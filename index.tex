% Options for packages loaded elsewhere
\PassOptionsToPackage{unicode}{hyperref}
\PassOptionsToPackage{hyphens}{url}
\PassOptionsToPackage{dvipsnames,svgnames,x11names}{xcolor}
%
\documentclass[
  letterpaper,
  DIV=11,
  numbers=noendperiod]{scrreprt}

\usepackage{amsmath,amssymb}
\usepackage{iftex}
\ifPDFTeX
  \usepackage[T1]{fontenc}
  \usepackage[utf8]{inputenc}
  \usepackage{textcomp} % provide euro and other symbols
\else % if luatex or xetex
  \usepackage{unicode-math}
  \defaultfontfeatures{Scale=MatchLowercase}
  \defaultfontfeatures[\rmfamily]{Ligatures=TeX,Scale=1}
\fi
\usepackage{lmodern}
\ifPDFTeX\else  
    % xetex/luatex font selection
\fi
% Use upquote if available, for straight quotes in verbatim environments
\IfFileExists{upquote.sty}{\usepackage{upquote}}{}
\IfFileExists{microtype.sty}{% use microtype if available
  \usepackage[]{microtype}
  \UseMicrotypeSet[protrusion]{basicmath} % disable protrusion for tt fonts
}{}
\makeatletter
\@ifundefined{KOMAClassName}{% if non-KOMA class
  \IfFileExists{parskip.sty}{%
    \usepackage{parskip}
  }{% else
    \setlength{\parindent}{0pt}
    \setlength{\parskip}{6pt plus 2pt minus 1pt}}
}{% if KOMA class
  \KOMAoptions{parskip=half}}
\makeatother
\usepackage{xcolor}
\ifLuaTeX
  \usepackage{luacolor}
  \usepackage[soul]{lua-ul}
\else
  \usepackage{soul}
  
\fi
\setlength{\emergencystretch}{3em} % prevent overfull lines
\setcounter{secnumdepth}{5}
% Make \paragraph and \subparagraph free-standing
\makeatletter
\ifx\paragraph\undefined\else
  \let\oldparagraph\paragraph
  \renewcommand{\paragraph}{
    \@ifstar
      \xxxParagraphStar
      \xxxParagraphNoStar
  }
  \newcommand{\xxxParagraphStar}[1]{\oldparagraph*{#1}\mbox{}}
  \newcommand{\xxxParagraphNoStar}[1]{\oldparagraph{#1}\mbox{}}
\fi
\ifx\subparagraph\undefined\else
  \let\oldsubparagraph\subparagraph
  \renewcommand{\subparagraph}{
    \@ifstar
      \xxxSubParagraphStar
      \xxxSubParagraphNoStar
  }
  \newcommand{\xxxSubParagraphStar}[1]{\oldsubparagraph*{#1}\mbox{}}
  \newcommand{\xxxSubParagraphNoStar}[1]{\oldsubparagraph{#1}\mbox{}}
\fi
\makeatother


\providecommand{\tightlist}{%
  \setlength{\itemsep}{0pt}\setlength{\parskip}{0pt}}\usepackage{longtable,booktabs,array}
\usepackage{calc} % for calculating minipage widths
% Correct order of tables after \paragraph or \subparagraph
\usepackage{etoolbox}
\makeatletter
\patchcmd\longtable{\par}{\if@noskipsec\mbox{}\fi\par}{}{}
\makeatother
% Allow footnotes in longtable head/foot
\IfFileExists{footnotehyper.sty}{\usepackage{footnotehyper}}{\usepackage{footnote}}
\makesavenoteenv{longtable}
\usepackage{graphicx}
\makeatletter
\newsavebox\pandoc@box
\newcommand*\pandocbounded[1]{% scales image to fit in text height/width
  \sbox\pandoc@box{#1}%
  \Gscale@div\@tempa{\textheight}{\dimexpr\ht\pandoc@box+\dp\pandoc@box\relax}%
  \Gscale@div\@tempb{\linewidth}{\wd\pandoc@box}%
  \ifdim\@tempb\p@<\@tempa\p@\let\@tempa\@tempb\fi% select the smaller of both
  \ifdim\@tempa\p@<\p@\scalebox{\@tempa}{\usebox\pandoc@box}%
  \else\usebox{\pandoc@box}%
  \fi%
}
% Set default figure placement to htbp
\def\fps@figure{htbp}
\makeatother
% definitions for citeproc citations
\NewDocumentCommand\citeproctext{}{}
\NewDocumentCommand\citeproc{mm}{%
  \begingroup\def\citeproctext{#2}\cite{#1}\endgroup}
\makeatletter
 % allow citations to break across lines
 \let\@cite@ofmt\@firstofone
 % avoid brackets around text for \cite:
 \def\@biblabel#1{}
 \def\@cite#1#2{{#1\if@tempswa , #2\fi}}
\makeatother
\newlength{\cslhangindent}
\setlength{\cslhangindent}{1.5em}
\newlength{\csllabelwidth}
\setlength{\csllabelwidth}{3em}
\newenvironment{CSLReferences}[2] % #1 hanging-indent, #2 entry-spacing
 {\begin{list}{}{%
  \setlength{\itemindent}{0pt}
  \setlength{\leftmargin}{0pt}
  \setlength{\parsep}{0pt}
  % turn on hanging indent if param 1 is 1
  \ifodd #1
   \setlength{\leftmargin}{\cslhangindent}
   \setlength{\itemindent}{-1\cslhangindent}
  \fi
  % set entry spacing
  \setlength{\itemsep}{#2\baselineskip}}}
 {\end{list}}
\usepackage{calc}
\newcommand{\CSLBlock}[1]{\hfill\break\parbox[t]{\linewidth}{\strut\ignorespaces#1\strut}}
\newcommand{\CSLLeftMargin}[1]{\parbox[t]{\csllabelwidth}{\strut#1\strut}}
\newcommand{\CSLRightInline}[1]{\parbox[t]{\linewidth - \csllabelwidth}{\strut#1\strut}}
\newcommand{\CSLIndent}[1]{\hspace{\cslhangindent}#1}

\usepackage{booktabs}
\usepackage{caption}
\usepackage{longtable}
\usepackage{colortbl}
\usepackage{array}
\usepackage{anyfontsize}
\usepackage{multirow}
\KOMAoption{captions}{tableheading}
\makeatletter
\@ifpackageloaded{tcolorbox}{}{\usepackage[skins,breakable]{tcolorbox}}
\@ifpackageloaded{fontawesome5}{}{\usepackage{fontawesome5}}
\definecolor{quarto-callout-color}{HTML}{909090}
\definecolor{quarto-callout-note-color}{HTML}{0758E5}
\definecolor{quarto-callout-important-color}{HTML}{CC1914}
\definecolor{quarto-callout-warning-color}{HTML}{EB9113}
\definecolor{quarto-callout-tip-color}{HTML}{00A047}
\definecolor{quarto-callout-caution-color}{HTML}{FC5300}
\definecolor{quarto-callout-color-frame}{HTML}{acacac}
\definecolor{quarto-callout-note-color-frame}{HTML}{4582ec}
\definecolor{quarto-callout-important-color-frame}{HTML}{d9534f}
\definecolor{quarto-callout-warning-color-frame}{HTML}{f0ad4e}
\definecolor{quarto-callout-tip-color-frame}{HTML}{02b875}
\definecolor{quarto-callout-caution-color-frame}{HTML}{fd7e14}
\makeatother
\makeatletter
\@ifpackageloaded{bookmark}{}{\usepackage{bookmark}}
\makeatother
\makeatletter
\@ifpackageloaded{caption}{}{\usepackage{caption}}
\AtBeginDocument{%
\ifdefined\contentsname
  \renewcommand*\contentsname{Table of contents}
\else
  \newcommand\contentsname{Table of contents}
\fi
\ifdefined\listfigurename
  \renewcommand*\listfigurename{List of Figures}
\else
  \newcommand\listfigurename{List of Figures}
\fi
\ifdefined\listtablename
  \renewcommand*\listtablename{List of Tables}
\else
  \newcommand\listtablename{List of Tables}
\fi
\ifdefined\figurename
  \renewcommand*\figurename{Figure}
\else
  \newcommand\figurename{Figure}
\fi
\ifdefined\tablename
  \renewcommand*\tablename{Table}
\else
  \newcommand\tablename{Table}
\fi
}
\@ifpackageloaded{float}{}{\usepackage{float}}
\floatstyle{ruled}
\@ifundefined{c@chapter}{\newfloat{codelisting}{h}{lop}}{\newfloat{codelisting}{h}{lop}[chapter]}
\floatname{codelisting}{Listing}
\newcommand*\listoflistings{\listof{codelisting}{List of Listings}}
\makeatother
\makeatletter
\makeatother
\makeatletter
\@ifpackageloaded{caption}{}{\usepackage{caption}}
\@ifpackageloaded{subcaption}{}{\usepackage{subcaption}}
\makeatother

\usepackage{bookmark}

\IfFileExists{xurl.sty}{\usepackage{xurl}}{} % add URL line breaks if available
\urlstyle{same} % disable monospaced font for URLs
\hypersetup{
  pdftitle={Pediatrics Notes},
  pdfauthor={Department of Child Health   SMS, KNUST   KATH},
  colorlinks=true,
  linkcolor={blue},
  filecolor={Maroon},
  citecolor={Blue},
  urlcolor={Blue},
  pdfcreator={LaTeX via pandoc}}


\title{Pediatrics Notes}
\author{Department of Child Health SMS, KNUST KATH}
\date{2025-11-17}

\begin{document}
\maketitle

\renewcommand*\contentsname{Table of contents}
{
\hypersetup{linkcolor=}
\setcounter{tocdepth}{2}
\tableofcontents
}

\bookmarksetup{startatroot}

\chapter*{Preface}\label{preface}
\addcontentsline{toc}{chapter}{Preface}

\markboth{Preface}{Preface}

Medical knowledge has evolved tremendously over the past century. With
the advent of the internet, sharing medical knowledge has become very
easy. However, most medical literature available originates from
developed countries, thus devoid of the relevant local content for many
underdeveloped regions. This is critical for medical education and
practice in that these texts often do not consider the cultural
relevance of these pathologies, disease patterns and risks, local
treatment options and idiosyncrasies of the medical delivery systems.
Undoubtedly, these have significant effects on medical education and
practice.

This text was birthed out of the need to bridge this gap. Though the
content is primarily directed at undergraduate medical teaching, it can
be beneficial to postgraduate trainees as well.

\bookmarksetup{startatroot}

\chapter*{List of contributors}\label{list-of-contributors}
\addcontentsline{toc}{chapter}{List of contributors}

\markboth{List of contributors}{List of contributors}

This book was written by the untiring effort of the following persons

\begin{center}\rule{0.5\linewidth}{0.5pt}\end{center}

\pandocbounded{\includegraphics[keepaspectratio]{images/prof-sampson-antwi.jpg}}

\textbf{Prof.~Sampson Antwi}

MB ChB, FWACP, FGCPS

Prof Sampson Antwi is a Professor of Pediatric Nephrology at the
Department of Child Health, School of Medical Sciences, Kwame Nkrumah
University of Science and Technology in Ghana. He is currently the Head
of Department and a fellow of the International Pediatric Nephrology
Association.

\begin{center}\rule{0.5\linewidth}{0.5pt}\end{center}

\pandocbounded{\includegraphics[keepaspectratio]{images/prof-daniel-ansong.jpg}}

\textbf{Prof.~Daniel Ansong}

MB ChB, FWACP, FGCPS

Prof.~Daniel Ansong is a Professor of Pediatrics at the Department of
Child Health, School of Medical Sciences, Kwame Nkrumah University of
Science and Technology in Ghana. He is an ardent researcher.

\begin{center}\rule{0.5\linewidth}{0.5pt}\end{center}

\pandocbounded{\includegraphics[keepaspectratio]{images/prof-alex-osei-akoto.jpg}}

\textbf{Prof.~Alex Osei Akoto}

MB ChB, FWACP, FGCPS

Prof.~Alex Osei Akoto is a Professor of paediatrics at the Department of
Child Health, School of Medical Sciences, Kwame Nkrumah University of
Science and Technology in Ghana.

\begin{center}\rule{0.5\linewidth}{0.5pt}\end{center}

\pandocbounded{\includegraphics[keepaspectratio]{images/prof-addo-yobo.jpg}}

\textbf{Prof.~Emmanuel Otopa Danquah Addo-Yobo}

MB ChB, FWACP, FGCPS

Prof.~Emmanuel Otopa Danquah Addo-Yobo is a Professor of paediatrics at
the Department of Child Health, School of Medical Sciences, Kwame
Nkrumah University of Science and Technology in Ghana.

\begin{center}\rule{0.5\linewidth}{0.5pt}\end{center}

\pandocbounded{\includegraphics[keepaspectratio]{images/prof-gyikua-plange-rhule.jpg}}

\textbf{Prof.~(Mrs) Gyikua Plange-Rhule}

MB ChB, FWACP, FGCPS

Prof.~(Mrs) Gyikua Plange-Rhule is a Professor of paediatrics at the
Department of Child Health, School of Medical Sciences, Kwame Nkrumah
University of Science and Technology in Ghana.

\begin{center}\rule{0.5\linewidth}{0.5pt}\end{center}

\pandocbounded{\includegraphics[keepaspectratio]{images/prof-joslin-dogbe.jpg}}

\textbf{Prof.~Joslin Alexei Dogbe}

MB ChB, FWACP, FGCPS

Prof.~Joslin Alexei Dogbe is a Professor of paediatrics at the
Department of Child Health, School of Medical Sciences, Kwame Nkrumah
University of Science and Technology in Ghana.

\begin{center}\rule{0.5\linewidth}{0.5pt}\end{center}

\pandocbounded{\includegraphics[keepaspectratio]{images/dr-samuel-blay-nguah.jpg}}

\textbf{Dr.~Samuel Blay Nguah}

MB ChB, FWACP, FGCPS

Dr.~Samuel Blay Nguah is a Senior Specialist in Pediatrics at the
Department of Child Health, School of Medical Sciences, Kwame Nkrumah
University of Science and Technology in Ghana.

\begin{center}\rule{0.5\linewidth}{0.5pt}\end{center}

\pandocbounded{\includegraphics[keepaspectratio]{images/dr-emmanuel-ameyaw.jpg}}

\textbf{Dr.~Emmanuel Ameyaw}

MB ChB, FWACP, FGCPS

Dr.~Emmanuel Ameyaw is a Senior Specialist in Pediatrics at the
Department of Child Health, School of Medical Sciences, Kwame Nkrumah
University of Science and Technology in Ghana. He is also a pediatric
endocrinologist

\begin{center}\rule{0.5\linewidth}{0.5pt}\end{center}

\pandocbounded{\includegraphics[keepaspectratio]{images/dr-anthony-enimil.jpg}}

\textbf{Dr.~Anthony Enimil}

MB ChB, FWACP, FGCPS

Dr Anthony Enimil is a Senior Specialist in Pediatrics at the Department
of Child Health, School of Medical Sciences, Kwame Nkrumah University of
Science and Technology in Ghana. He is also a pediatric infectious
diseases specialist

\begin{center}\rule{0.5\linewidth}{0.5pt}\end{center}

\begin{center}\rule{0.5\linewidth}{0.5pt}\end{center}

\pandocbounded{\includegraphics[keepaspectratio]{images/eno.jpg}}

\textbf{Dr.~Adwoa Pokua Boakye-Yiadom}

MB ChB, FWACP, FGCPS

Dr.~Adwoa Pokua Boakye-Yiadom is a Senior Lecturer in Pediatrics at the
Department of Child Health, School of Medical Sciences, Kwame Nkrumah
University of Science and Technology in Ghana. She is also a pediatric
neonatologist specialist

\begin{center}\rule{0.5\linewidth}{0.5pt}\end{center}

\pandocbounded{\includegraphics[keepaspectratio]{images/dr-sandra-kwarteng-owusu.jpg}}

\textbf{Dr.~(Mrs) Sandra Kwarteng Owusu}

MB ChB, FWACP, FGCPS

Dr.~Sandra Kwarteng Owusu is a Senior Specialist in Pediatrics at the
Department of Child Health, School of Medical Sciences, Kwame Nkrumah
University of Science and Technology in Ghana. She is also a pediatric
pulmonologist.

\begin{center}\rule{0.5\linewidth}{0.5pt}\end{center}

\pandocbounded{\includegraphics[keepaspectratio]{images/dr-akua-afriyie.jpg}}

\textbf{Dr.~(Mrs) Akua Afriyie Ocran}

MB ChB, FWACP, FGCPS

Dr.~Ocran is a Senior Specialist in Pediatrics at the Department of
Child Health, School of Medical Sciences, Kwame Nkrumah University of
Science and Technology in Ghana. She is also a Neonatologist

\begin{center}\rule{0.5\linewidth}{0.5pt}\end{center}

\pandocbounded{\includegraphics[keepaspectratio]{images/dr-naana-ayiwa-wireko-brobby.jpg}}

\textbf{Dr.~Naana Ayiwa Wireko Brobby}

MB ChB, FWACP, FGCPS

Dr.~Wireko Brobby is a Senior Specialist in Pediatrics at the Department
of Child Health, School of Medical Sciences, Kwame Nkrumah University of
Science and Technology in Ghana. She is also a Neonatologist

\begin{center}\rule{0.5\linewidth}{0.5pt}\end{center}

\pandocbounded{\includegraphics[keepaspectratio]{images/dr-vivian-paintsil.jpg}}

\textbf{Dr.~(Mrs) Vivian Paintsil}

MB ChB, FWACP, FGCPS

Dr.~(Mrs) Paintsil is a Senior Specialist in Pediatrics at the
Department of Child Health, School of Medical Sciences, Kwame Nkrumah
University of Science and Technology in Ghana. She is also a Pediatric
Oncologist

\begin{center}\rule{0.5\linewidth}{0.5pt}\end{center}

\pandocbounded{\includegraphics[keepaspectratio]{images/dr-charles-hammond.jpg}}

\textbf{Dr.~Charles Hammond}

MB ChB, FWACP, FGCPS

Dr.~Hammond is a Senior Specialist in Pediatrics at the Department of
Child Health, School of Medical Sciences, Kwame Nkrumah University of
Science and Technology in Ghana. She is also a Pediatric Neurologist

\begin{center}\rule{0.5\linewidth}{0.5pt}\end{center}

\pandocbounded{\includegraphics[keepaspectratio]{images/dr-serwaa-asafo-agyei.jpg}}

\textbf{Dr.~Serwah Bonsu Asafo-Agyei}

MB ChB, FWACP, FGCPS

Dr.~Asafo-Agyei is a Senior Specialist in Pediatrics at the Department
of Child Health, School of Medical Sciences, Kwame Nkrumah University of
Science and Technology in Ghana. She is also a Pediatric Endocrinologist

\begin{center}\rule{0.5\linewidth}{0.5pt}\end{center}

\pandocbounded{\includegraphics[keepaspectratio]{images/female-silhouette.jpg}}

\textbf{Dr.~Akua Yeboah Senyah}

MB ChB, FWACP, FGCPS

Dr.~Senyah is a Senior Specialist at the Directorate of Child Health of
the Komfo Anokye Teaching Hospital. She is also a Pediatric
Dermatologist

\begin{center}\rule{0.5\linewidth}{0.5pt}\end{center}

\pandocbounded{\includegraphics[keepaspectratio]{images/male-silhouette.jpg}}

\textbf{Dr.~Justice Sylverken}

MB ChB, FWACP, FGCPS

Dr.~Sylverken is a Senior Specialist at the Directorate of Child Health
of the Komfo Anokye Teaching Hospital. He has a special interest in
Pediatric Emergency

\begin{center}\rule{0.5\linewidth}{0.5pt}\end{center}

\pandocbounded{\includegraphics[keepaspectratio]{images/dr-John-adabie-appiah.jpg}}

\textbf{Dr.~John Adabie Appiah}

MB ChB, FWACP, FGCPS

Dr.~Appiah is a Senior Specialist in Pediatrics at the Department of
Child Health, Komfo Anokye Teaching Hospital. He is also a Pediatric
Critical Care Specialist.

\begin{center}\rule{0.5\linewidth}{0.5pt}\end{center}

\pandocbounded{\includegraphics[keepaspectratio]{images/akua-andzie-quainoo.jpg}}

\textbf{Dr.~Akua Andzie-Quainoo}

MB ChB, MWACP, MGCPS

Dr.~Andzie Quainoo is a Specialist in Pediatrics at the Department of
Child Health, Komfo Anokye Teaching Hospital. She is has special
interest in Paediatric Cardiology.

\begin{center}\rule{0.5\linewidth}{0.5pt}\end{center}

\pandocbounded{\includegraphics[keepaspectratio]{images/female-silhouette.jpg}}

\textbf{Dr.~Betty Nkansah Osei Mensah}

MB ChB BSc

Dr.~Betty Nkansah Osei Mensah is a resident pediatrician at the
Department of Child Health, Komfo Anokye Teaching.

\begin{center}\rule{0.5\linewidth}{0.5pt}\end{center}

\pandocbounded{\includegraphics[keepaspectratio]{images/male-silhouette.jpg}}

\textbf{Dr.~Joshua Sarkodie-Addo}

MB ChB BSc

Dr.~Sarkodie-Addo is a resident pediatrician at the Department of Child
Health, Komfo Anokye Teaching.

\begin{center}\rule{0.5\linewidth}{0.5pt}\end{center}

\pandocbounded{\includegraphics[keepaspectratio]{images/male-silhouette.jpg}}

\textbf{Dr.~Theodora-Ann Ellis}

MB ChB BSc

Dr.~Ellis is a resident pediatrician at the Department of Child Health,
Komfo Anokye Teaching.

\bookmarksetup{startatroot}

\chapter{Child History \& Examination}\label{child-history-examination}

\section{Introduction}\label{introduction}

To clerk a case is to take a comprehensive \textbf{history,} perform a
thorough \textbf{physical examination}, form an impression about the
most plausible diagnosis, known as a provisional diagnosis, order
investigations that will confirm this provisional diagnosis, and then
plan a \textbf{definitive treatment} based on the confirmed diagnosis.

\section{The Focus of Pediatrics
Clerkship}\label{the-focus-of-pediatrics-clerkship}

After a complete clerking of a patient, a reasonable Provisional
Diagnosis must be arrived at, which can then be confirmed with relevant
investigations. The nutritional status of any child should be determined
at the end of each pediatric case clerked.

\textbf{Differential diagnosis}: In most patient clerking situations,
more than one plausible investigation may be considered after a history
and physical examination. In such situations, all the plausible
diagnoses are differential diagnoses of each other.

\textbf{Different, distinctive diagnoses}: In some situations, the
patient may present with more than one unrelated disease. For example, a
patient who presents with pneumonia may also have septic arthritis. Such
different and distinct diagnoses are not differentials of each other

History

History-taking is pivotal in all medical encounters, whether in
emergencies or in ``stable'' encounters or consultations.

Paediatric History is essentially the same as adult history, albeit with
five additional segments:

\begin{enumerate}
\def\labelenumi{\arabic{enumi}.}
\tightlist
\item
  Informant
\item
  Pregnancy, Birth \& Neonatal History~~~~~
\item
  Immunization History~~
\item
  Dietary History~~
\item
  Developmental History
\end{enumerate}

\section{The 15 Elements of Pediatrics
History}\label{the-15-elements-of-pediatrics-history}

A full history must consist of the following 15 elements:

\begin{enumerate}
\def\labelenumi{\arabic{enumi}.}
\tightlist
\item
  The Informant
\item
  Demographics
\item
  Presenting Complaint
\item
  History of Presenting Complaint (HPC)
\item
  Direct Question
\item
  Systemic Enquiry
\item
  Past Medical History
\item
  Drug History
\item
  Pregnancy/Birth \& Neonatal History
\item
  Immunization History
\item
  Dietary History
\item
  Developmental History
\item
  Family History
\item
  Social History
\item
  Summary
\end{enumerate}

\subsection{Informant}\label{informant}

The role and importance of an informant in paediatric history are as
follows:

\begin{itemize}
\tightlist
\item
  To help with the provision of history. In infants and young children,
  the history is essentially given by the informant. In older children
  and adolescents, the sick child could contribute to the history.
\item
  To be reasonably sure whether the history so obtained is reliable or
  not
\end{itemize}

The following information should be requested of the informant regarding
themselves: their name, relationship to the patient, level of education,
and whether they witnessed the current illness from its onset.

\subsection{Demographics of Patient}\label{demographics-of-patient}

This should cover the patient's name, sex, age, residential address,
religion, and NHIS enrolment. ~The importance of patients' demographics
is foremost for patient identification. Additionally, the sex \& age of
a patient may point to certain disease types and rule out others. For
example, an infant girl straining at micturition has a form of urethral
obstruction, but it cannot be a posterior urethral valve since PUV
occurs exclusively in males.~ Similarly, bronchiolitis is not a usual
consideration for a 5-year-old with cough, breathlessness, and wheeze
since bronchiolitis occurs almost always in children under 2 years old.
The residential address can help identify the diseases to which the
patient is at risk. For example, a patient may be residing in an area
where certain diseases are endemic. For patients coming from a slum, the
insanitary conditions and congestion predispose them to diarrhoeal and
skin diseases.

\textbf{Note:} A family's religion may warn about potential conflicts
with certain treatments, such as blood transfusions. Having an active
insurance plan ensures the affordability of care to some extent.

\subsection{Presenting Complaint (PC) or Chief
Complaint}\label{presenting-complaint-pc-or-chief-complaint}

This elicits the symptoms the patient is presenting with, together with
their duration, in a chronological manner. The importance of the
presenting complaint lies in its ability to open up the whole history to
the system(s) involved in the disease process and the potential disease
under discussion. It serves as the gateway to the disease process under
review, as every disease manifests in its unique way. Identifying the
system(s) involved allows asking further questions about that system in
direct questioning (ODQ).

\subsection{History of Presenting
Complaint(HPC)}\label{history-of-presenting-complainthpc}

This provides detailed accounts of each symptom reported in the PC,
presented in chronological order. It also gives details of the
characteristics of the symptoms, relieving and aggravating factors, as
well as treatments that have been sought so far. The importance of the
HPC is to obtain a complete picture of each symptom. The characteristics
of the symptoms may provide clues to the disease under review.

\subsection{On Direct Questioning
(ODQ)}\label{on-direct-questioning-odq}

This is a focused questioning technique that asks for further symptoms
from the system(s) implicated by the presenting complaint, to narrow
down to the likely diagnosis. Where no particular system is identified,
direct questioning is conducted to identify the likely system of
infection or inflammation. For example, if fever is the only symptom
presented in the presenting complaint (PC), direct questioning is
conducted to identify the likely system of infection or inflammation.
For instance, symptoms suggestive of infection in the respiratory,
gastrointestinal (GIT), musculoskeletal, central nervous system (CNS),
genitourinary, ear, nose, and throat (ENT), etc., should be asked to
narrow down to the likely disease. If any positive symptom is elicited
in the ODQ, its duration and characteristics should be stated as well.

\subsection{Systemic Enquiry (Review of
Systems)}\label{systemic-enquiry-review-of-systems}

This section examines the ``unaffected systems'' to determine if they
are involved (through concurrent diseases or as a result of
complications from the primary disease). Here, key diagnostic symptoms
are asked in each of the systems (GIT, cardiovascular, respiratory, CNS,
musculoskeletal, genitourinary, integumentary, endocrine). Review of
Systems also allows obtaining all the symptoms the patient has (e.g., in
case we did not take note of some symptoms the informant mentioned in
the PC, or the informant stopped short of all the symptoms the patient
had)

\subsection{Past Medical History}\label{past-medical-history}

This section assesses four (4) things:

\begin{itemize}
\tightlist
\item
  Any medical condition the patient is known to have, current or
  previous, e.g., sickle cell disease, Asthma, Epilepsy, Tuberculosis,
  hypertension, diabetes, etc.
\item
  Previous hospital admissions, and if so, when and for what condition?
\item
  Previous blood transfusion
\item
  If the condition s/he is currently presenting with has occurred in the
  past. This is important as the recurrence of a symptom may give a clue
  to diagnosis. For example, recurrence of bodily swelling may point to
  relapsing nephrotic syndrome, and recurrence of afebrile seizure may
  point to epilepsy.
\end{itemize}

\subsection{Drug History}\label{drug-history}

This section assesses four (4) things:

\begin{enumerate}
\def\labelenumi{\arabic{enumi}.}
\tightlist
\item
  Medications taken so far for this current illness (usually captured in
  the HPC)
\item
  Whether a patient is on long-term medications. This may give a clue to
  an underlying medical condition, even if the informant failed to
  mention the condition in the past medical history
\item
  Any \ul{known} drug \& food allergies. If a patient has known
  allergies and a doctor fails to extract that information and goes
  ahead to give that forbidden medication, the practitioner may be
  liable for disciplinary actions and even prosecution should the
  patient suffer grave effects of the allergy.
\item
  Any herbal medications, habitually or acutely, for this illness.~ The
  importance is that the herbs may be responsible for the illness under
  review, since the potential toxic effects of most herbs have not been
  elucidated and documented, unlike orthodox medicine, whose side
  effects could easily be elucidated
\end{enumerate}

\subsection{Pregnancy, Birth \& Neonatal
History}\label{pregnancy-birth-neonatal-history}

This section assesses the patient's history during their conception,
delivery, and neonatal life. It is not the pregnancy that the mother may
be carrying at the time of clerking the sick child. The importance lies
in the fact that any disease a mother suffered during pregnancy could
affect the offspring of that pregnancy. The same applies to the
consequences of labor and delivery, as well as the neonatal life of that
child. The following information should be sought for in the pregnancy,
labour, and neonatal life:

\subsubsection{Pregnancy}\label{pregnancy}

When booking for antenatal care (ANC) was done, any significant diseases
encountered by the mother during that pregnancy (e.g., diabetes,
hypertension, rash, febrile illness, jaundice, admissions, and if so,
for what condition) should be noted. Febrile illness, rash, and jaundice
could all point to a possible TORCH infection in the mother. Whether the
pregnancy was carried to term or not should be inquired about.

\subsubsection{Birth}\label{birth}

The gestation of that pregnancy (term or not), the mode of delivery of
that child (spontaneous vaginal, induced labor, C/S, etc.), the baby's
condition at delivery, whether the baby cried at birth, whether the baby
needed resuscitation, and how soon after birth the baby was discharged
could all indicate how well or otherwise the baby was at birth. The
birth weight should also be asked for.

\subsubsection{The Neonatal Life}\label{the-neonatal-life}

Any neonatal illnesses. In particular, ~jaundice, febrile illness, or
admissions would be important to ask.

\subsection{Immunization}\label{immunization}

~Immunization is an important tool in preventing infectious diseases in
children. A child who is not immunized is at significant risk of
acquiring a severe form of infectious disease and of dying within the
first few years of life. It is essential to determine whether the
child's immunization is up to date or completed, based on the child's
age and the national immunization schedule. This information should be
cross-checked with the immunization card, if available. The presence of
a BCG scar should be checked to ensure that at least some vaccinations
have been initiated. Also, a BCG scar failure in those immunized may
lead to failure of BCG immunization in a tiny minority. If some
immunizations are detected to have been missed, the reasons for the
missed immunizations should be investigated, and if it is still within
the appropriate vaccination age, the child should be administered those
immunizations.

\subsection{Dietary History}\label{dietary-history}

Since children are growing species that require nutrients for both
growth and development, a comprehensive dietary history is a key
component of the pediatric history. Information to be sought under this
section includes: the breastfeeding history (in all cases) and a typical
24-hour dietary history, with an emphasis on complete meals rather than
just the main food type, e.g., rice with tomato stew and fish for lunch,
rather than just ``rice''. The meal should be assessed for both quality
(balanced meal) and quantity. For toddlers, types of complementary foods
and frequency of feeds, including night feeds, should be sought. Fruits
and Snack intake (meals taken in between main meals) should all be
assessed. Whether a child was fed breastmilk exclusively or was exposed
to cow's milk early in life has implications for diseases in later
childhood, e.g., allergic conditions, metabolic conditions, etc, hence
the importance of breastfeeding history in all cases.

\subsection{Developmental History}\label{developmental-history}

At any point in a child's life, their level of development must be
assessed to determine whether they are progressing at an appropriate
rate for their age.

\textbf{Four} areas of development should be assessed, namely:

\begin{enumerate}
\def\labelenumi{\arabic{enumi}.}
\tightlist
\item
  Gross motor development, e.g., sitting, standing, walking
\item
  Fine motor development (use of hands in coordination with vision)
\item
  Hearing \& Speech development
\item
  Social development (interactions with parents \& others + bladder and
  bowel control/continence)
\end{enumerate}

Where a developmental abnormality is detected, it is essential to review
the pregnancy, birth, neonatal history, as well as nutritional and past
medical history, to identify potential insults to the brain that may
have occurred during this period.

\subsection{Family History}\label{family-history}

The family history assesses for any diseases in the family that could
potentially have been transmitted to the patient, either through
heredity or environmental factors (common risk factors). In particular,
diseases such as sickle cell disease, asthma, epilepsy, tuberculosis,
HIV, hypertension, and diabetes in a parent or sibling are important to
inquire about. Acute illnesses, such as a runny nose, diarrhea, and
febrile convulsions (if a child has a history of these), would also be
important to note from the family. The family tree may be assessed in
the family history or under the social history, as given below.

\subsection{Social History}\label{social-history}

The family's social status is a significant predictor of risk factors
for the child's disease, as well as the family's ability to manage the
prescribed treatment effectively.~~

Information sought for under the social history covers the following six
(6) areas:

\begin{enumerate}
\def\labelenumi{\arabic{enumi}.}
\tightlist
\item
  \textbf{Parents}: their ages, level of education, and occupation.
  Whether they are in a stable marriage or not. If a parent has passed,
  the circumstances of their passing and the likely cause should be
  sought.
\item
  \textbf{Siblings}: Number, ages, and sexes, as well as their current
  school status. If a sibling has passed away, the circumstances of
  their passing should be noted.
\item
  \textbf{Residential facility}: The number of sleeping rooms, the
  number of people who sleep with the patient, the ventilation of the
  room (including windows), and the use of mosquito nets.
\item
  \textbf{Water \& Sewage}: Source of drinking water, toilet facilities,
  and means of waste disposal
\item
  \textbf{Financial support} for the child's upkeep
\item
  \textbf{Social habits of parents} like smoking \& drinking (Home
  environmental risk factors for diseases, e.g., a child heavily exposed
  to smoking will be at risk of respiratory diseases like asthma and
  pneumonia)
\end{enumerate}

\subsection{Summary of History}\label{summary-of-history}

The summary uses not only the symptoms elicited but also any other
relevant information obtained from any segment of the history that is
worthy of note. Typically, it mentions the patient's name, age,
presenting complaint, and all other essential information in a sentence
or two.

\section{Physical Examination}\label{physical-examination}

This follows after obtaining the comprehensive history. Always begins
the physical examination with \textbf{Anthropometry}:

\begin{enumerate}
\def\labelenumi{\arabic{enumi}.}
\tightlist
\item
  Weight, weight-for-age SD score, its interpretation (normal, abnormal,
  etc)
\item
  Height, height-for-age SD score, its interpretation (normal, abnormal,
  etc)
\item
  Weight-for-height SD score (for children up to 5 years), its
  interpretation (normal, abnormal, etc)
\item
  BMI (for children \textgreater{} 5 years), BMI-for-age centile, its
  interpretation (normal, abnormal, etc)
\item
  Mid-Upper-Arm-Circumference (MUAC, for 6 months to 5 years), its
  interpretation
\item
  Head circumference percentile (for children up to 5 years) and what it
  means
\end{enumerate}

~Of note, WHO simplifies the definitions of anthropometry as follows:

Findings outside the borders of -2SD and +2SD are abnormal, and values
-2SD to +2SD are normal. If the values are outside -3SD and +3SD, then
they are severely abnormal, e.g., moderate underweight if WFA is
\textless{} -2SD, severe underweight if WFA \textless-3SD, overweight if
WFA \textgreater+2SD, and obese if WFA \textgreater+3SD.

\textbf{Vital Signs:} Temperature, Pulse rate, volume \& rhythm,
Respiratory rate, Oxygen saturation (SPO2), Blood Pressure

\subsection{General Examination}\label{general-examination}

This assessment evaluates the general state of the patient, including
their appearance, distress level, position in bed, nutritional status,
state of consciousness, and other relevant factors. It then assesses for
pallor, jaundice, lymph node enlargement, pedal edema, hydration status,
warmth in the hands, capillary refill, clubbing, or any other stigmata
of disease. Additionally, it examines the skin for rashes, pigmentation,
and any eruptions.

\subsection{System-by-System
Examination}\label{system-by-system-examination}

This should cover \ul{at least the four major systems: Cardiovascular,
Respiratory, Gastrointestinal/Abdomen, and the central nervous system}.
Note that all four systems must be examined for every case clerked,
regardless of the system affected by the disease. Other systems to note
include the musculoskeletal, genitourinary, integumentary (comprising
skin and mucous membranes), and endocrine systems.

\subsubsection{Respiratory System
Examination}\label{respiratory-system-examination}

\textbf{Inspection}: Always begin by assessing the respiratory rate and
respiratory effort (quiet or distressful). Then check for cyanosis and
the shape of the chest.

\textbf{Palpation:} Palpate the chest wall for tenderness, centrality of
the trachea, and lymph nodes (if not already assessed during the general
examination). Additionally, assess chest expansion (for older children
only) and tactile fremitus (for older children only).

\textbf{Percussion:} The percussion fingers should always be placed
horizontally along the intercostal spaces and NEVER across the ribs or
scapulae. All chest zones, anteriorly, posteriorly, and laterally,
should be percussed. It is more convenient to finish the anterior and
lateral chest examination before moving to the back.~

\textbf{Auscultation:} The assessment should report on the volume of air
(adequacy of air entering lungs), nature of breath sounds, any
additional sounds, and vocal fremitus

\subsection{Cardiovascular System
Examination}\label{cardiovascular-system-examination}

A convenient style of cardiovascular examination is to move from the
\textbf{hands} (for warmth, CRT, clubbing, and cyanosis), wrist (for
pulse), arm (for blood pressure), neck (for distended veins), and then
settle on the heart or coronary arteries. However, the patient's general
position in bed (propped up or not), use of supplementary oxygen,
respiratory effort, and mouth for central cyanosis should all be noted,
if not already captured during the general examination.

\subsection{The COR/Heart}\label{the-corheart}

\textbf{Inspection:} Inspect for any bulge, precordial pulsations

\textbf{Palpation:} Palpate for the Apex beat, heaves, and thrills

\textbf{Auscultation:} Auscultate over all four areas for quality of the
heart sounds 1 \& 2, rhythm of the heartbeat, murmurs, and any added
sounds. If a murmur is detected, its characteristics must be reported,
and the point of maximal sound (which may indicate the valve affected or
position of a shunt lesion) as well as its radiation.

Typically, there is no \textbf{percussion} in COR examination!

\subsection{Gastrointestinal System
Examination}\label{gastrointestinal-system-examination}

A typical GIT examination is from mouth to anus, covering the abdomen
(liver and intestines).

However, GIT examination may be limited to the Abdomen.

\subsection{Abdominal Examination}\label{abdominal-examination}

\textbf{Inspection:} Exposure - The abdomen should be reasonably exposed
from the nipple level down to both inguinal creases. The genitals must
be covered for privacy, but must be inspected. The size of the abdomen
(using the chest wall as a reference, both anteroposteriorly and
laterolaterally), position of the umbilicus, movement with respiration,
presence or absence of distended veins or scars, and the presence of
hernia orifices should be commented on. If the patient is edematous,
check for edema of the genitals.

\textbf{Palpation:} Perform light palpation to assess tenderness in all
nine regions. If masses are detected during this examination, they
should be reported. Deep palpation for the Liver, Spleen, kidneys, and
any other masses felt during light palpation

\textbf{Percussion:} Percuss for fluid (use shifting dullness if the
fluid is judged to be mild to moderate, and use fluid thrill if the
fluid is judged to be severe). Always percuss at the level of the
umbilicus with the fingers spread out.

If percussion note is tympanitic across the hemi-abdomen to the flank,
there is no fluid. If it is dull all through, then the abdomen may be
full of fluid, in which case, fluid thrill will be preferred to use. If
there is perceived dullness at the flank but the dull note does not
change to tympanitic at the shift of the patient to the opposite site,
then there is no fluid in the abdomen.

\textbf{Auscultation:} Auscultate for bowel sounds by using the
diaphragm of the stethoscope around the umbilicus. Report on the
presence or absence of bowel sounds and their pitch. Bowel sound
auscultation is typically performed over 2 minutes. If no bowel sounds
are heard over this period, the bowel sounds are presumed to be absent.
In some cases, bruit can also be auscultated.

\subsection{Central Nervous System Physical
Examination}\label{central-nervous-system-physical-examination}

Here, five (5) areas should be examined and reported on, namely:

\begin{enumerate}
\def\labelenumi{\arabic{enumi}.}
\tightlist
\item
  \textbf{The level of Consciousness:} The Blantyre coma scale may be
  used (for children under 5 years) or the modified Glasgow coma scale.
\item
  \textbf{Signs of meningeal irritation}: Check for Neck stiffness,
  Kernig's sign, and Brudzinski's sign. Bulging fontanel may be elicited
  in babies, but this is often a late sign. Fever with irritability is a
  specific indicator of meningitis in this special group.
\item
  \textbf{Cranial nerves examination:} Examination of the cranial nerves
  should be performed
\item
  \textbf{Motor system:} Assess for the Tone, Power \& Reflexes
  {[}TPR{]})
\item
  \textbf{Sensory system:} Assess fine and deep touch, coordination,
  gait, and joint position sense
\end{enumerate}

\section{Provisional Diagnosis \& Differential
Diagnosis}\label{provisional-diagnosis-differential-diagnosis}

Information from history and examination is synthesized to arrive at a
likely diagnosis {[}provisionally{]} + all other potential diagnoses.
Notably, if there are two or more separate diagnoses, such as malaria,
pneumonia, and otitis, these remain separate diagnoses and not
differential diagnoses of one another. Differential diagnoses are
usually (but not always) exclusive of one another; for example, is it
pneumonia or heart failure?

\section{Investigations}\label{investigations}

\textbf{Main and supportive}

Always start with the main investigations that will lead to the
confirmation of the diagnosis before coming to the supportive
(ancillary) tests. For example, a Chest X-ray is diagnostic for
pneumonia, while a full blood count looking for neutrophil leukocytosis
is supportive.

\section{Definitive Diagnosis}\label{definitive-diagnosis}

Based on the results of investigations, a definitive diagnosis is then
made.

\section{Treatment Plan}\label{treatment-plan}

Based on the suspected or definitive diagnosis, a treatment plan is
formulated:

\textbf{Main treatment}: The specific treatment recommended for the
particular disease. For example, antibiotics for infectious diseases

\textbf{Supportive treatment}: Those treatments that relieve symptoms.
For example, analgesics for pain, antipyretics for fever

Note on \textbf{Empiric Treatment:} At the point of provisional
diagnosis, while awaiting confirmation of the disease through
appropriate diagnostic investigations, treatment is usually initiated
based on the most likely anticipated diagnosis. Such treatment
intervention is called empiric treatment. For infectious diseases in
which cultures have been taken and the results are waiting, the likely
isolate with known antibiotic susceptibility is usually initiated, which
will be reviewed either for continuation or discontinuation based on the
culture and sensitivity results obtained.

\textbf{Emergency management:}

Where a case is life-threatening and requires emergency intervention, it
may not be necessary to wait and go through the details of clerking
outlined above. In such emergency cases, a brief history may be taken,
and depending on the life-threatening issues identified, emergency
interventions may be instituted to stabilize the patient before
proceeding with a full clinical review.

USEFUL BOOKS

\href{https://shop.elsevier.com/books/hutchisons-clinical-methods/glynn/978-0-7020-8265-8}{Clinical
Methods by Hutchison and Macleod}

\bookmarksetup{startatroot}

\chapter{Growth and Development}\label{growth-and-development}

\section{Introduction}\label{introduction-1}

Growth and development are fundamental indicators of a child's overall
health and well-being. As medical students in Ghana, it is crucial to
comprehend the physiological processes of growth and development, their
milestones, and how socio-economic and environmental factors specific to
Ghana impact these processes.

\textbf{Growth} refers to an increase in physical size (height, weight,
head circumference). In contrast, \textbf{development} refers to the
acquisition of skills and functions such as motor abilities, language,
cognition, and social behaviour. Both occur simultaneously and are
influenced by genetic, nutritional, hormonal, environmental, and
psychosocial factors.

\section{Principles of Growth and
Development}\label{principles-of-growth-and-development}

\begin{enumerate}
\def\labelenumi{\arabic{enumi}.}
\tightlist
\item
  \textbf{Cephalocaudal progression}: Development proceeds from head to
  toe. For example, infants gain head control before they can sit or
  walk.
\item
  \textbf{Proximodistal progression}: Development proceeds from the
  center of the body outwards. Gross motor skills develop before fine
  motor skills.
\item
  \textbf{Sequential and Predictable}: Milestones follow a predictable
  pattern, although the pace may vary.
\item
  \textbf{Critical periods}: There are periods when the child is
  especially sensitive to environmental stimuli.
\item
  \textbf{Individual variability}: Normal children may achieve
  milestones at slightly different ages.
\end{enumerate}

\section{Stages of Growth and
Development}\label{stages-of-growth-and-development}

\textbf{1. Neonatal Period (Birth -- 28 Days)}

\textbf{Growth:}

\begin{itemize}
\tightlist
\item
  \textbf{Weight}: Average birth weight is 2.5--4.0 kg. Infants may lose
  up to 10\% of their birth weight in the first week but regain it by
  day 10.
\item
  \textbf{Length}: \textasciitilde50 cm at birth.
\item
  \textbf{Head circumference}: \textasciitilde35 cm at birth.
\end{itemize}

\textbf{Development:}

\begin{itemize}
\tightlist
\item
  Primitive reflexes: Rooting, sucking, Moro, palmar grasp, stepping
  reflex.
\item
  Sensory abilities: Can see up to 20--30 cm, prefer human faces,
  respond to loud sounds.
\item
  Motor: Moves all limbs symmetrically, exhibits flexed posture.
\end{itemize}

\textbf{2. Infancy (1 month -- 1 year)}

\textbf{Growth:}

\begin{itemize}
\tightlist
\item
  Weight doubles by 5--6 months and triples by 1 year.
\item
  Length increases by 50\% in the first year.
\item
  Head circumference increases \textasciitilde1 cm/month in the first 6
  months.
\end{itemize}

\textbf{Development:}

\begin{itemize}
\item
  \textbf{Gross motor}:

  \begin{itemize}
  \tightlist
  \item
    3 months: Head control.
  \item
    6 months: Rolls over.
  \item
    9 months: Sits without support, crawls.
  \item
    12 months: Stands, may begin walking
  \end{itemize}
\item
  \textbf{Fine motor}

  \begin{itemize}
  \tightlist
  \item
    4--6 months: Reaches for objects.
  \item
    9 months: Pincer grasp begins
  \item
    12 months: Transfers objects between hands, bangs objects together.
  \end{itemize}
\item
  \textbf{Language}:

  \begin{itemize}
  \tightlist
  \item
    2 months: Coos.
  \item
    6 months: Babbles.
  \item
    9--12 months: Says ``mama,'' ``dada'' (non-specific), understands
    ``no.''
  \end{itemize}
\item
  \textbf{Social}:

  \begin{itemize}
  \tightlist
  \item
    2 months: Social smile.
  \item
    6 months: Stranger anxiety.
  \item
    12 months: Waves ``bye-bye,'' enjoys peek-a-boo.
  \end{itemize}
\end{itemize}

\textbf{3. Toddler (1 -- 3 years)}

\textbf{Growth:}

\begin{itemize}
\tightlist
\item
  Gains 2--3 kg per year.
\item
  Height increases by \textasciitilde12 cm per year.
\item
  Head growth slows; the anterior fontanelle closes by 18 months.
\end{itemize}

\textbf{Development:}

\begin{itemize}
\item
  \textbf{Gross motor}:

  \begin{itemize}
  \tightlist
  \item
    15 months: Walks independently.
  \item
    18 months: Climbs stairs with help.
  \item
    2 years: Runs, kicks a ball.
  \item
    3 years: Rides tricycle, climbs stairs alternating feet.
  \end{itemize}
\item
  \textbf{Fine motor}:

  \begin{itemize}
  \tightlist
  \item
    Builds tower of 3 (18 months) to 9 (3 years) cubes.
  \item
    Can feed themselves with a spoon.
  \item
    Begins to draw lines and circles.
  \end{itemize}
\item
  \textbf{Language}:

  \begin{itemize}
  \tightlist
  \item
    18 months: 10--20 words.
  \item
    2 years: 2-word phrases, \textasciitilde50 words.
  \item
    3 years: Sentences of 3--4 words.
  \end{itemize}
\item
  \textbf{Social}:

  \begin{itemize}
  \tightlist
  \item
    Parallel play.
  \item
    Temper tantrums, strong desire for independence.
  \item
    Recognizes self in mirror.
  \end{itemize}
\end{itemize}

\textbf{4. Preschool (3 -- 5 years)}

\textbf{Growth:}

\begin{itemize}
\tightlist
\item
  Gains 2 kg/year.
\item
  Height increases \textasciitilde6--8 cm/year.
\end{itemize}

\textbf{Development:}

\begin{itemize}
\tightlist
\item
  \textbf{Gross motor}: Hops on one foot, skips, throws, and catches a
  ball.
\item
  \textbf{Fine motor}: Copies shapes, uses scissors, dresses self with
  help.
\item
  \textbf{Language}: Clear speech, tells stories, knows names, age, and
  gender.
\item
  \textbf{Cognitive}: Magical thinking, learns numbers, colours.
\item
  \textbf{Social}: Cooperative play starts forming friendships.
\end{itemize}

\textbf{5. School-Age (6 -- 12 years)}

\textbf{Growth:}

\begin{itemize}
\tightlist
\item
  Steady growth of \textasciitilde5--7 cm/year and 2--3 kg/year.
\item
  Permanent teeth begin to erupt around age 6.
\end{itemize}

\textbf{Development:}

\begin{itemize}
\tightlist
\item
  \textbf{Gross motor}: Coordination improves, participates in sports.
\item
  \textbf{Fine motor}: Writes well, does crafts, and is independent in
  dressing and eating.
\item
  \textbf{Cognitive}: Concrete operational stage (Piaget) -- can think
  logically about tangible objects.
\item
  \textbf{Social}: Peer relationships become central; starts forming
  moral values.
\item
  \textbf{Emotional}: Develops self-esteem; compares self to others.
\end{itemize}

\textbf{6. Adolescence (13 -- 18 years)}

Divided into early (10--13), middle (14--16), and late (17--19)
adolescence.

\textbf{Growth:}

\begin{itemize}
\tightlist
\item
  \textbf{Pubertal growth spurt}:

  \begin{itemize}
  \tightlist
  \item
    Girls: Peak at 11--12 years.
  \item
    Boys: Peak at 13--14 years.
  \end{itemize}
\item
  Growth completes by 18--20 years.
\item
  Sexual maturation: Tanner staging is used to assess the progression of
  puberty.
\end{itemize}

\textbf{Tanner Staging Overview:}

\begin{itemize}
\tightlist
\item
  \textbf{Stage 1}: Prepubertal.
\item
  \textbf{Stage 2}: Breast bud (girls); testicular enlargement (boys).
\item
  \textbf{Stage 3--5}: Progressive pubic hair growth, breast and genital
  development.
\end{itemize}

\textbf{Development:}

\begin{itemize}
\tightlist
\item
  \textbf{Cognitive}: Formal operational stage -- abstract thinking.
\item
  \textbf{Psychosocial} (Erikson): Identity vs.~Role Confusion.
\item
  \textbf{Emotional}: Self-awareness, mood swings, peer pressure.
\item
  \textbf{Social}: Increased independence, interest in opposite sex,
  development of personal values.
\end{itemize}

\section{Factors Influencing Growth and Development in
Ghana}\label{factors-influencing-growth-and-development-in-ghana}

\textbf{1. Nutrition}

\begin{itemize}
\tightlist
\item
  Malnutrition remains a significant cause of stunting and wasting in
  Ghana.
\item
  Exclusive breastfeeding for 6 months followed by appropriate
  complementary feeding is critical.
\item
  Micronutrient deficiencies (iron, vitamin A, iodine) are common.
\end{itemize}

\textbf{2. Health and Disease}

\begin{itemize}
\tightlist
\item
  Frequent infections (malaria, diarrheal disease, respiratory
  infections) affect growth.
\item
  Helminthic infestations (e.g., \emph{Ascaris}, \emph{hookworm}) can
  cause anemia and malabsorption.
\item
  HIV and chronic illnesses impact weight gain and development.
\end{itemize}

\textbf{3. Immunization}

\begin{itemize}
\tightlist
\item
  Vaccines under Ghana's EPI (Expanded Programme on Immunization)
  protect against major childhood illnesses.
\item
  Delays in vaccination can predispose children to infections that
  impair growth.
\end{itemize}

\textbf{4. Socioeconomic Status}

\begin{itemize}
\tightlist
\item
  Poverty, poor housing, and low parental education levels contribute to
  undernutrition and developmental delays.
\item
  Urban-rural disparities exist, with rural children at higher risk of
  poor outcomes.
\end{itemize}

\textbf{5. Environmental Factors}

\begin{itemize}
\tightlist
\item
  Poor sanitation increases the risk of repeated infections.
\item
  Environmental toxins (e.g., lead exposure in certain mining
  communities) can cause neurodevelopmental issues.
\end{itemize}

\textbf{6. Parental Care and Stimulation}

\begin{itemize}
\tightlist
\item
  Emotional support, play, and verbal interaction are key for early
  brain development.
\item
  Neglect, abuse, and trauma can lead to delayed speech and cognitive
  skills
\end{itemize}

\section{Clinical Assessment of Growth and
Development}\label{clinical-assessment-of-growth-and-development}

\textbf{Anthropometric Measurements}

\begin{itemize}
\tightlist
\item
  \textbf{Weight}: Measured at every visit. Weight-for-age is a good
  screening tool.
\item
  \textbf{Height/Length}: Height-for-age assesses linear growth; used to
  detect stunting.
\item
  \textbf{Head circumference}: Measured in children under 2 years;
  useful in assessing brain growth.
\item
  \textbf{Mid-upper arm circumference (MUAC)}: Used in children aged
  6--59 months to screen for acute malnutrition.
\end{itemize}

\textbf{Growth Charts}

\begin{itemize}
\tightlist
\item
  WHO growth standards are used in Ghana.
\item
  Plotted regularly to monitor trends over time.
\item
  Red flags: crossing percentiles downward, weight loss, or faltering
  height.
\end{itemize}

\section{Red Flags in Growth and
Development}\label{red-flags-in-growth-and-development}

Medical students should be alert to signs that may indicate problems:

\begin{itemize}
\tightlist
\item
  No head control by 4 months.
\item
  No sitting by 9 months of age.
\item
  No walking by 18 months.
\item
  No single words by 15 months.
\item
  Regression of previously attained milestones.
\item
  Persistent failure to thrive despite nutritional intervention.
\item
  Rapid head growth or microcephaly.
\item
  Poor school performance in school-aged children.
\end{itemize}

\section{Role of the Health Worker}\label{role-of-the-health-worker}

In Ghana, health workers play a vital role in promoting optimal growth
and development:

\begin{enumerate}
\def\labelenumi{\arabic{enumi}.}
\tightlist
\item
  \textbf{Growth monitoring and promotion}: Routine weighing and
  charting in Child Welfare Clinics (CWC).
\item
  \textbf{Nutrition counseling}: Promote exclusive breastfeeding,
  weaning practices, and dietary diversity.
\item
  \textbf{Immunization}: Ensuring timely vaccination.
\item
  \textbf{Early identification and referral}: Recognizing signs of
  developmental delay and making timely referrals.
\item
  \textbf{Parental education}: Encouraging stimulation, responsive
  parenting, and early learning.
\item
  \textbf{School health services}: Routine screening in schools for
  hearing, vision, and dental problems.
\end{enumerate}

\section{Conclusion}\label{conclusion}

Understanding child growth and development is critical in paediatrics
and preventive health care. In Ghana, many preventable factors influence
a child's trajectory. As future medical practitioners, students must
recognize normal patterns, use appropriate tools for assessment, and
intervene early where deviations exist. Knowledge of cultural,
nutritional, and environmental influences is essential for
context-specific care. By prioritizing growth and development, we lay
the foundation for healthier futures in Ghana's children.

\bookmarksetup{startatroot}

\chapter{Pediatric Anthropometry}\label{pediatric-anthropometry}

\section{Introduction}\label{introduction-2}

Pediatric anthropometry is the scientific measurement of the physical
dimensions and composition of the human body in children. It is a
fundamental component of growth monitoring and nutritional assessment,
playing a crucial role in evaluating child health. For medical students
and healthcare providers in Ghana, mastering anthropometry is essential
for identifying malnutrition, developmental issues, and chronic diseases
in children. This clinical note will cover the principles, techniques,
indicators, interpretation, and clinical application of pediatric
anthropometry, with special attention to the Ghanaian context.

\section{Objectives of Paediatric
Anthropometry}\label{objectives-of-paediatric-anthropometry}

\begin{enumerate}
\def\labelenumi{\arabic{enumi}.}
\tightlist
\item
  Assess growth and nutritional status
\item
  Monitor development over time
\item
  Detect early signs of undernutrition or overnutrition
\item
  Evaluate the impact of health and nutrition interventions
\item
  Assist in diagnosing systemic illnesses
\item
  Provide evidence for public health surveillance and policy making
\end{enumerate}

\section{Key Anthropometric Measurements in
Children}\label{key-anthropometric-measurements-in-children}

\textbf{1. Weight}

\begin{itemize}
\tightlist
\item
  \textbf{Importance}: Reflects body mass and is sensitive to acute
  changes in health and nutrition.
\item
  \textbf{Equipment}:

  \begin{itemize}
  \tightlist
  \item
    Infants: Electronic infant scale or beam balance (accurate to ±10g).
  \item
    Older children: Digital or beam scale (accurate to ±100g).
  \end{itemize}
\item
  \textbf{Procedure}:

  \begin{itemize}
  \tightlist
  \item
    Remove clothing and shoes.
  \item
    For infants, weigh naked or with minimal clothing.
  \item
    Ensure the scale is calibrated and on a flat surface.
  \end{itemize}
\item
  \textbf{Interpretation}:

  \begin{itemize}
  \tightlist
  \item
    Compare with WHO growth standards using Weight-for-Age (WFA),
    Weight-for-Height (WFH), and Body Mass Index (BMI).
  \end{itemize}
\end{itemize}

\textbf{2. Length/Height}

\begin{itemize}
\item
  \textbf{Length (children \textless2 years)}:

  \begin{itemize}
  \tightlist
  \item
    Measured using an infantometer.
  \item
    Child lies supine with head held against the fixed headboard and
    legs fully extended.
  \end{itemize}
\item
  \textbf{Height (children ≥2 years)}:

  \begin{itemize}
  \tightlist
  \item
    Use a stadiometer or wall-mounted measuring board.
  \item
    The child stands erect without shoes, heels together, and looks
    straight ahead
  \end{itemize}
\item
  \textbf{Accuracy}: ±0.1 cm
\item
  \textbf{Interpretation}:

  \begin{itemize}
  \tightlist
  \item
    Compare with Height-for-Age (HFA) standard.
  \item
    Used to detect stunting (chronic malnutrition)
  \end{itemize}
\end{itemize}

\textbf{3. Mid-Upper Arm Circumference (MUAC)}

\begin{itemize}
\item
  \textbf{Importance}: A rapid screening tool for acute malnutrition in
  children aged 6--59 months.
\item
  \textbf{Equipment}: MUAC tape (color-coded for easy interpretation).
\item
  \textbf{Procedure}:

  \begin{itemize}
  \tightlist
  \item
    Locate the midpoint between the acromion and the olecranon process.
  \item
    Measure the circumference of the left upper arm
  \end{itemize}
\item
  \textbf{Interpretation}:

  \begin{itemize}
  \tightlist
  \item
    MUAC \textless11.5 cm: Severe Acute Malnutrition (SAM).
  \item
    11.5--12.5 cm: Moderate Acute Malnutrition (MAM).
  \item
    ≥12.5 cm: Normal
  \end{itemize}
\end{itemize}

\textbf{4. Head Circumference}

\begin{itemize}
\tightlist
\item
  \textbf{Importance}: Reflects brain growth, especially in the first
  two years.
\item
  \textbf{Equipment}: Non-stretchable measuring tape.
\item
  \textbf{Procedure}:

  \begin{itemize}
  \tightlist
  \item
    Place the tape above the eyebrows and ears, and around the occipital
    prominence.
  \end{itemize}
\item
  \textbf{Interpretation}:

  \begin{itemize}
  \tightlist
  \item
    Compare with age- and sex-specific World Health Organization (WHO)
    standards.
  \item
    Used to identify microcephaly or macrocephaly.
  \end{itemize}
\end{itemize}

\textbf{5. Chest Circumference}

\begin{itemize}
\tightlist
\item
  \textbf{Less frequently used}.
\item
  Normally, head circumference exceeds chest circumference at birth;
  both become equal by 1 year.
\item
  May help in nutritional assessments.
\end{itemize}

\textbf{6. Body Mass Index (BMI)}

\begin{itemize}
\tightlist
\item
  \textbf{Formula}: BMI = Weight (kg) / Height² (m²).
\item
  \textbf{Use}: Detects overweight and obesity.
\item
  \textbf{Interpretation (children ≥5 years)}:

  \begin{itemize}
  \tightlist
  \item
    \textless5th percentile: Underweight.
  \item
    5th--85th percentile: Normal.
  \item
    85th--95th percentile: Overweight.
  \item
    95th percentile: Obese.
  \end{itemize}
\end{itemize}

\section{Anthropometric Indices and
Indicators}\label{anthropometric-indices-and-indicators}

These indices compare the child's measurement with reference values to
classify nutritional status.

\textbf{1. Weight-for-Age (WFA)}

\begin{itemize}
\tightlist
\item
  Detects underweight.
\item
  Sensitive to both acute and chronic malnutrition.
\item
  Limitation: Does not distinguish between stunting and wasting
\end{itemize}

\textbf{2. Height-for-Age (HFA)}

\begin{itemize}
\tightlist
\item
  Reflects linear growth.
\item
  Low HFA = \textbf{Stunting} (chronic malnutrition).
\item
  Not useful for detecting acute malnutrition.
\end{itemize}

\textbf{3. Weight-for-Height (WFH)}

\begin{itemize}
\tightlist
\item
  Identifies \textbf{wasting} (acute malnutrition).
\item
  Independent of age.
\item
  Used in emergencies and hospital settings.
\end{itemize}

\textbf{4. BMI-for-Age}

\begin{itemize}
\tightlist
\item
  Preferred index for children over 5 years.
\item
  Classifies thinness, normal weight, overweight, and obesity.
\end{itemize}

\textbf{5. Head Circumference-for-Age}

\begin{itemize}
\tightlist
\item
  Used in infants to assess brain development and detect congenital
  anomalies or infections (e.g., hydrocephalus, microcephaly).
\end{itemize}

\section{WHO Growth Standards}\label{who-growth-standards}

\begin{itemize}
\tightlist
\item
  WHO standards are based on healthy children from multiple countries,
  including Ghana.
\item
  Charts available for boys and girls separately.
\item
  Include percentiles and \textbf{Z-scores} (standard deviations from
  the median).
\end{itemize}

\textbf{Z-Score Interpretation:}

\begin{longtable}[]{@{}ll@{}}
\toprule\noalign{}
\endhead
\bottomrule\noalign{}
\endlastfoot
\textbf{Z-score} & \textbf{Classification} \\
≥ --1 SD to ≤ +1 SD & Normal growth \\
\textless{} --2 SD & Moderate malnutrition \\
\textless{} --3 SD & Severe malnutrition \\
\textgreater{} +2 SD & Overweight \\
\textgreater{} +3 SD & Obese \\
\end{longtable}

\textbf{Z-scores} are preferred over percentiles for clinical and public
health use because they are more statistically robust.

\section{Anthropometry in Ghana: Local
Context}\label{anthropometry-in-ghana-local-context}

\textbf{Nutritional Issues in Ghana}

\begin{itemize}
\tightlist
\item
  \textbf{Undernutrition}: Common in Northern and some rural regions due
  to food insecurity.
\item
  \textbf{Stunting}: Affects \textasciitilde19\% of children under 5
  (per recent DHS data).
\item
  \textbf{Wasting}: Acute malnutrition is less common but serious in
  emergencies.
\item
  \textbf{Overweight/Obesity}: Emerging problem in urban areas
\end{itemize}

\textbf{Common Causes:}

\begin{itemize}
\tightlist
\item
  Poverty, food insecurity, and poor weaning practices.
\item
  Frequent infections (e.g., malaria, diarrhea).
\item
  Inadequate maternal education.
\item
  Cultural beliefs affecting feeding.
\end{itemize}

\textbf{Public Health Programs:}

\begin{itemize}
\tightlist
\item
  Child Welfare Clinics (CWC): Regular growth monitoring, including
  weight and MUAC measurements.
\item
  Community-Based Management of Acute Malnutrition (CMAM).
\item
  School feeding programs.
\item
  Health education on infant and young child feeding (IYCF).
\end{itemize}

\section{Clinical Applications}\label{clinical-applications}

\subsection{Case Scenarios:}\label{case-scenarios}

\textbf{Case 1: Underweight Child}

\begin{itemize}
\tightlist
\item
  \textbf{Age}: 18 months
\item
  \textbf{Weight}: 6.5 kg
\item
  \textbf{WFA Z-score}: --3.2
\item
  \textbf{MUAC}: 11.2 cm
\item
  \textbf{Diagnosis}: Severe underweight and severe acute malnutrition.
\item
  \textbf{Action}: Admit to NRU (Nutritional Rehabilitation Unit);
  initiate therapeutic feeding
\end{itemize}

\textbf{Case 2: Overweight Child}

\begin{itemize}
\tightlist
\item
  \textbf{Age}: 10 years
\item
  \textbf{Weight}: 40 kg
\item
  \textbf{Height}: 1.35 m
\item
  \textbf{BMI}: 21.9 → \textgreater95th percentile
\item
  \textbf{Diagnosis}: Childhood obesity
\item
  \textbf{Action}: Diet and lifestyle counseling; screen for
  comorbidities like hypertension, type 2 diabetes.
\end{itemize}

\section{Challenges in Anthropometric Assessment in
Ghana}\label{challenges-in-anthropometric-assessment-in-ghana}

\begin{itemize}
\tightlist
\item
  \textbf{Equipment shortages}: In rural clinics, proper weighing scales
  or stadiometers may be lacking.
\item
  \textbf{Lack of training}: Some healthcare workers and students may
  not receive adequate training in accurate measurement techniques.
\item
  \textbf{Poor record-keeping}: Growth monitoring charts are often
  incomplete or misinterpreted.
\item
  \textbf{Cultural barriers}: Some communities resist exposing children
  for weighing or measurement.
\item
  \textbf{Inconsistent standards}: Some facilities still use outdated or
  non-standard growth charts.
\end{itemize}

\section{Tips for Medical Students}\label{tips-for-medical-students}

\begin{enumerate}
\def\labelenumi{\arabic{enumi}.}
\tightlist
\item
  \textbf{Practice correct technique}: Learn hands-on from skilled
  clinicians.
\item
  \textbf{Use WHO charts}: Understand how to plot and interpret
  Z-scores.
\item
  \textbf{Observe growth trends}: One-time measurements are less
  informative than trends over time.
\item
  \textbf{Correlate with clinical findings}: Anthropometry should
  complement physical exam and dietary history.
\item
  \textbf{Educate caregivers}: Explain growth status in simple language;
  encourage regular CWC visits.
\end{enumerate}

\textbf{Summary Table of Key Measures}

\begin{longtable}[]{@{}
  >{\raggedright\arraybackslash}p{(\linewidth - 8\tabcolsep) * \real{0.1905}}
  >{\raggedright\arraybackslash}p{(\linewidth - 8\tabcolsep) * \real{0.1429}}
  >{\raggedright\arraybackslash}p{(\linewidth - 8\tabcolsep) * \real{0.1810}}
  >{\raggedright\arraybackslash}p{(\linewidth - 8\tabcolsep) * \real{0.1905}}
  >{\raggedright\arraybackslash}p{(\linewidth - 8\tabcolsep) * \real{0.2952}}@{}}
\toprule\noalign{}
\endhead
\bottomrule\noalign{}
\endlastfoot
\textbf{Measurement} & \textbf{Age Group} & \textbf{Tool} &
\textbf{Indicator} & \textbf{Interpretation} \\
Weight & All & Infant/Beam Scale & WFA, WFH, BMI & Underweight, wasting,
obesity \\
Length & \textless2 yrs & Infantometer & HFA & Stunting \\
Height & ≥2 yrs & Stadiometer & HFA, BMI & Stunting, overweight \\
MUAC & 6--59 months & MUAC tape & Acute malnutrition & SAM, MAM \\
Head Circumference & 0--2 yrs & Measuring tape & HC-for-age &
Micro/macrocephaly \\
\end{longtable}

\section{Conclusion}\label{conclusion-1}

Pediatric anthropometry is an indispensable clinical tool for assessing
child health and nutrition. In the Ghanaian context, it is vital for
early detection of malnutrition and guiding appropriate interventions.
As a medical student, mastering these measurements, understanding their
interpretation, and applying them in both clinical and public health
settings are crucial skills. Consistent, accurate anthropometric
assessment can drastically improve child survival and long-term
developmental outcomes in Ghana.

\bookmarksetup{startatroot}

\chapter{Acid-Base Disorders}\label{acid-base-disorders}

\section{Introduction}\label{introduction-3}

Acid-base balance is vital for normal cellular metabolism and
physiological function. In children, acid-base disturbances can arise
from a variety of causes and often signal serious underlying pathology.
The developing physiology of infants and children also makes them
particularly vulnerable to imbalances.

This guide aims to provide medical students, particularly in Ghana, with
a comprehensive understanding of acid-base disorders, their causes,
clinical manifestations, diagnosis, and management, with a focus on
conditions commonly encountered in paediatric practice in
resource-limited settings.

\section{Physiology of Acid-Base
Balance}\label{physiology-of-acid-base-balance}

\subsection{Normal pH and Buffer}\label{normal-ph-and-buffer}

\begin{itemize}
\tightlist
\item
  Normal arterial blood pH: \textbf{7.35 -- 7.45}
\item
  Key buffer systems:

  \begin{itemize}
  \tightlist
  \item
    \textbf{Bicarbonate buffer} (HCO₃⁻ / H₂CO₃)
  \item
    \textbf{Phosphate buffer}
  \item
    \textbf{Protein buffer (e.g., haemoglobin)}
  \end{itemize}
\end{itemize}

\subsection{Regulation Mechanisms}\label{regulation-mechanisms}

\begin{enumerate}
\def\labelenumi{\arabic{enumi}.}
\tightlist
\item
  \textbf{Lungs}: Excrete CO₂ (volatile acid)
\item
  \textbf{Kidneys}: Reabsorb bicarbonate and excrete H⁺ (non-volatile
  acids)
\item
  \textbf{Buffers}: Immediate but temporary pH regulation
\end{enumerate}

\section{Classification of Acid-Base
Disorders}\label{classification-of-acid-base-disorders}

Acid-base disorders are classified as:

\begin{enumerate}
\def\labelenumi{\arabic{enumi}.}
\tightlist
\item
  \textbf{Metabolic Acidosis}
\item
  \textbf{Metabolic Alkalosis}
\item
  \textbf{Respiratory Acidosis}
\item
  \textbf{Respiratory Alkalosis}
\end{enumerate}

Each has \textbf{compensatory mechanisms} that attempt to restore pH
toward normal.

\section{Metabolic Acidosis}\label{metabolic-acidosis}

\subsection{Definition}\label{definition}

Characterized by \textbf{decreased pH and bicarbonate} (\textless{} 22
mmol/L)

\subsection{Causes}\label{causes}

\textbf{In Ghana and other resource-limited settings, common causes
include:}

\begin{longtable}[]{@{}
  >{\raggedright\arraybackslash}p{(\linewidth - 2\tabcolsep) * \real{0.5667}}
  >{\raggedright\arraybackslash}p{(\linewidth - 2\tabcolsep) * \real{0.4333}}@{}}
\toprule\noalign{}
\endhead
\bottomrule\noalign{}
\endlastfoot
\textbf{High Anion Gap} & \textbf{Normal Anion Gap (Hyperchloremic)} \\
Diabetic ketoacidosis (DKA) & Diarrhea (bicarbonate loss) \\
Lactic acidosis (sepsis, hypoxia, severe anaemia) & Renal tubular
acidosis (RTA) \\
Inborn errors of metabolism & Early renal failure \\
Uraemia & Use of carbonic anhydrase inhibitors \\
\end{longtable}

\textbf{Anion Gap (AG) = Na⁺ -- (Cl⁻ + HCO₃⁻)}\\
Normal AG: 8 -- 12 mmol/L

\subsection{Clinical Features}\label{clinical-features}

\begin{itemize}
\tightlist
\item
  Kussmaul breathing (deep, rapid)
\item
  Lethargy, confusion
\item
  Hypotension
\item
  Signs of dehydration
\end{itemize}

\subsection{Diagnosis}\label{diagnosis}

\begin{itemize}
\tightlist
\item
  Arterial blood gas (ABG): ↓pH, ↓HCO₃⁻
\item
  Serum electrolytes
\item
  Urine analysis (in RTA)
\item
  Blood glucose and ketones (in DKA)
\end{itemize}

\subsection{Management}\label{management}

\begin{itemize}
\tightlist
\item
  Treat the \textbf{underlying cause}
\item
  Rehydration (e.g., with normal saline)
\item
  \textbf{DKA}: Insulin therapy, fluids, potassium replacement
\item
  \textbf{Severe acidosis (pH \textless{} 7.1)}: Consider sodium
  bicarbonate cautiously
\item
  Monitor electrolytes, especially \textbf{K⁺}
\end{itemize}

\section{Metabolic Alkalosis}\label{metabolic-alkalosis}

\subsection{Definition}\label{definition-1}

Elevated pH and bicarbonate (\textgreater{} 28 mmol/L)

\subsection{Causes}\label{causes-1}

\begin{longtable}[]{@{}ll@{}}
\toprule\noalign{}
\endhead
\bottomrule\noalign{}
\endlastfoot
\textbf{Chloride-Responsive} & \textbf{Chloride-Resistant} \\
Vomiting or nasogastric suction & Primary hyperaldosteronism \\
Diuretic therapy & Congenital adrenal hyperplasia \\
Volume depletion & Severe hypokalemia \\
\end{longtable}

\subsection{Clinical Features}\label{clinical-features-1}

\begin{itemize}
\tightlist
\item
  Muscle cramps, weakness
\item
  Tetany (due to hypocalcemia)
\item
  Hypoventilation (compensatory)
\item
  Confusion, seizures (severe)
\end{itemize}

\subsection{Diagnosis}\label{diagnosis-1}

\begin{itemize}
\tightlist
\item
  ABG: ↑pH, ↑HCO₃⁻
\item
  Serum electrolytes: Look for \textbf{hypokalemia},
  \textbf{hypochloremia}
\item
  Urine chloride
\end{itemize}

\subsection{Management}\label{management-1}

\begin{itemize}
\tightlist
\item
  Volume replacement with normal saline
\item
  Potassium supplementation
\item
  Correct underlying cause
\item
  If resistant to saline: consider aldosterone antagonists (e.g.,
  spironolactone)
\end{itemize}

\section{Respiratory Acidosis}\label{respiratory-acidosis}

\subsection{Definition}\label{definition-2}

↓pH and ↑pCO₂ (\textgreater{} 45 mmHg)

\subsection{Causes}\label{causes-2}

\textbf{Due to hypoventilation:}

\begin{itemize}
\tightlist
\item
  CNS depression (head injury, infections)
\item
  Neuromuscular disorders (e.g., Guillain-Barré syndrome)
\item
  Chest wall deformities
\item
  Airway obstruction (e.g., asthma, foreign body)
\item
  Respiratory muscle fatigue
\end{itemize}

\subsection{Clinical Features}\label{clinical-features-2}

\begin{itemize}
\tightlist
\item
  Altered mental status
\item
  Headache
\item
  Tachycardia
\item
  Cyanosis
\item
  Papilledema (chronic)
\end{itemize}

\subsection{Diagnosis}\label{diagnosis-2}

\begin{itemize}
\item
  ABG: ↓pH, ↑pCO₂
\item
  Evaluate oxygenation (PaO₂)
\item
  Chest X-ray, pulmonary function tests (if available)
\end{itemize}

\subsection{Management}\label{management-2}

\begin{itemize}
\tightlist
\item
  Support ventilation (e.g., oxygen, non-invasive or mechanical
  ventilation)
\item
  Treat the \textbf{underlying cause} (e.g., bronchodilators for asthma)
\item
  Caution: Over-oxygenation can suppress respiratory drive in chronic
  cases
\end{itemize}

\section{Respiratory Alkalosis}\label{respiratory-alkalosis}

\subsection{Definition}\label{definition-3}

\begin{itemize}
\tightlist
\item
  ↑pH and ↓pCO₂ (\textless{} 35 mmHg)
\end{itemize}

\subsection{Causes}\label{causes-3}

\begin{itemize}
\tightlist
\item
  Anxiety, pain (hyperventilation)
\item
  Fever
\item
  Sepsis
\item
  Salicylate poisoning (early)
\item
  Central causes (e.g., meningitis)
\item
  High altitude (rare in Ghana)
\end{itemize}

\subsection{Clinical Features}\label{clinical-features-3}

\begin{itemize}
\tightlist
\item
  Light-headedness, dizziness
\item
  Perioral numbness
\item
  Muscle cramps
\item
  Tachypnoea
\end{itemize}

\subsection{Diagnosis}\label{diagnosis-3}

\begin{itemize}
\tightlist
\item
  ABG: ↑pH, ↓pCO₂
\item
  Serum calcium and phosphate (often decreased)
\end{itemize}

\subsection{Management}\label{management-3}

\begin{itemize}
\tightlist
\item
  Address the underlying cause
\item
  Calm the child (rebreathing bag if appropriate)
\item
  Treat fever or infections
\item
  Sedation may be necessary in extreme anxiety
\end{itemize}

\section{Mixed Acid-Base Disorders}\label{mixed-acid-base-disorders}

Children can present with \textbf{more than one disorder}
simultaneously, especially in critical illness.

\textbf{Examples:}

\begin{itemize}
\tightlist
\item
  \textbf{DKA with vomiting} → Metabolic acidosis + metabolic alkalosis
\item
  \textbf{Sepsis with respiratory failure} → Metabolic acidosis +
  respiratory acidosis
\end{itemize}

\textbf{Clues to Mixed Disorders:}

\begin{itemize}
\tightlist
\item
  pH is normal, but CO₂ and HCO₃⁻ are abnormal
\item
  Compensation appears inadequate or excessive
\end{itemize}

Use \textbf{Winter's formula} to assess expected respiratory
compensation in metabolic acidosis:

\textbf{Expected pCO₂ = (1.5 × HCO₃⁻) + 8 ± 2}

\section{Pediatric Considerations}\label{pediatric-considerations}

\begin{itemize}
\tightlist
\item
  \textbf{Neonates} have immature kidneys → limited ability to excrete
  acid
\item
  \textbf{Dehydration} is a common cause of acid-base disturbances
\item
  \textbf{Malaria, severe diarrhoea, and pneumonia} are leading
  paediatric conditions in Ghana that may present with acid-base
  imbalance
\end{itemize}

\section{Laboratory Evaluation}\label{laboratory-evaluation}

Key investigations:

\begin{enumerate}
\def\labelenumi{\arabic{enumi}.}
\tightlist
\item
  \textbf{ABG Analysis}
\item
  \textbf{Serum electrolytes} (Na⁺, K⁺, Cl⁻, HCO₃⁻)
\item
  \textbf{Anion gap calculation}
\item
  \textbf{Urine pH and electrolytes} (in RTA)
\item
  \textbf{Lactate}, \textbf{ketones}, \textbf{glucose}
\end{enumerate}

\section{Approach to a Child with Suspected Acid-Base
Disorder}\label{approach-to-a-child-with-suspected-acid-base-disorder}

\begin{enumerate}
\def\labelenumi{\arabic{enumi}.}
\tightlist
\item
  \textbf{Assess airway, breathing, and circulation (ABC)}
\item
  \textbf{Clinical history}

  \begin{itemize}
  \tightlist
  \item
    Diarrhoea, vomiting, fever, polyuria
  \item
    Diabetes, drug use
  \end{itemize}
\item
  \textbf{Examination}

  \begin{itemize}
  \tightlist
  \item
    Level of consciousness
  \item
    Respiratory pattern (Kussmaul, hypoventilation)
  \item
    Signs of dehydration or oedema
  \end{itemize}
\item
  \textbf{ABG + Electrolytes}
\item
  \textbf{Determine primary disorder and compensation}
\item
  \textbf{Treat the cause and monitor}
\end{enumerate}

\section{Resource-Limited Considerations (Ghana
Context)}\label{resource-limited-considerations-ghana-context}

\begin{itemize}
\tightlist
\item
  ABGs may not be widely available --- rely on \textbf{clinical signs},
  \textbf{serum bicarbonate}, \textbf{venous blood gases}
\item
  In emergencies, treat based on likely diagnosis (e.g., give fluids for
  suspected DKA even before lab confirmation)
\item
  \textbf{Point-of-care testing} (glucometers, lactate meters) can aid
  rapid decision-making
\end{itemize}

\section{Summary Table}\label{summary-table}

\begin{longtable}[]{@{}
  >{\raggedright\arraybackslash}p{(\linewidth - 8\tabcolsep) * \real{0.2805}}
  >{\raggedright\arraybackslash}p{(\linewidth - 8\tabcolsep) * \real{0.0976}}
  >{\raggedright\arraybackslash}p{(\linewidth - 8\tabcolsep) * \real{0.1341}}
  >{\raggedright\arraybackslash}p{(\linewidth - 8\tabcolsep) * \real{0.2073}}
  >{\raggedright\arraybackslash}p{(\linewidth - 8\tabcolsep) * \real{0.2805}}@{}}
\toprule\noalign{}
\endhead
\bottomrule\noalign{}
\endlastfoot
\textbf{Disorder} & \textbf{pH} & \textbf{HCO₃⁻} & \textbf{pCO₂} &
\textbf{Compensation} \\
Metabolic Acidosis & ↓ & ↓ & ↓ (respiratory) & Hyperventilation \\
Metabolic Alkalosis & ↑ & ↑ & ↑ & Hypoventilation \\
Respiratory Acidosis & ↓ & ↑ (renal) & ↑ & Renal HCO₃⁻ retention \\
Respiratory Alkalosis & ↑ & ↓ (renal) & ↓ & Renal HCO₃⁻ loss \\
\end{longtable}

\section{Conclusion}\label{conclusion-2}

Understanding acid-base disorders in children is crucial for early
recognition and effective treatment, particularly in acute settings like
emergency departments and paediatric wards. In Ghana, common
contributors include dehydration from diarrhoea, infections, and
diabetic ketoacidosis (DKA). A systematic clinical and laboratory
approach allows timely diagnosis and management, even in
resource-constrained environments.

\part{{Basic Critical Care}}

\chapter{Paediatric Critical Care}\label{paediatric-critical-care}

\section{Background}\label{background}

In resource-limited resource settings, such as those lacking access to
healthcare staff, equipment, and resources, pediatric emergency and
critical care (PECC) play an essential role. These settings bear the
highest burden of severe acute illness and life-threatening injuries.
Still, due to underappreciation of its importance, critical care has not
been incorporated as an essential part of the health system.(Sakaan et
al. 2022)

An infant, child, or adolescent with an illness, injury, or
post-operative state that increases the risk for or results in acute
physiological instability (abnormal vital signs, laboratory values, or
clinical findings) falls under acute paediatric critical illness.(Appiah
et al. 2018)\\
\strut \\
Many paediatric lives can be saved with low-cost resuscitation
techniques and procedures, supportive care and referral pathways
available at all levels of care. Over sixty per cent of children die
because of poor emergency and critical care. SDG may not be achieved
unless PECC service is available to every acutely ill child.

By providing prompt and adequate medical care to critically ill
children, paediatric emergency and critical care services can contribute
to a decrease in childhood death rates.

\section{Reasons for PECC}\label{reasons-for-pecc}

Children who require specialised care due to severe diseases or injuries
should ideally be referred or managed at the paediatric emergency or
critical care (PICU). Any space available should and can be used to
offer such services. In that sense, any clinician can advocate for or
institute measures to provide PECC.

The following are typical reasons for PICU admissions:

\begin{itemize}
\tightlist
\item
  \textbf{Acute respiratory distress and failure} e.g., (acute
  respiratory distress syndrome (ARDS), pneumonia, severe exacerbations
  of asthma, or any cause are all considered respiratory distress.
\item
  \textbf{Shock}: hypovolemic shock, septic shock, (e.g., from severe
  dehydration or haemorrhage). cardiogenic shock (e.g., from myocarditis
  or heart failure),
\item
  \textbf{Trauma}: accidental and non-accidental polytrauma injuries,
  burns, or other causes.
\item
  \textbf{Neurological emergencies}: Seizures, severe brain damage,
  cerebral bleeding, meningitis, or encephalitis are examples.
\item
  \textbf{Cardiovascular}: Heart failure, myocarditis, arrhythmias, and
  exacerbations of congenital heart disease are examples.
\item
  \textbf{Sepsis}: Systemic infections caused by bacteria, viruses, or
  fungi that result in septic shock or sepsis.
\item
  \textbf{Metabolic disorders} include metabolic acidosis, severe
  electrolyte imbalances, diabetic ketoacidosis (DKA), and inborn
  metabolic abnormalities.
\item
  Conditions that result in the failure or dysfunction of numerous organ
  systems, including multiple organ failure, are called~multi-organ
  dysfunction.
\item
  \textbf{Post-operative support}: Children undergoing difficult
  procedures, especially those requiring close cardiovascular monitoring
  of organ function.
\item
  \textbf{Haematological and oncological disorders}: coagulation issues,
  thrombocytopenia, or severe anaemia, side effects of cancer treatment,
  such as tumour lysis syndrome or neutropenia caused~by chemotherapy.
\item
  Exposure to poisonous substances, overdosing, or poisoning are
  examples of \textbf{toxicological} emergencies.
\item
  \textbf{Endocrine}: include thyroid storm, adrenal crises, and
  diabetic emergencies.\\
  Gastrointestinal: intussusception, typhoid ileal perforation,
  necrotizing enterocolitis, or extreme dehydration are examples of.
\item
  \textbf{Renal} \textbf{disorders:} include electrolyte abnormalities,
  acute kidney damage, and renal failure.
\end{itemize}

\section{Evaluation of critically
ill}\label{evaluation-of-critically-ill}

A methodical and comprehensive approach is necessary when evaluating a
severely unwell child to promptly recognize and treat life-threatening
diseases. The~clinical plan that is frequently applied in emergency and
critical care settings for children:

Initial assessment and stabilisation

\textbf{ABCDE technique}: This methodical technique entails evaluating
the airway, breathing, circulation, disability, and exposure.

A -- check for airway patency. Determine at risk or not at risk of
obstruction.

B -- the breathing sufficient

C -- assess circulation for perfusion

D -- evaluate the neurological status using GCS or AVPU

E -- expose the infant while keeping them warm.\\

\hfill\break
Vital Signs: Take your blood pressure, temperature, heart rate,
breathing rate, oxygen saturation and pain score assessment.

During this assessment, any irregularities ought to be taken care of
right away before the next category is done.

\subsection{History}\label{history}

A focused history that covers the child's presenting complaints, medical
history, current medications, recent illnesses, immunization status, and
events leading up to the current presentation should be obtained from
caregivers.\\
Inquire about symptoms like fever, coughing, breathing problems
diarrhoeaa, vomiting, lethargy, seizures, trauma, or toxin intake.

\subsection{Physical examination}\label{physical-examination-1}

Examine the child thoroughly, beginning with a general assessment of
their look and state of consciousness.\\
Perform a comprehensive examination from head to toe, encompassing
evaluation of the respiratory, neurological, gastrointestinal,
abdominal, cardiovascular, renal, and musculoskeletal systems.\\
Keep an eye out for symptoms of trauma, dehydration, altered mental
status, cyanosis, irregular breath sounds, abnormal heart sounds, and
respiratory distress.

\subsection{Focused assessment and
investigations}\label{focused-assessment-and-investigations}

\begin{itemize}
\item
  Prioritize testing including blood tests (e.g., complete blood count,
  electrolytes, blood gas analysis), imaging studies (e.g., chest X-ray,
  ultrasound, CT scan), and
\item
  other tests (e.g., ECG, echocardiogram, lumbar puncture) based on the
  first assessment and history.\\
  If available and appropriate, take into consideration bedside testing
  such as point-of-care ultrasound or quick diagnostics for infectious
  disorders.
\end{itemize}

\subsection{Management and treatment}\label{management-and-treatment}

Based on the results of the diagnostic tests and clinical findings,
begin the necessary treatment. This can involve the use of supportive
care measures, oxygen therapy, fluid resuscitation, and drug delivery
(such as antibiotics, inotropes, analgesia and sedatives).

Prioritize treating life-threatening disorders including shock,
respiratory failure, and obstruction of the airway.

\subsection{Monitoring and
reassessment}\label{monitoring-and-reassessment}

Monitoring is a cardinally important part of critical care. It allows
for evaluation of the patient's response to interventions. Patients with
critical illness or injury require more frequent monitoring in the first
24-48 to identify deterioration patients or those not responding to
treatment. Keep a close eye on the child's vital signs, reaction to
therapies, and general state of health.

Regularly reevaluate the child and modify the management plan in
response to treatment response and changes in the child's clinical
status. If resources permit all patients should be on patient monitor
for continuous vital signs evaluation. Additional staffing may be
required or assigned to monitor and alert other clinicians of abnormal
values. This will ensure a timely response.

\subsection{Multi-disciplinary and
referral}\label{multi-disciplinary-and-referral}

Critical illness may involve or result or lead to multiple pathologies
as a result multidisciplinary team approach to care cannot be
over-emphasised. If additional evaluation and management are needed,
consult with the appropriate specialists (paediatric surgeons,
paediatric intensivists, neurologists, infectious disease specialists,
etc.).

MDT in critical care involves discussing patients in detail with the
appropriate specialist(s) to determine the best care for the desired
outcome. If the child's condition calls for specialized interventions or
resources not provided by the current facility, consider transferring to
a higher level of care.

\subsection{Family centred care}\label{family-centred-care}

The family's role in providing care for a sick child must always be
considered during management. Recall that the patient is best known by
their parents, guardians, or relatives. Their engagement from the
patient's history to recognizing and comprehending their condition will
be optimal.

Involve caregivers in decision-making wherever possible, respecting
their choices and cultural views. Communicate effectively with the
child's caregivers, giving them updates on the child's status, outlining
the planned interventions, and answering any worries or questions they
may have.

\section{Interventions}\label{interventions}

Children who are critically ill require emergent and timely
life-sustaining interventions. Resuscitation and supportive care are two
interventions that are usually utilized to revive patients and assist
vital organs to recover until they can function within the survivable
level.

\subsection{Resuscitation}\label{resuscitation}

The process of treating a critically ill patient's physiological
abnormalities is known as resuscitation, and it occurs in every hospital
department. It involves various~methods that call for a broad range of
capabilities that have been systematically developed and applied over
time.(Arias et al. 2024)

It refers to the act of reviving someone from apparent death or
unconsciousness because of cardiorespiratory arrest, often involving
procedures such as chest compression, ventilation, cardiac massage, and
the use of a defibrillator.(Lewis and McConnell 2018)

Clinical presentation of cardiac arrest in children presents differently
from adults. Cardiac (shock), respiratory (respiratory tract infection)
and gastrointestinal (diarrhoea and vomiting) conditions are the common
predisposing risks that get children into cardiorespiratory
arrest.(Meghani 2021) Children frequently move to cardiac arrest through
respiratory or circulatory failure, the pre-arrest period in children is
typically characterized by respiratory distress or failure, shock, or a
combination of these factors.

\subsection{Critical organ dysfunction or
failure}\label{critical-organ-dysfunction-or-failure}

Manifestation of children's pre-cardiac arrest stage includes:

\begin{enumerate}
\def\labelenumi{\arabic{enumi}.}
\tightlist
\item
  \textbf{Respiratory distress}

  \begin{enumerate}
  \def\labelenumii{\alph{enumii}.}
  \tightlist
  \item
    \textbf{Increased work of breathing}: Using auxiliary muscles, nasal
    flaring, grunting, and retractions (intercostal, subcostal, and
    suprasternal).
  \item
    \textbf{Abnormal breath} sounds including crackles, stridor,
    wheezing, or reduced breath sounds.
  \item
    \textbf{Alterations in breathing pattern}: Bradypnea (slow
    breathing) or tachypnoea (fast breathing).
  \end{enumerate}
\item
  \textbf{Respiratory failure}

  \begin{enumerate}
  \def\labelenumii{\alph{enumii}.}
  \tightlist
  \item
    \textbf{Cyanosis:} Bluish discolouration of the skin, especially
    around the lips and fingertips, indicating hypoxemia.
  \item
    \textbf{Altered Mental Status:} Lethargy, irritability, or
    unresponsiveness due to inadequate oxygenation.
  \item
    \textbf{Apnoea:} Periods of stopped breathing or significant pauses
    in breathing.
  \end{enumerate}
\item
  \textbf{Circulatory impairment/failure (shock)}

  \begin{enumerate}
  \def\labelenumii{\alph{enumii}.}
  \tightlist
  \item
    \textbf{Tachycardia} or abnormally rapid heart rate is frequently
    the result of an early compensatory mechanism.
  \item
    \textbf{Hypotension}: Low blood pressure in children is a late and
    concerning sign.
  \item
    \textbf{Poor perfusion}: Cold, clammy skin; longer than two seconds
    for capillary refill; sluggish or non-existent peripheral pulses.
  \item
    \textbf{Altered mental status}: Incomprehension, agitation, or a
    reduction in reactivity.
  \end{enumerate}
\item
  \textbf{Neurological dysfunction}

  \begin{enumerate}
  \def\labelenumii{\alph{enumii}.}
  \tightlist
  \item
    Altered Mental State: Diminished level of alertness, agitation, or
    lethargy.
  \item
    Seizures: In children with known seizure disorders, a new or
    alterations in seizure patterns.
  \end{enumerate}
\item
  \textbf{Vital Sign Changes}

  \begin{enumerate}
  \def\labelenumii{\alph{enumii}.}
  \tightlist
  \item
    \textbf{Abnormal Heart Rate:} Tachycardia is common, whilst
    bradycardia is usually a late sign suggesting severe deterioration.
  \item
    \textbf{Abnormal Respiratory Rate:} Both tachypnoea and bradypnea
    are worrying.
  \item
    \textbf{Hypotension:} in children, it is frequently a late sign of
    shock.
  \end{enumerate}
\item
  \textbf{Laboratory and Monitoring Data:}

  \begin{enumerate}
  \def\labelenumii{\arabic{enumii}.}
  \item
    \textbf{Hypoxemia:} arterial saturation as measured by pulse
    oximetry.
  \item
    \textbf{Hypercapnia:} Elevated carbon dioxide readings, indicating
    inadequate ventilation.
  \item
    \textbf{Metabolic Acidosis:} Blood gas analysis shows a drop in
    blood pH and bicarbonate levels, indicating poor tissue perfusion
    and oxygenation.
  \end{enumerate}
\end{enumerate}

Management of pre-cardiac arrest and cardiac arrest involves prevention
by identifying patients at risk, monitoring and managing complications,
cardiopulmonary resuscitation, and in advanced cases supportive care,

\subsection{Early Recognition and
Monitoring:}\label{early-recognition-and-monitoring}

\begin{itemize}
\tightlist
\item
  Frequent or continuous monitoring of vital signs, including HR, RR,
  CRT, and SpO2 with or without BP.
\item
  Repeated assessments of mental status and perfusion indicators.
\item
  Following assessment and monitoring, any abnormalities should be
  responded to with appropriate interventions such as low SpO2 should be
  given oxygen and respiratory support for respiratory failure.
\end{itemize}

\subsection{Supportive therapy}\label{supportive-therapy}

Aside from resuscitation among patients with cardiac arrest, those in
pre-arrest or return of spontaneous circulation post-resuscitation need
supportive care.

\textbf{Respiratory Support:}

\begin{itemize}
\tightlist
\item
  Providing oxygen therapy, non-invasive ventilation (e.g., CPAP or
  BiPAP), or mechanical ventilation if needed.
\item
  Clearing airway obstructions and ensuring proper airway management.
\end{itemize}

\textbf{Circulatory Support:}

\begin{itemize}
\tightlist
\item
  Administering fluids judiciously to treat hypovolemia.
\item
  Using vasoactive medications like epinephrine or norepinephrine for
  shock.
\end{itemize}

\textbf{Treating Underlying Causes:}

\subsection{Optimisation of oxygen
delivery}\label{optimisation-of-oxygen-delivery}

Prevention of cardiac arrest or restoration after cardiac arrest of a
patient to health requires optimisation of the oxygen-carrying capacity
of blood and flow of blood to the tissues. These require

\begin{enumerate}
\def\labelenumi{\arabic{enumi}.}
\tightlist
\item
  Provide oxygen to the lungs so that red blood cells can take it
\item
  Sufficient haemoglobin levels to bind oxygen
\item
  Improve cardiac contractility to guarantee enough blood pumping
  optimum vascular tone for all tissue beds to be perfused
\item
  Effective intravascular volume to provide sufficient blood flow to the
  heart's right side.
\end{enumerate}

Children with critical illness require early identification, institution
of timely intervention, assessment and effective monitoring.
~Understanding the complex interaction of clinical, infrastructural,
workforce, family, and systemic factors affects the outcomes of
pediatric critical illness in hospitals. Improving these parameters
using focused interventions, distribution of resources, education, and
modifications to policies can greatly enhance the standard of care and
survival rates for children with critical illness.

\chapter{Cardiovascular Dysfunction}\label{cardiovascular-dysfunction}

\section{Introduction}\label{introduction-4}

Cardiac and vascular dysfunction may be classified as hemodynamic
impairment or failure. Hemodynamic failure is also referred to as shock.
The cardiovascular system (CVS) comprises the blood vessels, blood and
the heart. Cardiovascular physiological reserve allows the system to
function optimally even in some disease states. When physiological
reserve is exhausted the body enters a state of dysfunction. Patients
presenting in all acute disease states must have the CVS evaluated to
identify dysfunction early, institute the appropriate treatment to
prevent irreversible complications and avoid death.

\section{Case Presentation}\label{case-presentation}

A 7-month-old female infant, presented with a 2-week history of fever
and cough, and a day's history of poor feeding, vomiting (2 episodes,
non-bloody, non-bilous), and diarrhoea associated with lethargy. She
presented to the clinic with severe respiratory distress,
SpO\textsubscript{2}-98\% on a non-rebreather mask at 15L/min.
\emph{Chest} - she had reduced air entry with coarse crackles in the
middle and lower zones, more on the Right. \emph{Circulation} - Heart
Rate was 184 beats per minute, with cold extremities, Capillary Refill
Time of 4 seconds, and weak pulse volume. Glasgow Coma Score was 14/15,
temperature was 36.7\textsuperscript{o}C. He was given intramuscular
penicillin before being transferred to the tertiary hospital. Arterial
blood gases revealed pH of 7.227, pCO\textsubscript{2} 33.1,
pO\textsubscript{2} 59, and HCO\textsubscript{3} 14.

\section{Shock}\label{shock}

\subsection{Definition}\label{definition-4}

Shock is an acute process characterised by the body's inability to
deliver adequate oxygen to meet the metabolic demands of vital organs
and tissues.

\subsection{Pathophysiology}\label{pathophysiology}

Among the key functions of the CVS is the delivery of blood to tissues,
and perfusion. Figure~\ref{fig-oxygen-delivery} summarizes the concept
of oxygen delivery and adequate blood flow to the tissues with enough
oxygen in the blood. This ensures that oxygen and nutrients are sent to
the tissues to meet metabolic demands. Blood flow is dependent on stroke
volume (mechanical function of the heart) and blood oxygen content of
blood. Failure of the CVS to perform this function results in an
imbalance in oxygen supply and tissue demand leading to anaerobic
metabolism. This alternative energy production pathway to sustain life
is associated with unwanted metabolites including lactic acid disturbing
the homeostasis environment. When prolonged and not reversed promptly it
usually leads to irreversible tissue damage and ultimately, death. Shock
is a major cause of morbidity and mortality in children and a leading
cause of emergency and intensive care unit admission. Stroke volume is
dependent on the volume of blood at the end-diastole (preload), cardiac
muscle integrity (myocardial contractility) and the resistance against
which the heart pumps (afterload) ~is influenced by Oxygen carried by
haemoglobin (Hb X Arterial oxygen saturation, SpO2) + Dissolved oxygen
in the blood (0.003 × Arterial partial pressure of oxygen,
PaO\textsubscript{2}). The oxygen delivery to tissues equation is
represented by

\begin{figure}

\centering{

\pandocbounded{\includegraphics[keepaspectratio]{images/cc-oxygen-delivery-diagram.jpg}}

}

\caption{\label{fig-oxygen-delivery}Oxygen delivery (DO2) equation}

\end{figure}%

\emph{CO = Cardiac output\textbf{,} CaO\textsubscript{2} = arterial
oxygen content, HR = Heart rate,} \emph{SV = stroke volume,
SpO\textsubscript{2} = arterial oxygen saturation, PaO\textsubscript{2}
= partial pressure of arterial oxygen tension.}

\subsection{Clinical stages of shock}\label{clinical-stages-of-shock}

In impaired hemodynamic states, the body is not able to deliver blood,
oxygen and nutrients to the tissues. This results in a cascade of
biomedical processes initiated by anaerobic metabolism and if not
corrected the processes lead to vascular endothelial injury and
irreversible damage Figure 2. Shock may be categorized into stages to
assist with diagnosis and management

\begin{longtable}[]{@{}
  >{\raggedright\arraybackslash}p{(\linewidth - 4\tabcolsep) * \real{0.2055}}
  >{\raggedright\arraybackslash}p{(\linewidth - 4\tabcolsep) * \real{0.4247}}
  >{\raggedright\arraybackslash}p{(\linewidth - 4\tabcolsep) * \real{0.3699}}@{}}
\caption{Stages of shock with characteristics and clinical
features}\label{tbl-cc-shock-stages}\tabularnewline
\toprule\noalign{}
\begin{minipage}[b]{\linewidth}\raggedright
Stage
\end{minipage} & \begin{minipage}[b]{\linewidth}\raggedright
Pathophysiology
\end{minipage} & \begin{minipage}[b]{\linewidth}\raggedright
Clinical Features
\end{minipage} \\
\midrule\noalign{}
\endfirsthead
\toprule\noalign{}
\begin{minipage}[b]{\linewidth}\raggedright
Stage
\end{minipage} & \begin{minipage}[b]{\linewidth}\raggedright
Pathophysiology
\end{minipage} & \begin{minipage}[b]{\linewidth}\raggedright
Clinical Features
\end{minipage} \\
\midrule\noalign{}
\endhead
\bottomrule\noalign{}
\endlastfoot
Pre-shock & There is impaired hemodynamic status, but the body can fall
on its physiological reserve and intrinsic autoregulation to compensate
mechanisms to attenuate its untoward undesired effect. & Tachycardia,
Tachypnoea, Cool core to periphery temperature \\
Shock & Ensues when the compensatory reserve is exhausted with
full-blown clinical manifestation of the cardiovascular system & Above
\textbf{plus} Capillary refill time \textgreater{} 3 seconds, +/-
hypotension \\
End-organ dysfunction or failure & During this stage distant organs are
affected. Usually, the terminal process results from loss of
autoregulation leading to death or permanent organ damage in patients
who survive. & \emph{Neurology} (altered level of consciousness)
\emph{Genito-urinary} (reduced urine output) \emph{Laboratory} (evidence
of organ dysfunction - liver, renal, coagulation etc.) \\
\end{longtable}

\subsection{Classification}\label{classification}

\begin{itemize}
\item
  At the bedside, shock is conveniently classified as either compensated
  or uncompensated. The former refers to the early stages of shock when
  the body can mount a response to maintain perfusion. Uncompensated
  shock refers to the loss of the body's compensatory response.
\item
  Other classifications based on core-to-peripheral body temperature are
  cold and warm shock. Clinical presentation may vary depending on the
  state of the patient and may drift in and out of these
  classifications. Without intervention and monitoring for stability,
  there should not be seen an improvement.
\end{itemize}

\begin{figure}

\centering{

\includegraphics[width=7.9in,height=8.48in]{cc-cardiovascular-dysfunction_files/figure-latex/mermaid-figure-1.png}

}

\caption{\label{fig-ccShockCascade}Simplified cascade processes
resulting from shock hypoxia, free radical formation, and hypoxia lead
to inefficient anaerobic metabolism}

\end{figure}%

\subsection{Types of Shock}\label{types-of-shock}

Based on etiology shock may grouped into 5:

\begin{enumerate}
\def\labelenumi{\arabic{enumi}.}
\tightlist
\item
  \textbf{Hypovolemic shock} - the most common cause of shock in
  paediatrics. There is a reduction in venous return (preload) as a
  result of internal or external losses. Can be from fluid
  loss/redistribution, e.g.~Vomiting, diarrhoea, burns, third-spacing,
  blood loss/hemorrhagic shock from trauma, Gastrointestinal bleeds.
\item
  \textbf{Cardiogenic shock}- cardiac pump failure secondary to poor
  myocardial function. E.g. congenital heart diseases, cardiomyopathies-
  acute myocarditis, and arrhythmias.
\item
  In \textbf{obstructive shock} any mechanical impediment to adequate
  cardiac output. Eg. Tension pneumothorax, massive pulmonary embolism,
  cardiac tamponade.
\item
  \textbf{Distributive} \textbf{shock}- inadequate vasomotor tone,
  severe vasodilatation and increased capillary permeability resulting
  in fluid moving into the interstitium. Examples include f anaphylactic
  and shock neurogenic shock.
\item
  \textbf{Septic sho}ck is usually associated with a combination of
  distributive, hypovolemic and cardiogenic shock.
\end{enumerate}

\section{Case discussion}\label{case-discussion}

According to the World Health Organization, shock may be clinically
diagnosed when there are \textbf{cold extremities, a capillary refill
time of more than 3 seconds, and a fast weak pulse}. Hypotension is
usually a late sign and not a requirement to diagnose shock promptly.
Additional signs attributed to other organs impacted are

\begin{itemize}
\tightlist
\item
  altered mental status,
\item
  low urine output,
\item
  rising liver enzymes, lactic acidosis or base deficit,
\item
  low mixed and central venous oxygen saturation.
\end{itemize}

The 7-month-old baby presents with cold extremities, tachycardia, and
slow delayed refill time consistent with shock. It is important to know
the aetiology to manage appropriately. Hypovolemia, septic shock, and
obstructive may be implicated as the causes of this child's current
condition.

\begin{itemize}
\tightlist
\item
  The history of the illness includes fever cough, poor feeding,
  vomiting and diarrhoea.
\item
  Fever, poor feeding, vomiting and diarrhoea suggest possible fluid
  losses and therefore hypovolemia.
\item
  Similarly, fever and features suggestive of infection could result
  from sepsis as a cause of the shock.
\item
  The cough, respiratory distress requiring oxygen, a hint of
  respiratory tract infection and possible complications such as
  pneumothorax.
\item
  Lastly, penicillin causes anaphylaxis hence there is a potential
  anaphylactic.
\end{itemize}

The most probable cause of this infant's shock is septic shock. Given
fever, poor feeding, vomiting and diarrhoea the child may have low
intravascular volume. ~The suspected infection has progressed to severe
disease and the release of cytokines and other immune responses at the
tissue and cellular level leading to damaged endothelium and
consequently fluid leak from the intracellular space. Inflammatory
response and endotoxins may damage vasodilation and myocardium with
myocardial dysfunction.

The combination of these processes compromises perfusion to the tissues
all tissues especially vital organs, the brain, kidney, liver and
coagulation. A four-organ system model which included criteria for
respiratory, cardiovascular, coagulation, and neurologic dysfunction has
been developed to guide the diagnosis of sepsis. Whilst diagnosing
sepsis at this point may still not be early enough, the new criteria
provide new evidence that should sensitize clinicians to pick patients
with infection who are critically ill.

Acute gastroenteritis is the most common cause of shock in children.
Patients rapidly advance to dehydration and shock because of loss of
intravascular volume

\subsection{Clinical Presentation and
Diagnosis}\label{clinical-presentation-and-diagnosis}

Shock is a clinical diagnosis based on history and examination. It is
dependent on underlying pathology but there are overlaps in
presentation.

\subsection{History}\label{history-1}

\begin{itemize}
\tightlist
\item
  Hypovolemic shock- diarrhoea, vomiting, burns, bleeding, jaundice,
  trauma
\item
  Cardiogenic shock- known patient with congenital heart disease or
  cardiomyopathies, heart failure, easy fatiguability, darkened lips and
  fingers or toes, palpitations, chest pain, diaphoresis and taking
  frequent breaks during feeds
\item
  Distributive shock- drug history, allergies, spinal surgery,
\item
  Septic shock- fever, cough, jaundice, diarrhoea, vomiting, difficulty
  in breathing, loss of consciousness, seizures,
\end{itemize}

\subsection{Examination}\label{examination}

\begin{itemize}
\tightlist
\item
  Tachycardia
\item
  Poor peripheral perfusion- cold extremities, CRT\textgreater3s, weak
  pulse volume
\item
  Respiratory distress signs- tachypnea, use of accessory muscles,
  flaring of alae nasae, subcostal recession
\item
  Altered mental status
\item
  Hypotension (a late sign)
\item
  Oliguria
\item
  Signs of dehydration
\end{itemize}

Other findings- jaundice, fever, skin lesions, chest signs of pneumonia
or pneumothorax, upper airway obstruction murmurs or other heart sounds,
signs of raised ICP, oedema, hypo/hyperglycemia, low
SpO\textsubscript{2} \textless92\%

\subsection{Approach to management}\label{approach-to-management}

Shock is an emergency! Resuscitation must therefore be timely.

\subsection{Initial assessment}\label{initial-assessment}

\begin{enumerate}
\def\labelenumi{\arabic{enumi}.}
\tightlist
\item
  Triage by following assessment and ensuring the airway is patent and
  maintainable, and breathing is appropriate for age.
\item
  Administer oxygen
\item
  Admit the patient to the emergency room.
\item
  Secure intravenous access (preferably the largest vein possible).
  Intraosseous access is equally useful when IV is difficult to obtain.
\item
  The first line of most untreated shock is fluid. Recent evidence
  suggests that large fluid boluses increase mortality.

  \begin{enumerate}
  \def\labelenumii{\arabic{enumii}.}
  \tightlist
  \item
    5-10 ml/kg over 30 minutes, this may be repeated every 30 minutes as
    needed (non-dehydration related shock). Balanced crystalloids are
    the fluid of choice (Ringer's lactate is the crystalloid of
    preferable), Normal saline may be used in patients with head
    injuries. Avoid in with or at risk of acute kidney injury.
  \end{enumerate}
\end{enumerate}

\subsection{Second survey}\label{second-survey}

Evaluate the patient for causes of shock and treat accordingly.

\begin{enumerate}
\def\labelenumi{\arabic{enumi}.}
\tightlist
\item
  Hypovolemic

  \begin{enumerate}
  \def\labelenumii{\alph{enumii}.}
  \tightlist
  \item
    Severe dehydration because of acute gastroenteritis (diarrhoea and
    vomiting). Rehydrate using WHO Plan B or C as appropriate
  \item
    Severe anaemia -- transfuse blood.
  \item
    Hemorrhage -- transfuse blood without blood products as needed
  \end{enumerate}
\item
  Monitor response to the fluid bolus every 15 -- 30 minutes.
\end{enumerate}

\subsection{Investigations}\label{investigations-1}

Once the patient has been resuscitated tailored but detailed
investigations are needed.

\subsubsection{Laboratory}\label{laboratory}

\begin{enumerate}
\def\labelenumi{\arabic{enumi}.}
\tightlist
\item
  FBC- Anemia, Leukopenia/leukocytosis, thrombocytopenia
\item
  Arterial blood gases- acidosis (metabolic/respiratory), low PaO2,
  high/low PCO2, high lactate, high anion gap
\item
  Deranged liver and renal function test
\item
  Abnormal electrolytes (low Calcium, hyper/hypokalemia)
\item
  Abnormal clotting profile
\item
  Septic screen - blood, urine, nasopharyngeal swabs, stool, CSF
\item
  Chest x-ray indicated suspected lung pathology such as pneumothorax
  and infections.
\item
  Echo/ECG -- useful for diagnosis and monitoring of shock
\item
  Neuroimaging for suspected brain abscess, meningitis and encephalitis.
\item
  Markers of inflammation/infection- CRP, ESR and procalcitonin. These
  will be determined by circumstances. It may also be used to guide
  antibiotic therapy decisions. ~
\end{enumerate}

\subsection{Additional treatment}\label{additional-treatment}

For shock not responding to fluid of 30-40 ml/kg the following may be
added.

\begin{enumerate}
\def\labelenumi{\arabic{enumi}.}
\tightlist
\item
  \textbf{Cardiogenic shock} - Inotropes to improve contractility--
  adrenaline or dobutamine.
\item
  O\textbf{bstructive shock}

  \begin{enumerate}
  \def\labelenumii{\alph{enumii}.}
  \tightlist
  \item
    Tension pneumothorax -- thoracostomy/chest tube insertion
  \item
    massive pulmonary embolism -- anticoagulants
  \item
    cardiac tamponade - pericardiocentesis
  \end{enumerate}
\item
  \textbf{Distributive} \textbf{shock} -- adrenaline, phenylephrine
\item
  \textbf{Septic shock}

  \begin{enumerate}
  \def\labelenumii{\alph{enumii}.}
  \tightlist
  \item
    For fluid refractory - noradrenaline, adrenaline, dobutamine
  \item
    For vasopressor and inotrope refractory - steroid stress dose
  \item
    Antimicrobial therapy in sepsis within an hour.
  \end{enumerate}
\end{enumerate}

\subsection{Supportive therapy}\label{supportive-therapy-1}

In inotrope-refractory shock, steroids may be given for adrenal
insufficiency.

\begin{enumerate}
\def\labelenumi{\arabic{enumi}.}
\tightlist
\item
  Intensive Care Unit admission as required
\item
  Neuroprotective measures- normothermia, normoglycemia, anti-seizure(if
  indicated), hyperosmolar therapy
\item
  Providing ventilatory support- invasive vs non-invasive
\item
  Worsening cardiac function requires invasive treatment options like
  ECMO (Extracorporeal Membrane Oxygenation) and ventricular-assisted
  devices. The underlying cause should be looked at after resuscitation
  and stabilization
\item
  Correct all electrolyte abnormalities
\item
  Nutritional rehabilitation- early initiation of enteral feeds
  associated with good outcomes
\item
  Analgesia, antipyretics, anxiolytics. Hemotransfusion if indicated.
\end{enumerate}

\section{Monitoring}\label{monitoring}

The patients must be continuously monitored as treatment is instituted.
Inotropes titrated to effect per the average values per patient age
(pulse rate, CRT, BP, RR, SPO2 and urine output). In low-resource
facilities,

Monitoring of abnormal laboratory indices for improvement/deterioration.

\section{Conclusion}\label{conclusion-3}

Shock is an emergency requiring early recognition, triaging and
treatment. Initial treatment for almost all shock is careful restrictive
fluid resuscitation. Vasoactive drugs may be required in patients who
are refractory to fluid. Patients with shock may be fit from locations
where monitoring can be meticulously adhered to.

\part{{Neonatology}}

\chapter{Neonatal History \&
Examination}\label{neonatal-history-examination}

\section{Introduction}\label{introduction-5}

The newborn period, defined as the first 28 days of life, is a critical
phase in human development. It represents a time of rapid physiological
adaptation from intrauterine to extrauterine life, with major changes
occurring in respiration, circulation, nutrition, and thermoregulation.
During this period, morbidity and mortality are highest compared to any
other stage of childhood, particularly in low- and middle-income
countries such as Ghana.

For clinicians, the neonatal history and examination are essential tools
in identifying normal adaptation, detecting abnormalities early, and
guiding timely interventions. A detailed assessment requires not only
the direct clinical examination of the neonate but also a careful review
of maternal, antenatal, intrapartum, and immediate postnatal events.

\section{Importance of Neonatal History and
Examination}\label{importance-of-neonatal-history-and-examination}

\begin{itemize}
\tightlist
\item
  Early diagnosis of congenital anomalies -- many conditions can be
  subtle at birth but become evident on detailed examination.
\item
  Assessment of perinatal risk factors -- including maternal illnesses,
  infections, complications of labour, and prematurity.
\item
  Establishing baseline health status -- for growth monitoring and
  subsequent follow-up.
\item
  Building rapport with the mother and family -- ensuring continuity of
  care.
\item
  Guiding preventive strategies -- such as immunisation, exclusive
  breastfeeding, and infection control.
\end{itemize}

\section{Components of Neonatal
History}\label{components-of-neonatal-history}

The neonatal history is unique in that it depends heavily on information
from the mother and available records, since the newborn cannot
communicate symptoms. The history should be systematic and include the
following areas:

\subsection{Maternal History}\label{maternal-history}

\textbf{Demographic and Social Factors}

\begin{enumerate}
\def\labelenumi{\arabic{enumi}.}
\tightlist
\item
  Maternal age: Teenage and advanced maternal age pregnancies carry an
  increased risk.
\item
  Parity and gravidity: Provide context about reproductive history.
\item
  Socioeconomic status: influences access to care and nutrition.
\item
  Occupational exposures: Such as chemicals or radiation.
\end{enumerate}

\textbf{Maternal Medical History}

\begin{enumerate}
\def\labelenumi{\arabic{enumi}.}
\tightlist
\item
  Chronic illnesses: Diabetes, hypertension, renal disease, HIV,
  tuberculosis, and epilepsy.
\item
  Medications during pregnancy: Some drugs (e.g., anticonvulsants, ACE
  inhibitors) are teratogenic.
\item
  Substance use: Alcohol, tobacco, herbal medications, or recreational
  drugs.
\item
  Family history: Genetic disorders, congenital anomalies,
  consanguinity.
\end{enumerate}

\subsection{Antenatal History}\label{antenatal-history}

\textbf{Antenatal Care}

\begin{enumerate}
\def\labelenumi{\arabic{enumi}.}
\tightlist
\item
  Number and timing of visits.
\item
  Use of supplements (iron, folic acid, tetanus immunisation).
\end{enumerate}

\textbf{Maternal Illnesses in Pregnancy}

\begin{enumerate}
\def\labelenumi{\arabic{enumi}.}
\tightlist
\item
  Infections: TORCH (toxoplasmosis, rubella, cytomegalovirus, herpes,
  syphilis), malaria, urinary tract infections.
\item
  Gestational diabetes and pre-eclampsia.
\item
  Antepartum haemorrhage or polyhydramnios/oligohydramnios.
\end{enumerate}

\textbf{Fetal Wellbeing}

\begin{enumerate}
\def\labelenumi{\arabic{enumi}.}
\tightlist
\item
  Results of ultrasound scans (growth, anomalies, multiple gestation,
  amniotic fluid volume).
\item
  Reduced fetal movements.
\end{enumerate}

\subsection{Intrapartum History}\label{intrapartum-history}

\textbf{Labour and Delivery}

\begin{enumerate}
\def\labelenumi{\arabic{enumi}.}
\tightlist
\item
  Place of delivery (home, health centre, hospital).
\item
  Duration and course of labour.
\item
  Prolonged rupture of membranes (risk of infection).
\item
  Use of intrapartum medications or anaesthesia.
\item
  Mode of delivery: spontaneous vaginal delivery, assisted delivery, or
  caesarean section.
\end{enumerate}

\textbf{Condition of Baby at Birth}

\begin{enumerate}
\def\labelenumi{\arabic{enumi}.}
\tightlist
\item
  Apgar scores at 1 and 5 minutes.
\item
  Need for resuscitation.
\item
  Cord events (e.g., prolapse, nuchal cord).
\item
  Meconium-stained amniotic fluid (risk of aspiration).
\end{enumerate}

\subsection{Immediate Postnatal
History}\label{immediate-postnatal-history}

\begin{itemize}
\tightlist
\item
  Cry at birth (immediate and vigorous or delayed).
\item
  Initiation of breastfeeding and feeding adequacy.
\item
  Passage of urine and meconium.
\item
  Neonatal resuscitation or admission to neonatal intensive care unit
  (NICU).
\item
  Administration of vitamin K, eye prophylaxis, and immunisations (BCG,
  OPV, Hepatitis B).
\end{itemize}

\section{Components of Neonatal
Examination}\label{components-of-neonatal-examination}

A thorough neonatal examination should ideally be conducted within the
first 24 hours and repeated before discharge. It involves general
observation, measurement of growth parameters, a head-to-toe physical
examination, and a functional systems review.

\subsection{General Considerations}\label{general-considerations}

\begin{itemize}
\tightlist
\item
  Conduct in a warm, well-lit environment to avoid hypothermia.
\item
  Wash hands thoroughly and maintain asepsis.
\item
  Involve the mother to reduce stress and promote bonding.
\item
  Examine systematically from head to toe.
\end{itemize}

\subsection{General Observation}\label{general-observation}

\begin{itemize}
\tightlist
\item
  Appearance: alert, active, lethargic, floppy.
\item
  Colour: pink, pale, jaundiced, cyanosed.
\item
  Cry: strong and lusty vs weak or absent.
\item
  Breathing pattern: regular or irregular, presence of grunting, nasal
  flaring, or retractions.
\item
  Movements: spontaneous, symmetrical, abnormal posturing.
\end{itemize}

\subsection{Anthropometric
Measurements}\label{anthropometric-measurements}

\begin{itemize}
\tightlist
\item
  Weight: normal term 2.5--4.0 kg.
\item
  Length: 48--52 cm.
\item
  Head circumference: 33--35 cm.
\item
  Chest circumference: slightly less than head circumference. These
  values are plotted on neonatal growth charts.
\end{itemize}

\subsection{Skin and Subcutaneous
Tissue}\label{skin-and-subcutaneous-tissue}

\begin{itemize}
\tightlist
\item
  Look for vernix caseosa, lanugo hair, birthmarks (Mongolian spots,
  café-au-lait spots), and congenital anomalies.
\item
  Assess for jaundice, petechiae, cyanosis, or dehydration.
\item
  Palpate for oedema (suggests renal or cardiac disease).
\end{itemize}

\subsection{Head and Face}\label{head-and-face}

\begin{itemize}
\tightlist
\item
  Shape and size: microcephaly, macrocephaly, cranial swellings (caput
  succedaneum, cephalohaematoma).
\item
  Fontanelles and sutures: size, tension (bulging may indicate raised
  intracranial pressure).
\item
  Eyes: red reflex (absent in congenital cataract or retinoblastoma),
  discharge, conjunctival haemorrhage.
\item
  Ears: position, size, and anomalies (low-set ears suggest chromosomal
  syndromes).
\item
  Nose: patency (choanal atresia if blocked).
\item
  Mouth: cleft lip/palate, Epstein pearls, ankyloglossia.
\end{itemize}

\subsection{Neck}\label{neck}

\begin{itemize}
\tightlist
\item
  Masses such as cystic hygroma.
\item
  Neck mobility (torticollis).
\end{itemize}

\subsection{Chest}\label{chest}

\begin{itemize}
\tightlist
\item
  Inspection: chest shape, symmetry, retractions.
\item
  Auscultation: breath sounds equal? murmurs present?
\item
  Palpation: heart apex position, thrills, or heaves.
\end{itemize}

\subsection{Abdomen}\label{abdomen}

\begin{itemize}
\tightlist
\item
  Shape: scaphoid, distended.
\item
  Umbilical cord: number of vessels, infection, hernia.
\item
  Palpation: liver (normally 1--2 cm below costal margin), spleen,
  kidneys, masses.
\item
  Auscultation: bowel sounds.
\end{itemize}

\subsection{Genitalia and Anus}\label{genitalia-and-anus}

\begin{itemize}
\tightlist
\item
  Male: testicular descent, hypospadias, phimosis.
\item
  Female: labial size, vaginal discharge (may be normal
  pseudo-menstruation).
\item
  Anus: patency, imperforate anus.
\end{itemize}

\subsection{Musculoskeletal System}\label{musculoskeletal-system}

\begin{itemize}
\tightlist
\item
  Assess posture, limb movements, joint stability.
\item
  Look for polydactyly, syndactyly, clubfoot.
\item
  Check clavicles for fracture.
\item
  Hip stability (Ortolani and Barlow manoeuvres).
\end{itemize}

\subsection{Neurological Examination}\label{neurological-examination}

\begin{itemize}
\tightlist
\item
  Tone: normal flexor tone vs hypotonia or hypertonia.
\item
  Primitive reflexes:

  \begin{itemize}
  \tightlist
  \item
    Moro reflex
  \item
    Rooting reflex
  \item
    Sucking reflex
  \item
    Palmar grasp
  \item
    Stepping reflex
  \end{itemize}
\item
  Behaviour: alertness, consolability, irritability.
\end{itemize}

\section{Special Considerations in Preterm
Infants}\label{special-considerations-in-preterm-infants}

Preterm babies (\textless37 weeks) require special attention. History
should highlight maternal risk factors for preterm labour, and
examination must assess:

\begin{itemize}
\tightlist
\item
  Skin thin and translucent with little subcutaneous fat.
\item
  Lanugo hair more abundant. - Ear cartilage soft, pinna remains folded.
\item
  Breast buds small or absent.
\item
  Genitalia: undescended testes in males, prominent labia minora in
  females.
\item
  Poor muscle tone and weak reflexes.
\end{itemize}

\section{Neonatal Screening and Preventive
Measures}\label{neonatal-screening-and-preventive-measures}

In many centres, neonatal assessment is complemented by screening tests:

\begin{enumerate}
\def\labelenumi{\arabic{enumi}.}
\tightlist
\item
  Metabolic screening: for congenital hypothyroidism, phenylketonuria
  (where available).
\item
  Hearing screening: Otoacoustic emission tests.
\item
  Pulse oximetry: to detect critical congenital heart disease.
\item
  Blood sugar: in infants of diabetic mothers or small/large for
  gestational age.
\end{enumerate}

\section{Documentation and
Communication}\label{documentation-and-communication}

\begin{itemize}
\tightlist
\item
  Findings must be documented systematically in the neonatal record.
\item
  Abnormal findings should be clearly communicated to senior clinicians
  and to the parents in a sensitive manner.
\item
  Recommendations for follow-up, investigations, or referrals must be
  made.
\end{itemize}

\section{Challenges in Resource-Limited
Settings}\label{challenges-in-resource-limited-settings}

\begin{itemize}
\tightlist
\item
  Inadequate access to prenatal care records.
\item
  Limited diagnostic facilities for neonatal screening.
\item
  High burden of home deliveries without skilled attendants.
\item
  Cultural practices influencing early care and feeding.
\end{itemize}

\section{Conclusion}\label{conclusion-4}

The neonatal history and examination form the foundation of paediatric
practice. They provide critical information about the newborn's
adaptation, detect congenital anomalies, and guide early interventions.
For medical students and clinicians in Ghana, mastering these skills is
essential in reducing neonatal morbidity and mortality. A systematic
approach, attention to detail, and sensitivity to family concerns are
the cornerstones of effective neonatal assessment.

\chapter{Neonatal Delivery
Pathologies}\label{neonatal-delivery-pathologies}

\section{The health newborn}\label{the-health-newborn}

\begin{itemize}
\tightlist
\item
  Cries / Breathes normally
\item
  Pink all over
\item
  Well-flexed \& moves all limbs spontaneously
\item
  Suckles well at the breast
\item
  Birth weight 2.5 -- 4.0kg
\item
  Normal vitals signs
\end{itemize}

\section{Occurrences at birth}\label{occurrences-at-birth}

\begin{itemize}
\tightlist
\item
  The fluid in the alveoli is absorbed and replaced by air. If the
  transition is not smooth, it results in insufficient oxygen delivery
  to the vital organs\ldots{}
\item
  Poor muscle tone
\item
  Respiratory distress or depression
\item
  Slow heart rate
\item
  Low BP
\item
  Cyanosis
\end{itemize}

\section{Birth Asphyxia}\label{birth-asphyxia}

\subsection{Definition}\label{definition-5}

\begin{tcolorbox}[enhanced jigsaw, toprule=.15mm, left=2mm, leftrule=.75mm, opacitybacktitle=0.6, opacityback=0, arc=.35mm, toptitle=1mm, colbacktitle=quarto-callout-important-color!10!white, title=\textcolor{quarto-callout-important-color}{\faExclamation}\hspace{0.5em}{World Health Organisation definition}, titlerule=0mm, breakable, colframe=quarto-callout-important-color-frame, bottomtitle=1mm, colback=white, rightrule=.15mm, bottomrule=.15mm, coltitle=black]

Birth Asphyxia is the medical condition resulting from deprivation of
oxygen in the newborn that lasts long enough during the birth process to
cause harm, usually to the brain.

\end{tcolorbox}

\subsection{Risk factors}\label{risk-factors}

Any condition that will lead to impairment of oxygenation or blood flow
to the newborn's brain in the perinatal period. These include:

\begin{itemize}
\tightlist
\item
  Prolonged labour (CPD)
\item
  Placental failure
\item
  Cord around the neck
\item
  Problems with oxygenation of maternal blood / maternal disease
\item
  Anaemia and bleeding in the baby
\item
  Congenital heart disease
\item
  Infections
\item
  Deficient medical skills and or knowledge
\end{itemize}

\subsection{Presentation}\label{presentation}

The asphyxiated baby may have any of the following:

\begin{itemize}
\tightlist
\item
  Poor Apgar Scores
\item
  May not cry at birth
\item
  Floppy/spastic
\item
  Breathing problems
\item
  Unresponsive
\item
  Seizures
\item
  Irritable
\end{itemize}

\subsection{The APGAR Score}\label{the-apgar-score}

It is an objective method of quantifying the newborn's condition. And is
useful for conveying information about the newborn's overall status and
response to resuscitation

\begin{longtable}[]{@{}
  >{\raggedright\arraybackslash}p{(\linewidth - 6\tabcolsep) * \real{0.1717}}
  >{\raggedright\arraybackslash}p{(\linewidth - 6\tabcolsep) * \real{0.1414}}
  >{\raggedright\arraybackslash}p{(\linewidth - 6\tabcolsep) * \real{0.2727}}
  >{\raggedright\arraybackslash}p{(\linewidth - 6\tabcolsep) * \real{0.4141}}@{}}
\caption{The APGAR Score}\label{tbl-apgar-score}\tabularnewline
\toprule\noalign{}
\begin{minipage}[b]{\linewidth}\raggedright
\end{minipage} & \begin{minipage}[b]{\linewidth}\raggedright
0
\end{minipage} & \begin{minipage}[b]{\linewidth}\raggedright
1
\end{minipage} & \begin{minipage}[b]{\linewidth}\raggedright
2
\end{minipage} \\
\midrule\noalign{}
\endfirsthead
\toprule\noalign{}
\begin{minipage}[b]{\linewidth}\raggedright
\end{minipage} & \begin{minipage}[b]{\linewidth}\raggedright
0
\end{minipage} & \begin{minipage}[b]{\linewidth}\raggedright
1
\end{minipage} & \begin{minipage}[b]{\linewidth}\raggedright
2
\end{minipage} \\
\midrule\noalign{}
\endhead
\bottomrule\noalign{}
\endlastfoot
\textbf{Heart Rate} & 0 & \textless100 & \textgreater=100 \\
\textbf{Respiration} & 0 & Weak or Irregular & Good Cry \\
\textbf{Reaction} & None & Slight & Good \\
\textbf{Colour} & Blue or Pale & Pink body limbs blue & All pink \\
\textbf{Tone} & Limp & Some movement & Active movement, limbs well
flexed \\
\end{longtable}

8-10 = No Asphyxia\\
5-7 = Mild Asphyxia\\
3-4 = Moderate Asphyxia\\
0-2 = Severe Asphyxia\\

\subsection{Management}\label{management-4}

\begin{itemize}
\tightlist
\item
  Largely supportive
\item
  Newborn resuscitation/oxygenation
\item
  Correction of fluid \& electrolyte imbalances including shock
\item
  Control of seizures
\item
  Treatment of any underlying infection
\item
  Look out for birth injuries
\item
  Active cooling found to improve neurological outcome
\item
  Temperature maintenance
\item
  Full Blood Count, Culture \& Sensitivity, Blood glucose etc
\item
  Serum electrolytes: Na, K, Ca \& Mg
\item
  Start empiric 1st line antibiotics according to protocol: / X'pen \&
  Gentamycin
\item
  Start IV Fluids at 50ml/kg (Plain 10\% Dextrose).
\item
  Pass a urethral catheter and monitor the baby's urine output.
\item
  The target temperature of the baby is 36.50C -- 37.50C
\end{itemize}

\subsection{Hypoxemic Ischaemic
Encephalopathy}\label{hypoxemic-ischaemic-encephalopathy}

\begin{itemize}
\tightlist
\item
  The most important consequence of birth asphyxia is
\item
  The outcome ranges from complete recovery to death
\item
  25 - 30\% end up with permanent damage like Cerebral palsy \& Mental
  retardation
\item
  Prognosis dependent on gestational age, management of metabolic \&
  cardiopulmonary complications \& the severity of the encephalopathy
\item
  Subsequent competent care and available facilities also influence the
  outcome
\end{itemize}

\section{Birth Injuries}\label{birth-injuries}

A birth injury can simply be referred to as any form of damage incurred
by the baby during the birthing process. Injury may occur as a result of
inappropriate or deficient medical skill or attention or may occur
despite skilled and competent obstetric care.

Predisposing conditions include:

\begin{itemize}
\tightlist
\item
  Cephalopelvic disproportion (CPD) / Small maternal stature /
  Primiparity
\item
  Macrosomia
\item
  Shoulder Dystocia
\item
  Prematurity
\item
  Prolonged or precipitous labour
\item
  Abnormal presentation
\item
  Instrumentation
\item
  Handling after delivery
\end{itemize}

\subsection{Fracture}\label{fracture}

Generally, the affected limb looks deformed or swollen, and the baby
barely moves it on account of pain

\subsubsection{Clavicle}\label{clavicle}

This is the most fractured bone during delivery; mostly during delivery
of the shoulder in vertex and of the extended arms in the breech. Signs
of a fracture may include no free arm movement on the affected side,
crepitus and bony irregularity, and absent Moro reflex. It has an
excellent prognosis, even though it is commonly missed. Treatment, if
any, includes immobilization of the arm and shoulder as shown below.

This is the first type of humeral fracture

\begin{figure}

\centering{

\pandocbounded{\includegraphics[keepaspectratio]{images/neo-humeral-fracture.jpg}}

}

\caption{\label{fig-HumralFractureSplinted}Humeral Fracture
(immobilised)}

\end{figure}%

\subsubsection{Humerus}\label{humerus}

The x-ray below shows another commonly fractured bone, the humerus.

\begin{figure}

\centering{

\pandocbounded{\includegraphics[keepaspectratio]{images/humeral-fracture-2.jpg}}

}

\caption{\label{fig-HumeralFractureXray}X-ray of a humeral fracture}

\end{figure}%

\subsubsection{Femur}\label{femur}

Risk factors: big baby, breech presentation, incompetency. The affected
thigh looks deformed, swollen and may be reddened. The main mode of
management involves splinting the limb from the waist to below the knee.

\begin{figure}

\centering{

\pandocbounded{\includegraphics[keepaspectratio]{images/neo-femural-fracture.jpg}}

}

\caption{\label{fig-FemoralFracture}Femoral Fracture}

\end{figure}%

\begin{figure}

\centering{

\includegraphics[width=2.91667in,height=2.91667in]{images/neo-femuralfracture-splinting.jpg}

}

\caption{\label{fig-FemoralFractureSplint}Femoral Fracture Splinting}

\end{figure}%

\section{Nerve injuries}\label{nerve-injuries}

\subsection{Brachial plexus injuries}\label{brachial-plexus-injuries}

The nerves of the brachial plexus may be compressed, stretched or torn
in a difficult delivery. Paralysis occurs as a result of nerve
compression from either haemorrhage or oedema. Permanent paralysis can
occur from the tearing of the nerve or avulsion of the nerve root from
the spinal cord or oedema. Erb's palsy (C5-C6) is the most common type
of BPI and is associated with a lack of shoulder motion. The involved
extremity lies adducted, prone, and internally rotated. Grasp reflex is
usually present and prognosis is generally good. Also described as the
Waiter's tip position.

\begin{figure}

\centering{

\pandocbounded{\includegraphics[keepaspectratio]{images/neo-erbs-palsy.jpg}}

}

\caption{\label{fig-ErbsPalsy}Erb's Palsy}

\end{figure}%

\subsection{Klumpke's paralysis}\label{klumpkes-paralysis}

\subsection{Facial nerve paralysis}\label{facial-nerve-paralysis}

Loss of voluntary muscle movement in the face on account of pressure on
the facial nerve during the delivery process. Risk factors include
instrumental delivery, poor delivery skills, big baby etc. Usually
resolves spontaneously after a few months

\section{Scalp Injuries}\label{scalp-injuries}

\subsection{Cephalhematoma}\label{cephalhematoma}

Tearing or disruption of the superficial veins under the periosteum
leads to haemorrhage and subsequent swelling. Suture lines confine the
cephalhematoma and limit the extent of the bleeding. There could be an
underlying linear skull fracture. Prognosis is good with most of them
resolving between 2 weeks to 3 months.

\begin{figure}

\centering{

\pandocbounded{\includegraphics[keepaspectratio]{images/neo-cephalhematoma.jpg}}

}

\caption{\label{fig-Cephalhematoma}Cephalhematoma}

\end{figure}%

\subsection{Subgaleal Hemorrhage}\label{subgaleal-hemorrhage}

The subgaleal space is located between the galea aponeurotica \& the
periosteum. The space extends from the orbital ridges to the nape of the
neck and laterally to the ears. Bleeding is caused by damage to the
large emissary veins located in the subaponeurotic layer. The bleeding
associated with subgaleal haemorrhages can be extensive. Clinically, the
baby may present with pallor and lethargy, followed by tachycardia,
tachypnea and hypotension. The scalp may appear tight and boggy and
complications include anemia, hypovolemic shock and jaundice.

\begin{figure}

\centering{

\pandocbounded{\includegraphics[keepaspectratio]{images/neo-subgaleal-bleed.jpg}}

}

\caption{\label{fig-SubgalealBleed}Subgaleal Bleed}

\end{figure}%

\section{Visceral injuries}\label{visceral-injuries}

\subsection{Liver and spleen}\label{liver-and-spleen}

Usually results from pressure on the liver during delivery of the head
in breech presentations. Risk factors include macrosomia, intrauterine
asphyxia, extreme prematurity, and hepatomegaly

\section{Respiratory Distress}\label{respiratory-distress}

\subsection{Meconium aspiration
syndrome}\label{meconium-aspiration-syndrome}

Fetuses sometimes pass meconium whilst in utero as a result of some form
of stress. If the stress has been going on for a while and the fetus has
been passing meconium for a few days, the cord, skin and nails may be
stained. Occurs when the fetus passes meconium into the surrounding
liquor and then aspirates this into the lungs. Tends to happen in term
and post-date babies. Distressed fetuses tend to pass meconium either
just before or during the delivery process. The smaller the amniotic
fluid volume, and the more meconium the baby passes, the thicker the
fluid and the more dangerous it becomes if aspirated.

History of Pregnancy and delivery looking for predisposing factors such
as fetal distress, post-maturity, meconium-stained liquor etc. Physical
examination looking for signs of meconium-staining on the baby, and
evidence of respiratory distress (fast breathing, chest indrawing,
cyanosis etc.). Investigations include FBC, Blood C\&S, and sometimes a
chest X-ray depending on the severity. Management is mainly supportive.
Includes antibiotics, respiratory support, supportive treatment, IVFs
and nutrition.

\subsection{Transient tachypnoea of the
newborn}\label{transient-tachypnoea-of-the-newborn}

Caused by delay in clearance of fetal lung fluid. Typically resolves
within 72 hours. Often associated with Caesarean Section Delivery.
Severity varies but is often mild with just tachypnoea. Management
involves supportive treatment of the respiratory with oxygen.

\chapter{Neonatal Emergencies}\label{neonatal-emergencies}

\section{Introduction}\label{introduction-6}

The usual newborn cries on delivery, is pink all over, well flexed and
moves all limbs spontaneously, suckles well at the breast, breaths
normally and weighs between 2.5 -- 4.0kg. Neonatal emergencies are not
uncommon and encompass a wide range of conditions occurring in the first
28 days of life. The classical mnemonic for these is \textbf{THE}
\textbf{MISFITS.}

\textbf{T}rauma/Abuse, \textbf{H}eart \& Lung, \textbf{E}ndocrine,
\textbf{M}etabolic disturbances, \textbf{I}nborn errors of metabolism,
\textbf{S}epsis, \textbf{F}ormula, \textbf{I}ntestinal, \textbf{T}oxins,
\textbf{T}risomies, and \textbf{S}eizures.

\section{Approach to newborn
emergencies}\label{approach-to-newborn-emergencies}

Presenting features of many serious neonatal disorders are nonspecific.
The history and physical examination are essential in the overall
approach to the patient. Prenatal, perinatal and postnatal history play
a huge role in neonatal assessments They guide and inform the health
worker on the most appropriate investigations, which would eventually
lead to a correct diagnosis. A complete history may unmask the likely
cause of symptoms and guide further questioning, for example, sepsis.

Examination on the other hand involves assessing the Airway, Breathing,
Circulation, Random Blood Sugar, provision of Oxygen and checking Oxygen
saturation. This can be pre- and post-ductal. Others include temperature
checks and other vital signs. The weight, current weight, head
circumference, and length. Intravenous access should be obtained for
possible further treatment. Appropriate investigations should also be
done.

Requisite investigations should also be done accordingly.

\section{Respiratory Emergencies}\label{respiratory-emergencies}

This is one of the most common and includes:

\begin{enumerate}
\def\labelenumi{\arabic{enumi}.}
\tightlist
\item
  Primary pulmonary Hypertension,
\item
  Meconium Aspiration Syndrome,
\item
  Congenital Pneumonia
\item
  Birth Asphyxia and
\item
  Respiratory Distress Syndrome
\end{enumerate}

\subsection{Respiratory Distress
Syndrome}\label{respiratory-distress-syndrome}

RDS is due mainly to a lack of surfactant in the lungs. Surfactants are
essential for reducing the surface area of the lungs, thus helping in
breathing. Incidence and severity increase with decreasing gestational
age. Other risk factors include prematurity, male gender, multiple
gestations, being born to a mother with diabetes mellitus and
hypothermia. Signs of RDS include tachypnea, grunting, recessions and
cyanosis. Prevention involves preventing preterm births and
administering corticosteroids to the mother of gestation between 24 to
34 weeks before delivery. Treatment however involves the administration
of surfactant.

Note that this condition is different from \emph{Respiratory Distress}
in a newborn. Respiratory distress is a more generalised term used as a
single or a combination of signs as a result of increased work of
breathing. It can result from pulmonary as well as non-pulmonary causes.
These include cardiac, neurological (eg. Asphyxia), haematological
(Anemia) and sepsis. It occurs in both term and preterm children.

\section{Endocrine Emergencies}\label{endocrine-emergencies}

Neonatal jaundice is the most prominent endocrine disorder in this
section. This is appropriately discussed in Section~\ref{sec-nnj-intro}

\section{Metabolic Emergencies}\label{metabolic-emergencies}

\subsection{Hypoglycemia}\label{hypoglycemia}

Hypoglycemia is common in the stressed neonate and glucose levels should
be monitored regularly. In the newborn period, it is defined as a random
blood glucose of \textless{} 2.6mmol/l. Risk factors include sepsis,
Infant of a diabetic mother, prematurity, Intrauterine growth
restriction, birth asphyxia and hypothermia. Signs include Lethargy,
poor feeding, seizures, and apnea. Neurological damage may result from
hypoglycemia in neonates.

Neonatal hypoglycemia is most commonly seen in macrosomic infants and
infants of diabetic mothers. For these babies, during pregnancy,
maternal glucose crosses the placenta to cause fetal hyperglycaemia. The
fetal pancreas responds by increasing insulin production. Following
delivery, the hyperglycaemic stimulus is instantly removed but insulin
production may take longer to slow down. This results in an increased
risk and incidence of hypoglycemia at the early newborn period.

Management involves initially checking the airway, breathing and
circulation. 2ml/kg of IV 10\% Dextrose or 5ml/kg of 5\% Dextrose may be
given PR if IV access is unavailable. Ensure the baby has a normal body
temperature (temperature target: 36.50 -- 37.50C) as hypothermia prone
the baby to hypoglycaemia. Always look for the underlying cause of the
hypoglycemia and treat it appropriately.

\section{Neonatal Sepsis}\label{neonatal-sepsis}

Sepsis in the neonate kills more than a million babies worldwide every
year. It is a clinical syndrome characterized by signs of infection with
accompanying bacteremia in the first month of life. It can be
categorized into \emph{Early Onset Neonatal Sepsis} (EOS), which refers
to the presence of signs of infection accompanied by a positive culture
within the first 72 hours of life, and \emph{Late Onset Neonatal Sepsis}
(LOS), which signifies the onset of signs of infection with a positive
culture after 72 hours of life.

\emph{Causative organisms}: Early-onset sepsis is typically caused by
organisms from the maternal genital tract, whereas late-onset sepsis is
caused by organisms in the caregiving environment or community. Common
organisms are \emph{Klebsiella pneumoniae,} E. coli, and Coagulase
Negative Staphylococcus among others. For many of these organisms, the
resistance rate to antibiotics is alarmingly going up.

Antenal risk factors known to be associated with neonatal jaundice
include spontaneous rupture of membranes, less than 37 completed weeks
of gestation, spontaneous preterm labour, rupture of membranes greater
than 18 hours before delivery, maternal chorioamnionitis, maternal fever
of 38 degrees Celsius or more, maternal invasive bacterial infection
requiring antibiotics, pre-labour rupture of membranes, Group B
Streptococcus infection in a previous baby, or current pregnancy,
meconium-stained amniotic fluid and foul-smelling liquor.

Postnatally, risk factors for neonatal sepsis include prolonged
resuscitation at birth, prematurity, invasive procedures, mechanical
ventilation, excessive handling, home delivery, lack of hand washing and
inadequate Infection prevention control. Others include overcrowding and
prolonged hospital stay.

Signs of neonatal sepsis include

\begin{enumerate}
\def\labelenumi{\arabic{enumi}.}
\tightlist
\item
  Abnormal colour: Pale, cyanotic, mottled appearance, jaundice, grey
\item
  Temperature instability
\item
  Abdominal signs: distension, poor feeding, vomiting, diarrhoea
\item
  Respiratory: Apnea, respiratory distress, gasping (Abnormal breathing)
\item
  Hypo - or hyperglycemia
\item
  Cardiovascular: Shock, tachycardia (HR \textgreater{} 180),
  Bradycardia (HR \textless{} 80)
\item
  Abnormal bleeding
\item
  Central Nervous System: Excessive crying, irritability, seizures,
  altered tone, lethargy,
\end{enumerate}

Management of neonatal sepsis should be comprehensive. It should include
an initial evaluation for resuscitation of the airway, breathing and
circulation. Further, the blood sugar should be measured. Early reversal
of the shock state by administering an initial bolus of 10ml/kg of
crystalloid or its equivalent should be done in the shock present.
Vasopressors or inotropes should be used in septic shock only after
appropriate volume resuscitation has been done. The goals of the
resuscitation should be Normal Cap refill (less than 2 seconds), normal
pulses, warm extremities, and appropriate urine output (greater than
1mL/kg/hr).

Increased successful treatment of neonatal sepsis requires early
recognition and urgent administration of appropriate antibiotics.

After resuscitation, ongoing management usually starts with a detailed
history to assess risk factors and other presentations. A thorough
physical examination will then be performed, looking for and documenting
specific signs indicating severity.

Investigations usually include a blood culture. This is considered the
gold standard for diagnosis. Ideally, a culture should always be done
before the first dose of antibiotics. Other auxiliary investigations
include a complete blood count, blood gases, urine culture, and a lumbar
puncture. The threshold for performing a lumber puncture in all
symptomatic newborns suspected of sepsis should be encouraged. It should
however be deferred in neonates considered too unstable to tolerate the
procedure, or where there is an absolute contraindication.

Supportive treatment is essential for a good outcome in neonatal sepsis.
Antibiotics are not the entire solution to their treatment. Nutrition or
breastfeeding should be optimized. The environment should be
thermo-neutral and oxygen saturation should be maintained within the
normal range (89 -- 95\%). Intravenous fluids should be used if the
infant is hemodynamically unstable. Monitoring of the blood glucose
levels should be instituted. Packed red cells and fresh frozen plasma
should be used in the event of anaemia or bleeding.

\section{Gastrointestinal
emergencies}\label{gastrointestinal-emergencies}

Gastrointestinal emergencies in newborns can be broadly divided into the
following:

\emph{Obstructive}: Some of these are Tracheoesophageal fistula,
Duodenal Atresia, Hirschsprung's Disease, Biliary Atresia, Pyloric
Stenosis, Intestinal Volvulus, Imperforate Anus and Necrotizing
enterocolitis.

\begin{figure}

\centering{

\pandocbounded{\includegraphics[keepaspectratio]{images/neo-abdominal-distension.jpg}}

}

\caption{\label{fig-neo-abdominal-distension}Abdominal distension
secondary to intestinal obstruction in a newborn}

\end{figure}%

\emph{Abdominal wall defect}: The most notable examples here are
Omphalocele and Gastroschisis

\subsection{Gastroschisis}\label{gastroschisis}

This is an anterior abdominal wall defect, located to the right of the
umbilicus, and contains herniated intestines that have no material
covering the sac. It occurs in approximately 1 in 10000 births. Rarely,
it is associated with other genetic syndromes. However, it may be
associated with intestinal atresia, stenosis and malrotation. Other
associations include prematurity (50-60\%)and cryptochidism (31\%).
Generally, it has a better prognosis compared to an omphalocele. The
prognosis is excellent for small defects. Mortality is expertise and
facility-dependent but generally around 5 to 10\%. Necrotizing
enterocolitis is a well-recognised complication, occurring in as much as
18\%.

\begin{figure}

\centering{

\pandocbounded{\includegraphics[keepaspectratio]{images/neo-gastroschisis.jpg}}

}

\caption{\label{fig-gastroschisis}Gastroschisis in a newborn}

\end{figure}%

\subsection{Omphalocele}\label{omphalocele}

\begin{figure}

\centering{

\pandocbounded{\includegraphics[keepaspectratio]{images/omphalocele.jpg}}

}

\caption{\label{fig-omphalocele}Omphalocele in a newborn}

\end{figure}%

\chapter{Neonatal Jaundice}\label{neonatal-jaundice}

\section{Introduction}\label{sec-nnj-intro}

Jaundice is the yellowish discoloration of the skin, eyes and mucous
membranes, caused by a pigment called bilirubin in the blood. Out of 10
term and 10 preterm newborns, 6 and 8 will develop jaundice
respectively, all in the 1st couple of weeks of life. Universally
accepted as one of the commonest causes of admission and readmission in
the first month of life. At Komfo Anokye Teaching Hospital Mother Baby
Unit, monthly admissions average between 300 and 400 and about 15 to
25\% of all these admissions are cases of neonatal jaundice. Whereas the
developed world describes kernicterus as a rare condition,
unfortunately, the same cannot be said for us in developing countries.
On average, cases of severe Neonatal Jaundice have ranged from 2.2\% to
30.8\% of all jaundice cases, with the monthly mortality ranging from
2.8\% to 15.2\%(REFERENCE). Remember, kernicterus is the only
preventable cause of cerebral palsy!

\section{Bilirubin metabolism}\label{bilirubin-metabolism}

Humans continuously form bilirubin and the liver is the main organ
responsible for the metabolism of bilirubin. For every gram of
haemoglobin, 35mg of bilirubin is produced. The bilirubin is conjugated
by the UGT enzyme, making it water-soluble, which is then released into
the bile before being excreted in the stool (and urine). It can also be
broken down in the intestine by bacterial enzymes like E. coli. However,
at birth, the newborn has several challenges. The liver is immature, and
the levels of bilirubin uridine diphosphate glucuronosyltransferase
(bilirubin-UGT) enzyme are low. Newborns have β-glucuronidase in the
intestinal mucosa/brush border, which deconjugates the conjugated
bilirubin found in the meconium. The unconjugated bilirubin can now be
reabsorbed through the intestinal wall and recycled back into the
circulation. This process is known as the ``enterohepatic circulation of
bilirubin''. The gut is sterile and, subsequently, infants have far
fewer bacteria in the gut, and so very little, if any, bilirubin is
reduced to urobilin and stercobilin.

Specifically in newborns, more bilirubin is produced, on account of the
short life span of Red Blood Cells and high Hemoglobin levels. The liver
is immature. They also have fewer bacteria and low intestinal enzymatic
activity in the intestine

\section{Types of bilirubin}\label{types-of-bilirubin}

There are two types:

\subsection{Conjugated (Direct)
Bilirubin}\label{conjugated-direct-bilirubin}

This is water soluble, excreted in the urine and stool, and not toxic to
the brain. However, high amounts could indicate underlying liver disease
or injury.

\subsection{Unconjugated (Indirect)
Bilirubin}\label{unconjugated-indirect-bilirubin}

This is lipid soluble, can cross the blood-brain barrier and is toxic in
high amounts to the brain.

In very high concentrations, unconjugated bilirubin, which is
lipid-soluble, is toxic to the developing brain. Once it crosses the
blood-brain barrier, it binds to brain tissue and deposits in the
developing brain. Since this is an irreversible process, it leads to
long-term neurological issues and even death.

\section{Types of Jaundice}\label{types-of-jaundice}

There are two main types of jaundice:

\begin{enumerate}
\def\labelenumi{\arabic{enumi}.}
\tightlist
\item
  Physiological jaundice and
\item
  Pathological jaundice.
\end{enumerate}

There are three main mechanisms for jaundice:

\begin{enumerate}
\def\labelenumi{\arabic{enumi}.}
\tightlist
\item
  Increased bilirubin production
\item
  Decreased bilirubin clearance and
\item
  Increased enterohepatic circulation.
\end{enumerate}

\subsection{Physiological jaundice}\label{physiological-jaundice}

\subsubsection{Increased bilirubin
production}\label{increased-bilirubin-production}

in term newborn infants, bilirubin production is 2 to 3 times higher
than in adults. This occurs because newborns have more RBCs and fetal
RBCs have a shorter life span than those in adults. Unfortunately, the
liver being immature, cannot conjugate and excrete all the bilirubin
from the breakdown of all the excess RBCs, thereby resulting in
spillover of bilirubin into the blood.

\subsubsection{Bilirubin clearance or
excretion}\label{bilirubin-clearance-or-excretion}

This is decreased in newborns, mainly due to the low levels of the UGT
enzyme in the liver. UGT activity in term infants at day 7 of age is
approximately 1\% of that of the adult liver and does not reach adult
levels until about 14 weeks of age.

\subsubsection{Enterohepatic
circulation}\label{enterohepatic-circulation}

The presence of the ß-glucuronidase results in an increase in the
enterohepatic circulation of bilirubin, further increasing the bilirubin
load in the infant. This is a diagnosis of exclusion

\subsection{Pathological jaundice}\label{pathological-jaundice}

\subsubsection{Definition}\label{definition-6}

Neonatal jaundice is said to be pathologic if:

\begin{itemize}
\tightlist
\item
  Jaundice in the 1st 24 - 48 hours of life.
\item
  Rate of SB rise \textgreater{} 0.5 mg/dL (8.5µmol/L) per hour
\item
  Jaundice all over the body (including palms \& soles)
\item
  Presence of a danger sign
\item
  History of previous siblings having had jaundice at birth
\item
  Jaundice in a term newborn after 2 weeks of age or in a preterm infant
  after 3 weeks of age
\item
  Direct (conjugated) bilirubin concentration \textgreater{} 20\% of the
  total
\end{itemize}

It can be caused by certain pathologic conditions or exaggeration of the
mechanisms responsible for physiologic neonatal jaundice. Identification
of what is causing the jaundice is useful in guiding management,
including counselling of the parents and what to expect for the next
pregnancy. Most common cause is increased bilirubin production due to
haemolytic disease processes that include the following:

\begin{itemize}
\tightlist
\item
  Isoimmune-mediated haemolysis (e.g., ABO or Rhesus D incompatibility)
\item
  Erythrocyte enzymatic defects, e.g.~G6PD deficiency
\item
  Sepsis, especially Urinary Tract Infection
\item
  Polycythaemia
\item
  Birth Injuries resulting in sequestration of blood within a closed
  space, e.g.~cephalohematoma, subgaleal bleed.
\end{itemize}

\subsubsection{ABO incompatibility}\label{abo-incompatibility}

This is one of the most common causes of isoimmune hemolytic disease
during the neonatal period. Infants with blood group A or B, carried by
blood group O mother, will have a positive antibody because of maternal
anti-A or anti-B transfer into the fetal circulation.

\subsubsection{Rhesus Incompatibility}\label{rhesus-incompatibility}

Rh incompatibility can occur when an Rh-negative pregnant mother is
exposed to Rh-positive fetal red blood cells secondary to feto-maternal
haemorrhage during pregnancy/delivery. As a result, the mother's blood
gets exposed to the fetal circulation and sensitization occurs leading
to maternal antibody production against the foreign Rh antigen. Once
produced, maternal Rh (IgG) antibodies may cross freely from the
placenta to the fetal circulation, where they form antigen-antibody
complexes with Rh- positive fetal RBCs and eventually are destroyed,
resulting in a fetal alloimmune-induced hemolytic anaemia and jaundice.
The first pregnancy is usually not affected, but more antibodies are
produced with each pregnancy making the jaundice worse with each
pregnancy.

\subsubsection{Decreased clearance}\label{decreased-clearance}

Inherited defects in the gene that encodes the UGT liver enzyme (eg,
Gilbert Syndrome), decrease bilirubin conjugation (eg Crigglar Najjar).
In physiological jaundice, the levels are naturally low, but here, in
addition to the low levels the UGT enzyme is either defective, absent or
has a reduced function. This reduces hepatic bilirubin metabolism and
its clearance thereby increasing the total serum unconjugated bilirubin
levels.

\subsubsection{Increased enterohepatic
circulation}\label{increased-enterohepatic-circulation}

The major causes are\\

\begin{itemize}
\tightlist
\item
  Breastfeeding jaundice
\item
  Breast milk jaundice
\item
  Impaired intestinal motility is caused by functional or anatomic
  obstruction.
\item
  Congenital hypothyroidism also causes increased enterohepatic
  circulation on account of reduced gut motility.
\end{itemize}

\section{Assessing for Neonatal
Jaundice}\label{assessing-for-neonatal-jaundice}

\begin{itemize}
\tightlist
\item
  Baby should be assessed in natural daylight
\item
  Look for yellow eyes \& skin, check the white part of the eyes only if
  the baby opens the eyes voluntarily.
\item
  You may blanch the skin on the bridge of the nose or the palms/soles
  of the feet if they turn yellow\ldots{}
\item
  Remember that the yellowing spreads from head to toe\ldots{}
\item
  Do not rely on visual inspection alone to estimate the bilirubin level
  in a baby with jaundice!!! It can be very subjective!!
\end{itemize}

\section{Clinical features}\label{clinical-features-4}

The clinical features of neonatal jaundice may include:

\begin{itemize}
\tightlist
\item
  Baby looks yellow! The yellowness appears cephalocaudal.
\item
  May not be as active as he/she used to be
\item
  Lethargic/hypotonic
\item
  Weak cry, irritable
\item
  Poor feeding
\item
  High-pitched cry / poor cry
\item
  Seizures
\item
  Arching of the neck/back
\end{itemize}

Thus to evaluate a child with jaundice we:

\begin{itemize}
\tightlist
\item
  Determine birth weight, gestation and postnatal age (in hours)
\item
  Assess clinical condition (well or ill)
\item
  Degree of jaundice (visual inspection, SBR etc)
\item
  Look for evidence of kernicterus / BIND
\end{itemize}

\section{Management}\label{management-5}

The general principle of treatment includes

\begin{itemize}
\tightlist
\item
  Encourage frequent exclusive breastfeeding.
\item
  Start Intravenous fluids only when there are signs of dehydration
\item
  Watch out for danger signs
\item
  Pathologic Neonatal jaundice is treated with

  \begin{itemize}
  \tightlist
  \item
    Phototherapy
  \item
    Exchange Blood Transfusion (EBT)
  \item
    Antibiotics
  \end{itemize}
\end{itemize}

Be interested in the cause as this will serve as a guide in the
management of the baby and direct your counselling as well as impact on
subsequent pregnancies Loads of information in the maternal and child
health record book, Gravidity and Parity, G6PD status, maternal Blood
group \& Rhesus status etc

\subsection{Investigations}\label{investigations-2}

This should include but not be restricted to

\begin{itemize}
\tightlist
\item
  Serum Bilirubin (conjugated, unconjugated and total)
\item
  Full Blood Count
\item
  G6PD screening
\item
  Blood Culture \& Sensitivity
\item
  Baby's blood group (only necessary if mother's blood group is O)
\item
  Others include Direct Coomb's test, Urine C \& S etc
\end{itemize}

\subsection{Phototherapy}\label{phototherapy}

Phototherapy is the use of visible light to treat high levels of serum
bilirubin in the newborn.

\begin{figure}

\centering{

\pandocbounded{\includegraphics[keepaspectratio]{images/neo-phototherapy.jpg}}

}

\caption{\label{fig-PhototherapyUnit}Phototherapy Unit}

\end{figure}%

The dose of phototherapy is a key factor in how quickly it works. The
dose in turn is determined by:

\begin{itemize}
\tightlist
\item
  The wavelength of the light
\item
  The intensity of the light (irradiance)
\item
  The distance between the light and the baby
\item
  The body's surface area is exposed to the light.
\end{itemize}

Effective phototherapy lowers serum bilirubin levels by converting the
lipid-soluble bilirubin into water-soluble forms that can easily be
excreted in the stool and urine Phototherapy also prevents the need for
an Exchange Blood Transfusion and prevents bilirubin from depositing in
the brain. The breakdown of bilirubin begins almost instantaneously when
the skin is exposed to light, hence, phototherapy should be started as
early as possible.

In initiating phototherapy, always note the time the baby's SBR sample
is being taken and estimate the age in hours up until that time.
Interpret bilirubin levels according to the baby's postnatal age in
hours and manage the bilirubin levels according to the threshold table
Start phototherapy if the SBR plots on or above the line appropriate for
age (in hours) and gestational age If the SBR plots just underneath the
line, repeat the SBR after 6 hours or start phototherapy if a repeat is
not feasible. Repeat the SBR at least 24 to 48 hours after initiation of
phototherapy. Discontinue phototherapy when the SBR plots below the
line.

The side effects of phototherapy include:

\begin{itemize}
\tightlist
\item
  Increase insensible water loss
\item
  Loose stools
\item
  Skin rash
\item
  Bronze baby syndrome
\item
  Hypo- or Hyperthermia
\item
  Interruption of mother-baby bonding
\end{itemize}

\subsection{Sunlight Therapy}\label{sunlight-therapy}

Works for physiological jaundice, however, one can never tell by looking
at a baby what kind of jaundice a baby has Err on the side of caution,
at least always have the SBR checked first Remember prolonged exposure
to UV rays can be harmful to the developing skin Baby cannot be put in
the light for more than 30 minutes in a day Even most of the available
literature and studies that recommend sunlight still advice that if the
jaundice is severe, the baby must be managed in the hospital!! A serum
bilirubin high enough to warrant treatment should be managed in the
hospital.

\subsection{Exchange Blood
Transfusion}\label{exchange-blood-transfusion}

Provides a means of rapid reduction of circulating bilirubin in the
blood. Involves manual removal of the baby's blood and simultaneously
replacing it with compatible donor blood.

\begin{figure}

\centering{

\pandocbounded{\includegraphics[keepaspectratio]{images/neo-ebt.jpg}}

}

\caption{\label{fig-ExchangeBloodTransfusion}Exchange Blood Transfusion}

\end{figure}%

In addition to reducing bilirubin levels, EBT removes partially
hemolyzed RBCs, RBCs coated with antibodies and circulating
immunoglobulins.

\begin{figure}

\centering{

\pandocbounded{\includegraphics[keepaspectratio]{images/ebt-chart.jpg}}

}

\caption{\label{fig-bilirubin-chart}Bilirubin Graph (\textgreater{} 38
weeks)}

\end{figure}%

Complications of exchange blood transfusion include:

\begin{itemize}
\tightlist
\item
  Cardiac \& respiratory disorders
\item
  Shock due to bleeding or inadequate replacement of blood infection
\item
  Catheter-related complications
\item
  Changes in the composition of the blood (high or low potassium, low
  calcium, low glucose, changes in pH)
\item
  Thrombocytopenia
\item
  And the rare but serious complications of air embolism, portal
  hypertension, and necrotizing enterocolitis.
\end{itemize}

\subsection{Intravenous Immunoglobins}\label{intravenous-immunoglobins}

Treatment with intravenous immunoglobulin (IVIG) has been suggested as
an alternative therapy to Exchange Blood Transfusion for isoimmune
hemolytic jaundice to reduce the need for Exchange Blood Transfusion and
duration of phototherapy and hospitalization in isoimmune hemolytic
disease of the newborn. It has been proposed that IVIG blocks the
binding of the antibody to the antigen. With this blockade, hemolysis no
longer occurs.

\section{Long term complications}\label{long-term-complications}

The effects of bilirubin toxicity include

\begin{itemize}
\tightlist
\item
  Hearing loss
\item
  Cerebral palsy
\item
  Mental retardation
\item
  Dental complications
\item
  Delayed developmental milestones
\item
  Seizure and visual disorders
\end{itemize}

\section{Recommendations}\label{recommendations}

\begin{itemize}
\tightlist
\item
  Always err on the side of caution
\item
  An SBR is always more objective
\item
  Look out for danger signs
\item
  As much breastmilk as possible by any means necessary
\item
  Sunlight therapy is not recommended, if the baby is yellow enough for
  you to want to put him/her under the sun, then the baby needs to be
  brought to the hospital!
\end{itemize}

\chapter{Newborn Delivery \&
Resuscitation}\label{newborn-delivery-resuscitation}

\section{Introduction}\label{introduction-7}

The delivery of a newborn is one of the most critical moments in
medicine, requiring skill, vigilance, and readiness. The transition from
intrauterine to extrauterine life involves complex physiological changes
that must occur within seconds. In most deliveries, this transition is
smooth and spontaneous. However, about \textbf{10\% of newborns require
some form of assistance}, and approximately \textbf{1\% need extensive
resuscitation}.\\
Understanding the physiology of transition, preparation for delivery,
and the systematic approach to neonatal resuscitation is therefore vital
for every healthcare provider involved in childbirth.

\section{Physiology of Fetal to Neonatal
Transition}\label{physiology-of-fetal-to-neonatal-transition}

The fetus depends on the placenta for gas exchange, nutrient delivery,
and waste removal. At birth, these functions must shift rapidly to the
infant's lungs and other organs.

Key physiological changes: - \textbf{Lung expansion}: With the first
breaths, alveoli expand and fluid is replaced by air, allowing gas
exchange.\\
- \textbf{Circulatory changes}: - Closure of the \textbf{foramen
ovale}.\\
- Functional closure of the \textbf{ductus arteriosus} as pulmonary
resistance drops and oxygen tension rises.\\
- Closure of the \textbf{ductus venosus}, redirecting blood through the
liver.\\
- \textbf{Thermoregulation}: The newborn's ability to maintain
temperature is limited, necessitating early warmth and drying.

Failure of any of these adaptations can lead to respiratory distress and
hypoxia.

\section{Preparation for Delivery}\label{preparation-for-delivery}

Every birth, regardless of risk status, must have a \textbf{prepared
resuscitation team and equipment}.

\subsection{Personnel}\label{personnel}

\begin{itemize}
\tightlist
\item
  \textbf{At least one skilled person} trained in neonatal resuscitation
  should be present at every delivery.\\
\item
  For high-risk deliveries (preterm, meconium-stained liquor, multiple
  gestation), \textbf{two or more trained personnel} should be
  available.
\end{itemize}

\subsection{Equipment and Environment}\label{equipment-and-environment}

Preparation should follow the \textbf{``warm, clean, ready''} principle:
- \textbf{Warmth}: Preheat radiant warmer; ensure room temperature
≥25°C.\\
- \textbf{Cleanliness}: Use sterile instruments and maintain a clean
surface.\\
- \textbf{Readiness}: - Functioning \textbf{suction device}.\\
- \textbf{Bag and mask} appropriately sized.\\
- \textbf{Oxygen supply} and blender if available.\\
- \textbf{Clock or timer} for monitoring response.\\
- \textbf{Sterile cord clamps}, towels, gloves, and stethoscope.

\section{Immediate Care at Birth}\label{immediate-care-at-birth}

Immediately after delivery, attention should focus on \textbf{rapid
assessment and prevention of hypothermia}.

\subsection{Initial Steps (within 30
seconds)}\label{initial-steps-within-30-seconds}

\begin{enumerate}
\def\labelenumi{\arabic{enumi}.}
\tightlist
\item
  \textbf{Provide warmth} -- place under radiant warmer.\\
\item
  \textbf{Position the head} in slight extension (``sniffing
  position'').\\
\item
  \textbf{Clear the airway} only if obstructed.\\
\item
  \textbf{Dry and stimulate} -- rubbing the back or flicking soles can
  initiate breathing.\\
\item
  \textbf{Evaluate breathing and heart rate}.
\end{enumerate}

If the newborn is term, breathing, and with good tone, proceed with
\textbf{routine care}: - Keep warm, initiate \textbf{skin-to-skin
contact}, and encourage \textbf{early breastfeeding}.

If \textbf{not breathing or gasping}, proceed to \textbf{resuscitation}.

\section{Neonatal Resuscitation
Algorithm}\label{neonatal-resuscitation-algorithm}

The process follows the \textbf{``Golden Minute'' principle}: all
interventions up to effective ventilation should occur within the first
minute of life.

\subsection{1. Initial Assessment}\label{initial-assessment-1}

Ask three questions: - Is the baby \textbf{term}? - Is the baby
\textbf{breathing or crying}? - Does the baby have \textbf{good tone}?

If ``yes'' to all → Routine care.\\
If ``no'' to any → Begin resuscitation steps.

\subsection{2. Initial Actions}\label{initial-actions}

\begin{itemize}
\tightlist
\item
  Warm, position, clear airway (if necessary), dry, and stimulate.\\
\item
  Reassess after 30 seconds.
\end{itemize}

If breathing starts → continue observation.\\
If \textbf{not breathing or heart rate \textless100 bpm}, start
\textbf{Positive Pressure Ventilation (PPV)}.

\subsection{3. Ventilation (The Most Critical
Step)}\label{ventilation-the-most-critical-step}

\begin{itemize}
\tightlist
\item
  Use \textbf{bag and mask ventilation} with room air (21\%) initially;
  increase O₂ if no improvement.\\
\item
  Deliver 40--60 breaths/min.\\
\item
  Observe for \textbf{chest rise} --- if none, check mask seal, airway
  position, or increase pressure.\\
\item
  After 30 seconds of effective ventilation, reassess:

  \begin{itemize}
  \tightlist
  \item
    HR \textgreater100 bpm → support spontaneous breathing.\\
  \item
    HR 60--100 bpm → continue ventilation and reassess.\\
  \item
    HR \textless60 bpm → start \textbf{chest compressions}.
  \end{itemize}
\end{itemize}

\subsection{4. Chest Compressions}\label{chest-compressions}

\begin{itemize}
\tightlist
\item
  Coordinate with ventilation in a \textbf{3:1 ratio} (90 compressions +
  30 breaths per minute).\\
\item
  Compress one-third of the chest depth using \textbf{two thumbs} on the
  lower sternum.\\
\item
  After 60 seconds, reassess heart rate.
\end{itemize}

\subsection{5. Medications}\label{medications}

\begin{itemize}
\tightlist
\item
  If HR \textless60 bpm despite 30 sec of effective ventilation and 60
  sec of compressions, administer \textbf{epinephrine} (0.01--0.03 mg/kg
  IV/IO; 1:10,000 dilution).\\
\item
  \textbf{Volume expansion} (normal saline 10 mL/kg IV) if hypovolemia
  suspected.
\end{itemize}

\section{Post-Resuscitation Care}\label{post-resuscitation-care}

After stabilization: - Maintain \textbf{normal temperature}
(36.5--37.5°C).\\
- Provide \textbf{oxygen titrated} to maintain saturation (target
90--95\% after 10 minutes).\\
- Monitor \textbf{blood glucose} to prevent hypoglycaemia.\\
- Observe for \textbf{respiratory distress, seizures, or shock}.\\
- If resuscitation was prolonged, consider admission to a
\textbf{neonatal intensive care unit (NICU)} for ongoing support.

\section{Common Pitfalls in Neonatal
Resuscitation}\label{common-pitfalls-in-neonatal-resuscitation}

\begin{itemize}
\tightlist
\item
  \textbf{Failure to anticipate risk}.\\
\item
  \textbf{Delay in initiating ventilation} --- the most common cause of
  poor outcome.\\
\item
  \textbf{Ineffective ventilation} due to poor mask seal or incorrect
  technique.\\
\item
  \textbf{Excessive suctioning}, leading to vagal bradycardia.\\
\item
  \textbf{Overuse of oxygen}, which can cause oxidative injury,
  especially in preterm infants.
\end{itemize}

\section{Special Situations}\label{special-situations}

\subsection{Meconium-Stained Amniotic
Fluid}\label{meconium-stained-amniotic-fluid}

\begin{itemize}
\tightlist
\item
  If the baby is vigorous (crying, good tone), proceed with routine
  care.\\
\item
  If not vigorous, do \textbf{not delay ventilation} for suctioning;
  clear airway only if obstructed.
\end{itemize}

\subsection{Preterm Newborn}\label{preterm-newborn}

\begin{itemize}
\tightlist
\item
  Risk of hypothermia and respiratory distress is high.\\
\item
  Use \textbf{polyethylene wrap} or \textbf{warm humidified gas}.\\
\item
  Oxygen titration and \textbf{gentle ventilation} to avoid barotrauma.
\end{itemize}

\subsection{Multiple Births}\label{multiple-births}

\begin{itemize}
\tightlist
\item
  Ensure multiple sets of resuscitation equipment and personnel.
\end{itemize}

\subsection{Congenital Anomalies}\label{congenital-anomalies}

\begin{itemize}
\tightlist
\item
  Some, such as diaphragmatic hernia, require \textbf{intubation without
  bag-mask ventilation} to prevent gastric distension.
\end{itemize}

\section{Equipment Checklist}\label{equipment-checklist}

\begin{itemize}
\tightlist
\item
  Suction device, masks (sizes 0 and 1), self-inflating bag, oxygen
  source.\\
\item
  Umbilical venous catheter, syringes, epinephrine, normal saline.\\
\item
  Radiant warmer, towels, caps, polyethylene wraps.\\
\item
  Stethoscope, timer, pulse oximeter (if available).
\end{itemize}

\section{Documentation and Prognosis}\label{documentation-and-prognosis}

Accurate documentation of time, interventions, and outcomes is
essential. Most babies respond promptly to resuscitation; however,
prolonged asphyxia may lead to \textbf{hypoxic--ischaemic
encephalopathy}, \textbf{cerebral palsy}, or \textbf{neurodevelopmental
delay}.

\section{Summary}\label{summary}

Neonatal resuscitation is a time-critical, life-saving skill built on
preparation, effective ventilation, and teamwork. In most cases,
ensuring a warm environment, clearing the airway only when needed, and
establishing effective ventilation within the first minute of life can
mean the difference between survival and death.

\chapter{Preterm and Low Birth
Weight}\label{preterm-and-low-birth-weight}

\section{Introduction}\label{introduction-8}

Preterm birth and low birth weight are major contributors to neonatal
morbidity and mortality worldwide, particularly in low- and
middle-income countries such as Ghana. Advances in perinatal care,
antenatal monitoring, and neonatal intensive care have improved
survival; however, the burden remains high. Both conditions often
overlap --- most low birth weight (LBW) babies are preterm, but some are
born at term and small for gestational age due to intrauterine growth
restriction (IUGR). Understanding their causes, physiology,
complications, and management is crucial for improving outcomes.

\begin{center}\rule{0.5\linewidth}{0.5pt}\end{center}

\section{Definitions}\label{definitions}

\textbf{Preterm birth} refers to any birth occurring before \textbf{37
completed weeks of gestation} (less than 259 days from the first day of
the last menstrual period).

Preterm infants are further categorized as: - \textbf{Late preterm:} 34
to \textless37 weeks\\
- \textbf{Moderate preterm:} 32 to \textless34 weeks\\
- \textbf{Very preterm:} 28 to \textless32 weeks\\
- \textbf{Extremely preterm:} \textless28 weeks

\textbf{Low Birth Weight (LBW)} is defined by the World Health
Organization as a birth weight \textbf{less than 2500 grams},
irrespective of gestational age.

Subcategories include: - \textbf{Very Low Birth Weight (VLBW):}
\textless1500 g\\
- \textbf{Extremely Low Birth Weight (ELBW):} \textless1000 g

\begin{center}\rule{0.5\linewidth}{0.5pt}\end{center}

\section{Epidemiology and Global
Burden}\label{epidemiology-and-global-burden}

Globally, about 15 million babies are born preterm every year ---
roughly 11\% of all live births. Over 60\% occur in South Asia and
sub-Saharan Africa. In Ghana, preterm delivery rates range from
10--15\%, with low birth weight accounting for approximately 9--12\% of
births. Both conditions contribute significantly to neonatal deaths,
particularly from respiratory distress, sepsis, and hypothermia.

Factors such as inadequate antenatal care, infections, poor maternal
nutrition, multiple pregnancies, and adolescent pregnancy increase the
risk. Neonatal care resources --- such as incubators, surfactant
therapy, and continuous positive airway pressure (CPAP) --- are often
limited, worsening outcomes in resource-constrained settings.

\begin{center}\rule{0.5\linewidth}{0.5pt}\end{center}

\section{Embryological and Physiological
Considerations}\label{embryological-and-physiological-considerations}

Normal fetal growth depends on adequate \textbf{placental function},
\textbf{maternal nutrition}, and \textbf{fetal genetic potential}.
Disruption of any of these can impair fetal growth or lead to premature
birth.

\begin{itemize}
\tightlist
\item
  \textbf{Placental development:} The placenta serves as the interface
  for nutrient and oxygen exchange. Abnormal trophoblastic invasion or
  uteroplacental insufficiency can limit nutrient delivery, resulting in
  intrauterine growth restriction.
\item
  \textbf{Lung development:} The fetal lungs undergo several stages ---
  embryonic, pseudoglandular, canalicular, saccular, and alveolar.
  Infants born before 34 weeks have surfactant deficiency, leading to
  respiratory distress.
\item
  \textbf{Thermoregulation:} Preterm infants have a large surface
  area-to-weight ratio, thin skin, and minimal subcutaneous fat,
  predisposing them to hypothermia.
\item
  \textbf{Metabolic adaptation:} Immature hepatic enzymes, low glycogen
  stores, and poor feeding contribute to hypoglycaemia and metabolic
  instability.
\item
  \textbf{Neurological immaturity:} Poor reflexes (suck, swallow, gag)
  and underdeveloped autonomic control affect feeding and
  cardiorespiratory stability.
\end{itemize}

\begin{center}\rule{0.5\linewidth}{0.5pt}\end{center}

\section{Aetiology and Risk Factors}\label{aetiology-and-risk-factors}

\subsection{Maternal Factors}\label{maternal-factors}

\begin{itemize}
\tightlist
\item
  Poor nutritional status and anaemia\\
\item
  Infections such as malaria, urinary tract infection, or HIV\\
\item
  Hypertensive disorders of pregnancy (preeclampsia, eclampsia)\\
\item
  Smoking, alcohol, or substance abuse\\
\item
  Short inter-pregnancy intervals\\
\item
  Low socioeconomic status\\
\item
  Teenage or advanced maternal age
\end{itemize}

\subsection{Fetal Factors}\label{fetal-factors}

\begin{itemize}
\tightlist
\item
  Multiple gestations (twins, triplets)\\
\item
  Congenital anomalies or chromosomal abnormalities\\
\item
  Intrauterine infections (TORCH, syphilis)
\end{itemize}

\subsection{Placental Factors}\label{placental-factors}

\begin{itemize}
\tightlist
\item
  Placenta previa or abruption\\
\item
  Placental insufficiency\\
\item
  Umbilical cord anomalies
\end{itemize}

\begin{center}\rule{0.5\linewidth}{0.5pt}\end{center}

\section{Pathophysiology}\label{pathophysiology-1}

The underlying mechanism varies depending on whether the infant is
\textbf{preterm}, \textbf{small for gestational age}, or both.

\begin{itemize}
\tightlist
\item
  \textbf{Preterm birth} results from early initiation of labour due to
  uterine overdistension, infection, or hormonal imbalance.
\item
  \textbf{Low birth weight due to IUGR} stems from chronic hypoxia and
  nutrient deprivation secondary to placental insufficiency.
\item
  \textbf{Combined preterm and growth restriction} exacerbate
  vulnerability to hypoxia, sepsis, and metabolic instability.
\end{itemize}

Physiologically, the immature organs of a preterm infant --- lungs,
brain, gut, kidneys, and liver --- are unable to perform their functions
optimally. The result is a cascade of complications such as respiratory
distress, patent ductus arteriosus, necrotizing enterocolitis, and
intraventricular haemorrhage.

\begin{center}\rule{0.5\linewidth}{0.5pt}\end{center}

\section{Clinical Features}\label{clinical-features-5}

\textbf{Preterm infants} are typically: - Small, thin, with less
subcutaneous fat\\
- Skin reddish and translucent\\
- Large head relative to body\\
- Weak cry and poor tone\\
- Incomplete flexion of limbs\\
- Absent or weak primitive reflexes (suck, grasp, Moro)

\textbf{Low birth weight infants (IUGR)} may appear: - Small but mature
(if term)\\
- With wasted muscles, loose skin, and relatively large head\\
- Sometimes meconium-stained due to chronic hypoxia

\begin{center}\rule{0.5\linewidth}{0.5pt}\end{center}

\section{Complications}\label{complications}

Complications may be \textbf{immediate}, \textbf{early neonatal}, or
\textbf{long-term}.

\subsection{Early Complications}\label{early-complications}

\begin{itemize}
\tightlist
\item
  Respiratory distress syndrome (RDS)
\item
  Apnoea of prematurity
\item
  Hypothermia
\item
  Hypoglycaemia
\item
  Electrolyte imbalance
\item
  Sepsis
\item
  Necrotizing enterocolitis
\item
  Jaundice (due to immature liver conjugation)
\item
  Patent ductus arteriosus
\item
  Intraventricular haemorrhage
\end{itemize}

\subsection{Long-Term Complications}\label{long-term-complications-1}

\begin{itemize}
\tightlist
\item
  Chronic lung disease (bronchopulmonary dysplasia)
\item
  Retinopathy of prematurity
\item
  Neurodevelopmental delay or cerebral palsy
\item
  Growth failure
\item
  Learning disabilities and visual or hearing impairment
\end{itemize}

\begin{center}\rule{0.5\linewidth}{0.5pt}\end{center}

\section{Diagnosis and Assessment}\label{diagnosis-and-assessment}

\subsection{Determination of Gestational
Age}\label{determination-of-gestational-age}

\begin{itemize}
\tightlist
\item
  \textbf{Maternal history:} Last menstrual period, early ultrasound\\
\item
  \textbf{New Ballard Score:} Based on neuromuscular and physical
  maturity\\
\item
  \textbf{Anthropometry:} Weight, length, and head circumference
\end{itemize}

\subsection{Investigations}\label{investigations-3}

\begin{itemize}
\tightlist
\item
  Blood glucose, electrolytes, calcium\\
\item
  Full blood count and CRP (if infection suspected)\\
\item
  Chest X-ray for respiratory distress\\
\item
  Cranial ultrasound for intraventricular haemorrhage\\
\item
  Screening for congenital infections if indicated
\end{itemize}

\begin{center}\rule{0.5\linewidth}{0.5pt}\end{center}

\section{Management}\label{management-6}

Management involves \textbf{stabilization, supportive care, and
prevention of complications}.\\
The guiding principles are \textbf{warmth, feeding, infection
prevention, and monitoring}.

\subsection{1. Immediate Stabilization at
Birth}\label{immediate-stabilization-at-birth}

\begin{itemize}
\tightlist
\item
  Dry and wrap the baby immediately to prevent hypothermia\\
\item
  Assess breathing and initiate resuscitation if necessary\\
\item
  Maintain airway and oxygenation\\
\item
  Early cord clamping if stable
\end{itemize}

\subsection{2. Temperature Regulation}\label{temperature-regulation}

\begin{itemize}
\tightlist
\item
  Use of incubator or radiant warmer\\
\item
  Kangaroo mother care (skin-to-skin contact) is highly effective and
  feasible in low-resource settings
\end{itemize}

\subsection{3. Feeding and Nutrition}\label{feeding-and-nutrition}

\begin{itemize}
\tightlist
\item
  Encourage early breastfeeding if the baby can suck\\
\item
  Expressed breast milk via cup or nasogastric tube for immature
  infants\\
\item
  Parenteral nutrition if gut immaturity prevents enteral feeding\\
\item
  Monitor glucose and electrolytes closely
\end{itemize}

\subsection{4. Prevention of Infection}\label{prevention-of-infection}

\begin{itemize}
\tightlist
\item
  Strict hand hygiene and aseptic techniques\\
\item
  Avoid unnecessary invasive procedures\\
\item
  Antibiotic therapy for suspected or proven sepsis
\end{itemize}

\subsection{5. Monitoring}\label{monitoring-1}

\begin{itemize}
\tightlist
\item
  Regular temperature, respiratory rate, and heart rate\\
\item
  Daily weight and urine output\\
\item
  Observation for apnoea or feeding intolerance
\end{itemize}

\subsection{6. Management of Specific
Complications}\label{management-of-specific-complications}

\begin{itemize}
\tightlist
\item
  Surfactant replacement and CPAP for RDS\\
\item
  Phototherapy for jaundice\\
\item
  Caffeine for apnoea\\
\item
  Blood transfusion for anaemia if needed
\end{itemize}

\begin{center}\rule{0.5\linewidth}{0.5pt}\end{center}

\section{Discharge Planning and
Follow-Up}\label{discharge-planning-and-follow-up}

\textbf{Discharge Criteria:} - Stable temperature in open cot for 24--48
hours\\
- Feeding well and gaining weight\\
- No apnoea or cardiorespiratory instability\\
- Parents trained in home care, including kangaroo care

\textbf{Follow-Up:} - Weekly or biweekly reviews until adequate weight
gain\\
- Monitor for developmental milestones, vision, hearing, and growth\\
- Immunizations per national schedule (adjusted for weight and age as
necessary)

\begin{center}\rule{0.5\linewidth}{0.5pt}\end{center}

\section{Prevention}\label{prevention}

\begin{itemize}
\tightlist
\item
  Adequate \textbf{antenatal care} and early detection of high-risk
  pregnancies\\
\item
  \textbf{Maternal nutrition} and treatment of infections (especially
  malaria and syphilis)\\
\item
  \textbf{Prevention of teenage pregnancy} and family planning\\
\item
  \textbf{Antenatal corticosteroids} for women at risk of preterm
  delivery\\
\item
  \textbf{Tocolytic therapy} to delay labour when feasible\\
\item
  \textbf{Facility-based delivery} with neonatal resuscitation readiness
\end{itemize}

\begin{center}\rule{0.5\linewidth}{0.5pt}\end{center}

\section{Prognosis}\label{prognosis}

Survival depends on gestational age, birth weight, and available
neonatal care.\\
- Infants \textgreater32 weeks or \textgreater1500 g have good survival
with appropriate support.\\
- Extremely preterm (\textless28 weeks) and ELBW infants have high
mortality and morbidity, especially in low-resource settings.\\
- Long-term outcomes include growth failure, cognitive delay, and
chronic lung disease, emphasizing the need for continuous follow-up.

\begin{center}\rule{0.5\linewidth}{0.5pt}\end{center}

\section{Conclusion}\label{conclusion-5}

Preterm birth and low birth weight remain major challenges in neonatal
care, particularly in resource-limited settings such as Ghana. A
comprehensive approach --- involving antenatal prevention, skilled
perinatal care, thermal protection, infection control, nutritional
support, and long-term follow-up --- is essential. With improved
maternal health programs, wider adoption of kangaroo mother care, and
enhanced neonatal intensive care capacity, the survival and quality of
life of these vulnerable infants can continue to improve.

\chapter{Breastfeeding -- Theory and
Practice}\label{breastfeeding-theory-and-practice}

\section{Introduction}\label{introduction-9}

Breastfeeding is more than just a method of feeding infants; it is a
sophisticated biological process that nurtures, protects, and connects
both mother and child. Over recent decades, scientific discoveries have
deepened our understanding of breastmilk's dynamic nature and its
systemic benefits. This chapter examines the theoretical foundations and
practical applications of breastfeeding, focusing on its physiological,
immunological, nutritional, and psychosocial aspects. It also outlines
strategies to overcome common challenges, ensuring healthcare workers
can support mothers effectively.

\section{Learning Objectives}\label{learning-objectives}

By the end of this chapter, students should be able to:

\begin{itemize}
\tightlist
\item
  State Ghana's National Breastfeeding Policy.
\item
  Describe the protective systems in breastmilk and their mechanisms.
\item
  Explain the physiology of breastfeeding and how it supports maternal
  and infant health.
\item
  Identify common breastfeeding challenges and outline appropriate
  management strategies.
\end{itemize}

\section{The National Breastfeeding
Policy}\label{the-national-breastfeeding-policy}

Ghana's policy promotes optimal breastfeeding practices, which include:

\begin{itemize}
\tightlist
\item
  Early Initiation of Breastfeeding (EIB) with early skin-to-skin
  contact and initiation of breastfeeding within the first hour of life.
\item
  Exclusive Breastfeeding for the first six months, with no other foods
  or liquids, not even water.
\item
  Appropriate Complementary Feeding by introducing nutrient-dense foods
  at six months, while continuing to breastfeed.
\item
  Continue breastfeeding for up to two years or even beyond, if this is
  desirable for both mother and baby
\end{itemize}

This policy is grounded in evidence that links breastfeeding to improved
neonatal, childhood, and even long-term health outcomes.

\section{The Dynamic Nature of
Breastmilk}\label{the-dynamic-nature-of-breastmilk}

Breast milk is an amazing, living, and dynamic fluid. Its composition
adapts:

\begin{itemize}
\tightlist
\item
  From one feed to the next (foremilk vs.~hindmilk),
\item
  From day to night,
\item
  Depending on the baby's gestational age (preterm vs.~term),
\item
  In response to the mother's health and environmental exposures.
\item
  With the age of the baby
\end{itemize}

These adaptations ensure optimal nutrition, immune protection, and
developmental support for the infant.

\section{Some biological systems in Breastmilk that promote the health
of the
newborn.}\label{some-biological-systems-in-breastmilk-that-promote-the-health-of-the-newborn.}

\subsection{Infection Prevention}\label{infection-prevention}

Breastmilk contains secretory IgA (SIgA), critical for mucosal immunity.
Unlike formula, breastmilk actively defends the infant against
bacterial, viral, and other infections such as:

\begin{itemize}
\tightlist
\item
  \href{id-hiv.qmd}{\textbf{HIV}}\textbf{:}~Transmission risk is reduced
  significantly with exclusive breastfeeding under antiretroviral
  therapy.
\item
  \textbf{Hepatitis B:}~Not transmitted via breastmilk; vaccination
  prevents perinatal transmission
\end{itemize}

During the COVID-19 pandemic, breastmilk was shown to contain antibodies
against SARS-CoV-2, with no evidence of viral transmission, further
reinforcing its immunological role.

\subsection{Gut Microbiota and Disease
Prevention}\label{gut-microbiota-and-disease-prevention}

Human milk oligosaccharides (HMOs) act as prebiotics, fostering healthy
gut flora that:

\begin{itemize}
\tightlist
\item
  Shapes immunity,
\item
  Reduces the risk of allergies, \href{resp-asthma.qmd}{asthma}, and
  dermatitis,
\item
  Contributes to neurodevelopment and emotional regulation via the
  gut-brain axis.
\end{itemize}

\subsection{Brain Development}\label{brain-development}

Breastmilk supports brain growth through components such as:

\begin{itemize}
\tightlist
\item
  \textbf{Sphingomyelin:}~Vital for myelination,
\item
  \textbf{Sialic acid:}~Enhances cognitive function,
\item
  \textbf{Myo-inositol:}~Boosts neuronal connectivity
\end{itemize}

Studies using MRI have shown that breastfed infants demonstrate superior
white matter development and cognitive outcomes compared to formula-fed
peers.

\subsection{4.~Support for Preterm
Infants}\label{support-for-preterm-infants}

Breastmilk of mothers of preterm babies contains:

\begin{itemize}
\tightlist
\item
  Higher energy content,
\item
  Increased lactoferrin for iron absorption,
\item
  HMOs and glycosaminoglycans that prevent necrotizing enterocolitis
  (NEC),
\item
  More bioactive molecules for immune support.
\end{itemize}

\section{The Physiology of
Breastfeeding}\label{the-physiology-of-breastfeeding}

Breastfeeding involves the coordinated action of:

\begin{itemize}
\tightlist
\item
  \textbf{Prolactin:}~Stimulates milk production in the mammary gland
\item
  \textbf{Oxytocin:}~Facilitates milk ejection by causing contraction of
  the smooth muscle surrounding the milk ducts. (the ``let-down''
  reflex).
\end{itemize}

\textbf{Early Initiation of Breastfeeding (EIB)}~serves as a crucial
step in initiating and establishing effective lactation. Efficient
breast emptying is the key to sustaining milk production.

\section{Some benefits of
Breastfeeding}\label{some-benefits-of-breastfeeding}

\subsection{\texorpdfstring{\textbf{For
Infants:}}{For Infants:}}\label{for-infants}

\begin{itemize}
\tightlist
\item
  Reduced incidence of infections (e.g.,
  \href{resp-pneumonia.qmd}{Pneumonia}, otitis media),
\item
  Lower risk of chronic conditions ~in adulthood (e.g.,
  \href{endo-dm.qmd}{Diabetes}, Leukemia, obesity),
\item
  Enhanced cognitive development.
\end{itemize}

\subsection{\texorpdfstring{\textbf{For
Mothers:}}{For Mothers:}}\label{for-mothers}

\begin{itemize}
\tightlist
\item
  Reduced postpartum bleeding,
\item
  Delayed return of fertility,
\item
  Lower risks of breast and ovarian cancer,
\item
  Decreased risk of type 2 \href{endo-dm.qmd}{Diabetes} and
  cardiovascular disease.
\end{itemize}

\section{Practical Challenges and
Solutions}\label{practical-challenges-and-solutions}

\subsection{EARLY INITIATION OF BREASTFEEDING
(EIB)}\label{early-initiation-of-breastfeeding-eib}

Early Initiation of Breastfeeding is essential, as it significantly
decreases the risk of neonatal mortality. Unless there is a complication
with the mother or the baby that prevents early initiation of
breastfeeding (EIB), the baby should be delivered directly onto the
mother's abdomen and allowed to ``crawl'' to the breast and start
suckling. This process is like a light switch that kicks start the
process of establishing successful breastfeeding.

\subsection{Bottle Feeding}\label{bottle-feeding}

Health workers must be cautious about recommending feeding bottles, as
mothers may not always be able to clean them due to inadequate water
supplies and facilities for boiling and sterilizing. Feeding bottles and
teats may also lead to nipple confusion, causing difficulties with
latching.

\subsection{Prematurity}\label{prematurity}

Feeding methods vary (tube, cup, cup and spoon) based on gestational age
and coordination of the suck-swallow reflex. Expressed breast milk is
the food of choice for every preterm baby, unless there is a genuine
contraindication, such as an inborn error of metabolism, in which case
breastmilk is contraindicated.

\subsection{Mouth Abnormalities}\label{mouth-abnormalities}

Mouth conditions, such as cleft lip and/or palate, or severe oral
thrush, may necessitate expressed milk and alternative feeding methods.

\subsection{Multiple Births}\label{multiple-births-1}

Frequent feeding stimulates supply. Twins and triplets can be
exclusively breastfed with proper support for the mother. Higher
multiples should also start with exclusive breastfeeding, but are likely
to outgrow a mother's milk supply rapidly and require supplementation.
Support for the mother, making sure she is relieved of as many other
chores as possible, is key to successful breastfeeding in multiple
pregnancies.

\subsection{Perceived or Real Milk
Insufficiency}\label{perceived-or-real-milk-insufficiency}

It is common for mothers, especially first-time mothers, to lack
confidence in their ability to breastfeed and to feel they do not have
enough breast milk. Once the baby is gaining weight and is generally
well, support and counseling are essential and are often all that is
needed. True breastmilk insufficiency is rare but distressing when it
occurs. Good expression techniques can help maintain a sufficient milk
supply.

\subsection{Maternal Illness}\label{maternal-illness}

Most conditions, including maternal tuberculosis and HIV (with
precautions), are not necessarily contraindications to breastfeeding.
Support and education are critical. When a mother is ill, it is
important not to assume that she cannot breastfeed, but rather to
objectively assess the risk to the baby as against the many benefits the
baby will receive from breastfeeding. National guidelines, where
available, should be consulted, and each mother and baby dyad assessed
carefully. The decision not to breastfeed should never be made lightly,
as even where the family can afford and correctly prepare infant
formula, the risk of illnesses such as \href{resp-asthma.qmd}{asthma}
and allergies may be increased. Where the ability to sustain adequate
formula feeding is a challenge to the family, the effect on the child's
health can be disastrous.

\subsection{Mothers in Formal
Employment}\label{mothers-in-formal-employment}

Supportive workplaces (Baby Friendly Workplaces), extended paid
maternity leave, effective use of hand expression, and good-quality
breast pumps are some of the ways to help mothers who work outside the
home to continue breastfeeding. Breast pump technology has evolved over
the years so that there are, for example, wearable hands-free breast
pumps which can discreetly pump breast milk whilst the mother is at
work. However simple, correctly done hand expression of breast milk is
very effective. Breast milk can then be stored at room temperature for 6
hours, in a good fridge with a temperature of 6-7\textsuperscript{o}C
for 24 hours, and in a deep freezer or fridge freezer at a temperature
of -17\textsuperscript{o}C and below for 6 months. The milk can then be
fed to the baby by cup by whoever is caring for the child.

\section{Counteracting Challenges to
Breastfeeding}\label{counteracting-challenges-to-breastfeeding}

Barriers to breastfeeding in Ghana include:

\begin{itemize}
\tightlist
\item
  Aggressive and inappropriate marketing and promotion of Infant Formula
  by companies that manufacture and sell Infant Formula.
\item
  Negative cultural attitudes to breastfeeding.
\item
  Lack of support from health professionals or family.
\end{itemize}

\subsection{Solutions}\label{solutions}

\begin{itemize}
\tightlist
\item
  Community education,
\item
  Health worker training,
\item
  Advocacy for breastfeeding-friendly policies.
\end{itemize}

\section{Conclusion}\label{conclusion-6}

Breastfeeding is a public health priority with far-reaching benefits for
infants, mothers, and society. Despite its challenges, successful
breastfeeding is achievable with informed support, early initiation, and
continued advocacy. As science reveals more about the biology of
breastmilk, our responsibility to protect and promote breastfeeding
becomes ever more urgent.

\section{Recommended Reading and
Viewing}\label{recommended-reading-and-viewing}

\begin{itemize}
\tightlist
\item
  The Ghana National Breastfeeding Policy
\item
  The Lancet Breastfeeding
  Series~(\href{https://www.thelancet.com/series-do/breastfeeding}{2016}
  and
  \href{https://www.aims.org.uk/journal/item/lancet-breastfeeding-series}{2023})
\item
  \href{https://globalhealthmedia.org/videos/}{Global Health Media
  videos on breastfeeding}
\item
  \href{https://www.mdpi.com/2673-3897/2/2/11}{Human Milk and Brain
  Development in Infants}
\item
  \href{https://hivpreventioncoalition.unaids.org/en/resources/ghana-aids-commission-national-hiv-and-aids-policy\#:~:text=Ghana\%E2\%80\%99s\%20updated\%20national\%20policy\%20is\%20guided\%20by\%20four,collaboration\%20with\%20public\%2C\%20private\%2C\%20local\%20and\%20international\%20institutions.}{The
  Ghana National Policy on PMTCT of HIV}
\end{itemize}

\part{{Pulmonology}}

\chapter{Basics}\label{basics}

\section{Introduction}\label{introduction-10}

Pediatric pulmonology is a vital subspecialty of pediatrics that focuses
on the structure and function of the lungs and respiratory tract in
infants, children, and adolescents. To understand pediatric respiratory
diseases and their clinical manifestations, a solid grasp of the
\textbf{anatomical}, \textbf{physiological}, \textbf{embryological},
\textbf{biochemical}, and \textbf{pathophysiological} underpinnings of
the pediatric respiratory system is essential.

This foundational knowledge enables clinicians to recognize what is
normal, anticipate how and why diseases develop, and determine
appropriate investigations and interventions. This write-up provides a
focused introduction to these aspects, tailored to the context of
medical education in Ghana.

\section{Anatomy of the Pediatric Respiratory
System}\label{anatomy-of-the-pediatric-respiratory-system}

The pediatric respiratory system consists of the \textbf{upper airway},
\textbf{lower airway}, and \textbf{lungs}, with supporting structures
including the thoracic cage and diaphragm.

\subsection{Upper Airway:}\label{upper-airway}

Includes:

\begin{itemize}
\tightlist
\item
  \textbf{Nasal cavity} -- filters, humidifies, and warms inspired air
\item
  \textbf{Nasopharynx, oropharynx, and laryngopharynx} -- direct airflow
  toward the larynx
\item
  \textbf{Larynx} -- houses the vocal cords; functions in phonation and
  protection during swallowing
\end{itemize}

\textbf{Clinical relevance:} Infants are obligate nose breathers. Even
mild nasal congestion can lead to significant respiratory distress.

\subsection{Lower Airway:}\label{lower-airway}

Includes:

\begin{itemize}
\tightlist
\item
  \textbf{Trachea} -- extends from the cricoid cartilage to the carina
\item
  \textbf{Bronchi} -- right main bronchus is shorter and more vertical
\item
  \textbf{Bronchioles} -- terminal and respiratory
\item
  \textbf{Alveolar ducts and alveoli} -- site of gas exchange
\end{itemize}

\textbf{Age-related note:} The airway diameter in neonates is narrow,
which increases resistance and the risk of obstruction.

\subsection{Lungs:}\label{lungs}

\begin{itemize}
\tightlist
\item
  Right lung has \textbf{three lobes}, left lung has \textbf{two lobes}
\item
  Lungs are surrounded by a pleural membrane
\item
  Richly supplied with blood vessels and lymphatics
\end{itemize}

\subsection{Thoracic Cage and
Diaphragm:}\label{thoracic-cage-and-diaphragm}

\begin{itemize}
\tightlist
\item
  Ribs are more horizontal in infants
\item
  Diaphragm is the main muscle of respiration; intercostal muscles
  assist with increasing age
\end{itemize}

\section{Physiology of the Pediatric Respiratory
System}\label{physiology-of-the-pediatric-respiratory-system}

Respiratory physiology involves \textbf{ventilation},
\textbf{perfusion}, and \textbf{gas exchange}, as well as
\textbf{control of breathing} and \textbf{defense mechanisms}.

\subsection{Ventilation:}\label{ventilation}

The process of moving air into and out of the lungs.

\begin{itemize}
\tightlist
\item
  \textbf{Tidal volume (VT):} Volume of air moved in and out per breath
  (\textasciitilde6--8 mL/kg in children)
\item
  \textbf{Minute ventilation:} VT × respiratory rate
\item
  \textbf{Compliance:} Children have \textbf{high chest wall
  compliance}, meaning it deforms easily, but \textbf{low lung
  compliance}, especially in neonates
\end{itemize}

\textbf{Clinical note:} High compliance of the chest wall predisposes
neonates to respiratory fatigue.

\section{Gas Exchange:}\label{gas-exchange}

Occurs at the alveolar-capillary interface:

\begin{itemize}
\tightlist
\item
  \textbf{Oxygen (O₂)} diffuses from alveoli to blood
\item
  \textbf{Carbon dioxide (CO₂)} diffuses from blood to alveoli
\end{itemize}

Dependent on:

\begin{itemize}
\tightlist
\item
  Surface area of alveoli
\item
  Thickness of the alveolar-capillary membrane
\item
  Adequate ventilation-perfusion (V/Q) matching
\end{itemize}

\section{Control of Breathing:}\label{control-of-breathing}

Controlled by centers in the \textbf{medulla} and \textbf{pons},
modulated by:

\begin{itemize}
\tightlist
\item
  \textbf{Chemoreceptors} (central: respond to CO₂; peripheral: respond
  to O₂)
\item
  \textbf{Stretch receptors} in the lungs
\item
  Voluntary control is limited in neonates
\end{itemize}

\textbf{Age-specific physiology:}

\begin{itemize}
\tightlist
\item
  Infants have periodic breathing and are prone to apneas
\item
  Immature respiratory drive increases risk of hypoventilation
\end{itemize}

\subsection{Defense Mechanisms:}\label{defense-mechanisms}

\begin{itemize}
\tightlist
\item
  \textbf{Nasal hairs and mucosa} trap particles
\item
  \textbf{Mucociliary clearance} moves mucus upward toward the
  oropharynx
\item
  \textbf{Cough reflex} clears lower airways
\item
  \textbf{Immune defense:} IgA in secretions, macrophages in alveoli
\end{itemize}

\section{Embryology of the Respiratory
System}\label{embryology-of-the-respiratory-system}

\subsection{Development Timeline:}\label{development-timeline}

\begin{itemize}
\tightlist
\item
  \textbf{Week 4}: Respiratory diverticulum (lung bud) arises from
  foregut endoderm
\item
  \textbf{Week 5--7}: Formation of primary, secondary, and tertiary
  bronchi
\item
  \textbf{Week 16}: Terminal bronchioles formed
\item
  \textbf{Week 24}: Respiratory bronchioles begin to develop
\item
  \textbf{Week 28--36}: Alveolar ducts and primitive alveoli form
\item
  \textbf{Birth to 8 years}: Postnatal alveolar multiplication (from
  \textasciitilde20 million at birth to \textasciitilde300 million)
\end{itemize}

\subsection{Embryological Germ Layers:}\label{embryological-germ-layers}

\begin{itemize}
\tightlist
\item
  \textbf{Endoderm}: Forms the epithelium of the airways and alveoli
\item
  \textbf{Mesoderm}: Forms connective tissue, cartilage, smooth muscle,
  and blood vessels
\end{itemize}

\subsection{Lung Maturation Stages:}\label{lung-maturation-stages}

\begin{enumerate}
\def\labelenumi{\arabic{enumi}.}
\tightlist
\item
  \textbf{Pseudoglandular (weeks 5--17):} Branching of airways; no gas
  exchange possible
\item
  \textbf{Canalicular (weeks 16--25):} Formation of airspaces; capillary
  network appears
\item
  \textbf{Saccular (weeks 24--36):} Terminal sacs form; beginning of
  surfactant production
\item
  \textbf{Alveolar (week 36 to 8 years):} Alveoli mature and multiply
\end{enumerate}

\subsection{Surfactant:}\label{surfactant}

Produced by \textbf{type II pneumocytes} from \textasciitilde week 24,
with sufficient amounts by \textasciitilde week 34.

\textbf{Function:} Reduces surface tension in alveoli, preventing
collapse during expiration

\textbf{Clinical relevance:} Premature infants often lack surfactant, a
substance that can lead to respiratory distress.

\section{Biochemistry of the Respiratory
System}\label{biochemistry-of-the-respiratory-system}

\subsection{Gas Transport:}\label{gas-transport}

\begin{itemize}
\tightlist
\item
  \textbf{Oxygen Transport:}

  \begin{itemize}
  \tightlist
  \item
    98\% carried by hemoglobin
  \item
    Oxyhemoglobin dissociation curve describes the relation between PaO₂
    and SaO₂
  \item
    Fetal hemoglobin (HbF) has a \textbf{higher affinity} for oxygen
    than adult hemoglobin
  \end{itemize}
\item
  \textbf{Carbon Dioxide Transport:}

  \begin{itemize}
  \tightlist
  \item
    Dissolved in plasma (\textasciitilde10\%)
  \item
    Bound to hemoglobin as carbaminohemoglobin (\textasciitilde20\%)
  \item
    As \textbf{bicarbonate ions} (\textasciitilde70\%) via carbonic
    anhydrase reaction:
  \end{itemize}
\end{itemize}

\section{Acid-Base Balance:}\label{acid-base-balance}

\begin{itemize}
\tightlist
\item
  \textbf{Lungs regulate pH} by excreting CO₂
\item
  \textbf{Respiratory acidosis}: from hypoventilation (↑CO₂)
\item
  \textbf{Respiratory alkalosis}: from hyperventilation (↓CO₂)
\end{itemize}

Maintaining proper ventilation is crucial to acid-base homeostasis in
children.

\section{Surfactant Biochemistry:}\label{surfactant-biochemistry}

\begin{itemize}
\tightlist
\item
  Composed mainly of \textbf{phospholipids} (especially
  \textbf{dipalmitoylphosphatidylcholine - DPPC})
\item
  Also contains \textbf{surfactant proteins (SP-A, SP-B, SP-C, SP-D)}
  that help spread and regulate surfactant
\end{itemize}

\textbf{Synthesis is cortisol-dependent}, which is why maternal
corticosteroids are given antenatally in preterm labor.

\section{Pathophysiology of the Pediatric Respiratory
System}\label{pathophysiology-of-the-pediatric-respiratory-system}

Pathophysiology describes the \textbf{functional changes} that occur in
response to disease or injury. Understanding these responses helps to
explain signs such as \textbf{wheezing}, \textbf{cough},
\textbf{hypoxia}, and \textbf{tachypnea}.

\subsection{Airway Obstruction:}\label{airway-obstruction}

\begin{itemize}
\tightlist
\item
  Can occur \textbf{extrathoracically} (e.g., larynx) or
  \textbf{intrathoracically} (e.g., bronchioles)
\item
  Narrow pediatric airways mean even minor swelling or secretions cause
  significant resistance
\item
  Leads to increased work of breathing, wheezing, or stridor
\end{itemize}

\subsection{Ventilation-Perfusion (V/Q)
Mismatch:}\label{ventilation-perfusion-vq-mismatch}

\begin{itemize}
\tightlist
\item
  Ideal: ventilation matches perfusion
\item
  In disease (e.g., mucus plugging, consolidation), mismatch occurs

  \begin{itemize}
  \tightlist
  \item
    \textbf{Low V/Q}: alveoli are perfused but not ventilated →
    hypoxemia
  \item
    \textbf{High V/Q}: alveoli are ventilated but not perfused → wasted
    ventilation
  \end{itemize}
\end{itemize}

\subsection{Hypoventilation:}\label{hypoventilation}

\begin{itemize}
\tightlist
\item
  Due to fatigue, CNS depression, or neuromuscular disease
\item
  Leads to \textbf{hypercapnia} and \textbf{respiratory acidosis}
\end{itemize}

\subsection{Surfactant Deficiency:}\label{surfactant-deficiency}

\begin{itemize}
\tightlist
\item
  Causes alveolar collapse (atelectasis)
\item
  Reduces lung compliance
\item
  Seen in premature infants or inactivation by infection/inflammation
\end{itemize}

\subsection{Immature Immune System:}\label{immature-immune-system}

\begin{itemize}
\tightlist
\item
  Neonates have limited production of IgA, poor neutrophil function
\item
  Makes them vulnerable to respiratory infections
\end{itemize}

\section{Conclusion}\label{conclusion-7}

The pediatric respiratory system is uniquely structured and regulated,
necessitating a comprehensive understanding of its \textbf{anatomy},
\textbf{development}, \textbf{biochemistry}, \textbf{physiology}, and
\textbf{response to disease}. Medical students should appreciate how
these fundamental sciences interact in the context of health and
disease. This understanding lays the groundwork for clinical reasoning,
diagnosis, and management of respiratory illnesses in children.

In Ghana, where pediatric respiratory conditions are prevalent, this
foundational knowledge becomes particularly crucial. As a future
clinician, you are encouraged to integrate basic science with clinical
practice to improve child health outcomes.

\chapter{Respiratory Failure in
Children}\label{respiratory-failure-in-children}

\section{Introduction}\label{introduction-11}

Respiratory failure is a life-threatening condition in which the
respiratory system fails to maintain adequate oxygenation and/or carbon
dioxide elimination. In paediatrics it is a frequent final common
pathway of many severe illnesses --- particularly pneumonia,
bronchiolitis, severe asthma, and sepsis --- and is a major contributor
to childhood mortality in low- and middle-income countries including
Ghana. Recognition of early signs, understanding the underlying
pathophysiology, and prompt institution of supportive measures are
essential skills for the medical student and junior clinician.

Children differ from adults in airway anatomy, chest wall compliance,
metabolic rate and reserve, which makes them prone to rapid
deterioration. Where resources are limited, timely clinical assessment,
oxygen therapy, and basic respiratory support often determine outcome.

\section{Classification and Basic
Concepts}\label{classification-and-basic-concepts}

Respiratory failure is commonly classified by the dominant gas-exchange
abnormality:

\begin{itemize}
\tightlist
\item
  \textbf{Hypoxaemic (Type I)} respiratory failure --- impaired
  oxygenation (PaO₂ \textless{} 60 mmHg) with normal or low PaCO₂.
  Typical causes include pneumonia, acute respiratory distress syndrome
  (ARDS), pulmonary oedema and large shunts.
\item
  \textbf{Hypercapnic (Type II)} respiratory failure --- inadequate
  alveolar ventilation resulting in raised PaCO₂ (\textgreater50 mmHg).
  Causes include severe airway obstruction (status asthmaticus),
  respiratory muscle fatigue, central depression of respiration, and
  neuromuscular disease.
\item
  \textbf{Mixed respiratory failure} combines both elements and is
  common in advanced respiratory disease or severe sepsis.
\end{itemize}

Understanding the difference is practical: hypoxaemia requires
restoration of oxygenation (oxygen and recruitment of alveoli), while
hypercapnia indicates a need to improve ventilation (support minute
ventilation).

\section{Pathophysiology --- how failure
develops}\label{pathophysiology-how-failure-develops}

Effective respiration requires airway patency, adequate ventilatory
drive and muscle function, properly functioning lung units for
diffusion, and coordinated perfusion. Disruption to any of these leads
to failure.

In \textbf{ventilatory failure}, the work of breathing exceeds the
capacity of respiratory muscles; progressive fatigue causes
hypoventilation and CO₂ retention. Children with severe asthma, upper
airway obstruction, or neuromuscular weakness may decompensate rapidly.

In \textbf{oxygenation failure}, processes such as alveolar
consolidation (pneumonia), surfactant deficiency (preterm infants),
pulmonary oedema, or widespread inflammation (ARDS) reduce the effective
surface area for oxygen diffusion. Ventilation-perfusion mismatch and
intrapulmonary shunting contribute to refractory hypoxaemia.

Neonates and infants are particularly vulnerable because of small
functional residual capacity, high oxygen consumption, and immature
control of breathing --- they desaturate quickly once compromise begins.

\section{Aetiology --- common causes in
Ghana}\label{aetiology-common-causes-in-ghana}

The spectrum of causes varies with age and setting. In Ghanaian
paediatric practice, the most frequent precipitants are:

\begin{itemize}
\tightlist
\item
  \textbf{Infectious lower respiratory disease:} severe
  community-acquired pneumonia, TB in older children, and bronchiolitis
  in infants.
\item
  \textbf{Asthma exacerbations:} poorly controlled asthma presenting
  with severe bronchospasm.
\item
  \textbf{Sepsis and severe malaria:} systemic illness that increases
  oxygen demand and may cause ARDS or metabolic acidosis.
\item
  \textbf{Upper airway obstruction:} foreign body aspiration, croup, or
  deep neck infections.
\item
  \textbf{Neonatal causes:} surfactant deficiency, meconium aspiration,
  congenital pneumonia, and persistent pulmonary hypertension.
\item
  \textbf{Neuromuscular disease or central depression:} e.g., head
  injury, meningitis, or drug overdose.
\end{itemize}

Resource constraints, delayed presentation, and coexisting malnutrition
or anaemia often worsen the clinical picture.

\section{Clinical presentation and early
recognition}\label{clinical-presentation-and-early-recognition}

Respiratory failure may present subtly. Early identification hinges on
careful observation and monitoring. Important clinical features are:

\begin{itemize}
\tightlist
\item
  \textbf{Increased work of breathing:} tachypnoea, nasal flaring,
  intercostal/subcostal retractions, tracheal tug, and use of accessory
  muscles.
\item
  \textbf{Abnormal breathing patterns:} grunting (infants), prolonged
  expiratory phase (asthma), or shallow irregular respirations.
\item
  \textbf{Hypoxia signs:} central cyanosis (late), restlessness,
  agitation, poor perfusion, and tachycardia progressing to bradycardia.
\item
  \textbf{Hypercapnia signs:} headache and drowsiness in older children;
  in infants, poor feeding and lethargy are common.
\item
  \textbf{Failure to feed, pallor, and altered consciousness} indicate
  severe or advanced disease.
\end{itemize}

Because children compensate well until late, a sudden collapse may
occur. Routine use of pulse oximetry at triage helps detect hypoxaemia
before clinical cyanosis appears.

\section{Investigations --- practical
approach}\label{investigations-practical-approach}

Confirmatory investigation is an arterial blood gas (ABG), which defines
oxygenation and ventilation status and reveals acid-base disturbance. In
many settings, ABG may be unavailable, so clinical assessment and pulse
oximetry guide most initial decisions.

Additional useful tests:

\begin{itemize}
\tightlist
\item
  \textbf{Pulse oximetry:} continuous SpO₂ monitoring.
\item
  \textbf{Chest X-ray:} identifies consolidation, pneumothorax, pleural
  effusion or cardiomegaly.
\item
  \textbf{Blood tests:} full blood count, blood cultures, electrolytes,
  lactate and blood glucose.
\item
  \textbf{Viral testing} or nasopharyngeal aspirate for bronchiolitis
  where available.
\item
  \textbf{Point-of-care tests:} malaria rapid tests and HIV testing as
  clinically indicated.
\item
  \textbf{Ultrasound:} lung ultrasound can detect consolidation and
  effusion at the bedside.
\end{itemize}

Interpret findings in the clinical context: a high PaCO₂ points to
ventilatory failure and need for ventilatory support; severe hypoxaemia
with low PaCO₂ suggests shunt physiology and oxygenation failure.

\section{Management principles}\label{management-principles}

Management aims to reverse hypoxaemia and/or hypercapnia, treat the
underlying cause, and prevent complications. Interventions should be
guided by severity and available resources.

\subsection{Immediate actions}\label{immediate-actions}

Apply the ABC approach. Ensure airway patency, give supplemental oxygen
early, and support ventilation if there are signs of respiratory
compromise. Keep the child warm, monitor glucose, and establish IV
access (or IO in emergencies). Treat reversible causes such as severe
asthma with bronchodilators and steroids, or sepsis with timely
antibiotics.

\subsection{Oxygen therapy}\label{oxygen-therapy}

Oxygen is the cornerstone for hypoxaemic children. Start with low-flow
oxygen via nasal prongs or face mask, titrating to target SpO₂ levels
appropriate for age (generally ≥92\% in older children; lower targets
may apply for certain neonates/conditions). Where available, high-flow
nasal cannula (HFNC) or CPAP provides effective respiratory support in
moderate distress, reducing the need for intubation in many cases.

\subsection{Non-invasive and invasive
ventilation}\label{non-invasive-and-invasive-ventilation}

When oxygen alone is insufficient (persistent hypoxaemia on high FiO₂,
rising PaCO₂, or respiratory fatigue), provide ventilatory support.
Non-invasive options such as CPAP and BiPAP are useful for selected
patients. Intubation and mechanical ventilation become necessary for
respiratory arrest, severe hypercapnia, or inability to protect the
airway. Mechanical ventilation requires skilled staff and monitoring to
avoid complications such as barotrauma and ventilator-associated
pneumonia.

\subsection{Specific therapies}\label{specific-therapies}

Treat the underlying pathology: antibiotics for bacterial pneumonia,
bronchodilators and systemic steroids for asthma, surfactant for
neonatal respiratory distress syndrome (where indicated), diuretics for
cardiogenic pulmonary oedema, and bronchoscopy for airway foreign
bodies.

\subsection{Supportive care}\label{supportive-care}

Adequate hydration, nutritional support, correction of anaemia, and
seizure control (if present) are vital. Prevent and manage complications
like pneumothorax promptly. Keep meticulous infection control practices
and consider early transfer to higher-level care when advanced
ventilation or paediatric intensive care is needed.

\section{Monitoring and escalation}\label{monitoring-and-escalation}

Continuous monitoring of oxygen saturation, heart rate and respiratory
rate is essential. Frequent reassessment of work of breathing, mental
state, and perfusion identifies deterioration. ABG monitoring guides
ventilatory adjustments where available. Escalate care promptly if
hypoxaemia persists despite maximal non-invasive support, or if CO₂
retention or acidosis worsens.

\section{Complications and long-term
outcomes}\label{complications-and-long-term-outcomes}

Untreated or prolonged respiratory failure can cause hypoxic brain
injury, multi-organ dysfunction, and death. Survivors may develop
chronic lung disease, neurodevelopmental impairment (especially after
neonatal respiratory failure), or recurrent respiratory morbidity.
Prevention of complications, early protective ventilation strategies and
rehabilitation improve long-term outcomes.

\section{Prevention and public health
considerations}\label{prevention-and-public-health-considerations}

Reducing the burden of respiratory failure in Ghana requires both
clinical and public health measures. Strengthening immunisation
(pneumococcal, Haemophilus influenzae type b, measles, pertussis, and
influenza), improving indoor air quality, promoting exclusive
breastfeeding, and early care-seeking for respiratory symptoms reduce
disease incidence. At the facility level, training in paediatric
emergency care, widespread availability of pulse oximetry, oxygen
concentrators, and basic CPAP devices have high impact even in
resource-limited hospitals.

\section{Practical tips for the Ghanaian
setting}\label{practical-tips-for-the-ghanaian-setting}

Simple interventions save lives: triage with pulse oximetry, give oxygen
early to any child with respiratory distress, use CPAP for neonates and
infants when available, and ensure rapid antibiotic administration for
suspected severe pneumonia. Implementing standardised early warning
signs, training in paediatric airway management, and protocols for
escalation of care greatly improve outcomes.

\section{Conclusion}\label{conclusion-8}

Respiratory failure in children is a medical emergency that demands
prompt recognition and decisive action. Familiarity with the
physiological differences of children, common causes in the local
context, and a stepwise approach to oxygenation and ventilation are
essential competencies for medical students and clinicians. While
advanced therapies exist, many deaths from respiratory failure are
preventable with timely basic interventions, improved public health
measures, and strengthened paediatric acute care capacity across Ghana.

\chapter{Asthma}\label{asthma}

\section{Introduction}\label{introduction-12}

Asthma is a chronic inflammatory disorder of the airways, characterized
by variable and recurring symptoms, airflow obstruction, bronchial
hyperresponsiveness, and underlying inflammation. It is one of the most
common chronic diseases in children worldwide, including in Ghana.
Effective management is essential in pediatric care, especially due to
its impact on the quality of life, school attendance, and healthcare
utilization.

Understanding asthma in children is crucial for early diagnosis,
effective management, and the prevention of complications. This note
outlines the epidemiology, pathophysiology, clinical features,
diagnosis, differential diagnoses, management, and public health
implications of childhood asthma, with a focus on the context of
Ghanaian healthcare.

\section{Epidemiology}\label{epidemiology}

Asthma affects an estimated 10-15\% of children in Ghana, although its
prevalence varies by region, urbanization, and environmental factors.
Urban areas such as Accra and Kumasi report higher cases due to
increased pollution, lifestyle changes, and indoor allergens.

\textbf{Risk Factors:}

\begin{itemize}
\tightlist
\item
  \textbf{Genetics:} Family history of asthma or atopy (eczema, allergic
  rhinitis).
\item
  \textbf{Environmental exposures:} Dust, smoke (including biomass
  fuel), pollution, and cockroach or mould allergens.
\item
  \textbf{Infections:} Respiratory syncytial virus (RSV), influenza.
\item
  \textbf{Socioeconomic status:} Overcrowded housing, poor ventilation.
\item
  \textbf{Early weaning or formula feeding.}
\end{itemize}

\section{Pathophysiology}\label{pathophysiology-2}

Asthma is mainly an inflammatory disease that affects the airways. In
children, this airway inflammation is often eosinophilic and results in:

\begin{enumerate}
\def\labelenumi{\arabic{enumi}.}
\tightlist
\item
  \textbf{Airway Hyperresponsiveness:} Increased sensitivity to triggers
  such as cold air, dust, or exercise.
\item
  \textbf{Bronchoconstriction:} Constriction of bronchial smooth muscles
  causes narrowing of airways.
\item
  \textbf{Airway Remodelling (in chronic cases):} Thickening of the
  basement membrane, increased mucus secretion, and smooth muscle
  hypertrophy.
\end{enumerate}

These changes contribute to the classic symptoms: wheezing, cough, chest
tightness, and shortness of breath.

\section{Clinical Features}\label{clinical-features-6}

The presentation of asthma in children may vary based on age and
severity. Key symptoms include:

\begin{itemize}
\tightlist
\item
  \textbf{Wheezing:} High-pitched whistling sound, often during
  expiration.
\item
  \textbf{Coughing:} Worse at night, early morning, or after exercise.
\item
  \textbf{Shortness of breath:} This is especially noticeable during
  exertion or with infections.
\item
  \textbf{Chest tightness or pain.}
\end{itemize}

\textbf{Patterns of Childhood Asthma:}

\begin{itemize}
\tightlist
\item
  \textbf{Intermittent asthma:} Symptoms occur less than twice a week.
\item
  \textbf{Persistent asthma:} Symptoms occur more frequently and may
  impact daily activities.
\item
  \textbf{Exercise-induced asthma:} Triggered by physical activity.
\item
  \textbf{Nocturnal asthma:} Symptoms worsen at night.
\item
  \textbf{Viral-induced wheeze:} Common in toddlers; often resolves with
  age.
\end{itemize}

In Ghana, children may also present late or with severe symptoms due to
poor access to healthcare or misdiagnosis.

\section{Diagnosis}\label{diagnosis-4}

Asthma is primarily a clinical diagnosis in children, especially those
under 5 years of age.

\textbf{1. History:}

\begin{itemize}
\tightlist
\item
  Recurrent episodes of cough, wheeze, and breathlessness.
\item
  Family or personal history of allergies.
\item
  Symptoms triggered by cold, dust, exercise, or smoke.
\end{itemize}

\textbf{2. Physical Examination:}

\begin{itemize}
\tightlist
\item
  Wheezing on auscultation.
\item
  Use of accessory muscles in severe cases.
\item
  Hyperresonance on percussion in chronic cases.
\end{itemize}

\textbf{3. Investigations:}

\begin{itemize}
\tightlist
\item
  \textbf{Spirometry (in children \textgreater5 years):} Shows
  reversible airway obstruction (FEV1/FVC ratio \textless{} 80\%).
\item
  \textbf{Peak Expiratory Flow Rate (PEFR):} Helps monitor asthma
  control.
\item
  \textbf{Chest X-ray:} To exclude other conditions (e.g., foreign body,
  pneumonia).
\item
  \textbf{Allergy testing:} Useful in atopic children (skin prick or
  serum IgE)
\end{itemize}

\textbf{Diagnostic Challenge in Ghana:}

\begin{itemize}
\tightlist
\item
  Limited access to spirometry in rural settings.
\item
  Reliance on clinical judgment.
\item
  Misdiagnosis as pneumonia or bronchitis is common.
\end{itemize}

\section{Differential Diagnosis}\label{differential-diagnosis}

\begin{itemize}
\tightlist
\item
  \textbf{Bronchiolitis:} Common in infants; usually due to viral
  infections.
\item
  \textbf{Foreign body aspiration:} Sudden onset of wheeze with
  localized findings.
\item
  \textbf{Pneumonia:} Fever with cough; may have focal crepitations or
  consolidation.
\item
  \textbf{Congenital anomalies:} E.g., tracheomalacia or vascular rings.
\item
  \textbf{Tuberculosis:} Chronic cough, weight loss, and a history of
  contact.
\end{itemize}

\section{Management}\label{management-7}

\textbf{1. Education and Self-Management}

\begin{itemize}
\tightlist
\item
  Educate caregivers and older children on:

  \begin{itemize}
  \tightlist
  \item
    Nature of asthma.
  \item
    Avoidance of triggers.
  \item
    Proper inhaler technique.
  \item
    Recognition of early warning signs.
  \item
    Importance of medication adherence
  \end{itemize}
\end{itemize}

\textbf{2. Pharmacologic Management}

\textbf{a. Reliever Medications:}

\begin{itemize}
\tightlist
\item
  \textbf{Short-acting beta2-agonists (SABA):} e.g., \emph{Salbutamol}.

  \begin{itemize}
  \tightlist
  \item
    First-line for acute symptoms.
  \item
    Delivered via metered-dose inhaler (MDI) with a spacer.
  \end{itemize}
\end{itemize}

\textbf{b. Controller Medications}

\begin{itemize}
\item
  \textbf{Inhaled corticosteroids (ICS):} e.g., \emph{Beclomethasone,
  Budesonide}

  \begin{itemize}
  \tightlist
  \item
    First-line for persistent asthma.
  \end{itemize}
\item
  \textbf{Leukotriene receptor antagonists (LTRA):} e.g.,
  \emph{Montelukast}.

  \begin{itemize}
  \tightlist
  \item
    Useful for allergic or exercise-induced asthma.
  \end{itemize}
\item
  \textbf{Long-acting beta2-agonists (LABA):} Used in combination with
  ICS in older children with poor control.
\end{itemize}

\textbf{c.~Systemic corticosteroids:}

\begin{itemize}
\tightlist
\item
  \emph{Prednisolone} for acute exacerbations (short course).
\end{itemize}

\textbf{3. Non-Pharmacological Measures}

\begin{itemize}
\tightlist
\item
  Avoid known allergens (dust, cockroach, pet dander).
\item
  Reduce exposure to cigarette smoke and biomass fuel.
\item
  Immunization (including flu vaccine where available).
\item
  Treatment of comorbidities (e.g., allergic rhinitis).
\end{itemize}

\section{Acute Exacerbations}\label{acute-exacerbations}

\textbf{Signs:}

\begin{itemize}
\tightlist
\item
  Rapid breathing, use of accessory muscles.
\item
  Inability to speak in full sentences.
\item
  Cyanosis or drowsiness (life-threatening).
\end{itemize}

\textbf{Management:}

\begin{enumerate}
\def\labelenumi{\arabic{enumi}.}
\tightlist
\item
  \textbf{Assess severity (mild, moderate, severe, life-threatening).}
\item
  \textbf{Oxygen therapy:} Maintain SpO₂ \textgreater{} 92\%.
\item
  \textbf{Nebulized SABA:} e.g., Salbutamol every 20 minutes for 1 hour.
\item
  \textbf{Oral corticosteroids:} Prednisolone 1--2 mg/kg/day for 3--5
  days.
\item
  \textbf{Ipratropium bromide:} In severe cases, combined with SABA.
\item
  \textbf{Magnesium sulphate IV:} In very severe or unresponsive cases.
\item
  \textbf{Referral:} If there is a poor response or worsening symptoms
\end{enumerate}

\section{Monitoring and Follow-up}\label{monitoring-and-follow-up}

\begin{itemize}
\tightlist
\item
  Review asthma control every 1--3 months.
\item
  Monitor growth in children on long-term corticosteroids.
\item
  PEFR monitoring for older children.
\item
  Step-up or step-down therapy based on control.
\end{itemize}

\section{Asthma Control Criteria (based on
GINA):}\label{asthma-control-criteria-based-on-gina}

\begin{itemize}
\tightlist
\item
  Daytime symptoms ≤2 times/week.
\item
  No night waking.
\item
  No limitation of activity.
\item
  Minimal reliever use.
\item
  No exacerbations.
\end{itemize}

\section{Challenges in the Ghanaian
Context}\label{challenges-in-the-ghanaian-context}

\begin{itemize}
\tightlist
\item
  \textbf{Limited diagnostic tools:} Lack of spirometry or PEFR in rural
  facilities.
\item
  \textbf{Access to medication:} Inhalers may be expensive or
  unavailable.
\item
  \textbf{Cultural beliefs:} Asthma is often attributed to spiritual
  causes.
\item
  \textbf{Poor adherence} Due to a lack of understanding or medication
  side effects.
\item
  \textbf{Stigma:} Especially among school children using inhalers.
\item
  \textbf{Environmental triggers:} Open burning, indoor smoke, and dust.
\end{itemize}

\section{Public Health Interventions}\label{public-health-interventions}

\begin{itemize}
\tightlist
\item
  \textbf{Health education:} Community sensitization on asthma and
  triggers.
\item
  \textbf{School health programs:} Identification and management of
  asthma in schools.
\item
  \textbf{Policy support:} Include essential asthma medications in the
  National Health Insurance Scheme (NHIS).
\item
  \textbf{Training healthcare providers:} On asthma diagnosis and
  management.
\end{itemize}

\section{Conclusion}\label{conclusion-9}

Asthma in children is a significant public health issue in Ghana. Early
recognition, accurate diagnosis, and comprehensive management can
greatly enhance outcomes. Medical students need to be prepared to
address asthma in both urban and rural environments, understand the
unique challenges in Ghana, and advocate for improved care across all
levels of the healthcare system.

Key Takeaways for Medical Students\textbf{:}

\begin{itemize}
\tightlist
\item
  Always consider asthma in a child with recurrent cough or wheeze.
\item
  A detailed history and clinical examination are often sufficient for
  diagnosis.
\item
  Use inhale corticosteroids for long-term control and SABAs for quick
  relief.
\item
  Educate families and monitor regularly.
\item
  Advocate for improved access to asthma care in Ghana.
\end{itemize}

\chapter{Bronchiolitis}\label{bronchiolitis}

\section{Introduction}\label{introduction-13}

Bronchiolitis is a common viral infection of the lower respiratory tract
that primarily affects infants and young children. It is the leading
cause of hospitalization for children under 2 years of age worldwide. In
Ghana and other sub-Saharan African countries, bronchiolitis
significantly contributes to infant morbidity and mortality, especially
during the harmattan season when respiratory infections are more
prevalent.

Understanding bronchiolitis is essential for medical students,
particularly in environments where diagnostic tools are scarce and
treatment depends largely on clinical skills and supportive care.

\section{Definition}\label{definition-7}

\textbf{Bronchiolitis} is defined as an \textbf{acute viral infection of
the lower respiratory tract}, primarily affecting the
\textbf{bronchioles}. It leads to inflammation, edema, and increased
mucus production, resulting in \textbf{airway obstruction},
\textbf{wheezing}, and \textbf{respiratory distress}.

\section{Epidemiology}\label{epidemiology-1}

\begin{itemize}
\tightlist
\item
  \textbf{Age group}: Primarily affects children \textbf{under 2 years},
  most commonly \textbf{under 6 months}
\item
  \textbf{Peak incidence}: During the \textbf{cold and dry months}
  (November to February in Ghana)
\item
  \textbf{Transmission}: Highly contagious; spread via
  \textbf{respiratory droplets}, \textbf{direct contact}, or
  \textbf{contaminated surfaces}
\item
  \textbf{High-risk groups}:

  \begin{itemize}
  \tightlist
  \item
    Premature infants
  \item
    Infants with congenital heart disease
  \item
    Children with chronic lung disease
  \item
    Immunocompromised children
  \item
    Children exposed to tobacco smoke or indoor air pollution
  \end{itemize}
\end{itemize}

\section{Etiology (Causative Agents)}\label{etiology-causative-agents}

The most common cause is \textbf{Respiratory Syncytial Virus (RSV)},
responsible for 50--80\% of cases.

\textbf{Other viruses:}

\begin{itemize}
\tightlist
\item
  Human metapneumovirus
\item
  Parainfluenza virus
\item
  Influenza virus
\item
  Rhinovirus
\item
  Adenovirus
\item
  Coronavirus
\end{itemize}

\section{Pathophysiology}\label{pathophysiology-3}

\begin{enumerate}
\def\labelenumi{\arabic{enumi}.}
\tightlist
\item
  \textbf{Viral infection} of the \textbf{nasal and lower respiratory
  epithelium}
\item
  Inflammation and edema of the bronchioles
\item
  Necrosis and sloughing of epithelial cells
\item
  Increased mucus production and plugging of small airways
\item
  Air trapping and \textbf{hyperinflation}, leading to:

  \begin{itemize}
  \tightlist
  \item
    Increased work of breathing
  \item
    Impaired gas exchange
  \item
    \textbf{Hypoxia} and, in severe cases, \textbf{respiratory failure}
  \end{itemize}
\end{enumerate}

\section{Clinical Features}\label{clinical-features-7}

\textbf{History}

\begin{itemize}
\tightlist
\item
  Starts as an \textbf{upper respiratory tract infection} (e.g., runny
  nose, mild cough)
\item
  Progresses over 2--3 days to:

  \begin{itemize}
  \tightlist
  \item
    \textbf{Cough}
  \item
    \textbf{Tachypnea}
  \item
    \textbf{Wheezing}
  \item
    \textbf{Poor feeding}
  \item
    \textbf{Apnea} (especially in premature or very young infants)
  \item
    \textbf{Fever} (may or may not be present
  \end{itemize}
\end{itemize}

\textbf{Examination}

\begin{itemize}
\tightlist
\item
  \textbf{Tachypnea}
\item
  \textbf{Nasal flaring}
\item
  \textbf{Chest retractions} (intercostal, subcostal, suprasternal)
\item
  \textbf{Wheezing} and \textbf{crackles} on auscultation
\item
  \textbf{Hypoxia} (low oxygen saturation)
\item
  \textbf{Dehydration}
\item
  \textbf{Cyanosis} in severe cases
\end{itemize}

\section{Differential Diagnosis}\label{differential-diagnosis-1}

\begin{longtable}[]{@{}
  >{\raggedright\arraybackslash}p{(\linewidth - 2\tabcolsep) * \real{0.5000}}
  >{\raggedright\arraybackslash}p{(\linewidth - 2\tabcolsep) * \real{0.5000}}@{}}
\toprule\noalign{}
\endhead
\bottomrule\noalign{}
\endlastfoot
\textbf{Condition} & \textbf{Key Features} \\
Asthma & Older children (\textgreater2 years), recurrent episodes,
personal/family history of atopy \\
Pneumonia & Fever, focal crackles, lobar consolidation on chest X-ray \\
Foreign body aspiration & Sudden onset, localized wheeze, asymmetric
breath sounds \\
Congenital heart disease & Cyanosis, poor weight gain, murmur \\
Pertussis & Paroxysmal cough, whoop, post-tussive vomiting \\
\end{longtable}

\section{Diagnosis}\label{diagnosis-5}

\textbf{Clinical diagnosis is key in most settings, especially where
investigations are limited.}

\textbf{Investigations (if available)}

\begin{itemize}
\tightlist
\item
  \textbf{Pulse oximetry}: Assess oxygen saturation
\item
  \textbf{Chest X-ray} (not routinely indicated): May show
  hyperinflation, peribronchial thickening, patchy atelectasis
\item
  \textbf{Nasopharyngeal swab} for viral testing (e.g., RSV) -- rarely
  available in Ghana
\item
  \textbf{Complete blood count}: To rule out bacterial infection if
  fever is high or toxic appearance
\item
  \textbf{Serum electrolytes}: In severely ill or dehydrated children
\end{itemize}

\section{Severity Assessment}\label{severity-assessment}

\textbf{Mild}

\begin{itemize}
\tightlist
\item
  Normal feeding
\item
  Mild tachypnea, minimal retractions
\item
  Oxygen saturation ≥ 92\%
\end{itemize}

\textbf{Moderate}

\begin{itemize}
\tightlist
\item
  Poor feeding
\item
  Moderate tachypnea and retractions
\item
  Wheezing or crackles
\item
  Oxygen saturation 90--92\%
\end{itemize}

\textbf{Severe}

\begin{itemize}
\tightlist
\item
  Marked retractions, grunting, nasal flaring
\item
  Apnea
\item
  Cyanosis
\item
  Oxygen saturation \textless{} 90\%
\item
  Lethargy or altered mental status
\end{itemize}

\section{Management}\label{management-8}

\subsection{General Principles}\label{general-principles}

\begin{itemize}
\tightlist
\item
  Most cases are \textbf{self-limiting} and can be managed with
  \textbf{supportive care}
\item
  Hospitalization is required for:

  \begin{itemize}
  \tightlist
  \item
    Moderate to severe disease
  \item
    Apnea
  \item
    Inability to feed
  \item
    Oxygen saturation \textless{} 90\%
  \item
    High-risk infants
  \end{itemize}
\end{itemize}

\subsection{Outpatient (Home-Based)
Management}\label{outpatient-home-based-management}

\begin{itemize}
\tightlist
\item
  Ensure \textbf{adequate hydration} and feeding
\item
  Educate caregivers on danger signs:

  \begin{itemize}
  \tightlist
  \item
    Rapid breathing
  \item
    Chest in-drawing
  \item
    Inability to feed
  \item
    Cyanosis
  \item
    Lethargy
  \end{itemize}
\item
  Clear nasal secretions with saline drops/suction
\item
  Follow-up in 24--48 hours
\end{itemize}

\subsection{Inpatient (Hospital)
Management}\label{inpatient-hospital-management}

\textbf{1. Supportive Care}

\begin{itemize}
\tightlist
\item
  \textbf{Oxygen therapy}:

  \begin{itemize}
  \tightlist
  \item
    Give oxygen if SpO₂ \textless{} 90\%
  \item
    Via nasal prongs or face mask
  \end{itemize}
\item
  \textbf{Hydration and nutrition}:

  \begin{itemize}
  \tightlist
  \item
    Encourage breastfeeding or oral feeds
  \item
    NG tube feeding or IV fluids if unable to feed orally
  \end{itemize}
\item
  \textbf{Monitoring}:

  \begin{itemize}
  \tightlist
  \item
    Respiratory rate
  \item
    Oxygen saturation
  \item
    Fluid status
  \item
    Level of consciousness
  \end{itemize}
\end{itemize}

\subsection{Medications (Avoid routine
use)}\label{medications-avoid-routine-use}

\begin{longtable}[]{@{}
  >{\raggedright\arraybackslash}p{(\linewidth - 2\tabcolsep) * \real{0.5000}}
  >{\raggedright\arraybackslash}p{(\linewidth - 2\tabcolsep) * \real{0.5000}}@{}}
\toprule\noalign{}
\endhead
\bottomrule\noalign{}
\endlastfoot
\textbf{Medication} & \textbf{Recommendation} \\
\textbf{Bronchodilators (e.g., salbutamol)} & Not routinely recommended;
trial may be considered in wheezing children \textgreater12 months \\
\textbf{Steroids} & Not beneficial in uncomplicated bronchiolitis \\
\textbf{Antibiotics} & Not indicated unless bacterial co-infection
suspected (e.g., pneumonia, otitis media) \\
\textbf{Nebulized hypertonic saline} & Limited evidence; not routinely
used in Ghana \\
\textbf{Antiviral agents} & Not routinely available or used in Ghana \\
\end{longtable}

\section{Complications}\label{complications-1}

\begin{itemize}
\tightlist
\item
  \textbf{Apnea}
\item
  \textbf{Respiratory failure}
\item
  \textbf{Dehydration and poor nutrition}
\item
  \textbf{Secondary bacterial infections}
\item
  \textbf{Recurrent wheezing or asthma-like episodes} later in life
\item
  \textbf{Death} (in severe, untreated cases, particularly in high-risk
  infants)
\end{itemize}

\section{Prevention}\label{prevention-1}

\textbf{1. Infection Control}

\begin{itemize}
\tightlist
\item
  \textbf{Hand hygiene}
\item
  \textbf{Avoid crowding}, especially in daycares and nurseries
\item
  Educate caregivers on \textbf{cough etiquette}
\end{itemize}

\textbf{2. Breastfeeding}

\begin{itemize}
\tightlist
\item
  Exclusive breastfeeding for the first \textbf{6 months} provides
  protective antibodies
\end{itemize}

\textbf{3. Avoid Smoke Exposure}

\begin{itemize}
\tightlist
\item
  Avoid smoking near infants
\item
  Reduce indoor air pollution (e.g., smoke from firewood)
\end{itemize}

\textbf{4. Immunization}

\begin{itemize}
\tightlist
\item
  Ensure up-to-date \textbf{vaccination}, especially:

  \begin{itemize}
  \tightlist
  \item
    \textbf{Influenza vaccine}
  \item
    \textbf{Pneumococcal vaccine}
  \item
    \textbf{Pertussis vaccine}
  \end{itemize}
\end{itemize}

\textbf{5. Prophylaxis (Palivizumab)}

\begin{itemize}
\tightlist
\item
  A monoclonal antibody used for \textbf{RSV prophylaxis}
\item
  Expensive and not readily available in Ghana
\item
  Considered only for \textbf{very high-risk infants} in specialized
  centers
\end{itemize}

\section{Prognosis}\label{prognosis-1}

\begin{itemize}
\tightlist
\item
  Most children \textbf{recover fully within 7--10 days}
\item
  \textbf{Cough} may persist for 2--3 weeks
\item
  Infants with severe disease may have \textbf{recurrent wheezing or
  asthma}
\end{itemize}

\section{Special Considerations in
Ghana}\label{special-considerations-in-ghana}

\begin{itemize}
\tightlist
\item
  \textbf{Overcrowded homes} and \textbf{poor air quality} increase risk
\item
  Health-seeking behavior may be delayed due to cultural beliefs or
  access issues
\item
  Resource limitations often mean:

  \begin{itemize}
  \tightlist
  \item
    Reliance on clinical diagnosis
  \item
    Limited access to oxygen and pulse oximetry
  \end{itemize}
\item
  Need for \textbf{education of caregivers} about early signs of
  respiratory distress
\item
  Emphasize \textbf{community-based health interventions} (e.g., CHPS
  compounds)
\end{itemize}

\section{Case Scenario}\label{case-scenario}

\textbf{Case: 4-month-old male infant}

\textbf{Presentation:}

\begin{itemize}
\tightlist
\item
  3-day history of cough, runny nose, and poor feeding
\item
  Developed fast breathing and wheezing today
\item
  No fever
\item
  No significant past medical history
\end{itemize}

\textbf{On examination:}

\begin{itemize}
\tightlist
\item
  RR: 68 breaths/min
\item
  Chest retractions present
\item
  O₂ saturation: 88\% on room air
\item
  Nasal flaring, scattered wheeze
\end{itemize}

\textbf{Diagnosis:}

\begin{itemize}
\tightlist
\item
  Likely \textbf{moderate to severe bronchiolitis}
\end{itemize}

\textbf{Management:}

\begin{itemize}
\tightlist
\item
  Admit for supportive care
\item
  Oxygen via nasal prongs
\item
  NG tube feeding due to poor suck
\item
  Monitor vitals and oxygen saturation
\item
  Educate mother on hand hygiene and signs of deterioration
\end{itemize}

\section{Summary Table}\label{summary-table-1}

\begin{longtable}[]{@{}
  >{\raggedright\arraybackslash}p{(\linewidth - 2\tabcolsep) * \real{0.3151}}
  >{\raggedright\arraybackslash}p{(\linewidth - 2\tabcolsep) * \real{0.6849}}@{}}
\toprule\noalign{}
\endhead
\bottomrule\noalign{}
\endlastfoot
\textbf{Feature} & \textbf{Bronchiolitis} \\
Age group & \textless{} 2 years (commonest \textless{} 6 months) \\
Onset & Gradual, following URTI \\
Common virus & RSV \\
Main symptoms & Cough, wheeze, tachypnea, and feeding difficulty \\
Diagnosis & Clinical \\
Mainstay of treatment & Supportive care \\
Antibiotics & Not routinely indicated \\
Oxygen & If SpO₂ \textless{} 90\% \\
Prognosis & Excellent in most cases \\
\end{longtable}

\section{Conclusion}\label{conclusion-10}

Bronchiolitis is a common and potentially severe illness affecting
infants and young children in Ghana. Early recognition and supportive
management are essential to preventing complications. Medical students
need to be familiar with its presentation, clinical evaluation, and
evidence-based treatment, especially in resource-limited healthcare
settings where advanced diagnostics may not be available.

\chapter{Croup}\label{croup}

\section{Introduction}\label{introduction-14}

Croup, medically known as laryngotracheobronchitis, is a common acute
upper respiratory illness in children, characterized by inspiratory
stridor, a barking cough, and hoarseness. It typically results from a
viral infection that causes inflammation of the larynx, trachea, and
bronchi. Though it is usually self-limiting, it can occasionally lead to
life-threatening airway obstruction. Croup is particularly important for
medical students and healthcare providers in Ghana, where respiratory
infections are a leading cause of childhood morbidity, particularly
during the rainy season when viral infections peak.

\section{Epidemiology}\label{epidemiology-2}

\begin{itemize}
\tightlist
\item
  \textbf{Age group}: Primarily affects children between \textbf{6
  months and 5 years}. The peak incidence occurs around \textbf{2 years
  of age}.
\item
  \textbf{Gender}: Males are slightly more affected than females.
\item
  \textbf{Seasonality}: Most cases occur during the rainy or cold
  seasons (June to October in Ghana), coinciding with an increase in
  viral respiratory infections.
\item
  \textbf{Prevalence}: Although there is limited Ghana-specific data,
  studies across sub-Saharan Africa indicate that viral croup accounts
  for a significant proportion of paediatric respiratory admissions,
  particularly in urban areas such as Accra and Kumasi.
\end{itemize}

\section{Etiology}\label{etiology}

Viral infections most commonly cause croup. The \textbf{Parainfluenza
virus type 1} is the most frequent cause globally and in Ghana.

\textbf{Common viral agents:}

\begin{itemize}
\tightlist
\item
  \textbf{Parainfluenza viruses} (types 1, 2, 3)
\item
  \textbf{Respiratory syncytial virus (RSV)}
\item
  \textbf{Influenza A and B}
\item
  \textbf{Adenoviruses}
\item
  \textbf{Rhinoviruses}
\item
  \textbf{Coronavirus (including some SARS-CoV-2 variants)}
\end{itemize}

These viruses infect and inflame the epithelial lining of the
\textbf{upper airway}, leading to swelling, increased mucus, and
narrowed air passages, especially in the subglottic region.

\section{Pathophysiology}\label{pathophysiology-4}

The hallmark of croup is \textbf{subglottic inflammation}. In the
paediatric airway, the narrowest part is the \textbf{subglottic space},
located just below the vocal cords. Viral infection triggers:

\begin{itemize}
\tightlist
\item
  Mucosal oedema
\item
  Cellular infiltration
\item
  Increased mucus production
\end{itemize}

These changes reduce airway diameter, particularly during
\textbf{inspiration}, leading to:

\begin{itemize}
\tightlist
\item
  \textbf{Stridor} (turbulent airflow)
\item
  \textbf{Barking cough} (from irritated vocal cords)
\item
  \textbf{Respiratory distress} in severe cases
\end{itemize}

Young children are especially vulnerable due to their \textbf{smaller
airway diameter} and less developed respiratory musculature.

\section{Clinical Features}\label{clinical-features-8}

The classic presentation involves:

\textbf{Prodromal Phase:}

\begin{itemize}
\item
  Begins with \textbf{non-specific upper respiratory symptoms}:

  \begin{itemize}
  \tightlist
  \item
    Nasal congestion
  \item
    Rhinorrhoea
  \item
    Low-grade fever
  \item
    Mild cough
  \end{itemize}
\end{itemize}

\textbf{Croup Syndrome:}

\begin{itemize}
\tightlist
\item
  \textbf{Barking cough} (seal-like)
\item
  \textbf{Hoarseness}
\item
  \textbf{Inspiratory stridor} (worse with agitation or crying)
\item
  \textbf{Respiratory distress} (tachypnoea, nasal flaring, retractions)
\item
  \textbf{Fever} (low to moderate)
\end{itemize}

Symptoms often \textbf{worsen at night}, leading to sudden parental
concern.

\textbf{Severity Classification:}

\begin{enumerate}
\def\labelenumi{\arabic{enumi}.}
\tightlist
\item
  \textbf{Mild}:

  \begin{itemize}
  \tightlist
  \item
    Occasional barking cough
  \item
    No stridor at rest
  \item
    No retractions
  \end{itemize}
\item
  \textbf{Moderate}:

  \begin{itemize}
  \tightlist
  \item
    Frequent cough
  \item
    Stridor at rest
  \item
    Mild to moderate chest wall retractions
  \end{itemize}
\item
  \textbf{Severe}

  \begin{itemize}
  \tightlist
  \item
    Marked stridor at rest
  \item
    Severe retractions
  \item
    Agitation or lethargy
  \item
    Hypoxia (SpO₂ \textless{} 92\%)
  \end{itemize}
\item
  \textbf{Impending respiratory failure}:

  \begin{itemize}
  \tightlist
  \item
    Decreased level of consciousness
  \item
    Fatigue
  \item
    Cyanosis
  \item
    Silent chest
  \end{itemize}
\end{enumerate}

\section{Differential Diagnoses}\label{differential-diagnoses}

Croup must be differentiated from other \textbf{causes of upper airway
obstruction}:

\begin{longtable}[]{@{}
  >{\raggedright\arraybackslash}p{(\linewidth - 2\tabcolsep) * \real{0.2871}}
  >{\raggedright\arraybackslash}p{(\linewidth - 2\tabcolsep) * \real{0.7129}}@{}}
\toprule\noalign{}
\endhead
\bottomrule\noalign{}
\endlastfoot
\textbf{Condition} & \textbf{Key Differences} \\
\textbf{Epiglottitis} & Sudden onset, high fever, toxic appearance,
drooling, ``tripod'' posture \\
\textbf{Foreign body aspiration} & Sudden choking episode, unilateral
breath sounds \\
\textbf{Bacterial tracheitis} & High fever, purulent secretions, toxic
look \\
\textbf{Peritonsillar abscess} & Older children, muffled voice,
difficulty opening mouth \\
\textbf{Retropharyngeal abscess} & Neck stiffness, drooling, visible
swelling on imaging \\
\end{longtable}

\section{Diagnosis}\label{diagnosis-6}

Croup is primarily a \textbf{clinical diagnosis}, especially in
resource-limited settings like many areas in Ghana.

\textbf{Clinical Evaluation:}

\begin{itemize}
\tightlist
\item
  Vital signs: look for tachypnoea, fever
\item
  Oxygen saturation (pulse oximetry)
\item
  General appearance: level of alertness, work of breathing
\end{itemize}

Investigations

\begin{itemize}
\tightlist
\item
  \textbf{Neck X-ray (AP view)}: May reveal the classic ``steeple sign''
  (subglottic narrowing), although it is not routinely needed.
\item
  \textbf{CBC, CRP}: Not usually necessary unless bacterial
  superinfection is suspected.
\item
  \textbf{Nasopharyngeal swabs}: Can confirm viral aetiology, but are
  rarely done due to cost and availability.
\end{itemize}

\section{Management}\label{management-9}

Management depends on \textbf{severity}. The key principles are:

\begin{itemize}
\tightlist
\item
  Relieve airway obstruction
\item
  Reduce inflammation
\item
  Minimize agitation
\item
  Monitor for deterioration
\end{itemize}

\subsection{\texorpdfstring{\textbf{General
Measures:}}{General Measures:}}\label{general-measures}

\begin{itemize}
\tightlist
\item
  \textbf{Keep the child calm}: Crying worsens stridor.
\item
  \textbf{Humidified air}: Traditionally used, though evidence is weak.
\item
  \textbf{Supplemental oxygen}: For SpO₂ \textless{} 92\% or signs of
  hypoxia.
\end{itemize}

\subsection{\texorpdfstring{\textbf{Pharmacologic
Treatment}}{Pharmacologic Treatment}}\label{pharmacologic-treatment}

\textbf{1. Corticosteroids}

Mainstay of treatment, regardless of severity.

\begin{itemize}
\tightlist
\item
  \textbf{Dexamethasone} (preferred):

  \begin{itemize}
  \tightlist
  \item
    Dose: 0.15--0.6 mg/kg PO/IM/IV (max 10 mg)
  \item
    Long half-life (\textasciitilde36--72 hours), a single dose is often
    enough
  \end{itemize}
\item
  \textbf{Prednisolone} (if dexamethasone unavailable):

  \begin{itemize}
  \tightlist
  \item
    Dose: 1 mg/kg/day PO for 3--5 days
  \end{itemize}
\end{itemize}

\textbf{Corticosteroids reduce airway inflammation, decrease hospital
admissions, and shorten the duration of illness.}

\textbf{2. Nebulized Epinephrine (Racemic or L-epinephrine)}

\begin{itemize}
\tightlist
\item
  ~Used for moderate to severe croup:
\item
  Dose: 0.5 mL of 2.25\% racemic epinephrine or 5 mL of 1:1000
  L-epinephrine via nebulizer.
\item
  Acts quickly but temporarily (1--2 hours), often used while waiting
  for the corticosteroid effect.
\item
  Observe the child for 3--4 hours after administration for any rebound
  symptoms.
\end{itemize}

\textbf{3. Antibiotics}

Not indicated unless there is a suspicion of bacterial tracheitis or a
secondary infection (high fever, toxic appearance, purulent secretions).

\textbf{Monitoring and Admission Criteria}

\textbf{Admit if:}

\begin{itemize}
\tightlist
\item
  ~Persistent stridor at rest following epinephrine
\item
  Hypoxia (SpO₂ \textless{} 92\% on room air)
\item
  Severe work of breathing
\item
  Inadequate oral intake
\item
  Age under 6 months
\item
  Pre-existing comorbidities (e.g., sickle cell disease, malnutrition)
\end{itemize}

In Ghana, \textbf{admission should also be considered if reliable
follow-up is uncertain}, especially in rural or underserved areas.

\section{\texorpdfstring{\textbf{Complications}}{Complications}}\label{complications-2}

\begin{itemize}
\tightlist
\item
  Respiratory failure
\item
  Secondary bacterial tracheitis
\item
  Dehydration
\item
  Rarely, death (usually in severe, untreated cases)
\end{itemize}

\section{\texorpdfstring{\textbf{Prevention}}{Prevention}}\label{prevention-2}

\begin{itemize}
\tightlist
\item
  \textbf{Routine immunization}: Influenza and measles vaccines reduce
  incidence
\item
  \textbf{Hand hygiene and cough etiquette}
\item
  \textbf{Avoid exposure to sick contacts}, especially during viral
  seasons
\end{itemize}

\section{\texorpdfstring{\textbf{Public Health Considerations in
Ghana}}{Public Health Considerations in Ghana}}\label{public-health-considerations-in-ghana}

\begin{itemize}
\tightlist
\item
  \textbf{Limited access to nebulizers or corticosteroids} in rural
  facilities may delay treatment.
\item
  \textbf{Overcrowding and poor ventilation} increase the transmission
  of respiratory viruses.
\item
  \textbf{Training} \textbf{community health workers} in the recognition
  and referral of severecases is crucial.
\item
  Integration of \textbf{Integrated Management of Childhood Illness
  (IMCI)} strategies can help guide early treatment at the primary care
  level.
\end{itemize}

\section{\texorpdfstring{\textbf{Conclusion}}{Conclusion}}\label{conclusion-11}

Croup is a common, self-limiting pediatric illness that can become
life-threatening without prompt recognition and management. Medical
students and practitioners in Ghana should be proficient in diagnosing
croup based on clinical features and effectively managing it with
corticosteroids and supportive care. Knowing when to escalate care is
crucial, particularly in resource-constrained settings.

\chapter{Pneumonia}\label{pneumonia}

\section{Introduction}\label{introduction-15}

Pneumonia is an acute infection of the lung parenchyma, leading to
inflammation and consolidation of the alveoli. It remains a major cause
of childhood morbidity and mortality worldwide and is especially
significant in low- and middle-income countries such as Ghana. Despite
progress in immunization and child health services, pneumonia continues
to account for a large proportion of paediatric hospital admissions and
deaths, particularly among children under five years of age.

Understanding its causes, clinical presentation, and management is
essential for medical students and young clinicians. The disease
spectrum ranges from mild, self-limiting illness to severe,
life-threatening conditions requiring intensive care.

\section{Epidemiology and Burden}\label{epidemiology-and-burden}

Globally, pneumonia is responsible for approximately 14\% of all deaths
in children under five. In sub-Saharan Africa, the burden is
disproportionately high due to limited access to healthcare,
malnutrition, and environmental risk factors such as indoor air
pollution.

In Ghana, pneumonia is among the top five causes of under-five
mortality. Both viral and bacterial pneumonias are common, and
coinfections such as malaria, tuberculosis, or HIV-associated infections
complicate the picture. The disease shows a seasonal pattern, often
peaking during the rainy seasons when respiratory viruses are more
prevalent. Neonates and young infants, malnourished children, and those
with underlying chronic conditions such as congenital heart disease or
HIV are at greater risk.

\section{Aetiology}\label{aetiology}

The causes of pneumonia vary with age, immune status, and environment.

\subsection{\texorpdfstring{\textbf{In
Neonates:}}{In Neonates:}}\label{in-neonates}

\begin{itemize}
\tightlist
\item
  \emph{Bacterial:} Group B Streptococcus, Escherichia coli, Klebsiella
  species, Listeria monocytogenes.
\item
  \emph{Viral:} Respiratory syncytial virus (RSV) and cytomegalovirus
  (in congenital infection).
\end{itemize}

\subsection{\texorpdfstring{\textbf{In Infants and Young
Children:}}{In Infants and Young Children:}}\label{in-infants-and-young-children}

\begin{itemize}
\tightlist
\item
  \emph{Bacterial:} \textbf{Streptococcus pneumoniae} (pneumococcus) is
  the most common; \textbf{Haemophilus influenzae type b (Hib)} is
  important where vaccination coverage is low. \textbf{Staphylococcus
  aureus} causes severe, necrotizing pneumonia with empyema.
\item
  \emph{Viral:} RSV, parainfluenza, influenza, adenovirus, and human
  metapneumovirus are frequent, especially in the first two years of
  life.
\item
  \emph{Atypical:} \emph{Mycoplasma pneumoniae} and \emph{Chlamydia
  pneumoniae} appear more often in older children and adolescents.
\end{itemize}

\subsection{\texorpdfstring{\textbf{In Immunocompromised
Children:}}{In Immunocompromised Children:}}\label{in-immunocompromised-children}

\begin{itemize}
\tightlist
\item
  Opportunistic infections such as \emph{Pneumocystis jirovecii},
  cytomegalovirus, and fungal pneumonias may occur, particularly in
  HIV-positive children.
\end{itemize}

Environmental exposures, malnutrition, passive smoking, and crowded
living conditions amplify susceptibility.

\section{Pathophysiology}\label{pathophysiology-5}

The lungs normally maintain sterility through effective mucociliary
clearance, immune defenses, and cough reflexes. Pneumonia develops when
these defenses are breached --- by overwhelming microbial inoculation,
impaired clearance, or immune compromise.

Microorganisms reach the alveoli by inhalation, aspiration, or via the
bloodstream. The host immune response leads to inflammation, exudation
of fluid and cells into the alveolar spaces, and impaired gas exchange.

\textbf{Typical bacterial pneumonia} leads to alveolar consolidation ---
a process in which alveoli are filled with exudate containing
neutrophils and fibrin. This impedes oxygen diffusion and causes
hypoxaemia.

\textbf{Viral pneumonia}, in contrast, causes interstitial inflammation,
airway oedema, and epithelial necrosis, predisposing to secondary
bacterial infection.

The degree of impairment depends on the virulence of the organism and
host factors such as nutritional status, immunization history, and
presence of comorbidities.

\section{Clinical Features}\label{clinical-features-9}

The presentation of pneumonia varies with age and severity.

\textbf{General symptoms include:}

\begin{itemize}
\tightlist
\item
  Fever, often high-grade.
\item
  Cough, which may be dry or productive in older children.
\item
  Difficulty in breathing, nasal flaring, and grunting in infants.
\item
  Poor feeding, irritability, or lethargy.
\end{itemize}

\textbf{Physical findings:}

\begin{itemize}
\tightlist
\item
  Tachypnoea is the most sensitive clinical sign. The WHO defines
  tachypnoea as:

  \begin{itemize}
  \tightlist
  \item
    \textgreater60 breaths/min in infants \textless2 months
  \item
    \textgreater50 breaths/min in 2--12 months
  \item
    \textgreater40 breaths/min in 1--5 years
  \end{itemize}
\item
  Chest indrawing, nasal flaring, or grunting indicate severe disease.
\item
  Auscultation may reveal crackles, bronchial breath sounds, or
  decreased air entry.
\item
  Cyanosis, hypoxia, and altered sensorium are signs of respiratory
  failure.
\end{itemize}

Infants and neonates may have nonspecific presentations; temperature
instability, apnea, or poor feeding, making a high index of suspicion
essential.

\section{Differential Diagnosis}\label{differential-diagnosis-2}

Several other conditions can mimic pneumonia, and distinguishing them is
crucial for correct management:

\begin{itemize}
\tightlist
\item
  \textbf{Bronchiolitis} (in infants under 2 years) --- wheezing and
  diffuse crackles rather than localized findings.
\item
  \textbf{Asthma or viral-induced wheeze} --- recurrent episodes with
  reversible airway obstruction.
\item
  \textbf{Pulmonary tuberculosis} --- chronic cough, weight loss, and
  failure to thrive, often with contact history.
\item
  \textbf{Severe malaria} --- fever and respiratory distress due to
  metabolic acidosis.
\item
  \textbf{Congestive heart failure} --- history of cardiac disease and
  signs of cardiomegaly or murmurs.
\end{itemize}

Clinical judgement, aided by investigations, guides differentiation.

\section{Investigations}\label{investigations-4}

Diagnosis is often clinical, especially in resource-limited settings.
However, investigations help confirm and classify pneumonia, identify
complications, and guide therapy.

\textbf{Basic Investigations:}

\begin{enumerate}
\def\labelenumi{\arabic{enumi}.}
\tightlist
\item
  \textbf{Pulse oximetry} --- to assess oxygen saturation; hypoxaemia
  (\textless92\%) indicates severe disease.
\item
  \textbf{Chest X-ray} --- shows lobar consolidation, interstitial
  infiltrates, or pleural effusion.
\item
  \textbf{Full blood count} --- leukocytosis with neutrophilia suggests
  bacterial infection; lymphocytosis may indicate viral infection.
\item
  \textbf{Blood culture} --- useful for identifying pathogens but often
  low yield.
\item
  \textbf{Nasopharyngeal aspirate or PCR testing} --- for viral
  pathogens where available.
\end{enumerate}

\textbf{Further investigations} in selected cases:

\begin{enumerate}
\def\labelenumi{\arabic{enumi}.}
\tightlist
\item
  \textbf{Sputum culture or tracheal aspirate} (in ventilated patients).
\item
  \textbf{HIV testing} in children with recurrent or severe pneumonia.
\item
  \textbf{Ultrasound or CT scan} if empyema, abscess, or foreign body is
  suspected.
\end{enumerate}

In Ghana, reliance is often on clinical diagnosis supported by simple
tests due to cost and availability limitations.

\section{Treatment}\label{treatment}

The management of pneumonia involves supportive care, antimicrobial
therapy, and treatment of complications.

\subsection{\texorpdfstring{\textbf{1. Supportive
Management}}{1. Supportive Management}}\label{supportive-management}

\begin{itemize}
\tightlist
\item
  \textbf{Oxygen therapy} for hypoxaemia using nasal prongs or face
  mask.
\item
  \textbf{Hydration:} maintain fluid balance; overhydration may worsen
  pulmonary oedema.
\item
  \textbf{Antipyretics} (e.g., paracetamol) for fever.
\item
  \textbf{Nutritional support} to prevent catabolism.
\item
  \textbf{Monitoring} of respiratory rate, SpO₂, and consciousness
  level.
\end{itemize}

\subsection{\texorpdfstring{\textbf{2. Antibiotic
Therapy}}{2. Antibiotic Therapy}}\label{antibiotic-therapy}

Empiric treatment is based on the likely pathogen and local resistance
patterns:

\begin{itemize}
\tightlist
\item
  \textbf{Neonates:} Ampicillin plus gentamicin for 7--10 days.
\item
  \textbf{Infants and older children:}

  \begin{itemize}
  \tightlist
  \item
    Outpatient: Oral amoxicillin for 5--7 days for mild pneumonia.
  \item
    Inpatient (severe): IV ampicillin (or penicillin) plus gentamicin;
    add cloxacillin or ceftriaxone if \emph{S. aureus} or Gram-negative
    sepsis is suspected.
  \item
    Macrolides (e.g., azithromycin) for atypical infections.
  \end{itemize}
\end{itemize}

Treatment is modified based on clinical response or culture results.

\subsection{\texorpdfstring{\textbf{3. Management of
Complications}}{3. Management of Complications}}\label{management-of-complications}

\begin{itemize}
\tightlist
\item
  \textbf{Pleural effusion/empyema:} chest tube drainage and
  antibiotics.
\item
  \textbf{Lung abscess:} prolonged antibiotic therapy; drainage if
  necessary.
\item
  \textbf{Septicemia:} aggressive IV antibiotics and supportive care.
\item
  \textbf{Respiratory failure:} CPAP or mechanical ventilation as
  indicated.
\end{itemize}

\subsection{\texorpdfstring{\textbf{4. Discharge and
Follow-up}}{4. Discharge and Follow-up}}\label{discharge-and-follow-up}

Discharge once afebrile, feeding well, and maintaining oxygen saturation
in room air. Follow-up in 1--2 weeks ensures full recovery and detects
post-pneumonia complications such as bronchiectasis.

\section{Complications}\label{complications-3}

Complications occur more commonly with delayed treatment or virulent
organisms. They include:

\begin{itemize}
\tightlist
\item
  Parapneumonic effusion or empyema
\item
  Lung abscess
\item
  Pneumatocele formation
\item
  Septicemia and metastatic abscesses
\item
  Bronchiectasis or chronic lung disease
\item
  Acute respiratory failure and death
\end{itemize}

Prompt recognition and intervention are vital to prevent long-term
morbidity.

\section{Prevention}\label{prevention-3}

Preventive measures are among the most cost-effective interventions in
child health.

\textbf{Immunization} plays a key role:

\begin{enumerate}
\def\labelenumi{\arabic{enumi}.}
\tightlist
\item
  Pneumococcal conjugate vaccine (PCV13)
\item
  Haemophilus influenzae type b (Hib) vaccine
\item
  Measles and pertussis vaccines
\item
  Annual influenza vaccination in at-risk groups
\end{enumerate}

\textbf{Nutrition:} Exclusive breastfeeding for the first six months,
adequate complementary feeding, and vitamin A supplementation strengthen
immunity.

\textbf{Environmental control:} Reducing exposure to tobacco smoke and
indoor air pollution from biomass fuels.

\textbf{Early treatment of illnesses} such as malaria, HIV, and
malnutrition decreases susceptibility to pneumonia.

Community education on danger signs and early health-seeking behavior
significantly reduces mortality.

\section{Prognosis}\label{prognosis-2}

The outcome depends on the child's age, nutritional status, immune
function, causative agent, and timeliness of treatment.

Most children with uncomplicated pneumonia recover fully with
appropriate therapy. However, mortality remains high among neonates,
severely malnourished children, and those with HIV or delayed
presentation.

Recurrent or chronic infections may lead to lasting lung damage.
Strengthening preventive strategies and early intervention are therefore
crucial to reducing the burden of pneumonia in Ghana.

\section{Conclusion}\label{conclusion-12}

Pneumonia remains one of the most important paediatric health challenges
in Ghana and across Africa. Understanding its pathophysiology, timely
diagnosis, and appropriate management are fundamental skills for every
medical student.

While antibiotics and supportive care remain the mainstay of treatment,
prevention through vaccination, nutrition, and improved living
conditions offers the greatest hope for sustainable reduction in disease
burden. A holistic approach that integrates clinical excellence with
strong public health measures is the surest way to protect Ghana's
children from this preventable killer.

\chapter{Bronchopulmonary Dysplasia}\label{bronchopulmonary-dysplasia}

\section{Definition}\label{definition-8}

Bronchopulmonary dysplasia (BPD) is a chronic lung disease that affects
newborns, especially those born prematurely and requiring oxygen
therapy. It damages the lungs and airways, causing tissue destruction in
the lung's tiny air sacs. While most infants recover from BPD, some may
have long-term breathing difficulties.

\section{Incidence}\label{incidence}

Globally, the incidence in extremely preterm infants(\textless{} 28
weeks) ranges between 10-89\%, while 40\% of extremely low birth weight
infants (\textless1000g) will develop BPD. At the Komfo Anokye Teaching
Hospital, 44 out of 171 preterm babies admitted from January to April
2024 had Respiratory Distress Syndrome.

\section{Aetiology}\label{aetiology-1}

The causes of BPD vary and can be divided into:

\textbf{Pre-natal} - These include lack of maternal steroids, maternal
smoking, Pregnancy-induced hypertension, preeclampsia, chorioamnionitis,
hypoxia, congenital anomaly causing pulmonary hypoplasia, and genetic
susceptibility.

\textbf{Post-natal} - These include prematurity, immature lungs, apnea,
sepsis, need for mechanical ventilation and a \href{cvs-pda.qmd}{Patent
Ductus Arteriosus}

\section{Pathogenesis}\label{pathogenesis}

BPD is a multifactorial process and is linked to immature lung tissue,
prenatal factors and postnatal factors. Injury from mechanical
ventilation and reactive oxygen species to the premature lungs in the
presence of antenatal factors predisposing the lung to BPD forms the
basis of the pathogenesis in preterm infants. This leads to an
inflammatory response with an increase in pro-inflammatory cytokines
like IL-6 IL-8, and TNF alpha, along with growth factors (Transforming
growth factors,), angiogenic factors (vascular endothelial growth
factors, angiopoietin 2), which result in aberrant tissue repair and
arrest in lung development. Dysregulated vascular and arrested alveolar
development form the basis of the pathology seen in the new BPD.
Histologically, BPD occurs when lung development arrests in the late
canaliculi to the saccular stage of lung development. The pathology
characteristically demonstrates decreased septation and alveoli
hypoplasia resulting in simplified large alveoli and reduced
availability for gaseous exchange.

\section{Signs and symptoms}\label{signs-and-symptoms}

Initial findings in BPD are consistent with Respiratory Distress
Syndrome. These include respiratory distress, tachypnea, chest
retractions, tachycardia, and paradoxical breathing. Others include
intermittent expiratory wheezing, crackles and frequent desaturations.
There might be significant weight loss during the first 10 days of life.

\section{Investigations}\label{investigations-5}

A Chest radiograph is often the first investigative modality employed
The lung field may show a sponginess and decreased lung volumes. Others
include areas of hyperventilation, atelectasis, pulmonary oedema, and
pulmonary interstitial emphysema. A high-resolution CT Scan demonstrates
abnormalities not readily seen with routine chest radiography. Infants
with moderate or severe BPD must be screened for pulmonary hypertension
at 36 post-menstrual age using an echocardiogram. In the intensive care
unit, arterial blood gases may reveal the extent of hypoxia, hypercarbia
or acidosis.

\section{Treatment}\label{treatment-1}

Treatment is generally divided into two phases:

\textbf{Acute phase} - As previously mentioned, most cases of BPD
present as Respiratory Distress Syndrome. Hence its management starts
from this stage. This requires surfactant replacement with oxygen
supplementation, Continuous Positive Airway Pressure, and mechanical
ventilation when necessary. ~ Antibiotics are initiated if chronic
chorioamnionitis or an infective process is suspected. Others may insert
an indwelling arterial line for treatment administration and parenteral
nutrition

\textbf{Long-term} - Attention should be paid to the nutrition of the
infant. When necessary, a fluid restriction may be required. Also,
clinicians will need to minimise ventilator-associated and
oxygen-associated lung injury. Some pharmacological interventions may
include steroids, diuretics and bronchodilators.

\section{Complications}\label{complications-4}

Recognised complications include decreased pulmonary function and
defence, chronic reflux and microaspiration with a risk of aspiration
pneumonia and chronic inflammation. Others develop asthma-like symptoms,
exercise intolerance, pulmonary artery hypertension, systemic
hypertension, poor neurodevelopmental outcomes, left ventricular
hypertrophy and dysfunction.

\section{Prognosis}\label{prognosis-3}

Most babies with BPD recover completely but mortality ranges between 1\%
to 20\% during the first year of life.

\section{Differential diagnosis}\label{differential-diagnosis-3}

These include pulmonary atelectasis, pneumonia, pulmonary hypertension,
tracheomalacia and pulmonary interstitial emphysema.

\part{{Cardiology}}

\chapter{Basics}\label{basics-1}

\section{Anatomy}\label{anatomy}

The heart is located in the mediastinum of the chest, bounded anteriorly
by the sternum, posteriorly by the spine and laterally by the lungs.
Externally, the right ventricle is anterior. Most of the left ventricle,
left atrium and right atrium are posterior. Internally, the right and
left atria are separated by the tricuspid and mitral valves from the
right and left ventricles respectively. The arterial supply of the heart
is through the coronary arteries while venous drainage is through the
coronary sinus. The aorta and pulmonary arteries arise from the left and
right ventricles. The heart has three layers:

\begin{enumerate}
\def\labelenumi{\arabic{enumi}.}
\tightlist
\item
  Endocardium: Inner epithelial layer of the heart
\item
  Myocardium: Muscular part of the heart
\item
  Pericardium: Outer layers of the heart. Divided into the visceral and
  parietal pericardium.
\end{enumerate}

Venous blood enters the right atrium through the inferior and superior
vena cavae. It empties in atrial systole into the right ventricle
through the tricuspid valve. It then moves on through the pulmonary
valve in ventricular systole, to the pulmonary artery and then the
lungs. Blood returning from the lungs enters the left atrium through the
four pulmonary veins. In atrial systole, it moves into the left
ventricle through the mitral valve. Finally, it empties into the aorta
through the aortic valve.

\section{Conduction system}\label{conduction-system}

The heart has an inherent electrical system that automatically paces and
conducts depolarization throughout it. The parts are:

\begin{enumerate}
\def\labelenumi{\arabic{enumi}.}
\tightlist
\item
  \emph{Sinoatrial (SA) node}: This is the pacemaker of the heart and
  depolarizes the two atria.
\item
  \emph{Atrioventricular (AV) node}: Receives impulses from the SA node,
  and delays a bit before propagating it further.
\item
  \emph{His-purkinje fibre} system: Responsible for the spread of
  electrical impulses to the ventricles
\end{enumerate}

\section{\texorpdfstring{\textbf{Heart as a
pump}}{Heart as a pump}}\label{heart-as-a-pump}

There is a difference in the pumping action of the heart in utero and
after birth.

\begin{enumerate}
\def\labelenumi{\arabic{enumi}.}
\tightlist
\item
  Fetal

  \begin{itemize}
  \tightlist
  \item
    Most work is done by the right ventricle
  \item
    The right Ventricle is therefore relatively hypertrophic
  \item
    Only 15\% of the cardiac output is pumped into the lungs
  \end{itemize}
\item
  After birth

  \begin{itemize}
  \tightlist
  \item
    Gradual transition to Left ventricle dominance
  \item
    Gradual fall in pulmonary pressure (over 6 weeks)
  \item
    The left ventricle does most of the work and becomes more
    hypertrophic than the right
  \end{itemize}
\end{enumerate}

\section{\texorpdfstring{\textbf{Systolic and diastolic
functions}}{Systolic and diastolic functions}}\label{systolic-and-diastolic-functions}

\emph{Systole}: This is the contractile phase of the heart. It starts
after the atria is filled with blood. The atria then contract, emptying
its content into the ventricles. At this stage, the ventricle also
undergoes systole, which further empties the blood into the aorta and
pulmonary arteries.

\emph{Diastole}: This is the relaxation phase where the heart relaxes
and lets in blood. It also starts with the atrium and then the
ventricles.

\emph{Compliance}: This describes how easily the heart chamber relaxes
in response to the inflow of blood.

\section{\texorpdfstring{\textbf{Intracardiac
Pressures}}{Intracardiac Pressures}}\label{intracardiac-pressures}

The pressures in the heart vary for different ages and individuals.
Generally, the pressures in the atria are lower than the ventricles.
Also, the peak systolic pressure in the left ventricle is higher than in
the right. The diastolic pressure in the left ventricle is however lower
than the right ventricle. Also, both systolic and diastolic pressures in
the aorta are higher than that in the pulmonary artery.

Systolic pressure in general is generated by the ventricles. In
conditions such as coarctation of the aorta, aortic stenosis and
pulmonary hypertension, the ventricles increase their workload to
generate enough pressure. The diastolic pressure on the other hand is
maintained by the closure of the aortic and pulmonary valves. Thus, an
incompetent pulmonary or aortic valve leads to a decrease in diastolic
pressure in the two vessels respectively.

\section{Fetal circulation}\label{fetal-circulation}

The heart begins developing in the fetus as the cardiogenic area and
primitive blood vessels as early as 18 days old. By 20 days the paired
endocardial tubes are formed. By day 35 the heart would have completed
its looping with blood already flowing through.

Three main shunts exist in the fetal circulation. These are adaptations
to using the placenta rather than the lungs for oxygenation. These
shunts are the ductus venosus, ductus arteriosus and patent foramen. At
birth, these shunts regress. The ductus venosus regresses quickly to
form the ligamentum arteriosus. The foramen ovale closes at birth
physiologically but may take years to close anatomically. The patent
ductus arteriosus closes soon after birth but can physiologically stay
open for 72 hours. Beyond this period it is considered to be pathologic
if it continues to stay open. Patent ductus arteriosus are quite common
in preterm newborns.

\section{Pathologic classification}\label{pathologic-classification}

The pathology of pediatric cardiac disorders varies. Broadly, they can
be divided into these:

\begin{enumerate}
\def\labelenumi{\arabic{enumi}.}
\item
  \textbf{\emph{Congenital heart disorders}}: These are cardiac
  conditions that a child is born with. Thus they are present at birth.
  They form about 85\% of all pediatric heart diseases seen in the Komfo
  Anokye Teaching Hospital (KATH). It is further divided into:

  \begin{itemize}
  \item
    \emph{Acyanotic}: These are congenital heart diseases that are
    traditionally not known to be associated with cyanosis. Examples
    include ventricular septal defect (VSD), Atrial Septal Defect (ASD)
    and Patent Ductus Arteriosus (PDA)
  \item
    \emph{Cyanotic}: These on the other hand are associated with
    cyanosis and include Tetralogy of Fallot (ToF), Truncus Arteriosus
    and Tricuspid Atresia.
  \end{itemize}
\item
  \textbf{\emph{Acquired heart disorders}}: These are heart conditions
  that are not present at birth but develop afterwards. They include
  Infective Endocarditis (IE), Rheumatic Heart Disease (RHD) and
  Endomyocardial Fibrosis (EMF)
\item
  \textbf{\emph{Rhythm disorders}}: This set of disorders can present as
  either congenital or acquired. They affect the electrical system of
  the heart leading to an increase in heart rate (tachyarrhythmia),
  decrease in heart rate (bradyarrhythmia) or even normal heart rate.
\item
  \textbf{\emph{Secondary cardiac disorders}}: Some pathologies tend to
  affect the heart as a complication. Such conditions include some
  glycogen storage disorders resulting in cardiomyopathy and Rheumatoid
  arthritis resulting in pericardial effusion.
\end{enumerate}

\chapter{Evaluating Heart Diseases}\label{evaluating-heart-diseases}

To fully evaluate a child with a suspected cardiac condition, one needs
to go through the regular steps applicable in medicine. These are
outlined below:

\section{History}\label{history-2}

The history is traditionally divided into:

\subsection{Prenatal}\label{prenatal}

Prenatally, the history should delve into but not be limited to the
following:

\begin{enumerate}
\def\labelenumi{\arabic{enumi}.}
\tightlist
\item
  \textbf{\emph{Infections}}: Some infections are the well-known
  TORCHES. They include Toxoplasmosis, HIV, syphilis, parvovirus B19
  (fifth disease), varicella (chickenpox) and (Zika), Rubella,
  Cytomegalovirus, and Herpes simplex virus. Rubella, when acquired in
  the first trimester of pregnancy, is very well known to be associated
  with \href{cvs-pda.qmd}{PDA}s.
\item
  \textbf{\emph{Medications:}} The use of some medications, including
  herbs, predisposes to heart disease in newborns. Anticonvulsant such
  as phenytoin, carbamazepine, and valproic acid are highly
  teratogenic.(Kalisch-Smith, Ved, and Sparrow 2019)
\item
  \textbf{\emph{Recreational drugs}}: Excessive smoking, cocaine, and
  alcohol use in early pregnancy are all associated with teratogenic
  effects on the heart.
\item
  \textbf{\emph{Maternal illnesses:}} Maternal medical conditions during
  pregnancy may be associated with heart diseases in their fetuses.
  Diabetes mellitus is particularly well known, predisposing to
  hypertrophic cardiomyopathy, d-Transposition of the Great Arteries,
  etc. Autoimmune conditions such as Systemic Lupus Erythematosus may
  also predispose to rhythm disturbances in the fetus and child, even
  when the mother is not symptomatic.
\item
  \textbf{\emph{Family history of CHD}}: The recurrence of CHD in
  first-degree relatives varies but is almost always higher than in the
  rest of the population. For instance, having a first-degree relation
  with a cornoventricular defect is associated with a recurrence risk
  ratio of 24.3 (95\% CI,12.2 to 48.7), 7.1 (95\% CI, 4.5 to 11.1) for
  isolated \href{cvs-asd.qmd}{Atrial Septal Defect}, and 3.4 (95\% CI,
  2.2 to 5.3) for isolated \href{cvs-vsd.qmd}{Ventricular Septal
  Defect}.(Øyen et al. 2009)
\end{enumerate}

\subsection{Perinatal}\label{perinatal}

Perinatal history associated with heart disease may include the
following:

\begin{enumerate}
\def\labelenumi{\arabic{enumi}.}
\tightlist
\item
  \textbf{\emph{Birth weight}}: A high birth weight, often associated
  with a child of a diabetic mother, is also associated with an
  increased incidence of CHDs. Conversely, a low birth weight may also
  be associated with fetal alcohol syndrome or congenital rubella
  syndrome, both of which are associated with CHDs.
\item
  \textbf{\emph{Newborn resuscitation}}: Some critical CHDs can be
  similar to neonatal asphyxia in a newborn, thus requiring
  resuscitation.
\end{enumerate}

\subsection{After birth}\label{after-birth}

Ascertaining history after birth is the most extensive. Many of these
are directed to the features of heart failure. These include:

\begin{enumerate}
\def\labelenumi{\arabic{enumi}.}
\tightlist
\item
  \textbf{\emph{Growth failure}}: Poor weight gain is a very prominent
  feature of CHDs in children. Many clinically significant CHDs result
  in poor feeding, chronic metabolic demand on the patient and poor
  oxygenation in cyanotic CHDs. All these result in increased caloric
  demand, resulting in poor growth.
\item
  \textbf{\emph{Cyanotic spells}}: Some cyanotic CHDs are associated
  with recurrent periods where the child has increasing cyanosis,
  sometimes associated with weakness, fast breathing and even
  unconsciousness. The presence of these spells may be a pointer to a
  CHD.
\item
  \textbf{\emph{Squatting and exercise intolerance}}: Exercise
  intolerance is a common presentation of heart disease in children.
  However, for some cyanotic congenital heart diseases, most notably
  \href{cvs-tof.qmd}{Tetralogy of Fallot}, the added feature is frequent
  squatting when the child becomes fatigued.
\item
  \textbf{\emph{Delayed milestones}}: Growth failure, easy
  fatiguability, and other genetic syndromes may delay motor milestones
  in children.
\item
  \textbf{\emph{Others}}: \emph{Fast} and sometimes \emph{difficult
  breathing} are also common presentations of CHDs. Some children
  develop \emph{oedema}.(Figure~\ref{fig-cvs-pedal-edema}) This is
  predominantly seen in younger children's faces and older children's
  feet. Frequent \emph{lower respiratory infection} is also seen in
  children with heart diseases, especially those associated with
  \href{cvs-heart-failure.qmd}{heart failure}.
\item
  \textbf{\emph{Uncommon symptoms}}: Uncommon presentation of heart
  disease in children include:

  \begin{itemize}
  \tightlist
  \item
    \emph{Chest pain} is a rather feared symptom in adults but usually
    portends another diagnosis rather than heart disease in children.
  \item
    \emph{Syncope} can be observed in children with an arrhythmia or
    left or right ventricular obstruction. However, this is still not a
    common presentation in pediatric heart diseases.
  \item
    Older children report \emph{palpitations}.
  \item
    \emph{Joint swelling} does occur in Rheumatic Heart Disease, but
    again, it is not a common presentation in children with heart
    pathology.
  \end{itemize}
\end{enumerate}

\begin{figure}

\centering{

\pandocbounded{\includegraphics[keepaspectratio]{images/cvs-pedal-edema.jpg}}

}

\caption{\label{fig-cvs-pedal-edema}Pedal oedema in a child with heart
failure}

\end{figure}%

\section{Clinical examination}\label{clinical-examination}

\subsection{General}\label{general}

Clinical examination for a child with a suspected heart disease should
always start as general. One should first look out for life-threatening
signs and intervene quickly. Subsequent steps could include:

\begin{enumerate}
\def\labelenumi{\arabic{enumi}.}
\item
  \textbf{\emph{Nutritional status}} is vital as many children with
  chronic heart conditions with significant heart failure present with
  malnutrition. The growth pattern of the patient should always be
  evaluated.
\item
  \textbf{\emph{Dysmorphism}} is very critical in pediatric heart
  diseases. As much as 23\% of all children with CHD will have a
  chromosomal abnormality.(Wang et al. 2023) There are many genetic
  syndromes with well-documented recognisable heart defects. Below are
  just a few adapted from Ko (2015):
\end{enumerate}

\begin{longtable}[]{@{}
  >{\raggedright\arraybackslash}p{(\linewidth - 4\tabcolsep) * \real{0.3333}}
  >{\raggedright\arraybackslash}p{(\linewidth - 4\tabcolsep) * \real{0.3333}}
  >{\raggedright\arraybackslash}p{(\linewidth - 4\tabcolsep) * \real{0.3333}}@{}}
\caption{Common genetic syndromes associated with congenital heart
diseases}\label{tbl-genetic-synds}\tabularnewline
\toprule\noalign{}
\begin{minipage}[b]{\linewidth}\raggedright
Genetic syndrome
\end{minipage} & \begin{minipage}[b]{\linewidth}\raggedright
\% with CHD
\end{minipage} & \begin{minipage}[b]{\linewidth}\raggedright
Cardiac anomalies
\end{minipage} \\
\midrule\noalign{}
\endfirsthead
\toprule\noalign{}
\begin{minipage}[b]{\linewidth}\raggedright
Genetic syndrome
\end{minipage} & \begin{minipage}[b]{\linewidth}\raggedright
\% with CHD
\end{minipage} & \begin{minipage}[b]{\linewidth}\raggedright
Cardiac anomalies
\end{minipage} \\
\midrule\noalign{}
\endhead
\bottomrule\noalign{}
\endlastfoot
\href{synd-down.qmd}{Down Syndrome} & 40 to 50 & Atrial Septal Defect,
Ventricular Septal Defect, Atrioventricular Canal Defect, Patent Ductus
Arteriosus, Tetralogy of Fallot \\
\href{synd-turners.qmd}{Turner's syndrome} & 25 to 45 & Coarctation of
the Aorta. Bicuspid Aortic Valve, Aortic Stenosis, Hypoplastic left
heart syndrome \\
\href{synd-digeorge.qmd}{DiGeorge syndrome} & 70 to 75 & Aortic arch
anomalies, Truncus arteriosus, Tetralogy of Fallot \\
\href{synd-williams.qmd}{William's syndrome} & 75 to 80 & Supravalvar
Aortic Stenosis, Peripheral Pulmonary Stenosis \\
\href{synd-noonan.qmd}{Noonan syndrome} & 70 to 80 & Pulmonary Stenosis,
Hypertrophic Cardiomyopathy, Atrial Septal Defect \\
Kabuki syndrome & 31 to 55 & Coarctation of the Aorta, Atrial Septal
Defect, Aortic Stenosis, Mitral Stenosis, Hypoplastic left heart
syndrome \\
Alagille syndrome & 90 & Peripheral Pulmonary Stenosis, Pulmonary
Stenosis, Tetralogy of Fallot \\
\end{longtable}

\begin{enumerate}
\def\labelenumi{\arabic{enumi}.}
\setcounter{enumi}{2}
\tightlist
\item
  \textbf{\emph{Colour}}: The skin colour of a child with a CHD could
  hold signs of its presence. \emph{Cyanosis}, the blueish duskiness of
  the skin and mucous membranes can be seen in children with cyanotic
  congenital heart disease. This may not be easy on black skin and can
  only be observed in the mouth and tongue
  (Figure~\ref{fig-cvsCyanosisMouth}). Mild cyanosis is often not
  visible and may require pulse oximetry. \emph{Pallor} can be observed
  in patients with heart diseases, such as
  \href{cvs-inf-endo.qmd}{Infective Endocarditis}. \emph{Jaundice} can
  be observed in patients with \href{cvs-inf-endo.qmd}{Infective
  Endocarditis} or those with hepatic injury secondary to chronic
  \href{cvs-heart-failure.qmd}{heart failure}.
\end{enumerate}

\begin{figure}

\centering{

\pandocbounded{\includegraphics[keepaspectratio]{images/cvs-cyanosis-mouth.jpg}}

}

\caption{\label{fig-cvsCyanosisMouth}Cyanosis in the tongue of a child}

\end{figure}%

\begin{enumerate}
\def\labelenumi{\arabic{enumi}.}
\setcounter{enumi}{3}
\tightlist
\item
  \textbf{\emph{Clubbing}}: All four stages of digital clubbing are seen
  in children with cyanotic CHD or \href{cvs-inf-endo.qmd}{Infective
  Endocarditis}. (Figure~\ref{fig-cvsFingerClubbing}) Note that some
  cases of finger clubbing may be familial.
\end{enumerate}

\begin{figure}

\centering{

\pandocbounded{\includegraphics[keepaspectratio]{images/cvs_finger_clubbing.jpg}}

}

\caption{\label{fig-cvsFingerClubbing}Finger clubbing}

\end{figure}%

\begin{enumerate}
\def\labelenumi{\arabic{enumi}.}
\setcounter{enumi}{4}
\tightlist
\item
  \textbf{\emph{Respiratory signs}}: Respiratory signs commonly
  associated with heart diseases in children are \emph{tachypnoea},
  \emph{dyspnoea}, \emph{chest recessions,} and \emph{increased work of
  breathing}. These are especially true when there is associated
  \href{cvs-heart-failure.qmd}{heart failure}, which worsens with
  exercise or breastfeeding infants.
\item
  \textbf{\emph{Circulation}}: The circulation in a child with a
  suspected heart disease is critical. Reduced circulation can be
  assessed with the warmth of the extremities, capillary refill time,
  and blood pressure.
\item
  \textbf{\emph{Blood pressure}}: Low blood pressure is a late sign of
  circulatory failure and cardiogenic shock. Conversely, weak pulse may
  be associated with hypertensive heart disease as well. Wide pulse
  pressure, an abnormally wide difference between the systolic and
  diastolic blood pressures, may indicate a \href{cvs-pda.qmd}{patent
  ductus arteriosus}, aortic insufficiency or an aorticopulmonary
  window. Blood pressure should be checked in the upper and lower limbs
  as a higher BP in the upper limbs compared to the lower may indicate
  the presence of a \href{cvs-coa.qmd}{Coarctation of the Aorta}.
\item
  \textbf{\emph{Pulses}}: The radial pulse is the most routinely
  examined in cardiovascular examination. It should be checked for the
  rate, rhythm, volume and character. If they are difficult to examine,
  especially in young infants, the brachial pulsus can be used. Other
  pulses should be examined, including the brachial femoral and dorsalis
  pedis. Next, the synchronisation of the radio-femoral pulse should be
  determined for a delay. This happens in the coarctation of the aorta.
  Pulses that are challenging to palpate or inconsistent could be caused
  by large artery arteritis, such as Taksyasu's arteritis.
\end{enumerate}

\subsection{Precordial}\label{precordial}

\begin{enumerate}
\def\labelenumi{\arabic{enumi}.}
\tightlist
\item
  \textbf{\emph{Inspection}}: Inspection of the precordium yields a
  wealth of information in a child with a suspected heart disease. A
  midline bulge or left-sided bulge will often indicate a right or
  left-sided heart chamber dilatation. Visible precordial pulsation
  should be noted. Scars, especially from previous surgeries, are
  useful. Scarification, the usually small ``medicinal'' scars done on
  the chest as a means of treatment, should also be noted. A Harrison
  sulcus, depression of the lower part of the chest, is common in
  children with chronic heart failure and, thus, dyspnoea.
  Figure~\ref{fig-chest-bulge}
\end{enumerate}

\begin{figure}

\centering{

\pandocbounded{\includegraphics[keepaspectratio]{images/cvs-chest-bulge.jpg}}

}

\caption{\label{fig-chest-bulge}Chest bulge and Harrison's sulcus in
child}

\end{figure}%

\begin{enumerate}
\def\labelenumi{\arabic{enumi}.}
\setcounter{enumi}{1}
\item
  \textbf{\emph{Palpation}}: Palpation should be directed toward
  determining the presence of a thrill (a palpable murmur) heave at the
  apex or middle of the precordium. Also, a palpable heart sound,
  especially the second at the upper left sternal edge, may indicate
  pulmonary hypertension.
\item
  \textbf{\emph{Percussion}}: This is of very little relevance in
  examining the heart in children.
\item
  \textbf{\emph{Auscultation}}: Auscultation of the heart can yield a
  wealth of information. It should be done in a quiet environment, with
  the child as calm as possible. Auscultating can be performed with both
  bell and diaphragm. All four auscultatory areas need to be
  auscultated. Auscultating the back (between the scapulae) and over the
  carotids is always prudent. First, the regular two heart sounds should
  be determined. If muffled, they could indicate a pericardial effusion
  or sometimes obesity. Pulmonary hypertension and an
  \href{cvs-asd.qmd}{Atrial Septal Defect}, for instance could result in
  a loud or split-second heart sound. The presence of a third heart
  sound (S3) is not always pathologic in children, but an S4 is. The
  presence of a murmur needs to be determined. If present, it should be
  determined if it is systolic or diastolic and the point of maximal
  intensity. It needs to be graded (Table~\ref{tbl-murmur-grades}), and
  the presence of radiation must be ascertained.
  Table~\ref{tbl-murmur-location} indicates types of murmurs, their
  location and likely heart diseases. Diastolic murmurs are difficult to
  appreciate for the average medical student but may be present in
  aortic regurgitation, pulmonary regurgitation, and at the cardiac apex
  in cases of \href{cvs-heart-failure.qmd}{heart failure} secondary to a
  large left-to-right shunt. Notably, approximately 80\% of murmurs in
  children can be categorised as ``innocent murmurs'' as they are not
  associated with cardiac pathology.

  Other sounds need to be evaluated as well. These may include a
  pericardial rub, which occurs in pericarditis, an ejection click heart
  in early systole and cases of aortic or pulmonary stenosis.
\end{enumerate}

\begin{verbatim}
Warning: package 'tibble' was built under R version 4.5.1
\end{verbatim}

\begin{verbatim}
Warning: package 'purrr' was built under R version 4.5.1
\end{verbatim}

\begin{table}

\caption{\label{tbl-murmur-grades}Grades of murmurs}

\centering{

\fontsize{12.0pt}{14.4pt}\selectfont
\begin{tabular*}{\linewidth}{@{\extracolsep{\fill}}ll}
\toprule
Grade & Description \\ 
\midrule\addlinespace[2.5pt]
Grade I & Barely perceptible \\ 
Grade II & Soft, but easily audible \\ 
Grade III & Moderately loud but has not thrill \\ 
Grade IV & Loud and associated with a thrill \\ 
Grade V & Audible with stethoscope partially of the chest \\ 
Grade VI & Audible with stethoscope off the chest \\ 
\bottomrule
\end{tabular*}

}

\end{table}%

\begin{table}

\caption{\label{tbl-murmur-location}Heart diseases and their murmur
characteristics}

\centering{

\fontsize{12.0pt}{14.4pt}\selectfont
\begin{tabular*}{\linewidth}{@{\extracolsep{\fill}}lll}
\toprule
Murmur & Location & Condition \\ 
\midrule\addlinespace[2.5pt]
Pansystolic & LLSB & VSD, Tricuspid regurgitation \\ 
Pansystolic & Apex & Rheumatic Heart Disease, Mitral valve prolapse \\ 
Ejection systolic & URSB & Aortic stenosis \\ 
Ejection systolic & ULSB & ASD, Pulmonary stenosis, Tetralogy of Fallot, 
        Coarctation of the aorta \\ 
Continuous & 2nd left ICS & Patent Dutus Arteriosus \\ 
\bottomrule
\end{tabular*}

}

\end{table}%

\section{Investigation}\label{investigation}

Various investigations used in the diagnosis and management of heart
diseases are:

\begin{enumerate}
\def\labelenumi{\arabic{enumi}.}
\item
  \textbf{\emph{Pulse Oximetry}}: Pulse oximetry helps determine heart
  rate and oxygen saturation. An oxygen saturation lower than expected
  (\textless95\% outside the early neonatal period) is considered
  abnormal and a strong indication of cyanotic heart disease in the
  presence of a non-pathologic lung.
\item
  \textbf{\emph{Electrocardiogram}}: The electrocardiogram is a common
  modality for the bedside investigation of heart diseases in all age
  groups. It indicates the heart rate, rhythm, chamber dilatation, wall
  thickness, laterality of the chambers, electrolyte abnormalities and
  even the presence of a head injury. It is often not conclusive in many
  heart conditions but serves as a good auxiliary test in children,
  especially post-surgery. There are various types: The routine ECG
  takes just a few minutes to perform on a resting patient, usually
  lying supine. On the other hand, the stress ECG is traditionally done
  with the heart under stress, as may happen during an aerobic exercise.
  The Holter ECG, on the other hand, is attached to the patient and
  continuously monitors the heart for 24 to 48 hours. This usually gives
  a better reflection of the heart's electrical activity over a
  prolonged period instead of just a brief period.
\item
  \textbf{\emph{X-ray}}: A chest x-ray is very useful in diagnosing and
  managing heart diseases in children. Fortunately, it is readily
  available in many parts of Ghana. Both posteroanterior and lateral
  chest X-rays can be useful in assessing the individual chamber and
  overall heart sizes. Cardiomegaly, assessed with a cardiothoracic
  ratio \textgreater{} 60\% in children, is seen in many cases of heart
  disease. Chest X-rays also indicate ling pathologies, often showing as
  opacification or silhouetting. Increased lung markings, for instance,
  can be found in patients with VSD and ASD, while decreased lung
  markings are often seen in the \href{cvs-tof.qmd}{Tetralogy of Fallot}
  and pulmonary stenosis. The shape of the heart is often an indicator
  of the underlying cardiac pathology. A Boot-shaped heart may indicate
  a \href{cvs-tof.qmd}{Tetraloy of Fallot}
  (Figure~\ref{fig-cvs-x-ray-boot-shaped}), while a globular-shaped
  heart points to dilated cardiomyopathy or pericardial effusion.
  Figure~\ref{fig-cvs-globular-heart}
\end{enumerate}

\begin{figure}

\centering{

\pandocbounded{\includegraphics[keepaspectratio]{images/cvs-globular-heart.jpg}}

}

\caption{\label{fig-cxrCardiomegaly}Chest x-ray showing cardiomegaly and
lung shadowing}

\end{figure}%

\begin{figure}

\centering{

\pandocbounded{\includegraphics[keepaspectratio]{images/cvs-GlobularHeart.jpg}}

}

\caption{\label{fig-cvs-globular-heart}Chest x-ray showing a globular
heart}

\end{figure}%

\begin{figure}

\centering{

\pandocbounded{\includegraphics[keepaspectratio]{images/cvs-x-ray-boot-shaped.jpg}}

}

\caption{\label{fig-cvs-x-ray-boot-shaped}Boot-shaped heart of Chest
X-X-ray}

\end{figure}%

\begin{enumerate}
\def\labelenumi{\arabic{enumi}.}
\setcounter{enumi}{3}
\item
  \textbf{\emph{Echocardiogram}}: This is a rather old but new
  investigatory modality. It is old because it has been around since the
  1960s and new because it is relatively new in Ghana. It is, however, a
  beneficial modality of investigation. An echocardiogram is essentially
  an ultrasound of the heart and great vessels. Its most significant
  advantage is its ability to visualise the heart in real-time, assess
  systolic and diastolic functions, measure chamber sizes and wall
  thickness, detect defects such as ventricular septal defect, determine
  valvular abnormalities such as stenosis and regurgitation, and even
  assess all these under stressful situation (stress echocardiogram).
  The video below illustrates the various echocardiographic views used
  in children. Unfortunately, since it is very user-dependent, it is not
  commonly available in Ghana, with pediatric echocardiography only
  currently available in Accra, Kumasi, Cape Coast and Tamale.

  \url{https://www.youtube.com/watch?v=WpJuARIoR6s&t=39s}
\item
  \textbf{\emph{Computerised tomography (CT) scan and Magnetic resonance
  Imaging (MRI)}}: These are more advanced modalities available for use,
  especially when an echocardiogram is inconclusive or further study of
  the patient is necessary. A CT scan generates the image using a series
  of X-rays taken at different angles without the complication of
  significant X-ray radiation exposure. Both modalities can employ
  contrast to delineate vessels.
\item
  \textbf{\emph{Others}}: Other specialised investigatory modalities are
  used as required, including cardiac catheterisation in a specialised
  catheter laboratory.
\end{enumerate}

\chapter{Heart Failure}\label{heart-failure}

\section{\texorpdfstring{\textbf{Definition}}{Definition}}\label{definition-9}

The inability of the heat to provide enough output to the body.

\section{\texorpdfstring{\textbf{Causes}}{Causes}}\label{causes-4}

Varies, especially in children. They can occur in both structurally
normal hearts and in congenital cardiac malformations. There are three
main groups of causes:

\begin{enumerate}
\def\labelenumi{\arabic{enumi}.}
\item
  \textbf{Ventricular dysfunction} results from either systolic or
  diastolic dysfunction of the ventricles. Systolic dysfunction is more
  commonly encountered compared to diastolic ones. Examples are:

  \begin{itemize}
  \tightlist
  \item
    Cardiomyopathy (dilated, restrictive and hypertrophic)
  \item
    Myocarditis
  \item
    Arrhythmias
  \item
    Coronary artery anomalies
  \item
    Post-op cardiac dysfunction
  \end{itemize}
\item
  \textbf{Volume overload} occurs in conditions associated with
  increased volume (preload) in the heart, especially the ventricles.
  The ventricle must, therefore, eject an increased blood volume,
  leading to tachycardia. It may or may not be associated with
  ventricular dysfunction. Examples include:

  \begin{itemize}
  \tightlist
  \item
    Ventricular septal defect (left to right shunt)
  \item
    Atrial septal defect
  \item
    Patent ductus arteriosus
  \item
    Aortic regurgitation (left ventricle)
  \item
    Mitral regurgitation (Left atrium)
  \end{itemize}
\item
  \textbf{Pressure overload} is when heart failure is caused by an
  increased pressure (afterload) in the heart. Ventricles must,
  therefore, contract against higher pressures. It may or may not be
  associated with ventricular dysfunction. These include:

  \begin{itemize}
  \tightlist
  \item
    Hypertension
  \item
    Aortic valve stenosis
  \item
    Pulmonary stenosis
  \item
    Coarctation of the aorta
  \end{itemize}
\end{enumerate}

In all these, the result is decreased cardiac output and pulmonary
oedema.

\section{Classification}\label{classification-1}

The symptoms of heart failure vary significantly, with infants and young
children having different presentations compared to older children. The
classification of heart failure there is not uniform. The most
well-known classification is the NYHA, which is appropriate for older
children. It is shown below:

\begin{longtable}[]{@{}
  >{\raggedright\arraybackslash}p{(\linewidth - 2\tabcolsep) * \real{0.5000}}
  >{\raggedright\arraybackslash}p{(\linewidth - 2\tabcolsep) * \real{0.5000}}@{}}
\caption{NYHA
Classification}\label{tbl-nyha-hf-classification}\tabularnewline
\toprule\noalign{}
\begin{minipage}[b]{\linewidth}\raggedright
Class
\end{minipage} & \begin{minipage}[b]{\linewidth}\raggedright
Patient Symptoms
\end{minipage} \\
\midrule\noalign{}
\endfirsthead
\toprule\noalign{}
\begin{minipage}[b]{\linewidth}\raggedright
Class
\end{minipage} & \begin{minipage}[b]{\linewidth}\raggedright
Patient Symptoms
\end{minipage} \\
\midrule\noalign{}
\endhead
\bottomrule\noalign{}
\endlastfoot
Class I (Mild) & No limitation on physical activity. Ordinary physical
activity does not cause undue fatigue, palpitation or dyspnoea \\
Class II (Mild) & Slight limitation of physical activity. Comfortable at
rest but ordinary physical activity results in fatigue, palpitation or
dyspnoea \\
Class III (Moderate) & Marked limitation of physical activity.
Comfortable at rest but less than ordinary physical activity causes
fatigue, palpitation or dyspnoea \\
Class IV (Severe) & Unable to carry out any physical activity without
discomfort. Symptoms of cardiac insufficiency at rest. If any physical
activity is undertaken discomfort is increased \\
\end{longtable}

On the other hand, the Ross classification shown below is more suited
for infants and young children.

\begin{longtable}[]{@{}
  >{\raggedright\arraybackslash}p{(\linewidth - 2\tabcolsep) * \real{0.5000}}
  >{\raggedright\arraybackslash}p{(\linewidth - 2\tabcolsep) * \real{0.5000}}@{}}
\caption{Modified Ross
Classification}\label{tbl-ross-classification}\tabularnewline
\toprule\noalign{}
\begin{minipage}[b]{\linewidth}\raggedright
Class
\end{minipage} & \begin{minipage}[b]{\linewidth}\raggedright
Symptoms
\end{minipage} \\
\midrule\noalign{}
\endfirsthead
\toprule\noalign{}
\begin{minipage}[b]{\linewidth}\raggedright
Class
\end{minipage} & \begin{minipage}[b]{\linewidth}\raggedright
Symptoms
\end{minipage} \\
\midrule\noalign{}
\endhead
\bottomrule\noalign{}
\endlastfoot
Class I & Asymptomatic \\
Class II & Mild tachypnoea or diaphoresis in feeding in infants Dyspnoea
on exertion in older children \\
Class III & Marked tachypnoea or sweating with feeding in infants Marked
dyspnoea on exertion Prolonged feeding times with growth failure  \\
Class IV & Symptoms such as tachycardia, retraction, grunting, or
diaphoresis at rest \\
\end{longtable}

\section{Pathophysiology}\label{pathophysiology-6}

A schematic drawing of the various processes involved is shown below:

\begin{figure}

\centering{

\includegraphics[width=8.26in,height=7.56in]{cvs-heart-failure_files/figure-latex/mermaid-figure-1.png}

}

\caption{\label{fig-HeartFailure}Pathophysiology of heart failure}

\end{figure}%

\section{Signs and symptoms}\label{signs-and-symptoms-1}

The symptoms of heart failure are variable and age-dependent. For
infants, the symptoms include poor feeding, sweating with breastfeeding,
prolonged feeding time, tachypnoea, poor weight gain and dyspnoea. For
young children symptoms include recurrent respiratory tract infection,
recurrent wheezing, fatigue, exercise intolerance, facial and recurrent
cough. Older children have symptoms that more resemble those of adults.
These include tachypnoea, tachycardia, recurrent wheezing, pedal
swelling, palpitations, and vomiting.

Signs of heart failure also vary with age. These include for infants,
failure to thrive, tachycardia, tachypnoea, hepatomegaly, displaced apex
(cardiomegaly), S3 gallop, oedema (pedal in older children and facial or
abdominal distension in older children).

\section{Investigation}\label{investigation-1}

The investigations required are generally towards the likely underlying
pathology. Some of them would include:

\textbf{Chest x-ray}: This may show cardiomegaly, increased pulmonary
lung markings, pulmonary oedema, pleural effusion and heart shape.

\textbf{Electrocardiogram}: This helps to identify chamber enlargement
and dysrhythmias that may be the cause or consequent to the heart
failure

\textbf{Echocardiogram}: This identifies and quantifies the function of
the ventricle as well as the chamber sizes

\textbf{Blood test}: The complete blood count helps to identify anaemia
or polycethemia. The serum urea and creatinine identify possible renal
dysfunction. Other tests include BNP (Brain Naturetic Peptide) and
Troponin both of which are elevated in heart failure.

\textbf{Other investigatory modalities: These} include Magnetic
Resonance Imaging, Cardiac catheterization,

\section{Treatment of Heart Failure}\label{treatment-of-heart-failure}

This is done with some goals:

\begin{enumerate}
\def\labelenumi{\arabic{enumi}.}
\tightlist
\item
  Improve the quality of life
\item
  Arrest and possibly reverse the heart failure
\item
  Sustain till other definitive therapeutic interventions are employed,
  including surgery.
\end{enumerate}

The treatment for heart failure is dependent on the pathophysiology,
clinical features and stage of the disease.

\subsection{Non-pharmacological
treatment}\label{non-pharmacological-treatment}

This includes fluid restriction in case of congestion and fluid
overload, intubation and mechanical ventilation to help support
breathing and reduce the workload on the heart and patient. Others
include cardiac Resynchronization Therapy, Ventricular Assisted Devices
and Extracorporeal Membrane Oxygenation. Heart transplantation is the
last option in some cases of heart failure.

\subsection{Pharmacological treatment}\label{pharmacological-treatment}

Treatment depends on the clinical presentation and cause of the heart
failure. There are 2 main groups to be considered:

\subsection{Acute decompensated heart
failure}\label{acute-decompensated-heart-failure}

\begin{longtable}[]{@{}
  >{\raggedright\arraybackslash}p{(\linewidth - 2\tabcolsep) * \real{0.5000}}
  >{\raggedright\arraybackslash}p{(\linewidth - 2\tabcolsep) * \real{0.5000}}@{}}
\caption{Drugs used in acute decompensated heart
failure}\label{tbl-acute-hf-tmt-drug-class}\tabularnewline
\toprule\noalign{}
\begin{minipage}[b]{\linewidth}\raggedright
Drug
\end{minipage} & \begin{minipage}[b]{\linewidth}\raggedright
Action
\end{minipage} \\
\midrule\noalign{}
\endfirsthead
\toprule\noalign{}
\begin{minipage}[b]{\linewidth}\raggedright
Drug
\end{minipage} & \begin{minipage}[b]{\linewidth}\raggedright
Action
\end{minipage} \\
\midrule\noalign{}
\endhead
\bottomrule\noalign{}
\endlastfoot
Diuretics & Notable here is furosemide. The aim is to help decongest the
lungs, reduce preload by vasodilatation and improve heart failure
symptoms. \\
Inotropes & These include adrenaline, noradrenaline, dopamine and
dobutamine. They help improve the contractility of the heart, increase
heart rate, and increase peripheral vascular resistance, thus
maintaining the blood pressure and cardiac output. They are usually
Intravenous medications. \\
\end{longtable}

\subsection{Chronic heart failure}\label{chronic-heart-failure}

These are usually oral medications given to treat heart failure on an
outpatient basis

\begin{longtable}[]{@{}
  >{\raggedright\arraybackslash}p{(\linewidth - 2\tabcolsep) * \real{0.5000}}
  >{\raggedright\arraybackslash}p{(\linewidth - 2\tabcolsep) * \real{0.5000}}@{}}
\caption{Drugs for chronic heart failure
treatment}\label{tbl-chronic-hf-tmt-drug-class}\tabularnewline
\toprule\noalign{}
\begin{minipage}[b]{\linewidth}\raggedright
Group
\end{minipage} & \begin{minipage}[b]{\linewidth}\raggedright
Action
\end{minipage} \\
\midrule\noalign{}
\endfirsthead
\toprule\noalign{}
\begin{minipage}[b]{\linewidth}\raggedright
Group
\end{minipage} & \begin{minipage}[b]{\linewidth}\raggedright
Action
\end{minipage} \\
\midrule\noalign{}
\endhead
\bottomrule\noalign{}
\endlastfoot
Diuretics & These are given to decongest the lungs, liver and other
edematous organs. The most commonly used is furosemide. \\
Aldosterone antagonists & These counteract the aldosterone effect of
water and sodium retention. They decrease afterload while helping in
reversing cardiac remodelling. \\
ACE-I/ARB & Angiotensin-converting enzyme inhibitors and Angiotensin II
receptor blockers counteract the renin effects of increasing afterload.
They thus decrease the afterload and help reverse and prevent cardiac
remodelling \\
Digoxin & This is probably the oldest anti-heart failure medication. It
has negative chronotropic and positive inotropic effects. Thus
increasing contractility and reducing heart rate. \\
\(\beta\) -adrenergic blocking & These are adrenergic-blocking agents
that work by decreasing sympathetic activity to the heart, decreasing
heart rate, and thus decreasing oxygen demand. Examples are Propranolol,
Atenolol and Carvedilol. \\
\end{longtable}

\section{Complications}\label{complications-5}

Complications of heart failure include renal failure, hepatic failure,
pulmonary hypertension, arrhythmia, and thromboembolic effects.

\chapter{Atrial Septal Defect}\label{atrial-septal-defect}

\section{Introduction}\label{introduction-16}

An Atrial Septal Defect (ASD) is a defect in the wall separating the
left and right atria.

\section{Incidence/Prevalence}\label{incidenceprevalence}

It is the second most common congenital heart disease and may occur in
as much as 25\% of all congenital heart disease patients. It is thought
to have a small female preponderance. Still, in a compilation of all
electrocardiogram cases in Kumasi, it formed 26\% of the patients and
showed no difference in incidence between sexes. There are four main
types: Secundum ASD (50-70\%), Primum ASD (\textasciitilde30\%), Sinus
Venosus ASD and Coronary sinus ASD. Figure~\ref{fig-csv-asd}

\begin{figure}

\centering{

\pandocbounded{\includegraphics[keepaspectratio]{images/cvs-asd.jpg}}

}

\caption{\label{fig-csv-asd}Schematic drawing of an Atrial Septal
Defect}

\end{figure}%

\section{Aetiology}\label{aetiology-2}

Most ASDs are thought to occur sporadically, but there are recorded
associations with some genetic defects and syndromes. (Caputo et al.
2005) Among these are Holt-Oram, \href{synd-noonan.qmd}{Noonan},
\href{synd-down.qmd}{Down}, and Budd-Chiari syndrome.

\section{Pathophysiology}\label{pathophysiology-7}

Since the pressure in the left atrium is higher than that of the right,
the high oxygen-content blood shunts from the left atrium to the low
oxygen-content right atrium. Thus an uncomplicated ASD is
\textbf{acyanotic}. The shunting also leads to volume overload of the
right atrium, right ventricle, pulmonary artery and lungs. This
consequently results in dilatation of the right atrium, right ventricle
and pulmonary artery, and pulmonary oedema. The low pressure in the
atria implies low pressure in the right ventricle, pulmonary artery and
lungs. This reduces the extent of pulmonary oedema and, subsequently,
overt heart failure in a child with ASD, compared to other congenital
heart defects such as \href{cvs-vsd.qmd}{ventricular septal defects}.
However, longstanding long-standing lesions or relatively large ones,
especially those with a pulmonary-to-systemic flow ratio of 2 or more,
could lead to heart failure and pulmonary hypertension after about 15 to
20 years with subsequent reversal of the shunt.

\section{Signs and symptoms}\label{signs-and-symptoms-2}

Most children with an atrial septal defect are without overt symptoms.
However, those with relatively large defects with a high Qp:Qs may
result in \href{cvs-heart-failure.qmd}{heart failure}. Therefore, many
of these patients have been diagnosed incidentally when they, for
instance, report to the health institution for another complaint. They
tend to have slender bodies and reduced exercise tolerance.

Auscultation usually reveals a widely fixed-split second heart sound and
an ejection systolic murmur of grade 2/3 to 3/6, loudest at the left
upper sternal border. Unfortunately, many ASDs are silent as well. These
properties lead to many undiagnosed ASD that are subsequently seen in
adulthood.

\section{Investigations}\label{investigations-6}

\begin{itemize}
\tightlist
\item
  At the bedside, pulse oximetry would likely reveal a normal
  SpO\textsubscript{2} as this is an acyanotic congenital heart disease.
\item
  A chest X-ray could show cardiomegaly with dilatation of the right
  side of the heart. Prominence of the pulmonary artery and an increase
  in vascular markings may also be present.
\item
  The electrocardiogram will likely show a right axis deviation due to
  the dilated right ventricle and a right atrial enlargement.
\item
  An echocardiogram is diagnostic as it visualises the defect,
  quantifies the shunt and other chamber sizes, and identifies possible
  complications.
\item
  Cardiac catheterisation is often done in long-standing cases to detect
  complications, possibly pulmonary hypertension, and quantify the shunt
  volume.
\end{itemize}

\section{Treatment}\label{treatment-2}

Treatment depends on the age at diagnosis and the size of the defect.
For a small defect with no signs of heart failure, and in a child less
than 3 years, counselling and regular review may be what is required.
The echocardiogram should be repeated at 4 years, and surgical or device
closure should be considered if the defect persists. For large defects,
heart failure medications can be started as planning of immediate
surgical closure is being made. Closure of the ASD is by use of a device
or, as may occur more often in Ghana, by open heart surgery. There is no
need for exercise restriction or prophylaxis for endocarditis.

\section{Prevention}\label{prevention-4}

There is no known mode of prevention of ASDs. However, early and
appropriate treatment of the associated symptoms and complications go a
long way in improving quality of life.

\section{Complication}\label{complication}

\begin{itemize}
\tightlist
\item
  Many patients with ASDs may not grow appropriately and instead become
  thin.
\item
  Arrhythmias may arise because of the dilated right atrium.
\item
  Though there are reported cases of paradoxical strokes in patients
  with ASDs, it remains an uncommon occurrence.
\item
  \href{cvs-inf-endo.qmd}{Infective endocarditis} is rare in ASDs.
\end{itemize}

\section{Prognosis}\label{prognosis-4}

Most ASDs will close spontaneously by 4 years, with smaller ones having
a higher closure rate than bigger ones. A long-standing large defect,
however, leads to chronic heart failure and pulmonary hypertension in
early adulthood. Prognosis is generally good, with many living into
adulthood, even without corrective surgery. Post-surgical mortality is
currently less than 0.5\%. The patient will need little long-term
follow-up after the corrective surgery.

\section{Differential diagnosis}\label{differential-diagnosis-4}

Differential diagnosis includes pulmonary stenosis and a pink
\href{cvs-tof.qmd}{Tetralogy of Fallot}.

\chapter{Ventricular Septal Defect}\label{ventricular-septal-defect}

\section{Introduction}\label{introduction-17}

This is the most common Congenital Heart Disease (CHD), being seen in
about 15-20\% of all CHDs. VSD occurs in different anatomical locations.
The most common is the peri-membranous. Others are inlet, outlet,
muscular and infundibular. They also appear in different shapes and
sizes as well.

\begin{figure}

\centering{

\pandocbounded{\includegraphics[keepaspectratio]{images/cvs-vsd.jpg}}

}

\caption{\label{fig-cvs-vsd}Ventricular Septal Defect}

\end{figure}%

\section{Pathophysiology}\label{pathophysiology-8}

Typically VSD without the presence of another congenital heart
malformation results in a left to right ventricle shunting lesion
because of the pressure difference. Thus, a VSD is usually an acyanotic
congenital heart lesion. This leads to a volume overload of the
pulmonary artery, lungs, left atrium and left ventricle. These chambers
subsequently dilated. Pulmonary edema develops from lung congestion
leading to signs of heart failure.

Secondly, depending on the size of the defect the pressure in the left
ventricle will get transmitted to the right. How much pressure is
transmitted depends on the size of the defect with bigger defects
transmitting more than smaller ones. This can result in increased right
ventricular pressure, and hypertrophy. It also worsens the pulmonary
oedema already mentioned above.

Persistent pressure and volume overload cause remodelling of the
pulmonary vasculature, resulting in permanent changes and pulmonary
hypertension. When the pulmonary pressure rises significantly higher
than the systemic pressure, a reversal of the shunt results, leading to
decreased oxygen saturation.

\section{Clinical presentation}\label{clinical-presentation}

The clinical presentation of VSDs is variable and depends on the size
and position.

\textbf{Position}: A perimembranous VSD of comparable size may exhibit
more signs of heart failure than one that is mid-muscular or apical.

\textbf{Size}: Generally, the sizes of VSDs determine the extent of the
volume and pressure overload of the right heart and lungs. Larger ones
result in relatively higher pressure and volume. Small VSDs are usually
asymptomatic with no volume or pressure overload of the lung, pulmonary
artery and right heart. Some volume and pressure overload usually
accompanies moderate-sized VSDs. They are thus often accompanied by some
\href{cvs-heart-failure.qmd}{heart failure} and recurrent lower
respiratory infections. Large defects are accompanied by severe volume
and pressure overload. They present with persistent
\href{cvs-heart-failure.qmd}{heart failure} and failure to thrive,
exercise intolerance. When longstanding, they often end up with
significant pulmonary hypertension.

Newborns with VSD may not have a murmur at birth. This is due to
relatively high pulmonary pressure in the first weeks of life. The
intensity of the murmur may increase as the pulmonary pressure
decreases, usually over 4 weeks. The patient then becomes more
symptomatic with a louder pansystolic murmur loudest at the lower left
sternal border. Patients with a large defect may present with an apical
diastolic rumble. Cyanosis may present in a long-standing large VSD with
pulmonary hypertension.

\section{Investigations}\label{investigations-7}

\begin{itemize}
\tightlist
\item
  At the bedside pulse oximetry would likely reveal a normal
  SpO\textsubscript{2} as this is an acyanotic congenital heart disease.
\item
  A chest X-ray could show cardiomegaly with dilatation of the left side
  of the heart. Increased vascular markings may also be present.
\item
  The electrocardiogram will likely show features of left ventricular
  dilatation and left atrial enlargement.
\item
  An echocardiogram is diagnostic as it visualises the defect,
  quantifies the shunt and other chamber sizes, and identifies possible
  complications.
\item
  Cardiac catheterisation is often done in cases of moderate to large
  defects. to detect complications such as possibly pulmonary
  hypertension, and quantify the shunt volume.
\end{itemize}

\section{Treatment}\label{treatment-3}

Treatment modalities depend on the patient's age, defect size and
location, associated symptoms and complications. Small defects in the
very young, without signs of heart failure, can be treated by watchful
waiting as some may close spontaneously. Defects accompanied by heart
failure symptoms should be treated for heart failure while planning for
possible surgical ligation is being done. Large defects are treated the
same but with more aggressive heart failure management and urgent
surgical therapy. This is because these are often accompanied by a
failure to thrive, poor feeding and generally poor health.

Current surgical treatment involves device closure, but some centers
such as Ghana still close the defects with an open heart surgery. Also,
there are certain circumstances where a device closure cannot be done,
thus open heart surgery becomes the only option.

\section{Prognosis}\label{prognosis-5}

In many developed countries, the prognosis of VSDs has become excellent
with surgical and device treatment. (Jortveit et al. 2016) However, in
Ghana, most patients do not realistically have the chance of early
surgical correction. Thus the prognosis of VSDs with significant
symptoms tends to be poor.

\section{Differential diagnosis}\label{differential-diagnosis-5}

Differential diagnosis of a ventricular septal defect is mitral
regurgitation and tricuspid regurgitation.

\section{Complications}\label{complications-6}

Notable complications of VSDs include \href{cvs-inf-endo.qmd}{Infective
Endocarditis}, aortic regurgitation, pulmonary hypertension, left
ventricular outflow tract obstruction and growth failure.

\chapter{Patent Ductus Arteriosus}\label{patent-ductus-arteriosus}

\section{Definition}\label{definition-10}

Patent Ductus Arteriosus (PDA) is an acyanotic congenital heart disease.
that results from the persistence into the post-natal life of the normal
fetal vascular conduit between the pulmonary and systemic arterial
systems. Figure~\ref{fig-cvs-pda} Normally, the ductus arteriosus
functionally closes within the first 1-3 days after birth. Structural
closure is usually completed by the 3\textsuperscript{rd} week of birth.

\begin{figure}

\centering{

\pandocbounded{\includegraphics[keepaspectratio]{images/cvs-pda.jpg}}

}

\caption{\label{fig-cvs-pda}Patent Ductus Arteriosus}

\end{figure}%

\section{Incidence/prevalence}\label{incidenceprevalence-1}

PDAs represent about 5-10\% of all congenital heart defects, with an
equal male-to-female ratio.(Borges-Lujan et al. 2022) It made up 23\% of
all electrocardiograph diagnoses of heart diseases seen in Kumasi, with
a male-female-ratio of 1:0.9. This high proportion of PDAs in this
cohort could be because a significant proportion of the children scanned
were preterm infants.

\section{Aetiology}\label{aetiology-3}

Although there are no recognized etiological factors, PDAs are
associated with a few recognisable conditions. These include:

\begin{enumerate}
\def\labelenumi{\arabic{enumi}.}
\tightlist
\item
  \textbf{Prematurity} - The more premature a baby is the higher the
  incidence of PDA. it is seen in as much as 80\% of babies born from 24
  to 28 weeks.
\item
  \textbf{Teratogenic agents} such as Congenital Rubella, fetal alcohol
  syndrome, fetal hydantoin syndrome and maternal phenylketonuria
\item
  \textbf{Genetic or familial} factors such as Trisomy 21, Trisomy 18,
  Trisomy 13, Noonan syndrome, CHARGE association, VATER association,
  Holt-Oram syndrome, Treacher Collins syndrome, PHACE Syndrome,
  Smith-Lemli-Opitz syndrome, Cri du chat syndrome
\item
  \textbf{Living at high altitudes} has long been associated with a
  higher incidence of PDAs
\item
  \textbf{Idiopathic} - Many patients with a PDA have no identifiable
  risk factors.
\end{enumerate}

\section{Pathophysiology}\label{pathophysiology-9}

Ductus arteriosus in fetal circulation is indispensable to allow
right-to-left shunting of nutrient-rich oxygenated blood from the
placenta to the fetal systemic circulation, bypassing the fetal
pulmonary circuit. At birth, the rise in PaO\textsubscript{2} and
decline in prostaglandin concentration cause closure of the ductus
arteriosus, typically beginning within the first 10 to 15 hours of life.
If this normal process does not occur, the ductus arteriosus will remain
patent. The PDA then results in excess blood shunting from the aorta,
across the duct and into the pulmonary artery. This shunting causes
volume overload. There is therefore a circuit of excess blood volume in
the pulmonary arteries, lungs, left atrium, left ventricle, and aorta.
This subsequently leads to dilatation of the left pulmonary artery, left
atrium and left ventricle as well as pulmonary edema and heart failure.

\section{Signs and symptoms}\label{signs-and-symptoms-3}

Symptoms vary based on the volume of additional blood flow to the lungs.

\begin{enumerate}
\def\labelenumi{\arabic{enumi}.}
\tightlist
\item
  The degree of the shunt depends on:

  \begin{itemize}
  \tightlist
  \item
    The size of the PDA (including diameter, length, and tortuosity).
    Bigger ducts with shorter lengths often result in worse symptoms.
    Conversely, patients with small PDAs are often asymptomatic.
  \item
    Pulmonary vascular resistance when high does not encourage shunting
    across the duct. However, as the resistance drops the shunt gets
    worse, with worsening symptoms.
  \item
    Moderate to larger shunts produce the symptoms of congestive heart
    failure as the pulmonary vascular resistance decreases over the
    first 6 to 8 weeks of life.
  \end{itemize}
\item
  The physical examination depends on the size of the shunt, and to a
  lesser extent the age and maturity of the patient.

  \begin{itemize}
  \item
    Premature infants may present with:

    \begin{itemize}
    \tightlist
    \item
      Tachypnoea, crackles, tachycardia
    \item
      Hyperdynamic precordium and bounding pulses with wide pulse
      pressure
    \item
      \textbf{\emph{Pansystolic}} murmur loudest at the left upper or
      mid-sternal border.
    \item
      With a large PDA and equalization of pressure between the main
      pulmonary artery and the aorta, no murmur may be heard.
    \item
      Soft tender hepatomegaly
    \end{itemize}
  \item
    Infants and older children with small PDAs may present with:

    \begin{itemize}
    \tightlist
    \item
      A pansystolic murmur loudest in the 2\textsuperscript{nd} left
      intercostal space
    \item
      Murmur becomes continuous as the pulmonary vascular resistance
      decreases over the first months of life.
    \end{itemize}
  \item
    Infants and older children with moderate to large PDA may present
    with:

    \begin{itemize}
    \tightlist
    \item
      Louder murmur with a harsh quality and acquires a
      \textbf{\emph{machine-like}} quality often being heard
      posteriorly. A systolic thrill may be felt at the left upper
      sternal border.
    \item
      Tachycardia, bounding pulses with a wide pulse pressure, and a
      mid-diastolic low-frequency rumbling murmur may be audible at the
      apex with a large PDA
    \item
      With severe left ventricular failure the classic PDA signs may
      disappear, but there will be findings consistent with congestive
      heart failure (tachycardia, S3 gallop at the apex, tachypnoea,
      soft tender hepatomegaly, bi-basal crackles)
    \item
      Pulmonary hypertension may occur in long-standing cases. In
      advanced cases of irreversible pulmonary vascular disease,
      cyanosis occurs with the reversal of shunting.
    \end{itemize}
  \end{itemize}
\end{enumerate}

\section{Investigations}\label{investigations-8}

\begin{enumerate}
\def\labelenumi{\arabic{enumi}.}
\tightlist
\item
  \textbf{Chest X-ray:} Varies from normal (small PDAs) to prominence of
  main and peripheral pulmonary arteries and vasculature. Findings are
  more pronounced with moderate to large PDAs and may show cardiomegaly,
  and increased pulmonary vascular markings proportional to the
  left-to-right shunt. A dilated pulmonary artery may be seen on the
  chest x-ray in long-standing cases. Figure~\ref{fig-cvs-cxr-pda}
  Pulmonary oedema can be seen with congestive heart failure.
\item
  \textbf{Electrocardiogram:} Findings vary from normal (small PDAs) to
  evidence of left atrial dilatation and left ventricular hypertrophy
  with moderate to large PDA. Evidence of bi-ventricular hypertrophy in
  long-standing cases. If pulmonary hypertension is present, evidence of
  right ventricular hypertrophy may be seen.
\end{enumerate}

\begin{figure}

\centering{

\pandocbounded{\includegraphics[keepaspectratio]{images/cvs-cxr-pda.jpg}}

}

\caption{\label{fig-cvs-cxr-pda}Chest X-ray showing dilated pulmonary
artery in a child with PDA}

\end{figure}%

\begin{enumerate}
\def\labelenumi{\arabic{enumi}.}
\setcounter{enumi}{2}
\tightlist
\item
  \textbf{Echocardiogram:} This delineates the PDA and assesses the size
  of the left atrium and ventricle. Useful for evaluating pulmonary
  hypertension. Doppler for determining the flow pattern.
\item
  \textbf{Cardiac catheterisation:} most often not essential for
  diagnosis. Can be performed for treatment using transcatheter closure
  techniques
\end{enumerate}

\section{Treatment}\label{treatment-4}

\begin{enumerate}
\def\labelenumi{\arabic{enumi}.}
\tightlist
\item
  Supportive treatment including careful use of oxygen and respiratory
  assistance
\item
  Management of CHF with diuretics commonly furosemide and
  spironolactone, digoxin and afterload reduction on a case-by-case
  basis
\item
  Pharmacologic closure of PDA: Nonsteroidal Anti-Inflammatory Drugs
  (NSAIDs) such as indomethacin or ibuprofen; are not usually effective
  in infants or older children but are in the early neonatal period.
  Contraindications to pharmacologic closure include co-existing
  congenital heart defects that are duct-dependent, renal impairment,
  thrombocytopenia and associated conditions such as NEC and IVH.
\item
  Surgical closure is indicated in symptomatic or haemodynamically
  significant PDA. Surgical closure is achieved by open surgical
  ligation and division, video-assisted thoracoscopic ligation or
  transcatheter occlusion with coils or other devices.
\end{enumerate}

\section{Complications}\label{complications-7}

The recognised complication of PDA includes
\href{cvs-inf-endo.qmd}{infectious endocarditis}, pulmonary hypertension
and heart failure.

\section{Prognosis}\label{prognosis-6}

\begin{enumerate}
\def\labelenumi{\arabic{enumi}.}
\tightlist
\item
  The chance of spontaneous closure in a preterm baby is about 95\%,
  while the likelihood of spontaneous closure in a term baby is more
  than 90\% by age 1.(Yuan et al. 2021) This is because PDA in term
  infants results from a structural abnormality of the ductal smooth
  muscles rather than a decrease in responsiveness of the ductal smooth
  muscles to oxygen.
\item
  Heart failure and risk of recurrent chest infections develop for large
  shunts
\item
  Large shunts are also a risk for the development of pulmonary
  hypertension
\item
  Surgical treatment now has an almost 100\% success rate in many
  centres.
\end{enumerate}

\section{Differential diagnosis}\label{differential-diagnosis-6}

Other acyanotic congenital heart diseases are possible differentials.
These include a large \href{cvs-asd.qmd}{atrial septal defect} and a
\href{cvs-coa.qmd}{coarctation of the aorta}.

\chapter{Coarctation of the Aorta}\label{coarctation-of-the-aorta}

\section{Definition}\label{definition-11}

Coarctation of the aorta (CoA) is a congenital heart defect
characterized by the narrowing of the aorta, the major artery
responsible for carrying oxygen-rich blood from the heart to the rest of
the body. This condition accounts for approximately 5-8\% of all
congenital heart defects in children and is more prevalent in males than
females. CoA can present with varying degrees of severity and may occur
as an isolated defect or associated with other cardiac anomalies, such
as bicuspid aortic valve, ventricular septal defect (VSD), or complex
syndromes like Turner syndrome.

\begin{figure}

\centering{

\pandocbounded{\includegraphics[keepaspectratio]{images/cvs-coa.jpg}}

}

\caption{\label{fig-cvs-coa}Coarctation of the aorta}

\end{figure}%

\section{\texorpdfstring{\textbf{Anatomy and
Pathophysiology}}{Anatomy and Pathophysiology}}\label{anatomy-and-pathophysiology}

The aorta plays a crucial role in distributing oxygenated blood from the
heart's left ventricle to the systemic circulation. In CoA, the
narrowing typically occurs at the isthmus of the aorta, which is located
distal to the left subclavian artery and near the ductus arteriosus, a
fetal blood vessel that normally closes after birth. The severity of CoA
depends on the degree of narrowing, which can obstruct blood flow and
increase the heart's afterload.

The left ventricle must work harder in children to pump blood through
the narrowed segment, leading to left ventricular hypertrophy. Prolonged
obstruction may result in high blood pressure (hypertension) in the
upper body and diminished blood flow to the lower body. Collateral
circulation often develops as the body compensates using smaller vessels
to bypass the narrowing, but this is not always sufficient to normalize
blood flow.

\section{\texorpdfstring{\textbf{Clinical
Presentation}}{Clinical Presentation}}\label{clinical-presentation-1}

The clinical manifestations of CoA in children vary based on the
severity of the narrowing. Severe cases may present in infancy, while
milder forms might remain undetected until adolescence or adulthood.

\subsection{\texorpdfstring{\textbf{Infants:}}{Infants:}}\label{infants}

\begin{itemize}
\item
  Severe CoA may cause critical illness within the first few weeks of
  life, especially after the ductus arteriosus closes.
\item
  Symptoms include poor feeding, failure to thrive, lethargy,
  respiratory distress, and signs of heart failure.
\item
  Pulses in the lower extremities may be weak or absent, and blood
  pressure measurements reveal significant upper-to-lower extremity
  discrepancies.
\end{itemize}

\subsection{\texorpdfstring{\textbf{Older
Children:}}{Older Children:}}\label{older-children}

\begin{itemize}
\item
  Milder CoA might be asymptomatic or present with less obvious
  symptoms, such as fatigue, leg pain during exercise (claudication),
  headaches, or nosebleeds.
\item
  Hypertension is common in older children and may be detected
  incidentally during routine health check-ups.
\item
  Physical examination often reveals a systolic murmur heard best over
  the back, diminished or delayed femoral pulses, and upper extremity
  hypertension relative to the lower extremities.
\end{itemize}

\section{\texorpdfstring{\textbf{Diagnostic
Evaluation}}{Diagnostic Evaluation}}\label{diagnostic-evaluation}

Timely diagnosis of CoA is critical to prevent complications and ensure
appropriate management. Several diagnostic tools are employed to confirm
the condition and assess its severity.

\begin{enumerate}
\def\labelenumi{\arabic{enumi}.}
\item
  \textbf{Physical Examination:}

  \begin{itemize}
  \tightlist
  \item
    Blood pressure measurements in all four extremities to identify
    discrepancies.
  \item
    Palpation of pulses to detect reduced or absent femoral pulses.
  \item
    Auscultation for murmurs and other abnormal heart sounds.
  \end{itemize}
\item
  \textbf{Imaging Studies:}

  \begin{itemize}
  \tightlist
  \item
    \textbf{Chest X-ray:} May show rib notching (caused by collateral
    vessels) and a characteristic ``3 sign'' of the aorta.
  \item
    \textbf{Echocardiography:} The primary diagnostic tool for
    visualizing the narrowed segment of the aorta and assessing
    associated anomalies.
  \item
    \textbf{Magnetic Resonance Imaging (MRI) or Computed Tomography
    (CT):} Provides detailed anatomical information, especially in older
    children or when planning surgical interventions.
  \end{itemize}
\item
  \textbf{Cardiac Catheterization:}

  \begin{itemize}
  \tightlist
  \item
    Invasive procedures are used for definitive diagnosis in certain
    cases and for interventional treatment.
  \item
    Measures pressure gradients across the narrowed segment and
    evaluates the severity of obstruction.
  \end{itemize}
\end{enumerate}

\section{\texorpdfstring{\textbf{Management}}{Management}}\label{management-10}

The treatment of CoA depends on the child's age, the severity of the
condition, and the presence of associated cardiac defects. The primary
goal is to relieve the obstruction, restore normal blood flow, and
prevent complications.

\begin{enumerate}
\def\labelenumi{\arabic{enumi}.}
\item
  \textbf{Medical Management:}

  \begin{itemize}
  \item
    In neonates with severe CoA and ductal-dependent circulation,
    prostaglandin E1 infusion maintains ductus arteriosus patency and
    ensures adequate lower body perfusion.
  \item
    Medications such as inotropes and diuretics may be administered to
    manage heart failure symptoms before definitive treatment.
  \end{itemize}
\item
  \textbf{Surgical Repair:}

  \begin{itemize}
  \item
    Surgical correction is often the preferred treatment for infants and
    young children with severe CoA.
  \item
    Techniques include resection of the narrowed segment with end-to-end
    anastomosis, subclavian flap aortoplasty, or patch augmentation.
  \item
    Surgery is typically performed during infancy or early childhood to
    minimize long-term complications and avoid the development of
    significant collateral circulation.
  \end{itemize}
\item
  \textbf{Catheter-Based Interventions:}

  \begin{itemize}
  \item
    Balloon angioplasty and stent placement are minimally invasive
    alternatives, particularly in older children and adolescents.
  \item
    These procedures are often used for coarctation after initial
    surgical repair or in cases where surgery is not feasible
  \end{itemize}
\end{enumerate}

\section{\texorpdfstring{\textbf{Complications}}{Complications}}\label{complications-8}

Untreated or inadequately treated CoA can lead to significant
complications, including:

\begin{itemize}
\tightlist
\item
  Persistent hypertension, even after successful repair.
\item
  Aortic aneurysm or dissection, particularly in cases of long-standing
  hypertension.
\item
  Heart failure due to left ventricular strain.
\item
  Premature coronary artery disease and cerebrovascular events, such as
  stroke.
\item
  Infective endocarditis, especially in cases with associated valvular
  abnormalities.
\end{itemize}

\section{\texorpdfstring{\textbf{Long-Term
Outcomes}}{Long-Term Outcomes}}\label{long-term-outcomes}

With advancements in diagnostic techniques and treatment modalities, the
prognosis for children with CoA has improved significantly. However,
long-term follow-up is essential to monitor for residual or recurrent
narrowing, hypertension, and other complications. Lifelong care under a
cardiologist familiar with congenital heart defects is recommended.

\section{\texorpdfstring{\textbf{Prognosis}}{Prognosis}}\label{prognosis-7}

The long-term outlook for children with CoA largely depends on the
timing and success of treatment. Early intervention typically results in
good outcomes, with most children leading normal or near-normal lives.
However, ongoing surveillance is crucial to address potential issues
such as:

\begin{itemize}
\tightlist
\item
  Residual or recurrent coarctation.
\item
  Systemic hypertension.
\item
  Associated cardiac or vascular anomalies.
\end{itemize}

\section{\texorpdfstring{\textbf{Prevention and Genetic
Considerations}}{Prevention and Genetic Considerations}}\label{prevention-and-genetic-considerations}

Since CoA is a congenital defect, prevention strategies focus on early
detection and management. Prenatal ultrasounds can sometimes identify
CoA in utero, especially in high-risk pregnancies. Genetic counseling
may be beneficial for families with a history of congenital heart
defects or syndromes like Turner syndrome, which are associated with a
higher risk of CoA.

\section{\texorpdfstring{\textbf{Conclusion}}{Conclusion}}\label{conclusion-13}

Coarctation of the aorta is a significant congenital heart defect that
poses challenges in diagnosis and management, particularly in infants
and young children. Advances in medical and surgical interventions have
greatly improved outcomes, but timely recognition and treatment remain
critical. Long-term follow-up and a multidisciplinary approach involving
pediatric cardiologists, surgeons, and primary care providers are
essential to optimize the health and well-being of affected children.

\begin{center}\rule{0.5\linewidth}{0.5pt}\end{center}

\chapter{Tetralogy of Fallot}\label{tetralogy-of-fallot}

\section{Definition}\label{definition-12}

Tetralogy of Fallot (ToF) is a conotruncal defect resulting from
anterior misalignment of the infundibular septum, giving rise to four
components:

\begin{itemize}
\tightlist
\item
  Large, nonrestrictive \href{cvs-vsd.qmd}{Ventricular Septal Defect}
\item
  The aorta overriding the interventricular septum
\item
  Right ventricular outflow tract obstruction, including at the
  infundibulum, main and branch pulmonary arteries
\item
  Right ventricular hypertrophy
\end{itemize}

\section{Incidence/prevalence}\label{incidenceprevalence-2}

It is the most common cyanotic congenital heart disease. About 1 in 3500
babies born in the US are born with ToF. Accounts for 7--10\% of all
congenital cardiac malformations. More common in Males. In a series of
echocardiograms done in Kumasi, Ghana, ToF was the most common cyanotic
congenital heart disease seen in 13\% of all heart diseases. There was a
significant male preponderance with a female vs male prevalence of 10\%
vs 15\%, respectively.

\section{Aetiology}\label{aetiology-4}

Although there is no known definite cause as to why some babies develop
ToF in utero, certain environmental and biological factors are known to
increase the risk:

\begin{itemize}
\item
  \textbf{Genetic}: CHARGE syndrome, chromosome 22q11 microdeletion
  (\href{synd-digeorge.qmd}{Di George syndrome}), \href{}{Down
  syndrome}, \href{synd-edwards.qmd}{Edward's syndrome}, or
  \href{synd-patau.qmd}{Patau syndrome}, VACTERL association
\item
  \textbf{Teratogens}: maternal diabetes mellitus, retinoic acid
  exposure, maternal phenylketonuria (PKU), Alcohol
  (\href{synd-fetal-alcohol.qmd}{fetal alcohol syndrome}), Warfarin
  (fetal warfarin syndrome), Trimethadione: antiepileptic drug
\end{itemize}

\section{Pathophysiology}\label{pathophysiology-10}

\begin{itemize}
\tightlist
\item
  The severity of clinical signs and symptoms depends on the proportion
  of the cardiac output going through the pulmonary artery, the relative
  pressures in the right and left ventricles, and the proportion of the
  aorta overriding the VSD.
\item
  The VSD is normally of a significant size, which causes the systolic
  pressures between the ventricles to equalize. In mild ToF, the left
  ventricular pressures remain higher than the right ventricle, thus,
  blood shunts from left to right through the VSD. These patients are
  normally acyanotic. In more severe diseases, due to increased right
  ventricular pressure (secondary to Pulmonary Stenosis), the shunt
  direction reverses from right to left, allowing the mixing of
  deoxygenated and oxygenated blood. This results in lower oxygenated
  blood in the systemic circulation, making patients cyanotic.
\item
  Pulmonary stenosis can be classified according to its location. The
  commonest site is the infundibular septum (50\%). The stenosis may
  also be valvular (10\%) or a combination (30\%). This results in
  impaired flow of deoxygenated blood into the main pulmonary artery. It
  may be severe enough to cause intermittent right ventricular outflow
  tract obstruction. This forms the basis of hypercyanotic episodes (tet
  spells).
\item
  Hypertrophy of the right ventricle occurs in response to the high
  pressures it must overcome to force deoxygenated blood through the
  right ventricular outflow tract obstruction. Compared to the normal
  heart, the aorta in ToF is dilated and displaced over the
  interventricular septum. Aortic dilatation is caused by an increase in
  blood flow through the aorta as it receives blood from both ventricles
  via the \href{cvs-vsd.qmd}{Ventricular Septal Defect}.
\end{itemize}

\section{Signs and symptoms}\label{signs-and-symptoms-4}

Patients with Tof are often not born with cyanosis. This may lead to the
diagnosis being missed at birth. However, progressive cyanosis occurs in
the first year of life. They usually have little or no signs of
\href{cvs-heart-failure.qmd}{heart failure}. Most will present with poor
exercise tolerance, which worsens as the child ages. In children who can
walk, frequent squatting is observed in unrepaired ToF. Other features
include poor feeding and poor weight gain.

Physical examination of children with ToF may reveal digital clubbing of
varying stages, central cyanosis, and plethora, usually seen in the
hands and eyes. Auscultation classically reveals the first and second
heart sounds with an ejection systolic murmur loudest at the upper to
middle left sternal edge.

\section{Investigations}\label{investigations-9}

Bedside pulse oximetry often reveals an oxygen saturation of less than
90\%. A chest X-ray shows a normal-sized heart with a classical boot
shape and decreased pulmonary vascular markings. In about 30\% of the
cases, a right arch is present.

\begin{figure}

\centering{

\pandocbounded{\includegraphics[keepaspectratio]{images/cvs-Tof.jpg}}

}

\caption{\label{fig-ToF}Boot-shaped heart of Tetralogy of Fallot}

\end{figure}%

An electrocardiogram, though non-specific, may show right axis
deviation, right ventricular hypertrophy, and right atrial enlargement.

An echocardiogram is the most useful diagnostic modality. It delineates
the defect by showing the anterior malalignment
\href{cvs-vsd.qmd}{Ventricular Septal Defect}, degree of infundibular
stenosis, state of the pulmonary artery and/or branches, and the
overriding aortic arch sidedness. Other anatomical abnormalities that
may co-exist, eg, Atrioventricular Canal Defects,
\href{cvs-asd.qmd}{Atrial Septal Defect}, and coronary artery
abnormalities can also be obtained. A CT angiogram and Magnetic
Resonance Angiography can be done in complex cases and in preparation
for surgery.

\section{Treatment}\label{treatment-5}

Since patients with ToF hardly experience
\href{cvs-heart-failure.qmd}{heart failure}, antifailure medications are
not the mainstay of treatment.

Untreated cyanotic congenital heart disease is associated with a chronic
hypoxic state, which leads to polycythemia. This leads to a high demand
for iron and increases susceptibility to iron deficiency. Furthermore,
it is well documented that iron deficiency increases the chance of a
hypercyanotic spell and stroke. Iron treatment in ToF is therefore key
in its outpatient management. Nutritional rehabilitation is done for
chronically malnourished patients.

Surgical therapy is the definitive treatment. A Blalock-Taussig shunt
can be done to improve oxygenation. A stand can also be inserted in
younger children when deemed necessary. Definitive corrective surgery
should be done for all children with ToF.

\section{Natural history}\label{natural-history}

Features of ToF are progressive if not corrected surgically. There is
usually progressive dyspnoea on exertion and cyanosis as the child ages.
However, some ``pink tets'' can live very well into adulthood.
Untreated, approximately 50\% will live to their 6th birthday.

\section{Complications}\label{complications-9}

A feared presentation in untreated children with a ToF is the
hypercyanotic spell (Tet spell). This is treated further below. Other
complications can result from the embolus effect, leading to stroke and
cerebral abscess. High hematocrit may lead to hyperviscosity, headache,
and dizziness. \href{cvs-inf-endo.qmd}{Infective endocarditis} is
another known complication. Long-term complications include right
ventricular dysfunction, coagulopathy, and arrhythmias.

\section{Prognosis}\label{prognosis-8}

Repaired, the 25-year survival is about 95\%.(Smith et al. 2019)
Unrepaired, most will die by their 10th birthday. This is especially so
for those with other genetic syndromes and associated malformations.

\section{Differential diagnosis}\label{differential-diagnosis-7}

Cyanosed ToF patients have a differential diagnosis of Transposition of
the great arteries, Tricuspid atresia, pulmonary atresia, etc. Pink ToFs
will have a differential diagnosis of a \href{cvs-vsd-qmd}{Ventricular
Septal Defect}

\section{Prevention}\label{prevention-5}

There is no known prevention for ToF. However, it is always prudent for
prospective mothers and those in the first trimester to avoid
recreational drugs, alcohol, and some over-the-counter medications.
Also, folic acid supplementation should be encouraged.

\section{Hypercyanotic spell}\label{hypercyanotic-spell}

A hypercyanotic spell (tet spell) is an emergency in children with ToF
and, to a lesser extent, other cyanotic congenital heart diseases. It
presents most commonly in children less than 2 years old.

\subsection{Presentation}\label{presentation-1}

Children with hypercyanotic spells present with paroxysms of increased
and deep breathing, irritability, prolonged, unsettled crying,
increasing cyanosis, seizures, and decreased intensity of the heart
murmur. In untreated cases, this might lead to brain damage or death.

\subsection{Pathophysiology}\label{pathophysiology-11}

A hypercyanitic spell can have many precipitating factors. These may
include fever, anemia, dehydration, prolonged crying, and eating. These
precipitants lead to decreased lung blood flow and progressively
increase right-to-left shunting. This leads to increasing cyanosis,
tachycardia, and reduced systemic vascular resistance. Carbon dioxide
accumulation stimulates the central respiratory center, leading to
increased and deep breathing. All these unfortunately cause further
right-to-left shunting, thus perpetuating the hypoxia.

\subsection{Treatment}\label{treatment-6}

The treatment goal is to increase preload and promote pulmonary blood
flow.

\begin{itemize}
\tightlist
\item
  First, the child should be placed in a knee-chest position. This
  increases systemic vascular resistance, temporarily raises the
  systemic pressure, and reduces the right-to-left shunting.
\item
  Oxygen can be administered, though it is of limited value and should
  not be forced on the patient if he is combative.
\item
  Next volume expansion with intravenous fluids should be administered.
  This raises the preload and increases systemic pressure, thus reducing
  right-to-left shunting.
\item
  Intramuscular or subcutaneous morphine can be administered. This aids
  in relaxing the infundibular muscle and thus promotes pulmonary blood
  flow. It also sedates the child, thus making him/her less acidotic.
  Acidosis perpetuates the hyperchaotic spell
\item
  Intravenous propranolol, esmolol, or metoprolol is administered to
  reduce the right ventricular outflow tract obstruction.
\item
  Some alpha-agonists, such as Phenylephrine, can be given to improve
  blood pressure in severe cases
\item
  Long-term treatment may include oral propranolol for prophylaxis, iron
  supplementation, and surgical correction.
\end{itemize}

\chapter{Rheumatic Heart Disease}\label{rheumatic-heart-disease}

\section{\texorpdfstring{\textbf{Introduction}}{Introduction}}\label{introduction-18}

Rheumatic heart disease (RHD) is a chronic condition resulting from
acute rheumatic fever (ARF), an autoimmune response to group A
beta-hemolytic streptococcal (GAS) pharyngitis. It is characterized by
permanent damage to the heart valves, particularly the mitral and aortic
valves, due to repeated episodes of inflammation and scarring. RHD
remains a significant cause of morbidity and mortality among children
and young adults in low- and middle-income countries. Early diagnosis
and management of ARF and RHD are critical to prevent severe
complications and improve outcomes.

\section{\texorpdfstring{\textbf{Incidence and
Prevalence}}{Incidence and Prevalence}}\label{incidence-and-prevalence}

RHD affects approximately \textbf{40 million people worldwide}, with the
highest burden in sub-Saharan Africa, South Asia, the Pacific Islands,
and parts of Latin America. The global prevalence in children aged 5--15
years is estimated at \textbf{1--3 per 1,000}, but it can exceed
\textbf{10 per 1,000} in high-risk populations. ARF, the precursor to
RHD, occurs most commonly between \textbf{5 and 15 years} of age, with
peak incidence following untreated or inadequately treated GAS
pharyngitis.

\section{\texorpdfstring{\textbf{Etiology}}{Etiology}}\label{etiology-1}

The primary etiology of RHD is \textbf{recurrent ARF episodes} triggered
by an \textbf{immune response to GAS infection}. The following factors
contribute to its development:

\begin{enumerate}
\def\labelenumi{\arabic{enumi}.}
\tightlist
\item
  \textbf{Infectious Agent}: GAS infection, particularly of the throat,
  is necessary to initiate the autoimmune process. Certain GAS strains
  (M-protein serotypes) are more rheumatogenic.
\item
  \textbf{Host Susceptibility}: Genetic predisposition plays a role,
  with family clustering observed in affected individuals.
\item
  \textbf{Environmental Factors}: Overcrowding, poor hygiene, and
  limited access to healthcare increase the risk of GAS infections and
  progression to ARF and RHD.
\end{enumerate}

RHD develops through the following sequence:

\begin{enumerate}
\def\labelenumi{\arabic{enumi}.}
\tightlist
\item
  \textbf{GAS Pharyngitis}: GAS infection elicits an immune response
  involving antibodies and T-cells targeting streptococcal antigens.
\item
  \textbf{Molecular Mimicry}: Cross-reactivity occurs between
  streptococcal antigens (e.g., M protein) and human proteins in the
  heart, joints, brain, and skin. Autoimmune inflammation leads to
  tissue damage.
\item
  \textbf{Acute Rheumatic Fever}: Pancarditis (endocarditis,
  myocarditis, and pericarditis) is the hallmark of ARF. The endocardium
  is most affected, leading to valvulitis.
\item
  \textbf{Chronic RHD}: Recurrent inflammation and scarring cause
  permanent valvular damage, predominantly affecting the mitral and
  aortic valves. Mitral stenosis is the most common lesion, followed by
  mitral regurgitation and aortic regurgitation.
\end{enumerate}

\section{\texorpdfstring{\textbf{Signs and
Symptoms}}{Signs and Symptoms}}\label{signs-and-symptoms-5}

The clinical presentation of RHD varies based on the severity of
valvular involvement and associated complications.

\begin{enumerate}
\def\labelenumi{\arabic{enumi}.}
\tightlist
\item
  \textbf{Symptoms}:

  \begin{itemize}
  \tightlist
  \item
    Fatigue and exercise intolerance
  \item
    Dyspnea, initially on exertion and later at rest
  \item
    Palpitations due to arrhythmias (e.g., atrial fibrillation)
  \item
    Cough and hemoptysis (in severe mitral stenosis)
  \item
    Edema and signs of heart failure in advanced cases
  \end{itemize}
\item
  \textbf{Signs}:

  \begin{itemize}
  \item
    \textbf{Cardiac Murmurs}:

    \begin{itemize}
    \tightlist
    \item
      Mitral stenosis: Low-pitched diastolic murmur with an opening
      snap.
    \item
      Mitral regurgitation: holosystolic murmur at the apex
    \item
      Aortic regurgitation: High-pitched diastolic murmur.
    \end{itemize}
  \item
    Cardiomegaly with a displaced apex beat
  \item
    Signs of pulmonary hypertension (e.g., loud pulmonary component of
    S2)
  \item
    Peripheral edema, hepatomegaly, and ascites in heart failure
  \end{itemize}
\item
  \textbf{History of ARF}:

  \begin{itemize}
  \tightlist
  \item
    Clinical features such as migratory polyarthritis, carditis, chorea,
    subcutaneous nodules, or erythema marginatum support prior episodes
    of ARF.
  \end{itemize}
\end{enumerate}

\section{\texorpdfstring{\textbf{Investigations}}{Investigations}}\label{investigations-10}

The diagnosis of RHD involves clinical assessment, laboratory tests, and
imaging studies.

\begin{enumerate}
\def\labelenumi{\arabic{enumi}.}
\tightlist
\item
  \textbf{Laboratory Tests}:

  \begin{itemize}
  \tightlist
  \item
    \textbf{Throat Culture or Rapid Antigen Test}: To confirm GAS
    infection if suspected.
  \item
    \textbf{Anti-Streptolysin O (ASO) Titer}: Elevated in recent GAS
    infections.
  \item
    \textbf{C-reactive protein (CRP) and Erythrocyte Sedimentation Rate
    (ESR)}: Markers of inflammation during ARF episodes.
  \end{itemize}
\item
  \textbf{Imaging}:

  \begin{itemize}
  \tightlist
  \item
    \textbf{Echocardiography}:

    \begin{itemize}
    \tightlist
    \item
      Key diagnostic tool for detecting valvular lesions and assessing
      severity.
    \item
      Common findings include leaflet thickening, restricted mobility,
      and regurgitation or stenosis
    \end{itemize}
  \item
    \textbf{Chest X-Ray}:

    \begin{itemize}
    \tightlist
    \item
      Cardiomegaly and pulmonary congestion in advanced disease.
    \end{itemize}
  \item
    \textbf{Electrocardiogram (ECG)}:

    \begin{itemize}
    \tightlist
    \item
      Prolonged PR interval (first-degree heart block) in ARF.
    \item
      Atrial fibrillation or other arrhythmias in chronic RHD
    \end{itemize}
  \end{itemize}
\item
  \textbf{Other Tests}:

  \begin{itemize}
  \tightlist
  \item
    Cardiac MRI in selected cases for detailed assessment of myocardial
    and valvular involvement.
  \end{itemize}
\end{enumerate}

\section{\texorpdfstring{\textbf{Treatment}}{Treatment}}\label{treatment-7}

Management of RHD aims to reduce symptoms, prevent disease progression,
and address complications.

\begin{enumerate}
\def\labelenumi{\arabic{enumi}.}
\tightlist
\item
  \textbf{Medical Management}:

  \begin{itemize}
  \tightlist
  \item
    \textbf{Antibiotic Prophylaxis}: Long-term benzathine penicillin G
    intramuscular injections every 3--4 weeks to prevent recurrent ARF
    episodes.
  \item
    \textbf{Heart Failure Management}: Diuretics, ACE inhibitors, and
    beta-blockers for symptomatic relief.
  \item
    \textbf{Anticoagulation}: Warfarin for patients with atrial
    fibrillation or mechanical valve replacement.
  \item
    \textbf{Anti-Inflammatory Therapy}: Aspirin or corticosteroids for
    active carditis.
  \end{itemize}
\item
  \textbf{Surgical and Interventional Treatment}:

  \begin{itemize}
  \tightlist
  \item
    \textbf{Valvuloplasty}:

    \begin{itemize}
    \tightlist
    \item
      Percutaneous balloon mitral valvotomy for mitral stenosis in
      suitable candidates
    \end{itemize}
  \item
    \textbf{Valve Repair or Replacement}:

    \begin{itemize}
    \tightlist
    \item
      Required for severe valvular dysfunction or when medical
      management fails.
    \end{itemize}
  \end{itemize}
\end{enumerate}

\section{\texorpdfstring{\textbf{Prevention}}{Prevention}}\label{prevention-6}

The cornerstone of RHD prevention is the timely diagnosis and treatment
of GAS pharyngitis and ARF.

\begin{enumerate}
\def\labelenumi{\arabic{enumi}.}
\tightlist
\item
  \textbf{Primary Prevention}:

  \begin{itemize}
  \tightlist
  \item
    Early recognition and antibiotic treatment of streptococcal
    pharyngitis with penicillin or amoxicillin.
  \item
    Improved hygiene and reduced overcrowding to lower transmission
    risk.
  \end{itemize}
\item
  \textbf{Secondary Prevention}:

  \begin{itemize}
  \tightlist
  \item
    Long-term antibiotic prophylaxis to prevent recurrent ARF.
  \item
    Duration of prophylaxis:

    \begin{itemize}
    \tightlist
    \item
      At least 10 years after the last episode of ARF or until the
      patient is 21 years old, whichever is longer.
    \item
      Life-long prophylaxis for severe valvular disease or post-surgical
      cases
    \end{itemize}
  \end{itemize}
\item
  \textbf{Community Interventions}

  \begin{itemize}
  \tightlist
  \item
    Public health programs to increase awareness and access to
    healthcare in high-burden regions.
  \end{itemize}
\end{enumerate}

\section{\texorpdfstring{\textbf{Complications}}{Complications}}\label{complications-10}

RHD can lead to severe complications if not adequately managed:

\begin{enumerate}
\def\labelenumi{\arabic{enumi}.}
\tightlist
\item
  \textbf{Heart Failure}: Due to progressive valvular dysfunction and
  increased cardiac workload.
\item
  \textbf{Atrial Fibrillation}: Common in mitral stenosis, leading to
  thromboembolic events like stroke.
\item
  \textbf{Pulmonary Hypertension}: Resulting from chronic left-sided
  valvular disease.
\item
  \textbf{Infective Endocarditis}: Increased risk in patients with
  damaged valves.
\item
  \textbf{Pregnancy Complications}: Significant maternal and fetal risks
  due to increased hemodynamic demands.
\end{enumerate}

\section{\texorpdfstring{\textbf{Prognosis}}{Prognosis}}\label{prognosis-9}

The prognosis of RHD depends on the severity of valvular involvement,
the effectiveness of secondary prophylaxis, and access to medical and
surgical care. Without intervention, severe RHD can result in
progressive heart failure, significant morbidity, and premature death.
With timely diagnosis and appropriate management, many children can
experience improved quality of life and survival.

\section{\texorpdfstring{\textbf{Differential
Diagnosis}}{Differential Diagnosis}}\label{differential-diagnosis-8}

Several conditions can mimic the clinical presentation of RHD and should
be considered:

\begin{enumerate}
\def\labelenumi{\arabic{enumi}.}
\tightlist
\item
  \textbf{Congenital Heart Disease}: Examples include atrial septal
  defect, ventricular septal defect, and patent ductus arteriosus.
\item
  \textbf{Infective Endocarditis}: Characterized by fever, new murmur,
  and signs of embolization.
\item
  \textbf{Kawasaki Disease}: Vasculitis with coronary artery
  involvement, fever, and mucocutaneous inflammation.
\item
  \textbf{Myocarditis}: Viral or autoimmune causes leading to cardiac
  inflammation.
\item
  \textbf{Mitral Valve Prolapse}:

  \begin{itemize}
  \tightlist
  \item
    Can mimic mitral regurgitation murmurs
  \end{itemize}
\end{enumerate}

\section{\texorpdfstring{\textbf{Conclusion}}{Conclusion}}\label{conclusion-14}

Rheumatic heart disease remains a major public health challenge in
developing countries, disproportionately affecting children and young
adults. Early recognition and treatment of GAS pharyngitis and
consistent secondary prophylaxis are essential to prevent the
progression to RHD. Multidisciplinary care, including medical, surgical,
and public health interventions, is crucial to improving outcomes and
reducing the global burden of this preventable disease.

\begin{center}\rule{0.5\linewidth}{0.5pt}\end{center}

\chapter{Infective Endocarditis}\label{infective-endocarditis}

\section{\texorpdfstring{\textbf{Definition}}{Definition}}\label{definition-13}

Infective endocarditis (IE) is an infection of the endocardial surface
of the heart, typically involving one or more heart valves. It can be
caused by bacteria, fungi, or other pathogens, leading to the formation
of vegetation on the heart valves or endocardium. A serious condition
that can result in significant morbidity and mortality without prompt
diagnosis and treatment.

\section{\texorpdfstring{\textbf{Incidence/Prevalence}}{Incidence/Prevalence}}\label{incidenceprevalence-3}

Rare in children, with an estimated incidence of 0.05--0.12 cases per
1,000 pediatric hospital admissions. More common in children with
underlying congenital heart disease (CHD), accounting for up to 80\% of
cases. Increasing prevalence due to improved survival rates of children
with CHD and the use of indwelling central venous catheters. Higher
incidence in children with prosthetic heart valves or those who have
undergone cardiac surgery.

\textbf{Aetiology}

\begin{itemize}
\item
  \textbf{Microorganisms}:

  \begin{itemize}
  \tightlist
  \item
    Bacteria: Most common cause, including \emph{Streptococcus
    viridans}, \emph{Staphylococcus aureus}, and \emph{Enterococcus}
    species.
  \item
    Fungi: Less common, but \emph{Candida} and \emph{Aspergillus} can
    cause IE, particularly in immunocompromised patients
  \end{itemize}
\item
  \textbf{Risk Factors}:

  \begin{itemize}
  \tightlist
  \item
    Congenital heart defects, particularly cyanotic lesions.
  \item
    Prosthetic heart valves.
  \item
    Indwelling devices (e.g., pacemakers, central venous catheters).
  \item
    Rheumatic heart disease (rare in developed countries).
  \item
    Immunosuppression or intravenous drug use (less common in
    pediatrics).
  \end{itemize}
\end{itemize}

\textbf{Pathophysiology}

\begin{itemize}
\tightlist
\item
  Initial endothelial damage due to turbulent blood flow or direct
  trauma (e.g., from catheters).
\item
  Formation of sterile thrombotic vegetations at the site of damage.
\item
  Colonization of vegetations by microorganisms during transient
  bacteremia.
\item
  Vegetations grow, consisting of microorganisms, fibrin, and platelets.
\item
  Can result in local destruction of heart structures, systemic
  embolization, and immune-mediated complications (e.g.,
  glomerulonephritis).
\end{itemize}

\textbf{Signs and Symptoms}

\begin{itemize}
\item
  \textbf{Non-specific symptoms}:

  \begin{itemize}
  \tightlist
  \item
    Fever (most common presenting symptom).
  \item
    Fatigue, malaise, anorexia, weight loss
  \end{itemize}
\item
  \textbf{Cardiac manifestations}:

  \begin{itemize}
  \tightlist
  \item
    New or changing heart murmur.
  \item
    Signs of heart failure (e.g., dyspnea, tachypnea, peripheral edema)
  \end{itemize}
\item
  \textbf{Systemic features}:

  \begin{itemize}
  \tightlist
  \item
    Petechiae, splinter hemorrhages.
  \item
    Osler nodes (painful nodules on fingers/toes).
  \item
    Janeway lesions (painless macules on palms/soles).
  \item
    Roth spots (retinal hemorrhages with central clearing).
  \end{itemize}
\item
  \textbf{Embolic phenomena}:

  \begin{itemize}
  \tightlist
  \item
    Stroke or other neurologic deficits.
  \item
    Splenic or renal infarction.
  \item
    Pulmonary emboli in right-sided IE.
  \end{itemize}
\item
  Symptoms may be less pronounced in children, particularly in chronic
  or subacute presentations.
\end{itemize}

\textbf{Investigations}

\begin{itemize}
\tightlist
\item
  \textbf{Blood cultures}:

  \begin{itemize}
  \tightlist
  \item
    Essential for diagnosis; obtain at least three sets before starting
    antibiotics.
  \item
    May reveal causative organism in \textgreater90\% of cases if
    appropriately timed.
  \end{itemize}
\item
  \textbf{Echocardiography}:

  \begin{itemize}
  \tightlist
  \item
    Transthoracic echocardiography (TTE): Initial investigation;
    non-invasive.
  \item
    Transesophageal echocardiography (TEE): Higher sensitivity,
    especially for prosthetic valves or difficult-to-image cases
  \end{itemize}
\item
  \textbf{Laboratory tests}:

  \begin{itemize}
  \tightlist
  \item
    Full blood count: May show anemia, leukocytosis, or
    thrombocytopenia.
  \item
    Inflammatory markers: Elevated C-reactive protein (CRP) and
    erythrocyte sedimentation rate (ESR)
  \item
    Renal function and urinalysis: May detect embolic phenomena or
    immune-mediated injury.
  \end{itemize}
\item
  \textbf{Imaging}:

  \begin{itemize}
  \tightlist
  \item
    Chest X-ray: Evaluate for heart failure or pulmonary embolism in
    right-sided IE
  \item
    CT or MRI: Assess embolic complications (e.g., stroke, abscess).
  \end{itemize}
\end{itemize}

\textbf{Treatment}

\begin{itemize}
\tightlist
\item
  \textbf{Antimicrobial therapy}:

  \begin{itemize}
  \tightlist
  \item
    Empiric treatment with broad-spectrum antibiotics after blood
    cultures are drawn.
  \item
    Tailor therapy based on culture results and sensitivity testing.
  \item
    Prolonged intravenous antibiotics (typically 4--6 weeks).
  \end{itemize}
\item
  \textbf{Surgical intervention}:

  \begin{itemize}
  \tightlist
  \item
    Indicated for severe valvular damage, heart failure, abscess
    formation, or persistent infection despite antibiotics.
  \item
    Often required for prosthetic valve endocarditis.
  \end{itemize}
\item
  \textbf{Supportive care}:

  \begin{itemize}
  \tightlist
  \item
    Management of heart failure or other complications.
  \item
    Anticoagulation generally avoided due to risk of embolization from
    vegetations
  \end{itemize}
\end{itemize}

\textbf{Complications}

\begin{itemize}
\tightlist
\item
  \textbf{Cardiac complications}:

  \begin{itemize}
  \tightlist
  \item
    Valvular dysfunction (regurgitation or stenosis).
  \item
    Heart failure.
  \item
    Perivalvular abscess.
  \item
    Conduction disturbances (e.g., heart block).
  \end{itemize}
\item
  \textbf{Embolic events}:

  \begin{itemize}
  \tightlist
  \item
    Stroke, myocardial infarction, or organ infarctions.
  \item
    Septic emboli causing abscesses.
  \end{itemize}
\item
  \textbf{Systemic complications}:

  \begin{itemize}
  \tightlist
  \item
    Immune complex-mediated glomerulonephritis.
  \item
    Disseminated infection or sepsis
  \end{itemize}
\item
  \textbf{Prosthetic valve complications}:

  \begin{itemize}
  \tightlist
  \item
    Dehiscence or dysfunction requiring reoperation
  \end{itemize}
\end{itemize}

\textbf{Prognosis}

\begin{itemize}
\tightlist
\item
  Depends on the underlying cause, diagnosis timeliness, and treatment
  appropriateness.
\item
  Mortality rates in children range from 10\% to 25\%, higher in fungal
  infections or prosthetic valve IE.
\item
  Early surgical intervention improves outcomes in high-risk cases.
\item
  Long-term follow-up necessary for valve function and detection of late
  complications.
\end{itemize}

\textbf{Differential Diagnosis}

\begin{itemize}
\tightlist
\item
  \textbf{Non-infective causes of endocarditis-like features}:

  \begin{itemize}
  \tightlist
  \item
    Non-bacterial thrombotic endocarditis (marantic endocarditis).
  \item
    Libman-Sacks endocarditis (associated with systemic lupus
    erythematosus)
  \end{itemize}
\item
  \textbf{Conditions with overlapping symptoms}:

  \begin{itemize}
  \tightlist
  \item
    Rheumatic fever.
  \item
    Systemic vasculitis (e.g., Kawasaki disease, polyarteritis nodosa).
  \item
    Malignancy (e.g., leukemia).
  \item
    Infectious diseases (e.g., osteomyelitis, septic arthritis,
    tuberculosis)
  \end{itemize}
\item
  \textbf{Other cardiac conditions}:

  \begin{itemize}
  \tightlist
  \item
    Myocarditis.
  \item
    Pericarditis.
  \item
    Congenital heart disease exacerbations.
  \end{itemize}
\end{itemize}

This outline provides a structured approach for understanding infective
endocarditis in children and serves as a foundation for deeper study.
Let me know if you need any sections expanded or clarified.

\chapter{Endomyocardial Fibrosis}\label{endomyocardial-fibrosis}

\section{\texorpdfstring{\textbf{Definition}}{Definition}}\label{definition-14}

Endomyocardial fibrosis (EMF) is a progressive, restrictive
cardiomyopathy characterized by fibrotic thickening of the endocardium,
predominantly affecting the inflow tracts of the right and/or left
ventricles. This leads to impaired diastolic filling, atrioventricular
valve regurgitation, and ultimately heart failure. EMF is most commonly
seen in tropical and subtropical regions, and it remains an important
cause of pediatric heart failure in these areas.

\section{\texorpdfstring{\textbf{Incidence/Prevalence}}{Incidence/Prevalence}}\label{incidenceprevalence-4}

\begin{itemize}
\tightlist
\item
  EMF is primarily seen in tropical and subtropical regions,
  particularly in sub-Saharan Africa, India, and parts of South America.
\item
  It is estimated to affect 10 million people worldwide, with the
  highest burden in children and young adults.
\item
  The prevalence in endemic areas ranges from 10\% to 20\% of all heart
  diseases.
\item
  The condition is more common in socioeconomically disadvantaged
  populations and is associated with malnutrition and infections.
\item
  While rare in developed countries, cases have been reported in
  immigrants from endemic regions.
\end{itemize}

\section{\texorpdfstring{\textbf{Aetiology}}{Aetiology}}\label{aetiology-5}

The exact cause of endomyocardial fibrosis remains unknown, but several
contributing factors have been proposed, including:

\begin{enumerate}
\def\labelenumi{\arabic{enumi}.}
\tightlist
\item
  \textbf{Infectious Causes:}

  \begin{itemize}
  \tightlist
  \item
    Parasitic infections such as \emph{Plasmodium falciparum} (malaria)
    and \emph{Schistosoma} species have been implicated in the
    pathogenesis.
  \item
    Viral infections, including Epstein-Barr virus and Coxsackie virus,
    have also been suggested.
  \end{itemize}
\item
  \textbf{Autoimmune Mechanisms}

  \begin{itemize}
  \tightlist
  \item
    Immune system dysregulation leading to chronic inflammation and
    fibrosis.
  \item
    Presence of eosinophilia in many patients suggests an allergic or
    immune-mediated response.
  \end{itemize}
\item
  \textbf{Nutritional Deficiencies:}

  \begin{itemize}
  \tightlist
  \item
    Chronic malnutrition, specifically deficiencies in magnesium,
    selenium, and protein, may predispose individuals to EMF.
  \item
    Exposure to toxic dietary substances such as cassava, which contains
    cyanogenic glycosides, is considered a potential factor
  \end{itemize}
\item
  \textbf{Genetic Predisposition:}

  \begin{itemize}
  \tightlist
  \item
    Familial clustering has been noted in some endemic areas, suggesting
    a genetic susceptibility to the disease.
  \end{itemize}
\item
  \textbf{Environmental Factors:}

  \begin{itemize}
  \tightlist
  \item
    Living in rural, low-income areas with high exposure to infections
    and dietary toxins.
  \end{itemize}
\end{enumerate}

\textbf{Pathophysiology}

\begin{itemize}
\tightlist
\item
  EMF predominantly affects the ventricular endocardium, leading to
  fibrosis that extends from the apex toward the atrioventricular
  valves.
\item
  \textbf{Right ventricular involvement} is more common than left, but
  both can be affected (biventricular disease).
\item
  Fibrosis results in:

  \begin{enumerate}
  \def\labelenumi{\arabic{enumi}.}
  \tightlist
  \item
    \textbf{Diastolic dysfunction:} The stiff ventricle cannot fill
    adequately, leading to elevated atrial pressures.
  \item
    \textbf{Atrioventricular valve regurgitation:} Fibrosis and
    restriction of valve movement lead to tricuspid or mitral
    regurgitation.
  \item
    \textbf{Thrombus formation:} The fibrotic endocardium is prone to
    thrombus development, which can embolize systemically or to the
    lungs.
  \item
    \textbf{Myocardial dysfunction:} Though the myocardium is often
    spared early in the disease, late-stage fibrosis can affect
    contractility and lead to heart failure.
  \end{enumerate}
\end{itemize}

\textbf{Signs and Symptoms}

The clinical presentation of EMF in children varies depending on the
extent of cardiac involvement and which ventricle is affected.

\textbf{Right Ventricular EMF (Most Common Presentation):}

\begin{itemize}
\tightlist
\item
  Signs of right heart failure:

  \begin{itemize}
  \item
    Hepatomegaly
  \item
    Ascites
  \item
    Peripheral edema
  \item
    Elevated jugular venous pressure
  \end{itemize}
\item
  Fatigue and exercise intolerance
\item
  Right upper quadrant pain due to liver congestion
\end{itemize}

\textbf{Left Ventricular EMF:}

\begin{itemize}
\tightlist
\item
  Signs of left heart failure:

  \begin{itemize}
  \tightlist
  \item
    Pulmonary congestion (dyspnea, orthopnea)
  \item
    Cough, hemoptysis (in advanced cases)
  \item
    Fatigue and poor growth in children
  \end{itemize}
\item
  Systemic embolization (e.g., stroke) from left-sided thrombus
  formation
\end{itemize}

\textbf{Biventricular Disease:}

\begin{itemize}
\tightlist
\item
  Severe heart failure with a combination of right- and left-sided
  symptoms
\item
  Anasarca (generalized edema)
\item
  Reduced cardiac output leading to shock in advanced case
\end{itemize}

\textbf{General Symptoms:}

\begin{itemize}
\tightlist
\item
  Failure to thrive
\item
  Recurrent respiratory infections
\item
  Cyanosis (in severe cases)
\end{itemize}

\textbf{Investigations}

\textbf{1. Blood Tests:}

\begin{itemize}
\tightlist
\item
  Eosinophilia: Found in a subset of patients.
\item
  Elevated inflammatory markers (CRP, ESR).
\item
  Liver function tests: Abnormal in cases with severe right heart
  failure.
\item
  Pro-BNP: Elevated in cases of heart failure.
\end{itemize}

\textbf{2. Electrocardiogram (ECG):}

\begin{itemize}
\tightlist
\item
  Low voltage QRS complexes.
\item
  Right or left atrial enlargement.
\item
  Conduction abnormalities (e.g., atrioventricular block)
\end{itemize}

\textbf{3. Echocardiography (Key Diagnostic Tool):}

\begin{itemize}
\tightlist
\item
  Thickened endocardium, especially in the apical region.
\item
  Atrioventricular valve regurgitation.
\item
  Restricted ventricular filling pattern.
\item
  Intracardiac thrombus formation.
\item
  Diastolic dysfunction with preserved systolic function in early
  stages.
\end{itemize}

\textbf{4. Cardiac MRI:}

\begin{itemize}
\tightlist
\item
  Provides detailed imaging of fibrotic areas.
\item
  Helps differentiate EMF from other restrictive cardiomyopathies.
\end{itemize}

\textbf{5. Cardiac Catheterization:}

\begin{itemize}
\tightlist
\item
  Confirms restrictive physiology with elevated end-diastolic pressure
\end{itemize}

\textbf{6. Endomyocardial Biopsy:}

\begin{itemize}
\tightlist
\item
  Rarely performed but can confirm fibrosis histologically.
\end{itemize}

\textbf{Treatment}

Treatment of endomyocardial fibrosis in children is primarily supportive
and aimed at symptom relief.

\textbf{1. Medical Management:}

\begin{itemize}
\tightlist
\item
  \textbf{Diuretics:} Reduce fluid overload and symptoms of heart
  failure.
\item
  \textbf{Anticoagulation:} Indicated for patients with atrial
  fibrillation or intracardiac thrombi.
\item
  \textbf{ACE inhibitors/ARBs:} Help reduce afterload and improve heart
  function.
\item
  \textbf{Nutritional support:} Address malnutrition with appropriate
  supplementation.
\end{itemize}

\textbf{2. Surgical Management:}

\begin{itemize}
\tightlist
\item
  Endocardial resection and valve repair or replacement in selected
  cases.
\item
  High surgical risk with variable outcomes in children.
\end{itemize}

\textbf{3. Symptomatic Care:}

\begin{itemize}
\tightlist
\item
  Management of complications such as arrhythmias and infections.
\item
  Regular follow-up for disease progression and heart failure
  management.
\end{itemize}

\textbf{Complications}

\begin{itemize}
\tightlist
\item
  \textbf{Heart Failure:} Progressive and refractory to medical therapy.
\item
  \textbf{Thromboembolism:} Stroke, mesenteric ischemia, or pulmonary
  embolism.
\item
  \textbf{Arrhythmias:} Atrial fibrillation or heart block leading to
  sudden cardiac death.
\item
  \textbf{Growth retardation:} Due to chronic illness and malnutrition.
\item
  \textbf{Infective endocarditis:} Due to damaged endocardial surfaces.
\end{itemize}

\textbf{Prognosis}

\begin{itemize}
\tightlist
\item
  EMF is a chronic, progressive condition with a poor long-term
  prognosis.
\item
  Early diagnosis and medical management can improve quality of life.
\item
  In children, the prognosis is worse if diagnosed late or if
  biventricular involvement is present.
\item
  Surgical intervention provides limited benefit and carries high
  perioperative risks.
\end{itemize}

\textbf{Differential Diagnosis}

\begin{itemize}
\tightlist
\item
  \textbf{Restrictive Cardiomyopathy:} Similar presentation but without
  endocardial fibrosis on imaging.
\item
  \textbf{Constrictive Pericarditis:} Presents with similar right heart
  failure symptoms but is distinguished by pericardial thickening on
  imaging.
\item
  \textbf{Rheumatic Heart Disease:} Can cause valvular regurgitation and
  heart failure but lacks endocardial thickening.
\item
  \textbf{Hypereosinophilic Syndrome:} Can mimic EMF but includes
  systemic involvement (e.g., skin, lungs).
\end{itemize}

\begin{figure}

\centering{

\pandocbounded{\includegraphics[keepaspectratio]{images/cvs-emf.jpg}}

}

\caption{\label{fig-emf}Endomyocardial Fibrosis showing classical
Egg-on-Stick appearance}

\end{figure}%

\chapter{Miscellaneous Cardiac
Conditions}\label{miscellaneous-cardiac-conditions}

\section{\texorpdfstring{\textbf{Introduction}}{Introduction}}\label{introduction-19}

Beyond the commonly discussed congenital and acquired heart diseases
such as septal defects, ductal anomalies, coarctation, and rheumatic
disease, children may present with a variety of other cardiac conditions
that, although less frequent, are clinically significant. These
disorders encompass abnormalities of rhythm, cardiomyopathies,
pericardial disease, pulmonary hypertension, and conditions secondary to
systemic illness.

In the Ghanaian context, where diagnostic resources are limited and late
presentations are common, awareness of these miscellaneous cardiac
conditions is vital for timely recognition, appropriate referral, and
improved outcomes. This chapter explores these diverse entities,
emphasising their pathophysiology, clinical features, diagnostic
approach, and management principles.

\section{\texorpdfstring{\textbf{Arrhythmias in
Children}}{Arrhythmias in Children}}\label{arrhythmias-in-children}

\subsection{\texorpdfstring{\textbf{Overview}}{Overview}}\label{overview}

Arrhythmias refer to disturbances in the heart's rhythm, either too
slow, too fast, or irregular. They can occur in structurally normal
hearts or as complications of congenital heart disease, myocarditis, or
postoperative states.

\subsection{\texorpdfstring{\textbf{Common
Types}}{Common Types}}\label{common-types}

\begin{itemize}
\tightlist
\item
  \textbf{Sinus bradycardia:} Often physiological in athletes or during
  sleep, but may occur with raised intracranial pressure or
  hypothyroidism.
\item
  \textbf{Sinus tachycardia:} Commonly secondary to fever, anaemia, or
  dehydration.
\item
  \textbf{Supraventricular tachycardia (SVT):} The most common
  pathological tachyarrhythmia in children, often due to re-entry
  mechanisms.
\item
  \textbf{Ventricular tachycardia (VT):} Rare but life-threatening, seen
  in myocarditis or cardiomyopathy.
\item
  \textbf{Heart block:} May be congenital or secondary to maternal
  lupus, cardiac surgery, or myocarditis.
\end{itemize}

\subsection{\texorpdfstring{\textbf{Clinical
Features}}{Clinical Features}}\label{clinical-features-10}

\begin{itemize}
\tightlist
\item
  Palpitations, dizziness, syncope, or chest discomfort
\item
  Cyanosis or heart failure in sustained tachyarrhythmia
\item
  Irregular pulse or variable heart rate on auscultation
\end{itemize}

\subsection{\texorpdfstring{\textbf{Diagnosis and
Management}}{Diagnosis and Management}}\label{diagnosis-and-management}

Diagnosis is made via \textbf{ECG}, \textbf{Holter monitoring}, or
\textbf{event recorders}.\\
Management includes:

\begin{itemize}
\tightlist
\item
  \textbf{Vagal manoeuvres} and \textbf{adenosine} for SVT
\item
  \textbf{Amiodarone} or \textbf{procainamide} for VT
\item
  \textbf{Pacemaker insertion} for complete heart block
\item
  Long-term follow-up with paediatric cardiology is essential.
\end{itemize}

\section{\texorpdfstring{\textbf{Myocarditis}}{Myocarditis}}\label{myocarditis}

\subsection{\texorpdfstring{\textbf{Definition and
Aetiology}}{Definition and Aetiology}}\label{definition-and-aetiology}

Myocarditis is inflammation of the myocardium that impairs
contractility. It may be viral, bacterial, autoimmune, or toxin-induced.

\textbf{Common causes include:}

\begin{itemize}
\tightlist
\item
  \textbf{Viral:} Coxsackie B, adenovirus, enterovirus, parvovirus B19
\item
  \textbf{Bacterial:} Diphtheria, Staphylococcus, Mycoplasma
\item
  \textbf{Others:} Kawasaki disease, autoimmune disorders, drug
  reactions
\end{itemize}

\subsection{\texorpdfstring{\textbf{Pathophysiology}}{Pathophysiology}}\label{pathophysiology-12}

Infectious agents cause direct myocyte injury or immune-mediated
destruction, leading to myocardial oedema, necrosis, and fibrosis. This
results in reduced systolic function and may progress to \textbf{dilated
cardiomyopathy}.

\subsection{\texorpdfstring{\textbf{Clinical
Features}}{Clinical Features}}\label{clinical-features-11}

\begin{itemize}
\tightlist
\item
  Fatigue, feeding difficulties, dyspnoea
\item
  Tachycardia disproportionate to fever
\item
  Gallop rhythm, hepatomegaly, or heart failure signs
\item
  In severe cases, \textbf{cardiogenic shock}
\end{itemize}

\subsection{\texorpdfstring{\textbf{Diagnosis}}{Diagnosis}}\label{diagnosis-7}

\begin{itemize}
\tightlist
\item
  \textbf{Elevated cardiac enzymes (CK-MB, troponin)}
\item
  \textbf{ECG:} ST-T changes, arrhythmias
\item
  \textbf{Echocardiogram:} Global hypokinesia, chamber dilation, reduced
  ejection fraction
\item
  \textbf{Viral studies} where available
\end{itemize}

\subsection{\texorpdfstring{\textbf{Management}}{Management}}\label{management-11}

\begin{itemize}
\tightlist
\item
  Supportive: oxygen, diuretics, inotropes
\item
  \textbf{Avoid excessive fluid loading.}
\item
  \textbf{IV immunoglobulin (IVIG)} or \textbf{steroids} may be
  considered in selected cases.
\item
  Long-term follow-up for ventricular function recovery
\end{itemize}

\section{\texorpdfstring{\textbf{Cardiomyopathies}}{Cardiomyopathies}}\label{cardiomyopathies}

Cardiomyopathies are diseases of the heart muscle not explained by
abnormal loading or coronary artery disease. They are classified based
on ventricular morphology and function.

\subsection{\texorpdfstring{\textbf{Dilated Cardiomyopathy
(DCM)}}{Dilated Cardiomyopathy (DCM)}}\label{dilated-cardiomyopathy-dcm}

\begin{itemize}
\tightlist
\item
  Most common in children; can follow viral myocarditis, genetic
  mutations, or nutritional deficiencies (e.g., selenium deficiency).
\item
  \textbf{Pathophysiology:} Progressive ventricular dilation and
  systolic dysfunction.
\item
  \textbf{Clinical features:} Fatigue, failure to thrive, dyspnoea,
  hepatomegaly, and heart failure.
\item
  \textbf{Management:} Standard heart failure therapy: ACE inhibitors,
  beta-blockers, diuretics, and occasionally anticoagulation.
\item
  \textbf{Prognosis:} Variable; some recover, others progress to chronic
  failure or require transplantation.
\end{itemize}

\subsection{\texorpdfstring{\textbf{Hypertrophic Cardiomyopathy
(HCM)}}{Hypertrophic Cardiomyopathy (HCM)}}\label{hypertrophic-cardiomyopathy-hcm}

\begin{itemize}
\tightlist
\item
  A genetic disorder characterized by asymmetric septal hypertrophy and
  diastolic dysfunction.
\item
  \textbf{May present with:} Syncope, exertional dyspnoea, or sudden
  cardiac death, particularly during exercise.
\item
  \textbf{Diagnosis:} ECG showing LV hypertrophy; echocardiography
  reveals a thickened septum and a small LV cavity.
\item
  \textbf{Management:} Beta-blockers or calcium channel blockers; avoid
  dehydration and strenuous activity.
\end{itemize}

\subsection{\texorpdfstring{\textbf{Restrictive Cardiomyopathy
(RCM)}}{Restrictive Cardiomyopathy (RCM)}}\label{restrictive-cardiomyopathy-rcm}

\begin{itemize}
\tightlist
\item
  Characterized by impaired ventricular filling with normal systolic
  function.
\item
  Rare in children, sometimes secondary to infiltrative diseases.
\item
  \textbf{Features:} Hepatomegaly, ascites, elevated jugular venous
  pressure.
\item
  \textbf{Management:} Diuretics for congestion; poor prognosis without
  transplant.
\end{itemize}

\section{\texorpdfstring{\textbf{Pericardial
Diseases}}{Pericardial Diseases}}\label{pericardial-diseases}

\subsection{\texorpdfstring{\textbf{Pericarditis}}{Pericarditis}}\label{pericarditis}

Inflammation of the pericardium often secondary to viral infection,
rheumatologic disease, or post-surgical states.

\textbf{Clinical features:}

\begin{itemize}
\tightlist
\item
  Sharp chest pain relieved by sitting forward
\item
  Pericardial rub on auscultation
\item
  Low-grade fever
\end{itemize}

\textbf{Diagnosis:} ECG showing diffuse ST elevation; echocardiogram may
reveal effusion.\\
\textbf{Management:} NSAIDs, rest, and treatment of underlying
infection.

\subsection{\texorpdfstring{\textbf{Pericardial Effusion and Cardiac
Tamponade}}{Pericardial Effusion and Cardiac Tamponade}}\label{pericardial-effusion-and-cardiac-tamponade}

Fluid accumulation within the pericardial sac impedes cardiac filling,
leading to tamponade.\\
\textbf{Causes:} Tuberculosis (common in Ghana), malignancy, uremia,
trauma.\\
\textbf{Clinical features:} Dyspnoea, tachycardia, muffled heart sounds,
distended neck veins, pulsus paradoxus.\\
\textbf{Management:} Pericardiocentesis (urgent in tamponade),
antituberculous therapy if indicated.

\section{\texorpdfstring{\textbf{Pulmonary Hypertension
(PH)}}{Pulmonary Hypertension (PH)}}\label{pulmonary-hypertension-ph}

\subsection{\texorpdfstring{\textbf{Definition and
Classification}}{Definition and Classification}}\label{definition-and-classification}

Pulmonary hypertension is a sustained elevation of pulmonary artery
pressure \textgreater25 mmHg at rest. It may be:

\begin{itemize}
\tightlist
\item
  \textbf{Primary (idiopathic)}: rare in children
\item
  \textbf{Secondary}: Due to chronic hypoxia, congenital heart disease,
  or pulmonary disorders
\end{itemize}

\subsection{\texorpdfstring{\textbf{Pathophysiology}}{Pathophysiology}}\label{pathophysiology-13}

Chronic elevation of pulmonary vascular resistance leads to right
ventricular hypertrophy and eventual right heart failure.

\subsection{\texorpdfstring{\textbf{Clinical
Features}}{Clinical Features}}\label{clinical-features-12}

\begin{itemize}
\tightlist
\item
  Exertional dyspnoea, fatigue, syncope
\item
  Loud pulmonary component of the second heart sound
\item
  Right ventricular heave and signs of failure
\end{itemize}

\subsection{\texorpdfstring{\textbf{Diagnosis}}{Diagnosis}}\label{diagnosis-8}

\begin{itemize}
\tightlist
\item
  \textbf{Echocardiography:} Estimates pulmonary pressures
\item
  \textbf{Cardiac catheterization:} Gold standard
\item
  \textbf{CXR:} Enlarged pulmonary arteries
\item
  \textbf{ECG:} Right axis deviation, RV hypertrophy
\end{itemize}

\subsection{\texorpdfstring{\textbf{Management}}{Management}}\label{management-12}

\begin{itemize}
\tightlist
\item
  Treat underlying cause (e.g., repair of shunt lesions)
\item
  \textbf{Oxygen therapy} for hypoxia
\item
  \textbf{Vasodilators:} Sildenafil, calcium channel blockers
\item
  \textbf{Anticoagulation} in selected patients
\item
  \textbf{Avoid dehydration and high altitude} exposure
\end{itemize}

Prognosis depends on the cause and reversibility of vascular changes.

\section{\texorpdfstring{\textbf{Kawasaki
Disease}}{Kawasaki Disease}}\label{kawasaki-disease}

An acute, self-limiting vasculitis of childhood, predominantly affecting
the coronary arteries. Most common in children under 5 years.

\subsection{\texorpdfstring{\textbf{Clinical
Features}}{Clinical Features}}\label{clinical-features-13}

\begin{itemize}
\tightlist
\item
  Persistent high fever \textgreater5 days
\item
  Conjunctival injection, strawberry tongue, cracked lips
\item
  Rash, cervical lymphadenopathy
\item
  Desquamation of fingers and toes
\item
  Coronary artery aneurysms may develop in untreated cases.
\end{itemize}

\subsection{\texorpdfstring{\textbf{Diagnosis and
Management}}{Diagnosis and Management}}\label{diagnosis-and-management-1}

Clinical diagnosis: Elevated ESR, CRP, and platelets support
inflammation.\\
\textbf{Echocardiogram} to detect coronary aneurysms.

Treatment includes:

\begin{itemize}
\tightlist
\item
  \textbf{IVIG (2 g/kg single dose)} within 10 days of onset
\item
  \textbf{High-dose aspirin} during acute phase, then low-dose for 6--8
  weeks
\item
  Long-term cardiology follow-up for aneurysm surveillance
\end{itemize}

\section{\texorpdfstring{\textbf{Cardiac
Tumours}}{Cardiac Tumours}}\label{cardiac-tumours}

\subsection{\texorpdfstring{\textbf{Types}}{Types}}\label{types}

\begin{itemize}
\tightlist
\item
  \textbf{Rhabdomyoma:} Most common; often associated with
  \textbf{tuberous sclerosis}
\item
  \textbf{Fibroma, Teratoma, Myxoma, Hemangioma:} Less common
\end{itemize}

\subsection{\texorpdfstring{\textbf{Clinical
Manifestations}}{Clinical Manifestations}}\label{clinical-manifestations}

\begin{itemize}
\tightlist
\item
  Obstructive symptoms (outflow tract obstruction)
\item
  Arrhythmias
\item
  Heart failure or sudden death
\end{itemize}

\textbf{Diagnosis:} Echocardiography reveals intracardiac mass; MRI
provides further characterization.\\
\textbf{Management:} Rhabdomyomas often regress; others may require
surgical excision.

\section{\texorpdfstring{\textbf{Systemic Diseases with Cardiac
Involvement}}{Systemic Diseases with Cardiac Involvement}}\label{systemic-diseases-with-cardiac-involvement}

Certain systemic illnesses have important cardiac manifestations.

\subsection{\texorpdfstring{\textbf{Anaemia and
Malnutrition}}{Anaemia and Malnutrition}}\label{anaemia-and-malnutrition}

Chronic anaemia leads to \textbf{high-output cardiac failure}, while
severe malnutrition can cause cardiac atrophy and reduced contractility.

\subsection{\texorpdfstring{\textbf{Sickle Cell
Disease}}{Sickle Cell Disease}}\label{sickle-cell-disease}

Recurrent anaemia, iron overload, and pulmonary hypertension contribute
to \textbf{cardiomegaly} and \textbf{diastolic dysfunction}.

\subsection{\texorpdfstring{\textbf{Sepsis}}{Sepsis}}\label{sepsis}

Severe sepsis may cause \textbf{myocardial depression} secondary to
cytokine release, which is reversible with recovery.

\subsection{\texorpdfstring{\textbf{Thyroid
Disorders}}{Thyroid Disorders}}\label{thyroid-disorders}

\begin{itemize}
\tightlist
\item
  \textbf{Hyperthyroidism:} Tachyarrhythmias and high-output failure
\item
  \textbf{Hypothyroidism:} Bradycardia and pericardial effusion
\end{itemize}

\subsection{\texorpdfstring{\textbf{HIV-Associated Cardiac
Disease}}{HIV-Associated Cardiac Disease}}\label{hiv-associated-cardiac-disease}

In resource-limited settings like Ghana, HIV-infected children may
develop:

\begin{itemize}
\tightlist
\item
  \textbf{Dilated cardiomyopathy}
\item
  \textbf{Pericardial effusion} (often tuberculous)
\item
  \textbf{Pulmonary hypertension}
\item
  \textbf{Drug-related cardiotoxicity}
\end{itemize}

Early ART initiation and cardiac monitoring are essential to reduce
morbidity.

\section{\texorpdfstring{\textbf{Drug-Induced
Cardiotoxicity}}{Drug-Induced Cardiotoxicity}}\label{drug-induced-cardiotoxicity}

Several medications used in paediatrics can cause cardiac dysfunction.

\textbf{Examples:}

\begin{itemize}
\tightlist
\item
  \textbf{Anthracyclines (e.g., doxorubicin):} Dose-dependent
  cardiomyopathy
\item
  \textbf{Cyclophosphamide:} Myocardial necrosis
\item
  \textbf{Antimalarials (chloroquine):} QT prolongation
\end{itemize}

\textbf{Prevention:} Regular ECG, echocardiography, and adherence to
cumulative dose limits.

\section{\texorpdfstring{\textbf{Prognosis and Long-Term
Care}}{Prognosis and Long-Term Care}}\label{prognosis-and-long-term-care}

The outcome of miscellaneous paediatric cardiac conditions varies
widely. Transient viral myocarditis may resolve completely, while
genetic cardiomyopathies often progress despite optimal therapy. Chronic
conditions such as pulmonary hypertension and postoperative arrhythmias
require ongoing multidisciplinary follow-up.

Early detection through echocardiography, routine screening in high-risk
children, and integration of paediatric cardiology into tertiary care
systems in Ghana are crucial to improving survival and quality of life.

\section{\texorpdfstring{\textbf{Conclusion}}{Conclusion}}\label{conclusion-15}

Miscellaneous paediatric cardiac conditions represent a heterogeneous
group that collectively contribute significantly to cardiac morbidity in
children. While individually less frequent than septal defects or
rheumatic disease, they often carry substantial diagnostic and
therapeutic challenges. For medical students and practitioners in Ghana,
a high index of suspicion, thorough clinical evaluation, and appropriate
use of echocardiography can facilitate early recognition and management.
Strengthening diagnostic infrastructure and ensuring access to
paediatric cardiology services remain essential priorities in improving
outcomes for these children.

\part{{Infectious Diseases}}

\chapter{Enteric Fever}\label{enteric-fever}

Typhoid and Paratyphoid Fevers

\hfill\break

\section{Definition}\label{definition-15}

Typhoid fever is a life-threatening infection caused by the bacterium
\emph{Salmonella Typh}i. It is usually spread through contaminated food
or water. Once Salmonella Typhi bacteria are ingested, they multiply and
spread into the bloodstream. It causes an acute generalised infection of
the reticuloendothelial system, intestinal lymphoid tissue, and the gall
bladder.

\section{Incidence/prevalence}\label{incidenceprevalence-5}

As of 2019 estimates, there were 9 million cases of typhoid fever
annually, resulting in about 110,000 deaths annually.(WHO 2023) In 2023,
information from LHIMS, Komfo Anokye Teaching Hospital, Kumasi indicated
a rate of 0.4\% or 4/1000 admissions through the Paediatric Emergency
Unit (PEU) were diagnosed as enteric fever.

\section{Aetiology}\label{aetiology-6}

An infectious feverish disease caused by the bacterium Salmonella typhi
(Salmonella enterica Serovar Typhi) and less commonly by Salmonella
paratyphi.

\section{Pathogenesis}\label{pathogenesis-1}

S. typhi and S. paratyphi are transmitted through ingestion of fecally
contaminated food or water, improper hygiene, and unsafe food/water
handling practices. Individual-level risk factors include contaminated
water supply, patronizing food vendors, ingestion of raw fruits and
vegetables and a history of contact with a case or a chronic carrier.
The risk of environmental transmission of typhoid fever is higher in the
rainy season, proximity to open sewers and highly contaminated water
bodies and residing in areas of low elevation.(Adesegun et al. 2020)
Ingested organisms survive exposure to gastric acid before gaining
access to the small bowel, where they penetrate the epithelium, enter
the lymphoid tissue, and disseminate via the lymphatic or hematogenous
route. A chronic carrier state is established in an estimated 1 to 5 per
cent of cases.(Andrews and Charles 2023)

\section{Signs and symptoms}\label{signs-and-symptoms-6}

The incubation period ranges from 7-14 days on average but can range
from 3 days to two months. Symptoms include prolonged high fever,
fatigue, headache, nausea, abdominal pain, constipation or diarrhoea,
and in some cases a rash. Typhoid can affect every system of the body.
Other manifestations include drowsiness, seizures, coma, psychosis,
meningitis, acute renal failure, osteomyelitis, and septic arthritis.
Severe cases may lead to serious complications including terminal ileal
perforation or even death.

\section{Investigations}\label{investigations-11}

In the first and second weeks of the presentation, blood culture and
sensitivity are recommended. Stool culture is also relevant more in the
first week as compared to the second week. Bone marrow aspirate and
culture are important after the second week. Recently, antibodies (IgM)
have been used as a diagnostic tool. Local studies are needed to
validate these antibody tests. Depending on the system involved, other
tests must be requested.

\section{Treatment}\label{treatment-8}

Treatment is supportive (Antipyretics, hydration, nutrition,
transfusion) and specific (antibiotics are given starting with the
empiric regimen: Third generation cephalosporin or quinolone eg.
Ciprofloxacin). The choice of antibiotics should be changed to a
narrower spectrum when culture and sensitivity results are available.
Where there is poor response attributed to a focus e.g.~abscess
formation, source control should be pursued. Safe water, sanitation, and
hygiene (WASH) interventions are critical to preventing the spread of
typhoid. Typhoid is spread via the faecal-oral route when bacteria pass
into people's mouths through food, water, hands, or objects contaminated
with faecal matter. Solutions such as water treatment or filtration,
installation and management of toilets and sanitation systems, and
education about proper handwashing and food-handling practices can save
lives and protect people from typhoid infection. Three types of typhoid
vaccines of demonstrated safety and efficacy are available on the
international market:-

\begin{enumerate}
\def\labelenumi{\arabic{enumi}.}
\tightlist
\item
  A conjugated vaccine in which the Vi polysaccharide vaccine is bound
  to a carrier protein,
\item
  A non-conjugated Vi polysaccharide vaccine, and
\item
  A live attenuated Ty21a vaccine.
\end{enumerate}

\section{Complications}\label{complications-11}

These include anicteric hepatitis, bone marrow suppression, paralytic
ileus, myocarditis, psychosis, cholecystitis, osteomyelitis,
peritonitis, pneumonia, haemolysis, and syndrome of inappropriate
release of antidiuretic hormone (SIADH)

\section{Prognosis}\label{prognosis-10}

The prognosis among persons with typhoid fever depends primarily on the
speed of diagnosis and initiation of correct treatment. Generally,
untreated typhoid fever carries a mortality rate of 15\%-30\%. In
properly treated diseases, the mortality rate is less than 1\%.
(emedicine 2024)

\section{Differential diagnosis}\label{differential-diagnosis-9}

Differential diagnosis will depend on the types of presentation. The
most common are malaria, liver abscesses, tuberculosis, and meningitis.

\section{Sample questions}\label{sample-questions}

\begin{enumerate}
\def\labelenumi{\arabic{enumi}.}
\item
  A 5-year-old boy complained of general body weakness, abdominal pain
  and a fever of two weeks duration. He had 2 courses of antimalarial
  treatment even though the RDT was negative. On examination, he was
  lethargic and had a body temperature of 39.9\textsuperscript{o}C. If
  you suspect enteric fever, what will be the best test to perform?

  \begin{enumerate}
  \def\labelenumii{\alph{enumii}.}
  \item
    Urine culture
  \item
    Stool culture
  \item
    Blood culture
  \item
    Widal test

    \emph{All the options are feasible but with the duration of illness,
    blood culture with sensitivity testing will provide the best yield.
    Widal tests are widely used in some facilities but have a high
    tendency of false positive results}.
  \end{enumerate}
\item
  A 10-year-old known sickle cell disease patient genotype SS presented
  with severe pain in the right leg of 3-week duration. She is on her
  routine medications but has yet to be initiated on hydroxyurea. On
  examination, there was tenderness in the right leg, especially at the
  knee joint with evidence of inflammation. Blood culture isolated
  Salmonella typhi. What is your best management approach?

  \emph{A more detailed history and examination is warranted. Before the
  blood culture results came out, the child would have been on empiric
  antibiotics. This must be changed to a narrower spectrum based on the
  sensitivity results. Remember to request for ultrasound of the
  inflamed knee for possible effusion. If there is fluid collection,
  this must be drained to achieve source control. This will optimize
  antibiotic response.}
\end{enumerate}

\section{Practice question}\label{practice-question}

Concerning question 2 (above) if the child was started on ceftriaxone,
discuss what must be done after microbiology provides the sensitivity
results as shown in the table below.

\begin{longtable}[]{@{}ll@{}}
\toprule\noalign{}
Sensitive & Resistant \\
\midrule\noalign{}
\endhead
\bottomrule\noalign{}
\endlastfoot
Meropenem Ciprofloxacin Amikacin & Ceftriaxone, Linezolid \\
\end{longtable}

\chapter{HIV}\label{hiv}

\section{Definition}\label{definition-16}

Human immunodeficiency virus (HIV) is an infection that attacks the
body's immune system, specifically the white blood cells called CD4
cells. HIV destroys these CD4 cells, weakening a person's immunity
against opportunistic infections, such as tuberculosis, fungal
infections, severe bacterial infections, and some cancers.(World Health
Organization (WHO) 2023)

\section{Incidence/prevalence}\label{incidenceprevalence-6}

Globally, 39 million people were living with HIV in 2022 out of which
1.5 million were children below 15 years. In Ghana, out of 354,927
people living with HIV in 2022, 7\% (24,845) were children below 15
years. There were 16,574 new HIV infections in 2022 out of which 17\%
(2,818) were children. Mother-to-child transmission of HIV at 6 weeks in
2022 was 9.12\% while the final MTCT rate at 18 months was 17.75\%. In
2022, there were 21,439 adolescents (10-19 years) living with HIV out of
which 1,791 were newly infected.(Ghana Health Service 2023)

\section{Aetiology}\label{aetiology-7}

HIV is a retrovirus with two main subtypes namely HIV 1 and HIV 2. HIV-1
is the most common type of HIV and accounts for 99\% of all infections
in Ghana, whereas HIV-2 is relatively uncommon (0.08\%) and less
infectious. Ghanaians co-infected with HIV 1 and HIV 2 form 0.02\% of
total infections. HIV-2 is mainly concentrated in West Africa and the
surrounding countries. HIV-2 is less fatal and progresses more slowly
than HIV-1. Modes of HIV transmission are sexual (80\% in Ghana) mainly
heterosexual but also same-sex; parenteral transmission (5\%) examples
include blood transmission, shared needles, and needle stick accidents,
and mother-to-child transmission (15\%) of which in-utero, intrapartum
and postpartum accounts for 10-25\%, 60-75\%, and 10-20\% respectively.

\section{Pathogenesis}\label{pathogenesis-2}

HIV entry, the first phase of the viral replication cycle, begins with
the adhesion of the virus to the host cell and ends with the fusion of
the cell and viral membranes with subsequent delivery of the viral core
into the cytoplasm. The intricate series of protein-protein interactions
that ultimately result in virus infection can be divided into several
phases, some of which are essential and others that may modulate the
efficiency of the process.(Wilen, Tilton, and Doms 2012)

Infection with HIV starts without symptoms or ill-feeling and is
accompanied by slight changes in the immune system. This stage spans up
to three months after infection until seroconversion where HIV-specific
antibodies can be detected in individuals following recent exposure. The
outcome of infection and duration of disease progression with clinical
symptoms may vary greatly between individuals, but often it progresses
fairly slowly. It takes several years from primary infection to the
development of symptoms of advanced HIV diseases and
immunosuppression.(Cunningham et al. 2000) Although individuals may look
healthy during primary infection, the virus replicates in infected
individuals' lymph nodes and bloodstream. As a result, the immune system
may get slowly damaged by the burst of viral load in their bodies.(Moir,
Chun, and Fauci 2011)

\section{Signs and symptoms}\label{signs-and-symptoms-7}

Signs and symptoms develop when the immune system's ability to fight the
disease is compromised due to viral replication and reduced CD4 cells.

\subsection{Early/ acute HIV}\label{early-acute-hiv}

Acute HIV infection, also known as primary HIV infection, can cause a
range of symptoms that can begin a few days after exposure to the virus
and last for a few days to several months

\subsection{Late chronic HIV}\label{late-chronic-hiv}

The late stage of HIV infection is AIDS (acquired immunodeficiency
syndrome), which occurs when the virus weakens the immune system.

\subsection{Staging of HIV}\label{staging-of-hiv}

\begin{longtable}[]{@{}
  >{\raggedright\arraybackslash}p{(\linewidth - 2\tabcolsep) * \real{0.5000}}
  >{\raggedright\arraybackslash}p{(\linewidth - 2\tabcolsep) * \real{0.5000}}@{}}

\caption{\label{tbl-hiv-staging}HIV Staging}

\tabularnewline

\toprule\noalign{}
\endhead
\bottomrule\noalign{}
\endlastfoot
\multicolumn{2}{@{}>{\raggedright\arraybackslash}p{(\linewidth - 2\tabcolsep) * \real{1.0000} + 2\tabcolsep}@{}}{%
\textbf{{Stage 1}}

{Asymptomatic}

{Generalised lymphadenopathy}} \\
\multicolumn{2}{@{}>{\raggedright\arraybackslash}p{(\linewidth - 2\tabcolsep) * \real{1.0000} + 2\tabcolsep}@{}}{%
\textbf{{Stage 2}}

{Unexplained persistent hepato-splenomegaly}

{Recurrent or chronic upper respiratory tract infections (otitis media,
otorrhoea, sinusitis, tonsillitis)}

{Herpes zoster}

{Linear gingival erythema}

{Recurrent oral ulceration}

{Papular pruritic eruption}

{Fungal nail infections}

{Extensive wart virus infection}

{Extensive Molluscum contangiosum}

{Unexplained persistent parotid enlargement}} \\
\textbf{{Stage 3}}

{Unexplained moderate malnutrition and not adequately responding to
standard therapy}

{Unexplained persistent diarrhoea (14 days or more)}

{Unexplained persistent fever (above 37.5°C, intermittent or constant,
for longer than one 1 month)}

{Persistent oral candidiasis (after first 6 weeks of life)}

{Oral hairy leukoplakia}

{Lymph node tuberculosis} & \textbf{{Stage 3}}

{Pulmonary tuberculosis}

{Severe recurrent bacterial pneumonia}

{Acute necrotising ulcerative gingivitis or}

{periodontitis}

{Unexplained anaemia (\textless8g/dl), neutropenia (\textless0.5x
10\textsuperscript{9}/l) and chronic thrombocytopenia (\textless50 x
10\textsuperscript{9}/l).~}

{Symptomatic lymphoid interstitial pneumonitis}

{Chronic HIV-associated lung disease, including}

{bronchiectasis}

{~} \\
\textbf{{Stage 4}}

{Unexplained severe wasting, stunting or severe malnutrition not
responding to standard therapy}

\emph{{Pneumocystis (jiroveci) pneumonia}}

{Recurrent severe bacterial infections (such as empyema, pyomyositis,
bone or joint infection, and meningitis, but excluding pneumonia)}

{Chronic herpes simplex infection (orolabial or cutaneous of more than 1
month's duration or}

{visceral at any site)}

{Oesophageal candidiasis (or candidiasis of the trachea, bronchi or
lungs)}

{Extrapulmonary tuberculosis}

{Kaposi sarcoma}

{Cytomegalovirus infection (retinitis or infection of other organs with
onset at age more than 1 month)}

{Central nervous system toxoplasmosis (after the neonatal period)} &
\textbf{{Stage 4}}

{HIV encephalopathy}

{Cytomegalovirus infection (retinitis or infection of other organs with
onset at age more than 1 month)}

{Extrapulmonary Cryptococcosis, including meningitis}

{Disseminated non-tuberculous mycobacterial infection}

{Progressive multifocal leukoencephalopathy}

{Chronic cryptosporidiosis (with diarrhoea)}

{Chronic Isosporiasis}

{Disseminated endemic mycosis (Extrapulmonary histoplasmosis,
coccidioidomycosis, penicilliosis)}

{Cerebral or B-cell non-Hodgkin lymphoma}

{HIV-associated nephropathy or cardiomyopathy}{~}

{~} \\

\end{longtable}

\section{Investigations}\label{investigations-12}

Depending on signs and symptoms, a patient could be suspected of HIV
infection and AIDS. Children are then tested for HIV depending on their
age.

\subsection{Newborn to \textless18 months}\label{newborn-to-18-months}

Younger children (\textless{} 18 months) who are exposed to HIV or are
suspected of HIV should be tested at any time they have symptoms of HIV
disease. The preferred test is the DNA PCR which is done using dried
blood spots (DBS) taken from pricking the heel of a child onto filter
paper. If the test is positive, the child should be started on
treatment, but a second confirmatory test must be done. Infants born to
HIV-positive mothers without symptoms are routinely tested within 6
weeks of birth, at 9 months of age using DNA PCR. At 18 months such
children are tested with antibody tests using a ``triple algorithm'' as
detailed below.

\subsection{18 months and above}\label{months-and-above}

Should be tested using the ``triple algorithm'' when a patient is
sequentially tested using first response, Oraquick, and SD biofilm test.
ALL three tests must be positive before a child is confirmed HIV
positive. Figure~\ref{fig-HivTestAlgorithm} below gives further
information

\begin{figure}

\centering{

\includegraphics[width=10.06in,height=11.81in]{id-hiv_files/figure-latex/mermaid-figure-1.png}

}

\caption{\label{fig-HivTestAlgorithm}HIV testing algorithm for
non-pregnant women and general population}

\end{figure}%

Baseline investigations are needed before initiating antiretroviral
treatment/ monitoring of the disease. These are listed in
Table~\ref{tbl-hiv-investigations}

\begin{verbatim}
Warning: package 'tibble' was built under R version 4.5.1
\end{verbatim}

\begin{verbatim}
Warning: package 'purrr' was built under R version 4.5.1
\end{verbatim}

\begin{table}

\caption{\label{tbl-hiv-investigations}Baseline investigations for HIV}

\centering{

\fontsize{10.5pt}{12.6pt}\selectfont
\begin{tabular*}{\linewidth}{@{\extracolsep{\fill}}ll}
\toprule
Test & Types \\ 
\midrule\addlinespace[2.5pt]
{\bfseries Haematological} & Full blood count \\ 
{\bfseries Biochemistry} & Blood Urea Electrolytes and Creatinine \\ 
{\bfseries } & Liver Function tests \\ 
{\bfseries } & Fasting Blood Sugar \\ 
{\bfseries } & Cholesterol and lipid profile \\ 
{\bfseries Routine} & Urinalysis (Urine R/E) \\ 
{\bfseries } & Stool R/E \\ 
{\bfseries Respiratory} & TB screening \\ 
{\bfseries } & GeneXpert \\ 
{\bfseries } & Chest X-ray \\ 
{\bfseries Serological} & Hepatitis B Surface antigen \\ 
{\bfseries Immunological} & CD4 \\ 
{\bfseries Optional (Patient dependent)} & Histology on skin and lymph node biopsy \\ 
{\bfseries } & Kidney biopsy \\ 
{\bfseries } & Screening for STIs \\ 
{\bfseries } & Pap smear, HPV DNA \\ 
{\bfseries } & Abdominal Ultrasound \\ 
\bottomrule
\end{tabular*}

}

\end{table}%

When children are on treatment viral load tests are done regularly. This
is done to detect early virological failure. When highly active
antiretroviral therapy (HAART) is started, the first viral load test is
done at 6 and 12 months. If there is virological suppression, then
monitoring is done yearly. The table below outlines what to do with
various viral load cut-offs and actions to take

\begin{figure}

\centering{

\includegraphics[width=8.41in,height=10.31in]{id-hiv_files/figure-latex/mermaid-figure-2.png}

}

\caption{\label{fig-viral-load-monitoring}Treatment Monitoring
Algorithm}

\end{figure}%

\section{Treatment}\label{treatment-9}

Before commencing HAART for any child or adolescent, counsel the
caregiver and the child if old enough to appreciate the counselling

\subsection{Infants born to HIV-positive mothers (post-exposure
prophylaxis)}\label{infants-born-to-hiv-positive-mothers-post-exposure-prophylaxis}

All newborns of HIV-positive mothers are called HIV-exposed babies.
These babies should be given antiretroviral prophylaxis namely
nevirapine and zidovudine daily starting from birth (within 24 hours)
for 3 months. At week 6, infants should be started on cotrimoxazole
daily till HIV infection is ruled out at 18 months. At any point in time
when a child tests positive for HIV, treatment with cotrimoxazole should
be continued for a longer duration till the child's immune system is
fully re-constituted and appropriate for age. The best marker is
appropriate CD4 for age. In the absence of CD4 testing, the child should
be virologically suppressed with no evidence of clinical
disease.@borges-lujan2022

\subsection{Treatment of HIV
infection}\label{treatment-of-hiv-infection}

Counselling caregivers and children is a must before initiating HAART.
Emphasise should be on HAART being a lifelong treatment. Adherence
counselling should be integrated into clinical care. The HAART consist
of 2 Nucleos(t)ide Reverse Transcriptase Inhibitor (N(t)RTI plus
Non-nucleoside Reverse Transcriptase Inhibitor (NNRTI) or Integrase
Strand Transfer Inhibitor (INSTI) or Protease Inhibitor (PI). HAART
regimen should have a minimum of three ARVs. The table below shows the
types of ARVs available in Ghana

\begin{longtable}[]{@{}
  >{\raggedright\arraybackslash}p{(\linewidth - 8\tabcolsep) * \real{0.2000}}
  >{\raggedright\arraybackslash}p{(\linewidth - 8\tabcolsep) * \real{0.2000}}
  >{\raggedright\arraybackslash}p{(\linewidth - 8\tabcolsep) * \real{0.2000}}
  >{\raggedright\arraybackslash}p{(\linewidth - 8\tabcolsep) * \real{0.2000}}
  >{\raggedright\arraybackslash}p{(\linewidth - 8\tabcolsep) * \real{0.2000}}@{}}

\caption{\label{tbl-oopo}Antiretroviral Groups}

\tabularnewline

\toprule\noalign{}
\endhead
\bottomrule\noalign{}
\endlastfoot
\textbf{{Group 1}} & \textbf{{Group 2}} &
\multicolumn{3}{>{\raggedright\arraybackslash}p{(\linewidth - 8\tabcolsep) * \real{0.6000} + 4\tabcolsep}@{}}{%
\textbf{{Group 3}}} \\
\textbf{{N(t)RTI}} & \textbf{{NRTI}} & \textbf{{NNRTI}} &
\textbf{{INSTI}} & \textbf{{PI}} \\
{Abacavir (ABC)} & {Lamivudine (3TC)} & {Efavirenz (EFV)}

{Weight \textgreater{} 10 kg} & {Dolutegravir (DTG)}

{Weight \textgreater3 kg} & {Ritonavir boosted Lopinavir (LPV/r)} \\
{Zidovudine (AZT)}

{(Hb\textgreater{} 8 g/dl)} & {Emtricitabine (FTC)} & {Nevirapine (NVP)}
& {Raltegravir (RAL)} & {Ritonavir boosted Atazanavir (ATV/r)} \\
{Tenofovir (TDF)}

{(Renal disease)}

{Weight=\textgreater30 kg} &
\multicolumn{3}{>{\raggedright\arraybackslash}p{(\linewidth - 8\tabcolsep) * \real{0.6000} + 4\tabcolsep}}{%
\textbf{{Read around the main side effects/contraindications of each
ARV}}} & {Ritonavir boosted Darunavir (DRV/r)} \\

\end{longtable}

HAART combination can be made easy by choosing one ARV from each group.

\begin{longtable}[]{@{}
  >{\raggedright\arraybackslash}p{(\linewidth - 6\tabcolsep) * \real{0.2500}}
  >{\raggedright\arraybackslash}p{(\linewidth - 6\tabcolsep) * \real{0.2500}}
  >{\raggedright\arraybackslash}p{(\linewidth - 6\tabcolsep) * \real{0.2500}}
  >{\raggedright\arraybackslash}p{(\linewidth - 6\tabcolsep) * \real{0.2500}}@{}}

\caption{\label{tbl-alternative-arv}Preferred and alternative first-line
ART regimens for adolescents, children and neonates}

\tabularnewline

\toprule\noalign{}
\endhead
\bottomrule\noalign{}
\endlastfoot
\textbf{{Populations}} & \textbf{{Preferred first-line}}

\textbf{{regimen}} & \textbf{{Alternative first-line}}

\textbf{{regimen}} & \textbf{{Special circumstances}} \\
{~}

\textbf{{Adolescents~}} & {TDF + 3TC (or FTC) + DTG} & {TDF + 3TC + EFV
400mg} & {TDF + 3TC (or FTC) + EFV600mg}

{AZT + 3TC + EFV 600 mg}

{TDF + 3TC (or FTC) + PI/r}

{TDF + 3TC (or FTC) + RAL}

{ABC + 3TC + DTG}

{TDF + 3TC (or FTC) + PI/r} \\
\textbf{{Children}} & {ABC + 3TC + DTG}

{~}

\textbf{{~}} & {ABC + 3TC + LPV/r}

{TDF+ 3TC (or FTC) + DTG}

{~} & {ABC + 3TC + EFV~}

{ABC + 3TC + RAL}

{AZT + 3TC + EFV~}

{AZT + 3TC + LPV/r (or RAL)} \\
\textbf{{Neonates}} & {AZT (or ABC) + 3TC + RAL(or DTG)} & {AZT + 3TC +
NVP} & {AZT + 3TC + LPV/r} \\

\end{longtable}

\subsection{Side effects of main ARVS}\label{side-effects-of-main-arvs}

Depending on the type of HAART, children may experience different side
effects. Regular clinical and laboratory monitoring will be needed to
identify side effects early. There are alternative ARVs in each group to
substitute if a child on HAART experiences a major side effect.

\begin{longtable}[]{@{}
  >{\raggedright\arraybackslash}p{(\linewidth - 2\tabcolsep) * \real{0.5000}}
  >{\raggedright\arraybackslash}p{(\linewidth - 2\tabcolsep) * \real{0.5000}}@{}}
\caption{Common ARV toxicities}\label{tbl-arv-toxicities}\tabularnewline
\toprule\noalign{}
\endfirsthead
\endhead
\bottomrule\noalign{}
\endlastfoot
\textbf{Haematological toxicity} & Drug-induced bone marrow suppression
is most commonly seen with AZT (anaemia, neutropenia). \\
\textbf{Mitochondrial Dysfunction} & Primarily seen with the NRTI drugs,
including lactic acidosis, hepatic toxicity, pancreatitis, peripheral
neuropathy, lipoatrophy, myopathy \\
\textbf{Renal Toxicity} & Renal tubular dysfunction is associated with
Tenofovir (TDF). ATV/r can also cause nephrolithiasis. \\
\textbf{Other Metabolic Abnormalities} & More common with PIs and
INSTIs. Include hyperlipidaemia, fat accumulation, insulin resistance,
diabetes and osteopenia. Lipodystrophy is also associated with
Zidovudine. The risk of cardiovascular events with Abacavir (ABC) is
still debatable. \\
\textbf{Allergic Reactions} & Skin rashes and hypersensitivity
reactions, are more common with the NNRTI drugs but also seen with
certain NRTI drugs, such as ABC and some PIs. \\
\textbf{Hepatic Toxicity} & Liver enzyme elevation with DTG especially
in patients with HBV or HCV co-infection. DRV/r also causes liver enzyme
elevation \\
\textbf{Muscular Toxicity} & Muscle weakness and sometimes
rhabdomyolysis are seen with RAL \\
\end{longtable}

\section{Complications}\label{complications-12}

The main complication of HIV infection is the progression to AIDS when
HAART is not initiated. Those children on treatment with non-adherence
or poor adherence to HAART can ultimately develop AIDS. This will be the
consequence of reduced CD4 cell count and increased viral load

\section{Prognosis}\label{prognosis-11}

Patients who start treatment early before immune dysgenesis and are
virologically controlled have a life expectancy like an HIV-negative
individual

\section{Differential diagnosis}\label{differential-diagnosis-10}

Acute HIV infection may be asymptomatic or may cause a
mononucleosis-like syndrome. It should be differentiated from similar
diseases that cause fever, fatigue, sore throat, myalgia, and
lymphadenopathy such as acute toxoplasmosis, acute CMV/EBV infections,
and acute viral hepatitis

\section{Further reading}\label{further-reading}

\href{https://ghs.gov.gh/wp-content/uploads/2022/10/Consolidated-HIV-Guidelines.pdf}{Consolidated
Guidelines for HIV care in Ghana}

\section{Sample case scenarios}\label{sample-case-scenarios}

\begin{enumerate}
\def\labelenumi{\arabic{enumi}.}
\tightlist
\item
  A 10-year-old boy presented to your facility with skin rashes, weight
  loss, fever, and cough of 3 months duration. On examination, he was
  semi-conscious. Chest X-ray was suggestive of pulmonary tuberculosis.
  CSF from Lumber puncture was positive on GeneXpert

  \begin{enumerate}
  \def\labelenumii{\alph{enumii}.}
  \tightlist
  \item
    How would you confirm HIV in this child?
  \item
    What is the appropriate WHO clinical staging
  \end{enumerate}
\item
  A 5-year-old boy has recently been diagnosed with HIV. You intend to
  start antiretrovirals. He weighs 25 kg with Hemoglobin of 6 gm/dl.~
  His renal and liver function is normal. Which option will be your best
  HAART?

  \begin{enumerate}
  \def\labelenumii{\alph{enumii}.}
  \tightlist
  \item
    ABC/3TC/EFV
  \item
    ABC/3TC/DTG
  \item
    AZT/3TC/LPV/r
  \item
    TDF/3TC/EFV
  \end{enumerate}
\item
  A 15-year-old male drug addict, newly diagnosed with HIV. He weighs 35
  kg. The renal and liver functions are normal. Based on the history,
  what additional questions will you ask? What additional test would you
  do?

  \begin{enumerate}
  \def\labelenumii{\alph{enumii}.}
  \tightlist
  \item
    Propose ARVs
  \end{enumerate}
\item
  A 6-month-old was admitted to your facility with a fever, poor weight
  gain and oral thrush. HIV antibody test came up positive in both
  mother and infant

  \begin{enumerate}
  \def\labelenumii{\alph{enumii}.}
  \tightlist
  \item
    Does this HIV antibody test confirm HIV in the child?
  \item
    What other tests are required in this child?
  \end{enumerate}
\item
  A 2-month-old newly diagnosed with HIV. Weight 5kg, Hb=12 g/dl normal
  renal and kidney function.~~~~~

  \begin{enumerate}
  \def\labelenumii{\alph{enumii}.}
  \tightlist
  \item
    Suggest ARVs for treatment
  \end{enumerate}
\end{enumerate}

\section{\texorpdfstring{\textbf{Answers to sample
questions}}{Answers to sample questions}}\label{answers-to-sample-questions}

\begin{enumerate}
\def\labelenumi{\arabic{enumi}.}
\item
  It is important to know the duration of the other symptoms if they are
  beyond 3 months. Ask about interventions, facilities visited and what
  was done for the patient. The top three possibilities are HIV/AIDS,
  Malignancy and tuberculosis. Malignancy should be ruled out (REFER TO
  ONCOLOGY LECTURES). Tuberculosis is confirmed in this child. Since HIV
  is a risk factor for developing TB, it is right to think of HIV in
  this child. Remember HIV is a family disease. Parents and other
  siblings MUST also be screened if they exist.

  To diagnose HIV test, this child must have positive tests on all tests
  using the ``triple algorithm'' namely first response, Oraquick, and SD
  bioline.

  This child has confirmed Pulmonary TB. PTB is airborne and therefore
  there is the need to screen close contacts. This will help identify
  the index case and put in necessary screening tests. Those who have
  the disease are treated while those exposed without the disease will
  need TB preventive therapy.

  This child's HIV test was positive. He has both pulmonary TB and TB
  meningitis. Referring to the notes on WHO clinical staging, PTB is
  stage 3 while TB Meningitis (TBM) is stage 4. This child therefore has
  clinical stage 4

  Remember that this child has an opportunist infection (TB). To prevent
  Immune Reconstitutive Inflammatory Syndrome (IRIS), this child has to
  start TB treatment before starting HAART, for PTB HAART is started
  preferably 2 weeks after TB medication. For TBM, HAART should be
  started 4-6 weeks after initiating TB medications.

  Note: Drug-drug interaction occurs between some ARVs and rifampicin.
  Dolutegravir, lopinavir/ritonavir, and nevirapine should be doubled
  when administered simultaneously with rifampicin. When TB treatment
  ends, extend the duration of the double dose of the ARVs for 2 weeks
  before reducing the dose to the expected age and weight of the
  children.
\item
  In deciding on the best HAART, an ARV should each be selected from
  groups 1,2 and 3. From group 1, the only feasible option is abacavir
  because HB is \textless{} 8g/dl and weight is \textless{} 30 kg(AZT
  and TCF cannot be used). Under group 2, either 3TC or FTC are possible
  but 3TC is readily available. Under group 3, DTG will be the best
  option because EFV has a lower resistance barrier and high community
  resistance. LPV/r is plausible but twice daily regimen makes
  compliance more difficult. The best option will be ABC/3TC/DTG (option
  b is the best choice)
\item
  Being a drug addict, he may be using intravenous injection which
  increases his risk of being infected with hepatitis B or hepatitis C.
  Again, he may be having sexual partners exposing him to STI. He,
  therefore, has to be investigated for hepatitis B/C, full blood count,
  and screened for other STIs. The choice of HAART will have to factor
  in hepatitis B status. TDF can be given because he weighs more than 30
  kg and renal function is normal. Based on the previous explanation,
  TDF/3TC/DTG is the best choice. This option comes as a Fixed dose
  combination so he will take just 1 tablet daily, improving compliance
  among adolescents. Should the adolescent be Hepatitis B positive,
  TDF/3TC are also active against hepatitis B. Adolescents should be
  referred for appropriate services if there are other co-morbidities.
  Referral to a clinical psychologist will also be needed because he is
  addicted to drugs
\item
  Antibodies are IgG and cross the placental to the baby. Newborns
  testing positive using antibody tests might be maternally transmitted.
  Antibody tests do not confirm HIV in children below 18 months. These
  antibodies are expected to disappear by 18 months. The recommended
  test for a 6-month-old baby is a DNA PCR test (refer to above). Other
  tests and timelines if the baby is negative at 6 months, are at 9
  months (DNA PCR) and 18 months (antibody test)
\item
  For a 2-month-old, an Hb of 12 is low; therefore, AZT will not be a
  good option. The HAART of choice is ABC/3TC/DTG
\end{enumerate}

\section{\texorpdfstring{\textbf{Self-assessment
questions}}{Self-assessment questions}}\label{self-assessment-questions}

\begin{enumerate}
\def\labelenumi{\arabic{enumi}.}
\tightlist
\item
  A 15-year-old takes HAART which he thinks are vitamins for SCD. How
  would you disclose his true status to him?
\item
  A 17-year-old adolescent girl is about to start HAART. Explain the
  content of your counselling.
\end{enumerate}

\chapter{Sepsis in Children}\label{sepsis-in-children}

\section{Definition}\label{definition-17}

Sepsis is a life-threatening condition characterized by organ
dysfunction resulting from a dysregulated host response to infection. In
children, it presents a range of disorders caused by bacterial, viral,
fungal, or parasitic infections. Septic shock, a severe form of sepsis,
is defined by persistent hypotension that requires vasopressors to
maintain a mean arterial pressure (MAP) of ≥ 65 mmHg and a serum lactate
level greater than 2 mmol/L, despite adequate fluid resuscitation.

\section{Incidence and Prevalence}\label{incidence-and-prevalence-1}

Sepsis continues to be a major cause of morbidity and mortality in
children around the globe. As stated by the World Health Organization
(WHO), sepsis plays a significant role in childhood mortality,
especially in low-resource environments. The incidence varies by region,
with higher rates observed in neonates and infants due to their immature
immune systems.

\section{Aetiology}\label{aetiology-8}

Sepsis can result from infections caused by bacteria, viruses, fungi,
and parasites. The most common bacterial pathogens vary by age:

\begin{itemize}
\tightlist
\item
  Early-Onset Neonatal Sepsis: \emph{Streptococcus agalactiae},
  \emph{Escherichia coli}, \emph{Haemophilus influenzae}, \emph{Listeria
  monocytogenes}.
\item
  Infant Sepsis: \emph{Haemophilus influenzae type b (Hib)},
  \emph{Streptococcus pneumoniae}, \emph{Neisseria meningitidis},
  \emph{Salmonella} species.
\item
  Late-Onset Neonatal Sepsis: \emph{Staphylococcus aureus}, \emph{E.
  coli}, \emph{Klebsiella} species, \emph{Pseudomonas aeruginosa},
  \emph{Candida} species.
\end{itemize}

\section{Pathogenesis}\label{pathogenesis-3}

Sepsis results from a dysregulated immune response to infection, leading
to widespread tissue injury. The process involves:

\begin{enumerate}
\def\labelenumi{\arabic{enumi}.}
\tightlist
\item
  \textbf{Immune Dysregulation:} Excessive release of pro-inflammatory
  and anti-inflammatory mediators.
\item
  \textbf{Microcirculatory Derangements:} Increased vascular
  permeability, leading to hypotension and organ dysfunction.
\item
  \textbf{End-Organ Damage:} Progression to multi-organ failure due to
  poor perfusion.
\end{enumerate}

\section{Signs and Symptoms}\label{signs-and-symptoms-8}

Early recognition is crucial, as sepsis can progress rapidly. Symptoms
include:

\begin{itemize}
\tightlist
\item
  Fever or Hypothermia
\item
  Tachycardia and Tachypnea
\item
  Altered Mental Status
\item
  Hypotension
\item
  Cool Extremities
\item
  Petechial or Purpuric Rash (suggestive of meningococcal sepsis)
\item
  Oliguria or Anuria (kidney dysfunction).
\end{itemize}

\section{Investigations}\label{investigations-13}

A thorough \textbf{laboratory workup} is essential for diagnosis:

\begin{itemize}
\tightlist
\item
  \textbf{Blood Tests:} Complete Blood Count (CBC), Blood Culture, Blood
  Gas Analysis (including lactate levels).
\item
  \textbf{Urine Analysis:} Dipstick, Routine Examination, Culture.
\item
  \textbf{Cerebrospinal Fluid (CSF) Analysis:} Culture and Sensitivity.
\item
  \textbf{Coagulation Studies:} To assess disseminated intravascular
  coagulation (DIC).
\item
  \textbf{Inflammatory Markers:} C-Reactive Protein (CRP), Procalcitonin
  (PCT), Interleukins (IL-1b, IL-6, IL-8), Tumor Necrosis Factor-alpha.
\end{itemize}

\section{Treatment}\label{treatment-10}

Early antibiotic therapy and fluid resuscitation are critical:

\begin{itemize}
\tightlist
\item
  \textbf{Empirical Broad-Spectrum Antibiotics:} Based on suspected
  pathogens.
\item
  \textbf{Fluid Resuscitation:} Crystalloids (e.g., normal saline or
  Ringer's lactate).
\item
  \textbf{Vasopressors:} If hypotension persists despite fluids.
\item
  \textbf{Supportive Care:} Oxygen therapy, mechanical ventilation,
  renal replacement therapy if needed.
\end{itemize}

\section{Complications}\label{complications-13}

Sepsis can lead to multi-organ failure and death if untreated. Common
complications include:

\begin{itemize}
\tightlist
\item
  Acute Respiratory Distress Syndrome (ARDS)
\item
  Disseminated Intravascular Coagulation (DIC)
\item
  Renal Failure
\item
  Cardiac Dysfunction
\item
  Neurological Sequelae (e.g., cognitive impairment post-sepsis).
\end{itemize}

\section{Prognosis}\label{prognosis-12}

The mortality rate varies based on early recognition and intervention.
Neonatal sepsis has a higher fatality rate, especially in low-resource
settings. Survivors may experience long-term complications, including
neurodevelopmental delays.

\section{Differential Diagnosis}\label{differential-diagnosis-11}

Sepsis must be distinguished from other conditions with similar
presentations:

\begin{itemize}
\tightlist
\item
  Meningitis
\item
  Severe Pneumonia
\item
  Hemorrhagic Shock
\item
  Metabolic Disorders
\item
  Autoimmune Diseases.
\end{itemize}

\section{References}\label{references}

\begin{enumerate}
\def\labelenumi{\arabic{enumi}.}
\tightlist
\item
  Bone RC. The sepsis syndrome: definition and general approach to
  management. Clin Chest Med. 1996 Jun;17(2):175-81. Available here.
\item
  Angus DC, van der Poll T. Severe sepsis and septic shock. N Engl J
  Med. 2013 Aug 29;369(9):840-51. Available here.
\item
  Goldstein B, Giroir B, Randolph A; International Consensus Conference
  on Pediatric Sepsis. International pediatric sepsis consensus
  conference: definitions for sepsis and organ dysfunction in
  pediatrics. Pediatr Crit Care Med. 2005 Jan;6(1):2-8. Available here
\end{enumerate}

\subsection{Clinical Scenario 1: Neonatal
Sepsis}\label{clinical-scenario-1-neonatal-sepsis}

A 5-day-old male infant presents with lethargy, poor feeding,
respiratory distress, and a fever of 38.5°C. On examination, the infant
appears jaundiced, with cool extremities and tachypnea. The mother had a
prolonged rupture of membranes (\textgreater18 hours) before delivery,
and the infant had meconium-stained amniotic fluid.

Key Considerations:

\begin{itemize}
\tightlist
\item
  \textbf{Etiology:} Early-onset neonatal sepsis, likely
  \emph{Streptococcus agalactiae} (Group B Strep) or \emph{Escherichia
  coli}.
\item
  \textbf{Investigations:} Blood cultures, CBC, CRP, procalcitonin,
  serum lactate, and lumbar puncture for CSF culture.
\item
  \textbf{Management:} \textbf{Empirical IV antibiotics} (Ampicillin +
  Gentamicin), fluid resuscitation, oxygen therapy, and supportive care.
\item
  \textbf{Complications:} Risk of meningitis and multi-organ failure if
  untreated.
\end{itemize}

\subsection{Clinical Scenario 2: Pediatric Septic
Shock}\label{clinical-scenario-2-pediatric-septic-shock}

A 6-year-old girl presents with a history of fever (40°C) for 3 days,
altered mental status, and poor urine output. She is tachycardic (HR:
150 bpm), hypotensive (BP: 75/50 mmHg), and has delayed capillary refill
(\textgreater3 seconds). A petechial rash is noted on the lower
extremities.

Key Considerations:

\begin{itemize}
\tightlist
\item
  \textbf{Etiology:} Meningococcal sepsis (\emph{Neisseria
  meningitidis}) suspected.
\item
  \textbf{Investigations:} Blood culture, CBC, coagulation studies,
  lactate, kidney function tests, and inflammatory markers.
\item
  \textbf{Management:} \textbf{IV Ceftriaxone}, aggressive fluid
  resuscitation, vasopressors if needed, and close monitoring in ICU.
\item
  \textbf{Complications:} Disseminated Intravascular Coagulation (DIC),
  Acute Respiratory Distress Syndrome (ARDS), multi-organ failure
\end{itemize}

\chapter{Paediatric Tuberculosis}\label{paediatric-tuberculosis}

\section{Definitions}\label{definitions-1}

Infection with Mycobacterium tuberculosis usually results from inhaling
infected droplets produced by someone who has Pulmonary Tuberculosis
(TB) and is coughing. The most infectious source cases are those with
sputum smear-positive disease. The closer the contact with this source
case, the greater the exposure and the greater the risk of getting
infected with tuberculosis.

\textbf{TB infection} occurs when a person carries the Mycobacterium
tuberculosis bacteria inside the body. Many people have TB but are well.
A positive tuberculin skin test (TST) suggests infection but a negative
TST does not exclude the possibility of infection.

\textbf{TB disease} occurs in someone with TB infection when the
bacteria inside the body start to multiply and become numerous enough to
damage one or more organs of the body. This damage causes clinical
symptoms and signs. This is referred to as ``tuberculosis'' or active
disease.

\textbf{Close contact} is defined as living in the same household as, or
in frequent contact with (e.g. caregiver, school staff), a source case
with PTB.

\textbf{Multidrug-resistant TB (MDR-TB)} is caused by M. tuberculosis
strains that are resistant to both \emph{isoniazid} and
rifa\emph{mpicin}.

\textbf{Pre-extensively drug-resistant TB (Pre-XDR)}: TB caused by M.
tuberculosis strains that fulfil the definition of multidrug-resistant
TB (MDR-TB) or rifampicin-resistant TB (RR-TB) and that are also
resistant to any fluoroquinolone.

\textbf{Extensively drug-resistant TB (XDR-TB):} TB caused by M.
tuberculosis strains that fulfil the definition of MDR/RR-TB and that
are also resistant to any fluoroquinolone and at least one additional
Group A medicine. (bedaquiline and linezolid)

\section{Incidence/prevalence}\label{incidenceprevalence-7}

According to the 2023 WHO global TB report; globally, a total of 10.6
million people fell ill with TB in 2022 of which children less than 15
years accounted for 12\%. According to WHO estimates for 2022, there
were 44,000 estimated incident cases in Ghana. Of this number, Ghana
notified 16,526 cases of which 10\% were expected to be paediatric
(0-14yrs). However, 5\% of paediatric cases were notified.

\section{Aetiology}\label{aetiology-9}

The Mycobacterium tuberculosis complex (MTBC) constitutes a
significantly genetically similar group of bacteria that cause
tuberculosis in various hosts. They are rod-shaped, acid-base-fast,
aerobic, slow-growing intracellular pathogens that destroy phagosomal
cells to maintain and evade the immune system. The major MTBC pathogenic
mycobacteria species include M. tuberculosis, M. bovis, M. africanum,
and M. microti.(Zhang et al. 2022)

\section{Pathogenesis}\label{pathogenesis-4}

Following the M. tuberculosis transmission to a new host, the bacilli
enter the lung and get ingested by macrophages. Further, immune cells
are recruited to wall off the infected macrophages, forming granuloma,
the hallmark of TB. Healthy individuals remain latently infected, and
the infection is kept at bay at this stage, but it is prone to the risk
of reactivation. As the granuloma develops, the bacilli emerge from the
macrophages. When the reactivation occurs, M. tb proliferates, the
bacterial load becomes overwhelmingly high, and the granuloma ruptures,
disseminating the bacteria to the airways. The bacilli are then
expectorated as contagious aerosol droplets, restarting the cycle, and
infecting other individuals.(Alsayed and Gunosewoyo 2023)

\section{Signs and symptoms}\label{signs-and-symptoms-9}

It is important to understand the risk for TB infection and TB disease.
Taking the history from children and caregivers must include questions
on risk factors.

\textbf{For an infection to occur, there are certain factors}:

\begin{enumerate}
\def\labelenumi{\arabic{enumi}.}
\tightlist
\item
  Contact with source case (Close contact and duration of contact)
\item
  Source case (Smear positivity: smear positive is more infectious;
  Cavitation on Chest X-ray: more infectious)
\item
  Increased exposure (Living in high TB endemic areas; children of
  families living with \href{id-hiv.qmd}{HIV})
\end{enumerate}

\textbf{Factors affecting TB disease}:

\begin{enumerate}
\def\labelenumi{\arabic{enumi}.}
\tightlist
\item
  Young age ( 2 years and below)
\item
  \href{id-hiv.qmd}{HIV} infection
\item
  Other immunosuppression (Malnutrition, Post-measles)
\item
  Not BCG vaccinated (Risk of disseminated TB or severe TB disease)
\end{enumerate}

\textbf{Pulmonary Tuberculosis}

\textbf{\emph{History --The major considerations}}

Make every effort to look for the close contact or the household contact
who is the source of the infection. It is helpful to note that close
contact may be at school in a classroom, dormitory/ school bus, or
church. Sometimes, it may be someone who frequently visits the child's
home or a caregiver. In childhood, it may take between 3 months to 2
years from the time of exposure to develop TB disease.

\textbf{History of symptoms suggestive of TB}

More commonly children with TB will present with the following symptoms:

\begin{enumerate}
\def\labelenumi{\arabic{enumi}.}
\tightlist
\item
  Cough of any duration or progressive non-remitting cough which may be
  dry or wet.
\item
  Fever (persistent or unexplained)
\item
  Lethargy/reduced playfulness/less active
\item
  Poor weight gain or weight loss or very low weight (failure to
  thrive), flattened growth curve is a very sensitive marker of disease.
  More specifically it is important to plot the measurement and compare
  it to previous charts on the growth charts in the child health record
  booklet
\item
  Night sweat. Since most children sweat at night, it is usually
  difficult to establish this symptom.
\end{enumerate}

\textbf{Physical Examination} \textbf{(Some clinical findings suggestive
of PTB)}

\textbf{\emph{General Examination}}

\begin{enumerate}
\def\labelenumi{\arabic{enumi}.}
\tightlist
\item
  Fever- Temperature that remains persistently high or irregular
  \textgreater37.5 (fever)
\item
  Weight- (confirm poor weight gain, recent weight loss): the weight
  should be plotted on the child's growth curve, and any child who
  ``falls off'' or is unable to maintain their usual line of growth
  should be considered as having possible TB
\item
  Length/Height is needed to determine the weight-for-length/height
  Z-scores (\textless-3 Z indicates severe wasting)
\item
  MUAC -Middle upper arm circumference of \textless{} 12.5 cm
\item
  Respiratory rate - (fast breathing) depends on the patient's age.
  (Children 0-2 months above 60cpm, 3 months to 12 months more than 50
  CPM and 1-5 years more than 40 CPM)
\item
  Signs of respiratory distress are not specific to TB but must raise
  the index of suspicion e.g Low oxygen saturation, stridor, and wheezes
\end{enumerate}

\textbf{Physical signs suggestive of Extra Pulmonary TB (EPTB) include}:

\begin{enumerate}
\def\labelenumi{\arabic{enumi}.}
\tightlist
\item
  Enlarged cervical lymph nodes which are not painful with or without
  fistula formation -- TB lymphadenopathy;
\item
  Presence of spinal kyphosis (angular swelling) -- spinal TB
  (``gibbous'');
\item
  Signs of non-acute meningitis with poor response to antibiotic
  treatment and/or with raised intracranial pressure -- TB Meningitis;
\item
  Pleural effusion, especially one-sided dullness with pleuritic pain in
  a child who is not acutely ill -- pleural TB;
\item
  Pericardial effusion, distant or muffled heart sounds or signs of
  new-onset heart failure -- pericardial TB;
\item
  Non-acute distended abdomen with or without ascites -- abdominal TB;
\item
  Non-tender swollen joints with painful or abnormal gait --
  osteoarticular TB.
\end{enumerate}

\section{Investigations}\label{investigations-14}

In addition to a detailed history and careful physical examination, all
children suspected to have TB will require additional investigations.
Investigations commonly used are grouped into the following categories.

\begin{enumerate}
\def\labelenumi{\arabic{enumi}.}
\tightlist
\item
  Bacteriological investigations
\item
  Radiologic investigations and
\item
  Immunologic investigations.
\end{enumerate}

\subsection{Bacteriological
investigations}\label{bacteriological-investigations}

\textbf{Xpert MTB/RIF Assay} is the recommended first-line investigation
for diagnosing TB in children. Results are rapid and determine if the
patient has a drug-sensitive or resistant organism. Various specimens
may be collected, including expectorated sputum, induced sputum, gastric
aspirate, bronchoalveolar lavage, transbronchial biopsies, pleural
aspirate urine, blood, cerebrospinal fluid tissue and, more recently,
stool. Other modalities for confirming TB are smear microscopy and TB
cultures.

\subsection{Radiologic investigations}\label{radiologic-investigations}

Children often have paucibacillary TB and therefore bacteriological
yields are low. Various imaging modalities can be suggestive of TB.
Chest X-ray is the most frequently used radiological imaging. The
presence of hilar lymphadenopathy (Figure~\ref{fig-id-cxr-ptb}),
effusions, and cavitations could all support the diagnosis of TB in
children. Ultrasound, CT scan, and MRI all have roles in suspected
extrapulmonary TB.

\begin{figure}

\centering{

\includegraphics[width=3.65625in,height=\textheight,keepaspectratio]{images/id-cxr-ptb.jpg}

}

\caption{\label{fig-id-cxr-ptb}Chest X-ray showing perihilar
lymphadenopathy suggestive of TB}

\end{figure}%

\subsection{Immunologic
investigations}\label{immunologic-investigations}

Immunological tests provide evidence for TB infection but not TB
disease. Two tests are widely used namely the \emph{Tuberculin skin
test} and the \emph{interferon-gamma release assay}.

\section{Treatment}\label{treatment-11}

\subsection{Antituberculous
medications}\label{antituberculous-medications}

There are two types of treatment namely \emph{TB disease treatment} and
\emph{TB preventive therapy.} The Paediatric TB Medicines comprises of 3
different formulations as follows:

\begin{enumerate}
\def\labelenumi{\arabic{enumi}.}
\tightlist
\item
  Rifampicin + Isoniazid + Pyrazinamide (RHZ) 75/50/150 mg
\item
  Ethambutol (E) 100 mg 3. Rifampicin + Isoniazid (RH) 75/50 mg
\end{enumerate}

Every child receives 2RHZE ( 2 months intensive phase)/4RH (4 months
continuation phase) for all forms of TB except TB meningitis and
osteoarticular TB where the continuation phase is extended for 10 months
(10 RH). For non-severe TB (refer to further reading) 2RHZE/2RH regimen
can be applied.

\textbf{\emph{Note:}}

Corticosteroids are often used as an adjunct in the treatment of these
forms of TB to prevent complications. These include TB Meningitis; TB
Pericarditis; and Pott disease/ TB Spondylitis. Pleural diseases, and
Endobronchial TB\\
Pyridoxine (Vitamin B6) supplement is necessary in some patients to
prevent peripheral neuropathy but recommended in ALL
\href{id-hiv.qmd}{HIV}-infected persons and severely acute malnourished
patients on isoniazid

\subsection{Major side effects}\label{major-side-effects}

Potential side effects of TB medications are:

\begin{enumerate}
\def\labelenumi{\arabic{enumi}.}
\tightlist
\item
  Rifampicin: Orange-colored urine, saliva or tears, jaundice
\item
  Pyrazinamide: GI disturbances, hepatotoxicity
\item
  Ethambutol: GI disturbances, blurred vision
\item
  Isoniazid: numbness and tingling in the extremities, GI disturbances,
  rash
\end{enumerate}

\subsection{TB Preventive Therapy}\label{tb-preventive-therapy}

Every person living with \href{id-hiv.qmd}{HIV} should be given TB
preventive therapy (TPT) after screening and ruling out active TB
disease. Other categories of children requiring TPT after ruling out TB
disease are:

\begin{enumerate}
\def\labelenumi{\arabic{enumi}.}
\tightlist
\item
  Newborns of mothers with TB,
\item
  All Children exposed to an index case with sputum-positive TB,
\item
  Long-term steroids, and immunocompromised children.
\end{enumerate}

If a patient develops TB disease, the patient should be investigated,
and treatment changed from TPT to full treatment. There are four(4)
options for TPT in children:

\begin{enumerate}
\def\labelenumi{\arabic{enumi}.}
\tightlist
\item
  Rifapentine + Isoniazid- weekly for 3 months
\item
  Rifampicin + isoniazid -- daily for 3 months
\item
  Rifampicin- daily for 4 months
\item
  Isoniazid only -- daily for 6 months
\end{enumerate}

\section{Complications}\label{complications-14}

The most common complication is chronic lung disease. TB can affect any
part of the body including the brain, spine and therefore can cause
other complications such as stroke, abscesses, impaired growth and so
forth.

\section{Prognosis}\label{prognosis-13}

With early identification and treatment, the prognosis is good.

\section{Differential diagnosis}\label{differential-diagnosis-12}

Common differentials are bacterial pneumonia, atypical pneumonia,
brucellosis, bronchogenic carcinoma, \href{id-hiv.qmd}{HIV}, and Hodgkin
lymphoma

\section{Further readings}\label{further-readings}

\href{https://www.who.int/publications/i/item/9789240046764}{WHO
consolidated guidelines on tuberculosis: module 5: management of
tuberculosis in children and adolescents}

\href{https://www.who.int/publications/i/item/9789240046832}{WHO
operational handbook on tuberculosis: module 5: management of
tuberculosis in children and adolescents}

\section{Sample case scenarios}\label{sample-case-scenarios-1}

\textbf{Question}

\begin{enumerate}
\def\labelenumi{\arabic{enumi}.}
\tightlist
\item
  A mother delivers a newborn at 40 weeks gestation. Within the last 4
  weeks of pregnancy, she started coughing. She bought cough syrup and
  amoxicillin at a dispensary. Her coughing got severe and she noted
  weight loss. At week 39, she visited the hospital and was diagnosed
  with TB (GeneXpert MTB positive and RIF sensitive). She was started on
  treatment immediately. She lives in a single room with her 3 other
  children who are 2 years, 8 years, and 10 years respectively who were
  all clinically well except the 2 years old who weighed 8 kgs.

  \begin{enumerate}
  \def\labelenumii{\arabic{enumii}.}
  \tightlist
  \item
    Identify risks of infection for the children
  \item
    Identify risks of disease in children
  \item
    How would you approach the management of the children
  \end{enumerate}
\end{enumerate}

\textbf{Answers}

\begin{enumerate}
\def\labelenumi{\alph{enumi}.}
\tightlist
\item
  The risks of infection in the above scenario are sputum-positive MTB
  on GeneXpert tests and the single room is occupied by a single mother
  and her children.
\item
  The risk of TB disease will be in the newborn and the 2-year-old
  sibling who is already failing to thrive. If the mother has
  \href{id-hiv.qmd}{HIV} or the children are not immunised with BCG,
  that would also be a risk factors for disease.
\item
  All the children have been exposed to TB through a close contact, who
  happens to be their mother. The mother has to be tested for
  \href{id-hiv.qmd}{HIV}. If she is positive, all the children should
  also be tested. All the children would have to be screened for TB. The
  newborn should ideally not be given BCG vaccine but put on TPT if TB
  disease is excluded. The recommended TPT will depend on the
  \href{id-hiv.qmd}{HIV} status of the newborn, but Isoniazid is an
  option for 6 months. After 6 months of INH, if there is no evidence
  that the newborn has been exposed to TB (Mantoux testing), the child
  can be given BCG vaccine. If the newborn has the disease, then full TB
  treatment should be given. The 2-year-old weighs 8 kg which is
  evidence of weight faltering. Plot the weight for age on the Z-score.
  Do other investigations such as Chest X-ray, stool for Xpert, and
  \href{id-hiv.qmd}{HIV} testing. If the clinical, bacteriological, and
  imaging are suggestive of TB, treat the 2-year-old. If
  \href{id-hiv.qmd}{HIV} is positive, remember to adjust the dose of
  ARVs that interact with rifampicin during TB treatment. If the
  screening of the 8 and 10 years is normal, then put them on TPT,
  otherwise treat them
\end{enumerate}

\textbf{Self-assessment questions}

\begin{enumerate}
\def\labelenumi{\arabic{enumi}.}
\tightlist
\item
  A 5-year-old boy diagnosed with TB and started on RHZE complained to
  his mother that he finds it difficult to see clearly what his school
  teacher has projected in class. Which of the following medications is
  likely responsible?

  \begin{enumerate}
  \def\labelenumii{\alph{enumii}.}
  \tightlist
  \item
    Isoniazid
  \item
    Pyrazinamide
  \item
    Ethambutol
  \item
    Rifampicin
  \end{enumerate}
\item
  Explain why bacteriological yield from children with suspected PTB is
  often very low.
\end{enumerate}

\chapter{Childhood Immunization}\label{childhood-immunization}

\section{Introduction}\label{introduction-20}

Childhood immunization is a cornerstone of public health, offering
lifesaving protection against infectious diseases. It is one of the most
cost-effective strategies for reducing morbidity and mortality in
children worldwide. Immunization is vital in achieving
\href{https://sdgs.un.org/goals/goal3}{Sustainable Development Goal
(SDG) 3}, which aims to reduce neonatal mortality to 12 per 1000 live
births and under-five mortality to 25 per 1000 live births by 2030.
Despite its effectiveness, challenges such as parental misconceptions,
healthcare accessibility, vaccine hesitancy, and logistical issues
persist.

\section{Key Immunization Concepts}\label{key-immunization-concepts}

\begin{itemize}
\tightlist
\item
  \textbf{Immunization}: The process of artificially conferring immunity
  against infectious diseases.
\item
  \textbf{Vaccination}: The act of introducing an antigenic material
  into the body to stimulate an immune response, leading to future
  protection upon exposure.
\item
  \textbf{Vaccine}: A biological preparation that enhances active
  acquired immunity against specific pathogens. Vaccines can be
  live-attenuated, inactivated, subunit, toxoid-based, or nucleic
  acid-based.
\end{itemize}

\section{Types of Vaccines}\label{types-of-vaccines}

Vaccines are classified based on their composition and mechanism:

\begin{enumerate}
\def\labelenumi{\arabic{enumi}.}
\tightlist
\item
  \textbf{Live Attenuated Vaccines}: Contain weakened pathogens (e.g.,
  Measles, Mumps, Rubella, BCG, Yellow Fever, OPV).
\item
  \textbf{Killed/Inactivated Vaccines:} Pathogens are killed but retain
  their immunogenic properties (e.g., Hepatitis A, Rabies, IPV).
\item
  \textbf{Toxoid-Based Vaccines}: Contain inactivated toxins to generate
  immunity (e.g., Tetanus, Diphtheria).
\item
  \textbf{Subunit/Conjugate Vaccines}: Use fragments of pathogens for
  immunity (e.g., Pneumococcal, Hib, Hepatitis B).
\item
  \textbf{Nucleic Acid-Based Vaccines}: Utilize mRNA or DNA to instruct
  cells to produce an immune response (e.g., SARS-CoV-2 vaccines).
\end{enumerate}

\section{Expanded Programme on Immunization (EPI) in
Ghana}\label{expanded-programme-on-immunization-epi-in-ghana}

Ghana's routine immunization schedule includes:

\begin{itemize}
\tightlist
\item
  \textbf{Birth:} BCG, OPV (0)
\item
  \textbf{6 Weeks:} Pentavalent (DTwP/HepB/Hib), OPV (1), PCV (1),
  Rotavirus (1)
\item
  \textbf{10 Weeks:} Pentavalent (2), OPV (2), PCV (2), Rotavirus (2)
\item
  \textbf{14 Weeks:} Pentavalent (3), OPV (3), IPV, PCV (3), Rotavirus
  (3)
\item
  \textbf{6-7 Months:} RTS,S (Malaria vaccine)
\item
  \textbf{9 Months:} Measles-Rubella (MR), Yellow Fever, RTS,S (3)
\item
  \textbf{18 Months:} MR Booster, Meningococcal A, RTS,S (4)
\item
  \textbf{Adolescents:} HPV vaccine for girls aged 10-13.
\end{itemize}

\section{Principles of Immunization}\label{principles-of-immunization}

\begin{itemize}
\tightlist
\item
  \textbf{Herd Immunity:} High vaccination coverage reduces disease
  transmission.
\item
  \textbf{Cold Chain Management:} Essential for vaccine potency.
\item
  \textbf{Missed Opportunities:} Ensuring timely vaccination enhances
  community protection.
\item
  \textbf{Vaccine Safety:} Strict protocols ensure vaccines are safe and
  effective.
\end{itemize}

\section{Contraindications to
Vaccination}\label{contraindications-to-vaccination}

While vaccines are generally safe, contraindications exist:

\begin{itemize}
\tightlist
\item
  \textbf{Live vaccines} should be avoided in \textbf{immunocompromised
  individuals}.
\item
  \textbf{Severe allergic reactions} (anaphylaxis) to vaccine
  components.
\item
  \textbf{Pregnancy} -- Certain live vaccines (e.g., Yellow Fever,
  Measles) are contraindicated.
\end{itemize}

\section{Vaccine-Preventable Diseases \& Control
Programs}\label{vaccine-preventable-diseases-control-programs}

Immunization programs target several infectious diseases:

\begin{itemize}
\tightlist
\item
  \textbf{Tuberculosis (BCG):} Protects against severe TB forms in
  infants.
\item
  \textbf{Polio:} Eradication efforts involve IPV and OPV.
\item
  \textbf{Pneumococcal Disease (PCV13):} Prevents pneumonia, meningitis.
\item
  \textbf{Rotavirus:} Reduces childhood diarrheal deaths.
\item
  \textbf{Measles, Rubella, Mumps (MR/MMR):} Prevents severe
  complications.
\item
  \textbf{Meningococcal Disease (Men A):} Prevents outbreaks.
\item
  \textbf{Yellow Fever:} Essential for endemic regions.
\item
  \textbf{HPV:} Prevents cervical cancer.
\end{itemize}

\section{Challenges \& Future
Directions}\label{challenges-future-directions}

\begin{itemize}
\tightlist
\item
  \textbf{Vaccine Hesitancy:} Addressing misinformation.
\item
  \textbf{New Vaccine Development:} Ongoing innovations like mRNA
  vaccines.
\item
  \textbf{Policy \& Surveillance:} Strengthening disease monitoring to
  prevent outbreaks.
\end{itemize}

\section{Specific vaccines}\label{specific-vaccines}

\subsection{BCG Vaccine (Tuberculosis)}\label{bcg-vaccine-tuberculosis}

\begin{itemize}
\tightlist
\item
  \textbf{Pathogen:} \emph{Mycobacterium tuberculosis} or \emph{M.
  bovis} (Non-sporing, rod-shaped, acid-fast bacillus)
\item
  \textbf{Type:} Live attenuated, freeze-dried
\item
  \textbf{Storage:} 2°C - 8°C (Never frozen)
\item
  \textbf{Administration:} Intradermal (deltoid)
\item
  \textbf{Shelf Life:} 12 - 18 months
\item
  \textbf{Usage:} Do not shake to mix; use within 2 hours
\item
  \textbf{Efficacy:}

  \begin{itemize}
  \tightlist
  \item
    0--80\% for Pulmonary TB (PTB)
  \item
    75--86\% for Miliary TB \& TB Meningitis
  \end{itemize}
\item
  \textbf{Indications:}

  \begin{itemize}
  \tightlist
  \item
    Infants, health personnel, and contacts with sputum-positive cases.
  \item
    Suspected exposure? Perform a tuberculin test before immunization.
  \item
    In cases of contact: Tuberculin test, repeat after 6 months. If
    positive ⇒ Chemoprophylaxis.
  \end{itemize}
\item
  \textbf{Notes:}

  \begin{itemize}
  \tightlist
  \item
    Duration of immunity is uncertain; it wanes over time.
  \item
    Protects children against meningitis \& disseminated TB but does not
    prevent primary infection or reactivation.
  \item
    Scar confirms vaccination but not protection. Absence of a scar may
    indicate the need for testing and revaccination.
  \item
    Up to 10\% scar failure rate is acceptable in properly vaccinated
    individuals.
  \item
    Tuberculin test: Uses Purified Protein Derivative (PPD) for
    Mantoux/Heaf test.
  \end{itemize}
\end{itemize}

\subsection{Polio Vaccines}\label{polio-vaccines}

\begin{itemize}
\tightlist
\item
  \textbf{Pathogen:} Poliovirus Types I, II, III.
\item
  \textbf{Adults:} More prone to \textbf{inapparent paralytic
  infections}.
\item
  \textbf{Virus Survival:} Inactivated at 55°C for 30 minutes (but
  inhibited by Mg++, milk, ice cream)
\item
  \textbf{Types of Vaccines:}

  \begin{itemize}
  \tightlist
  \item
    \textbf{Inactivated Polio Vaccine (IPV)} -- Salk (1956)
  \item
    \textbf{Live Attenuated Oral Polio Vaccine (OPV)} -- Sabin (1962)
  \item
    \textbf{Variants:}
  \item
    \textbf{tOPV (Trivalent OPV):} Contains live strains of all three
    virus types.
  \item
    \textbf{bOPV (Bivalent OPV):} Contains live strains of Types I \&
    III.
  \item
    \textbf{nOPV (Novel OPV):} A modified strain of Type II with
    enhanced stability.
  \end{itemize}
\item
  \textbf{Efficacy:}

  \begin{itemize}
  \tightlist
  \item
    90\% in industrialized nations.
  \item
    72--98\% in hot climates (\textbf{lower protection against Type
    III}).
  \end{itemize}
\item
  \textbf{Duration of Immunity:}

  \begin{itemize}
  \tightlist
  \item
    Lifelong if boosted by wild virus, otherwise shorter.
  \end{itemize}
\item
  \textbf{Vaccine-Associated Paralytic Poliomyelitis (VAPP):}
\item
  Polio Type II in tOPV linked to \textbf{VAPP}, undermining eradication
  efforts.
\item
  \textbf{Solution:} Withdraw Type II from OPV ⇒ Introduce bOPV + at
  least one IPV dose in routine schedules.
\item
  \textbf{Ghana (since June 2018):} bOPV + IPV at 14 weeks.
\item
  nOPV introduced as a \textbf{more antigenically stable} next-gen Type
  II vaccine.
\end{itemize}

\subsection{Pentavalent Vaccine
(``PENTA'')}\label{pentavalent-vaccine-penta}

\begin{itemize}
\tightlist
\item
  \textbf{Components:} Diphtheria, Whole-cell Pertussis, Hepatitis B,
  Hib (Haemophilus influenzae type
\item
  \textbf{Introduced:} 2001; used in Ghana since March 2002.
\item
  \textbf{Storage:}

  \begin{itemize}
  \tightlist
  \item
    Liquid DPT-HepB: Refrigerated at 2°C - 8°C (\textbf{not frozen}).
  \item
    Lyophilized Hib: Stored at -20°C or refrigerated at +2 - 8°C.
  \end{itemize}
\item
  \textbf{Preservation:} Contains preservatives, allowing reconstitution
  for extended use.
\end{itemize}

\subsection{Tetanus Vaccine}\label{tetanus-vaccine}

\begin{itemize}
\tightlist
\item
  \textbf{Pathogen:} \emph{Clostridium tetani} (Toxin-producing)
\item
  \textbf{Vaccine Type:} Toxoid (inactivated toxin)
\item
  \textbf{Schedule:}

  \begin{itemize}
  \tightlist
  \item
    \textbf{Children \& Adults:} 3 doses (one month apart); reinforced
    every 10 years with two doses for lifelong immunity.
  \item
    \textbf{Boosters:} Recommended at time of injury.
  \item
    \textbf{Maternal immunization:} Protects against neonatal tetanus.
  \end{itemize}
\end{itemize}

\subsection{Hepatitis B Vaccine}\label{hepatitis-b-vaccine}

\begin{itemize}
\tightlist
\item
  \textbf{Pathogen:} Hepadnavirus (Double-stranded DNA virus)
\item
  \textbf{Carrier Rate:} 2 - 10\% (higher in perinatal infections).
\item
  \textbf{Transmission:} Highly infectious among carriers with
  \textbf{HBeAg}.
\item
  \textbf{Vaccine:}

  \begin{itemize}
  \tightlist
  \item
    HBsAg adsorbed onto alum (adjuvant).
  \item
    Produced via recombinant DNA in yeast cells.
  \end{itemize}
\item
  \textbf{Pre-exposure Immunization}

  \begin{itemize}
  \tightlist
  \item
    Universal infant immunization.
  \item
    Catch-up vaccination for adolescents.
  \item
    Healthcare workers, hemodialysis patients, blood recipients, drug
    abusers, transplant candidates.
  \end{itemize}
\item
  \textbf{Post-exposure Prophylaxis:}

  \begin{itemize}
  \tightlist
  \item
    \textbf{HBIG (Hepatitis B Immunoglobulin):} Provides passive
    immunity (3-6 months).
  \item
    Best protection: \textbf{HBIG + Hep B vaccine} within \textbf{24
    hours} after exposure.
  \item
    \textbf{Routine infant vaccination:} HB vaccine alone is sufficient.
  \item
    Not needed for pre-transfusion prophylaxis due to modern blood
    screening.
  \end{itemize}
\end{itemize}

\subsection{Yellow Fever Vaccine}\label{yellow-fever-vaccine}

\begin{itemize}
\tightlist
\item
  \textbf{Pathogen:} \emph{Flavivirus} (RNA virus); spread by
  \emph{Aedes aegypti} mosquitoes.
\item
  \textbf{Vaccine:} Live attenuated, freeze-dried (17D strain, grown in
  chick embryo).
\item
  \textbf{Contains:} Neomycin, polymyxin.
\item
  \textbf{Contraindications:} Allergy to components.
\end{itemize}

\subsection{Measles-Rubella (MR) \& MMR
Vaccine}\label{measles-rubella-mr-mmr-vaccine}

\begin{itemize}
\tightlist
\item
  \textbf{Viruses:}

  \begin{itemize}
  \tightlist
  \item
    Measles, Mumps (\emph{Paramyxoviruses}, RNA)
  \item
    Rubella (\emph{Togavirus}, single-stranded RNA)
  \end{itemize}
\item
  \textbf{Purpose:} Prevent congenital rubella infection.
\item
  \textbf{Variants}

  \begin{itemize}
  \tightlist
  \item
    MR (Measles-Rubella) -- Used in Ghana.
  \item
    MMR (Measles-Mumps-Rubella).
  \end{itemize}
\item
  \textbf{Presentation:} Freeze-dried.
\item
  \textbf{Administration:} Subcutaneous injection.
\end{itemize}

\subsection{Pneumococcal Disease \&
Vaccines}\label{pneumococcal-disease-vaccines}

\begin{itemize}
\tightlist
\item
  \textbf{Pathogen:} \emph{Streptococcus pneumoniae} (Gram-positive
  diplococcus).
\item
  \textbf{Common Diseases:}

  \begin{itemize}
  \tightlist
  \item
    \textbf{Non-Invasive:} Otitis media, sinusitis, bronchitis.
  \item
    \textbf{Invasive (IPD):} Pneumonia, Bacteraemia, Meningitis.
  \end{itemize}
\end{itemize}

\begin{itemize}
\item
  \textbf{Vaccines}

  \begin{itemize}
  \tightlist
  \item
    \textbf{Polysaccharide (PPV23):} Short-lived immunity, recommended
    for high-risk individuals ≥2 years.
  \item
    \textbf{Conjugate (PCV13):} Provides longer-lasting immunity and is
    effective against pneumonia. Ghana uses PCV13 (``Prevenar'')
  \end{itemize}
\end{itemize}

\subsection{Rotavirus Vaccine}\label{rotavirus-vaccine}

\begin{itemize}
\tightlist
\item
  \textbf{Disease Impact:} Severe diarrheal illness in young children;
  major cause of dehydration.
\item
  \textbf{Transmission:} Ubiquitous (water and sanitation improvements
  do not prevent infection).
\item
  \textbf{Vaccine Options in Ghana:}
\item
  \textbf{Rotavac (May 2021):} 3 doses (6, 10, 14 weeks).
\item
  \textbf{Rotarix (GSK):} Monovalent, given orally in \textbf{two doses}
  (by 16 weeks, no later than 24 weeks).
\item
  \textbf{Rotateq:} Bovine-human reassortant vaccine; \textbf{three
  doses} at 2, 4, 6 months.
\item
  \textbf{RotaShield (Wyeth):} Withdrawn due to risk of
  \textbf{intussusception}.
\end{itemize}

\subsection{Human Papillomavirus (HPV)
Vaccine}\label{human-papillomavirus-hpv-vaccine}

\begin{itemize}
\tightlist
\item
  \textbf{Virus Type:} Small, double-stranded DNA virus.
\item
  \textbf{High-Risk Oncogenic Strains:} Types \textbf{16 \& 18} (cause
  70\% of cervical cancers).
\item
  \textbf{Vaccines:}

  \begin{itemize}
  \tightlist
  \item
    \textbf{Quadrivalent (HPV 6, 11, 16, 18)} -- Produced in yeast.
  \item
    \textbf{Bivalent (HPV 16, 18).}
  \end{itemize}
\item
  \textbf{Target Age Group:} 10-13-year-old girls (not a standard
  vaccination group).
\item
  \textbf{Catch-up Vaccination:} Not recommended in public health
  programs.
\end{itemize}

\subsection{Malaria Vaccines and Immunization
Strategy}\label{malaria-vaccines-and-immunization-strategy}

Malaria remains a significant global health challenge, particularly in
endemic regions. The \textbf{Plasmodium falciparum life cycle} involves
two distinct stages:

\begin{enumerate}
\def\labelenumi{\arabic{enumi}.}
\tightlist
\item
  \textbf{Asexual Stage (Human Host)} -- Sporozoites enter the
  bloodstream through mosquito bites, travel to the liver, and mature
  into merozoites before infecting red blood cells.
\item
  \textbf{Sexual Stage (Mosquito Vector)} -- Gametocytes ingested by
  mosquitoes undergo development, enabling transmission.
\end{enumerate}

\textbf{RTS,S/AS01 Malaria Vaccine}

The RTS,S/AS01 malaria vaccine provides partial protection against
\textbf{Plasmodium falciparum} infection.

\begin{itemize}
\tightlist
\item
  \textbf{Mechanism of Action}:

  \begin{itemize}
  \tightlist
  \item
    Induces \textbf{antibody production} to block sporozoite entry into
    liver cells.
  \item
    Activates \textbf{T-cell responses} to eliminate sporozoites that
    reach the liver.
  \end{itemize}
\item
  \textbf{Administration}:

  \begin{itemize}
  \item
    \textbf{Four-dose series}:

    \begin{itemize}
    \tightlist
    \item
      \textbf{1st dose} at \textbf{5 months} (not recommended for
      infants
    \item
      \textbf{2nd and 3rd doses} administered at \textbf{4-week
      intervals}.
    \item
      \textbf{4th dose} given between \textbf{15--18 months}
    \end{itemize}
  \item
    Can be co-administered with other vaccines in \textbf{national
    immunization programs}
  \end{itemize}
\item
  \textbf{Efficacy}: Less than \textbf{50\%}, but beneficial for
  reducing severe cases and mortality.
\item
  \textbf{Recommendations}: Targeted for \textbf{high-malaria-burden
  African countries} with existing control programs.
\end{itemize}

\subsection{Rabies Vaccines and Post-Exposure
Prophylaxis}\label{rabies-vaccines-and-post-exposure-prophylaxis}

Rabies is a fatal viral zoonosis transmitted through the bite of
infected animals (mainly carnivores and bats).

\begin{itemize}
\tightlist
\item
  \textbf{Disease Determinants}:

  \begin{itemize}
  \tightlist
  \item
    Severity of wound and viral inoculation.
  \item
    Proximity of bite to central nervous system (higher risk if near
    head).
  \item
    Timeliness of post-exposure prophylaxis (PEP).
  \end{itemize}
\item
  \textbf{Rabies Virus (RABV) Presence in Humans}:

  \begin{itemize}
  \tightlist
  \item
    Found in saliva, tears, urine, and nervous tissues.
  \item
    Not detected in blood.
  \end{itemize}
\item
  \textbf{Incubation Period}: 1-3 months, but may extend up to 1 year
\end{itemize}

\textbf{Post-Exposure Prophylaxis (PEP)}

\begin{itemize}
\tightlist
\item
  \textbf{Immediate wound cleansing}.
\item
  \textbf{Rabies vaccine series initiation}.
\item
  \textbf{Rabies immunoglobulin (Rabies IG)} infiltration around the
  wound (if indicated).
\end{itemize}

\textbf{Rabies Vaccine Types}

\begin{itemize}
\tightlist
\item
  Cell Culture or Embryonated Egg Vaccines (CCEECV):

  \begin{itemize}
  \tightlist
  \item
    Live attenuated, freeze-dried (propagated in human diploid or chick
    embryo).
  \item
    Administered intramuscularly (IM) or intradermally (ID) on Days 0,
    3, 7, 14, and 28.
  \item
    Preferred site of administration: Deltoid (adults) or anterolateral
    thigh (children)
  \end{itemize}
\item
  Nerve Tissue Vaccines (NTV) (Obsolete):
\item
  Derived from animal brain tissue.
\item
  Associated with Guillain-Barré Syndrome, encephalitis.
\item
  Not recommended by WHO.
\end{itemize}

\textbf{Note}: Chloroquine prophylaxis suppresses rabies vaccine
antibody response, particularly when given intradermally.

\subsection{Influenza and COVID-19
Vaccines}\label{influenza-and-covid-19-vaccines}

\subsubsection{Influenza Virus and Vaccine
Development}\label{influenza-virus-and-vaccine-development}

Influenza viruses undergo continuous genetic variations, requiring
annual vaccine updates.

\begin{itemize}
\tightlist
\item
  Antigenic Drift -- Small mutations producing minor variants (Influenza
  A, B) → epidemics.
\item
  Antigenic Shift -- Major genetic reassortments (Influenza A only) →
  pandemics.
\end{itemize}

\subsubsection{COVID-19 Vaccine
Technologies}\label{covid-19-vaccine-technologies}

\begin{itemize}
\tightlist
\item
  \textbf{Inactivated or Weakened Virus} -- Uses killed virus
  components.
\item
  \textbf{Protein-Based Vaccines} -- Uses harmless protein fragments to
  generate immunity.
\item
  \textbf{Viral Vector Vaccines} -- Genetically engineered virus
  produces spike proteins.
\item
  \textbf{RNA/DNA Vaccines} -- Uses mRNA/DNA encoding spike proteins
  (e.g., Moderna, Pfizer).
\end{itemize}

\section{Vaccine Reactions and Adverse Events Following Immunization
(AEFI)}\label{vaccine-reactions-and-adverse-events-following-immunization-aefi}

\subsection{Minor Vaccine Reactions}\label{minor-vaccine-reactions}

\begin{itemize}
\tightlist
\item
  Common reactions occur as part of the immune response:

  \begin{itemize}
  \tightlist
  \item
    Fever, injection-site swelling/pain, malaise.
  \item
    Most frequent with DPT vaccines.
  \item
    Symptoms self-resolve.
  \end{itemize}
\item
  Parents should be educated on symptom management.
\end{itemize}

\subsection{Severe Reactions (Rare)}\label{severe-reactions-rare}

\begin{itemize}
\tightlist
\item
  Anaphylaxis (1 per million doses) -- Requires urgent medical
  intervention (e.g., adrenaline).
\item
  BCG Osteitis -- Rare, vaccine-specific reaction.
\item
  Vaccine-Induced Fainting -- Common in adolescents, often
  misinterpreted as anaphylaxis.
\end{itemize}

\subsection{AEFI Classification}\label{aefi-classification}

\begin{enumerate}
\def\labelenumi{\arabic{enumi}.}
\tightlist
\item
  \textbf{Vaccine Reaction} -- Direct response to vaccine components.
\item
  \textbf{Program Error} -- Due to improper vaccine
  handling/administration
\item
  \textbf{Coincidental} -- Occurs post-immunization but is unrelated to
  vaccination.
\item
  \textbf{Injection Reaction} -- Pain/anxiety linked to the injection
  process.
\item
  \textbf{Unknown Cause} -- Unresolved cases
\end{enumerate}

\section{Vaccine Storage and Multidose Vial
Policy}\label{vaccine-storage-and-multidose-vial-policy}

\begin{itemize}
\tightlist
\item
  \textbf{Cold Chain Maintenance}: Essential for vaccine efficacy and
  stability.
\item
  \textbf{Heat Sensitivity}:

  \begin{itemize}
  \tightlist
  \item
    BCG, Measles, Polio can be frozen.
  \item
    Diluent-containing vaccines (DPT, TT, HepB) must NOT be frozen.
  \item
    Frozen vaccines may cause reduced immune response
  \end{itemize}
\item
  \textbf{Shelf Life}: Max 2 years under ideal storage.
\item
  \textbf{Vaccine Vial Monitors (VVM)}:
\item
  Monitor vaccine exposure to heat.
\item
  Discard if VVM reaches critical stages.
\end{itemize}

\subsection{Multidose Vial Policy (Current
Guidelines)}\label{multidose-vial-policy-current-guidelines}

\begin{itemize}
\tightlist
\item
  Vaccines usable for up to 4 weeks if:

  \begin{itemize}
  \tightlist
  \item
    Stored at 2--8°C.
  \item
    Aseptic techniques are used for administration.
  \item
    VVM remains intact.
  \end{itemize}
\item
  Reconstituted vaccines (BCG, Measles, Yellow Fever):
\item
  Must be discarded after 6 hours or at the end of the session.
\end{itemize}

\textbf{Contraindications to Vaccination}

\begin{itemize}
\tightlist
\item
  \textbf{Live vaccines contraindicated in}:

  \begin{itemize}
  \tightlist
  \item
    Immunocompromised individuals (HIV, malignancies).
  \item
    Pregnant women (risk of teratogenicity).
  \item
    Neurological disorders (avoid DPT in uncontrolled epilepsy).
  \end{itemize}
\item
  \textbf{Egg allergy}: Avoid Yellow Fever, Influenza, but alternative
  fibroblast-derived vaccines may be used.
\end{itemize}

\section{Vaccination in Special
Populations}\label{vaccination-in-special-populations}

\subsection{Preterm Infant
Immunization}\label{preterm-infant-immunization}

\begin{itemize}
\tightlist
\item
  Immunization response similar to term infants.
\item
  Start immunization at 2 months, irrespective of prematurity.
\item
  OPV delayed until discharge (reduces nursery transmission risks).
\end{itemize}

\subsection{Adolescent Immunization}\label{adolescent-immunization}

\begin{itemize}
\tightlist
\item
  HPV vaccine for girls aged 10--13.
\item
  Booster doses for waning childhood immunity (e.g., Tetanus).
\item
  Catch-up vaccination for missed/incomplete schedules.
\end{itemize}

\subsection{Pregnancy and Vaccination}\label{pregnancy-and-vaccination}

\begin{itemize}
\tightlist
\item
  Live viral vaccines generally avoided due to potential fetal risks.
\item
  Tdap recommended in the third trimester to protect against pertussis.
\end{itemize}

\section{Conclusion}\label{conclusion-16}

Immunization remains a cornerstone of preventive healthcare, reducing
the infectious disease burden globally. Maintaining vaccine quality,
storage protocols, and surveillance systems enhances safety and
efficacy. Addressing vaccine hesitancy, logistical challenges, and
misinformation remains crucial for improving immunization coverage.
Future advancements, including next-generation vaccines, aim to
strengthen global disease prevention efforts. This detailed narrative is
tailored for a professional audience and integrates key immunization
strategies, vaccine science, and best practices in public health. Let me
know if you need further elaboration on any aspect!

\textbf{References}

\begin{enumerate}
\def\labelenumi{\arabic{enumi}.}
\tightlist
\item
  \url{https://www.who.int/teams/immunization-vaccines-and-biologicals/policies/position-papers}
\item
  WHO Vaccine safety course:
  \href{https://sdgs.un.org/goals/goal3}{www.vaccine-safety-training.org}
\item
  \href{https://www.who.int/publications/i/item/immunization-in-practice-a-practical-guide-for-health-staff}{Immunization
  in Practice}
\item
  Pollard AJ, Bijker EM. A~guide~to~vaccinology: from basic principles
  to new developments. Nat Rev Immunol. 2021 Feb;21(2):83-100. doi:
  10.1038/s41577-020-00479-7. Epub 2020 Dec 22.PMID:~33353987
\item
  Bangura JB, Xiao S, Qiu D, Ouyang F, Chen L. Barriers to childhood
  immunization in sub-Saharan Africa: A systematic review. BMC Public
  Health. 2020 Jul 14;20(1):1108. doi: 10.1186/s12889-020-09169-4.
\end{enumerate}

\section{Practical work}\label{practical-work}

\subsection{Question}\label{question}

A 4-month-old baby is presented with fever, cough, and rhinorrhoea of 2
days duration. O/E:~ Active, healthy-looking child with occasional
smiles; axillary temp 37.8 °C.~ The baby is treated for the common
cold.~ Baby has no scar over the Left deltoid area and on enquiry has
visited the immunization clinic two times since birth, but has misplaced
the weighing card.

\begin{itemize}
\tightlist
\item
  Which vaccinations will the child require at that age?
\item
  Which vaccinations are contraindicated?
\item
  Which ones should the child receive before going home?
\item
  When should the child be returned for follow-up vaccinations? And for
  which vaccines?
\item
  What if the baby's mother is HIV-positive?
\item
  What if there is a parental history of SCD?
\end{itemize}

\subsection{Answers}\label{answers}

The baby should have received several routine vaccinations at four
months old according to Ghana's Expanded Programme on Immunisation
(EPI). Based on the standard schedule, the following vaccines are
typically required at this age:

\textbf{Vaccinations Required at Four Months}

\begin{itemize}
\tightlist
\item
  Oral Polio Vaccine (OPV) -- Third dose
\item
  Pentavalent Vaccine (DPT/HiB/HepB) -- Third dose (protects against
  diphtheria, pertussis, tetanus, Haemophilus influenzae type B, and
  hepatitis B)
\item
  Pneumococcal Conjugate Vaccine (PCV) -- Third dose
\item
  Rotavirus Vaccine -- Second dose
\end{itemize}

\textbf{Contraindicated Vaccines}

\begin{itemize}
\tightlist
\item
  Live vaccines such as BCG (for tuberculosis) and Measles-Rubella (MR)
  may be contraindicated if the child has certain immunodeficiencies,
  such as HIV/AIDS, or other medical conditions. However, specific
  contraindications should be assessed by a healthcare provider.
\end{itemize}

\textbf{Vaccines to Receive Before Going Home}

\begin{itemize}
\tightlist
\item
  Since the baby has visited the immunization clinic only twice and has
  no visible BCG scar, verifying which vaccines have been missed is
  crucial. Before discharge, the baby should receive any missed doses of
  the routine vaccines, particularly BCG if it was not previously
  administered.
\end{itemize}

\textbf{Follow-up Vaccinations and Schedule}

The baby should return for the next scheduled vaccinations:

\begin{itemize}
\tightlist
\item
  \textbf{At 6 months} -- Vitamin A supplementation
\item
  \textbf{At 9 months} -- Measles-Rubella (MR) and Yellow Fever vaccines
\item
  \textbf{At 12 months} -- Meningococcal vaccine (Men A) and second dose
  of Measles-Rubella (MR)
\end{itemize}

\textbf{Considerations for Special Cases}

\begin{itemize}
\tightlist
\item
  If the mother is HIV-positive: The baby may require additional
  monitoring and possible adjustments to the vaccination schedule. BCG
  may be contraindicated if the baby is symptomatic or severely
  immunocompromised.
\item
  If there is a parental history of Sickle Cell Disease (SCD): The baby
  should be screened for sickle cell status. Additional precautions may
  be needed if diagnosed with SCD, including early pneumococcal and
  meningococcal vaccines to prevent infections.
\end{itemize}

\chapter{Meningitis (bacterial)}\label{meningitis-bacterial}

\section{Definition}\label{definition-18}

Inflammation of the meninges due to bacterial infection. The onset of
symptoms is classified as acute (symptoms evolving rapidly over
1-24hrs), sub-acute (1-7 days), and Chronic (\textgreater{} 1 week).
Infants, children, and young adults are most likely to suffer from
bacterial meningitis.

\section{Incidence/prevalence}\label{incidenceprevalence-8}

An estimated 2.5 million cases of meningitis occur globally each year,
with approximately 250,000 deaths.(PATH 2021) In 2023, information from
LHIMS, Komfo Anokye Teaching Hospital, Kumasi indicated a rate of 1.5\%
or 15/1000 admissions through the Paediatric Emergency Unit (PEU) were
diagnosed as meningitis. Most of the KATH cases were not confirmed.

\section{Aetiology}\label{aetiology-10}

Several different bacteria can cause meningitis. \emph{Streptococcus
pneumoniae}, \emph{Haemophilus influenzae}, and \emph{Neisseria
meningitidis} are the most frequent ones. N. meningitidis, causing
meningococcal meningitis, has the potential to produce large epidemics.
12 serogroups of N. meningitidis have been identified, 6 of which (A, B,
C, W, X and Y) can cause epidemics.(Organization 2021) Viral, fungi and
Mycobacterium species can also cause meningitis. This write-up is
limited to bacteria.

\section{Pathogenesis}\label{pathogenesis-5}

The bacterial gain access to the central nervous system through

\begin{enumerate}
\def\labelenumi{\roman{enumi}.}
\tightlist
\item
  Invasion of mucosal surface (respiratory tract) then, hematogenous to
  the brain;
\item
  Spread from Para meningeal focus(otitis media, sinusitis); penetrating
  head trauma, and previous neurosurgical procedure.
\end{enumerate}

Bacterial meningitis is distinguished by the introduction of bacteria
into the cerebrospinal fluid (CSF) and the subsequent proliferation of
bacteria in this compartment, leading to inflammation both within the
CSF and in the brain tissue next to it. By production and/or release of
virulence factors into and stimulating the formation of inflammatory
cytokines within the central nervous system, meningeal pathogens
increase the permeability of the blood-brain barrier, thus allowing
protein and neutrophils to move into the subarachnoid space.(Hoffman and
Weber 2009)

\section{Signs and symptoms}\label{signs-and-symptoms-10}

\textbf{Infants:} Temperature instability, convulsions, meningeal
irritation (stiff neck, positive Kernig's sign, Positive Brudzinski's
sign), bulging fontanelles and increased head circumference are common
and may be late signs. Signs are very non-specific. Examine for spinal
or cranial abnormalities

\textbf{Older children}: Fever, headache, photophobia, Changes in mental
status (Irritability, lethargy, coma, and confusion), and End organ
dysfunction (Heart, Lung, Kidney, Liver). Meningeal irritation (neck
stiffness, positive Kernig's sign), cranial nerve palsies, and purpuric
rash -- meningococcal meningitis.

\section{Investigations}\label{investigations-15}

\textbf{Lumbar puncture}: White Blood Cells, Red Blood Cells, protein
content, Glucose content (2/3 blood glucose), Culture and sensitivity,
Serology (latex agglutination test), PCR

\begin{longtable}[]{@{}
  >{\raggedright\arraybackslash}p{(\linewidth - 8\tabcolsep) * \real{0.2245}}
  >{\raggedright\arraybackslash}p{(\linewidth - 8\tabcolsep) * \real{0.2551}}
  >{\raggedright\arraybackslash}p{(\linewidth - 8\tabcolsep) * \real{0.1939}}
  >{\raggedright\arraybackslash}p{(\linewidth - 8\tabcolsep) * \real{0.1633}}
  >{\raggedright\arraybackslash}p{(\linewidth - 8\tabcolsep) * \real{0.1633}}@{}}
\caption{Lumbar puncture findings in different
meningitis}\label{tbl-genetic-synds}\tabularnewline
\toprule\noalign{}
\begin{minipage}[b]{\linewidth}\raggedright
Item
\end{minipage} & \begin{minipage}[b]{\linewidth}\raggedright
Bacterial
\end{minipage} & \begin{minipage}[b]{\linewidth}\raggedright
Viral
\end{minipage} & \begin{minipage}[b]{\linewidth}\raggedright
Fungal
\end{minipage} & \begin{minipage}[b]{\linewidth}\raggedright
Tuberculous
\end{minipage} \\
\midrule\noalign{}
\endfirsthead
\toprule\noalign{}
\begin{minipage}[b]{\linewidth}\raggedright
Item
\end{minipage} & \begin{minipage}[b]{\linewidth}\raggedright
Bacterial
\end{minipage} & \begin{minipage}[b]{\linewidth}\raggedright
Viral
\end{minipage} & \begin{minipage}[b]{\linewidth}\raggedright
Fungal
\end{minipage} & \begin{minipage}[b]{\linewidth}\raggedright
Tuberculous
\end{minipage} \\
\midrule\noalign{}
\endhead
\bottomrule\noalign{}
\endlastfoot
\textbf{Opening pressure} & Elevated & Slightly elevated & Normal or
high & Unusually high \\
\textbf{Appearance} & Turbid & clear & Turbid & Cob-web \\
\textbf{Proteins} & Very high & Normal & High & High \\
\textbf{Glucose} & Low & Normal & Low & Low \\
\textbf{RBCs} & Few & None & None & None \\
\textbf{WBCs} & \textgreater200 & \textless200 & \textless50 & 20-30 \\
\textbf{Differential} & Polymorphonuclear cells & Monocytes & Monocytes
& Monocytes \\
\end{longtable}

\textbf{Blood}: Glucose, culture and sensitivity

\textbf{Imaging:} (CT Scan/MRI) helps identify: brain abscesses,
meningeal inflammation, infarction, haemorrhages, subdural effusion,
focal infections (sinusitis)

\section{\texorpdfstring{\textbf{Contraindications}}{Contraindications}}\label{contraindications}

Focal neurologic deficit or signs of increased intracranial pressure,
deep coma, protracted seizures, cranial nerve palsy, pupillary
dilatation, bleeding disorders, septic lesion at the site of LP.

\section{Treatment}\label{treatment-12}

\subsection{General and supportive
measures}\label{general-and-supportive-measures}

Close cardio-respiratory monitoring, frequent neurologic assessment,
strict fluid balance, frequent urine specific gravity assessment, nil
per os until neurologically stable, isolate until the organism is known,
daily weighing, and frequent BP monitoring may be needed, monitor and
treat for (hypoglycaemia, hyponatraemia, Acidosis, Septic shock, DIC,
Seizures, Increased intracranial pressure)

\subsection{Definitive treatment}\label{definitive-treatment}

Ceftriaxone is the drug of empiric choice beyond the neonatal period
(cefotaxime is preferred in the first 2 weeks of life). Modify after
culture and sensitivity results are available. Look out for focus if the
response to antibiotics is sub-optimal. Steroids such as dexamethasone
may be used depending on the organism isolated. Anticonvulsants
(phenobarbitone, diazepam, and midazolam) and analgesics may also be
required.

\subsection{Prophylaxis}\label{prophylaxis}

Close contacts especially for patients with Neisseria meningitidis
meningitis will need post-exposure prophylaxis preferably within 48
hours. Options are ciprofloxacin or rifampicin.

\subsection{\texorpdfstring{\textbf{Vaccines}}{Vaccines}}\label{vaccines}

Vaccines are available for use, especially during outbreaks

\section{Complications}\label{complications-15}

These include seizures, persistent focal seizures, neurological
deficits, cerebral oedema, visual impairment, ataxia, hearing loss,
hydrocephalus, cranial nerve palsy, mental retardation, severe
behavioural problems, syndrome of the inappropriate release of
antidiuretic hormone (SIADH), and vegetative state.

\section{Prognosis}\label{prognosis-14}

Even with timely, appropriate treatment, bacterial meningitis can be
fatal in 5 to 20\% of newborns and 5 to 15\% of older infants and
children.(Skar et al. 2024)

\section{Differential diagnosis}\label{differential-diagnosis-13}

Differential diagnoses include cerebral malaria, liver failure, brain
abscess, encephalitis, brain tumour, and subarachnoid haemorrhage.

\section{Sample questions}\label{sample-questions-1}

1. A neonate presented at 24 hours post-delivery with fever, floppiness,
and poor feeding. These were your CSF chemistry report

\begin{longtable}[]{@{}lllll@{}}
\toprule\noalign{}
Appearance & Proteins & Glucose & WBCs & Differentials \\
\midrule\noalign{}
\endhead
\bottomrule\noalign{}
\endlastfoot
Turbid & High & low & \textless50 & Monocytes \\
\end{longtable}

Which of the following diagnoses is most likely?

\begin{enumerate}
\def\labelenumi{\alph{enumi}.}
\tightlist
\item
  Bacterial
\item
  \textbf{Fungal}
\item
  Tuberculosis
\item
  Viral
\end{enumerate}

This CSF characteristic is more consistent with the fungal cause of
meningitis. There will be a need to investigate immunosuppression.

\begin{enumerate}
\def\labelenumi{\arabic{enumi}.}
\setcounter{enumi}{1}
\item
  You suspect a 10-year-old presenting with focal seizures and febrile
  illness had meningitis. You were not able to do a Lumbar puncture. You
  started treatment with ceftriaxone. Three days into treatment, the
  child still had a fever (39.7\textsuperscript{o}C) and focal seizures.
  What will be your best next step?

  \begin{enumerate}
  \def\labelenumii{\alph{enumii}.}
  \tightlist
  \item
    Increase the ceftriaxone dose
  \item
    Modify antipyretic dose
  \item
    Immediate lumbar puncture
  \item
    \textbf{Imaging of the head}
  \end{enumerate}

  Imaging (CT scan /MRI) will be the best option to help identify the
  focus of infection especially abscesses.
\end{enumerate}

\textbf{Practice question}

\begin{enumerate}
\def\labelenumi{\arabic{enumi}.}
\setcounter{enumi}{2}
\tightlist
\item
  You worked as Director of Public Health in a rural facility. There was
  a sudden increase in the number of students admitted to your facility
  from a particular secondary school with meningitis. Enumerate the
  steps you will take to stop the outbreak.
\end{enumerate}

\part{{Oncology}}

\chapter{General Principles}\label{general-principles-1}

\section{Introduction}\label{introduction-21}

Paediatric oncology is the branch of medicine that deals with the
diagnosis, treatment, and long-term follow-up of cancers in children and
adolescents. Although cancer is less common in children compared to
adults, it remains a significant cause of morbidity and mortality
worldwide. In many low- and middle-income countries, including those in
sub-Saharan Africa, paediatric cancers are increasingly recognised due
to improved awareness and diagnostic facilities.

Understanding the general principles of paediatric oncology is essential
for medical students, as it forms the foundation for appreciating the
biology, clinical behaviour, and management of childhood cancers.

\section{Epidemiology of Childhood
Cancers}\label{epidemiology-of-childhood-cancers}

\begin{itemize}
\tightlist
\item
  Childhood cancers account for about \textbf{1--4\% of all cancers
  worldwide}.\\
\item
  The incidence is approximately \textbf{100--150 cases per million
  children per year}.\\
\item
  In high-income countries, survival rates exceed \textbf{80\%}, but in
  low- and middle-income regions, survival may be \textbf{20--40\%} due
  to late presentation, limited resources, and treatment abandonment.\\
\item
  The most common paediatric cancers include:

  \begin{itemize}
  \tightlist
  \item
    \textbf{Leukaemias} (especially acute lymphoblastic leukaemia).\\
  \item
    \textbf{Brain tumours} (medulloblastoma, astrocytoma).\\
  \item
    \textbf{Lymphomas} (Burkitt's lymphoma, Hodgkin lymphoma).\\
  \item
    \textbf{Solid tumours} (Wilms' tumour, neuroblastoma,
    retinoblastoma).
  \end{itemize}
\end{itemize}

\section{Biology of Childhood
Cancers}\label{biology-of-childhood-cancers}

Unlike adult cancers, childhood malignancies:\\
- Often arise from \textbf{embryonal tissues} or primitive cells rather
than epithelial tissues.\\
- Show \textbf{fewer environmental associations} (e.g., smoking,
alcohol, carcinogens).\\
- Are more commonly associated with \textbf{genetic predispositions}
(e.g., RB1 mutations in retinoblastoma, TP53 in Li-Fraumeni syndrome).\\
- Tend to have \textbf{rapid growth rates}, making them highly
responsive to chemotherapy and radiotherapy.

\section{Clinical Presentation}\label{clinical-presentation-2}

Children with cancer may present with vague, non-specific symptoms that
mimic common infections. High suspicion is necessary.

\subsection{\texorpdfstring{General warning signs of cancer in children
(commonly remembered by the acronym \textbf{CHILD
CANCER}):}{General warning signs of cancer in children (commonly remembered by the acronym CHILD CANCER):}}\label{general-warning-signs-of-cancer-in-children-commonly-remembered-by-the-acronym-child-cancer}

\begin{itemize}
\tightlist
\item
  \textbf{C}ontinued, unexplained weight loss.\\
\item
  \textbf{H}eadaches with early morning vomiting.\\
\item
  \textbf{I}ncreased swelling or pain in bones/joints.\\
\item
  \textbf{L}ump or mass in abdomen, chest, or neck.\\
\item
  \textbf{D}evelopment of excessive bruising, bleeding, or rash.\\
\item
  \textbf{C}onstant infections.\\
\item
  \textbf{A} whitish glow in the eye (leukocoria).\\
\item
  \textbf{N}eurological symptoms (seizures, persistent dizziness).\\
\item
  \textbf{C}hanges in vision.\\
\item
  \textbf{E}nlarged lymph nodes or persistent fever.\\
\item
  \textbf{R}ecurrent unexplained fevers.
\end{itemize}

\section{Diagnosis and Staging}\label{diagnosis-and-staging}

Diagnosis of childhood cancer requires a \textbf{multidisciplinary
approach}.

\begin{enumerate}
\def\labelenumi{\arabic{enumi}.}
\tightlist
\item
  \textbf{Clinical evaluation} -- thorough history and examination, with
  attention to family history of cancers or syndromes.\\
\item
  \textbf{Laboratory tests} -- complete blood count, peripheral smear,
  biochemical markers (e.g., LDH, uric acid).\\
\item
  \textbf{Imaging} -- X-rays, ultrasound, CT, MRI, and PET scans
  depending on tumour location.\\
\item
  \textbf{Histopathology} -- biopsy for tissue diagnosis (except for
  retinoblastoma where clinical diagnosis is usually made).\\
\item
  \textbf{Molecular and cytogenetic studies} -- identification of
  chromosomal translocations (e.g., t(8;14) in Burkitt's lymphoma,
  t(12;21) in ALL).\\
\item
  \textbf{Staging} -- determines the extent of disease, using systems
  like:

  \begin{itemize}
  \tightlist
  \item
    Ann Arbor staging for lymphomas.\\
  \item
    INSS (International Neuroblastoma Staging System).\\
  \item
    TNM classification for some solid tumours.
  \end{itemize}
\end{enumerate}

\section{Principles of Treatment}\label{principles-of-treatment}

Treatment is multidisciplinary, involving oncologists, surgeons,
radiation oncologists, pathologists, radiologists, nurses, and
psychosocial support teams.

\subsection{\texorpdfstring{1.
\textbf{Surgery}}{1. Surgery}}\label{surgery}

\begin{itemize}
\tightlist
\item
  Plays a key role in diagnosis (biopsy) and treatment (resection of
  tumour).\\
\item
  Examples: nephrectomy in Wilms' tumour, enucleation in advanced
  retinoblastoma.
\end{itemize}

\subsection{\texorpdfstring{2.
\textbf{Chemotherapy}}{2. Chemotherapy}}\label{chemotherapy}

\begin{itemize}
\tightlist
\item
  Mainstay of treatment for most paediatric cancers.\\
\item
  Uses cytotoxic drugs targeting rapidly dividing cells.\\
\item
  Often given in \textbf{cycles} to allow normal tissues to recover.\\
\item
  Commonly used agents: vincristine, doxorubicin, cyclophosphamide,
  methotrexate, cytarabine.\\
\item
  Side effects: bone marrow suppression, alopecia, nausea, infections.
\end{itemize}

\subsection{\texorpdfstring{3.
\textbf{Radiotherapy}}{3. Radiotherapy}}\label{radiotherapy}

\begin{itemize}
\tightlist
\item
  Used in selected cancers (e.g., brain tumours, Hodgkin lymphoma).\\
\item
  Careful dosing required to avoid long-term growth and developmental
  complications.\\
\item
  Increasingly replaced by more precise modalities such as proton
  therapy where available.
\end{itemize}

\subsection{\texorpdfstring{4. \textbf{Stem Cell
Transplantation}}{4. Stem Cell Transplantation}}\label{stem-cell-transplantation}

\begin{itemize}
\tightlist
\item
  Indicated in high-risk or relapsed cases (e.g., relapsed leukaemia).\\
\item
  May involve autologous or allogeneic transplantation.
\end{itemize}

\subsection{\texorpdfstring{5. \textbf{Targeted Therapy and
Immunotherapy}}{5. Targeted Therapy and Immunotherapy}}\label{targeted-therapy-and-immunotherapy}

\begin{itemize}
\tightlist
\item
  Monoclonal antibodies (e.g., rituximab in B-cell lymphomas).\\
\item
  Tyrosine kinase inhibitors (e.g., imatinib in Philadelphia
  chromosome-positive ALL).\\
\item
  CAR-T cell therapy emerging in refractory cases.
\end{itemize}

\section{Supportive Care}\label{supportive-care-1}

Equally important as definitive treatment, supportive care ensures the
child tolerates therapy.

\begin{itemize}
\tightlist
\item
  \textbf{Infection prevention and treatment}: use of antibiotics,
  antifungals, and sometimes prophylaxis.\\
\item
  \textbf{Blood product support}: transfusions for anaemia and
  thrombocytopenia.\\
\item
  \textbf{Nutritional support}: maintaining adequate nutrition to aid
  recovery.\\
\item
  \textbf{Pain management}: opioids and adjuvants as required.\\
\item
  \textbf{Psychological support}: counselling for child and family.\\
\item
  \textbf{Management of treatment complications}: tumour lysis syndrome,
  neutropenic sepsis.
\end{itemize}

\section{Emergency Presentations in Paediatric
Oncology}\label{emergency-presentations-in-paediatric-oncology}

Certain cancer-related emergencies require immediate recognition and
intervention:\\
- \textbf{Febrile neutropenia} -- life-threatening infection during
chemotherapy-induced immunosuppression.\\
- \textbf{Tumour lysis syndrome} -- rapid cell breakdown causing
hyperkalaemia, hyperuricaemia, renal failure.\\
- \textbf{Mediastinal mass} -- airway compression in lymphomas or
leukaemia.\\
- \textbf{Spinal cord compression} -- neuroblastoma or vertebral
metastases.\\
- \textbf{Severe anaemia or bleeding} -- marrow infiltration by
leukaemia.

\section{Long-Term Follow-Up and
Survivorship}\label{long-term-follow-up-and-survivorship}

With improved survival, focus has shifted to long-term outcomes:\\
- \textbf{Late effects of therapy}:\\
- Growth retardation from cranial irradiation.\\
- Cardiomyopathy from anthracyclines.\\
- Infertility from alkylating agents.\\
- Secondary malignancies.\\
- \textbf{Rehabilitation and reintegration}: ensuring schooling and
social development.\\
- \textbf{Psychological support}: addressing anxiety, depression, and
stigma.

\section{Prevention and Early
Detection}\label{prevention-and-early-detection}

\begin{itemize}
\tightlist
\item
  Unlike adult cancers, \textbf{primary prevention is limited} in
  childhood cancers.\\
\item
  However, measures include:

  \begin{itemize}
  \tightlist
  \item
    Avoiding unnecessary exposure to ionising radiation in pregnancy and
    childhood.\\
  \item
    Vaccination against viruses that can indirectly influence cancer
    risk (e.g., HBV to reduce hepatocellular carcinoma).\\
  \item
    Screening in high-risk families with known cancer syndromes (e.g.,
    RB1 mutation carriers).\\
  \end{itemize}
\item
  Public health education on early signs of cancer is vital in improving
  outcomes in low-resource settings.
\end{itemize}

\section{Prognosis}\label{prognosis-15}

\begin{itemize}
\tightlist
\item
  Prognosis depends on:

  \begin{itemize}
  \tightlist
  \item
    Type of cancer.\\
  \item
    Stage at diagnosis.\\
  \item
    Response to therapy.\\
  \item
    Availability of supportive care.\\
  \end{itemize}
\item
  Survival is excellent in conditions like Hodgkin lymphoma
  (\textgreater90\%) but poorer in advanced neuroblastoma or
  late-presenting retinoblastoma.
\end{itemize}

\section{Conclusion}\label{conclusion-17}

Paediatric oncology integrates principles of cell biology, clinical
medicine, and multidisciplinary care. While outcomes have improved
remarkably in high-resource settings, challenges remain in low- and
middle-income countries, where late presentation and limited
infrastructure hinder survival. For medical students, an appreciation of
the \textbf{unique biology, presentation, and management} of childhood
cancers is essential in recognising cases early, guiding families, and
contributing to improved outcomes in resource-constrained environments.

\chapter{Oncological Emergencies}\label{oncological-emergencies}

\section{Introduction}\label{introduction-22}

Oncological emergencies are acute, potentially life-threatening
conditions that affect patients with cancer, either due to the
malignancy itself or as complications arising from its treatment. Early
recognition and prompt intervention are crucial in preventing morbidity
and mortality. Understanding these emergencies is essential for all
healthcare professionals, particularly in resource-limited settings like
Ghana, where delays in cancer diagnosis and treatment are prevalent.
Oncological emergencies are generally classified into three categories:

\begin{enumerate}
\def\labelenumi{\arabic{enumi}.}
\tightlist
\item
  \textbf{Metabolic Emergencies}
\item
  \textbf{Hematological Emergencies}
\item
  \textbf{Structural/Mechanical Emergencies}
\end{enumerate}

\section{Metabolic Emergencies}\label{metabolic-emergencies-1}

\subsection{Tumor Lysis Syndrome (TLS)}\label{tumor-lysis-syndrome-tls}

\textbf{Definition:} A life-threatening condition that occurs when
massive tumor cell lysis releases intracellular contents (potassium,
phosphate, uric acid) into the bloodstream, leading to acute kidney
injury and cardiac arrhythmias.

\textbf{Common Causes:}

\begin{itemize}
\tightlist
\item
  High-grade lymphomas (especially Burkitt lymphoma)
\item
  Acute leukemias (e.g., ALL)
\item
  Solid tumors with high tumor burden after chemotherapy
\end{itemize}

\textbf{Clinical Features:}

\begin{itemize}
\tightlist
\item
  Nausea and vomiting
\item
  Lethargy
\item
  Muscle cramps
\item
  Seizures
\item
  Oliguria or anuria
\item
  Arrhythmias
\end{itemize}

\textbf{Diagnostic Criteria (Cairo-Bishop):}

Laboratory TLS involves ≥2 of the following:

\begin{itemize}
\tightlist
\item
  Uric acid \textgreater{} 476 μmol/L
\item
  Potassium \textgreater{} 6.0 mmol/L
\item
  Phosphate \textgreater{} 1.45 mmol/L
\item
  Calcium \textless{} 1.75 mmol/L
\end{itemize}

\textbf{Management:}

\begin{itemize}
\tightlist
\item
  Aggressive IV hydratio
\item
  Allopurinol or rasburicase (rasburicase preferred)
\item
  Correction of electrolyte imbalances
\item
  Dialysis for refractory cases
\end{itemize}

\subsection{Hypercalcemia of
Malignancy}\label{hypercalcemia-of-malignancy}

\textbf{Definition:} Elevated serum calcium level (usually
\textgreater2.6 mmol/L) due to malignancy.

\textbf{Common Causes:}

\begin{itemize}
\tightlist
\item
  Breast cancer
\item
  Multiple myeloma
\item
  Lung cancer
\item
  Renal cell carcinoma
\item
  Parathyroid hormone-related protein (PTHrP) production
\end{itemize}

\textbf{Clinical Features:}

\begin{itemize}
\tightlist
\item
  Nausea, vomiting
\item
  Polyuria, polydipsia
\item
  Constipation
\item
  Confusion, coma
\item
  Shortened QT interval
\end{itemize}

\textbf{Management:}

\begin{itemize}
\tightlist
\item
  IV hydration with normal saline
\item
  Bisphosphonates (e.g., zoledronic acid)
\item
  Calcitonin for rapid reduction
\item
  Dialysis in severe cases
\end{itemize}

\subsection{Syndrome of Inappropriate Antidiuretic Hormone
(SIADH)}\label{syndrome-of-inappropriate-antidiuretic-hormone-siadh}

\textbf{Definition:} Excessive release of antidiuretic hormone leads to
water retention and hyponatremia.

\textbf{Common Causes:}

\begin{itemize}
\tightlist
\item
  Small cell lung carcinoma
\item
  CNS tumors
\end{itemize}

\textbf{Clinical Features:}

\begin{itemize}
\tightlist
\item
  Headache
\item
  Confusion
\item
  Seizures
\item
  Coma
\end{itemize}

\textbf{Management:}

\begin{itemize}
\tightlist
\item
  Fluid restriction
\item
  Hypertonic saline (3\%) in severe hyponatremia
\item
  Demeclocycline or vasopressin receptor antagonists in chronic cases
\end{itemize}

\section{Hematological Emergencies}\label{hematological-emergencies}

\subsection{Febrile Neutropenia}\label{febrile-neutropenia}

\textbf{Definition:} Fever (\textgreater38°C) with absolute neutrophil
count (ANC) \textless{} 0.5 × 10⁹/L in a cancer patient.

\textbf{Causes:}

\begin{itemize}
\tightlist
\item
  Chemotherapy-induced bone marrow suppression
\end{itemize}

\textbf{Clinical Features:}

\begin{itemize}
\tightlist
\item
  Fever (often the only sign)
\item
  Signs of infection may be subtle
\end{itemize}

\textbf{Management:}

\begin{itemize}
\tightlist
\item
  Broad-spectrum antibiotics within 1 hour (e.g., cefepime,
  piperacillin-tazobactam)
\item
  Risk stratification (MASCC score)
\item
  G-CSF in selected cases
\item
  Isolate and monitor closely
\end{itemize}

\subsection{Disseminated Intravascular Coagulation
(DIC)}\label{disseminated-intravascular-coagulation-dic}

\textbf{Definition:} Widespread activation of the coagulation system
leads to the consumption of clotting factors and platelets, resulting in
bleeding and thrombosis.

\textbf{Common Causes:}

\begin{itemize}
\tightlist
\item
  Acute promyelocytic leukemia (APL)
\item
  Metastatic cancers
\end{itemize}

\textbf{Clinical Features:}

\begin{itemize}
\tightlist
\item
  Bleeding (petechiae, ecchymosis, mucosal bleeding)
\item
  Thrombosis
\item
  Organ dysfunction
\end{itemize}

\textbf{Laboratory Findings:}

\begin{itemize}
\tightlist
\item
  Prolonged PT, aPTT
\item
  Low fibrinogen
\item
  Elevated D-dimer
\item
  Thrombocytopenia
\end{itemize}

\textbf{Management:}

\begin{itemize}
\tightlist
\item
  Treat underlying cause (e.g., ATRA for APL)
\item
  Transfusions (platelets, FFP)
\item
  Heparin in cases with thrombosis
\end{itemize}

\subsection{Hyperviscosity Syndrome}\label{hyperviscosity-syndrome}

\textbf{Definition:} Increased blood viscosity due to elevated cellular
or protein components.

\textbf{Common Causes:}

\begin{itemize}
\tightlist
\item
  Waldenström's macroglobulinemia (IgM)
\item
  Multiple myeloma
\item
  Leukemia with very high WBC
\end{itemize}

\textbf{Clinical Features:}

\begin{itemize}
\tightlist
\item
  Visual disturbances
\item
  Headache
\item
  Mucosal bleeding
\item
  Confusion
\item
  Heart failure
\end{itemize}

\textbf{Management:}

\begin{itemize}
\tightlist
\item
  Plasmapheresis
\item
  Hydration
\item
  Treat underlying cancer
\end{itemize}

\section{Structural/Mechanical
Emergencies}\label{structuralmechanical-emergencies}

\subsection{Superior Vena Cava (SVC)
Syndrome}\label{superior-vena-cava-svc-syndrome}

\textbf{Definition:}

Obstruction of blood flow through the superior vena cava, commonly due
to external compression by tumors.

\textbf{Common Causes:}

\begin{itemize}
\tightlist
\item
  Small cell lung cancer
\item
  Non-Hodgkin lymphoma
\item
  Metastatic mediastinal tumors
\end{itemize}

\textbf{Clinical Features:}

\begin{itemize}
\tightlist
\item
  Facial and upper limb swelling
\item
  Dyspnea
\item
  Distended neck veins
\item
  Cyanosis
\item
  Cough and hoarseness
\end{itemize}

\textbf{Diagnosis:}

\begin{itemize}
\tightlist
\item
  Chest X-ray: mediastinal widening
\item
  CT scan: to confirm compression
\item
  Biopsy of mass (if unknown etiology)
\end{itemize}

\textbf{Management:}

\begin{itemize}
\tightlist
\item
  Elevate the head
\item
  Steroids to reduce edema
\item
  Radiotherapy or chemotherapy, depending on etiology
\item
  Stenting in severe cases
\end{itemize}

\subsection{Spinal Cord Compression}\label{spinal-cord-compression}

\textbf{Definition:}

Compression of the spinal cord due to a tumor, leading to neurological
deficits.

\textbf{Common Causes:}

\begin{itemize}
\tightlist
\item
  Breast, prostate, and lung cancers
\item
  Lymphomas
\item
  Myeloma
\end{itemize}

\textbf{Clinical Features:}

\begin{itemize}
\tightlist
\item
  Back pain (worsened by lying down)
\item
  Weakness in limbs
\item
  Sensory loss
\item
  Bladder/bowel incontinence
\end{itemize}

\textbf{Diagnosis:}

\begin{itemize}
\tightlist
\item
  MRI spine (preferred)
\item
  Neurological exam
\end{itemize}

\textbf{Management:}

\begin{itemize}
\tightlist
\item
  High-dose corticosteroids (e.g., dexamethasone)
\item
  Emergency radiotherapy or surgery
\item
  Rehabilitation
\end{itemize}

\subsection{Pericardial Tamponade}\label{pericardial-tamponade}

\textbf{Definition:}

Accumulation of fluid in the pericardial sac impairs cardiac output.

\textbf{Common Causes:}

\begin{itemize}
\tightlist
\item
  Lung and breast cancers
\item
  Lymphomas
\item
  Metastatic cancers
\end{itemize}

\textbf{Clinical Features:}

\begin{itemize}
\tightlist
\item
  Dyspnea
\item
  Chest discomfort
\item
  Hypotension
\item
  Elevated JVP
\item
  Muffled heart sounds (Beck's triad)
\end{itemize}

\textbf{Diagnosis:}

\begin{itemize}
\tightlist
\item
  Echocardiography: diagnostic
\item
  ECG: low-voltage QRS or electrical alternans
\end{itemize}

\textbf{Management:}

\begin{itemize}
\tightlist
\item
  Urgent pericardiocentesis
\item
  Fluid resuscitation
\item
  Treat underlying malignancy
\end{itemize}

\subsection{Intestinal Obstruction}\label{intestinal-obstruction}

\textbf{Definition:} Partial or complete blockage of the bowel lumen.

\textbf{Common Causes:}

\begin{itemize}
\tightlist
\item
  Colorectal cancer
\item
  Ovarian cancer
\item
  Gastric cance
\item
  Peritoneal metastases
\end{itemize}

\textbf{Clinical Features:}

\begin{itemize}
\tightlist
\item
  Abdominal distension
\item
  Vomiting
\item
  Constipation
\item
  Colicky abdominal pain
\end{itemize}

\textbf{Diagnosis:}

\begin{itemize}
\tightlist
\item
  Abdominal X-ray or CT scan
\end{itemize}

\textbf{Management:}

\begin{itemize}
\tightlist
\item
  Nasogastric decompression
\item
  IV fluids and electrolytes
\item
  Surgery if obstruction is complete or complications arise
\item
  Stenting in selected cases
\end{itemize}

\section{Increased Intracranial Pressure
(ICP)}\label{increased-intracranial-pressure-icp}

\textbf{Causes:}

\begin{itemize}
\tightlist
\item
  Brain metastases (lung, breast, melanoma)
\item
  Primary CNS tumors
\item
  Leptomeningeal disease
\end{itemize}

\textbf{Clinical Features:}

\begin{itemize}
\tightlist
\item
  Headache
\item
  Vomiting (projectile)
\item
  Seizures
\item
  Altered mental status
\item
  Papilledema
\end{itemize}

\textbf{Diagnosis:}

\begin{itemize}
\tightlist
\item
  Brain imaging (CT or MRI)
\end{itemize}

\textbf{Management:}

\begin{itemize}
\tightlist
\item
  Corticosteroids (dexamethasone)
\item
  Mannitol for acute relief
\item
  Neurosurgical consultation
\item
  Radiotherapy/chemotherapy, depending on the cause
\end{itemize}

\section{Approach to the Patient with an Oncological
Emergency}\label{approach-to-the-patient-with-an-oncological-emergency}

\textbf{1. ABCDE Approach}

\begin{itemize}
\tightlist
\item
  \textbf{Airway:} Ensure patency, especially in patients with superior
  vena cava syndrome or airway tumors.
\item
  \textbf{Breathing:} Provide oxygen if hypoxic.
\item
  \textbf{Circulation:} Monitor for signs of shock (e.g., tamponade,
  disseminated intravascular coagulation).
\item
  \textbf{Disability:} Assess for neurological compromise (e.g., spinal
  cord compression, raised ICP).
\item
  \textbf{Exposure:} Full examination to identify other signs (e.g.,
  petechiae, masses).
\end{itemize}

\textbf{2. Laboratory and Imaging}

\begin{itemize}
\tightlist
\item
  CBC, U\&E, calcium, phosphate, uric acid
\item
  Coagulation profile
\item
  ECG and echocardiography
\item
  CT/MRI depending on clinical suspicion
\end{itemize}

\textbf{3. Specialist Referral}

\begin{itemize}
\tightlist
\item
  Oncology, surgery, radiotherapy, hematology, or palliative care,
  depending on diagnosis.
\end{itemize}

\textbf{Challenges in the Ghanaian Setting}

\begin{itemize}
\tightlist
\item
  Limited access to imaging (CT/MRI)
\item
  Delays in diagnosis and referral
\item
  Shortage of oncologists and hematologists
\item
  Limited availability of drugs (e.g., rasburicase, bisphosphonates)
\item
  Inadequate supportive care facilities (ICU, dialysis)
\end{itemize}

\textbf{Summary Table: Common Oncological Emergencies}

\begin{longtable}[]{@{}
  >{\raggedright\arraybackslash}p{(\linewidth - 4\tabcolsep) * \real{0.2404}}
  >{\raggedright\arraybackslash}p{(\linewidth - 4\tabcolsep) * \real{0.3942}}
  >{\raggedright\arraybackslash}p{(\linewidth - 4\tabcolsep) * \real{0.3654}}@{}}
\toprule\noalign{}
\endhead
\bottomrule\noalign{}
\endlastfoot
\textbf{Emergency} & \textbf{Main Feature} & \textbf{Key Management} \\
Tumor Lysis Syndrome & Electrolyte disturbances, renal failure &
Hydration, rasburicase \\
Hypercalcemia & Confusion, constipation & Hydration, bisphosphonate \\
SIAD & Hyponatremia, confusion & Fluid restriction, hypertonic saline \\
Febrile Neutropenia & Fever in neutropenia & Broad-spectrum
antibiotics \\
DIC & Bleeding, low platelets & Treat the cause, transfusion \\
SVC Syndrome & Facial swelling, JVP & Steroids, radiotherapy \\
Spinal Cord Compression & Back pain, limb weakness & Steroids, MRI,
radiotherapy \\
Tamponade & Hypotension, JVP & Pericardiocentesis \\
Bowel Obstruction & Abdominal pain, vomiting & NG tube, fluids, and
surgery \\
Increased ICP & Headache, vomiting & Steroids, mannitol, imaging \\
\end{longtable}

\textbf{Conclusion}

Oncological emergencies require prompt identification and urgent
management to prevent irreversible complications or death. In Ghana,
with rising cancer incidence and limited resources, medical students and
junior doctors must be adept at recognizing early signs and initiating
life-saving interventions. Close collaboration with oncology, radiology,
and surgical teams is essential for optimal outcomes.

\chapter{Leukaemia}\label{leukaemia}

Leukaemia is the most common childhood malignancy, representing about
one-third of all cancers diagnosed in children. It is a malignant
disorder of the blood and bone marrow characterised by uncontrolled
proliferation of abnormal white blood cells. These immature cells crowd
the marrow, impairing the production of normal blood cells and leading
to anaemia, thrombocytopenia, and neutropenia. For medical students,
understanding leukaemia requires an integrated view of its epidemiology,
aetiology, pathophysiology, clinical presentation, diagnosis, treatment,
and outcomes.

\section{Introduction}\label{introduction-23}

Leukaemia in children is a heterogeneous group of disorders arising from
the malignant transformation of haematopoietic precursor cells. It is
broadly divided into \textbf{acute lymphoblastic leukaemia (ALL)}, the
most common type, and \textbf{acute myeloid leukaemia (AML)}. Chronic
leukaemias are rare in childhood. The disease disrupts normal bone
marrow function and causes systemic manifestations due to infiltration
of organs by leukaemic cells.

\section{Incidence and Prevalence}\label{incidence-and-prevalence-2}

\begin{itemize}
\tightlist
\item
  Childhood leukaemia accounts for approximately \textbf{25--35\% of all
  paediatric cancers}.
\item
  The global incidence is about \textbf{3--4 per 100,000 children per
  year}.
\item
  ALL is more common than AML, with a peak incidence between ages
  \textbf{2 and 5 years}.
\item
  There is a slight male predominance and variation across ethnicities,
  with higher rates in high-income countries.
\item
  Improvements in treatment have markedly increased survival, with
  5-year survival rates for ALL exceeding \textbf{80\%} in developed
  settings.
\end{itemize}

\section{Aetiology}\label{aetiology-11}

The exact cause of leukaemia remains unclear, but multiple interacting
factors have been implicated:

\begin{itemize}
\tightlist
\item
  \textbf{Genetic predisposition}:

  \begin{itemize}
  \tightlist
  \item
    Children with syndromes such as Down syndrome, Li-Fraumeni syndrome,
    and Fanconi anaemia are at higher risk.
  \end{itemize}
\item
  \textbf{Chromosomal abnormalities}: Translocations such as
  \emph{t(12;21)} in ALL or \emph{t(8;21)} in AML are common.
\item
  \textbf{Environmental factors}: Ionising radiation, certain
  chemotherapeutic agents, and exposure to benzene.
\item
  \textbf{Infections and immune dysregulation}: Delayed immune
  development and abnormal immune responses to infections have been
  proposed.
\item
  \textbf{Familial risk}: Having a sibling with leukaemia increases the
  risk modestly.
\end{itemize}

\section{Pathophysiology}\label{pathophysiology-14}

Leukaemia results from genetic mutations in haematopoietic stem or
progenitor cells leading to:

\begin{enumerate}
\def\labelenumi{\arabic{enumi}.}
\tightlist
\item
  \textbf{Uncontrolled proliferation} of abnormal blasts.
\item
  \textbf{Failure of differentiation}, with accumulation of immature
  cells.
\item
  \textbf{Bone marrow failure}, causing:

  \begin{itemize}
  \tightlist
  \item
    Anaemia → fatigue, pallor.
  \item
    Neutropenia → infections.
  \item
    Thrombocytopenia → bleeding.
  \end{itemize}
\item
  \textbf{Tissue infiltration} by leukaemic cells:

  \begin{itemize}
  \tightlist
  \item
    Hepatosplenomegaly.
  \item
    Lymphadenopathy.
  \item
    CNS involvement (headache, vomiting, cranial nerve palsies).
  \item
    Bone pain from marrow expansion.
  \end{itemize}
\end{enumerate}

\section{Signs and Symptoms}\label{signs-and-symptoms-11}

Children typically present with non-specific symptoms, making early
recognition challenging.

\begin{itemize}
\tightlist
\item
  \textbf{General symptoms}: Fatigue, fever, anorexia, weight loss.
\item
  \textbf{Bone marrow failure manifestations}:

  \begin{itemize}
  \tightlist
  \item
    Pallor, tachycardia, and lethargy from anaemia.
  \item
    Easy bruising, petechiae, and mucosal bleeding from
    thrombocytopenia.
  \item
    Recurrent infections due to neutropenia.
  \end{itemize}
\item
  \textbf{Organ infiltration}:

  \begin{itemize}
  \tightlist
  \item
    Lymphadenopathy, hepatomegaly, splenomegaly.
  \item
    Bone or joint pain, limp.
  \item
    CNS signs: vomiting, seizures, headaches.
  \item
    Testicular enlargement (especially in ALL).
  \end{itemize}
\end{itemize}

\section{Differential Diagnosis}\label{differential-diagnosis-14}

Conditions that mimic leukaemia include:

\begin{itemize}
\tightlist
\item
  \textbf{Aplastic anaemia}.
\item
  \textbf{Infectious causes}: EBV, HIV, tuberculosis.
\item
  \textbf{Other malignancies}: Lymphomas, neuroblastoma.
\item
  \textbf{Rheumatological disorders}: Juvenile idiopathic arthritis.
\item
  \textbf{Storage disorders} with hepatosplenomegaly.
\end{itemize}

\section{Investigations}\label{investigations-16}

Workup includes both laboratory and imaging studies:

\begin{itemize}
\tightlist
\item
  \textbf{Initial tests}:

  \begin{itemize}
  \tightlist
  \item
    Full blood count (FBC) often shows anaemia, thrombocytopenia,
    leukocytosis or leukopenia.
  \item
    Blood film reveals circulating blasts.
  \end{itemize}
\item
  \textbf{Confirmatory tests}:

  \begin{itemize}
  \tightlist
  \item
    Bone marrow aspiration and biopsy showing \textgreater25\% blasts.
  \item
    Flow cytometry for immunophenotyping (B-cell vs T-cell ALL, AML
    subtypes).
  \end{itemize}
\item
  \textbf{Cytogenetics and molecular studies}: Prognostic significance
  (e.g., \emph{t(9;22)} Philadelphia chromosome).
\item
  \textbf{Additional workup}:

  \begin{itemize}
  \tightlist
  \item
    Lumbar puncture for CNS involvement.
  \item
    Chest X-ray to check for mediastinal mass.
  \item
    Biochemistry: uric acid, LDH, renal and liver function.
  \end{itemize}
\end{itemize}

\section{Treatment}\label{treatment-13}

Management of leukaemia is complex and requires a multidisciplinary
team. It can be categorised into stages:

\subsection{Emergency Management (at
presentation)}\label{emergency-management-at-presentation}

\begin{itemize}
\tightlist
\item
  Stabilisation: Manage anaemia, thrombocytopenia, and infections.
\item
  Blood product support: Packed RBCs, platelets.
\item
  Treatment of tumour lysis syndrome: Hydration, allopurinol or
  rasburicase.
\item
  Empirical antibiotics for febrile neutropenia.
\end{itemize}

\subsection{Ongoing Management (definitive
therapy)}\label{ongoing-management-definitive-therapy}

\begin{itemize}
\tightlist
\item
  \textbf{Chemotherapy} is the backbone of treatment:

  \begin{itemize}
  \tightlist
  \item
    Induction → achieve remission.
  \item
    Consolidation/intensification → eradicate residual disease.
  \item
    Maintenance → prevent relapse.
  \end{itemize}
\item
  \textbf{CNS prophylaxis} with intrathecal methotrexate.
\item
  AML requires more intensive regimens.
\item
  Targeted therapies (e.g., tyrosine kinase inhibitors for BCR-ABL
  positive ALL).
\end{itemize}

\subsection{3. Preparation for
Discharge}\label{preparation-for-discharge}

\begin{itemize}
\tightlist
\item
  Education of caregivers about infection prevention, medication
  adherence, and follow-up.
\item
  Arrangements for outpatient chemotherapy and monitoring.
\item
  Psychosocial support for the child and family.
\end{itemize}

\subsection{4. Long-Term Management}\label{long-term-management}

\begin{itemize}
\tightlist
\item
  Monitoring for relapse with clinical exam and minimal residual disease
  testing.
\item
  Managing late effects of chemotherapy: growth retardation,
  infertility, cardiotoxicity.
\item
  Vaccinations and infection prophylaxis.
\item
  Consideration of stem cell transplant in high-risk or relapsed cases.
\end{itemize}

\section{Complications}\label{complications-16}

\begin{itemize}
\tightlist
\item
  \textbf{Early}: Tumour lysis syndrome, febrile neutropenia, bleeding,
  sepsis.
\item
  \textbf{During therapy}: Chemotherapy toxicity (mucositis,
  hepatotoxicity, cardiotoxicity).
\item
  \textbf{Late}: Relapse, secondary malignancies, growth and endocrine
  abnormalities, learning difficulties.
\end{itemize}

\section{Prevention}\label{prevention-7}

\begin{itemize}
\tightlist
\item
  Currently, there are no definitive preventive strategies for most
  cases.
\item
  Avoidance of unnecessary radiation and known chemical carcinogens is
  recommended.
\item
  Genetic counselling for families with hereditary cancer syndromes.
\end{itemize}

\section{Prognosis}\label{prognosis-16}

\begin{itemize}
\tightlist
\item
  Prognosis depends on age, initial white cell count, cytogenetic
  abnormalities, and response to therapy.
\item
  \textbf{ALL}: 5-year survival \textgreater80\% in developed countries,
  lower in resource-limited settings.
\item
  \textbf{AML}: Lower survival (\textasciitilde60\%), requires more
  intensive therapy.
\item
  Relapse remains a major challenge, with outcomes poorer after
  recurrence.
\end{itemize}

\section{Conclusion}\label{conclusion-18}

Childhood leukaemia, though the most common paediatric cancer, is a
highly treatable condition with modern chemotherapy protocols. A good
understanding of its presentation, investigations, and treatment
approach is essential for practitioners, particularly in Ghana, where
delayed diagnosis and limited resources pose challenges. With early
recognition, appropriate supportive care, and treatment adherence,
survival rates continue to improve globally.

\chapter{Lymphoma}\label{lymphoma}

Lymphomas are malignant neoplasms arising from the lymphoid tissues and
are the \textbf{third most common childhood cancer} after leukaemia and
brain tumours. They account for approximately 10--15\% of childhood
malignancies. Lymphomas are broadly classified into \textbf{Hodgkin
lymphoma (HL)} and \textbf{non-Hodgkin lymphoma (NHL)}, with the latter
being more frequent in children. Among NHL subtypes, \textbf{Burkitt's
lymphoma} is particularly common in tropical Africa, including Ghana,
where it is one of the leading childhood cancers.

Understanding the epidemiology, aetiology, pathophysiology, clinical
features, and treatment of childhood lymphoma is vital for medical
students, as early diagnosis and timely intervention significantly
improve outcomes.

\section{Introduction}\label{introduction-24}

Lymphoma is a malignancy of the lymphoid system, originating from either
\textbf{B lymphocytes} or \textbf{T lymphocytes} at various stages of
differentiation. In children, the disease behaves more aggressively
compared to adult lymphomas, but it is also more curable with modern
treatment protocols.

\begin{itemize}
\tightlist
\item
  \textbf{Hodgkin lymphoma (HL)}: Characterised by the presence of
  Reed-Sternberg cells. Typically presents in older children and
  adolescents.
\item
  \textbf{Non-Hodgkin lymphoma (NHL)}: Includes Burkitt's lymphoma,
  lymphoblastic lymphoma, and large cell lymphoma. These are high-grade
  tumours with rapid proliferation.
\end{itemize}

\section{Incidence and Prevalence}\label{incidence-and-prevalence-3}

\begin{itemize}
\tightlist
\item
  Lymphomas account for 10--15\% of childhood cancers worldwide.
\item
  NHL is more common in children than HL, particularly in those under 10
  years of age.
\item
  In sub-Saharan Africa, \textbf{Burkitt's lymphoma is endemic} and
  accounts for up to \textbf{50\% of childhood cancers in some regions}.
\item
  Peak age for Burkitt's lymphoma: \textbf{5--10 years}.
\item
  Hodgkin lymphoma is less common in African children but occurs
  worldwide, often peaking in adolescence.
\item
  There is a slight male predominance, particularly in the case of NHL.
\end{itemize}

\section{Aetiology}\label{aetiology-12}

The development of lymphoma is multifactorial and involves both genetic
and environmental influences.

\begin{itemize}
\tightlist
\item
  \textbf{Genetic predisposition}: Mutations affecting oncogenes and
  tumour suppressor genes (e.g., \emph{MYC} translocation in Burkitt's
  lymphoma).
\item
  \textbf{Infectious agents}:

  \begin{itemize}
  \tightlist
  \item
    Epstein--Barr virus (EBV) is strongly associated with Burkitt's
    lymphoma and some cases of HL.
  \item
    Human immunodeficiency virus (HIV) predisposes to NHL.
  \end{itemize}
\item
  \textbf{Immunodeficiency states}: Congenital (e.g., Wiskott--Aldrich
  syndrome) or acquired (HIV/AIDS, post-transplant immunosuppression).
\item
  \textbf{Environmental factors}: Chronic malaria infection in endemic
  regions contributes to immune dysregulation, facilitating EBV-driven
  oncogenesis in Burkitt's lymphoma.
\end{itemize}

\section{Pathophysiology}\label{pathophysiology-15}

Lymphomas arise from \textbf{clonal proliferation of lymphoid cells}.

\begin{itemize}
\tightlist
\item
  \textbf{Hodgkin lymphoma}:

  \begin{itemize}
  \tightlist
  \item
    Originates from germinal centre B-cells that become transformed into
    Reed-Sternberg cells.
  \item
    These cells secrete cytokines that recruit inflammatory cells,
    explaining the prominent systemic symptoms.
  \end{itemize}
\item
  \textbf{Non-Hodgkin lymphoma}:

  \begin{itemize}
  \tightlist
  \item
    High-grade and rapidly proliferating.
  \item
    Burkitt's lymphoma is characterised by a translocation involving the
    \emph{MYC} gene on chromosome 8 (t(8;14) most common).
  \item
    Malaria-induced chronic immune stimulation reduces T-cell control
    over EBV-infected B-cells, facilitating malignant transformation.
  \end{itemize}
\item
  \textbf{Organ infiltration}: Lymphomas can spread to extranodal sites
  such as the bone marrow, CNS, and abdominal viscera.
\item
  The hallmark of Burkitt's lymphoma in Africa is \textbf{jaw
  involvement}, though abdominal presentations are also common.
\end{itemize}

\section{Signs and Symptoms}\label{signs-and-symptoms-12}

The presentation of lymphoma varies depending on the subtype and site of
involvement.

\begin{itemize}
\tightlist
\item
  \textbf{General features}:

  \begin{itemize}
  \tightlist
  \item
    Fever, weight loss, night sweats (``B symptoms'').
  \item
    Fatigue, anorexia.
  \end{itemize}
\item
  \textbf{Hodgkin lymphoma}:

  \begin{itemize}
  \tightlist
  \item
    Painless lymphadenopathy (often cervical or supraclavicular).
  \item
    Mediastinal mass causing cough, dyspnoea, or SVC obstruction.
  \item
    Hepatosplenomegaly.
  \end{itemize}
\item
  \textbf{Non-Hodgkin lymphoma}:

  \begin{itemize}
  \tightlist
  \item
    Rapidly enlarging lymph nodes, often extranodal.
  \item
    Abdominal involvement: distension, pain, palpable mass,
    intussusception, or bowel obstruction.
  \item
    CNS infiltration: seizures, cranial nerve palsies, spinal cord
    compression.\\
  \item
    \textbf{Burkitt's lymphoma}:

    \begin{itemize}
    \tightlist
    \item
      Endemic type: jaw/facial bone swelling, often bilateral.
    \item
      Sporadic type: abdominal masses, ileocecal involvement.
    \end{itemize}
  \end{itemize}
\end{itemize}

\section{Differential Diagnosis}\label{differential-diagnosis-15}

Conditions that mimic childhood lymphoma include:

\begin{itemize}
\tightlist
\item
  \textbf{Infectious diseases}: Tuberculosis, HIV lymphadenopathy, EBV
  infection.
\item
  \textbf{Other malignancies}: Leukaemia, neuroblastoma, Wilms' tumour.
\item
  \textbf{Rheumatological disorders}: Juvenile idiopathic arthritis,
  systemic lupus erythematosus.
\item
  \textbf{Benign causes of lymphadenopathy}: Reactive hyperplasia,
  cat-scratch disease.
\end{itemize}

\section{Investigations}\label{investigations-17}

Evaluation of suspected lymphoma requires a combination of laboratory,
imaging, and histological studies.

\begin{itemize}
\tightlist
\item
  \textbf{Laboratory tests}:

  \begin{itemize}
  \tightlist
  \item
    Full blood count: may reveal anaemia, cytopenias if marrow
    involvement.
  \item
    ESR and LDH: often elevated.
  \item
    Uric acid and renal function: to assess for tumour lysis risk.
  \end{itemize}
\item
  \textbf{Imaging}:

  \begin{itemize}
  \tightlist
  \item
    Chest X-ray: mediastinal mass.
  \item
    Ultrasound/CT/MRI: delineate abdominal or nodal masses.
  \end{itemize}
\item
  \textbf{Histology}:

  \begin{itemize}
  \tightlist
  \item
    Excisional lymph node biopsy is the gold standard.
  \item
    Reed-Sternberg cells → Hodgkin lymphoma.
  \item
    ``Starry sky'' appearance → Burkitt's lymphoma.
  \end{itemize}
\item
  \textbf{Bone marrow aspiration and biopsy}: To check for
  infiltration.\\
\item
  \textbf{Lumbar puncture}: Especially for Burkitt's lymphoma and
  lymphoblastic lymphoma to detect CNS disease.
\end{itemize}

\section{Treatment}\label{treatment-14}

Management depends on subtype, stage, and extent of disease.
Multidisciplinary care is essential.

\subsection{Emergency Management}\label{emergency-management}

\begin{itemize}
\tightlist
\item
  Stabilise airway, breathing, circulation.
\item
  Manage tumour lysis syndrome: hydration, allopurinol or rasburicase.
\item
  Empirical antibiotics for febrile neutropenia.
\item
  Blood products as required.
\end{itemize}

\subsection{Definitive (Ongoing)
Therapy}\label{definitive-ongoing-therapy}

\begin{itemize}
\tightlist
\item
  \textbf{Chemotherapy} is the mainstay:

  \begin{itemize}
  \tightlist
  \item
    HL: ABVD (adriamycin, bleomycin, vinblastine, dacarbazine) or
    equivalent protocols.
  \item
    NHL: intensive multiagent regimens (e.g., cyclophosphamide,
    vincristine, doxorubicin, methotrexate, cytarabine).
  \item
    Burkitt's lymphoma responds dramatically to short, intensive
    chemotherapy cycles.\\
  \end{itemize}
\item
  \textbf{Radiotherapy}: Occasionally used in HL but less so in children
  due to long-term side effects.
\item
  \textbf{CNS prophylaxis}: Intrathecal chemotherapy for NHL.
\item
  \textbf{Stem cell transplantation}: Considered in refractory or
  relapsed cases.
\end{itemize}

\subsection{Preparation for
Discharge}\label{preparation-for-discharge-1}

\begin{itemize}
\tightlist
\item
  Educate caregivers on infection prevention, adherence to chemotherapy,
  and recognition of complications.
\item
  Ensure follow-up schedules are clear.
\item
  Provide psychosocial and nutritional support.
\end{itemize}

\subsection{Long-Term Management}\label{long-term-management-1}

\begin{itemize}
\tightlist
\item
  Monitor for relapse with clinical examination and imaging as
  indicated.
\item
  Surveillance for late effects: growth disturbances, infertility,
  cardiotoxicity, secondary malignancies.
\item
  Ongoing psychosocial support and school reintegration.
\end{itemize}

\section{Complications}\label{complications-17}

\begin{itemize}
\tightlist
\item
  \textbf{Early}: Tumour lysis syndrome, airway obstruction from
  mediastinal masses, sepsis.
\item
  \textbf{During therapy}: Chemotherapy toxicities (mucositis,
  myelosuppression, cardiotoxicity).
\item
  \textbf{Late}: Relapse, secondary cancers, endocrine dysfunction,
  infertility, psychosocial issues.
\end{itemize}

\section{Prevention}\label{prevention-8}

\begin{itemize}
\tightlist
\item
  No definitive primary prevention strategies exist.
\item
  Reducing malaria transmission may indirectly lower Burkitt's lymphoma
  incidence.
\item
  HIV prevention and treatment help reduce NHL burden.
\item
  Early recognition and referral are key to improving survival.
\end{itemize}

\section{Prognosis}\label{prognosis-17}

\begin{itemize}
\tightlist
\item
  Prognosis depends on subtype, stage, and response to therapy.
\item
  \textbf{Hodgkin lymphoma}: Excellent prognosis with \textgreater90\%
  5-year survival in early-stage disease.
\item
  \textbf{Non-Hodgkin lymphoma}: Cure rates of 70--90\% with appropriate
  therapy.
\item
  \textbf{Burkitt's lymphoma}: Rapidly fatal if untreated, but highly
  curable with intensive short-course chemotherapy. Survival is
  significantly improved when diagnosed early and managed promptly.
\end{itemize}

\section{Conclusion}\label{conclusion-19}

Childhood lymphoma is a significant health problem, particularly in
sub-Saharan Africa, where \textbf{Burkitt's lymphoma is endemic}. It is
an aggressive but highly treatable malignancy. For medical students, key
learning points include recognition of clinical presentations,
understanding the role of EBV and malaria in endemic Burkitt's lymphoma,
and appreciating the importance of early diagnosis and intensive
chemotherapy. With improved healthcare infrastructure, supportive care,
and public health measures, outcomes for children with lymphoma can
continue to improve.

\chapter{Retinoblastoma}\label{retinoblastoma}

\section{Introduction}\label{introduction-25}

Retinoblastoma is the most common \textbf{primary intraocular malignancy
of childhood}, arising from the retina. It typically presents before the
age of 5 years and carries major implications for vision, survival, and
quality of life. The disease has become a paradigm for cancer genetics,
as the RB1 tumour suppressor gene was the first tumour gene identified
in humans.

Early diagnosis and treatment are critical: retinoblastoma is highly
curable if detected early, but advanced disease can be fatal. In
high-income countries, survival exceeds 95\%, while in many low- and
middle-income countries---including Ghana---delayed diagnosis often
results in poorer outcomes.

\section{Incidence and Prevalence}\label{incidence-and-prevalence-4}

\begin{itemize}
\tightlist
\item
  Global incidence: about \textbf{1 in 15,000--20,000 live births}.
\item
  Accounts for \textbf{3--4\% of all childhood cancers}.
\item
  Peak age:

  \begin{itemize}
  \tightlist
  \item
    Unilateral disease: 2--3 years.
  \item
    Bilateral disease: diagnosed earlier, often before 1 year.
  \end{itemize}
\item
  No sex predilection.
\item
  In Ghana and other sub-Saharan countries, children frequently present
  late, often with extraocular spread, which worsens survival rates.
\end{itemize}

\section{Aetiology}\label{aetiology-13}

The development of retinoblastoma is intimately linked to the
\textbf{RB1 gene} on chromosome 13q14.

\begin{itemize}
\tightlist
\item
  \textbf{Heritable form} (40\% of cases):

  \begin{itemize}
  \tightlist
  \item
    Germline mutation in one allele of RB1 gene is inherited.
  \item
    Second hit occurs somatically in retinal cells.
  \item
    Usually bilateral and multifocal.
  \item
    Associated with increased risk of secondary malignancies (e.g.,
    osteosarcoma).
  \end{itemize}
\item
  \textbf{Non-heritable form} (60\% of cases):

  \begin{itemize}
  \tightlist
  \item
    Both RB1 mutations occur somatically.
  \item
    Usually unilateral and unifocal.
  \end{itemize}
\end{itemize}

Rarely, retinoblastoma may arise from \textbf{MYCN amplification} even
without RB1 mutation.

\section{Pathophysiology}\label{pathophysiology-16}

The RB1 gene product regulates the \textbf{G1--S checkpoint} of the cell
cycle. Loss of both functional RB1 alleles leads to uncontrolled retinal
cell proliferation.

\begin{itemize}
\tightlist
\item
  Tumours originate from \textbf{retinal progenitor cells}.
\item
  Can grow in various patterns:

  \begin{itemize}
  \tightlist
  \item
    \textbf{Endophytic}: growing into the vitreous.
  \item
    \textbf{Exophytic}: growing beneath the retina, leading to retinal
    detachment.
  \item
    \textbf{Diffuse infiltrating}: rare, spreading through the retina
    without forming a discrete mass.
  \end{itemize}
\item
  Spread:

  \begin{itemize}
  \tightlist
  \item
    Local invasion (into optic nerve, choroid, sclera).
  \item
    Extraocular spread (orbit, brain via optic nerve, systemic
    metastasis to bone marrow, liver).
  \end{itemize}
\end{itemize}

\section{Clinical Features}\label{clinical-features-14}

Presentation varies depending on stage and extent of disease.

\begin{itemize}
\tightlist
\item
  \textbf{Most common presenting sign}:

  \begin{itemize}
  \tightlist
  \item
    \textbf{Leukocoria} (white pupillary reflex), often noticed in
    photographs with flash.
  \end{itemize}
\item
  Other features:

  \begin{itemize}
  \tightlist
  \item
    Strabismus (misalignment of eyes).
  \item
    Red, painful eye (from secondary glaucoma, uveitis, or tumour
    necrosis).
  \item
    Poor vision or blindness.
  \item
    Hyphema (blood in anterior chamber).
  \item
    Orbital swelling or proptosis (extraocular disease).
  \item
    Rare systemic symptoms in metastatic disease (bone pain, weight
    loss, fever).
  \end{itemize}
\end{itemize}

\section{Differential Diagnosis}\label{differential-diagnosis-16}

\begin{itemize}
\tightlist
\item
  Coats' disease (retinal telangiectasia with exudation).
\item
  Persistent hyperplastic primary vitreous.
\item
  Congenital cataract.
\item
  Toxocariasis.
\item
  Retinal detachment.
\item
  Medulloepithelioma of the ciliary body.
\end{itemize}

\section{Investigations}\label{investigations-18}

\begin{itemize}
\tightlist
\item
  \textbf{Ocular examination}:

  \begin{itemize}
  \tightlist
  \item
    Indirect ophthalmoscopy under anaesthesia (definitive for
    diagnosis).
  \end{itemize}
\item
  \textbf{Imaging}:

  \begin{itemize}
  \tightlist
  \item
    \textbf{Ultrasound B-scan}: reveals intraocular mass with
    calcification.
  \item
    \textbf{CT scan}: useful for calcifications but limited due to
    radiation risk.
  \item
    \textbf{MRI of orbits and brain}: preferred for local extension
    (optic nerve, CNS).
  \end{itemize}
\item
  \textbf{Laboratory tests}: not diagnostic, but baseline bloods useful
  before chemotherapy.
\item
  \textbf{Genetic testing}:

  \begin{itemize}
  \tightlist
  \item
    Detects RB1 mutation.
  \item
    Guides family counselling and screening of siblings.
  \end{itemize}
\end{itemize}

Biopsy of the eye is avoided due to risk of tumour spread.

\section{Staging}\label{staging}

Two main systems:

\begin{enumerate}
\def\labelenumi{\arabic{enumi}.}
\tightlist
\item
  \textbf{International Intraocular Retinoblastoma Classification
  (IIRC)} -- based on disease extent within the eye (Groups A--E).
\item
  \textbf{International Retinoblastoma Staging System (IRSS)} -- for
  post-enucleation staging, including extraocular spread and metastasis.
\end{enumerate}

\section{Treatment}\label{treatment-15}

Treatment depends on whether the disease is unilateral or bilateral,
intraocular or extraocular, and the aim (life preservation, eye salvage,
vision preservation).

\subsection{Emergency Care}\label{emergency-care}

\begin{itemize}
\tightlist
\item
  Treat secondary glaucoma for pain relief.
\item
  Manage raised intracranial pressure in cases of optic nerve invasion.
\end{itemize}

\subsection{Definitive and Ongoing
Management}\label{definitive-and-ongoing-management}

\begin{itemize}
\tightlist
\item
  \textbf{Enucleation}: removal of the affected eye. Standard for
  unilateral advanced disease.
\item
  \textbf{Focal therapies} (for small tumours):

  \begin{itemize}
  \tightlist
  \item
    Laser photocoagulation.
  \item
    Cryotherapy.
  \item
    Thermotherapy.
  \end{itemize}
\item
  \textbf{Chemotherapy}:

  \begin{itemize}
  \tightlist
  \item
    Systemic chemotherapy (vincristine, carboplatin, etoposide) for
    chemoreduction.
  \item
    Intra-arterial chemotherapy (direct to ophthalmic artery).
  \item
    Intravitreal chemotherapy (for vitreous seeds).
  \end{itemize}
\item
  \textbf{Radiotherapy}:

  \begin{itemize}
  \tightlist
  \item
    External beam (rare now, due to risk of secondary tumours in
    heritable cases).
  \item
    Plaque brachytherapy for selected cases.
  \end{itemize}
\item
  \textbf{Bilateral disease}: efforts made to preserve at least one eye
  with useful vision.
\end{itemize}

\subsection{Preparation for
Discharge}\label{preparation-for-discharge-2}

\begin{itemize}
\tightlist
\item
  Educate parents on prosthesis care after enucleation.
\item
  Importance of follow-up for recurrence detection.
\item
  Genetic counselling for families with heritable disease.
\end{itemize}

\subsection{Long-Term Management}\label{long-term-management-2}

\begin{itemize}
\tightlist
\item
  Regular ophthalmologic examinations under anaesthesia.
\item
  Screening for second malignancies in heritable cases.
\item
  Monitoring growth, vision development, and psychosocial adjustment.
\end{itemize}

\section{Complications}\label{complications-18}

\begin{itemize}
\tightlist
\item
  Local recurrence within the eye or orbit.
\item
  Extraocular spread with poor prognosis.
\item
  Metastases to CNS, bone marrow, or distant organs.
\item
  Vision loss, especially in bilateral disease.
\item
  Cosmetic issues after enucleation.
\item
  Secondary malignancies in heritable retinoblastoma, particularly
  osteosarcoma and soft tissue sarcomas (especially after radiotherapy).
\end{itemize}

\section{Prognosis}\label{prognosis-18}

\begin{itemize}
\tightlist
\item
  In high-income countries: \textgreater95\% survival.
\item
  In sub-Saharan Africa: survival often \textless40\%, mainly due to
  late presentation, extraocular disease, and limited treatment
  resources.
\item
  Prognosis is best with early detection, small intraocular tumours, and
  access to multimodal therapy.
\end{itemize}

\section{Prevention}\label{prevention-9}

\begin{itemize}
\tightlist
\item
  No known prevention for sporadic cases.
\item
  \textbf{Genetic counselling and testing}: vital in families with
  heritable retinoblastoma.
\item
  Screening of at-risk infants (regular eye exams from birth to 5
  years).
\item
  Avoid unnecessary exposure to ionising radiation in heritable cases.
\end{itemize}

\section{Conclusion}\label{conclusion-20}

Retinoblastoma is a highly curable childhood malignancy when detected
early. The disease highlights the importance of integrating
\textbf{clinical suspicion, imaging, genetic counselling, and multimodal
therapy} in management. In Ghana and similar settings, community
education to recognise leukocoria early, improved access to ophthalmic
oncology, and support for families can dramatically improve survival and
quality of life.

\chapter{Nephroblastoma (Wilms'
Tumour)}\label{nephroblastoma-wilms-tumour}

\section{Introduction}\label{introduction-26}

Nephroblastoma, commonly known as \textbf{Wilms' tumour}, is the most
common malignant renal tumour in childhood. It arises from embryonic
renal tissue and typically presents between ages 2 and 5 years. The
tumour has contributed greatly to the success story of paediatric
oncology, with survival rates improving significantly due to advances in
surgery, chemotherapy, and radiotherapy.\\
Although relatively rare compared to infections or malnutrition,
nephroblastoma remains an important cause of morbidity and mortality in
paediatrics, particularly in low- and middle-income countries where late
presentation is common.

\section{Incidence and Prevalence}\label{incidence-and-prevalence-5}

Wilms' tumour accounts for about \textbf{6--8\% of all childhood
cancers}.\\
- The annual incidence is approximately \textbf{8 cases per million
children} under 15 years.\\
- Peak age of presentation: \textbf{3--4 years}.\\
- Slight female predominance.\\
- Bilateral disease occurs in about \textbf{5--7\%} of cases.

In sub-Saharan Africa, including Ghana, the incidence is comparable to
global figures, but outcomes are poorer due to late presentation,
limited access to multimodal therapy, and higher rates of advanced
disease at diagnosis.

\section{Aetiology}\label{aetiology-14}

Most cases are sporadic, but both \textbf{genetic} and
\textbf{environmental} factors play roles.

\begin{itemize}
\tightlist
\item
  \textbf{Genetic factors}:

  \begin{itemize}
  \tightlist
  \item
    Mutations in WT1 (chromosome 11p13) and WT2 (11p15) are implicated.
  \item
    Other genes: WTX (X chromosome), CTNNB1 (beta-catenin pathway).
  \end{itemize}
\item
  \textbf{Syndromic associations}:

  \begin{itemize}
  \tightlist
  \item
    WAGR syndrome (Wilms tumour, Aniridia, Genitourinary anomalies,
    mental Retardation).
  \item
    Denys--Drash syndrome (gonadal dysgenesis, nephropathy, Wilms
    tumour).
  \item
    Beckwith--Wiedemann syndrome (organomegaly, hemihypertrophy,
    increased tumour risk).
  \end{itemize}
\item
  \textbf{Familial predisposition}: Rare but recognised, with siblings
  sometimes affected.
\item
  \textbf{Environmental factors}: No strong evidence, though
  intrauterine exposures have been explored.
\end{itemize}

\section{Pathophysiology}\label{pathophysiology-17}

Wilms' tumour develops from \textbf{persistent metanephric blastema},
the embryonic renal precursor tissue that fails to differentiate
normally.\\
- The tumour is typically a \textbf{triphasic neoplasm}, consisting
of:\\
- \textbf{Blastemal cells} (small round blue cells).\\
- \textbf{Stromal elements} (spindle cells, connective tissue).\\
- \textbf{Epithelial components} (tubules, glomeruloid structures).\\
- Some tumours may be monophasic, dominated by one component.\\
- Tumour growth can distort the kidney, invade renal vessels, extend
into the inferior vena cava, and metastasize, commonly to lungs, liver,
and lymph nodes.

A subset of tumours shows \textbf{anaplasia}, which carries a poorer
prognosis and resistance to therapy.

\section{Clinical Features}\label{clinical-features-15}

Presentation depends on tumour size, stage, and presence of metastases.

\begin{itemize}
\tightlist
\item
  \textbf{Most common feature}:

  \begin{itemize}
  \tightlist
  \item
    Painless \textbf{abdominal mass}, often noticed by parents during
    bathing or dressing.
  \end{itemize}
\item
  Other features:

  \begin{itemize}
  \tightlist
  \item
    Abdominal pain or discomfort.
  \item
    Hematuria (gross or microscopic).
  \item
    Hypertension (due to increased renin secretion).
  \item
    Anemia (from haemorrhage or bone marrow suppression).
  \item
    Weight loss, anorexia, malaise (less common).\\
  \end{itemize}
\item
  Advanced disease:

  \begin{itemize}
  \tightlist
  \item
    Cough, dyspnea (lung metastases).
  \item
    Hepatomegaly (liver metastases).
  \end{itemize}
\end{itemize}

Unlike neuroblastoma, Wilms' tumour rarely crosses the midline in the
abdomen.

\section{Differential Diagnosis}\label{differential-diagnosis-17}

Important conditions to consider when a child presents with an abdominal
mass:\\
- \textbf{Neuroblastoma} (usually crosses midline, calcification
common).\\
- Multicystic dysplastic kidney.\\
- Hydronephrosis.\\
- Mesoblastic nephroma (in neonates).\\
- Renal cell carcinoma (rare in children).\\
- Hepatoblastoma or hepatomegaly from other causes.

\section{Investigations}\label{investigations-19}

Workup aims at confirming diagnosis, assessing extent, and staging.

\begin{itemize}
\tightlist
\item
  \textbf{Laboratory tests}:

  \begin{itemize}
  \tightlist
  \item
    CBC (anemia, baseline counts).
  \item
    Renal function tests (creatinine, electrolytes).
  \item
    Liver function tests.
  \item
    Urinalysis (hematuria).\\
  \end{itemize}
\item
  \textbf{Imaging}:

  \begin{itemize}
  \tightlist
  \item
    \textbf{Abdominal ultrasound}: First-line; identifies renal origin
    of mass.
  \item
    \textbf{CT or MRI of abdomen}: Defines tumour extent, contralateral
    kidney involvement, vascular invasion.
  \item
    \textbf{Chest X-ray/CT}: Evaluate for lung metastases.
  \end{itemize}
\item
  \textbf{Histology}: Usually obtained after nephrectomy or biopsy in
  bilateral/advanced disease.
\end{itemize}

\section{Staging}\label{staging-1}

The \textbf{National Wilms' Tumor Study (NWTS)} staging system is widely
used:\\
- Stage I: Limited to kidney, completely resected.\\
- Stage II: Extends beyond kidney but completely resected.\\
- Stage III: Residual tumour confined to abdomen (lymph nodes,
peritoneal spillage).\\
- Stage IV: Hematogenous metastases (lung, liver, bone, brain).\\
- Stage V: Bilateral renal involvement.

\section{Treatment}\label{treatment-16}

Successful management requires a \textbf{multimodal approach}: surgery,
chemotherapy, and sometimes radiotherapy.

\subsection{Emergency Management}\label{emergency-management-1}

\begin{itemize}
\tightlist
\item
  Stabilise child if anaemic, hypertensive, or in respiratory distress.
\item
  Treat severe hypertension with antihypertensives.
\item
  Blood transfusion for anaemia.
\item
  Manage tumour rupture (can present with acute abdomen).
\end{itemize}

\subsection{Definitive and Ongoing
Treatment}\label{definitive-and-ongoing-treatment}

\begin{itemize}
\tightlist
\item
  \textbf{Surgery}: Radical nephrectomy is standard for unilateral
  disease.
\item
  \textbf{Chemotherapy}: Regimens typically include vincristine,
  actinomycin D, and doxorubicin (depending on stage and histology).
\item
  \textbf{Radiotherapy}: Reserved for higher-stage disease or anaplastic
  histology.
\item
  \textbf{Bilateral disease (Stage V)}: Initial chemotherapy to shrink
  tumour, followed by nephron-sparing surgery.
\end{itemize}

\subsection{Preparation for
Discharge}\label{preparation-for-discharge-3}

\begin{itemize}
\tightlist
\item
  Educate caregivers on:

  \begin{itemize}
  \tightlist
  \item
    Medication adherence.
  \item
    Infection prevention during chemotherapy.
  \item
    Monitoring for hypertension and renal function.
  \item
    Nutrition and follow-up visits.
  \end{itemize}
\end{itemize}

\subsection{Long-Term Management}\label{long-term-management-3}

\begin{itemize}
\tightlist
\item
  Regular follow-up for recurrence surveillance.
\item
  Monitor growth and development.
\item
  Monitor renal function (risk of chronic kidney disease, especially in
  bilateral disease).
\item
  Monitor for late effects of chemotherapy/radiotherapy (cardiotoxicity,
  infertility, secondary malignancies).
\end{itemize}

\section{Complications}\label{complications-19}

\begin{itemize}
\tightlist
\item
  Tumour rupture leading to haemorrhage and peritonitis.
\item
  Hypertension due to renin production.
\item
  Metastasis (lungs, liver).
\item
  Chemotherapy-related: myelosuppression, mucositis, cardiotoxicity.
\item
  Chronic renal impairment in bilateral disease.
\item
  Psychological and social impact on family.
\end{itemize}

\section{Prognosis}\label{prognosis-19}

Wilms' tumour is one of the \textbf{paediatric oncology success
stories}:\\
- Overall survival exceeds \textbf{85\%} in high-income countries.\\
- Prognosis depends on:\\
- Stage at diagnosis.\\
- Histology (anaplastic variants worse).\\
- Age of child.\\
- In sub-Saharan Africa, survival is significantly lower (20--50\%) due
to late presentation, limited resources, and treatment abandonment.

\section{Prevention}\label{prevention-10}

There are no established preventive measures for sporadic cases.
However:\\
- \textbf{Genetic counselling} for families with syndromic or familial
cases.\\
- \textbf{Surveillance imaging} (ultrasound every 3 months until age 7)
for high-risk children (e.g., WAGR, Denys--Drash, Beckwith--Wiedemann,
bilateral disease).\\
- Early detection and treatment significantly improve outcomes.

\section{Conclusion}\label{conclusion-21}

Nephroblastoma is the most common childhood renal malignancy and a
leading cause of paediatric abdominal masses. It exemplifies how
combined surgery, chemotherapy, and radiotherapy can yield excellent
outcomes when implemented effectively. For Ghana and similar settings,
the major challenge remains late presentation and limited access to
oncology services. Strengthening health systems, raising community
awareness, and improving access to multimodal therapy are essential to
bridge the survival gap.

\chapter{Other Pediatric Tumours}\label{other-pediatric-tumours}

\section{Introduction}\label{introduction-27}

Childhood cancers represent a diverse group of diseases distinct from
adult malignancies in their biology, clinical behaviour, and response to
treatment. While leukaemias, lymphomas, nephroblastoma, and
retinoblastoma are among the most common, several other solid tumours
also contribute significantly to paediatric cancer morbidity and
mortality worldwide.

Understanding these conditions is crucial for early recognition,
appropriate referral, and timely management. This chapter will cover
neuroblastoma, hepatic tumours, sarcomas, bone tumours, brain tumours,
and germ cell tumours, as well as selected rare entities.

\section{Neuroblastoma}\label{neuroblastoma}

\subsection{Introduction and
Epidemiology}\label{introduction-and-epidemiology}

Neuroblastoma is the most common \textbf{extracranial solid tumour of
childhood}, arising from neural crest cells of the sympathetic nervous
system.\\
- Accounts for about \textbf{8--10\% of childhood cancers}.\\
- Median age at diagnosis: \textbf{2 years}.\\
- Rare after 10 years.\\
- Common sites: adrenal medulla (40\%), paraspinal sympathetic chain,
posterior mediastinum.

\subsection{Pathophysiology}\label{pathophysiology-18}

\begin{itemize}
\tightlist
\item
  Originates from neural crest cells that fail to differentiate.\\
\item
  Tumour behaviour is highly variable: can spontaneously regress
  (especially in infants) or progress aggressively with widespread
  metastases.\\
\item
  Genetic features: amplification of \textbf{MYCN oncogene} is
  associated with poor prognosis.
\end{itemize}

\subsection{Clinical Features}\label{clinical-features-16}

\begin{itemize}
\tightlist
\item
  Abdominal mass (often firm, irregular, crossing midline).\\
\item
  Symptoms due to local invasion: constipation, urinary obstruction.\\
\item
  Metastases: bone pain, periorbital ecchymoses (``raccoon eyes''),
  hepatomegaly.\\
\item
  Paraneoplastic features: hypertension, diarrhoea (due to vasoactive
  intestinal peptide).
\end{itemize}

\subsection{Investigations}\label{investigations-20}

\begin{itemize}
\tightlist
\item
  Imaging: ultrasound, CT/MRI of abdomen.\\
\item
  MIBG scan: identifies tumour sites.\\
\item
  Biopsy for histology.\\
\item
  Elevated urinary catecholamines (VMA, HVA) in \textgreater90\%.
\end{itemize}

\subsection{Management}\label{management-13}

\begin{itemize}
\tightlist
\item
  Depends on risk stratification.\\
\item
  Low-risk: Surgery alone may cure.\\
\item
  Intermediate-risk: surgery + chemotherapy.\\
\item
  High-risk: intensive chemotherapy, surgery, radiotherapy, stem cell
  transplant, immunotherapy.
\end{itemize}

\subsection{Prognosis}\label{prognosis-20}

\begin{itemize}
\tightlist
\item
  Variable. Infants with localised disease may do very well.\\
\item
  High-risk disease has poorer survival despite aggressive therapy.
\end{itemize}

\section{Hepatic Tumours}\label{hepatic-tumours}

\subsection{Hepatoblastoma}\label{hepatoblastoma}

\begin{itemize}
\tightlist
\item
  Most common \textbf{primary liver tumour in children}.\\
\item
  Usually diagnosed in children under \textbf{3 years}.\\
\item
  Associated with prematurity and some genetic syndromes (e.g.,
  Beckwith--Wiedemann).
\end{itemize}

\textbf{Clinical Features}:\\
- Painless abdominal mass.\\
- Abdominal distension.\\
- Elevated \textbf{alpha-fetoprotein (AFP)} in most cases.

\textbf{Diagnosis}:\\
- Ultrasound/CT scan showing liver mass.\\
- Biopsy for confirmation.

\textbf{Treatment}:\\
- Surgical resection (hepatectomy).\\
- Neoadjuvant/adjuvant chemotherapy (cisplatin-based).\\
- Liver transplant if unresectable.

\textbf{Prognosis}:\\
- Good if complete surgical removal is possible.

\subsection{Hepatocellular Carcinoma
(HCC)}\label{hepatocellular-carcinoma-hcc}

\begin{itemize}
\tightlist
\item
  Less common in children but important in sub-Saharan Africa due to
  \textbf{hepatitis B infection}.\\
\item
  Presents in older children and adolescents.\\
\item
  AFP is elevated but less consistently than hepatoblastoma.\\
\item
  Prognosis is generally poor, as tumours are often unresectable.
\end{itemize}

\section{Rhabdomyosarcoma and Other Soft Tissue
Sarcomas}\label{rhabdomyosarcoma-and-other-soft-tissue-sarcomas}

\subsection{Rhabdomyosarcoma (RMS)}\label{rhabdomyosarcoma-rms}

\begin{itemize}
\tightlist
\item
  Most common \textbf{soft tissue sarcoma in children}.\\
\item
  Arises from primitive mesenchymal cells committed to skeletal muscle
  lineage.\\
\item
  Common sites: head and neck (orbit, nasopharynx), genitourinary tract,
  extremities.
\end{itemize}

\textbf{Clinical Features}:\\
- Mass on the affected site.\\
- Proptosis (orbital).\\
- Nasal obstruction, epistaxis (nasopharyngeal).\\
- Haematuria or vaginal bleeding (genitourinary).

\textbf{Diagnosis}:\\
- Imaging (MRI).\\
- Biopsy with histology (embryonal, alveolar, pleomorphic subtypes).\\
- Immunohistochemistry (desmin, myogenin positive).

\textbf{Treatment}:\\
- Multimodal: surgery, chemotherapy, radiotherapy.

\textbf{Prognosis}:\\
- Better in embryonal type.\\
- Depends on site, size, and extent.

\subsection{Other Soft Tissue
Sarcomas}\label{other-soft-tissue-sarcomas}

\begin{itemize}
\tightlist
\item
  Include fibrosarcoma, synovial sarcoma, and malignant peripheral nerve
  sheath tumour.\\
\item
  Less common but managed similarly with surgery, chemotherapy, and
  radiotherapy.
\end{itemize}

\section{Bone Tumours}\label{bone-tumours}

\subsection{Osteosarcoma}\label{osteosarcoma}

\begin{itemize}
\tightlist
\item
  Most common \textbf{primary malignant bone tumour} in children and
  adolescents.\\
\item
  Peak in adolescence during rapid bone growth.\\
\item
  Common sites: metaphyses of long bones (femur, tibia, humerus).
\end{itemize}

\textbf{Clinical Features}:\\
- Localised bone pain (worse at night).\\
- Swelling, mass, limitation of movement.\\
- Pathological fractures may occur.

\textbf{Investigations}:\\
- X-ray: mixed lytic-sclerotic lesion, periosteal reaction (``sunburst''
appearance, Codman triangle).\\
- MRI: local extent.\\
- Biopsy confirms diagnosis.

\textbf{Treatment}:\\
- Neoadjuvant chemotherapy, limb-sparing surgery (or amputation),
adjuvant chemotherapy.

\textbf{Prognosis}:\\
- Improved with multimodal therapy.\\
- Presence of metastases (lung) worsens outlook.

\subsection{Ewing Sarcoma}\label{ewing-sarcoma}

\begin{itemize}
\tightlist
\item
  Second most common bone tumour in children.\\
\item
  Arises from primitive neuroectodermal cells.\\
\item
  Typically affects diaphysis of long bones and pelvis.\\
\item
  Associated with \textbf{t(11;22) translocation}.
\end{itemize}

\textbf{Clinical Features}:\\
- Pain, swelling, systemic symptoms (fever, weight loss).\\
- Can mimic infection (osteomyelitis).

\textbf{Investigations}:\\
- X-ray: ``onion-skin'' periosteal reaction.\\
- MRI: extent of disease.\\
- Biopsy for histology and cytogenetics.

\textbf{Treatment}:\\
- Chemotherapy, surgery, and/or radiotherapy.

\textbf{Prognosis}:\\
- Fair with localised disease.\\
- Poor with metastases.

\section{Brain Tumours}\label{brain-tumours}

Brain tumours are the most common \textbf{solid tumours of childhood}.

\subsection{Medulloblastoma}\label{medulloblastoma}

\begin{itemize}
\tightlist
\item
  Most common malignant brain tumour in children.\\
\item
  Originates in cerebellum.\\
\item
  Highly radiosensitive.
\end{itemize}

\textbf{Features}: headache, vomiting, ataxia, papilloedema.\\
\textbf{Treatment}: surgery + craniospinal irradiation + chemotherapy.

\subsection{Astrocytomas}\label{astrocytomas}

\begin{itemize}
\tightlist
\item
  Low-grade astrocytomas (e.g., pilocytic astrocytoma) have excellent
  prognosis after surgical removal.\\
\item
  High-grade astrocytomas are aggressive and carry poorer outcomes.
\end{itemize}

\subsection{Ependymomas}\label{ependymomas}

\begin{itemize}
\tightlist
\item
  Arise from ependymal cells lining ventricles.\\
\item
  Commonly present with hydrocephalus due to obstruction.\\
\item
  Treatment: surgery and radiotherapy.
\end{itemize}

\section{Germ Cell Tumours}\label{germ-cell-tumours}

\subsection{Overview}\label{overview-1}

\begin{itemize}
\tightlist
\item
  Can occur in gonads (testis, ovary) or extragonadal sites
  (sacrococcygeal, mediastinum, retroperitoneum).\\
\item
  Derived from primordial germ cells.
\end{itemize}

\subsection{Clinical Features}\label{clinical-features-17}

\begin{itemize}
\tightlist
\item
  Testicular: painless testicular mass.\\
\item
  Ovarian: abdominal mass, pain, precocious puberty.\\
\item
  Sacrococcygeal: mass at base of spine, sometimes visible externally.
\end{itemize}

\subsection{Diagnosis}\label{diagnosis-9}

\begin{itemize}
\tightlist
\item
  Imaging (ultrasound, CT/MRI).\\
\item
  Serum tumour markers: AFP, β-HCG.\\
\item
  Biopsy (except in some gonadal cases where orchiectomy/oophorectomy is
  primary treatment).
\end{itemize}

\subsection{Treatment}\label{treatment-17}

\begin{itemize}
\tightlist
\item
  Surgery + chemotherapy (cisplatin-based).\\
\item
  Prognosis generally good, especially for localised disease.
\end{itemize}

\section{Other Rare Paediatric
Tumours}\label{other-rare-paediatric-tumours}

\begin{itemize}
\tightlist
\item
  \textbf{Adrenocortical tumours}: may present with virilisation or
  Cushing's syndrome.\\
\item
  \textbf{Thyroid carcinoma}: uncommon, but papillary carcinoma can
  occur, sometimes associated with prior radiation exposure.\\
\item
  \textbf{Malignant rhabdoid tumour of the kidney or brain}: rare and
  aggressive.
\end{itemize}

\section{Conclusion}\label{conclusion-22}

Beyond leukaemia, lymphoma, nephroblastoma, and retinoblastoma, several
other paediatric malignancies play a significant role in the spectrum of
childhood cancer. These include neuroblastoma, hepatic tumours,
sarcomas, bone tumours, brain tumours, and germ cell tumours. Although
they differ in biology and clinical behaviour, successful management
relies on \textbf{early diagnosis, multidisciplinary care, and
supportive management}.

In resource-limited settings such as Ghana, improving survival will
require increased awareness of early signs, timely referral, access to
diagnostic facilities, and strengthening of paediatric oncology units.

\part{{Nephrology}}

\chapter{Spectrum of Kidney Diseases in
Children}\label{spectrum-of-kidney-diseases-in-children}

\section{Introduction}\label{introduction-28}

The kidneys play a vital role in maintaining internal homeostasis
through the regulation of water, electrolytes, acid-base balance, and
the excretion of metabolic waste. In children, renal function is
essential not only for maintaining physiologic stability but also for
supporting growth and development. Kidney diseases in children encompass
a wide spectrum ranging from congenital and inherited disorders to
acquired glomerular, tubular, and systemic conditions.

In Ghana and other low- and middle-income countries, kidney diseases in
children are increasingly recognized as important causes of morbidity
and mortality. Limited diagnostic facilities, late presentation, and
inadequate access to nephrology services remain major challenges.
Understanding the diverse presentation and underlying mechanisms of
renal disease is, therefore, critical for early diagnosis and effective
management.

\section{Epidemiology and Burden}\label{epidemiology-and-burden-1}

Globally, the prevalence of paediatric kidney disease varies widely
depending on the specific condition. Acute kidney injury (AKI) is
estimated to occur in up to 25\% of hospitalized children, while chronic
kidney disease (CKD) affects 1--3 per 1,000 children. In sub-Saharan
Africa, data are scarce, but renal disease often presents late and is
associated with preventable causes such as infections, dehydration, and
toxins.

In Ghana, children frequently present with conditions such as nephrotic
syndrome, acute glomerulonephritis, urinary tract infections, and
congenital anomalies of the kidney and urinary tract (CAKUT). Early
childhood illnesses, poor sanitation, and limited access to pediatric
nephrology care contribute to adverse outcomes.

\section{Classification of Kidney Diseases in
Children}\label{classification-of-kidney-diseases-in-children}

Kidney diseases in children can be broadly classified into the following
categories:

\begin{enumerate}
\def\labelenumi{\arabic{enumi}.}
\tightlist
\item
  \textbf{Congenital and Structural Anomalies}

  \begin{itemize}
  \tightlist
  \item
    Congenital anomalies of the kidney and urinary tract (CAKUT)
  \item
    Obstructive uropathy (posterior urethral valves, ureteropelvic
    junction obstruction)
  \item
    Renal dysplasia or hypoplasia
  \item
    Polycystic kidney disease
  \end{itemize}
\item
  \textbf{Glomerular Diseases}

  \begin{itemize}
  \tightlist
  \item
    Nephrotic syndrome (minimal change, focal segmental
    glomerulosclerosis, membranoproliferative)
  \item
    Glomerulonephritis (post-streptococcal, IgA nephropathy, lupus
    nephritis)
  \item
    Rapidly progressive glomerulonephritis
  \end{itemize}
\item
  \textbf{Tubulointerstitial and Tubular Disorders}

  \begin{itemize}
  \tightlist
  \item
    Acute interstitial nephritis
  \item
    Renal tubular acidosis
  \item
    Fanconi syndrome
  \item
    Cystinosis and other metabolic tubular disorders
  \end{itemize}
\item
  \textbf{Infective and Postinfectious Conditions}

  \begin{itemize}
  \tightlist
  \item
    Urinary tract infection (UTI)
  \item
    Pyelonephritis
  \item
    Reflux nephropathy
  \item
    Schistosomiasis-related kidney disease
  \end{itemize}
\item
  \textbf{Systemic and Secondary Causes}

  \begin{itemize}
  \tightlist
  \item
    Hypertension
  \item
    Diabetes mellitus (rare in children but increasing)
  \item
    Sickle cell nephropathy
  \item
    HIV-associated nephropathy
  \item
    Hemolytic uremic syndrome (HUS)
  \end{itemize}
\item
  \textbf{Acute and Chronic Renal Failure}

  \begin{itemize}
  \tightlist
  \item
    Acute kidney injury (AKI)
  \item
    Chronic kidney disease (CKD) and end-stage renal disease (ESRD)
  \end{itemize}
\end{enumerate}

\section{Pathophysiological Overview}\label{pathophysiological-overview}

The kidney's response to injury varies depending on the site and nature
of the insult.

\subsection{\texorpdfstring{\textbf{Glomerular
Diseases}}{Glomerular Diseases}}\label{glomerular-diseases}

These involve inflammation or damage to the glomeruli, leading to
abnormal filtration.\\
- \textbf{Nephrotic syndrome} results from increased glomerular
permeability to proteins, producing heavy proteinuria, hypoalbuminaemia,
and oedema.\\
- \textbf{Glomerulonephritis}, on the other hand, causes hematuria,
hypertension, and varying degrees of renal impairment. Immune-mediated
mechanisms --- such as deposition of immune complexes following
infections --- play a central role.

\subsection{\texorpdfstring{\textbf{Tubulointerstitial and Tubular
Disorders}}{Tubulointerstitial and Tubular Disorders}}\label{tubulointerstitial-and-tubular-disorders}

Tubular diseases interfere with urine concentration, electrolyte
handling, and acid-base balance.\\
- \textbf{Renal tubular acidosis} leads to metabolic acidosis due to
defective hydrogen ion excretion or bicarbonate reabsorption.\\
- \textbf{Fanconi syndrome} affects multiple tubular transport
mechanisms, leading to glycosuria, aminoaciduria, and phosphate wasting.

\subsection{\texorpdfstring{\textbf{Vascular and Systemic
Disorders}}{Vascular and Systemic Disorders}}\label{vascular-and-systemic-disorders}

Conditions such as hemolytic uremic syndrome cause endothelial injury
leading to microangiopathic hemolysis and acute renal failure.
Hypertension, both a cause and consequence of renal disease, damages
glomeruli and accelerates progression to chronic kidney disease.

\subsection{\texorpdfstring{\textbf{Congenital and Structural
Anomalies}}{Congenital and Structural Anomalies}}\label{congenital-and-structural-anomalies}

CAKUT accounts for a significant proportion of pediatric renal failure.
These abnormalities arise during embryogenesis and include renal
agenesis, dysplasia, and obstructive lesions. Impaired nephron
development or chronic obstruction eventually results in renal scarring
and progressive dysfunction.

\section{Clinical Presentation}\label{clinical-presentation-3}

Renal diseases in children present with diverse features depending on
the site and extent of involvement.

Common presentations include: - \textbf{Oedema}, especially periorbital
and pedal, typical of nephrotic syndrome.\\
- \textbf{Haematuria} (macroscopic or microscopic), often seen in
glomerulonephritis.\\
- \textbf{Hypertension}, either as a primary finding or secondary to
renal pathology.\\
- \textbf{Oliguria or anuria}, indicating renal failure.\\
- \textbf{Polyuria and polydipsia}, suggestive of tubular dysfunction.\\
- \textbf{Recurrent urinary tract infections}, possibly pointing to
vesicoureteral reflux or obstruction.\\
- \textbf{Growth retardation and failure to thrive}, common in chronic
kidney disease.

Infants may present with nonspecific signs such as poor feeding,
vomiting, or failure to gain weight, necessitating a high index of
suspicion.

\section{Investigations}\label{investigations-21}

Diagnosis requires a combination of clinical assessment, laboratory
tests, and imaging.

\subsection{\texorpdfstring{\textbf{Laboratory
Tests}}{Laboratory Tests}}\label{laboratory-tests}

\begin{itemize}
\tightlist
\item
  \textbf{Urinalysis:} detects proteinuria, hematuria, or pyuria.\\
\item
  \textbf{Urine microscopy and culture:} identifies infection or
  casts.\\
\item
  \textbf{Serum urea and creatinine:} assess renal function.\\
\item
  \textbf{Electrolytes and bicarbonate:} for acid-base and electrolyte
  imbalances.\\
\item
  \textbf{Complement levels (C3, C4):} decreased in post-streptococcal
  glomerulonephritis.\\
\item
  \textbf{Autoantibody testing:} ANA, anti-dsDNA for lupus nephritis.\\
\item
  \textbf{24-hour urine protein or spot protein-to-creatinine ratio} to
  quantify proteinuria.
\end{itemize}

\subsection{\texorpdfstring{\textbf{Imaging}}{Imaging}}\label{imaging}

\begin{itemize}
\tightlist
\item
  \textbf{Renal ultrasound} for kidney size, structure, and
  obstruction.\\
\item
  \textbf{Voiding cystourethrogram (VCUG)} for reflux diagnosis.\\
\item
  \textbf{DMSA scan} for renal scarring.\\
\item
  \textbf{CT/MRI} in complex anomalies or masses.
\end{itemize}

\subsection{\texorpdfstring{\textbf{Renal
Biopsy}}{Renal Biopsy}}\label{renal-biopsy}

Indicated in cases of nephrotic syndrome unresponsive to steroids,
unexplained renal failure, or to confirm a specific glomerular disease.

\section{Management Principles}\label{management-principles-1}

The management of pediatric kidney disease depends on the underlying
cause but follows certain common principles.

\subsection{\texorpdfstring{\textbf{1. General Supportive
Care}}{1. General Supportive Care}}\label{general-supportive-care}

\begin{itemize}
\tightlist
\item
  Control of \textbf{blood pressure} using ACE inhibitors or calcium
  channel blockers.\\
\item
  Maintenance of \textbf{fluid and electrolyte balance}.\\
\item
  Correction of metabolic acidosis and anemia.\\
\item
  Adequate nutrition to support growth and prevent catabolism.
\end{itemize}

\subsection{\texorpdfstring{\textbf{2. Disease-Specific
Therapy}}{2. Disease-Specific Therapy}}\label{disease-specific-therapy}

\begin{itemize}
\tightlist
\item
  \textbf{Nephrotic syndrome:} corticosteroids are first-line; resistant
  cases may need cyclophosphamide or calcineurin inhibitors.\\
\item
  \textbf{Acute glomerulonephritis:} mainly supportive; antibiotics for
  streptococcal infection; control of hypertension and edema.\\
\item
  \textbf{UTIs:} treated with appropriate antibiotics and preventive
  measures such as hydration and bladder hygiene.\\
\item
  \textbf{Obstructive uropathy:} surgical intervention to relieve
  obstruction.\\
\item
  \textbf{AKI:} manage underlying cause, ensure adequate perfusion, and
  initiate dialysis when necessary.\\
\item
  \textbf{CKD:} slow progression through blood pressure control, treat
  anemia, and prepare for renal replacement therapy.
\end{itemize}

\subsection{\texorpdfstring{\textbf{3. Dialysis and Renal
Replacement}}{3. Dialysis and Renal Replacement}}\label{dialysis-and-renal-replacement}

Indicated in severe AKI or end-stage renal disease.\\
- \textbf{Peritoneal dialysis} is often preferred in children due to
simplicity and better hemodynamic tolerance.\\
- \textbf{Haemodialysis} is used in older children when facilities
permit.\\
- \textbf{Kidney transplantation} offers the best long-term outcome,
though access is limited in Ghana.

\subsection{\texorpdfstring{\textbf{4. Psychosocial and Family
Support}}{4. Psychosocial and Family Support}}\label{psychosocial-and-family-support}

Chronic kidney disease imposes psychological and financial burdens.
Family counselling, nutritional education, and social support are
integral to management.

\section{Complications}\label{complications-20}

Untreated or poorly managed kidney disease can result in severe
complications: - Hypertensive crisis\\
- Chronic kidney disease and end-stage renal failure\\
- Electrolyte disturbances (hyperkalaemia, hyponatraemia)\\
- Growth failure and bone disease\\
- Cardiovascular complications\\
- Infections from immunosuppression or dialysis\\
- Anaemia and fatigue

\section{Prevention}\label{prevention-11}

Many causes of renal disease in children are preventable.

Key preventive measures include: - \textbf{Antenatal care} to detect
congenital anomalies early.\\
- \textbf{Prompt treatment of infections}, particularly streptococcal
throat and skin infections.\\
- \textbf{Avoidance of nephrotoxic drugs} (e.g., aminoglycosides,
NSAIDs).\\
- \textbf{Adequate hydration} during diarrhoeal or febrile illnesses.\\
- \textbf{Health education} on hygiene and sanitation to prevent UTIs.\\
- \textbf{Early referral} for persistent oedema, hematuria, or
hypertension.

\section{Prognosis}\label{prognosis-21}

The outcome varies with the cause and stage at diagnosis.\\
- \textbf{Acute glomerulonephritis} generally resolves completely with
supportive care.\\
- \textbf{Steroid-sensitive nephrotic syndrome} has an excellent
prognosis though relapses are common.\\
- \textbf{Chronic kidney disease} progresses slowly but inevitably to
renal failure without intervention.\\
Early detection and multidisciplinary care significantly improve
survival and quality of life.

\section{Conclusion}\label{conclusion-23}

The spectrum of kidney diseases in children is wide and complex,
encompassing congenital, infectious, immune, and systemic disorders. In
Ghana and similar settings, late presentation and limited diagnostic
resources often worsen outcomes.

Medical students and young clinicians must develop a strong foundation
in recognizing early signs, performing appropriate investigations, and
instituting timely management. With better public health measures,
increased awareness, and improved access to paediatric nephrology
services, the burden of childhood renal disease can be significantly
reduced.

\chapter{Hypertension}\label{hypertension}

\section{Introduction}\label{introduction-29}

Blood pressure is the force exerted by the blood against any unit area
of the vessel wall. Physiologically,
\[BP = CO \times TPR = SV \times HR \times TPR\] Where:

\begin{itemize}
\tightlist
\item
  \(HR\) is the Heart Rate
\item
  \(BP\) is the Blood Pressure
\item
  \(TPR\) is the Total Peripheral Resistance
\item
  \(CO\) is the Cardiac Output
\item
  \(SV\) is the stroke volume
\end{itemize}

\section{Ways of measuring blood
pressure}\label{ways-of-measuring-blood-pressure}

\begin{enumerate}
\def\labelenumi{\arabic{enumi}.}
\tightlist
\item
  \textbf{Direct intra-arterial} measurements by placing a catheter into
  the vessel and measuring the pressure ``in line'' with the vessel
  (end-on-pressure). This method is used by physiologists and
  Intensivists. The principle is employed in the measurements of central
  venous pressure and intracranial pressure in clinical practice.
\item
  \textbf{The auscultatory method} is done with the use of a
  sphygmomanometer (either mercury or aneroid) and a stethoscope. This
  is the gold standard in clinical practice. Korotkoff sounds 1 and 5
  sounds are measured for systolic and diastolic bleed pressures
  respectively. Values obtained are generally lower than direct \&
  oscillometric measurements.
\item
  \textbf{The palpation method} (flush technique) is performed with the
  use of a sphygmomanometer and palpating finger. Largely unreliable.
  Only systolic blood pressure can be measured with this technique. The
  palpated pulse is generally lower than Korotkoff sound 1 by 10mmHg.
\item
  \textbf{The oscillometric method} uses a sphygmomanometer and a
  monitor e.g.~digital blood pressure devices and Dynamap. Here,
  pulsatile blood flow through arterial wall oscillations is transmitted
  to the cuff encircling the extremity. Korotkoff sound 1 is recorded at
  the point of rapid increase in oscillation amplitude. Korotkoff sound
  5 is recorded as the point of a sudden decrease in oscillation
  amplitude. Values obtained by oscillometric measurements are generally
  higher than auscultatory.
\item
  \textbf{Doppler ultrasound technique}: Here a Doppler ultrasound is
  held over the pulse to magnify the sound so that it is audible without
  a stethoscope. The sound detected may be 5mmHg higher than Korotkoff
  sound 1.
\item
  \textbf{Ambulatory blood pressure measurements}. Here, multiple
  measurements are recorded over time (e.g.~24 hours) with digital
  devices attached to the limb whilst the patient engages in normal
  activities outside the hospital. Results are analysed on a computer or
  paper tracer built into the device using the mean of the readings. It
  provides a truer picture of blood pressure trends useful in diagnosing
  ``white coat hypertension'' and nocturnal hypertension (absence of a
  normal physiological drop in blood pressure during sleep).
\end{enumerate}

\section{Definition of Hypertension in
children}\label{definition-of-hypertension-in-children}

\textbf{In adults}, the epidemiological definition is based on the risk
of adverse events (e.g.~Stroke) being\textgreater140/90mmHg. \textbf{In
children}, hypertension is defined statistically based on normative
data: ≥ 95th centile for age, height, and gender (Refer to height
centile chart and blood pressure levels). By this statistical
definition, 5\% of children will be classified as hypertensives. Other
definitions include:

\begin{itemize}
\item
  \textbf{Normal blood pressure}: \textless{} 90th centile for age,
  height, and sex.
\item
  \textbf{Pre-Hypertension}: 90th -- \textless95th centile for age,
  height, and sex
\item
  \textbf{Stage 1 Hypertension}: 95th - 99th + 5 mmHg
\item
  \textbf{Stage 2 Hypertension}: \emph{\textgreater{} 99th centile +
  5mmHg}

  A sample of the blood pressure chart is shown below.
\end{itemize}

\begin{figure}[H]

{\centering \pandocbounded{\includegraphics[keepaspectratio]{images/bpchart.jpg}}

}

\caption{Blood Pressure Centile Chart}

\end{figure}%

\section{Plotting the blood pressure
centile}\label{plotting-the-blood-pressure-centile}

\begin{enumerate}
\def\labelenumi{\arabic{enumi}.}
\tightlist
\item
  Measure the child's height.
\item
  Determine the height centile. If the height centile falls between 2
  centiles, use the closest centile. Otherwise, use the lower height
  centile.
\item
  Determine the blood pressure centile.
\item
  Classify blood pressure using the definitions above.
\end{enumerate}

\section{Hypertensive emergency}\label{hypertensive-emergency}

This is an acutely elevated blood pressure with evidence of threatening
end-organ damage involving the following organs:

\begin{itemize}
\tightlist
\item
  Brain (severe headache, visual changes, cranial nerve palsy,
  papilloedema)
\item
  Heart (acute chest pain and tightness, shortness of breath)
\item
  Kidney (decreased urine output acutely, proteinuria and haematuria on
  dipstick)
\end{itemize}

It is thus a symptomatic, severe Hypertension.

\section{Hypertensive Urgency}\label{hypertensive-urgency}

This is severe hypertension without evidence of end-organ damage or
symptoms. The blood pressure should nevertheless be treated urgently but
not aggressively like in a hypertensive emergency to prevent progression
into a hypertensive emergency. If possible, the patient should be
managed as in-patient.

\section{Rules of blood pressure
measurement}\label{rules-of-blood-pressure-measurement}

\begin{enumerate}
\def\labelenumi{\arabic{enumi}.}
\tightlist
\item
  Select the right cuff size.

  \begin{itemize}
  \tightlist
  \item
    The length of the inflation bladder should be at least 80\% of the
    mid-arm circumference.
  \item
    The width of the inflation bladder is at least 40th of the mid-arm
    circumference.
  \end{itemize}
\item
  The child should rest for at least 5 minutes in a comfortable
  environment and position.
\item
  Arm resting and supported at heart level (The reference level. Values
  outside this reference level are higher). The lower edge of the cuff
  is 2cm above the cubital fossa.
\item
  Bladder tubings should lie over the brachial artery.
\item
  The Bell of the stethoscope is used.
\item
  Korotkoff sounds 1 and 5 are used for systolic and diastolic
  respectively.
\item
  Multiple measurements are made (preferably at different settings) and
  the lowest reading is taken. For research purposes, 3 measurements are
  taken and an average of the last 2 used.
\end{enumerate}

Blood pressure readings obtained in the legs are 10-20mmHg higher than
the arm pressure in any individual. Arm blood pressure higher than leg
blood pressure occurs in aortic coarctation distal to ductus arteriosus.

\section{When to suspect
hypertension}\label{when-to-suspect-hypertension}

Suspect hypertension in any child with any of the following conditions:

\begin{itemize}
\tightlist
\item
  Alteration in consciousness including aggressive behavior and
  convulsion
\item
  Oedematous
\item
  Known kidney disease or evidence of abnormal urinalysis
\item
  Heart failure
\item
  Obesity
\item
  Failure to thrive
\item
  Stroke or other palsies including cranial nerve palsy
\item
  History of Low Birth Weight (small number of nephrons)
\item
  Unexplained anaemia, or blurred vision
\item
  Neurofibromatosis
\item
  Other syndromes like Turner \& Williams
\end{itemize}

\section{Aetiology of hypertension}\label{aetiology-of-hypertension}

Generally, childhood Hypertension is considered to be of secondary cause
until proven otherwise. This is particularly so among the very young and
the severely hypertensive. The majority (\textasciitilde80\%) are of
renal origin. However, the number of children with essential
Hypertension is on the rise, particularly among obese adolescents and
those with a positive family history.

Broadly, aetiology can be categorized into:

\begin{itemize}
\tightlist
\item
  Renal disease
\item
  Vascular disorders
\item
  Endocrine causes
\item
  Neurologic causes
\item
  Renal tumours
\item
  Catecholamine-secreting tumours
\item
  Drug-induced
\item
  Miscellaneous causes
\end{itemize}

However, since these are often age-specific categorizations are done by
age as below:

\subsection{Neonate to one-year}\label{neonate-to-one-year}

\textbf{Congenital}

\begin{itemize}
\tightlist
\item
  Congenital lesions of the vasculature

  \begin{itemize}
  \tightlist
  \item
    Renal Artery Stenosis
  \item
    Aortic coarctation
  \end{itemize}
\item
  Congenital lesions of renal parenchyma

  \begin{itemize}
  \tightlist
  \item
    Polycystic Kidney disease
  \item
    Dysplastic kidneys
  \item
    Obstructive uropathy
  \end{itemize}
\item
  Congenital Adrenal Hyperplasia

  \begin{itemize}
  \tightlist
  \item
    11-β hydroxylase deficiency
  \item
    17-αhydroxylase def
  \end{itemize}
\end{itemize}

\textbf{Acquired}

\begin{itemize}
\tightlist
\item
  Renal artery or vein thrombosis secondary to umbilical artery or vein
  catheterisation
\item
  Bronchopulmonary dysplasia
\item
  Medications

  \begin{itemize}
  \tightlist
  \item
    Theophylline/caffeine
  \item
    Phenylephrine and Ephedrine Nasal Drops in cold medications
  \item
    Steroids
  \item
    Vitamin D intoxication
  \end{itemize}
\item
  Total Parental Nutrition (high Ca2+)
\item
  Maternal drug use: Cocaine, heroin
\end{itemize}

\subsection{One- to five years}\label{one--to-five-years}

\begin{itemize}
\tightlist
\item
  Renal Artery Stenosis
\item
  Glomerulonephritis
\item
  Renal vein thrombosis
\item
  Wilms tumour
\item
  Neuroblastoma
\item
  Phaeochromocytoma
\item
  Cystic kidney disease
\item
  Monogenic Hypertension (e.g.~Liddle's syndrome)
\end{itemize}

\subsection{Five- to ten-years}\label{five--to-ten-years}

\begin{itemize}
\tightlist
\item
  Glomerulonephritis
\item
  Renal scars from reflux nephropathies or Urinary Tract Infections
\item
  Renal Artery Stenosis
\item
  Cystic renal disease
\item
  Endocrine tumours
\item
  Essential Hypertension
\item
  Obesity
\end{itemize}

\subsection{Ten- to twenty-years}\label{ten--to-twenty-years}

\begin{itemize}
\tightlist
\item
  Obesity
\item
  Essential hypertension
\item
  Reflux nephropathies with repeated Urinary Tract Infections
\item
  Glomerulonephritis
\item
  Renal Artery Stenosis
\item
  Endocrine tumours
\item
  Hyperthyroidism
\item
  Drugs (Oral Contraceptive Pill, illicit drugs)
\end{itemize}

\section{Evaluation of the Hypertensive
Child}\label{evaluation-of-the-hypertensive-child}

\begin{itemize}
\tightlist
\item
  Patient's history
\item
  Symptoms of renal disease (haematuria, oliguria, evidence of bodily
  swelling, polyuria, enuresis)
\item
  Symptoms of vasculitis or rheumatology ( Joint swelling \& rash)
\item
  Past medical history (umbilical artery/vein catheterisation, previous
  renal disease e.g. Previous swelling)
\item
  Drug History (steroids, Oral Contraceptive Pill, amphetamines, other
  illicit drugs)
\item
  Birth History: Low Birth Weight
\item
  Family History of Hypertension
\end{itemize}

Clues on physical examination include:

\begin{itemize}
\tightlist
\item
  Coarctation of the Aorta \& Takayasu:

  \begin{itemize}
  \tightlist
  \item
    Femoral artery delay or imperceptible
  \item
    Blood pressure discrepancy between arm \& leg →COA, Takayasu
    arteritis
  \end{itemize}
\item
  Neurofibromatosis

  \begin{itemize}
  \tightlist
  \item
    Cafѐ au lait spots
  \end{itemize}
\item
  RAS, Takayasu arteritis

  \begin{itemize}
  \tightlist
  \item
    Abdominal bruit
  \end{itemize}
\item
  Congenital adrenal hyperplasia

  \begin{itemize}
  \tightlist
  \item
    Ambiguous genitalia
  \end{itemize}
\item
  Dysmorphism suggestive of Turner or William syndromes
\item
  Signs of Chronic Renal Failure: Growth failure (stunted), renal
  rickets, anaemia, oedema
\item
  Bedside urine dipstick positive for protein and blood (± oedema)
\end{itemize}

\section{Investigations}\label{investigations-22}

The rationale is 2-fold:

\begin{enumerate}
\def\labelenumi{\arabic{enumi}.}
\tightlist
\item
  To define aetiology
\item
  To assess the presence of end-organ damage
\end{enumerate}

Some of the investigations include:

\begin{itemize}
\tightlist
\item
  Full blood count
\item
  Urine dipstick, microscopy and culture
\item
  BUE, Serum Creatinine, Ca, Mg, PO4, blood gases
\item
  Uric acid
\item
  KUB ultrasound and Doppler studies to rule out Renal Artery Stenosis
\item
  Chest X-ray for cardiomegaly
\item
  Echocardiogram for Left Ventricular Hypertrophy (end organ damage)
\item
  Fundoscopy
\item
  Plasma Renin Activity (PRA) for RAS \& renin secreting tumours
\item
  Pre/post captopril nuclear scan
\item
  MRA or CT Angiogram
\item
  DMSA scan for renal scars
\item
  Urine HVA \& VMA for catechol amine secreting tumours/MIBG
  scintigraphy
\end{itemize}

\section{Uric Acid and hypertension}\label{uric-acid-and-hypertension}

Uric acid is increasingly being implicated in the pathogenesis of
Hypertension in both adults and children. It is believed to cause
endothelial dysfunction leading to microvascular and inflammatory injury
to the kidneys. There are also reduced levels of endothelial-derived
nitric oxide and associated elevation of the
Renin-Aldosterone-Angiotensin System. Elevated uric acid levels in
hypertensive individuals are associated with adverse outcomes like
stroke. Allopurinol treatment is advocated for such individuals.

\section{Complication of
Hypertension}\label{complication-of-hypertension}

Some complications of Hypertension are listed below:

\begin{itemize}
\tightlist
\item
  Hypertensive encephalopathy
\item
  Left Ventricular Failure
\item
  Stroke
\item
  Subarachnoid haemorrhage
\item
  Secondary renal damage
\item
  Retinopathy
\end{itemize}

\section{Treatment of hypertension}\label{treatment-of-hypertension}

\subsection{Non-drug treatment}\label{non-drug-treatment}

\begin{itemize}
\tightlist
\item
  Reducing salt intake
\item
  Weight reduction for obesity-related hypertension
\item
  Intake of more vegetables on account of potassium richness
\end{itemize}

\subsection{Drug Treatment}\label{drug-treatment}

Principles of anti-hypertensive therapy:

\begin{itemize}
\tightlist
\item
  Long-acting (once-daily medication)
\item
  Maximise treatment dosage before adding on
\item
  Agents used will come from the ``ABCD'' group:

  \begin{itemize}
  \tightlist
  \item
    \textbf{A}CE inhibitor and ARBs (Avoid if RAS suspected or in
    hypovolaemia)
  \item
    \textbf{B}eta-blocker
  \item
    \textbf{C}alcium channel blocker
  \item
    \textbf{D}iuretic
  \item
    \textbf{E}very other drug (methyl dopa, alpha-blockers, vasodilators
    like hydralazine
  \end{itemize}
\end{itemize}

Generally, \textbf{A \& B} drugs are not combined for Blood pressure
control. Rather: \textbf{A + C + D} or \textbf{B + C + D}

\section{Hypertensive encephalopathy}\label{hypertensive-encephalopathy}

Hypertension with changes in mental status and/or seizures. Other
manifestations are:

\begin{itemize}
\tightlist
\item
  Facial palsy
\item
  Visual changes→blindness
\item
  Coma
\end{itemize}

\textbf{Pathophysiology}: Disruption of the normal autoregulatory
mechanisms of cerebral blood flow. The inability of cerebral vasculature
to constrict appropriately in response to the abrupt increase in
cerebral blood flow leads to cerebral hyperperfusion. Generally,
short-acting antihypertensives are preferred in the initial instance of
treatment so that any potentially harmful drop in blood pressure (which
could lead to Posterior Reversible Encephalopathy Syndrome \{PRES\})
could be reversed. Subsequently, long-acting agents could be used
Sublingual nifedipine could cause a precipitous drop in blood pressure
so it is best avoided or should be used with extreme caution.

Treatment outline:

\begin{itemize}
\tightlist
\item
  Use anti-hypertensive drugs
\item
  Blood pressure should be brought down slowly to a desirable level
  (?stage I) by 48hrs (though not to normal levels) as follows:

  \begin{itemize}
  \tightlist
  \item
    1/3 of total blood pressure reduction in 1st 12-hrs
  \item
    Next one-third of the subsequent 12-hrs
  \item
    Final one-third over 24-hrs
  \end{itemize}
\item
  Alternatively, by a quarter within 6 hours, and the rest in the next
  24-36hrs
\end{itemize}

Commonly preferred drugs include Labetalol infusion, Na nitroprusside
infusion, and IV hydralazine infusion. After achieving the desired blood
pressure target, oral antihypertensives are then started.

\chapter{Urinary Tract Infection}\label{urinary-tract-infection}

\section{Introduction}\label{introduction-30}

Urinary tract infection (UTI) is one of the most frequent bacterial
infections in childhood, second only to respiratory infections. It
represents an invasion of the urinary tract by pathogenic
microorganisms, most commonly \emph{Escherichia coli}.\\
In children, UTI can occur at any age and often presents with
non-specific symptoms, especially in neonates and infants. Because the
infection may indicate underlying structural or functional abnormalities
of the urinary tract, careful diagnosis and follow-up are essential.

Globally, and in Ghana, UTI contributes significantly to paediatric
morbidity and can lead to long-term complications such as renal
scarring, hypertension, and chronic kidney disease if not promptly
treated.

\section{Epidemiology}\label{epidemiology-3}

The incidence of UTI in children varies with age and sex: - In the
\textbf{neonatal period}, UTIs are more common in boys, particularly
those who are uncircumcised. - After infancy, the \textbf{female-to-male
ratio} increases sharply because of the shorter urethra and its
proximity to the anus. - Approximately \textbf{8\% of girls and 2\% of
boys} experience at least one symptomatic UTI before 7 years of age. -
Recurrence rates may reach \textbf{30--40\%}, especially among those
with vesicoureteral reflux (VUR) or bladder dysfunction.

In low- and middle-income countries such as Ghana, poor hygiene, delayed
treatment of fever, and limited access to imaging services may
contribute to underdiagnosis and recurrent infections.

\section{Aetiology and Risk Factors}\label{aetiology-and-risk-factors-1}

\subsection{Microbiology}\label{microbiology}

\begin{itemize}
\tightlist
\item
  \textbf{Gram-negative bacilli} are predominant:

  \begin{itemize}
  \tightlist
  \item
    \emph{Escherichia coli} (responsible for 70--90\% of cases)
  \item
    \emph{Klebsiella}, \emph{Proteus}, \emph{Enterobacter},
    \emph{Pseudomonas}\\
  \end{itemize}
\item
  \textbf{Gram-positive organisms} such as \emph{Enterococcus faecalis}
  and \emph{Staphylococcus saprophyticus} are less common.
\item
  In neonates, \emph{Group B Streptococcus} and \emph{Staphylococcus
  aureus} may be isolated.
\end{itemize}

\subsection{Predisposing Factors}\label{predisposing-factors}

\begin{enumerate}
\def\labelenumi{\arabic{enumi}.}
\tightlist
\item
  \textbf{Anatomical abnormalities} --- posterior urethral valves,
  vesicoureteral reflux, hydronephrosis.
\item
  \textbf{Functional abnormalities} --- neurogenic bladder,
  constipation, dysfunctional voiding.
\item
  \textbf{Incomplete bladder emptying} or \textbf{obstruction}.
\item
  \textbf{Poor perineal hygiene} and \textbf{urinary stasis}.
\item
  \textbf{Uncircumcised males} --- foreskin colonization increases
  bacterial adherence.
\item
  \textbf{Instrumentation} --- catheterization or cystoscopy.
\item
  \textbf{Systemic conditions} --- diabetes mellitus, immunodeficiency,
  malnutrition.
\item
  \textbf{Dehydration} and inadequate fluid intake.
\end{enumerate}

\section{Pathophysiology}\label{pathophysiology-19}

UTI occurs when microorganisms colonize the periurethral area and ascend
through the urethra into the bladder (cystitis) and, in some cases,
further to the kidneys (pyelonephritis).

\subsection{Mechanisms}\label{mechanisms}

\begin{itemize}
\tightlist
\item
  \textbf{Ascending infection:} the most common pathway; bacteria
  migrate from the perineum, facilitated by poor hygiene or reflux.
\item
  \textbf{Hematogenous spread:} less common; seen in neonates or
  immunocompromised children with bacteremia.
\item
  \textbf{Lymphatic spread:} rare and of uncertain significance.
\end{itemize}

\textbf{Virulence factors} of uropathogens include: - \textbf{Fimbriae
(pili):} enhance adherence to uroepithelial cells. - \textbf{Capsular
polysaccharides:} resist phagocytosis. - \textbf{Hemolysins and toxins:}
cause epithelial injury. - \textbf{Biofilm formation:} enables
persistence and recurrence.

\textbf{Host defenses} such as urine flow, mucosal IgA, and epithelial
turnover normally prevent infection. When these defenses are impaired,
infection takes hold.

\section{Classification}\label{classification-2}

UTIs are classified based on location, severity, and recurrence.

\subsection{Based on Location}\label{based-on-location}

\begin{itemize}
\tightlist
\item
  \textbf{Lower UTI (Cystitis):} infection confined to bladder and
  urethra.
\item
  \textbf{Upper UTI (Pyelonephritis):} infection involves renal
  parenchyma, usually with systemic features.
\end{itemize}

\subsection{Based on Severity}\label{based-on-severity}

\begin{itemize}
\tightlist
\item
  \textbf{Uncomplicated:} infection in an otherwise healthy urinary
  tract.
\item
  \textbf{Complicated:} associated with structural/functional
  abnormalities or systemic illness.
\end{itemize}

\subsection{Based on Recurrence}\label{based-on-recurrence}

\begin{itemize}
\tightlist
\item
  \textbf{Recurrent UTI:} ≥2 episodes in six months or ≥3 within a year.
\item
  \textbf{Relapse:} infection by same organism within two weeks of
  treatment.
\item
  \textbf{Reinfection:} infection by a new organism after successful
  therapy.
\end{itemize}

\section{Clinical Features}\label{clinical-features-18}

The presentation varies with age, making clinical suspicion critical.

\subsection{Neonates and Infants}\label{neonates-and-infants}

\begin{itemize}
\tightlist
\item
  Fever (may be absent in neonates)
\item
  Poor feeding, vomiting, lethargy
\item
  Jaundice
\item
  Failure to thrive
\item
  Hypothermia or irritability
\end{itemize}

\subsection{Older Children}\label{older-children-1}

\begin{itemize}
\tightlist
\item
  Dysuria, frequency, urgency
\item
  Suprapubic pain
\item
  Foul-smelling or cloudy urine
\item
  Hematuria
\item
  Fever and flank pain (if pyelonephritis)
\end{itemize}

\subsection{School-Age and
Adolescents}\label{school-age-and-adolescents}

\begin{itemize}
\tightlist
\item
  Classic lower tract symptoms (frequency, dysuria)
\item
  Abdominal or flank pain
\item
  Occasionally incontinence or enuresis
\end{itemize}

Because symptoms are often non-specific, any child with unexplained
fever, particularly under 2 years of age, should be evaluated for UTI.

\section{Differential Diagnosis}\label{differential-diagnosis-18}

\begin{itemize}
\tightlist
\item
  Viral cystitis\\
\item
  Vulvovaginitis or balanitis\\
\item
  Appendicitis\\
\item
  Gastroenteritis\\
\item
  Renal stones\\
\item
  Glomerulonephritis (if hematuria and proteinuria present)\\
\item
  Fever of unknown origin
\end{itemize}

\section{Investigations}\label{investigations-23}

\subsection{Urine Collection Methods}\label{urine-collection-methods}

Accurate diagnosis depends on obtaining a clean sample. -
\textbf{Clean-catch midstream urine} (toilet-trained children). -
\textbf{Catheterization} or \textbf{suprapubic aspiration} (infants). -
\textbf{Urine bag collection} --- often contaminated; used only for
screening.

\subsection{Laboratory Evaluation}\label{laboratory-evaluation-1}

\begin{enumerate}
\def\labelenumi{\arabic{enumi}.}
\tightlist
\item
  \textbf{Urinalysis}

  \begin{itemize}
  \tightlist
  \item
    \textbf{Leukocyte esterase and nitrite tests:} simple bedside
    screening.
  \item
    \textbf{Microscopy:} ≥5--10 WBCs per high-power field suggests
    infection; presence of bacteria reinforces diagnosis.
  \end{itemize}
\item
  \textbf{Urine Culture}

  \begin{itemize}
  \tightlist
  \item
    Gold standard for diagnosis.
  \item
    Significant growth:

    \begin{itemize}
    \tightlist
    \item
      ≥10⁵ CFU/mL (clean catch)\\
    \item
      ≥10⁴ CFU/mL (catheter specimen)
    \end{itemize}
  \item
    Identifies organism and antibiotic sensitivity.
  \end{itemize}
\item
  \textbf{Blood Tests}

  \begin{itemize}
  \tightlist
  \item
    Full blood count (raised WBC count).
  \item
    ESR or CRP (elevated in pyelonephritis).
  \item
    Renal function tests (urea, creatinine, electrolytes).
  \end{itemize}
\item
  \textbf{Imaging Studies}

  \begin{itemize}
  \tightlist
  \item
    \textbf{Renal and bladder ultrasound:} after first febrile UTI to
    detect structural anomalies.
  \item
    \textbf{Micturating cystourethrogram (MCUG):} for recurrent or
    atypical cases to identify vesicoureteral reflux.
  \item
    \textbf{DMSA scan:} assesses renal scarring and differential renal
    function.
  \end{itemize}
\end{enumerate}

\subsection{Diagnostic Criteria}\label{diagnostic-criteria}

Diagnosis requires both \textbf{clinical features} and
\textbf{microbiological evidence}.\\
In infants, UTI should be suspected in any febrile illness without an
obvious focus, and confirmed through culture before or soon after
antibiotic initiation.

\section{Management}\label{management-14}

Prompt diagnosis and appropriate therapy are crucial to prevent renal
damage.

\subsection{1. Acute (Emergency)
Management}\label{acute-emergency-management}

Children presenting with fever, dehydration, vomiting, or systemic
toxicity should be hospitalized and started on parenteral antibiotics
after urine collection. - \textbf{Initial antibiotics (parenteral):} -
Cefotaxime, ceftriaxone, or gentamicin (adjust to local resistance
patterns). - \textbf{Supportive care:} - Adequate hydration (IV or
oral). - Antipyretics and pain relief. - Monitor urine output and renal
function.

\subsection{2. Oral Therapy for Stable
Patients}\label{oral-therapy-for-stable-patients}

For older or less ill children with uncomplicated cystitis: -
\textbf{Oral agents:} amoxicillin-clavulanate, cefixime, or
cotrimoxazole (guided by culture). - Duration:\\
- Cystitis --- 5--7 days\\
- Pyelonephritis --- 10--14 days\\
Adjust antibiotics based on sensitivity results.

\subsection{3. Follow-Up and Ongoing
Management}\label{follow-up-and-ongoing-management}

\begin{itemize}
\tightlist
\item
  Reassess clinical response within 48--72 hours.
\item
  Repeat urinalysis and culture after completion of therapy.
\item
  Persistent fever or bacteriuria warrants imaging for obstruction or
  reflux.
\item
  Evaluate for underlying abnormalities after first febrile UTI,
  especially in children \textless2 years.
\end{itemize}

\subsection{4. Management of Recurrent
UTI}\label{management-of-recurrent-uti}

\begin{itemize}
\tightlist
\item
  Identify and treat predisposing factors such as constipation,
  dysfunctional voiding, or vesicoureteral reflux.
\item
  Prophylactic low-dose antibiotics (e.g., nightly nitrofurantoin or
  trimethoprim) may be considered for high-risk patients.
\item
  Encourage regular voiding and adequate hydration.
\item
  Periodic urine monitoring.
\end{itemize}

\subsection{5. Management of Complicated
UTI}\label{management-of-complicated-uti}

Complicated cases (e.g., with obstruction, abscess, or sepsis) require:
- Hospitalization - IV antibiotics (broader spectrum) - Possible
urologic intervention - Multidisciplinary care with paediatric
nephrology/urology teams.

\section{Complications}\label{complications-21}

Untreated or recurrent UTI can lead to: - \textbf{Renal scarring} and
cortical atrophy. - \textbf{Hypertension} (secondary to scarring). -
\textbf{Proteinuria and CKD.} - \textbf{Perinephric abscess.} -
\textbf{Urosepsis}, especially in neonates. - \textbf{Growth
retardation} in chronic cases.

\section{Prevention}\label{prevention-12}

Preventive measures are essential, especially in endemic and
resource-limited settings.

\subsection{Behavioural and Hygiene
Measures}\label{behavioural-and-hygiene-measures}

\begin{itemize}
\tightlist
\item
  Encourage frequent voiding and complete bladder emptying.
\item
  Ensure adequate hydration.
\item
  Teach proper perineal hygiene (front-to-back wiping for girls).
\item
  Avoid prolonged use of tight or synthetic clothing.
\item
  Manage constipation promptly.
\end{itemize}

\subsection{Medical Measures}\label{medical-measures}

\begin{itemize}
\tightlist
\item
  Early treatment of bladder dysfunction or obstruction.
\item
  Circumcision may reduce risk in recurrent UTI among boys.
\item
  Prophylactic antibiotics in selected high-risk children.
\item
  Immunization and prompt care of febrile illness.
\end{itemize}

\subsection{Community and Public Health
Measures}\label{community-and-public-health-measures}

\begin{itemize}
\tightlist
\item
  Improve sanitation and access to clean water.
\item
  Educate parents and caregivers on recognizing early signs of UTI.
\item
  Integrate UTI screening into child health programs.
\end{itemize}

\section{Prognosis}\label{prognosis-22}

With early diagnosis and appropriate management, most children recover
fully without long-term sequelae.\\
However, risk of renal damage increases with: - Delayed treatment
(\textgreater48 hours of fever) - Recurrent infections - Presence of
vesicoureteral reflux or obstruction - Poor adherence to therapy

Regular follow-up and imaging, where feasible, are key to preserving
renal function.

\section{Summary}\label{summary-1}

UTI in children is a common but potentially serious infection. The
clinical picture varies with age, making early recognition essential.\\
Diagnosis requires proper urine collection and culture confirmation.\\
Treatment should be prompt, guided by local bacterial sensitivity
patterns, and followed by evaluation for underlying structural
abnormalities.

In Ghana and other similar settings, emphasis must be placed on hygiene,
caregiver education, and accessible diagnostic services. Preventing
renal damage through timely treatment remains the ultimate goal of
managing childhood UTI.

\begin{center}\rule{0.5\linewidth}{0.5pt}\end{center}

\chapter{Hematuria}\label{hematuria}

\section{Introduction}\label{introduction-31}

Hematuria refers to the presence of red blood cells (RBCs) in the urine.
It is one of the most common and sometimes alarming urinary findings in
children. While it may be transient and benign in some cases, in others
it may signal significant renal or urinary tract pathology requiring
urgent evaluation.

In paediatrics, distinguishing between glomerular and non-glomerular
causes is crucial, as it guides both investigation and management. In
Ghana and other low- and middle-income countries, infections,
post-streptococcal glomerulonephritis, and schistosomiasis remain
prominent causes, although hereditary and structural causes are also
seen.

\section{Definitions and
Classification}\label{definitions-and-classification}

Hematuria can be:

\begin{itemize}
\tightlist
\item
  \textbf{Macroscopic (gross):} Urine is visibly red or cola-coloured.
\item
  \textbf{Microscopic:} RBCs are seen only under the microscope
  (\textgreater5 RBCs per high-power field in a centrifuged sample).
\end{itemize}

It may also be \textbf{transient} (short-lived, e.g., after fever,
exercise, or minor trauma) or \textbf{persistent} (detected on ≥3
separate occasions over weeks).

\section{Pathophysiology}\label{pathophysiology-20}

The appearance of RBCs in urine reflects disruption along any part of
the urinary tract.

\begin{itemize}
\tightlist
\item
  \textbf{Glomerular hematuria} occurs when the glomerular basement
  membrane (GBM) is damaged, allowing RBCs to pass into Bowman's space.
  These cells are often dysmorphic due to osmotic and mechanical stress
  as they traverse the nephron.\\
\item
  \textbf{Non-glomerular hematuria} arises from bleeding beyond the
  glomerulus --- the renal pelvis, ureter, bladder, or urethra where
  RBCs maintain their normal morphology.
\end{itemize}

Mechanisms include:

\begin{enumerate}
\def\labelenumi{\arabic{enumi}.}
\tightlist
\item
  \textbf{Inflammation:} As seen in glomerulonephritis or cystitis.
\item
  \textbf{Mechanical injury:} From stones or trauma.
\item
  \textbf{Vascular abnormalities:} Such as renal vein thrombosis.
\item
  \textbf{Neoplastic infiltration:} Tumours like Wilms' tumour or
  rhabdomyosarcoma.
\item
  \textbf{Coagulopathies:} Affecting hemostatic mechanisms.
\end{enumerate}

\section{Common Aetiological
Categories}\label{common-aetiological-categories}

\subsection{Glomerular Causes}\label{glomerular-causes}

Usually accompanied by proteinuria, hypertension, or oedema:

\begin{enumerate}
\def\labelenumi{\arabic{enumi}.}
\tightlist
\item
  \textbf{Post-streptococcal glomerulonephritis (PSGN):} Common in
  school-age children following throat or skin infection.
\item
  \textbf{IgA nephropathy:} Episodic hematuria following infections.
\item
  \textbf{Alport syndrome:} Familial nephritis with sensorineural
  hearing loss.
\item
  \textbf{Lupus nephritis:} Especially in adolescents.
\item
  \textbf{Henoch--Schönlein purpura (HSP):} Vasculitic process involving
  the kidneys.
\end{enumerate}

\subsection{Non-glomerular Causes}\label{non-glomerular-causes}

Typically associated with pain, dysuria, or clots in urine:

\begin{enumerate}
\def\labelenumi{\arabic{enumi}.}
\tightlist
\item
  \textbf{Urinary tract infection (UTI):} Very common in young children.
\item
  \textbf{Urolithiasis:} Can occur with dehydration or metabolic
  disorders.
\item
  \textbf{Trauma:} From catheterization, accidents, or abuse.
\item
  \textbf{Structural lesions:} Such as posterior urethral valves or
  hydronephrosis.
\item
  \textbf{Tumours:} Wilms' tumour, rhabdomyosarcoma.
\item
  \textbf{Schistosomiasis:} Common in endemic areas such as parts of
  northern Ghana.
\end{enumerate}

\subsection{Systemic Causes}\label{systemic-causes}

\begin{itemize}
\tightlist
\item
  \textbf{Bleeding diatheses:} e.g., platelet disorders, hemophilia.
\item
  \textbf{Sickle cell disease:} Due to papillary necrosis or
  microinfarction.
\item
  \textbf{Drugs:} e.g., cyclophosphamide, anticoagulants.
\end{itemize}

\section{Clinical Evaluation}\label{clinical-evaluation}

A meticulous history and examination are the cornerstone of assessment.

\subsection{History}\label{history-3}

Key aspects include:

\begin{enumerate}
\def\labelenumi{\arabic{enumi}.}
\tightlist
\item
  \textbf{Duration and pattern:} Is it single episode or recurrent?
\item
  \textbf{Associated symptoms:} Dysuria, fever, flank pain, oedema, or
  rash.
\item
  \textbf{Colour of urine:} Bright red (lower tract), cola-coloured
  (glomerular).
\item
  \textbf{Timing during micturition:} Initial (urethral), terminal
  (bladder neck), or total (upper tract).
\item
  \textbf{Recent infections:} Especially sore throat, skin lesions, or
  diarrhoea.
\item
  \textbf{Family history:} Kidney disease, hearing loss, or stones.
\item
  \textbf{Exposure:} To schistosomiasis, drugs, or toxins.
\end{enumerate}

\subsection{Examination}\label{examination-1}

Focus on:

\begin{enumerate}
\def\labelenumi{\arabic{enumi}.}
\tightlist
\item
  \textbf{General appearance:} Pallor (anemia), oedema
  (nephritis/nephrotic syndrome), or rash (vasculitis).
\item
  \textbf{Vital signs:} Hypertension suggests glomerular disease.
\item
  \textbf{Abdominal exam:} Palpable kidneys, masses, tenderness.
\item
  \textbf{ENT and hearing assessment:} For Alport syndrome.
\item
  \textbf{Skin and joint findings:} Indicate systemic disease (HSP,
  lupus).
\end{enumerate}

\section{Laboratory and Imaging
Investigations}\label{laboratory-and-imaging-investigations}

Investigations are guided by clinical suspicion.

\begin{enumerate}
\def\labelenumi{\arabic{enumi}.}
\tightlist
\item
  \textbf{Urinalysis:}

  \begin{itemize}
  \tightlist
  \item
    Confirm presence of RBCs.
  \item
    Assess for proteinuria, casts, or infection.
  \item
    Dysmorphic RBCs or red cell casts → glomerular origin.
  \end{itemize}
\item
  \textbf{Urine culture:}

  \begin{itemize}
  \tightlist
  \item
    Especially when infection suspected.
  \end{itemize}
\item
  \textbf{Urine microscopy:}

  \begin{itemize}
  \tightlist
  \item
    Crystals (stones), schistosome ova, or RBC morphology.
  \end{itemize}
\item
  \textbf{Blood tests:}

  \begin{itemize}
  \tightlist
  \item
    Full blood count (infection, anaemia).
  \item
    Serum creatinine and urea (renal function).
  \item
    Complement levels (low in PSGN).
  \item
    ASO titre (evidence of streptococcal infection).
  \item
    Autoimmune screen (ANA, anti-dsDNA).
  \end{itemize}
\item
  \textbf{Imaging:}

  \begin{itemize}
  \tightlist
  \item
    \textbf{Renal ultrasound:} Detects structural abnormalities, masses,
    or hydronephrosis.
  \item
    \textbf{CT scan:} For stones or tumours if indicated.
  \item
    \textbf{Cystoscopy:} Rarely needed in children.
  \end{itemize}
\item
  \textbf{Special tests:}

  \begin{itemize}
  \tightlist
  \item
    \textbf{Hearing test:} In suspected Alport syndrome.
  \item
    \textbf{Renal biopsy:} For persistent hematuria, nephritic/nephrotic
    syndrome, or unexplained renal impairment.
  \end{itemize}
\end{enumerate}

\section{Differential Diagnosis}\label{differential-diagnosis-19}

Differentiate between glomerular and non-glomerular hematuria and other
causes of red urine such as: - Hemoglobinuria or myoglobinuria (clear on
centrifugation, dipstick positive but no RBCs). - Beetroot ingestion or
drug-induced discoloration (rifampicin, phenazopyridine). - Porphyria
(rare).

\section{Management}\label{management-15}

The approach depends on the cause and severity.

\subsection{General Principles}\label{general-principles-2}

\begin{itemize}
\tightlist
\item
  \textbf{Reassurance and follow-up:} For isolated microscopic hematuria
  without other abnormalities.
\item
  \textbf{Treat underlying cause:} Infection, stones,
  glomerulonephritis, etc.
\item
  \textbf{Monitor renal function:} Especially in recurrent or persistent
  cases.
\end{itemize}

\subsection{Specific Management}\label{specific-management}

\begin{itemize}
\tightlist
\item
  \textbf{UTI:} Appropriate antibiotics based on culture.
\item
  \textbf{PSGN:} Rest, salt restriction, antihypertensives, diuretics if
  needed.
\item
  \textbf{HSP nephritis:} Supportive, steroids if severe.
\item
  \textbf{Alport syndrome:} ACE inhibitors to reduce proteinuria,
  monitor progression.
\item
  \textbf{Stones:} Hydration, pain relief, urologic intervention.
\item
  \textbf{Schistosomiasis:} Praziquantel, and public health measures.
\item
  \textbf{Bleeding disorders:} Correction with factor replacement or
  platelet transfusion.
\end{itemize}

\section{Complications}\label{complications-22}

If left untreated or unrecognized:

\begin{enumerate}
\def\labelenumi{\arabic{enumi}.}
\tightlist
\item
  \textbf{Chronic kidney disease (CKD).}
\item
  \textbf{Hypertension.}
\item
  \textbf{Anemia from recurrent bleeding.}
\item
  \textbf{Renal scarring} (after recurrent infection).
\item
  \textbf{End-stage renal disease} in hereditary nephropathies.
\end{enumerate}

\section{Prognosis}\label{prognosis-23}

The prognosis varies:

\begin{enumerate}
\def\labelenumi{\arabic{enumi}.}
\tightlist
\item
  Transient and benign causes (e.g., post-exercise, mild UTI) resolve
  completely.
\item
  Glomerulonephritis often resolves but may progress to CKD in severe
  cases.
\item
  Genetic or structural diseases require lifelong monitoring.
\item
  Early diagnosis and management significantly improve outcomes.
\end{enumerate}

\section{Prevention}\label{prevention-13}

Preventive strategies should address both infection-related and genetic
causes:

\begin{enumerate}
\def\labelenumi{\arabic{enumi}.}
\tightlist
\item
  Early treatment of streptococcal infections.
\item
  Improved sanitation and control of schistosomiasis.
\item
  Safe use of nephrotoxic drugs.
\item
  Genetic counselling for familial disorders.
\item
  Routine urine screening in school health programs.
\end{enumerate}

\section{Summary}\label{summary-2}

Hematuria in children should never be dismissed without evaluation. The
clinician must first confirm its presence, identify whether it is
glomerular or non-glomerular, and systematically search for the
underlying cause.

In Ghana, infection-related causes such as PSGN, UTI, and
schistosomiasis remain predominant, but clinicians should maintain a
broad differential including congenital and systemic conditions. A
structured approach beginning with good history-taking, urinalysis, and
targeted investigations---guides effective management and follow-up,
ensuring children retain optimal renal function into adulthood.

\chapter{Nephrotic Syndrome}\label{nephrotic-syndrome}

\section{Introduction}\label{introduction-32}

Nephrotic syndrome (NS) is a common renal disorder in childhood
characterised by \textbf{massive proteinuria}, \textbf{hypoalbuminemia},
\textbf{generalised oedema}, and \textbf{hyperlipidemia}. It results
from increased permeability of the glomerular basement membrane,
allowing abnormal leakage of plasma proteins into urine.

It is a major cause of morbidity among Ghanaian children, particularly
between the ages of \textbf{2 and 8 years}, and is a frequent reason for
referral to paediatric renal clinics. Early recognition and appropriate
management are critical to prevent complications such as infection,
thrombosis, and renal failure.

\section{Incidence and Epidemiology}\label{incidence-and-epidemiology}

Nephrotic syndrome is one of the most common glomerular diseases in
children worldwide, but its pattern varies by geography and race.

\begin{itemize}
\tightlist
\item
  \textbf{Age:} Most cases occur between \textbf{2 and 8 years}, with a
  peak around \textbf{4 years}.
\item
  \textbf{Sex:} There is a male predominance (M:F ≈ 2:1).
\item
  \textbf{Geographical variation:}

  \begin{itemize}
  \tightlist
  \item
    In \textbf{Europe and North America}, \textbf{minimal change disease
    (MCD)} accounts for 70--90\% of idiopathic cases.
  \item
    In \textbf{West Africa}, including Ghana, \textbf{focal segmental
    glomerulosclerosis (FSGS)} is more common, and steroid resistance
    rates are higher.
  \end{itemize}
\item
  \textbf{Secondary nephrotic syndrome} may be seen with infections such
  as \emph{hepatitis B}, \emph{malaria}, \emph{HIV}, or \emph{systemic
  diseases such as lupus nephritis}.
\end{itemize}

\section{Aetiology and
Classification}\label{aetiology-and-classification}

Nephrotic syndrome can be classified into \textbf{primary (idiopathic)}
and \textbf{secondary} forms.

\subsection{Primary (Idiopathic) Nephrotic
Syndrome}\label{primary-idiopathic-nephrotic-syndrome}

No identifiable systemic cause. Includes:

\begin{itemize}
\tightlist
\item
  \textbf{Minimal Change Disease (MCD)} -- the most common cause in
  younger children; characterised by normal light microscopy but
  podocyte effacement on electron microscopy.
\item
  \textbf{Focal Segmental Glomerulosclerosis (FSGS)} -- affects older
  children; may follow MCD or occur de novo; associated with steroid
  resistance.
\item
  \textbf{Membranoproliferative Glomerulonephritis (MPGN)} -- uncommon
  but causes persistent proteinuria and reduced complement levels.
\end{itemize}

\subsection{Secondary Nephrotic
Syndrome}\label{secondary-nephrotic-syndrome}

Occurs due to identifiable systemic or renal disorders:

\begin{itemize}
\tightlist
\item
  \textbf{Infections:} Hepatitis B, HIV, malaria, syphilis,
  tuberculosis.
\item
  \textbf{Systemic diseases:} Systemic lupus erythematosus (SLE),
  Henoch-Schönlein purpura.
\item
  \textbf{Drugs:} NSAIDs, gold salts, penicillamine.
\item
  \textbf{Metabolic and inherited conditions:} Diabetes mellitus,
  amyloidosis, Alport syndrome.
\end{itemize}

\chapter{Pathophysiology}\label{pathophysiology-21}

The glomerular filtration barrier, comprising the endothelium, basement
membrane, and podocyte layer, normally prevents large proteins, such as
albumin, from passing into the urine.

In nephrotic syndrome:

\begin{itemize}
\tightlist
\item
  \textbf{Podocyte injury} or \textbf{loss of charge selectivity} leads
  to massive proteinuria (\textgreater40 mg/m²/hr or \textgreater3.5
  g/day).
\item
  \textbf{Loss of plasma proteins}, especially albumin, reduces plasma
  oncotic pressure, resulting in fluid movement into interstitial spaces
  → \textbf{oedema}.
\item
  \textbf{Hypovolemia} triggers activation of the
  renin--angiotensin--aldosterone system and antidiuretic hormone,
  worsening sodium and water retention.
\item
  The liver's compensatory response increases lipoprotein synthesis →
  \textbf{hyperlipidemia} and \textbf{lipiduria}.
\end{itemize}

Thus, the four hallmark features are:

\begin{enumerate}
\def\labelenumi{\arabic{enumi}.}
\tightlist
\item
  Proteinuria
\item
  Hypoalbuminemia
\item
  Oedema
\item
  Hyperlipidemia
\end{enumerate}

\section{Clinical Features}\label{clinical-features-19}

\subsection{General Presentation}\label{general-presentation}

Onset is usually insidious over days to weeks.

\textbf{Main features:}

\begin{itemize}
\tightlist
\item
  \textbf{Oedema:} Initially periorbital (especially in the morning),
  then generalized (anasarca).
\item
  \textbf{Ascites and pleural effusion:} From fluid transudation.
\item
  \textbf{Weight gain:} Due to fluid retention.
\item
  \textbf{Reduced urine output (oliguria):} Often dark and frothy.
\item
  \textbf{Fatigue and anorexia.}
\end{itemize}

\subsection{Examination Findings}\label{examination-findings}

\begin{itemize}
\tightlist
\item
  Puffy face, periorbital and pedal oedema
\item
  Distended abdomen with ascites
\item
  Pleural effusion causing respiratory distress.
\item
  In advanced cases, scrotal or labial swelling
\item
  Pulse may be small due to hypovolemia.
\item
  Blood pressure is usually normal or slightly raised (if renal
  impairment develops)
\end{itemize}

\section{Differential Diagnosis}\label{differential-diagnosis-20}

Other causes of oedema in children must be considered:

\begin{longtable}[]{@{}
  >{\raggedright\arraybackslash}p{(\linewidth - 2\tabcolsep) * \real{0.3535}}
  >{\raggedright\arraybackslash}p{(\linewidth - 2\tabcolsep) * \real{0.6465}}@{}}
\toprule\noalign{}
\begin{minipage}[b]{\linewidth}\raggedright
\textbf{Condition}
\end{minipage} & \begin{minipage}[b]{\linewidth}\raggedright
\textbf{Distinguishing Features}
\end{minipage} \\
\midrule\noalign{}
\endhead
\bottomrule\noalign{}
\endlastfoot
Acute glomerulonephritis & Haematuria, hypertension, mild proteinuria,
elevated ASO titre \\
Congestive heart failure & Cardiomegaly, hepatomegaly, pulmonary
congestion \\
Liver disease & Jaundice, hepatomegaly, abnormal LFTs \\
Protein-losing enteropathy & Diarrhoea, malabsorption features \\
Severe malnutrition (Kwashiorkor) & Wasting, dermatosis, low total
protein and albumin \\
\end{longtable}

\section{Investigations}\label{investigations-24}

\subsection{Urine Studies}\label{urine-studies}

\begin{itemize}
\tightlist
\item
  \textbf{Dipstick:} Heavy proteinuria (≥3+).
\item
  \textbf{24-hour urinary protein:} \textgreater40 mg/m²/hr or spot
  protein/creatinine ratio \textgreater200 mg/mmol.
\item
  \textbf{Microscopy:} Few red cells and casts (in MCD, urine is bland).
\item
  \textbf{Lipiduria:} Fat droplets or ``Maltese crosses'' under
  polarised light.
\end{itemize}

\subsection{Blood Tests}\label{blood-tests}

\begin{itemize}
\tightlist
\item
  \textbf{Serum albumin:} \textless25 g/L.
\item
  \textbf{Serum cholesterol:} Often \textgreater6.5 mmol/L.
\item
  \textbf{Electrolytes, urea, creatinine:} Assess renal function.
\item
  \textbf{Complement levels (C3, C4):} Reduced in lupus and MPGN.
\item
  \textbf{ASO titre, Hepatitis B, HIV screening} as indicated.
\end{itemize}

\subsection{Imaging}\label{imaging-1}

\begin{itemize}
\tightlist
\item
  \textbf{Renal ultrasound:} Usually normal in MCD; may show increased
  echogenicity in chronic or secondary disease.
\item
  \textbf{Chest X-ray:} May reveal pleural effusion.
\end{itemize}

\subsection{Kidney Biopsy}\label{kidney-biopsy}

Indicated when:

\begin{itemize}
\tightlist
\item
  Age \textless1 year or \textgreater10 years
\item
  Gross haematuria
\item
  Persistent hypertension or renal failure
\item
  Low complement levels
\item
  Steroid resistance or frequent relapses
\end{itemize}

\section{Diagnosis}\label{diagnosis-10}

The diagnosis of nephrotic syndrome is made clinically and supported by
laboratory findings:

\textbf{Diagnostic criteria:}

\begin{enumerate}
\def\labelenumi{\arabic{enumi}.}
\tightlist
\item
  Proteinuria \textgreater3+ on dipstick
\item
  Serum albumin \textless25 g/L
\item
  Oedema
\item
  Hyperlipidemia
\end{enumerate}

The child is classified based on steroid responsiveness as:

\begin{itemize}
\tightlist
\item
  \textbf{Steroid-sensitive nephrotic syndrome (SSNS)}
\item
  \textbf{Steroid-resistant nephrotic syndrome (SRNS)}
\item
  \textbf{Frequent relapser} or \textbf{steroid-dependent}
\end{itemize}

\section{Management}\label{management-16}

The approach to treatment depends on the underlying cause and the
child's response to steroids.

\subsection{Emergency and Supportive
Care}\label{emergency-and-supportive-care}

\textbf{Hospital admission} is warranted for the first presentation or
severe relapse with anasarca.

\textbf{Fluid and salt management}

\begin{itemize}
\tightlist
\item
  Restrict sodium intake.
\item
  Maintain appropriate fluid balance, usually restricted to insensible
  loss plus urine output.
\item
  Daily weight and urine monitoring.
\end{itemize}

\textbf{Management of hypovolemia}

\begin{itemize}
\tightlist
\item
  Suspect if tachycardia, abdominal pain, or cold extremities occur.
\item
  Give \textbf{10--20 mL/kg of 4.5\% albumin} or normal saline
  cautiously, followed by furosemide.
\end{itemize}

\textbf{Infection prevention}

\begin{itemize}
\tightlist
\item
  High risk for peritonitis (commonly \emph{Streptococcus pneumoniae}),
  cellulitis, and sepsis.
\item
  Start \textbf{broad-spectrum antibiotics} if infection suspected.
\item
  Pneumococcal and varicella vaccination recommended.
\end{itemize}

\textbf{Nutritional support}

\begin{itemize}
\tightlist
\item
  Adequate calories and protein (1--2 g/kg/day).
\item
  Avoid high-fat diets to prevent exacerbation of hyperlipidemia.
\end{itemize}

\subsection{Specific Treatment}\label{specific-treatment}

\subsubsection{Corticosteroid Therapy}\label{corticosteroid-therapy}

Prednisolone remains the cornerstone of therapy for idiopathic nephrotic
syndrome.

\textbf{Initial episode (ISPN guidelines):}

\begin{itemize}
\tightlist
\item
  \textbf{Prednisolone 60 mg/m²/day} (max 60 mg) for \textbf{4 weeks},
  followed by
\item
  \textbf{40 mg/m² on alternate days} for another \textbf{4 weeks}, then
  taper gradually.
\end{itemize}

\textbf{Response monitoring:}

\begin{itemize}
\tightlist
\item
  Daily urine dipstick for protein.
\item
  \textbf{Complete remission:} 3 consecutive days of negative or trace
  proteinuria.
\end{itemize}

\subsubsection{Relapse}\label{relapse}

Reappearance of proteinuria (≥3+) for ≥3 days after remission. Treat
with \textbf{prednisolone 60 mg/m²/day} until remission, then taper.

\subsubsection{Steroid-Resistant Nephrotic Syndrome
(SRNS)}\label{steroid-resistant-nephrotic-syndrome-srns}

No remission after 4 weeks of adequate steroid therapy. - Evaluate for
\textbf{secondary causes} or \textbf{FSGS} (via biopsy).

\begin{itemize}
\tightlist
\item
  Consider \textbf{calcineurin inhibitors} (cyclosporine or tacrolimus),
  \textbf{mycophenolate mofetil}, or \textbf{cyclophosphamide}.
\item
  Manage under paediatric nephrology care.
\end{itemize}

\subsection{Management of
Complications}\label{management-of-complications-1}

\begin{longtable}[]{@{}
  >{\raggedright\arraybackslash}p{(\linewidth - 2\tabcolsep) * \real{0.3820}}
  >{\raggedright\arraybackslash}p{(\linewidth - 2\tabcolsep) * \real{0.6180}}@{}}
\toprule\noalign{}
\begin{minipage}[b]{\linewidth}\raggedright
\textbf{Complication}
\end{minipage} & \begin{minipage}[b]{\linewidth}\raggedright
\textbf{Management}
\end{minipage} \\
\midrule\noalign{}
\endhead
\bottomrule\noalign{}
\endlastfoot
\textbf{Infection} & Prompt antibiotic therapy; pneumococcal
prophylaxis \\
\textbf{Hypovolemia} & Albumin infusion + diuretics \\
\textbf{Thrombosis (DVT, renal vein)} & Anticoagulation (heparin →
warfarin) \\
\textbf{Acute renal failure} & Supportive care, treat underlying
cause \\
\textbf{Dyslipidemia} & Dietary modification; statins if persistent \\
\textbf{Hypertension} & ACE inhibitors (enalapril) beneficial for
proteinuria \\
\end{longtable}

\subsection{Preparation for
Discharge}\label{preparation-for-discharge-4}

Before discharge: - Ensure oedema resolution and stable renal function.

\begin{itemize}
\tightlist
\item
  Teach caregivers how to \textbf{check urine protein at home}.
\item
  Educate on \textbf{signs of relapse} and infection prevention.
\item
  Arrange follow-up schedule (weekly initially, then monthly).
\item
  Encourage vaccination where indicated.
\end{itemize}

\subsection{Long-Term Management and
Follow-Up}\label{long-term-management-and-follow-up}

\begin{itemize}
\tightlist
\item
  \textbf{Monitor relapses:} Up to 70\% of idiopathic cases relapse
  within 6 months.
\item
  \textbf{Minimise steroid toxicity}: Screen for growth retardation,
  obesity, cataracts, and hypertension.
\item
  \textbf{Address psychosocial impact:} School attendance and family
  anxiety.
\item
  \textbf{Nephrology referral:} For steroid dependence or resistance.
\end{itemize}

\section{Complications}\label{complications-23}

\textbf{Acute}

\begin{itemize}
\tightlist
\item
  Infection (spontaneous bacterial peritonitis, cellulitis)
\item
  Hypovolemia and shock
\item
  Thrombosis (renal vein, cerebral venous sinus)
\item
  Acute renal failure
\end{itemize}

\textbf{Chronic}

\begin{itemize}
\tightlist
\item
  Persistent proteinuria leading to chronic kidney disease
\item
  Growth retardation and delayed puberty
\item
  Steroid toxicity (hypertension, osteoporosis, Cushingoid features)
\end{itemize}

\section{Prevention}\label{prevention-14}

While idiopathic cases cannot be prevented, complications can be
minimized by: - Early diagnosis and appropriate steroid therapy

\begin{itemize}
\tightlist
\item
  Routine immunization, especially \textbf{pneumococcal and varicella
  vaccines}
\item
  Avoiding nephrotoxic drugs (e.g., NSAIDs)
\item
  Educating families on prompt infection treatment
\item
  Regular follow-up at renal clinics
\end{itemize}

\section{Prognosis}\label{prognosis-24}

\begin{itemize}
\tightlist
\item
  \textbf{Minimal Change Disease:} Excellent prognosis; over 90\%
  achieve remission with steroids.
\item
  \textbf{FSGS or secondary causes:} Higher risk of chronic renal
  failure.
\item
  \textbf{Steroid-dependent or frequent relapsers:} Often require
  second-line therapy but may maintain long-term renal function.
\end{itemize}

Relapses often decrease with age, and most children achieve permanent
remission by adolescence.

\section{Conclusion}\label{conclusion-24}

Nephrotic syndrome remains a significant paediatric renal disorder in
Ghana, accounting for substantial hospital admissions and morbidity. The
majority of cases respond well to corticosteroids, though
steroid-resistant forms, particularly FSGS, are increasingly recognized.
A sound understanding of its clinical features, complications, and
management principles is essential for medical students and
practitioners.\\
Timely diagnosis, infection control, family education, and close
follow-up remain the pillars of good outcomes in paediatric nephrotic
syndrome.

\chapter{Nephritic Syndrome}\label{nephritic-syndrome}

\section{Introduction}\label{introduction-33}

Nephritic syndrome is a clinical syndrome resulting from inflammation of
the glomeruli, leading to impaired renal filtration. It is characterised
by \textbf{haematuria, mild-to-moderate proteinuria, oedema,
hypertension, and varying degrees of renal impairment}. Unlike nephrotic
syndrome, in which protein loss predominates, nephritic syndrome
reflects glomerular injury due to immune-mediated inflammation, leading
to red cell leakage and reduced glomerular filtration.

In Ghana, as in many developing countries, \textbf{post-streptococcal
glomerulonephritis (PSGN)} remains the most common cause in children.
However, other glomerulonephritides such as lupus nephritis and IgA
nephropathy are also encountered.

Understanding nephritic syndrome is important because timely diagnosis
and appropriate management can prevent progression to chronic kidney
disease.

\section{Epidemiology}\label{epidemiology-4}

Nephritic syndrome can occur at any age but is most common in
\textbf{school-aged children between 5 and 12 years}.

\begin{itemize}
\tightlist
\item
  \textbf{Sex:} Slight male predominance (M:F ≈ 2:1).
\item
  \textbf{Geography:} Higher prevalence in areas with poor sanitation,
  overcrowding, and high incidence of streptococcal skin or throat
  infections.
\item
  \textbf{Seasonal variation:} Cases often peak following outbreaks of
  streptococcal infections, especially during the dry or cold season.
\end{itemize}

In Ghanaian children, \textbf{acute post-streptococcal
glomerulonephritis (APSGN)} accounts for the majority of nephritic
presentations.

\section{Aetiology and
Classification}\label{aetiology-and-classification-1}

Nephritic syndrome can be classified according to \textbf{clinical
course} (acute, rapidly progressive, or chronic) or \textbf{underlying
aetiology} (primary renal vs secondary systemic causes).

\subsection{Primary Glomerular Causes}\label{primary-glomerular-causes}

\begin{itemize}
\tightlist
\item
  \textbf{Acute post-streptococcal glomerulonephritis (APSGN):}\\
  The classic cause in children, occurring 1--3 weeks after group A
  β-haemolytic streptococcal pharyngitis or skin infection (impetigo).
\item
  \textbf{IgA nephropathy (Berger's disease):}\\
  Characterised by recurrent haematuria, often following upper
  respiratory tract infection.
\item
  \textbf{Membranoproliferative glomerulonephritis (MPGN):}\\
  Chronic immune complex--mediated inflammation causing persistent
  haematuria and proteinuria.
\item
  \textbf{Rapidly progressive glomerulonephritis (RPGN):}\\
  Severe form with crescent formation in glomeruli and rapid loss of
  renal function.
\end{itemize}

\subsection{2. Secondary Glomerular
Causes}\label{secondary-glomerular-causes}

\begin{itemize}
\tightlist
\item
  \textbf{Systemic lupus erythematosus (SLE):} Immune complex deposition
  in glomeruli.
\item
  \textbf{Henoch-Schönlein purpura (HSP):} IgA-mediated vasculitis
  involving the kidneys.
\item
  \textbf{Infections:} Hepatitis B/C, malaria, HIV.
\item
  \textbf{Endocarditis or shunt nephritis:} Chronic infection leading to
  immune complex glomerulonephritis.
\end{itemize}

\section{Pathophysiology}\label{pathophysiology-22}

Nephritic syndrome arises from \textbf{inflammation and proliferation
within the glomeruli}, usually mediated by \textbf{immune-complex}
deposition or \textbf{autoantibodies}.

The general sequence is as follows:

\begin{enumerate}
\def\labelenumi{\arabic{enumi}.}
\tightlist
\item
  \textbf{Immune complex formation} (e.g., streptococcal antigens with
  antibodies).
\item
  \textbf{Deposition in glomerular capillaries} and activation of the
  \textbf{complement system}.
\item
  \textbf{Inflammatory response} → neutrophil and macrophage
  infiltration.
\item
  \textbf{Glomerular injury} → capillary wall thickening, cellular
  proliferation, and reduced surface area for filtration.
\item
  \textbf{Reduced glomerular filtration rate (GFR)} → salt and water
  retention → oedema and hypertension.
\item
  \textbf{Leakage of red cells and some protein} into urine → haematuria
  and mild proteinuria.
\end{enumerate}

\section{Clinical Features}\label{clinical-features-20}

\subsection{History}\label{history-4}

The onset is usually acute, developing within days to weeks following a
\textbf{streptococcal throat or skin infection}.

\textbf{Key presenting symptoms:}

\begin{itemize}
\tightlist
\item
  \textbf{Haematuria:} Brown, smoky, or cola-coloured urine.
\item
  \textbf{Oliguria:} Decreased urine output due to reduced GFR.
\item
  \textbf{Facial puffiness:} Particularly periorbital oedema, worse in
  the morning.
\item
  \textbf{Mild generalized oedema.}
\item
  \textbf{Headache or visual disturbances:} Due to hypertension.
\item
  \textbf{History of sore throat or impetigo} 1--3 weeks prior to
  illness.
\end{itemize}

\subsection{Physical Examination}\label{physical-examination-2}

\begin{itemize}
\tightlist
\item
  \textbf{Oedema:} Usually mild, periorbital, and pedal.
\item
  \textbf{Blood pressure:} Elevated in most cases (may be severe).
\item
  \textbf{Urine colour:} Dark, tea-coloured urine.
\item
  \textbf{Signs of volume overload:} Raised jugular venous pressure,
  basal crepitations.
\item
  \textbf{Other findings:} Pallor (anaemia), mild hepatomegaly.
\end{itemize}

\section{Differential Diagnosis}\label{differential-diagnosis-21}

\begin{longtable}[]{@{}
  >{\raggedright\arraybackslash}p{(\linewidth - 2\tabcolsep) * \real{0.5000}}
  >{\raggedright\arraybackslash}p{(\linewidth - 2\tabcolsep) * \real{0.5000}}@{}}
\toprule\noalign{}
\begin{minipage}[b]{\linewidth}\raggedright
\textbf{Condition}
\end{minipage} & \begin{minipage}[b]{\linewidth}\raggedright
\textbf{Distinguishing Features}
\end{minipage} \\
\midrule\noalign{}
\endhead
\bottomrule\noalign{}
\endlastfoot
Nephrotic syndrome & Marked oedema, massive proteinuria, normal
complement \\
Haemolytic uraemic syndrome & Triad of anaemia, thrombocytopenia, and
renal failure following diarrhoea \\
IgA nephropathy & Recurrent macroscopic haematuria after respiratory
infection \\
SLE nephritis & Photosensitive rash, arthritis, positive ANA \\
Acute interstitial nephritis & Drug exposure, eosinophiluria \\
\end{longtable}

\section{Investigations}\label{investigations-25}

\subsection{Urine Tests}\label{urine-tests}

\begin{itemize}
\tightlist
\item
  \textbf{Urinalysis:}

  \begin{itemize}
  \tightlist
  \item
    \textbf{Haematuria:} Dysmorphic RBCs and red cell casts are
    diagnostic.
  \item
    \textbf{Proteinuria:} Usually mild to moderate (1--2+).
  \item
    \textbf{Specific gravity:} Often raised due to oliguria.
  \end{itemize}
\item
  \textbf{Microscopy:} RBC casts, WBCs.
\end{itemize}

\subsection{Blood Tests}\label{blood-tests-1}

\begin{itemize}
\tightlist
\item
  \textbf{Renal function:} Elevated urea and creatinine indicate
  impaired filtration.
\item
  \textbf{Complement levels (C3):} Low in post-streptococcal GN, normal
  in IgA nephropathy.
\item
  \textbf{ASO titre / anti-DNase B:} Elevated after streptococcal
  infection.
\item
  \textbf{Serum electrolytes:} May show hyperkalaemia or hyponatraemia.
\item
  \textbf{Full blood count:} Mild anaemia, leukocytosis possible.
\item
  \textbf{Antinuclear antibody (ANA):} For suspected lupus nephritis.
\end{itemize}

\subsection{Imaging}\label{imaging-2}

\begin{itemize}
\tightlist
\item
  \textbf{Renal ultrasound:} Normal or slightly enlarged kidneys with
  increased echogenicity.
\item
  \textbf{Chest X-ray:} Pulmonary oedema or cardiomegaly from volume
  overload.
\end{itemize}

\subsection{Kidney Biopsy}\label{kidney-biopsy-1}

Indications include:

\begin{itemize}
\tightlist
\item
  Atypical course (no recovery after 2--3 weeks)
\item
  Persistent renal dysfunction
\item
  Gross proteinuria (\textgreater3 g/day)
\item
  Absence of low complement
\item
  Suspected lupus nephritis or RPGN
\end{itemize}

\section{Diagnosis}\label{diagnosis-11}

Diagnosis is based on the presence of the following:

\begin{enumerate}
\def\labelenumi{\arabic{enumi}.}
\tightlist
\item
  \textbf{Haematuria} (microscopic or macroscopic)
\item
  \textbf{Oliguria} with elevated urea and creatinine
\item
  \textbf{Mild-to-moderate proteinuria}
\item
  \textbf{Hypertension}
\item
  \textbf{History of preceding infection}
\end{enumerate}

\section{Management}\label{management-17}

Treatment of nephritic syndrome is mainly \textbf{supportive}, aimed at
controlling hypertension, oedema, and preventing complications. The
underlying cause guides specific therapy.

\subsection{General Measures}\label{general-measures-1}

\begin{itemize}
\tightlist
\item
  \textbf{Hospital admission} for monitoring of blood pressure, urine
  output, and renal function.
\item
  \textbf{Bed rest} during the acute phase to reduce workload on
  kidneys.
\item
  \textbf{Fluid restriction} to match output + insensible loss (usually
  400--600 mL/m²/day).
\item
  \textbf{Sodium restriction} to prevent oedema and hypertension.
\end{itemize}

\subsection{Control of Oedema}\label{control-of-oedema}

\begin{itemize}
\tightlist
\item
  \textbf{Loop diuretics (furosemide 1--2 mg/kg)} if the child is not
  oliguric.
\item
  Avoid excessive diuresis in oliguric states to prevent hypovolemia.
\item
  \textbf{Fluid overload} with pulmonary oedema may require
  \textbf{frusemide + antihypertensive therapy} or dialysis.
\end{itemize}

\subsection{Hypertension Management}\label{hypertension-management}

Hypertension in nephritic syndrome is mainly volume-dependent.

\begin{itemize}
\item
  \textbf{First-line:} Loop diuretics.
\item
  \textbf{If persistent:} Add \textbf{nifedipine} or
  \textbf{hydralazine}.
\item
  \textbf{Severe hypertension or hypertensive encephalopathy:} Use IV
  antihypertensives cautiously (e.g., labetalol infusion).
\end{itemize}

\subsection{Infection Management}\label{infection-management}

\begin{itemize}
\tightlist
\item
  If there is an ongoing streptococcal infection, give
  \textbf{benzathine penicillin} or \textbf{amoxicillin}.
\item
  \textbf{Antibiotic prophylaxis} is not usually required once the
  infection has cleared.
\end{itemize}

\subsection{Dietary Advice}\label{dietary-advice}

\begin{itemize}
\tightlist
\item
  \textbf{Salt restriction} until oedema and hypertension resolve.
\item
  \textbf{Protein intake:} Normal for age (avoid restriction unless
  renal failure develops).
\item
  \textbf{Potassium restriction} if hyperkalaemia occurs.
\end{itemize}

\subsection{Dialysis Indications}\label{dialysis-indications}

Dialysis may be required if there is: - Refractory fluid overload.

\begin{itemize}
\tightlist
\item
  Severe hyperkalaemia
\item
  Rising urea/creatinine
\item
  Uremic encephalopathy or seizures
\end{itemize}

\section{Specific Therapy}\label{specific-therapy}

\subsection{Post-Streptococcal
Glomerulonephritis}\label{post-streptococcal-glomerulonephritis}

\begin{itemize}
\tightlist
\item
  Mainly supportive care.
\item
  Prognosis excellent; most recover fully within 6--8 weeks.
\item
  Persistent proteinuria or haematuria may last months.
\end{itemize}

\subsection{IgA Nephropathy and MPGN}\label{iga-nephropathy-and-mpgn}

\begin{itemize}
\tightlist
\item
  Often chronic; may need \textbf{ACE inhibitors} or
  \textbf{immunosuppressants} (prednisolone, azathioprine).
\item
  Regular follow-up of renal function is essential.
\end{itemize}

\subsection{Lupus Nephritis}\label{lupus-nephritis}

\begin{itemize}
\tightlist
\item
  Requires systemic corticosteroids and immunosuppressive therapy
  (mycophenolate mofetil or cyclophosphamide).
\item
  Managed in collaboration with a paediatric nephrologist.
\end{itemize}

\subsection{Rapidly Progressive
Glomerulonephritis}\label{rapidly-progressive-glomerulonephritis}

\begin{itemize}
\tightlist
\item
  Aggressive management with high-dose corticosteroids,
  cyclophosphamide, and sometimes plasma exchange.
\item
  Early treatment may prevent irreversible renal failure.
\end{itemize}

\section{Complications}\label{complications-24}

\begin{itemize}
\tightlist
\item
  \textbf{Hypertensive encephalopathy:} Seizures, vomiting, blurred
  vision.
\item
  \textbf{Acute renal failure:} Due to severe glomerular inflammation.
\item
  \textbf{Fluid overload:} Pulmonary oedema or heart failure.
\item
  \textbf{Electrolyte imbalances:} Hyperkalaemia, hyponatraemia.
\item
  \textbf{Chronic kidney disease:} From unresolved inflammation.
\end{itemize}

\section{Prognosis}\label{prognosis-25}

Most children with \textbf{post-streptococcal nephritic syndrome}
recover completely.

\begin{itemize}
\tightlist
\item
  \textbf{Microscopic haematuria} may persist for up to 6--12 months.
\item
  \textbf{Renal function} returns to normal in \textgreater95\% of
  cases.
\item
  \textbf{Poor prognostic factors:} Persistent hypertension, heavy
  proteinuria, reduced complement for \textgreater8 weeks, and
  histologic evidence of crescentic GN.
\end{itemize}

Chronic or secondary causes (e.g., lupus, MPGN) may progress to
\textbf{end-stage kidney disease} if not adequately managed.

\section{Prevention}\label{prevention-15}

\begin{itemize}
\tightlist
\item
  Early diagnosis and treatment of streptococcal infections
  (pharyngitis, impetigo).
\item
  Improved sanitation and hygiene to reduce transmission.
\item
  Community health education to promote prompt medical attention for
  children with swelling or dark urine.
\item
  Regular follow-up for children with known glomerular diseases.
\end{itemize}

\section{Conclusion}\label{conclusion-25}

Nephritic syndrome in children, particularly post-streptococcal
glomerulonephritis, remains a significant cause of acute kidney disease
in Ghana. It typically follows streptococcal infection and manifests
with haematuria, oedema, and hypertension. While most cases resolve
spontaneously with supportive care, early recognition and careful
monitoring are essential to prevent complications and chronic kidney
damage. A solid grasp of its pathophysiology and management principles
is crucial for medical students and paediatric practitioners, ensuring
timely and appropriate care for affected children.

\chapter{Kidney Failure}\label{kidney-failure}

Kidney failure in children represents a significant cause of morbidity
and mortality, especially in low- and middle-income countries such as
Ghana. It may occur acutely, following a transient insult to the
kidneys, or chronically, as the final stage of progressive renal
disease. Understanding the causes, pathophysiology, clinical features,
investigations, and management of kidney failure is critical for all
medical students and healthcare providers involved in child health.

\section{Introduction}\label{introduction-34}

Kidney failure refers to the inability of the kidneys to perform their
normal regulatory, excretory, and endocrine functions. In children, it
can present as \textbf{acute kidney injury (AKI)} or \textbf{chronic
kidney disease (CKD)}. AKI involves a rapid decline in renal function
over hours or days, while CKD involves irreversible deterioration over
months or years. The distinction is important because their causes,
management, and outcomes differ.

In sub-Saharan Africa, including Ghana, the prevalence of kidney failure
is underestimated due to limited diagnostic resources. However,
hospital-based data show that AKI is a common complication of severe
infections, dehydration, and nephrotoxic exposure in children, while CKD
often results from congenital anomalies or glomerular diseases.

\section{Classification}\label{classification-3}

Kidney failure in children can be classified as:

\begin{itemize}
\tightlist
\item
  \textbf{Acute Kidney Injury (AKI):}

  \begin{itemize}
  \tightlist
  \item
    Rapid onset (hours to days).
  \item
    Often reversible if recognized and treated promptly.
  \item
    Common causes: dehydration, sepsis, nephrotoxic drugs, malaria, and
    hemolytic uremic syndrome.
  \end{itemize}
\item
  \textbf{Chronic Kidney Disease (CKD):}

  \begin{itemize}
  \tightlist
  \item
    Gradual and irreversible loss of kidney function lasting over three
    months.
  \item
    Often associated with congenital anomalies, reflux nephropathy, or
    glomerular diseases.
  \item
    May progress to end-stage renal disease (ESRD), requiring dialysis
    or transplantation.
  \end{itemize}
\end{itemize}

\section{Epidemiology}\label{epidemiology-5}

Global data show that kidney failure accounts for approximately 1--3\%
of paediatric hospital admissions, though regional variations exist. In
Ghana and similar settings, AKI is more frequent than CKD, with
mortality rates reaching up to 30\% due to late presentation and limited
access to dialysis. CKD is less common but often underdiagnosed until
advanced stages.

\section{Aetiology}\label{aetiology-15}

\subsection{\texorpdfstring{\textbf{Acute Kidney
Injury}}{Acute Kidney Injury}}\label{acute-kidney-injury}

\begin{enumerate}
\def\labelenumi{\arabic{enumi}.}
\tightlist
\item
  \textbf{Pre-renal causes (impaired perfusion)}

  \begin{itemize}
  \tightlist
  \item
    Severe dehydration (e.g., diarrhoeal diseases)
  \item
    Septic shock
  \item
    Congestive heart failure
  \item
    Hemorrhage
  \end{itemize}
\item
  \textbf{Intrinsic renal causes}

  \begin{itemize}
  \tightlist
  \item
    Acute glomerulonephritis
  \item
    Hemolytic uremic syndrome
  \item
    Acute tubular necrosis
  \item
    Nephrotoxins (e.g., aminoglycosides, NSAIDs, herbal medicines)
  \item
    Malaria (especially falciparum infection)
  \end{itemize}
\item
  \textbf{Post-renal causes (obstruction)}

  \begin{itemize}
  \tightlist
  \item
    Posterior urethral valves
  \item
    Kidney stones
  \item
    Tumours compressing urinary outflow
  \end{itemize}
\end{enumerate}

\subsection{\texorpdfstring{\textbf{Chronic Kidney
Disease}}{Chronic Kidney Disease}}\label{chronic-kidney-disease}

\begin{enumerate}
\def\labelenumi{\arabic{enumi}.}
\tightlist
\item
  \textbf{Congenital anomalies of the kidney and urinary tract (CAKUT)}

  \begin{itemize}
  \tightlist
  \item
    Renal dysplasia
  \item
    Obstructive uropathy
  \item
    Vesicoureteral reflux
  \end{itemize}
\item
  \textbf{Glomerular diseases}

  \begin{itemize}
  \tightlist
  \item
    Nephrotic and nephritic syndromes
  \item
    Focal segmental glomerulosclerosis
  \item
    Chronic glomerulonephritis
  \end{itemize}
\item
  \textbf{Inherited and metabolic diseases}

  \begin{itemize}
  \tightlist
  \item
    Polycystic kidney disease
  \item
    Cystinosis
  \item
    Alport syndrome
  \end{itemize}
\item
  \textbf{Systemic diseases}

  \begin{itemize}
  \tightlist
  \item
    Lupus nephritis
  \item
    Sickle cell nephropathy
  \end{itemize}
\end{enumerate}

\section{Pathophysiology}\label{pathophysiology-23}

The kidneys maintain homeostasis through filtration, tubular
reabsorption, secretion, and endocrine regulation. In kidney failure,
these functions are impaired, leading to:

\begin{itemize}
\tightlist
\item
  \textbf{Accumulation of waste products:} Elevated blood urea nitrogen
  (BUN) and creatinine.
\item
  \textbf{Fluid imbalance:} Volume overload, edema, and hypertension.
\item
  \textbf{Electrolyte abnormalities:} Hyperkalemia, hyponatremia, and
  metabolic acidosis.
\item
  \textbf{Endocrine dysfunction:} Reduced erythropoietin production
  causes anemia; decreased vitamin D activation leads to hypocalcemia
  and bone disease.
\end{itemize}

In CKD, progressive nephron loss leads to compensatory hyperfiltration
in the remaining nephrons, which accelerates further damage, forming a
vicious cycle that leads to end-stage disease.

\section{Clinical Features}\label{clinical-features-21}

\subsection{\texorpdfstring{\textbf{Acute Kidney
Injury}}{Acute Kidney Injury}}\label{acute-kidney-injury-1}

\begin{itemize}
\tightlist
\item
  Oliguria or anuria
\item
  Peripheral or generalised oedema
\item
  Vomiting, lethargy, and poor feeding
\item
  Hypertension
\item
  Pallor (from anemia)
\item
  Seizures due to uremic encephalopathy or electrolyte imbalance
\end{itemize}

\subsection{\texorpdfstring{\textbf{Chronic Kidney
Disease}}{Chronic Kidney Disease}}\label{chronic-kidney-disease-1}

\begin{itemize}
\tightlist
\item
  Growth retardation
\item
  Persistent pallor and fatigue
\item
  Polyuria and nocturia
\item
  Bone deformities (renal osteodystrophy)
\item
  Pruritus, anorexia, nausea, or vomiting
\item
  Late-stage uremic symptoms: confusion, pericarditis, and bleeding
  tendency
\end{itemize}

\section{Investigations}\label{investigations-26}

\begin{enumerate}
\def\labelenumi{\arabic{enumi}.}
\tightlist
\item
  \textbf{Basic Laboratory Tests}

  \begin{itemize}
  \tightlist
  \item
    Serum creatinine, urea, electrolytes (Na⁺, K⁺, Cl⁻, HCO₃⁻)
  \item
    Urinalysis (proteinuria, hematuria, specific gravity)
  \item
    Urine output monitoring
  \item
    Full blood count (for anemia and infection)
  \end{itemize}
\item
  \textbf{Special Tests}

  \begin{itemize}
  \tightlist
  \item
    Renal ultrasound: kidney size, echogenicity, and obstruction
  \item
    Renal biopsy (for glomerular diseases)
  \item
    24-hour urine protein estimation
  \item
    Serum calcium, phosphate, alkaline phosphatase, and parathyroid
    hormone in CKD
  \end{itemize}
\item
  \textbf{Imaging}

  \begin{itemize}
  \tightlist
  \item
    Voiding cystourethrogram (VCUG) for reflux nephropathy
  \item
    DMSA or MAG3 scans for renal scarring or differential function
  \end{itemize}
\end{enumerate}

\section{Management}\label{management-18}

\subsection{\texorpdfstring{\textbf{Emergency Management
(AKI)}}{Emergency Management (AKI)}}\label{emergency-management-aki}

\begin{itemize}
\tightlist
\item
  \textbf{Stabilization:}

  \begin{itemize}
  \tightlist
  \item
    Secure airway, breathing, and circulation.
  \item
    Correct fluid deficits cautiously to avoid overload.
  \item
    Treat underlying causes: antibiotics for sepsis, antimalarials, or
    stop nephrotoxic drugs.
  \end{itemize}
\item
  \textbf{Monitor urine output} with a catheter.
\item
  \textbf{Manage complications:}

  \begin{itemize}
  \tightlist
  \item
    Hyperkalemia: calcium gluconate, insulin with glucose, and sodium
    bicarbonate.
  \item
    Hypertension: diuretics, antihypertensives.
  \item
    Dialysis if unresponsive or in severe uremia.
  \end{itemize}
\end{itemize}

\subsection{\texorpdfstring{\textbf{Ongoing Management (CKD and AKI
Recovery)}}{Ongoing Management (CKD and AKI Recovery)}}\label{ongoing-management-ckd-and-aki-recovery}

\begin{itemize}
\tightlist
\item
  Maintain fluid and electrolyte balance.
\item
  Nutritional support, adequate calories and protein.
\item
  Control hypertension using ACE inhibitors or ARBs.
\item
  Manage anaemia with erythropoietin and iron supplements.
\item
  Prevent bone disease with phosphate binders and vitamin D analogues.
\end{itemize}

\subsection{\texorpdfstring{\textbf{Renal Replacement
Therapy}}{Renal Replacement Therapy}}\label{renal-replacement-therapy}

Indications include:

\begin{itemize}
\tightlist
\item
  Persistent hyperkalemia
\item
  Severe metabolic acidosis
\item
  Fluid overload unresponsive to diuretics
\item
  Uremic complications (encephalopathy, pericarditis) Modalities:
\item
  Peritoneal dialysis (preferred in infants and young children)
\item
  Hemodialysis - Kidney transplantation (definitive for ESRD)
\end{itemize}

\subsection{\texorpdfstring{\textbf{Preparation for
Discharge}}{Preparation for Discharge}}\label{preparation-for-discharge-5}

\begin{itemize}
\tightlist
\item
  Educate caregivers on dietary and medication adherence.
\item
  Schedule follow-up with a paediatric nephrologist.
\item
  Screen for underlying or recurrent causes.
\end{itemize}

\subsection{\texorpdfstring{\textbf{Long-Term
Management}}{Long-Term Management}}\label{long-term-management-4}

\begin{itemize}
\tightlist
\item
  Regular monitoring of growth, blood pressure, and kidney function.
\item
  Early management of infections.
\item
  Psychosocial support for the child and family.
\item
  Planning for eventual renal transplantation.
\end{itemize}

\section{Complications}\label{complications-25}

\begin{itemize}
\tightlist
\item
  Electrolyte disturbances (hyperkalemia, acidosis)
\item
  Hypertension
\item
  Chronic anemia
\item
  Growth failure
\item
  Bone deformities
\item
  Cardiovascular disease (secondary to uremia)
\item
  Infections due to immunosuppression
\item
  Progression to ESRD
\end{itemize}

\section{Prevention}\label{prevention-16}

\begin{itemize}
\tightlist
\item
  Prompt treatment of infections such as malaria and urinary tract
  infections.
\item
  Avoidance of nephrotoxic drugs.
\item
  Good hydration during illness.
\item
  Early detection and referral of congenital renal anomalies.
\item
  Regular follow-up for children with known renal disease.
\end{itemize}

\section{Prognosis}\label{prognosis-26}

The outcome depends on the underlying cause, promptness of treatment,
and availability of renal replacement therapy.

\begin{itemize}
\item
  \textbf{AKI} has a good prognosis if managed early, though severe
  cases may progress to CKD.
\item
  \textbf{CKD} often progresses to ESRD without timely intervention.\\
  Access to dialysis and transplantation markedly improves survival and
  quality of life.
\end{itemize}

\section{Conclusion}\label{conclusion-26}

Kidney failure in children is a major clinical problem requiring early
recognition and multidisciplinary management. In resource-limited
settings like Ghana, prevention through early diagnosis and community
awareness remains the most effective strategy. Improving access to
dialysis and transplant facilities is essential for long-term survival
and better quality of life for affected children.

\chapter{Obstructive Uropathy}\label{obstructive-uropathy}

\section{Introduction}\label{introduction-35}

Obstructive uropathy refers to any structural or functional impediment
to the normal flow of urine along the urinary tract, leading to an
increase in intraluminal pressure and, ultimately, damage to the
kidneys. The obstruction may occur at any point from the renal pelvis to
the urethral meatus and may be \textbf{acute or chronic},
\textbf{partial or complete}, \textbf{unilateral or bilateral}.

In children, the condition is of particular concern because prolonged
obstruction, even if partial, can lead to \textbf{irreversible renal
parenchymal damage}, which has lifelong implications for growth,
development, and renal function. Early recognition and prompt
intervention are therefore essential.

In Ghana and other parts of sub-Saharan Africa, obstructive uropathy in
children is not uncommon. The causes vary with age, ranging from
\textbf{posterior urethral valves} in neonates and infants to
\textbf{ureteropelvic junction obstruction} and acquired causes such as
\textbf{stones} or \textbf{iatrogenic injury} in older children.

\section{Basic Anatomy and
Physiology}\label{basic-anatomy-and-physiology}

The urinary tract consists of:

\begin{itemize}
\tightlist
\item
  \textbf{Kidneys:} Filter blood to form urine.
\item
  \textbf{Ureters:} Conduct urine from the kidneys to the bladder.
\item
  \textbf{Bladder:} Stores urine temporarily.
\item
  \textbf{Urethra:} Expels urine from the body.
\end{itemize}

Normal urine flow is maintained by a combination of:

\begin{enumerate}
\def\labelenumi{\arabic{enumi}.}
\tightlist
\item
  The peristaltic activity of the ureters,
\item
  The presence of competent valves (such as the vesicoureteric
  junction),
\item
  Coordinated bladder contraction and sphincter relaxation.
\end{enumerate}

Any lesion disrupting these mechanisms can cause obstruction and
consequent back pressure on the kidneys, resulting in
\textbf{hydronephrosis} and possible \textbf{renal impairment}.

\section{Pathophysiology}\label{pathophysiology-24}

When urine flow is obstructed, several changes occur in the urinary
tract and renal parenchyma:

\begin{enumerate}
\def\labelenumi{\arabic{enumi}.}
\tightlist
\item
  \textbf{Increased pressure proximal to the obstruction:} Leads to
  dilatation of the collecting system (hydronephrosis) and stretching of
  the renal capsule.
\item
  \textbf{Altered renal blood flow:} Initially, renal blood flow
  increases but eventually declines as interstitial pressure rises,
  leading to ischemic injury.
\item
  \textbf{Tubular dysfunction:} Impaired concentrating ability and
  reduced glomerular filtration rate (GFR).
\item
  \textbf{Inflammation and fibrosis:} Chronic obstruction leads to
  tubular atrophy, interstitial inflammation, and fibrosis, resulting in
  progressive loss of renal function.
\end{enumerate}

The reversibility of renal damage depends on the \textbf{duration and
severity} of the obstruction. Complete obstruction for more than a few
weeks can cause \textbf{irreversible renal scarring}.

\section{Classification}\label{classification-4}

\subsection{Based on Duration}\label{based-on-duration}

\begin{itemize}
\tightlist
\item
  \textbf{Acute Obstructive Uropathy:} Sudden onset (e.g., calculus
  obstruction, trauma).
\item
  \textbf{Chronic Obstructive Uropathy:} Long-standing (e.g., posterior
  urethral valves, congenital stenosis).
\end{itemize}

\subsection{Based on Site}\label{based-on-site}

\begin{itemize}
\tightlist
\item
  \textbf{Upper tract obstruction:} Affecting renal pelvis or ureter.
\item
  \textbf{Lower tract obstruction:} Involving the bladder or urethra.
\end{itemize}

\subsection{Based on Laterality}\label{based-on-laterality}

\begin{itemize}
\tightlist
\item
  \textbf{Unilateral:} One kidney affected (may preserve overall renal
  function).
\item
  \textbf{Bilateral:} Both kidneys affected (risk of renal failure).
\end{itemize}

\subsection{Based on Nature of Lesion}\label{based-on-nature-of-lesion}

\begin{itemize}
\tightlist
\item
  \textbf{Intrinsic:} Due to lesion within the urinary tract (e.g.,
  valves, stricture, stone).
\item
  \textbf{Extrinsic:} Due to external compression (e.g., tumour,
  retroperitoneal fibrosis).
\end{itemize}

\section{Aetiology}\label{aetiology-16}

The causes of obstructive uropathy in children vary by age group:

\subsection{Neonates and Infants}\label{neonates-and-infants-1}

\begin{itemize}
\tightlist
\item
  \textbf{Posterior urethral valves (PUV):} The most common cause in
  male infants.
\item
  \textbf{Ureteropelvic junction (UPJ) obstruction}
\item
  \textbf{Ureterovesical junction (UVJ) obstruction}
\item
  \textbf{Prune-belly syndrome}
\item
  \textbf{Congenital megaureter}
\end{itemize}

\subsection{Older Children and
Adolescents}\label{older-children-and-adolescents}

\begin{itemize}
\tightlist
\item
  \textbf{Urolithiasis}
\item
  \textbf{Urethral stricture (post-infective or traumatic)}
\item
  \textbf{Pelvi-ureteric junction obstruction (if not detected earlier)}
\item
  \textbf{External compression (tumours, retroperitoneal fibrosis)}
\item
  \textbf{Neurogenic bladder dysfunction}
\end{itemize}

\subsection{Acquired Causes (relevant in Ghana and sub-Saharan
Africa)}\label{acquired-causes-relevant-in-ghana-and-sub-saharan-africa}

\begin{itemize}
\tightlist
\item
  \textbf{Schistosomiasis (S. haematobium):} Chronic infection can lead
  to ureteric fibrosis and obstruction.
\item
  \textbf{Tuberculosis:} Genitourinary TB may cause ureteric strictures.
\item
  \textbf{Trauma:} Pelvic trauma from road traffic accidents or
  traditional circumcision mishaps.
\item
  \textbf{Iatrogenic injuries:} Following urethral catheterisation or
  pelvic surgery.
\end{itemize}

\section{Epidemiology and Local
Context}\label{epidemiology-and-local-context}

In Ghana, the exact incidence of obstructive uropathy in children is not
well documented due to limited national registry data. However, tertiary
centres such as \textbf{Komfo Anokye Teaching Hospital} and
\textbf{Korle Bu Teaching Hospital} frequently report cases, often
presenting late with significant renal compromise.

Key local factors contributing to delayed diagnosis include:

\begin{itemize}
\tightlist
\item
  Limited access to antenatal ultrasound screening,
\item
  Late referral from peripheral hospitals,
\item
  Reliance on traditional herbal treatments before hospital
  presentation,
\item
  Scarcity of paediatric urologists and specialized imaging facilities.
\end{itemize}

Commonly observed patterns:

\begin{itemize}
\tightlist
\item
  \textbf{Posterior urethral valves} remain the most frequent congenital
  cause in male infants.
\item
  \textbf{Schistosomiasis-related strictures} and \textbf{ureteric
  calculi} are notable acquired causes in endemic rural areas,
  especially along the Volta Basin and northern Ghana.
\end{itemize}

\section{Clinical Features}\label{clinical-features-22}

The presentation of obstructive uropathy depends on:

\begin{enumerate}
\def\labelenumi{\arabic{enumi}.}
\tightlist
\item
  The site and degree of obstruction,
\item
  The duration (acute vs.~chronic),
\item
  The age of the child.
\end{enumerate}

\subsection{Neonates and Infants}\label{neonates-and-infants-2}

\begin{itemize}
\tightlist
\item
  \textbf{Antenatal hydronephrosis:} Detected on prenatal ultrasound.
\item
  \textbf{Poor urine stream} or \textbf{dribbling of urine} (especially
  in boys with PUV).
\item
  \textbf{Palpable bladder} or \textbf{abdominal distension}.
\item
  \textbf{Failure to thrive} or \textbf{recurrent urinary tract
  infections (UTIs)}.
\item
  \textbf{Azotaemia} or features of renal failure (e.g., vomiting,
  lethargy).
\end{itemize}

\subsection{Older Children}\label{older-children-2}

\begin{itemize}
\tightlist
\item
  \textbf{Flank or abdominal pain} (colicky if due to stones).
\item
  \textbf{Urinary frequency, urgency, or incontinence}.
\item
  \textbf{Haematuria}.
\item
  \textbf{Recurrent UTIs}.
\item
  \textbf{Palpable kidney or bladder} on examination.
\item
  \textbf{Hypertension} in chronic cases.
\item
  \textbf{Signs of chronic renal insufficiency} (pallor, growth
  retardation).
\end{itemize}

\section{Investigations}\label{investigations-27}

Investigations aim to:

\begin{enumerate}
\def\labelenumi{\arabic{enumi}.}
\tightlist
\item
  Confirm the presence and site of obstruction,
\item
  Identify the underlying cause,
\item
  Assess renal function and the degree of damage.
\end{enumerate}

\subsection{Laboratory Investigations}\label{laboratory-investigations}

\begin{itemize}
\tightlist
\item
  \textbf{Urinalysis:} Proteinuria, haematuria, pyuria, or evidence of
  infection.
\item
  \textbf{Urine culture and sensitivity:} To guide antibiotic therapy.
\item
  \textbf{Serum urea, creatinine, and electrolytes:} Assess renal
  function.
\item
  \textbf{Full blood count:} Anaemia or infection.
\item
  \textbf{Urine specific gravity:} Low in chronic cases due to tubular
  dysfunction.
\end{itemize}

\subsection{Imaging Studies}\label{imaging-studies}

\subsubsection{Ultrasound (USS)}\label{ultrasound-uss}

\begin{itemize}
\tightlist
\item
  First-line investigation.
\item
  Detects hydronephrosis, hydroureter, bladder wall thickening, and
  residual urine.
\item
  Antenatal ultrasound can detect hydronephrosis as early as the second
  trimester.
\end{itemize}

\subsubsection{Micturating Cystourethrogram
(MCUG)}\label{micturating-cystourethrogram-mcug}

\begin{itemize}
\tightlist
\item
  Essential for diagnosing \textbf{posterior urethral valves} and
  \textbf{vesicoureteric reflux}.
\item
  Should be performed after treating any active UTI.
\end{itemize}

\subsubsection{Diuretic Renogram (using MAG3 or
DTPA)}\label{diuretic-renogram-using-mag3-or-dtpa}

\begin{itemize}
\tightlist
\item
  Differentiates between obstructive and non-obstructive dilatation.
\item
  Provides functional information on differential renal function.
\end{itemize}

\subsubsection{Intravenous Urography
(IVU)}\label{intravenous-urography-ivu}

\begin{itemize}
\tightlist
\item
  May show delayed excretion, dilated calyces, or level of obstruction.
\item
  Limited use in children due to radiation exposure and replaced largely
  by renography.
\end{itemize}

\subsubsection{Other Imaging}\label{other-imaging}

\begin{itemize}
\tightlist
\item
  \textbf{CT Urography or MRI Urography:} For complex cases or extrinsic
  causes.
\item
  \textbf{Cystoscopy:} Direct visualization of posterior urethral valves
  or strictures.
\end{itemize}

\section{Management}\label{management-19}

Management depends on the \textbf{site, cause, and severity} of the
obstruction, as well as the \textbf{presence or absence of renal
failure}.

\subsection{General Principles}\label{general-principles-3}

\begin{enumerate}
\def\labelenumi{\arabic{enumi}.}
\tightlist
\item
  \textbf{Prompt relief of obstruction.}
\item
  \textbf{Preservation of renal function.}
\item
  \textbf{Treatment of infection and prevention of recurrence.}
\item
  \textbf{Correction of underlying cause.}
\item
  \textbf{Long-term follow-up} for renal growth and function.
\end{enumerate}

\subsection{Initial Stabilization}\label{initial-stabilization}

\begin{itemize}
\tightlist
\item
  \textbf{Assess hydration status} and \textbf{correct electrolyte
  imbalance}.
\item
  \textbf{Treat infections} aggressively with appropriate antibiotics.
\item
  \textbf{Bladder decompression} using catheterization if there is lower
  tract obstruction.
\item
  \textbf{Nephrostomy} or \textbf{ureterostomy} for upper tract
  obstruction if necessary.
\end{itemize}

\subsection{Specific Treatment Based on
Cause}\label{specific-treatment-based-on-cause}

\subsubsection{Posterior Urethral
Valves}\label{posterior-urethral-valves}

\begin{itemize}
\tightlist
\item
  \textbf{Initial management:} Catheterization for bladder drainage.
\item
  \textbf{Definitive management:} Endoscopic valve ablation
  (fulguration).
\item
  In cases with severe renal impairment, \textbf{temporary vesicostomy}
  may be indicated.
\end{itemize}

\subsubsection{Ureteropelvic Junction (UPJ)
Obstruction}\label{ureteropelvic-junction-upj-obstruction}

\begin{itemize}
\tightlist
\item
  \textbf{Observation} if mild and renal function preserved.
\item
  \textbf{Surgical correction (pyeloplasty)} if obstruction is
  significant or progressive.
\end{itemize}

\subsubsection{Ureterovesical Junction (UVJ) Obstruction /
Megaureter}\label{ureterovesical-junction-uvj-obstruction-megaureter}

\begin{itemize}
\tightlist
\item
  \textbf{Reimplantation surgery} or \textbf{tailoring} of the ureter as
  needed.
\end{itemize}

\subsubsection{Urolithiasis}\label{urolithiasis}

\begin{itemize}
\tightlist
\item
  \textbf{Hydration and analgesia.}
\item
  \textbf{Medical expulsive therapy} for small distal stones.
\item
  \textbf{Surgical removal} (ureteroscopy or open surgery) for larger or
  impacted stones.
\end{itemize}

\subsubsection{Schistosomiasis}\label{schistosomiasis}

\begin{itemize}
\tightlist
\item
  \textbf{Praziquantel} (40 mg/kg single dose) for all infected
  individuals.
\item
  Management of resultant strictures may require \textbf{endoscopic or
  surgical correction}.
\end{itemize}

\subsubsection{Neurogenic Bladder}\label{neurogenic-bladder}

\begin{itemize}
\tightlist
\item
  \textbf{Clean intermittent catheterization (CIC)}.
\item
  \textbf{Anticholinergic medications} to reduce detrusor overactivity.
\item
  \textbf{Bladder augmentation} in refractory cases.
\end{itemize}

\subsection{Chronic Management}\label{chronic-management}

\begin{itemize}
\tightlist
\item
  \textbf{Monitoring renal function} periodically.
\item
  \textbf{Blood pressure control} with antihypertensives if needed.
\item
  \textbf{Treatment of recurrent infections.}
\item
  \textbf{Nutritional support} to promote growth.
\item
  \textbf{Parental counseling} regarding long-term prognosis.
\end{itemize}

\section{Complications}\label{complications-26}

\begin{itemize}
\tightlist
\item
  \textbf{Hydronephrosis}
\item
  \textbf{Recurrent UTIs}
\item
  \textbf{Hypertension}
\item
  \textbf{Chronic kidney disease (CKD)}
\item
  \textbf{Bladder dysfunction}
\item
  \textbf{Growth retardation}
\item
  \textbf{Electrolyte disturbances}
\end{itemize}

In Ghana, late presentation often means that children may already have
advanced CKD by the time of diagnosis. This underscores the importance
of early detection through \textbf{antenatal screening} and
\textbf{newborn examination}.

\section{Prevention and Early
Detection}\label{prevention-and-early-detection-1}

\begin{enumerate}
\def\labelenumi{\arabic{enumi}.}
\tightlist
\item
  \textbf{Antenatal ultrasound} for detection of hydronephrosis.
\item
  \textbf{Neonatal screening} for poor urinary stream in male infants.
\item
  \textbf{Public education} on early symptoms and the need for prompt
  medical evaluation.
\item
  \textbf{Training of peripheral health workers} to identify signs of
  obstructive uropathy.
\item
  \textbf{Mass treatment and prevention of schistosomiasis} in endemic
  regions.
\end{enumerate}

\section{Prognosis}\label{prognosis-27}

Prognosis depends on:

\begin{itemize}
\tightlist
\item
  Age at diagnosis,
\item
  Duration and completeness of obstruction,
\item
  Residual renal function,
\item
  Presence of infections.
\end{itemize}

With \textbf{early diagnosis} and \textbf{appropriate intervention},
many children can achieve good long-term outcomes. However, delayed
cases may progress to \textbf{chronic renal failure}, requiring dialysis
or renal transplantation --- services that are still limited in many
parts of Ghana.

\section{Key Points}\label{key-points}

\begin{itemize}
\tightlist
\item
  Obstructive uropathy is a major cause of preventable renal failure in
  children.
\item
  Posterior urethral valves are the most common congenital cause in male
  infants.
\item
  Early detection through antenatal and postnatal screening is critical.
\item
  Management requires a multidisciplinary approach involving
  paediatricians, urologists, and nephrologists.
\item
  In endemic areas, schistosomiasis remains a significant preventable
  contributor.
\end{itemize}

\section{Further Reading}\label{further-reading-1}

\begin{enumerate}
\def\labelenumi{\arabic{enumi}.}
\tightlist
\item
  Anochie IC, Eke FU. Paediatric obstructive uropathy in Port Harcourt,
  Nigeria. \emph{Nigerian Journal of Paediatrics.} 2005;32(1):1--6.
\item
  Ducket JW. Posterior urethral valves: Concepts and management. \emph{J
  Urol.} 1992;148(5):1671--1675.
\item
  Odetunde OI, et al.~Spectrum and outcome of paediatric urological
  disorders in Enugu, Nigeria. \emph{Afr J Paediatr Surg.}
  2010;7(2):84--87.
\item
  Osifo OD, Okolo JC. Posterior urethral valves in Benin City, Nigeria.
  \emph{Ann Afr Med.} 2010;9(2):82--86.
\item
  Anumba DOC, et al.~Prenatal diagnosis of urinary tract abnormalities
  in Africa: Prospects and challenges. \emph{Trop Med Int Health.}
  2013;18(11):1373--1381.
\item
  Cheeseman SH, Shortliffe LD. Obstructive uropathy in children. In:
  Behrman RE, Kliegman RM, Jenson HB (eds). \emph{Nelson Textbook of
  Pediatrics.} 21st ed.~Elsevier; 2020.
\item
  Ghana Health Service. \emph{National Guidelines for Schistosomiasis
  Control in Ghana.} Accra: GHS;

  \begin{enumerate}
  \def\labelenumii{\arabic{enumii}.}
  \setcounter{enumii}{2020}
  \tightlist
  \item
  \end{enumerate}
\item
  World Health Organization. \emph{Paediatric Urology: Global Overview
  and Recommendations.} Geneva: WHO; 2020.
\end{enumerate}

\part{{Neurology}}

\chapter{Basic Neruroscience}\label{basic-neruroscience}

\section{\texorpdfstring{\textbf{Introduction}}{Introduction}}\label{introduction-36}

Understanding paediatric neurology begins with grasping the fundamentals
of the nervous system's anatomy and physiology. The nervous system in
children is dynamic and continuously developing, exhibiting distinct
features compared to that of adults.

\begin{enumerate}
\def\labelenumi{\alph{enumi}.}
\tightlist
\item
  \textbf{Structure and Development}

  \begin{itemize}
  \tightlist
  \item
    The \textbf{nervous system} comprises the \textbf{central nervous
    system (CNS)}---brain and spinal cord---and the \textbf{peripheral
    nervous system (PNS)}---cranial and spinal nerves.
  \item
    \textbf{Neural tube development} begins in the third week of
    gestation, giving rise to the brain and spinal cord.
  \item
    \textbf{Myelination}, the process of forming the myelin sheath
    around neurons, continues from the prenatal stage into adolescence.
    In children, the degree of myelination affects neurological function
    and should be taken into consideration during assessments.
  \end{itemize}
\item
  \textbf{Brain Regions and Functions}

  \begin{itemize}
  \tightlist
  \item
    \textbf{Cerebrum}: Higher functions like cognition, voluntary
    movement, and perception.
  \item
    \textbf{Cerebellum}: Coordination, balance, and motor control.
  \item
    \textbf{Brainstem}: Regulates vital functions like respiration,
    heart rate, and consciousness.
  \item
    \textbf{Spinal cord}: Conveys messages between the brain and the
    rest of the body.
  \end{itemize}
\item
  \textbf{Peripheral Nervous System}

  \begin{itemize}
  \tightlist
  \item
    Consists of \textbf{motor, sensory}, and \textbf{autonomic} nerves.
  \item
    Motor neurons control muscle activity, while sensory neurons
    transmit information like pain, temperature, and proprioception.
  \item
    Autonomic nerves regulate involuntary functions (e.g., heart rate,
    digestion).
  \end{itemize}
\item
  \textbf{Neurotransmitters}

  \begin{itemize}
  \tightlist
  \item
    \textbf{Acetylcholine}, \textbf{dopamine}, \textbf{GABA}, and
    \textbf{glutamate} play crucial roles in neural signalling.
  \item
    Imbalances are implicated in various neurological disorders like
    epilepsy, movement disorders, and developmental conditions.
  \end{itemize}
\end{enumerate}

\section{Pathological Processes in
Neurology}\label{pathological-processes-in-neurology}

Paediatric neurological diseases result from a variety of underlying
mechanisms:

\begin{enumerate}
\def\labelenumi{\alph{enumi}.}
\tightlist
\item
  \textbf{Congenital Disorders}

  \begin{itemize}
  \tightlist
  \item
    Neural tube defects (NTDs), e.g., spina bifida and anencephaly, due
    to folate deficiency in pregnancy. cy.
  \item
    \textbf{Cerebral palsy (CP)}: A group of permanent movement
    disorders from non-progressive disturbances in the developing foetal
    or infant brain.
  \end{itemize}
\item
  \textbf{Genetic and Metabolic Disorders}

  \begin{itemize}
  \tightlist
  \item
    \textbf{Neurocutaneous syndromes}: e.g., Tuberous Sclerosis and
    Neurofibromatosis.
  \item
    \textbf{Inborn errors of metabolism}: Can cause neurodegeneration or
    developmental delay (e.g., phenylketonuria, Tay-Sachs disease).
  \end{itemize}
\item
  \textbf{Infectious Causes}

  \begin{itemize}
  \tightlist
  \item
    \textbf{Meningitis}, \textbf{encephalitis}, and \textbf{brain
    abscesses} are common in low-resource settings.
  \item
    Causative agents include \emph{Neisseria meningitidis},
    \emph{Streptococcus pneumoniae}, \emph{Herpes Simplex Virus}, and
    \emph{Plasmodium falciparum} (cerebral malaria).
  \end{itemize}
\item
  \textbf{Inflammatory and Autoimmune Conditions}

  \begin{itemize}
  \tightlist
  \item
    \textbf{Acute disseminated encephalomyelitis (ADEM)}:
    Post-infectious or post-vaccination immune response.
  \item
    \textbf{Guillain-Barré Syndrome (GBS)}: Acute polyneuropathy causing
    weakness, often post-infection.
  \end{itemize}
\item
  \textbf{Epilepsy and Seizure Disorders}

  \begin{itemize}
  \tightlist
  \item
    \textbf{Febrile seizures} are common in Ghanaian children aged 6
    months to 5 years.
  \item
    \textbf{Epilepsy} can be idiopathic, structural, metabolic, or
    secondary to infection or trauma.
  \end{itemize}
\item
  \textbf{Trauma}

  \begin{itemize}
  \tightlist
  \item
    Head injury from road traffic accidents or falls is a leading cause
    of morbidity.
  \item
    May result in skull fractures, intracranial haemorrhage, or brain
    oedema.
  \end{itemize}
\item
  \textbf{Tumours}

  \begin{itemize}
  \tightlist
  \item
    Paediatric CNS tumours include \textbf{medulloblastoma},
    \textbf{astrocytoma}, and \textbf{ependymoma}.
  \item
    Present with increased intracranial pressure, focal deficits, or
    seizures.
  \end{itemize}
\end{enumerate}

\section{Neurological Signs and Symptoms in
Children}\label{neurological-signs-and-symptoms-in-children}

Children may not accurately describe their symptoms, so observation and
parental history are crucial.

\begin{enumerate}
\def\labelenumi{\alph{enumi}.}
\tightlist
\item
  \textbf{Seizures}

  \begin{itemize}
  \tightlist
  \item
    Generalized (tonic-clonic, absence) or focal.
  \item
    Observe the duration, type, postictal state, and presence of
    triggers (fever, sleep deprivation).
  \end{itemize}
\item
  \textbf{Developmental Delay}

  \begin{itemize}
  \tightlist
  \item
    Failure to achieve motor, cognitive, language, or social milestones.
  \item
    Global developmental delay affects multiple domains.
  \end{itemize}
\item
  \textbf{Headache}

  \begin{itemize}
  \tightlist
  \item
    Can result from infection (meningitis), intracranial pressure, or
    tension.
  \item
    Red flags: morning headache, vomiting, visual changes, altered
    consciousness.
  \end{itemize}
\item
  \textbf{Ataxia and Gait Abnormalities}

  \begin{itemize}
  \tightlist
  \item
    Unsteady gait or coordination problems can suggest cerebellar
    disease or vestibular dysfunction.
  \item
    Sudden onset may indicate infection, tumor, or intoxication.
  \end{itemize}
\item
  \textbf{Altered Consciousness}

  \begin{itemize}
  \tightlist
  \item
    Ranges from drowsiness to coma.
  \item
    Common causes: CNS infection, trauma, metabolic disturbances (e.g.,
    hypoglycemia), or seizures.
  \end{itemize}
\item
  \textbf{Motor Weakness and Paralysis}

  \begin{itemize}
  \tightlist
  \item
    May be upper motor neuron (spasticity, brisk reflexes) or lower
    motor neuron (flaccidity, fasciculations).
  \item
    Acute flaccid paralysis is notifiable (e.g., poliomyelitis, GBS).
  \end{itemize}
\item
  \textbf{Sensory Disturbances}

  \begin{itemize}
  \tightlist
  \item
    Less commonly reported in young children.
  \item
    May include numbness, tingling, or loss of proprioception.
  \end{itemize}
\item
  \textbf{Abnormal Movements}

  \begin{itemize}
  \tightlist
  \item
    Includes tremors, chorea, dystonia, or tics.
  \item
    Seen in conditions like Sydenham chorea (post-streptococcal),
    dystonic CP, or genetic syndromes.
  \end{itemize}
\end{enumerate}

\section{Localization Along the
Neuro-axis}\label{localization-along-the-neuro-axis}

Understanding where a lesion is located helps narrow the differential
diagnosis.

\begin{enumerate}
\def\labelenumi{\alph{enumi}.}
\tightlist
\item
  \textbf{Cerebral Cortex}

  \begin{itemize}
  \tightlist
  \item
    Lesions cause \textbf{hemiparesis}, \textbf{seizures}, or
    \textbf{language deficits}.
  \item
    Can result from ischemia, trauma, infection, or malformations.
  \end{itemize}
\item
  \textbf{Basal Ganglia}

  \begin{itemize}
  \tightlist
  \item
    Involved in movement control.
  \item
    Disorders cause \textbf{involuntary movements} (e.g., chorea,
    dystonia).
  \end{itemize}
\item
  \textbf{Brainstem}

  \begin{itemize}
  \tightlist
  \item
    Cranial nerve deficits (e.g., facial palsy), eye movement
    abnormalities, and vital sign instability.
  \item
    Lesions are often life-threatening.
  \end{itemize}
\item
  \textbf{Cerebellum}

  \begin{itemize}
  \tightlist
  \item
    \textbf{Ataxia}, \textbf{dysmetria}, and \textbf{intention tremor}.
  \item
    Tumours or infections like cerebellitis are common culprits.
  \end{itemize}
\item
  \textbf{Spinal Cord}

  \begin{itemize}
  \tightlist
  \item
    \textbf{Motor and sensory level deficits}, reflex changes, and
    incontinence.
  \item
    Trauma or transverse myelitis is a common cause.
  \end{itemize}
\item
  \textbf{Peripheral Nerves}

  \begin{itemize}
  \tightlist
  \item
    Symmetrical weakness, absent reflexes, and distal sensory loss.
  \item
    Seen in GBS or hereditary neuropathies.
  \end{itemize}
\item
  \textbf{Neuromuscular Junction}

  \begin{itemize}
  \tightlist
  \item
    Fluctuating weakness, especially ocular and bulbar muscles.
  \item
    Example: Myasthenia Gravis.
  \end{itemize}
\item
  \textbf{Muscle}

  \begin{itemize}
  \tightlist
  \item
    Proximal weakness and hypotonia.
  \item
    Seen in \textbf{muscular dystrophies}, \textbf{myopathies}, and
    metabolic muscle diseases.
  \end{itemize}
\end{enumerate}

\section{Basic Neurological
Investigations}\label{basic-neurological-investigations}

Timely and appropriate investigations help confirm clinical suspicion.

\begin{enumerate}
\def\labelenumi{\alph{enumi}.}
\tightlist
\item
  \textbf{Neuroimaging}

  \begin{itemize}
  \tightlist
  \item
    \textbf{CT scan}: Good for acute trauma or haemorrhages. Widely
    available in Ghana, but it has been linked to radiation exposure.
  \item
    \textbf{MRI}: Better for soft tissue detail, congenital
    malformations, or tumours. Less accessible but ideal for subacute
    and chronic conditions.
  \end{itemize}
\item
  \textbf{Electroencephalography (EEG)}

  \begin{itemize}
  \tightlist
  \item
    Assesses the brain's electrical activity.
  \item
    Useful in seizure evaluation, epilepsy classification, and
    encephalopathy.
  \end{itemize}
\item
  \textbf{Lumbar Puncture}

  \begin{itemize}
  \tightlist
  \item
    Essential for evaluating CNS infections.
  \item
    CSF analysis helps differentiate bacterial, viral, or tuberculous
    meningitis.
  \item
    Ensure no signs of raised intracranial pressure before performing.
  \end{itemize}
\item
  \textbf{Blood Tests}

  \begin{itemize}
  \tightlist
  \item
    \textbf{CBC}, \textbf{electrolytes}, \textbf{blood glucose},
    \textbf{renal/liver function}, \textbf{malaria test}.
  \item
    Metabolic screening for inborn errors if available.
  \end{itemize}
\item
  \textbf{Nerve Conduction Studies (NCS)/Electromyography (EMG)}

  \begin{itemize}
  \tightlist
  \item
    Evaluate peripheral nerve and muscle function.
  \item
    Useful in GBS, neuropathies, and myopathies.
  \end{itemize}
\item
  \textbf{Genetic Testing}

  \begin{itemize}
  \tightlist
  \item
    For suspected inherited or syndromic conditions.
  \item
    May be limited in availability and affordability.
  \end{itemize}
\end{enumerate}

\section{Basic Neurological
Procedures}\label{basic-neurological-procedures}

These are diagnostic and sometimes therapeutic.

\begin{enumerate}
\def\labelenumi{\alph{enumi}.}
\tightlist
\item
  \textbf{Lumbar Puncture (Spinal Tap)}

  \begin{itemize}
  \tightlist
  \item
    Performed in suspected meningitis, encephalitis, or to measure
    intracranial pressure.
  \item
    Atraumatic technique is important. Avoid in cases of suspected
    elevated ICP or spinal deformities.
  \end{itemize}
\item
  \textbf{EEG Application}

  \begin{itemize}
  \tightlist
  \item
    Involves placing electrodes on the scalp using the 10-20 system.
  \item
    Should be interpreted by trained personnel.
  \end{itemize}
\item
  \textbf{Neuroimaging Requests}

  \begin{itemize}
  \tightlist
  \item
    Clinicians should provide clear clinical information and suspected
    diagnosis when requesting CT/MRI.
  \item
    Sedation may be required in children for MRI.
  \end{itemize}
\item
  \textbf{Muscle Biopsy}

  \begin{itemize}
  \tightlist
  \item
    Used in diagnosing myopathies.
  \item
    Requires sterile technique and histopathological expertise.
  \end{itemize}
\item
  \textbf{Botulinum Toxin Injections}

  \begin{itemize}
  \tightlist
  \item
    Used for spasticity in conditions like CP.
  \item
    Requires experience and is often performed under sedation.
  \end{itemize}
\item
  \textbf{CSF Shunt Insertion}

  \begin{itemize}
  \tightlist
  \item
    Performed by neurosurgeons in hydrocephalus.
  \item
    Ventriculoperitoneal (VP) shunt is common. Risks include infection
    and obstruction.
  \end{itemize}
\end{enumerate}

\section{Final Notes for Medical Students in
Ghana}\label{final-notes-for-medical-students-in-ghana}

\begin{itemize}
\tightlist
\item
  Paediatric neurological disorders are common, and early recognition is
  crucial for better outcomes.
\item
  Focus on \textbf{comprehensive history-taking} and \textbf{detailed
  neurological examination}.
\item
  Understand the \textbf{developmental context}: some signs that are
  abnormal in adults are normal in infants (e.g., primitive reflexes).
\item
  Pay attention to \textbf{public health issues} like malaria,
  meningitis, malnutrition, and trauma, which are common in Ghana and
  major contributors to paediatric neurological morbidity.
\item
  Always consider \textbf{preventable causes}---promote antenatal care,
  immunizations (e.g., against polio, Hib, pneumococcus), good
  nutrition, and road safety.
\item
  Collaborate with neurology, paediatrics, radiology, and rehabilitation
  teams when managing neurological cases.
\end{itemize}

\chapter{Spectrum of Neurological Disorders in
Children}\label{spectrum-of-neurological-disorders-in-children}

\section{Introduction}\label{introduction-37}

Neurological disorders in children represent a diverse group of
conditions that affect the brain, spinal cord, peripheral nerves, or
neuromuscular junctions. These disorders can lead to developmental
delays, motor dysfunction, seizures, and cognitive impairments. In
Ghana, the burden of childhood neurological disorders is significant due
to factors such as limited resources, perinatal complications,
infections, malnutrition, and lack of early diagnosis and intervention.

Understanding the spectrum of pediatric neurological conditions is vital
for early identification, diagnosis, treatment, and referral.

\section{Classification of Pediatric Neurological
Disorders}\label{classification-of-pediatric-neurological-disorders}

Neurological disorders in children can be classified into the following
major categories:

\begin{enumerate}
\def\labelenumi{\arabic{enumi}.}
\tightlist
\item
  Neurodevelopmental Disorders
\item
  Epileptic Disorders
\item
  Cerebrovascular Disorders
\item
  Neuromuscular Disorders
\item
  Infectious and Post-infectious Disorders
\item
  Metabolic and Genetic Disorders
\item
  Neurocutaneous Syndromes
\item
  Brain Tumors and Space-occupying Lesions
\item
  Head Trauma and Acquired Brain Injuries
\end{enumerate}

\section{Neurodevelopmental
Disorders}\label{neurodevelopmental-disorders}

These disorders typically manifest early in development and are
characterized by impairments in personal, social, academic, or
occupational functioning.

\begin{enumerate}
\def\labelenumi{\arabic{enumi}.}
\item
  \textbf{Cerebral Palsy (CP)}

  \begin{enumerate}
  \def\labelenumii{\arabic{enumii}.}
  \tightlist
  \item
    \textbf{Definition:} A group of permanent movement disorders due to
    non-progressive disturbances in the developing brain.
  \item
    \textbf{Types:} Spastic (most common), dyskinetic, ataxic, and
    mixed.
  \item
    \textbf{Causes:}

    \begin{itemize}
    \tightlist
    \item
      Perinatal asphyxia (common in Ghana)
    \item
      Premature birth
    \item
      Neonatal jaundice (kernicterus)
    \item
      Infections (TORCH)
    \end{itemize}
  \end{enumerate}
\item
  \textbf{Clinical Features:}

  \begin{itemize}
  \tightlist
  \item
    Delayed milestones
  \item
    Abnormal tone (increased or decreased)
  \item
    Reflex abnormalities
  \end{itemize}
\item
  \textbf{Management:}

  \begin{itemize}
  \tightlist
  \item
    Multidisciplinary approach: physiotherapy, occupational therapy,
    antispastic medications (e.g., baclofen), orthopedic interventions.
  \item
    Early intervention programs are critical.
  \end{itemize}
\end{enumerate}

\textbf{B. Autism Spectrum Disorder (ASD)}

\begin{itemize}
\tightlist
\item
  \textbf{Definition:} Neurodevelopmental disorder characterized by
  deficits in social interaction and communication, with restricted,
  repetitive behaviors.
\item
  \textbf{Diagnosis:} Based on DSM-5 criteria.
\item
  \textbf{Onset:} Before age 3.
\item
  \textbf{Red Flags:}

  \begin{itemize}
  \tightlist
  \item
    Lack of eye contact
  \item
    No single words by 16 months
  \item
    No two-word phrases by 2 years
  \end{itemize}
\item
  \textbf{Management:}

  \begin{itemize}
  \item
    Behavioral therapy
  \item
    Speech therapy
  \item
    Structured educational support
  \end{itemize}
\end{itemize}

\textbf{C. Attention-Deficit/Hyperactivity Disorder (ADHD)}

\begin{itemize}
\tightlist
\item
  \textbf{Symptoms:} Inattention, hyperactivity, impulsiveness.
\item
  \textbf{Diagnosis:} Based on clinical history and observation.
\item
  \textbf{Management:}

  \begin{itemize}
  \tightlist
  \item
    Behavioral therapy
  \item
    Medications (e.g., methylphenidate---rarely used in Ghana due to
    availability)
  \end{itemize}
\end{itemize}

\section{Epileptic Disorders}\label{epileptic-disorders}

Epilepsy is a common neurological disorder in Ghanaian children due to
high rates of perinatal insults, CNS infections, and trauma.

\textbf{A. Seizure Classification}

\begin{itemize}
\tightlist
\item
  Focal (Partial) Seizures
\item
  Generalized Seizures
\item
  Absence Seizures
\item
  Febrile Seizures
\item
  Infantile Spasms (West Syndrome)
\end{itemize}

\textbf{B. Etiology}

\begin{itemize}
\tightlist
\item
  Idiopathic (genetic)
\item
  Structural (trauma, tumor)
\item
  Metabolic (hypoglycemia, electrolyte imbalance)
\item
  Infectious (meningitis, cerebral malaria)
\end{itemize}

\textbf{C. Diagnosis}

\begin{itemize}
\item
\item
  Clinical history
\item
\item
  EEG
\item
\item
  Neuroimaging (CT or MRI if accessible)
\item
\item
  Blood tests for metabolic derangements
\item
\end{itemize}

\textbf{D. Management}

\begin{itemize}
\item
\item
  Acute seizure: Diazepam or lorazepam
\item
\item
  Long-term: Carbamazepine, sodium valproate, phenobarbital (commonly
  used in Ghana)
\item
\item
  Treat the underlying cause
\item
\item
  Educate caregivers
\item
\end{itemize}

\section{Cerebrovascular Disorders}\label{cerebrovascular-disorders}

Relatively rare but essential to consider.

\textbf{A. Stroke in Children}

\begin{itemize}
\item
\item
  \textbf{Causes:}

  \begin{itemize}
  \item
  \item
    Sickle Cell Disease (SCD) is common in Ghana
  \item
  \item
    Congenital heart disease
  \item
  \item
    Infections (e.g., meningitis, endocarditis)
  \item
  \end{itemize}
\item
\item
  \textbf{Signs:}

  \begin{itemize}
  \item
  \item
    Hemiplegia
  \item
  \item
    Altered consciousness
  \item
  \end{itemize}
\item
\item
  \textbf{Diagnosis:}

  \begin{itemize}
  \item
  \item
    Neuroimaging
  \item
  \item
    Blood tests (e.g., sickling test)
  \item
  \end{itemize}
\item
\item
  \textbf{Management:}

  \begin{itemize}
  \item
  \item
    Supportive care
  \item
  \item
    Antiplatelets (aspirin)
  \item
  \item
    Transfusion in SCD
  \item
  \end{itemize}
\item
\end{itemize}

\section{Neuromuscular Disorders}\label{neuromuscular-disorders}

These affect motor nerves, neuromuscular junctions, or muscles.

\textbf{A. Duchenne Muscular Dystrophy (DMD)}

\begin{itemize}
\item
\item
  \textbf{Genetic disorder:} X-linked recessive
\item
\item
  \textbf{Onset:} 2--5 years
\item
\item
  \textbf{Signs:}

  \begin{itemize}
  \item
  \item
    Gower's sign
  \item
  \item
    Proximal muscle weakness
  \item
  \item
    Calf pseudohypertrophy
  \item
  \end{itemize}
\item
\item
  \textbf{Diagnosis:}

  \begin{itemize}
  \item
  \item
    Elevated CK
  \item
  \item
    Genetic testing (if available)
  \item
  \end{itemize}
\item
\item
  \textbf{Management:}

  \begin{itemize}
  \item
  \item
    Steroids
  \item
  \item
    Physiotherapy
  \item
  \item
    Monitor respiratory and cardiac function
  \item
  \end{itemize}
\item
\end{itemize}

\textbf{B. Guillain-Barré Syndrome (GBS)}

\begin{itemize}
\item
\item
  \textbf{Acute autoimmune polyneuropathy}
\item
\item
  \textbf{Trigger:} Often post-infectious (e.g., Campylobacter, CMV)
\item
\item
  \textbf{Symptoms:}

  \begin{itemize}
  \item
  \item
    Ascending paralysis
  \item
  \item
    Areflexia
  \item
  \end{itemize}
\item
\item
  \textbf{Management:}

  \begin{itemize}
  \item
  \item
    Supportive care
  \item
  \item
    IVIG (limited availability)
  \item
  \item
    Monitor respiratory function
  \item
  \end{itemize}
\item
\end{itemize}

\section{Infectious and Post-Infectious
Disorders}\label{infectious-and-post-infectious-disorders}

A significant cause of neurological disease in children in Ghana.

\textbf{A. Bacterial Meningitis}

\begin{itemize}
\item
\item
  \textbf{Causes:} \emph{S. pneumoniae}, \emph{N. meningitidis},
  \emph{H. influenzae}
\item
\item
  \textbf{Symptoms:}

  \begin{itemize}
  \item
  \item
    Fever, neck stiffness, bulging fontanelle (infants)
  \item
  \item
    Altered consciousness, seizures
  \item
  \end{itemize}
\item
\item
  \textbf{Diagnosis:}

  \begin{itemize}
  \item
  \item
    CSF analysis
  \item
  \end{itemize}
\item
\item
  \textbf{Complications:}

  \begin{itemize}
  \item
  \item
    Hydrocephalus
  \item
  \item
    Hearing loss
  \item
  \item
    Epilepsy
  \item
  \end{itemize}
\item
\item
  \textbf{Treatment:}

  \begin{itemize}
  \item
  \item
    IV antibiotics (ceftriaxone)
  \item
  \item
    Supportive care
  \item
  \end{itemize}
\item
\end{itemize}

\textbf{B. Cerebral Malaria}

\begin{itemize}
\item
\item
  \textbf{Caused by:} \emph{Plasmodium falciparum}
\item
\item
  \textbf{Symptoms:}

  \begin{itemize}
  \item
  \item
    Seizures
  \item
  \item
    Coma
  \item
  \end{itemize}
\item
\item
  \textbf{Diagnosis:}

  \begin{itemize}
  \item
  \item
    Blood smear
  \item
  \end{itemize}
\item
\item
  \textbf{Treatment:}

  \begin{itemize}
  \item
  \item
    IV artesunate
  \item
  \item
    Anticonvulsants
  \item
  \end{itemize}
\item
\item
  \textbf{Prevention:}

  \begin{itemize}
  \item
  \item
    Insecticide-treated nets
  \item
  \item
    IPT in pregnancy
  \item
  \end{itemize}
\item
\end{itemize}

\textbf{C. Tuberculous Meningitis}

\begin{itemize}
\item
\item
  \textbf{Symptoms:}

  \begin{itemize}
  \item
  \item
    Gradual onset of fever, headache, vomiting, neck stiffness
  \item
  \end{itemize}
\item
\item
  \textbf{Diagnosis:}

  \begin{itemize}
  \item
  \item
    CSF (high protein, low glucose)
  \item
  \item
    GeneXpert (if available)
  \item
  \end{itemize}
\item
\item
  \textbf{Treatment:}

  \begin{itemize}
  \item
  \item
    Anti-TB drugs for 12 months
  \item
  \item
    Steroids (dexamethasone)
  \item
  \end{itemize}
\item
\end{itemize}

\section{Metabolic and Genetic
Disorders}\label{metabolic-and-genetic-disorders}

Rare, but often underdiagnosed in Ghana due to a lack of advanced
diagnostics.

\textbf{A. Phenylketonuria (PKU), Tay-Sachs, and Others}

\begin{itemize}
\item
\item
  \textbf{Symptoms:}

  \begin{itemize}
  \item
  \item
    Developmental delay
  \item
  \item
    Seizures
  \item
  \item
    Hypotonia
  \item
  \end{itemize}
\item
\item
  \textbf{Diagnosis:}

  \begin{itemize}
  \item
  \item
    Requires metabolic screening (Guthrie test---limited availability)
  \item
  \end{itemize}
\item
\item
  \textbf{Management:}

  \begin{itemize}
  \item
  \item
    Dietary modifications
  \item
  \item
    Genetic counseling
  \item
  \end{itemize}
\item
\end{itemize}

\textbf{B. Mitochondrial Disorders}

\begin{itemize}
\item
\item
  Often present with multisystem involvement.
\item
\item
  Poor feeding, lactic acidosis, seizures.
\item
\end{itemize}

\section{Neurocutaneous Syndromes}\label{neurocutaneous-syndromes}

\textbf{A. Tuberous Sclerosis Complex (TSC)}

\begin{itemize}
\item
\item
  \textbf{Signs:}

  \begin{itemize}
  \item
  \item
    Seizures (infantile spasms)
  \item
  \item
    Skin lesions (ash leaf spots, shagreen patches)
  \item
  \item
    Intellectual disability
  \item
  \end{itemize}
\item
\item
  \textbf{Diagnosis:}

  \begin{itemize}
  \item
  \item
    Clinical + imaging (MRI may show cortical tubers)
  \item
  \end{itemize}
\item
\item
  \textbf{Management:}

  \begin{itemize}
  \item
  \item
    Seizure control
  \item
  \item
    Multidisciplinary follow-up
  \item
  \end{itemize}
\item
\end{itemize}

\textbf{B. Neurofibromatosis Type 1}

\begin{itemize}
\item
\item
  \textbf{Signs:}

  \begin{itemize}
  \item
  \item
    Café-au-lait spots
  \item
  \item
    Neurofibromas
  \item
  \item
    Learning disabilities
  \item
  \end{itemize}
\item
\item
  \textbf{Complications:}

  \begin{itemize}
  \item
  \item
    Optic gliomas
  \item
  \end{itemize}
\item
\item
  \textbf{Management:}

  \begin{itemize}
  \item
  \item
    Surveillance for tumors
  \item
  \item
    Genetic counseling
  \item
  \end{itemize}
\item
\end{itemize}

\section{Brain Tumors and Space-Occupying
Lesions}\label{brain-tumors-and-space-occupying-lesions}

Though rare, brain tumors must be considered, especially in children
with persistent headaches, vomiting, or seizures.

\textbf{Common Types in Children:}

\begin{itemize}
\item
\item
  Medulloblastoma
\item
\item
  Astrocytoma
\item
\end{itemize}

\textbf{Symptoms:}

\begin{itemize}
\item
\item
  Headache
\item
\item
  Morning vomiting
\item
\item
  Papilledema
\item
\item
  Focal deficits
\item
\end{itemize}

\textbf{Diagnosis:}

\begin{itemize}
\item
\item
  CT/MRI
\item
\item
  Biopsy (if accessible)
\item
\end{itemize}

\textbf{Management:}

\begin{itemize}
\item
\item
  Surgical resection
\item
\item
  Radiotherapy
\item
\item
  Chemotherapy
\item
\end{itemize}

\section{Head Trauma and Acquired Brain
Injury}\label{head-trauma-and-acquired-brain-injury}

Frequent in Ghana due to road traffic accidents and falls.

\textbf{A. Types:}

\begin{itemize}
\item
\item
  Concussion
\item
\item
  Contusion
\item
\item
  Hematomas (epidural, subdural)
\item
\end{itemize}

\textbf{B. Signs:}

\begin{itemize}
\item
\item
  Loss of consciousness
\item
\item
  Vomiting
\item
\item
  Seizures
\item
\item
  Pupillary changes
\item
\end{itemize}

\textbf{C. Management:}

\begin{itemize}
\item
\item
  ABCs (Airway, Breathing, Circulation)
\item
\item
  Neuroimaging
\item
\item
  Neurosurgical referral
\item
\end{itemize}

\textbf{D. Prevention:}

\begin{itemize}
\item
\item
  Use of helmets, child restraints
\item
\item
  Public health education
\item
\end{itemize}

\section{Diagnostic Approach to a Child with Neurological
Symptoms}\label{diagnostic-approach-to-a-child-with-neurological-symptoms}

\begin{enumerate}
\def\labelenumi{\arabic{enumi}.}
\item
\item
  \textbf{History Taking}

  \begin{itemize}
  \item
  \item
    Antenatal, birth, and developmental history
  \item
  \item
    Family history
  \item
  \item
    Onset and progression of symptoms
  \item
  \end{itemize}
\item
\item
  \textbf{Physical Examination}

  \begin{itemize}
  \item
  \item
    Neurological exam: tone, reflexes, cranial nerves
  \item
  \item
    General exam: dysmorphic features, skin lesions
  \item
  \end{itemize}
\item
\item
  \textbf{Investigations}

  \begin{itemize}
  \item
  \item
    Imaging (CT/MRI)
  \item
  \item
    EEG
  \item
  \item
    CSF analysis
  \item
  \item
    Laboratory (CBC, glucose, electrolytes, infection screening)
  \item
  \item
    Genetic/metabolic workup (if available)
  \item
  \end{itemize}
\item
\end{enumerate}

\section{Challenges in Ghana}\label{challenges-in-ghana}

\begin{itemize}
\item
\item
  Limited access to pediatric neurologists
\item
\item
  Inadequate neuroimaging facilities in rural areas
\item
\item
  Delayed diagnosis and referral
\item
\item
  Financial constraints
\item
\item
  Stigmatization and lack of awareness
\item
\end{itemize}

\section{Recommendations for Medical Students and Health
Workers}\label{recommendations-for-medical-students-and-health-workers}

\begin{itemize}
\item
\item
  Recognize early signs of neurological disorders
\item
\item
  Take a thorough developmental and birth history
\item
\item
  Use clinical judgment when diagnostic tools are limited
\item
\item
  Educate parents and caregivers
\item
\item
  Refer early to tertiary centers (e.g., Korle-Bu, Komfo Anokye Teaching
  Hospital)
\item
\item
  Advocate for public health interventions and awareness
\item
\end{itemize}

\section{Conclusion}\label{conclusion-27}

Neurological disorders in children in Ghana are diverse, with
significant morbidity. Early recognition, timely referral, and
multidisciplinary care are crucial. Despite resource limitations,
medical students and practitioners can make a significant impact through
clinical acumen, advocacy, and community education.

\chapter{CNS Infections}\label{cns-infections}

Central nervous system (CNS) infections are among the most serious
medical conditions affecting children, often leading to high morbidity
and mortality if not promptly diagnosed and treated. They include
infections that involve the brain (encephalitis), meninges (meningitis),
or both (meningoencephalitis). In sub-Saharan Africa, including Ghana,
CNS infections are common due to the high burden of bacterial, viral,
parasitic, and fungal diseases. Understanding the pathophysiology,
presentation, and management of these infections is crucial for all
medical students and clinicians caring for children.

\section{Introduction}\label{introduction-38}

CNS infections encompass a spectrum of conditions caused by
microorganisms that invade the central nervous system, leading to
inflammation of the brain, spinal cord, or their protective coverings.
The major forms are:

\begin{itemize}
\tightlist
\item
  \textbf{Meningitis:} Inflammation of the meninges, often bacterial or
  viral.
\item
  \textbf{Encephalitis:} Inflammation of the brain parenchyma, usually
  viral.
\item
  \textbf{Meningoencephalitis:} Combination of both meningitis and
  encephalitis.
\item
  \textbf{Brain abscess:} Localized collection of pus within the brain.
\item
  \textbf{Subdural or epidural empyema:} Collection of pus between
  meningeal layers or between dura and skull.
\end{itemize}

These infections are medical emergencies. Early diagnosis and aggressive
management can prevent death and long-term neurological sequelae such as
hearing loss, seizures, and cognitive impairment.

\section{Epidemiology}\label{epidemiology-6}

CNS infections occur worldwide but are particularly common in
low-resource settings.\\
In \textbf{Ghana and West Africa}, they remain leading causes of
hospitalization and death in children under five years.

\begin{itemize}
\tightlist
\item
  \textbf{Bacterial meningitis} is endemic in the ``meningitis belt,''
  which includes northern Ghana, and outbreaks occur periodically.
\item
  \textbf{Viral infections}, especially enteroviruses, herpes simplex
  virus (HSV), and arboviruses (e.g., West Nile virus), are also
  significant.
\item
  \textbf{Parasitic infections} such as \textbf{cerebral malaria} remain
  a major cause of CNS involvement in endemic regions.
\item
  \textbf{Fungal infections} (Cryptococcus neoformans, Candida species)
  occur primarily in immunocompromised children.
\end{itemize}

The burden of CNS infections in African countries is magnified by
delayed presentation, limited access to diagnostic tools, and poor
vaccination coverage.

\section{Aetiology}\label{aetiology-17}

\subsection{\texorpdfstring{\textbf{Bacterial
Causes}}{Bacterial Causes}}\label{bacterial-causes}

Common organisms vary by age:

\begin{itemize}
\tightlist
\item
  \textbf{Neonates:} Group B \emph{Streptococcus}, \emph{Escherichia
  coli}, \emph{Listeria monocytogenes}.
\item
  \textbf{Infants and young children:} \emph{Streptococcus pneumoniae},
  \emph{Neisseria meningitidis}, \emph{Haemophilus influenzae} type b
  (Hib).
\item
  \textbf{Older children and adolescents:} \emph{Neisseria meningitidis}
  and \emph{Streptococcus pneumoniae}.
\end{itemize}

\subsection{\texorpdfstring{2. \textbf{Viral
Causes}}{2. Viral Causes}}\label{viral-causes}

\begin{itemize}
\tightlist
\item
  Enteroviruses (Coxsackie, Echovirus)
\item
  Herpes simplex virus type 1 and 2
\item
  Varicella-zoster virus
\item
  Mumps and measles viruses
\item
  Arboviruses (e.g., Japanese encephalitis, West Nile virus)
\end{itemize}

\subsection{\texorpdfstring{3. \textbf{Parasitic
Causes}}{3. Parasitic Causes}}\label{parasitic-causes}

\begin{itemize}
\tightlist
\item
  \emph{Plasmodium falciparum} (cerebral malaria)
\item
  \emph{Toxoplasma gondii}
\item
  \emph{Trypanosoma brucei} (African sleeping sickness)
\end{itemize}

\subsection{\texorpdfstring{4. \textbf{Fungal
Causes}}{4. Fungal Causes}}\label{fungal-causes}

\begin{itemize}
\tightlist
\item
  \emph{Cryptococcus neoformans}
\item
  \emph{Candida albicans}
\item
  \emph{Aspergillus} species (rare)
\end{itemize}

\section{Pathophysiology}\label{pathophysiology-25}

The CNS is normally protected by the \textbf{blood-brain barrier (BBB)}
and the \textbf{meningeal layers}. Infection occurs when microorganisms
breach these defenses via:

\begin{enumerate}
\def\labelenumi{\arabic{enumi}.}
\tightlist
\item
  \textbf{Hematogenous spread}: The most common route, from systemic
  infection or nasopharyngeal colonization.
\item
  \textbf{Contiguous spread}: From otitis media, sinusitis, or
  mastoiditis.
\item
  \textbf{Direct inoculation}: From trauma, surgery, or congenital
  defects (e.g., spina bifida).
\item
  \textbf{Retrograde neuronal spread}: seen with HSV and rabies.
\end{enumerate}

Once pathogens enter the CNS:

\begin{itemize}
\tightlist
\item
  They trigger an \textbf{inflammatory response} involving cytokines,
  prostaglandins, and leukocyte infiltration.
\item
  This causes \textbf{cerebral edema}, \textbf{increased intracranial
  pressure (ICP)}, \textbf{reduced cerebral perfusion}, and
  \textbf{neuronal injury}.
\item
  In meningitis, inflammation of the leptomeninges leads to
  \textbf{disruption of CSF flow} and \textbf{hydrocephalus}.
\item
  In encephalitis, direct infection of neurons and glial cells causes
  \textbf{necrosis} and \textbf{demyelination}.
\end{itemize}

\section{Clinical Features}\label{clinical-features-23}

The presentation depends on the child's age and the specific infection
but can be broadly grouped.

\subsection{\texorpdfstring{\textbf{In
Neonates}}{In Neonates}}\label{in-neonates-1}

\begin{itemize}
\tightlist
\item
  Fever or hypothermia
\item
  Poor feeding and lethargy
\item
  Irritability, high-pitched cry
\item
  Bulging fontanelle
\item
  Seizures
\item
  Apnea or cyanosis
\end{itemize}

\subsection{\texorpdfstring{\textbf{In Older Infants and
Children}}{In Older Infants and Children}}\label{in-older-infants-and-children}

\begin{itemize}
\tightlist
\item
  Fever, headache, vomiting
\item
  Neck stiffness (meningism)
\item
  Photophobia
\item
  Altered level of consciousness
\item
  Seizures
\item
  Signs of raised intracranial pressure (ICP): papilledema, bradycardia,
  hypertension, and irregular respiration (Cushing's triad)
\item
  Focal neurological deficits (in encephalitis or abscess)
\end{itemize}

\subsection{\texorpdfstring{\textbf{Specific
Clues}}{Specific Clues}}\label{specific-clues}

\begin{itemize}
\tightlist
\item
  \textbf{Petechial rash} → \emph{Neisseria meningitidis} infection.
\item
  \textbf{Paralysis, ataxia, movement disorders} → viral encephalitis.
\item
  \textbf{Severe anemia and coma} → cerebral malaria.
\end{itemize}

\section{Differential Diagnosis}\label{differential-diagnosis-22}

CNS infections must be differentiated from other causes of altered
sensorium or seizures in children:

\begin{itemize}
\tightlist
\item
  Cerebral malaria
\item
  Epilepsy or status epilepticus
\item
  Febrile seizures
\item
  Metabolic disorders (hypoglycemia, hyponatremia, uremia)
\item
  Intracranial hemorrhage or tumor
\item
  Toxic encephalopathy
\end{itemize}

\section{Investigations}\label{investigations-28}

\subsection{\texorpdfstring{\textbf{Laboratory
Investigations}}{Laboratory Investigations}}\label{laboratory-investigations-1}

\begin{enumerate}
\def\labelenumi{\arabic{enumi}.}
\tightlist
\item
  \textbf{Full blood count:} Leukocytosis in bacterial infections;
  lymphocytosis in viral causes.
\item
  \textbf{Blood cultures:} Identify causative bacteria in 30--50\% of
  cases.
\item
  \textbf{Lumbar puncture (LP):}

  \begin{itemize}
  \tightlist
  \item
    Gold standard for meningitis diagnosis unless contraindicated (e.g.,
    raised ICP, focal neurological signs).
  \item
    \textbf{CSF findings:}

    \begin{itemize}
    \tightlist
    \item
      \textbf{Bacterial meningitis:} High protein, low glucose, turbid
      appearance, neutrophil predominance.
    \item
      \textbf{Viral meningitis:} Normal or mildly elevated protein,
      normal glucose, lymphocyte predominance.
    \item
      \textbf{Fungal/TB meningitis:} Elevated protein, low glucose,
      lymphocytes, positive India ink or acid-fast bacilli.
    \end{itemize}
  \end{itemize}
\item
  \textbf{CSF Gram stain and culture:} Identifies specific bacteria.
\item
  \textbf{Polymerase chain reaction (PCR):} Detects viral DNA/RNA (HSV,
  enteroviruses).
\item
  \textbf{Rapid antigen tests:} Useful for \emph{Neisseria
  meningitidis}, \emph{H. influenzae}, and \emph{S. pneumoniae}.
\end{enumerate}

\subsection{\texorpdfstring{\textbf{Neuroimaging}}{Neuroimaging}}\label{neuroimaging}

\begin{itemize}
\tightlist
\item
  \textbf{CT or MRI brain} before LP if raised ICP or focal signs are
  suspected.
\item
  May reveal cerebral edema, abscess, or hydrocephalus.
\end{itemize}

\subsection{\texorpdfstring{\textbf{Other
Tests}}{Other Tests}}\label{other-tests}

\begin{itemize}
\tightlist
\item
  Blood glucose and electrolytes.
\item
  Malaria smear or rapid diagnostic test (to exclude cerebral malaria).
\item
  HIV screening in chronic or atypical infections.
\end{itemize}

\section{Management}\label{management-20}

CNS infections constitute a \textbf{medical emergency}. Prompt empirical
therapy, supportive care, and control of complications are vital.

\subsection{\texorpdfstring{\textbf{1. Initial
Stabilization}}{1. Initial Stabilization}}\label{initial-stabilization-1}

\begin{itemize}
\tightlist
\item
  Airway, breathing, and circulation support.
\item
  Control seizures with intravenous diazepam or phenobarbital.
\item
  Manage raised ICP: elevate head, restrict fluids, give mannitol if
  needed.
\item
  Correct dehydration, hypoglycemia, and electrolyte imbalance.
\end{itemize}

\subsection{\texorpdfstring{\textbf{2. Empirical Antimicrobial
Therapy}}{2. Empirical Antimicrobial Therapy}}\label{empirical-antimicrobial-therapy}

Start antibiotics \textbf{immediately after blood and CSF samples} are
collected (or sooner if LP delayed).

\subsubsection{\texorpdfstring{\textbf{Empirical Antibiotic
Regimens:}}{Empirical Antibiotic Regimens:}}\label{empirical-antibiotic-regimens}

\begin{itemize}
\tightlist
\item
  \textbf{Neonates:} Ampicillin + Gentamicin or Cefotaxime (covering
  GBS, \emph{E. coli}, \emph{Listeria}).
\item
  \textbf{Infants and children:} Ceftriaxone or Cefotaxime ± Vancomycin
  (for \emph{S. pneumoniae} resistance).
\item
  \textbf{Suspected meningococcal infection:} Add high-dose Penicillin G
  or continue ceftriaxone.
\item
  \textbf{TB meningitis:} Standard anti-TB therapy (HRZE) with
  adjunctive corticosteroids.
\item
  \textbf{Fungal infections:} Amphotericin B or Fluconazole.
\item
  \textbf{Cerebral malaria:} Intravenous artesunate.
\end{itemize}

\subsubsection{\texorpdfstring{\textbf{Antiviral
Therapy:}}{Antiviral Therapy:}}\label{antiviral-therapy}

\begin{itemize}
\tightlist
\item
  For suspected HSV encephalitis → IV \textbf{Acyclovir} 10 mg/kg every
  8 hours for 14--21 days.
\end{itemize}

\section{Supportive Care}\label{supportive-care-2}

\begin{itemize}
\tightlist
\item
  \textbf{Antipyretics} for fever.
\item
  \textbf{Fluid management:} Maintain euvolemia; avoid fluid overload.
\item
  \textbf{Nutritional support:} Enteral feeding as soon as tolerated.
\item
  \textbf{Seizure control:} Maintain anticonvulsant therapy.
\item
  \textbf{Corticosteroids:} Dexamethasone 0.15 mg/kg every 6 hours for 4
  days in bacterial meningitis due to \emph{H. influenzae} or \emph{S.
  pneumoniae} (reduces hearing loss risk).
\item
  \textbf{Monitoring:} Vital signs, neurological status, urine output,
  and electrolyte balance.
\end{itemize}

\section{Complications}\label{complications-27}

CNS infections can lead to devastating outcomes if not promptly managed.

\textbf{Early complications:} - Seizures and status epilepticus

\begin{itemize}
\tightlist
\item
  Cerebral edema and herniation
\item
  Hydrocephalus
\item
  Subdural effusion or empyema
\item
  Shock and multi-organ failure
\end{itemize}

\textbf{Late complications:}

\begin{itemize}
\tightlist
\item
  Hearing impairment
\item
  Developmental delay and learning difficulties
\item
  Epilepsy
\item
  Vision loss
\item
  Motor deficits (paresis, ataxia)
\item
  Behavioral and cognitive problems
\end{itemize}

\section{Prevention}\label{prevention-17}

\subsection{\texorpdfstring{\textbf{Immunization}}{Immunization}}\label{immunization-1}

Vaccination remains the most effective preventive measure:

\begin{itemize}
\tightlist
\item
  \textbf{Haemophilus influenzae type b (Hib) vaccine}, introduced into
  Ghana's EPI, has drastically reduced cases.
\item
  \textbf{Pneumococcal conjugate vaccine (PCV13)} protects against
  \emph{Streptococcus pneumoniae}.
\item
  \textbf{Meningococcal vaccine} useful during outbreaks in northern
  Ghana.
\item
  \textbf{Measles and mumps vaccines} prevent postinfectious
  encephalitis.
\end{itemize}

\subsection{\texorpdfstring{\textbf{Chemoprophylaxis}}{Chemoprophylaxis}}\label{chemoprophylaxis}

Close contacts of meningococcal meningitis cases should receive
\textbf{Rifampicin}, \textbf{Ciprofloxacin}, or \textbf{Ceftriaxone} as
prophylaxis.

\subsection{\texorpdfstring{\textbf{Public Health
Measures}}{Public Health Measures}}\label{public-health-measures}

\begin{itemize}
\tightlist
\item
  Early detection and treatment of ear and respiratory infections.
\item
  Improved sanitation and reduced overcrowding.
\item
  Health education and prompt healthcare-seeking behavior.
\end{itemize}

\section{Prognosis}\label{prognosis-28}

The prognosis of CNS infections depends on:

\begin{itemize}
\tightlist
\item
  The causative organism
\item
  The age of the child
\item
  Speed of diagnosis and initiation of treatment
\item
  Availability of intensive care and rehabilitation
\end{itemize}

Mortality from bacterial meningitis in Africa remains between 15--30\%.
Survivors frequently suffer long-term sequelae, including hearing loss,
epilepsy, and neurodevelopmental delays.

Viral meningitis usually has a good prognosis, while HSV encephalitis
can cause permanent neurological impairment despite treatment. Cerebral
malaria remains a leading cause of childhood neurological disability in
endemic regions.

\section{Conclusion}\label{conclusion-28}

Central nervous system infections in children represent a critical
emergency that demands prompt recognition and treatment. In Ghana and
other tropical settings, bacterial meningitis, cerebral malaria, and
viral encephalitis are the significant causes. Improved immunization
coverage, early diagnosis, and effective antimicrobial therapy are
essential to reduce mortality and prevent neurological sequelae.
Strengthening laboratory capacity and surveillance systems will further
enhance early detection and appropriate management.

\chapter{Cerebral Palsy}\label{cerebral-palsy}

\section{Introduction}\label{introduction-39}

\subsection{Definition}\label{definition-19}

Cerebral palsy describes a group of permanent disorders of movement
and/or posture resulting from non-progressive (static) disturbances
(insult/injury) to the developing foetal or infant brain.

Although the injury is non-progressive (static), the neurological
manifestations evolve.

Disturbances of sensation, perception, cognition, communication, and
behaviour, by epilepsy, and by secondary musculoskeletal problems, often
accompany the motor disorders of cerebral palsy. Cerebral palsy was
first described by Dr William Little in 1860. It is also known as
Little's disease, cerebral paralysis, or static encephalopathy. It is
usually evident by the time a child is 2 years old. The incidence is
approximately 2.2 per 1,000 live births in well-resourced countries. The
incidence is higher in lower- and middle-income countries (LMIC).
Cerebral palsy is the most common cause of physical disability in
children. It is associated with multiple comorbidities.

\subsection{Classification of cerebral
palsy}\label{classification-of-cerebral-palsy}

Cerebral palsy is traditionally classified based on the nature of the
motor disorder (tone abnormality) and its topographical distribution.
Using the motor disorder classification, cerebral palsy is categorized
as spastic (pyramidal), dyskinetic (extrapyramidal or athetoid), ataxic,
hypotonic, or mixed. Using the topographic distribution, cerebral palsy
is classified as quadriplegia, diplegia, hemiplegia, triplegia, and
monoplegia.

Recent classifications of cerebral palsy describe the functional
assessment of motor abilities using an objective scale. Examples
include:

\begin{itemize}
\tightlist
\item
  Gross Motor Functional Classification Scale (GMFCS) for gross motor
  function
\item
  Manual Ability Classification System (MACS) for upper limb function
\item
  Communication Functional Classification System (CFCS) for
  communication
\item
  Eating and Drinking Ability Classification System (EDACS) for eating
  and Drinking
\end{itemize}

\subsection{Aetiology and Risk
Factors}\label{aetiology-and-risk-factors-2}

The risk factors and aetiologies of CP can be grouped based on the
timing of the insult as genetic, preconceptional, prenatal, perinatal,
or postnatal.

Genetic causes/risk factors include

\begin{itemize}
\tightlist
\item
  Chromosomal syndromes
\item
  Single gene or microdeletion syndromes
\end{itemize}

Preconceptional risk factors include:

\begin{itemize}
\tightlist
\item
  Maternal illness---seizures, intellectual disability, thyroid disease,
  iodine deficiency
\item
  Maternal history of stillbirth or neonatal death
\item
  Maternal age over 40
\item
  Low socio-economic status
\end{itemize}

Prenatal risk factors

\begin{itemize}
\tightlist
\item
  Intrauterine infections (TORCHES)
\item
  Placental abnormalities
\item
  Bleeding in the second or third trimesters
\item
  Eclampsia/pre-eclampsia
\end{itemize}

Chorioamnionitis

\begin{itemize}
\tightlist
\item
  Maternal drug use, e.g., alcohol, cocaine, cigarette
\item
  Oligo/polyhydramnios
\item
  Intrauterine growth restriction
\item
  Multiple pregnancy
\end{itemize}

Perinatal risk factors

\begin{itemize}
\tightlist
\item
  Foetal distress
\item
  Difficult deliveries
\item
  Breech delivery
\item
  Prolonged labour
\item
  Instrumental deliveries incl.~emergency caesarean section
\item
  Meconium aspiration
\item
  Birth asphyxia à Hypoxic ischaemic encephalopathy
\item
  Birth injuries affecting the brain
\end{itemize}

Postnatal risk factors

\begin{itemize}
\tightlist
\item
  Prematurity
\item
  Low birth weight
\item
  Respiratory distress syndrome
\item
  Hypoglycaemia
\item
  Kernicterus à bilirubin-induced neurological dysfunction
\item
  Neonatal seizures
\item
  Infections, e.g., meningitis, encephalitis
\item
  Trauma à head injuries
\item
  Child abuse, e.g., shaken baby syndrome
\item
  Strokes
\item
  Submersion injuries
\end{itemize}

\section{Types of cerebral palsy}\label{types-of-cerebral-palsy}

\subsection{Spastic Cerebral Palsy}\label{spastic-cerebral-palsy}

This is the most common form of CP, accounting for about 70\% of cases.
It results from injury to the corticospinal (pyramidal) tract or the
motor cortex. The features are usually not present at birth, but develop
within the first 2 years of life, and include:

\begin{itemize}
\tightlist
\item
  Delayed motor milestones
\item
  Hypertonia (spasticity)
\item
  Hyperreflexia
\item
  Seizures
\item
  Intellectual/learning disability
\end{itemize}

Spastic CP is further classified based on the anatomical distribution
as:

\begin{itemize}
\tightlist
\item
  Monoplegia
\item
  Diplegia
\item
  Hemiplegia
\item
  Triplegia
\item
  Quadriplegia
\end{itemize}

\subsubsection{Spastic hemiplegia}\label{spastic-hemiplegia}

This is when the weakness is on one side of the body. On the affected
side, the upper limb is usually more severely affected than the lower
limb. It results from focal pathology in the cerebral cortex, often due
to cerebral malformations or vascular causes such as intrauterine
haemorrhage.

Early manifestations of spastic hemiplegia include:

\begin{itemize}
\tightlist
\item
  Fisting on the affected side
\item
  Early handedness (lateralization before 1 year of age)
\item
  Delayed sitting (falls over as affected leg hyper-extends)
\item
  The child does not bear weight on the affected side when held upright
\item
  May not be recognized until 5-6 months of age or later
\end{itemize}

Late manifestations of spastic hemiplegia include:

\begin{itemize}
\tightlist
\item
  Spastic hemiplegic CP is the most ambulatory form of CP. They
  typically walk independently by the age of 3.
\item
  Tiptoe walking in the affected foot
\item
  Hemiplegic gait/hemiparesis
\item
  Seizures may occur in 50-60\% of patients, usually in the first 2
  years of life.
\item
  Intelligence may not be affected
\end{itemize}

\subsubsection{Spastic diplegia}\label{spastic-diplegia}

This affects all four limbs, but the lower limbs are severely affected,
leaving the upper limbs relatively spared. Common etiologic or risk
factors are prematurity and hypoxic-ischemic encephalopathy in the
preterm infant. The pathological finding in the brain is periventricular
leukomalacia.

Early manifestations in infants with spastic diplegia:

\begin{itemize}
\tightlist
\item
  They are usually alert and have good socialization
\item
  Their hands open (no fisting)
\item
  They have a normal tone (or even hypotonia) during the first 4 months
\item
  Delayed motor milestones (They delay in sitting and often extend their
  legs when pulled to sit).
\end{itemize}

Late manifestations of spastic diplegia:

\begin{itemize}
\tightlist
\item
  They later develop increased tone in the lower limbs (especially in
  the hip adductors, hamstrings, and gastrocnemius)
\item
  Also, increased deep tendon reflexes, clonus, and Babinski sign in the
  lower limbs.
\item
  Commando crawling, bottom shuffling, or rolling movement
\item
  Scissoring gait and tiptoeing when pulled to stand
\item
  Delayed walking
\item
  Hip subluxation in children with severe lower limb spasticity
\item
  Usually normal intelligence
\item
  Usually no seizures
\end{itemize}

\subsubsection{Spastic quadriplegia}\label{spastic-quadriplegia}

In this type of cerebral palsy, all four limbs are affected. It also
affects the face, neck, and trunk. It is caused by diffuse brain injury,
such as:

\begin{itemize}
\tightlist
\item
  Hypoxic-ischemic damage in a term infant
\item
  Intrauterine disease
\item
  Cerebral malformations
\end{itemize}

The typical pathological findings are watershed infarcts.

Early manifestations of spastic quadriplegic cerebral palsy are

\begin{itemize}
\tightlist
\item
  Poor socialization
\item
  Poor neck control
\item
  Infantile reflexes (Moro and tonic-neck) are obligatory, stereotyped,
  and persist after age 6 months
\item
  The patient may be hypotonic in infancy and later evolve into
  spasticity.
\item
  Cortical fisting in both hands
\end{itemize}

Late manifestations of spastic quadriplegia include:

\begin{itemize}
\tightlist
\item
  Severe to profound global developmental
\item
  Microcephaly
\item
  Seizures are common
\item
  Severe intellectual disability
\item
  Diffused increased tone
\item
  Increased deep tendon reflexes, clonus, Babinski sign
\item
  Supranuclear bulbar palsy (dysphagia, dysarthria) presenting as
  drooling and recurrent aspirations.
\item
  They may never walk or sit alone
\item
  They tend to have cortical visual impairment, disturbances in ocular
  motility, and hearing impairment.
\end{itemize}

Less common forms of spastic cerebral palsy are spastic triplegia and
spastic monoplegia.

Figure~\ref{fig-CerebralPalsyTypes} summarises the types of spastic
cerebral palsy.

\begin{figure}

\centering{

\pandocbounded{\includegraphics[keepaspectratio]{images/neu-ceberal-palsy-distributions.jpg}}

}

\caption{\label{fig-CerebralPalsyTypes}Types of spastic cerebral palsy}

\end{figure}%

\subsection{Dyskinetic cerebral palsy}\label{dyskinetic-cerebral-palsy}

This is also called extrapyramidal/choreoathetoid cerebral palsy. Common
aetiologies for this form of cerebral palsy include kernicterus and
sudden hypoxic-ischaemic episodes, as in uterine rupture or placenta
abruptio. The brain pathology shows damage in the extrapyramidal system
and the basal ganglia.

Early manifestations of dyskinetic cerebral palsy include:

\begin{itemize}
\tightlist
\item
  No choreoathetosis in the first 2 years of life
\item
  Often hypotonic, but sometimes presents with fluctuating muscle tone
\item
  Delayed motor milestones
\item
  Poor neck control
\item
  Sensorineural deafness
\item
  Normal socialization
\end{itemize}

Late manifestations include:

\begin{itemize}
\tightlist
\item
  Movement disorders: Choreoathetosis, dystonia
\item
  Symmetrical distribution
\item
  Dental enamel dysplasia
\item
  Difficulty with speech (dysarthria)
\item
  Swallowing difficulty leading to drooling
\item
  Affected children may or may not walk independently
\item
  Intelligence can be normal
\item
  Seizures are not common
\end{itemize}

\subsection{Ataxic cerebral palsy}\label{ataxic-cerebral-palsy}

This results from damage to the cerebellum. Patients experience problems
with balance and deep perception, presenting with an unsteady gait and
difficulty with movement that requires a lot of control, such as
writing. Their muscle tone may be increased or decreased.

\subsection{Mixed cerebral palsy}\label{mixed-cerebral-palsy}

Patients with mixed cerebral palsy have more than one type of cerebral
palsy, usually a mixture of spasticity and athetoid movements, with
tight muscle tone and involuntary reflexes. In mixed CP, different parts
of the brain are affected.
Figure~\ref{fig-TopographicalDistributionCerebralPalsy} shows the
topographical and physiologic classification of cerebral palsy.

\begin{figure}

\centering{

\pandocbounded{\includegraphics[keepaspectratio]{images/neu-physiologic-cerebral-palsy.jpg}}

}

\caption{\label{fig-TopographicalDistributionCerebralPalsy}Topographical
and physiologic classification of cerebral palsy}

\end{figure}%

\section{Comorbidities of cerebral
palsy}\label{comorbidities-of-cerebral-palsy}

These are conditions that occur with greater frequency in children with
cerebral palsy than in the general population. They include:

\begin{itemize}
\tightlist
\item
  Developmental delays
\item
  Seizures
\item
  Ophthalmologic/visual abnormalities including Cortical visual
  impairment, Disorders of ocular motility, Refractive errors, and Optic
  atrophy
\item
  Hearing impairment
\item
  Speech defects, including delayed speech, poor articulation, loss of
  voice modulation
\item
  Learning/intellectual disability
\item
  Feeding problems/swallowing difficulties
\item
  Aspiration pneumonia
\item
  Gastroesophageal reflux
\item
  Gait abnormalities
\item
  Contractures
\item
  Hip subluxation
\item
  Failure to thrive/neglect/abuse
\item
  Behavioural and emotional problems such as ADHD, depression, and ASD
\end{itemize}

\section{Diagnosis}\label{diagnosis-12}

The diagnosis of CP is clinical. It is based on the constellation of
symptoms and signs in the affected child. Special investigations have a
limited role in confirming the diagnosis but may contribute to
determining the aetiology and the timing of the insult. Investigations
may also help to exclude differential diagnoses and to identify
comorbidities.

Investigations that may be employed include various modalities of
neuroimaging, such as transcranial USG, brain CT scan, and brain MRI

\textbf{Transcranial ultrasound}: This is useful in the neonate up to 6
months, for the detection of large structural abnormalities

\textbf{Brain CT Scan/MRI}: These give a better definition of
structures. The CT is used in children \textgreater1 year old due to the
risk associated with radiation in younger children, while the MRI is
helpful at any age. However, the MRI is more challenging to perform due
to its limited availability, high cost, and the prolonged sedation
required.

\section{Features that suggest a progressive CNS disorder rather than
CP}\label{features-that-suggest-a-progressive-cns-disorder-rather-than-cp}

\begin{itemize}
\tightlist
\item
  Abnormally increasing head circumference: think hydrocephalus, tumour,
  leukodystrophies
\item
  Eye abnormalities such as cataract, retinal pigmentary degeneration,
  optic atrophy: think neurodegenerative disease
\item
  Skin abnormalities such as hypopigmentation, café-au-lait spots, nevus
  flammeus, etc: think neurocutaneous disorders, e.g., Sturge-Weber
  syndrome, neurofibromatosis, etc.
\item
  Hepatomegaly with or without splenomegaly: think storage disease
\item
  Sensory abnormalities: think peripheral nerve disorders.
\end{itemize}

Figure~\ref{fig-ddCerebralPalsy} lists a few examples of disorders that
are sometimes misdiagnosed as cerebral palsy

\begin{figure}

\centering{

\pandocbounded{\includegraphics[keepaspectratio]{images/neu-dd-cerebral-palsy.jpg}}

}

\caption{\label{fig-ddCerebralPalsy}Disorders that are sometimes
misdiagnosed as cerebral palsy}

\end{figure}%

\section{Management of the child with cerebral
palsy}\label{management-of-the-child-with-cerebral-palsy}

In the management of the child with cerebral palsy, the main goals of
treatment should be to help the child reach their full potential by:

\begin{enumerate}
\def\labelenumi{\arabic{enumi}.}
\tightlist
\item
  Maximizing mobility through physical therapy
\item
  Providing physical support using aids such as splints, walkers, and
  wheelchairs
\item
  Speech and occupational therapy
\item
  Surgery to correct abnormalities, improve mobility, and reduce
  spasticity
\item
  Special educational services
\end{enumerate}

It is essential to adopt a multidisciplinary approach in the management
of .the child with cerebral palsy. In contrast, the child remains under
the care of one paediatrician, usually a developmental paediatrician.

\begin{figure}

\centering{

\pandocbounded{\includegraphics[keepaspectratio]{images/neu-cerebral-palsy-mdt-3.jpg}}

}

\caption{\label{fig-mdtcerebralPalsy}MDT management of cerebral palsy}

\end{figure}%

The role of the multidisciplinary team (MDT) members in the management
of cerebral palsy includes:

\begin{enumerate}
\def\labelenumi{\arabic{enumi}.}
\tightlist
\item
  Developmental paediatrician -- monitors a child's development and
  coordinates multidisciplinary care for the patient.
\item
  Neurologist -- management of neurological disorders, including
  seizures and movement disorders.
\item
  Audiologist -- hearing assessment
\item
  Speech therapist --assessment of speech and swallowing
\item
  Ophthalmologist -- visual defects
\item
  Nutritionist/Dietician -- growth failure, nutritional deficiencies
\item
  Specialist teachers
\item
  Physiotherapist -- addressing spasticity, posture, and gait
  abnormalities, among others.
\item
  Orthopaedic surgeon -- structural deformities, contractures,
  scoliosis.
\item
  Neurosurgeon -- surgical management of spasticity and scoliosis.
\item
  Occupational therapist -- difficulties with fine movements and
  activities of daily living (ADLs)
\item
  Clinical psychologist
\item
  Social/community worker
\end{enumerate}

\subsection{Management of spasticity}\label{management-of-spasticity}

Spasticity may be focal or generalised, and treatment modalities are
either reversible or permanent. For generalised spasticity, some
reversible modalities used are oral therapy (baclofen, diazepam, etc.)
and intrathecal baclofen pumps. Permanent modalities for generalised
spasticity include deep brain stimulation and selective dorsal
rhizotomy.

For focal spasticity, botulinum toxin A injection provides reversible
relief, whereas focal surgeries, such as tendon release, provide
permanent relief.

These are summarised in Figure~\ref{fig-SpasticityMgt} below.

\begin{figure}

\centering{

\pandocbounded{\includegraphics[keepaspectratio]{images/neu-SpasticityMgt.jpg}}

}

\caption{\label{fig-SpasticityMgt}Management of spasticity in cerebral
palsy}

\end{figure}%

\section{Preventive strategies for cerebral
palsy}\label{preventive-strategies-for-cerebral-palsy}

\begin{enumerate}
\def\labelenumi{\arabic{enumi}.}
\tightlist
\item
  Antenatal measures

  \begin{itemize}
  \tightlist
  \item
    Use of magnesium sulphate for neuroprotection
  \item
    Use of antenatal corticosteroids in anticipated preterm deliveries
  \item
    Tocolysis
  \item
    Measures to prevent preterm births
  \item
    Appropriate antibiotic use in PROM
  \item
    Vaccination against rubella and congenital infections
  \item
    Regular ANC check-ups
  \item
    Avoid alcohol, tobacco, and drugs
  \item
    Maintaining a healthy lifestyle for the mother
  \end{itemize}
\item
  Perinatal/neonatal care

  \begin{itemize}
  \tightlist
  \item
    Early detection and management of neonatal care
  \item
    Neuroprotection with moderate hypothermia for newborns with HIE
  \item
    Avoidance of unnecessary oxygen supplementation
  \item
    Early detection and treatment of neonatal hypoglycaemia
  \end{itemize}
\item
  Postnatal and childhood care

  \begin{itemize}
  \tightlist
  \item
    Measures to prevent head injury, including the use of baby care
    seats, the prevention of falls, and shaking.
  \item
    Genetic screening and counselling for families with cerebral palsy
  \end{itemize}
\end{enumerate}

\section{Current innovations in the management of cerebral
palsy}\label{current-innovations-in-the-management-of-cerebral-palsy}

Systemic hypothermia: Controlled medical cooling of the body's core
temperature may protect the brain and decrease the rate of death and
disability from brain injuries. Hypothermia is effective in treating
neurologic symptoms in babies with hypoxic-ischemic encephalopathy (HIE)

Stem cell therapy is being investigated as a potential treatment for
cerebral palsy. Stem cells are capable of differentiating into various
cell types within the body. Scientists are hopeful that stem cells may
be able to repair damaged nerves and brain tissues. Clinical studies are
examining the safety and tolerability of umbilical cord blood stem cell
infusion in children with cerebral palsy.

\section{Prognosis}\label{prognosis-29}

Cerebral palsy is not a progressive condition, and so living into old
age is possible. Regression or worsening of long-term symptoms is not
characteristic. Prognosis varies according to the severity of the
disorder. The lifespan of patients with cerebral palsy is usually
reduced by complications such as reduced mobility, feeding difficulties,
respiratory infections, and epilepsy. In the US, the average life span
of patients with cerebral palsy was reported as 35 years in 2008. ~

\chapter{Seizure Disorders}\label{seizure-disorders}

\section{Introduction}\label{introduction-40}

The term seizure vaguely refers to anything that ``seizes'' or ``takes
hold'' of a person. These may be epileptic or non-epileptic events. In
this chapter, unless otherwise specified, the term seizure is used to
refer to epileptic events.

\section{Definitions}\label{definitions-2}

\subsection{Seizures}\label{seizures}

The International League Against Epilepsy (ILAE) defines epileptic
seizure as a \emph{transient occurrence of signs and/or symptoms due to
abnormal, excessive, or synchronous neuronal activity in the brain.}
These may manifest as paroxysmal motor, sensory, autonomic, and/or
behavioral or cognitive function abnormalities or impaired
consciousness.

\subsection{Convulsions}\label{convulsions}

These are the motor manifestations of a seizure. These include tonic
(stiffening), clonic (jerking), myoclonic (massive jerking), vibratory
(trembling), or hypermotor (thrashing about). Seizures with no motor
manifestations are termed non-convulsive and include motor arrest, e.g.,
unresponsive stare or drop attacks. Sensory disturbances during a
seizure may include visual, auditory, or tactile disturbances. Some
patients may describe changes in smell or taste.

Non-epileptic seizures in children include many events such as cardiac
syncope, vasovagal syncope, breath-holding spells, infantile
gratification, shuddering spells, etc. These are sometimes referred to
as seizure mimics.

\section{Classification of Seizures}\label{classification-of-seizures}

The ILAE published its most recent classification of seizures in 2017.
In this classification, seizures are broadly classified based on their
onset within the brain as focal, generalized, or unknown onset.

\begin{itemize}
\tightlist
\item
  Focal-onset seizures originate from a focus within one hemisphere.
\item
  Generalized-onset seizures originate from both hemispheres.
\item
  Unknown onset -- where the onset is not known at the time of the
  evaluation.
\end{itemize}

\textbf{Focal-onset seizures} (previously known as partial seizures) are
further subclassified based on whether awareness is preserved or lost
during the seizure and secondary generalization.

\begin{itemize}
\tightlist
\item
  Focal seizures with intact awareness (or focal aware seizures): These
  are focal seizures in which awareness is preserved. This was
  previously known as simple partial seizures.
\item
  Focal seizures with loss of awareness (or focal unaware seizures):
  These are focal seizures in which awareness is lost during the event.
  They were previously referred to as complex partial seizures.
\item
  Focal seizures evolving into bilateral tonic-clonic seizures: These
  are focal seizures that go on to become generalized. They were
  previously referred to as focal seizures with secondary
  generalization.
\end{itemize}

\textbf{Generalised-onset seizures} are further subclassified based on
their manifestations as:

\begin{itemize}
\tightlist
\item
  Clonic seizures: having repetitive jerky movements.
\item
  Tonic seizures: characterised by increased tone.
\item
  Tonic-clonic seizures: They have two phases: a tonic phase and a
  clonic phase.
\item
  Tonic-clonic-tonic, or other combinations
\item
  Atonic seizures: These are characterized by loss of muscle tone,
  leading to drop attacks.
\item
  Myoclonic seizures: These are characterized by sudden jerks of a group
  of muscles.
\item
  Absence seizures: These are characterized by blank stares during which
  the patient loses awareness. They may be associated with lip-smacking
  or eyelid fluttering. A typical absence seizure starts abruptly, lasts
  about 5-15 seconds, and ends abruptly with no postictal events. An
  atypical absence may last \textgreater15 seconds or may have a slow
  recovery.~
\item
  Epileptic spasms: These are characterized by repetitive flexor (or
  extensor) jerks of the limbs and trunk, occurring in clusters.
\end{itemize}

\begin{figure}

\centering{

\pandocbounded{\includegraphics[keepaspectratio]{images/neu-seizure-classification.jpg}}

}

\caption{\label{fig-SeizureClassification}ILAE classification of
seizures}

\end{figure}%

\section{Febrile seizures}\label{febrile-seizures}

\subsection{Definition}\label{definition-20}

A febrile seizure is accompanied by fever in a child aged 6 months to 6
years without intracranial infection/inflammation {[}ref{]}. It affects
about 3\% of all children between 6 months and 6 years old.

Biological Basis

The exact mechanism leading to febrile seizures is not clearly
understood; biology is related to the immature brain in young children,
fever, and genetic predisposition. The fever activates the cytokine
networks (IL-1 alpha, IL-1 beta), increasing neuronal excitability.
There is a complex inheritance involving multiple genes and
environmental factors for children with genetic predisposition.
Monozygotic twins have a higher concordance rate than dizygotic twins.

The risk for a first febrile seizure in a child is increased in those
with

\begin{itemize}
\tightlist
\item
  Delayed neonatal hospital discharge
\item
  Slow neurological development as judged by the parent
\item
  Family history of febrile seizures (especially in a first-degree
  relative)
\item
  Attendance at day care
\end{itemize}

\subsection{Etiology}\label{etiology-2}

Common causes of febrile seizures include malaria, URTI, otitis media,
pharyngitis, transient viral infections, and gastroenteritis. Some
children develop a fever following vaccination with whole-cell DPT or
measles vaccine.

\subsection{Classification}\label{classification-5}

Febrile seizures are classified as simple or complex.
Table~\ref{tbl-simple-complex-seizure} below shows the differences
between simple and complex febrile seizures.

\begin{longtable}[]{@{}
  >{\raggedright\arraybackslash}p{(\linewidth - 4\tabcolsep) * \real{0.1961}}
  >{\raggedright\arraybackslash}p{(\linewidth - 4\tabcolsep) * \real{0.4216}}
  >{\raggedright\arraybackslash}p{(\linewidth - 4\tabcolsep) * \real{0.3824}}@{}}
\caption{Simple versus Complex Febrile
Seizures}\label{tbl-simple-complex-seizure}\tabularnewline
\toprule\noalign{}
\begin{minipage}[b]{\linewidth}\raggedright
Feature
\end{minipage} & \begin{minipage}[b]{\linewidth}\raggedright
Simple febrile seizure
\end{minipage} & \begin{minipage}[b]{\linewidth}\raggedright
Complex febrile seizure
\end{minipage} \\
\midrule\noalign{}
\endfirsthead
\toprule\noalign{}
\begin{minipage}[b]{\linewidth}\raggedright
Feature
\end{minipage} & \begin{minipage}[b]{\linewidth}\raggedright
Simple febrile seizure
\end{minipage} & \begin{minipage}[b]{\linewidth}\raggedright
Complex febrile seizure
\end{minipage} \\
\midrule\noalign{}
\endhead
\bottomrule\noalign{}
\endlastfoot
Seizure type & Generalized & Maybe focal \\
Duration & Brief (\textless15 min) & Prolonged \\
Number of episodes & Single episode during the febrile illness &
Repeated episodes in the same illness \\
Outcomes & Do not cause brain damage & Increased risk of brain damage \\
\end{longtable}

\subsection{Risk of recurrence}\label{risk-of-recurrence}

In children with febrile seizures, the risk factors for recurrence are

\begin{itemize}
\tightlist
\item
  Young age at the time of the first febrile seizure (\textless{} 15
  months)
\item
  Family history of febrile seizures (first-degree relative)
\item
  Low temperature at the time of the seizure (\textless{} 40 degrees)
\item
  Short duration of illness before the seizure ~~~~~~~~ {[}Berg et al,
  1997{]}
\end{itemize}

Having a complex febrile seizure and neurologic dysfunction are not
consistent predictors of recurrence.

\subsection{Risk of subsequent
epilepsy}\label{risk-of-subsequent-epilepsy}

The risk of epilepsy in children with febrile seizures is slightly
higher than the incidence in the general population. The risk is
increased with:

\begin{itemize}
\tightlist
\item
  Complex febrile seizures (focal, prolonged, repeated within a single
  illness)
\item
  Developmental delay or neurologic dysfunction
\item
  Family history of epilepsy~
  ~~~~~~~~~~~~~~~~~~~~~~~~~~~~~~~~~~~~~~~~~~~~~~~~~~~~~~~~~~~~~~~~~~~
\end{itemize}

The number of febrile seizures is not a predictor of subsequent
epilepsy.

\subsection{Long-term cognitive and behavioural
outcomes}\label{long-term-cognitive-and-behavioural-outcomes}

Febrile seizures are ``benign''. Studies have shown that children with
febrile seizures have the same academic progress, intellect, and
behaviour as other children. ~

\section{Acute scenarios: prolonged seizures versus status
epilepticus}\label{acute-scenarios-prolonged-seizures-versus-status-epilepticus}

\subsection{Definitions:}\label{definitions-3}

\textbf{Prolonged seizures:} Most convulsive seizures in children are
brief (lasting \textless{} 3 minutes). However, some may go on beyond 5
minutes. These are termed prolonged seizures.

\textbf{Status epilepticus}, on the other hand, is a prolonged seizure
lasting more than 30 minutes or multiple seizures without recovery of
consciousness. This definition applies to convulsive seizures. The
duration of non-convulsive seizures differs based on the seizure type.

\subsection{Pathophysiology of status
epilepticus}\label{pathophysiology-of-status-epilepticus}

In the initial seizure phase, the body undergoes autoregulation, leading
to increased heart rate, cardiac output, and cerebral perfusion. This
ensures that there is increased oxygen and glucose delivery to the
brain. Also, increased cerebral perfusion helps remove carbon dioxide
and metabolic waste, which build up in the brain during seizures. Beyond
30 minutes, this autoregulation breaks down, reducing cardiac output,
decreasing systemic blood pressure, and decreasing cerebral perfusion.
In the end, there is decreased oxygenation (hypoxia) and buildup of
metabolic waste, which trigger a cascade of events with resultant
neuronal deaths. This is why it is essential to stop any seizure from
progressing into a status epilepticus.

\begin{figure}

\centering{

\pandocbounded{\includegraphics[keepaspectratio]{images/neu-seizure-timing.jpg}}

}

\caption{\label{fig-SeizureTiming}Timing of seizures in the acute
scenario and the clinical implications}

\end{figure}%

\subsection{The Do's and Don'ts in acute
scenarios}\label{the-dos-and-donts-in-acute-scenarios}

In the first few minutes of a seizure, it is essential to undertake
basic first aid measures to protect the patient. These are termed the
\textbf{Do's} and include:

\begin{itemize}
\tightlist
\item
  Protect the person from injury (eg, remove harmful objects from
  nearby, cushion their heads, etc.)
\item
  When the seizure is over, gently place the patient in recovery. This
  positioning aids breathing by ensuring the patient does not choke on
  their secretions.
\item
  Stay with the patient until the seizure is over.
\item
  Stay calm and reassuring.
\end{itemize}

Some traditional practices can be harmful and should be discouraged.
These are referred to as the \textbf{Don'ts} and include:

\begin{itemize}
\tightlist
\item
  Do not restrain the person's movements. Forcefully restraining their
  movement does not stop the seizures and may instead result in needless
  injuries like fractures and dislocations.~
\item
  Do not put anything in their mouth. This could lead to the patient
  biting on the spoon or spatula and causing damage to their teeth,
  gums, or palate. Also, please do not put your finger in their mouth. A
  forceful bite on your figure can lead to severe injuries, including
  amputation.
\item
  Do not try to move them unless they are in danger.
\item
  Do not give them anything to eat or drink until they fully recover.
\item
  Do not attempt to bring them around. Practices such as pouring cold
  water on the patient, smearing them with garlic or other noxious
  substances are harmful and do not stop the seizures.~
\end{itemize}

\textbf{Indications for bringing the patient to the emergency room}
include:

\begin{enumerate}
\def\labelenumi{\arabic{enumi}.}
\tightlist
\item
  If this is the person's first seizure.
\item
  If the seizure continues for more than five minutes.
\item
  If one seizure follows another without the person regaining
  consciousness between seizures.
\item
  If the person is injured.
\item
  If you believe the person needs urgent medical attention.
\end{enumerate}

\subsection{General principles of management of Prolonged Seizures and
Status
Epilepticus}\label{general-principles-of-management-of-prolonged-seizures-and-status-epilepticus}

In the first 5 minutes, critical interventions should include:

\begin{itemize}
\item
  General emergency measures including

  \begin{itemize}
  \tightlist
  \item
    A = airway
  \item
    B = breathing
  \item
    C = circulation
  \item
    D = disability (such as hypoglycemia)
  \item
    E = exposure
  \end{itemize}
\item
  Check blood glucose and correct hypoglycemia, if present.
\item
  Remember the Do's and Don'ts.
\end{itemize}

If the seizure persists after 5 minutes, initiate drug management.

\begin{itemize}
\tightlist
\item
  1\textsuperscript{st} line drugs are benzodiazepines: these commonly
  include diazepam (rectal/IV, midazolam (buccal/nasal), and lorazepam
  (IV/rectal). These may be repeated after 10 minutes if the seizure
  persists.
\item
  2\textsuperscript{nd} line drugs include phenobarbital (IV), phenytoin
  (IV), valproate (IV), or levetiracetam (IV).
\item
  3\textsuperscript{rd} line drugs include anesthetic agents and should
  be given in the intensive care unit or in a center where the patient's
  breathing can be supported. They include thiopentone, propofol, and
  ketamine. Alternatively, continuous infusions of midazolam or repeated
  intravenous boluses of phenobarbitone may be used.
\end{itemize}

This stepwise approach to the management of ongoing seizures is
illustrated in Table~\ref{tbl-stepwise-epilepticus} below:

\begin{longtable}[]{@{}
  >{\raggedright\arraybackslash}p{(\linewidth - 4\tabcolsep) * \real{0.2424}}
  >{\raggedright\arraybackslash}p{(\linewidth - 4\tabcolsep) * \real{0.3737}}
  >{\raggedright\arraybackslash}p{(\linewidth - 4\tabcolsep) * \real{0.3737}}@{}}
\caption{Stepwise approach to status
epilepticus}\label{tbl-stepwise-epilepticus}\tabularnewline
\toprule\noalign{}
\begin{minipage}[b]{\linewidth}\raggedright
\textbf{Time}
\end{minipage} & \begin{minipage}[b]{\linewidth}\raggedright
\textbf{Action required}
\end{minipage} & \begin{minipage}[b]{\linewidth}\raggedright
\textbf{Notes}
\end{minipage} \\
\midrule\noalign{}
\endfirsthead
\toprule\noalign{}
\begin{minipage}[b]{\linewidth}\raggedright
\textbf{Time}
\end{minipage} & \begin{minipage}[b]{\linewidth}\raggedright
\textbf{Action required}
\end{minipage} & \begin{minipage}[b]{\linewidth}\raggedright
\textbf{Notes}
\end{minipage} \\
\midrule\noalign{}
\endhead
\bottomrule\noalign{}
\endlastfoot
t = 0 (seizure onset) & \begin{minipage}[t]{\linewidth}\raggedright
Ensure patient's safety (remember the Do's and Don'ts) Check the ABCDs.

\begin{itemize}
\tightlist
\item
  Ensure airway patency.
\item
  Monitor vitals (HR, RR, BP, SpO2)
\item
  If SpO2 \textless90\%, give supplemental O2
\item
  Secure venous access
\item
  Check RBS. If there is hypoglycemia, correct with dextrose.
\end{itemize}
\end{minipage} & Most seizures will stop on their own within 3 minutes
of onset. \\
t = 5 min & Start 1st line pharmacologic treatment with a
benzodiazepine. & \begin{minipage}[t]{\linewidth}\raggedright
Commonly used benzodiazepines include:

\begin{enumerate}
\def\labelenumi{\arabic{enumi}.}
\tightlist
\item
  Diazepam (rectal or IV)
\item
  Midazolam (buccal or nasal)
\item
  Lorazepam (rectal or IV)
\end{enumerate}
\end{minipage} \\
t = 15 min & If seizure persists to time t = 15 min (10 minutes after
the first benzodiazepine), repeat the dose of benzodiazepine. &
Benzodiazepines may be repeated once only. Avoid \textgreater2 doses of
benzodiazepines. \\
t = 25 min & Start the 2nd line pharmacologic treatment. &
\begin{minipage}[t]{\linewidth}\raggedright
Commonly used 2nd line drugs are:

\begin{itemize}
\tightlist
\item
  Phenobarbitone (IV)
\item
  Phenytoin (IV)
\item
  Sodium valproate (IV)
\item
  Levetiracetam (IV)
\end{itemize}
\end{minipage} \\
t = 40-60 min & Start the 3rd line pharmacologic treatment. & This stage
employs anesthetic agents. It should be done in a center where the
patient's breathing can be supported. \\
\end{longtable}

\subsection{Prognosis of Status Epilepticus in
Children}\label{prognosis-of-status-epilepticus-in-children}

In patients who present with convulsive status epilepticus (CSE), there
is a mortality rate of 3-9\% within 30 days of the CSE. The mortality
rate is worse in low- and middle-income countries.

In those who survive, the neurological sequelae depend on the type and
duration of the seizure, the patient's age, and the underlying etiology.

\begin{itemize}
\tightlist
\item
  Duration of the SE -- the single most important determinant of
  prognosis. Worse outcome for prolonged SE.
\item
  Age -- worse neurological sequelae in infants (30\% vs.~6\% in older
  children)
\item
  Seizure type -- worse outcome for convulsive SE
\item
  Underlying etiology
\end{itemize}

\section{Epilepsy}\label{epilepsy}

\subsection{Definition:}\label{definition-21}

Epilepsy is a disease of the brain defined by any of the following
conditions:

\begin{enumerate}
\def\labelenumi{\alph{enumi}.}
\tightlist
\item
  At least two unprovoked seizures occurring \textgreater24 hours apart.
\item
  One unprovoked seizure and an increased probability of further
  seizures
\item
  Diagnosis of an epileptic syndrome
\end{enumerate}

It affects over 50 million people worldwide, and 1 in 200 children
worldwide.

\subsection{Classification}\label{classification-6}

The ILAE classifies Epilepsy into four main types based on the seizure
type (or types) {[}3{]}. These are:

\begin{itemize}
\tightlist
\item
  Focal epilepsy
\item
  Generalised epilepsy
\item
  Combined generalised and focal epilepsy
\item
  Epilepsy with unknown seizure type(s)
\end{itemize}

\subsection{Causes}\label{causes-5}

All causes of epilepsy can be grouped into six main categories as
follows:

\begin{enumerate}
\def\labelenumi{\alph{enumi}.}
\tightlist
\item
  Structural: e.g., lissencephaly, tumours, calcifications, post-stroke,
  tuberous sclerosis complex, etc.
\item
  Genetic: e.g., familial neonatal seizures, Dravet syndrome, etc.
\item
  Infections: e.g., neurocysticercosis, HIV, post-meningitis, etc.
\item
  Metabolic: e.g., hypoglycemia, electrolyte imbalance, inborn errors of
  metabolism, vitamin deficiency, etc.
\item
  Immune-mediated: autoimmune encephalitis
\item
  Unknown
\end{enumerate}

\begin{figure}

\centering{

\pandocbounded{\includegraphics[keepaspectratio]{images/neu-classification-of-epilapsy.jpg}}

}

\caption{\label{fig-epilapsyClassification}ILAE classification of
epilepsies}

\end{figure}%

\subsection{Clinical evaluation of the child with
epilepsy}\label{clinical-evaluation-of-the-child-with-epilepsy}

In evaluating a child with epilepsy, it is important to obtain a good
history, perform a thorough physical examination, and then form a
clinical diagnosis (or impression). Investigations are then employed to
define the seizure type(s), confirm the clinical diagnosis, identify the
underlying etiology, assess the effect of treatment, rule out
differential diagnoses, and evaluate comorbidities.

\textbf{History:} An eyewitness account is useful when taking the
history of a child with epilepsy. Ask the older child for an account of
the episodes. Home videos of the seizure episodes, if available, will
complement the history. Key aspects of the seizure history should
include:

\begin{itemize}
\tightlist
\item
  Age at onset: This is important in making an epilepsy syndrome
  diagnosis, as specific epilepsy syndromes have different ages at
  onset.
\item
  Patient's baseline neurological status: This may differ from patient
  to patient and becomes essential in clinical decision-making. For
  example, the approach to a 3-year-old previously healthy child
  presenting with seizures will be different from that of a 3-year-old
  known with cerebral palsy, which will also be different from another
  3-year-old known with HIV infection.
\item
  Seizure semiology: This is a detailed description of what happened
  during the seizure. It is useful to describe the seizure semiology in
  terms of the different stages, namely the aura (if any), the ictal
  event, and the postictal event.
\item
  The aura is often a sensory feeling that the patient experiences at
  the start of the seizure. It is seen in focal-onset seizures. Please
  note that younger children may not be able to describe an aura.
\item
  Description of the ictal event should include the seizure type (or
  types), non-motor manifestations (such as behavioral arrest, autonomic
  dysfunctions, etc.), seizure progression and duration, as well as
  timing of the events (e.g.~shortly after going to sleep, during sleep,
  on awakening, whilst watching television, etc.).
\item
  Post-ictal events: This is a description of what the patient does
  after the seizures have stopped. They include events such as falling
  into deep sleep, changes in behaviour or mood, transient focal
  weakness (Todd paralysis), etc. Please note that some seizure types
  (e.g., absence seizures) have abrupt recovery with no post-ictal
  phase. If there are post-ictal events, the duration should also be
  noted.
\item
  Developmental/cognitive outcome and co-morbidities: The seizure
  history should describe the patient's development and any cognitive
  fallouts or neurobehavioral comorbidities noted since the onset of the
  seizures.
\item
  Drug history: This should include a detailed list of all anti-seizure
  medications used in the past, their contribution to seizure control,
  and the reason for stopping the drug. Other long-term medications
  (including herbs) and known allergies should also be described.
\item
  Family history: If other family members have seizures, they should be
  described in detail, including their relation to the index patient,
  the seizure type(s), response to treatment, prognosis, etc.~
\end{itemize}

\textbf{Physical examination}

The purpose of the physical examination is to identify risk factors and
comorbidities associated with epilepsy and evaluate treatment
effectiveness. The initial examination should be comprehensive and
include anthropometry, a skin examination for neurocutaneous
manifestations, a detailed neurological assessment, a developmental
evaluation, and an examination of other systems.

\subsection{Investigations}\label{investigations-29}

It is important to remember that the diagnosis of epilepsy is mostly
clinical and that the clinician should not overrely on investigations.
However, investigations may be useful in diagnosing epilepsy syndrome,
identifying the underlying etiology, assessing the effect of treatment,
ruling out differential diagnoses, and evaluating comorbidities.

Basic investigations in the evaluation of the child with epilepsy
include:

\begin{itemize}
\tightlist
\item
  Blood counts
\item
  Serum electrolytes
\item
  Liver function and renal function tests
\end{itemize}

Advanced investigations include:

\begin{itemize}
\tightlist
\item
  Electroencephalogram (EEG)
\item
  Neuroimaging (CT, MRI, SPECT, PET)
\item
  Metabolic screening
\item
  Autoimmune antibody assays
\item
  Molecular genetic testing
\end{itemize}

\subsection{Treatment of Epilepsy}\label{treatment-of-epilepsy}

The various modalities for treating epilepsy include:

\begin{enumerate}
\def\labelenumi{\arabic{enumi}.}
\tightlist
\item
  Medical treatment -- traditional and newer Anti-seizure medications
  (ASMs)
\item
  Hormonal treatment -- ACTH or Steroids (Prednisolone) for epileptic
  spasms
\item
  Vitamins -- pyridoxine, folinic acid, and biotin for
  vitamin-responsive seizures
\item
  Surgical treatment
\item
  Dietary treatment (ketogenic diet)~
\end{enumerate}

\textbf{Medical treatment:} This uses traditional or newer Anti-Seizure
Medications (ASMs). Not all children with epilepsy require ASM therapy.
When needed, ASM selection should be based on seizure type, epilepsy
syndrome, and potential side effects. In a low-resource country,
availability and affordability should be considered in ASM selection.

Monotherapy is always preferred. However, a few patients will require
rational polypharmacy. It is essential to select ASM at the minimum
dosage that provides reasonable seizure control with minimal adverse
effects. ASM therapy may be discontinued after the patient has had 2
years of seizure freedom.

ASMs have a variety of side effects. Some are dose-related and others
are idiosyncratic (non-dose related)

Table~\ref{tbl-classification-of-epilepsy} below shows the traditional
anti-seizure medications and their specific indications.

\begin{longtable}[]{@{}
  >{\raggedright\arraybackslash}p{(\linewidth - 4\tabcolsep) * \real{0.1900}}
  >{\raggedright\arraybackslash}p{(\linewidth - 4\tabcolsep) * \real{0.4000}}
  >{\raggedright\arraybackslash}p{(\linewidth - 4\tabcolsep) * \real{0.4000}}@{}}
\caption{Traditional Anti-seizure medications
(ASMs)}\label{tbl-classification-of-epilepsy}\tabularnewline
\toprule\noalign{}
\begin{minipage}[b]{\linewidth}\raggedright
\textbf{Drug}
\end{minipage} & \begin{minipage}[b]{\linewidth}\raggedright
\textbf{Indications}
\end{minipage} & \begin{minipage}[b]{\linewidth}\raggedright
\textbf{Side Effects}
\end{minipage} \\
\midrule\noalign{}
\endfirsthead
\toprule\noalign{}
\begin{minipage}[b]{\linewidth}\raggedright
\textbf{Drug}
\end{minipage} & \begin{minipage}[b]{\linewidth}\raggedright
\textbf{Indications}
\end{minipage} & \begin{minipage}[b]{\linewidth}\raggedright
\textbf{Side Effects}
\end{minipage} \\
\midrule\noalign{}
\endhead
\bottomrule\noalign{}
\endlastfoot
Phenobarbitone & Status epilepticus

Neonatal seizures

Epilepsy (both focal and generalized seizures) & \textbf{Dose-related:}
Drowsiness, respiratory depression

\textbf{Idiosyncratic:} Rash, Stevens-Johnson syndrome \\
Phenytoin & Status epilepticus

Neonatal seizures

Peri-operative seizures & \textbf{Dose-related:} Drowsiness

\textbf{Chronic toxicity:} Gingival hyperplasia, coarse facial features,
hirsutism, neuropathy, megaloblastic anemia

\textbf{Idiosyncratic:} Rash, Stevens-Johnson syndrome, serum
sickness \\
Sodium valproate & Broad spectrum

Generalized seizures

Absence seizures & \textbf{Dose-related:} Intention tremor, weight gain,
polycystic ovaries, teratogenicity (risk of neural tube defects)

\textbf{Idiosyncratic:} Hepatic failure, pancreatitis \\
Carbamazepine & Focal seizures

Avoid in the absence and myoclonic seizures & \textbf{Dose-related:}
Dizziness, drowsiness, diplopia, ataxia

\textbf{Idiosyncratic:} Aplastic anemia, rash, Stevens-Johnson
syndrome \\
Ethosuximide & Absence seizures & \textbf{Dose-related:} Dizziness,
nausea, weight loss \\
\end{longtable}

Some newer ASMs include Lamotrigine, Topiramate, Levetiracetam,
Clonazepam, Clobazam, Vigabatrin (for the treatment of epileptic
spasms), Gabapentin, Pregabalin, Felbamate, Oxcarbazepine, Fosphenytoin,
Lacosamide, Stiripentol, Tiagabine, and Zonisamide.

\textbf{Hormonal therapy} in the treatment of epilepsy includes:

\begin{itemize}
\tightlist
\item
  Adrenocorticotropic hormone (ACTH) -- for treatment of epileptic
  spasms
\item
  Prednisone or prednisolone -- for treatment of epileptic spasms,
  Landau-Kleffner syndrome, etc.
\end{itemize}

\textbf{Vitamins} may be employed in the treatment of some epilepsies
(known as vitamin-responsive or vitamin-dependent seizures). These
include:

\begin{itemize}
\tightlist
\item
  Pyridoxine
\item
  Folinic acid
\item
  Biotin
\end{itemize}

\textbf{Surgical treatment} of epilepsy includes:

\begin{itemize}
\tightlist
\item
  Focal resection
\item
  Lobectomy
\item
  Hemispherectomy
\item
  Corpus callosotomy
\item
  Vagus nerve stimulation
\end{itemize}

\textbf{Ketogenic diet:} This employs high-fat, low-carbohydrate, and
low-protein diets in treating patients with epilepsy. When used, the
patient assumes a fasting state, and the brain relies on fatty acids
instead of glucose as the primary source of energy. The exact mechanism
of action is not known, but it leads to a reduction in seizure frequency
and duration. It is effective for all seizure types.

\section{Epilepsy syndromes}\label{epilepsy-syndromes}

An epilepsy syndrome is defined as a characteristic cluster of clinical
and electroencephalographic (EEG) features, often supported by specific
etiological findings (structural, genetic, metabolic, immune, and
infectious) {[}ILAE, 2022{]}

This is to say that some epilepsies can be clustered together as a
syndrome based on their features, such as etiology, age at onset,
seizure type(s), EEG findings, response to treatment, comorbidities, and
long-term prognosis.

Epilepsy syndromes often have age-dependent presentations, may have
age-dependent remission, and are usually strongly correlated with other
co-morbidities.

Common epilepsy syndromes in children include:

\begin{itemize}
\tightlist
\item
  Early infantile epileptic encephalopathy (Ohtahara syndrome)
\item
  Infantile epileptic spasms syndrome (West syndrome)
\item
  Severe myoclonic epilepsy of infancy (Dravet syndrome)
\item
  Lennox-Gastaut syndrome
\item
  Myoclonic astatic epilepsy (Doose syndrome)
\item
  Epilepsy-aphasia (Landau-Kleffner syndrome)
\item
  Childhood absence epilepsy
\item
  Juvenile absence epilepsy
\item
  Self-limiting epilepsy with centro-temporal spikes (previously Benign
  Rolandic epilepsy)
\item
  Juvenile myoclonic epilepsy
\end{itemize}

\section{Neonatal Seizures}\label{neonatal-seizures}

\subsection{Definition}\label{definition-22}

These are seizures in newborns 0-28 days (up to 2 months in clinical
practice). They are often poorly organized and difficult to distinguish
from normal activity. Clinical patterns include:

\begin{itemize}
\tightlist
\item
  Tonic stiffening of the body
\item
  Tonic deviation of the eyes
\item
  Apnea
\item
  Focal clonic movements of one limb or both limbs on one side
\item
  Myoclonic jerks
\item
  Paroxysmal laughing
\item
  Cycling movement of the limbs
\end{itemize}

Generalized tonic-clonic movements do not occur in the neonatal period.

The term \textbf{subtle seizures} refers to all the different patterns
without tonic or clonic movement of the limbs.

\subsection{Causes of seizures in neonates by
age}\label{causes-of-seizures-in-neonates-by-age}

Within 24 hours:

\begin{itemize}
\tightlist
\item
  Hypoxic-ischaemic encephalopathy
\item
  Intrauterine infections
\item
  Intracranial haemorrhage (IVH or SAH)
\item
  Metabolic disorders (commonly pyridoxine deficiency)
\end{itemize}

24 to 72 hours

\begin{itemize}
\tightlist
\item
  Neonatal sepsis (including meningitis)
\item
  Drug withdrawal
\item
  Metabolic disorders
\item
  Congenital malformations (cerebral dysgenesis)
\end{itemize}

After 72 hours

\begin{itemize}
\tightlist
\item
  Familial neonatal seizures
\item
  Kernicterus
\item
  Cerebral malformations
\item
  Metabolic disorders~~~~~~~~~~~~~~~~~~~~~~~~~~~~~~~~~~
\item
  Congenital malformations (cerebral dysgenesis)
\end{itemize}

\subsection{Management of neonatal
seizures}\label{management-of-neonatal-seizures}

\begin{itemize}
\tightlist
\item
  Remember ABCDs
\item
  Phenobarbital is the first-line drug of choice
\item
  Avoid benzodiazepine (unless in a specialist center with respiratory
  support)
\item
  Repeated Phenobarbital, Phenytoin, or IV infusion of Midazolam may be
  used as 2\textsuperscript{nd} line or 3\textsuperscript{rd} line
\item
  3\textsuperscript{rd} line: must be in the NICU where ventilators are
  available.
\item
  Give vitamins for vitamin-responsive or vitamin-dependent seizures.
  These include pyridoxine, biotin, and folinic acid.~~~~~~~~~~
\item
  Identify and treat the underlying cause.
\end{itemize}

\section{Seizure mimics}\label{seizure-mimics}

These are ``events'' that resemble seizures and may be misdiagnosed as
seizures if not carefully evaluated. Common seizure mimics in neonates
include:

\begin{itemize}
\tightlist
\item
  Benign sleep myoclonus
\item
  Jitteriness
\item
  Opisthotonos
\item
  Apnea (especially in preterm newborns)
\end{itemize}

In infants and older children, common seizure mimics include:

\begin{itemize}
\tightlist
\item
  Psychogenic non-epileptic seizures (PNES)
\item
  Jitteriness
\item
  Sandifer syndrome (GERD)
\item
  Breath-holding spells
\item
  Movement disorders (Tics, chorea, paroxysmal dyskinesias, etc.)
\item
  Benign sleep myoclonus
\item
  Opsoclonus myoclonus syndrome
\item
  Migraine variants
\item
  Parasomnias
\item
  Syncope (vasovagal or cardiac)
\item
  Self-gratification
\item
  Hypnic jerks
\item
  Hypertonicity in a patient with CP or anoxic brain injury
\item
  Cataplexy
\end{itemize}

\section{References}\label{references-1}

\begin{enumerate}
\def\labelenumi{\arabic{enumi}.}
\item
  Fisher, R.S., van Emde Boas, W., Blume, W., Elger, C., Genton, P.,
  Lee, P., Engel, J., 2005. Epileptic seizures and epilepsy: definitions
  proposed by the International League Against Epilepsy (ILAE) and the
  International Bureau for Epilepsy (IBE). Epilepsia 46, 470--472.
\item
  Fisher, R.S., Cross, J.H., French, J.A., Higurashi, N., Hirsch, E.,
  Jansen, F.E., Lagae, L., Moshé, S.L., Peltola, J., Roulet Perez, E.,
  Scheffer, I.E., Zuberi, S.M., 2017, Operational classification of
  seizure types by the International League Against Epilepsy: Position
  Paper of the ILAE Commission for Classification and Terminology.
  Epilepsia 58(4), 522- 530.
\item
  Fisher, R.S., Acevedo, C., Arzimanoglou, A., Bogacz, A., Cross, J.H.,
  Elger, J.H., Engel, J. Jr., Forsgren, L., French, J.A., Glynn, M.,
  Hesdorffer, D.C., Lee, B.I., Mathern, G.W., Moshé, S.L., Perucca, E.,
  Scheffer, I.E., Tomson, T., Watanabe, M., Wiebe, S., 2014. ILAE
  official report: a practical clinical definition of epilepsy.
  Epilepsia 55(4), 475-82
\item
  Newton, RW and Giles, A. Neurological Disorders. In: Lissauer and
  Carroll (ed). Illustrated textbook of Paediatrics, 5th edition.
  Elsevier. 2018: 501-24
\end{enumerate}

\chapter{Neuromuscular Disorders}\label{neuromuscular-disorders-1}

Neuromuscular disorders (NMDs) encompass a wide range of conditions that
affect the motor unit, the functional link between the nervous system
and muscles responsible for movement. They are characterized primarily
by muscle weakness, hypotonia, and impaired motor development. These
disorders may be inherited or acquired, acute or chronic, and involve
any level of the motor pathway from the anterior horn cell in the spinal
cord to the muscle fiber itself.

In Ghana and other sub-Saharan African countries, neuromuscular
disorders are often underdiagnosed or misdiagnosed due to limited access
to electrophysiological, genetic, and imaging studies. Nonetheless,
careful clinical assessment can identify most cases early and guide
appropriate referral and management.

\section{Common Terminologies}\label{common-terminologies}

Understanding basic terms is crucial for describing and classifying
neuromuscular disorders:

\begin{itemize}
\tightlist
\item
  \textbf{Motor unit:} The combination of a motor neuron, its axon, and
  all the muscle fibers it innervates.
\item
  \textbf{Hypotonia:} Reduced muscle tone leading to floppiness and poor
  resistance to passive movement.
\item
  \textbf{Weakness:} Reduction in the ability to generate voluntary
  muscle force.
\item
  \textbf{Fatigability:} Diminished ability to sustain muscle
  contraction during repeated or prolonged activity.
\item
  \textbf{Atrophy:} Loss of muscle bulk, usually due to disuse or
  denervation.
\item
  \textbf{Fasciculations:} Fine, involuntary muscle twitches visible
  under the skin, often seen in anterior horn cell disease.
\item
  \textbf{Pseudohypertrophy:} Apparent muscle enlargement due to fatty
  or fibrous replacement, as in Duchenne muscular dystrophy.
\item
  \textbf{Myopathy:} Primary disorder of the muscle fiber.
\item
  \textbf{Neuropathy:} Disease of peripheral nerves.
\end{itemize}

\section{Common Presentations}\label{common-presentations}

Children with neuromuscular disorders may present in various ways
depending on the site and nature of pathology. Typical features include:

\begin{itemize}
\tightlist
\item
  \textbf{Delayed motor milestones} -- not sitting, standing, or walking
  at expected ages.
\item
  \textbf{Hypotonia} -- often noted at birth or during infancy (``floppy
  baby'').
\item
  \textbf{Muscle weakness} -- proximal (difficulty rising from sitting)
  or distal (foot drop).
\item
  \textbf{Reduced or absent deep tendon reflexes.}
\item
  \textbf{Gait abnormalities} -- waddling or high-stepping gait.
\item
  \textbf{Cranial nerve involvement} -- facial weakness, ptosis,
  difficulty swallowing.
\item
  \textbf{Respiratory distress} -- due to diaphragmatic or intercostal
  weakness.
\item
  \textbf{Skeletal deformities} -- scoliosis, contractures.
\end{itemize}

A key clinical clue is that cognition and sensation are usually
\textbf{preserved} in primary muscle diseases but may be affected in
neuropathies or central lesions.

\section{The Floppy Infant Syndrome}\label{the-floppy-infant-syndrome}

\textbf{Definition:}\\
The term ``floppy infant'' describes a neonate or young infant with
decreased resistance to passive movement of limbs and trunk due to
hypotonia.

\subsection{\texorpdfstring{\textbf{Causes}}{Causes}}\label{causes-6}

Floppiness may be due to central or peripheral causes:

\begin{itemize}
\tightlist
\item
  \textbf{Central (CNS):} Hypoxic-ischemic encephalopathy, intracranial
  hemorrhage, metabolic encephalopathy, chromosomal syndromes (e.g.,
  Down syndrome, Prader--Willi).
\item
  \textbf{Peripheral:} Disorders of the motor unit --- anterior horn
  cell (SMA), peripheral nerves, neuromuscular junction (congenital
  myasthenic syndrome), or muscle (congenital myopathy, muscular
  dystrophy).
\end{itemize}

\subsection{\texorpdfstring{\textbf{Clinical
Features}}{Clinical Features}}\label{clinical-features-24}

\begin{itemize}
\tightlist
\item
  Frog-leg posture.
\item
  Head lag on traction.
\item
  Poor spontaneous movement but preserved alertness (suggesting
  peripheral cause).
\item
  Weak cry and suck.
\item
  Respiratory difficulty in severe cases.
\end{itemize}

\subsection{\texorpdfstring{\textbf{Evaluation}}{Evaluation}}\label{evaluation}

\begin{itemize}
\tightlist
\item
  Assess tone, reflexes, and antigravity movement.
\item
  Presence of brisk reflexes suggests central cause; diminished reflexes
  indicate peripheral lesion.
\item
  Look for dysmorphic features or systemic illness.
\end{itemize}

\section{Site of Lesion}\label{site-of-lesion}

Neuromuscular disorders are localized anatomically by identifying where
in the motor unit the pathology lies:

\begin{longtable}[]{@{}
  >{\raggedright\arraybackslash}p{(\linewidth - 4\tabcolsep) * \real{0.3333}}
  >{\raggedright\arraybackslash}p{(\linewidth - 4\tabcolsep) * \real{0.3333}}
  >{\raggedright\arraybackslash}p{(\linewidth - 4\tabcolsep) * \real{0.3333}}@{}}
\toprule\noalign{}
\begin{minipage}[b]{\linewidth}\raggedright
\textbf{Level of Lesion}
\end{minipage} & \begin{minipage}[b]{\linewidth}\raggedright
\textbf{Examples}
\end{minipage} & \begin{minipage}[b]{\linewidth}\raggedright
\textbf{Characteristic Features}
\end{minipage} \\
\midrule\noalign{}
\endhead
\bottomrule\noalign{}
\endlastfoot
\textbf{Anterior horn cell} & Spinal muscular atrophy, poliomyelitis &
Flaccid weakness, fasciculations, atrophy, absent reflexes \\
\textbf{Peripheral nerve} & Guillain--Barré syndrome,
Charcot--Marie--Tooth disease & Distal weakness, areflexia, sensory
loss \\
\textbf{Neuromuscular junction} & Myasthenia gravis, botulism &
Fatigable weakness, normal reflexes, cranial involvement \\
\textbf{Muscle} & Muscular dystrophies, congenital myopathies,
dermatomyositis & Proximal weakness, preserved sensation, normal or
reduced reflexes \\
\end{longtable}

Accurate localization is the first step in diagnosis and guides further
investigations such as electromyography (EMG) and genetic testing.

\section{Specific Neuromuscular
Disorders}\label{specific-neuromuscular-disorders}

\subsection{\texorpdfstring{\textbf{Spinal Muscular Atrophy
(SMA)}}{Spinal Muscular Atrophy (SMA)}}\label{spinal-muscular-atrophy-sma}

A \textbf{genetic disorder} caused by homozygous deletion or mutation of
the \emph{SMN1} gene, leading to degeneration of anterior horn cells.

\textbf{Clinical Types:}

\begin{itemize}
\item
  \textbf{Type I (Werdnig--Hoffmann):} Onset before 6 months, severe
  hypotonia (``floppy infant''), poor feeding, respiratory distress,
  early death.
\item
  \textbf{Type II:} Onset 6--18 months; able to sit but not walk
  unaided. - \textbf{Type III (Kugelberg--Welander):} Late onset; mild
  weakness, able to walk initially.
\end{itemize}

\textbf{Investigations:} EMG (denervation), genetic testing.\\
\textbf{Management:} Supportive care, respiratory support,
physiotherapy, new therapies (nusinersen, gene therapy) where available.

\subsection{\texorpdfstring{\textbf{Poliomyelitis}}{Poliomyelitis}}\label{poliomyelitis}

A viral infection caused by \emph{poliovirus} that selectively destroys
anterior horn cells. Though now rare due to immunization, sporadic cases
or vaccine-derived poliovirus can still occur.

\textbf{Clinical features:} - Fever and malaise followed by asymmetric,
flaccid paralysis. - No sensory loss. - Reflexes absent in affected
limbs.

\textbf{Diagnosis:} Isolation of poliovirus from stool or throat
swabs.\\
\textbf{Prevention:} Oral and inactivated polio vaccines (OPV and
IPV).\\
\textbf{Public health relevance:} Surveillance is vital as Ghana remains
part of the global polio eradication initiative.

\subsection{\texorpdfstring{\textbf{Guillain--Barré Syndrome
(GBS)}}{Guillain--Barré Syndrome (GBS)}}\label{guillainbarruxe9-syndrome-gbs}

An acute, immune-mediated demyelinating polyneuropathy, often
post-infectious.

\textbf{Presentation:}

\begin{itemize}
\tightlist
\item
  Rapidly progressive, symmetrical ascending weakness.
\item
  Areflexia.
\item
  May involve respiratory and autonomic dysfunction.
\end{itemize}

\textbf{Investigations:}

\begin{itemize}
\tightlist
\item
  CSF: High protein, normal cells (albuminocytologic dissociation).
\item
  Nerve conduction studies: Demyelination.
\end{itemize}

\textbf{Management:}

\begin{itemize}
\tightlist
\item
  Supportive care and monitoring for respiratory failure.
\item
  IV immunoglobulin (IVIG) or plasmapheresis.
\end{itemize}

\textbf{Prognosis:} Good in children; most recover fully.

\subsection{\texorpdfstring{\textbf{Charcot--Marie--Tooth Disease
(CMT)}}{Charcot--Marie--Tooth Disease (CMT)}}\label{charcotmarietooth-disease-cmt}

A \textbf{hereditary motor and sensory neuropathy} caused by
demyelination or axonal degeneration.

\textbf{Features:}

\begin{itemize}
\tightlist
\item
  Distal weakness (especially peroneal muscles).
\item
  Foot drop and high-arched feet (pes cavus).
\item
  Reduced reflexes and distal sensory loss.
\item
  Slowly progressive.
\end{itemize}

\textbf{Diagnosis:} Nerve conduction studies, genetic testing.\\
\textbf{Management:} Supportive --- physiotherapy, orthotics, and
counseling.

\subsection{\texorpdfstring{\textbf{Myasthenia
Gravis}}{Myasthenia Gravis}}\label{myasthenia-gravis}

An autoimmune disorder affecting the \textbf{neuromuscular junction},
where antibodies attack acetylcholine receptors.

\textbf{Features:} - Fluctuating weakness, worsening with exertion. -
Ptosis, diplopia, facial weakness. - Bulbar symptoms (dysphagia,
dysarthria). - Normal sensation and reflexes.

\textbf{Diagnosis:} Edrophonium test, AChR antibodies, EMG.\\
\textbf{Treatment:} Pyridostigmine, corticosteroids, thymectomy in
selected cases.

\subsection{\texorpdfstring{\textbf{Duchenne Muscular Dystrophy
(DMD)}}{Duchenne Muscular Dystrophy (DMD)}}\label{duchenne-muscular-dystrophy-dmd}

An \textbf{X-linked recessive} myopathy due to absence of
\textbf{dystrophin} protein.

\textbf{Features:}

\begin{itemize}
\tightlist
\item
  Onset at 2--5 years.
\item
  Progressive proximal weakness, calf pseudohypertrophy.
\item
  Gowers' sign (using hands to ``walk up'' thighs).
\item
  Wheelchair-bound by early teens.
\item
  Cardiomyopathy and respiratory failure later.
\end{itemize}

\textbf{Investigations:} Markedly elevated CK, genetic confirmation.\\
\textbf{Management:} Corticosteroids, physiotherapy, cardiac and
respiratory care.

\subsection{\texorpdfstring{\textbf{Congenital
Myopathies}}{Congenital Myopathies}}\label{congenital-myopathies}

A group of inherited disorders characterized by structural defects in
muscle fibers (e.g., nemaline, central core, myotubular myopathy).

\textbf{Features:}

\begin{itemize}
\item
  Neonatal hypotonia (``floppy infant'').
\item
  Delayed motor milestones.
\item
  Facial and ocular weakness.
\end{itemize}

Usually non-progressive or slowly progressive. Diagnosis is confirmed by
muscle biopsy and genetic analysis.

\subsection{\texorpdfstring{\textbf{Metabolic
Myopathies}}{Metabolic Myopathies}}\label{metabolic-myopathies}

Caused by enzyme defects in muscle metabolism (e.g., glycogen storage
diseases, fatty acid oxidation defects).

\textbf{Features:}

\begin{itemize}
\tightlist
\item
  Exercise intolerance.
\item
  Recurrent rhabdomyolysis or myoglobinuria.
\item
  Hypoglycemia in systemic types.
\end{itemize}

Management focuses on dietary modification and avoiding fasting or
strenuous exercise.

\subsection{\texorpdfstring{\textbf{Dermatomyositis}}{Dermatomyositis}}\label{dermatomyositis}

An \textbf{autoimmune inflammatory myopathy} affecting both skin and
muscle.

\textbf{Features:}

\begin{itemize}
\tightlist
\item
  Symmetrical proximal weakness.
\item
  Heliotrope rash (purple discolouration of eyelids).
\item
  Gottron's papules (over knuckles).
\item
  May involve myocardium and lungs.
\end{itemize}

\textbf{Investigations:} Elevated CK, EMG, muscle biopsy, autoantibody
screen.\\
\textbf{Treatment:} Corticosteroids, immunosuppressants, physiotherapy.

\section{Acute Flaccid Paralysis
(AFP)}\label{acute-flaccid-paralysis-afp}

\subsection{\texorpdfstring{\textbf{Case
Definition}}{Case Definition}}\label{case-definition}

Acute flaccid paralysis is defined as \textbf{sudden onset of weakness
or paralysis in a child under 15 years}, with reduced or absent muscle
tone and reflexes. It includes \textbf{poliomyelitis} and
\textbf{non-polio causes} such as Guillain--Barré syndrome and
transverse myelitis.

\subsection{\texorpdfstring{\textbf{Differential
Diagnosis}}{Differential Diagnosis}}\label{differential-diagnosis-23}

\begin{itemize}
\tightlist
\item
  Poliomyelitis
\item
  Guillain--Barré syndrome
\item
  Transverse myelitis
\item
  Traumatic neuritis
\item
  Myasthenia gravis
\item
  Botulism
\end{itemize}

\subsection{\texorpdfstring{\textbf{Public Health
Importance}}{Public Health Importance}}\label{public-health-importance}

AFP surveillance is central to \textbf{polio eradication} efforts.

\begin{itemize}
\tightlist
\item
  Every case of AFP must be reported within 24 hours to health
  authorities.
\item
  Stool specimens (two within 14 days of onset) are collected for viral
  isolation.
\item
  Early detection enables rapid outbreak response and immunization
  campaigns.
\end{itemize}

In Ghana, AFP surveillance has been instrumental in maintaining
polio-free certification and detecting vaccine-derived strains.

\section{Summary and Key Points}\label{summary-and-key-points}

\begin{itemize}
\tightlist
\item
  Neuromuscular disorders affect any part of the motor unit, causing
  weakness and hypotonia.
\item
  Careful clinical localisation is essential since advanced diagnostics
  are often unavailable.
\item
  Common disorders in children include SMA, DMD, GBS, and myasthenia
  gravis.
\item
  The floppy infant syndrome is a key early presentation of several
  NMDs.
\item
  Acute flaccid paralysis remains a \textbf{notifiable condition} due to
  its link with polio surveillance.
\item
  Multidisciplinary care, physiotherapy, respiratory and nutritional
  support---is crucial to improving outcomes.
\end{itemize}

\section{References}\label{references-2}

\begin{enumerate}
\def\labelenumi{\arabic{enumi}.}
\tightlist
\item
  Mah JK. Overview of neuromuscular disorders in children.
  \emph{Paediatr Child Health.} 2017.
\item
  Pearn J. Neuromuscular disorders in childhood. \emph{J Child Neurol.}
  2003.
\item
  WHO. \emph{Acute Flaccid Paralysis Surveillance and Polio Eradication
  Guidelines.} 2022.
\item
  Bushby K et al.~Diagnosis and management of Duchenne muscular
  dystrophy. \emph{Lancet Neurol.} 2010.
\item
  Ghana Health Service. \emph{Guidelines on Child Neurology and
  Disability Care}, 2022.
\end{enumerate}

\chapter{Neurocutaneous Syndromes}\label{neurocutaneous-syndromes-1}

\section{Introduction}\label{introduction-41}

Neurocutaneous syndromes, also known as \emph{phakomatoses}, are a
diverse group of congenital disorders characterised by abnormalities of
the skin and nervous system, often involving other organs such as the
eyes, bones, and endocrine glands. They result primarily from genetic
mutations that affect the development and differentiation of tissues
derived from the ectoderm. Because both the skin and the nervous system
originate from the ectodermal layer during embryogenesis, a
developmental insult to this germ layer can explain the frequent
coexistence of neurological and cutaneous manifestations.

Recognition of these disorders is crucial in paediatrics, as cutaneous
findings may serve as visible markers of significant underlying
neurological disease. Early diagnosis allows for timely surveillance,
prevention of complications, and genetic counselling for families. The
most common and well-studied neurocutaneous syndromes include
\textbf{Neurofibromatosis type 1 and 2}, \textbf{Tuberous sclerosis
complex}, \textbf{Sturge--Weber syndrome}, and \textbf{Von
Hippel--Lindau disease}. Other less frequent syndromes include
\textbf{Ataxia telangiectasia}, \textbf{Incontinentia pigmenti}, and
\textbf{Hypomelanosis of Ito}.

\section{Embryological and Pathophysiological
Basis}\label{embryological-and-pathophysiological-basis}

During the third to fourth week of embryonic development, the ectoderm
differentiates into two main tissues: the \textbf{neural ectoderm},
which forms the nervous system, and the \textbf{surface ectoderm}, which
gives rise to the skin and its appendages. Genetic defects that disrupt
the proliferation or migration of neural crest cells, progenitors for
melanocytes, Schwann cells, meninges, and components of the peripheral
nervous system, underlie most neurocutaneous disorders.

These mutations often affect genes involved in \textbf{tumour
suppression}, \textbf{cell growth regulation}, and \textbf{signal
transduction pathways}, such as the \textbf{RAS/MAPK} and \textbf{mTOR}
pathways. The result is dysregulated cell proliferation, hamartoma
formation, and predisposition to both benign and malignant tumours.

The underlying pathophysiological mechanisms therefore include:

\begin{itemize}
\tightlist
\item
  Hamartomatous growths in skin, brain, and other organs
\item
  Vascular malformations
\item
  Abnormal neuronal migration
\item
  Predisposition to neoplasia
\end{itemize}

\section{General Clinical Features}\label{general-clinical-features}

Neurocutaneous syndromes present variably, but some general features
include:

\begin{itemize}
\tightlist
\item
  \textbf{Cutaneous findings:} café-au-lait macules, hypopigmented
  macules, angiofibromas, shagreen patches, or port-wine stains.
\item
  \textbf{Neurological manifestations:} seizures, developmental delay,
  learning disability, and focal neurological deficits.
\item
  \textbf{Ocular features:} retinal hamartomas, optic gliomas, and
  choroidal angiomas.
\item
  \textbf{Skeletal anomalies:} scoliosis, pseudoarthrosis, and bone
  cysts.
\item
  \textbf{Endocrine abnormalities:} precocious puberty or adrenal
  lesions in certain syndromes.
\end{itemize}

Diagnosis relies on recognizing these characteristic associations,
supported by neuroimaging and genetic testing.

\section{Major Neurocutaneous
Syndromes}\label{major-neurocutaneous-syndromes}

\subsection{Neurofibromatosis Type 1
(NF1)}\label{neurofibromatosis-type-1-nf1}

NF1, also known as \emph{von Recklinghausen disease}, is the most common
neurocutaneous disorder, occurring in about 1 in 3,000 live births. It
is caused by mutations in the \textbf{NF1 gene} on chromosome 17, which
encodes \textbf{neurofibromin}, a tumour suppressor that regulates the
RAS pathway.

\textbf{Key clinical features:}

\begin{itemize}
\tightlist
\item
  ≥6 café-au-lait macules (\textgreater5 mm prepubertal, \textgreater15
  mm postpubertal)
\item
  Axillary or inguinal freckling
\item
  ≥2 neurofibromas or one plexiform neurofibroma
\item
  Lisch nodules (iris hamartomas)
\item
  Optic pathway glioma
\item
  Bony lesions (e.g., sphenoid dysplasia, pseudoarthrosis)
\item
  A first-degree relative with NF1
\end{itemize}

\textbf{Neurological features:} learning difficulties, attention
deficit, seizures, and risk of intracranial tumours.\\
\textbf{Complications:} hypertension due to renal artery stenosis or
pheochromocytoma, malignant peripheral nerve sheath tumours, and
scoliosis.

\textbf{Management:} multidisciplinary, dermatologic surveillance,
annual ophthalmology, developmental assessment, and blood pressure
monitoring.

\subsection{Neurofibromatosis Type 2
(NF2)}\label{neurofibromatosis-type-2-nf2}

NF2 results from mutations in the \textbf{NF2 gene} on chromosome 22,
which encodes \textbf{merlin (schwannomin)}, another tumour suppressor.
It is rarer (1 in 25,000 births) and typically presents in adolescence
or early adulthood.

\textbf{Hallmark features:}

\begin{itemize}
\tightlist
\item
  Bilateral vestibular schwannomas causing hearing loss and imbalance
\item
  Meningiomas, spinal schwannomas, and ependymomas
\item
  Posterior subcapsular cataracts in children
\end{itemize}

\textbf{Cutaneous findings} are less prominent than in NF1.\\
\textbf{Management:} MRI surveillance, hearing monitoring, and
neurosurgical intervention when necessary.

\subsection{Tuberous Sclerosis Complex
(TSC)}\label{tuberous-sclerosis-complex-tsc}

TSC is an autosomal dominant disorder due to mutations in \textbf{TSC1
(hamartin)} or \textbf{TSC2 (tuberin)} genes, which regulate the
\textbf{mTOR} signalling pathway. It affects about 1 in 6,000 births.

\textbf{Cutaneous features:}

\begin{itemize}
\tightlist
\item
  Hypomelanotic macules (``ash leaf'' spots)
\item
  Facial angiofibromas (adenoma sebaceum)
\item
  Shagreen patch (connective tissue nevus)
\item
  Periungual fibromas
\end{itemize}

\textbf{Neurological manifestations:}

\begin{itemize}
\tightlist
\item
  Cortical tubers, subependymal nodules, and subependymal giant cell
  astrocytomas (SEGAs)
\item
  Epilepsy (infantile spasms common)
\item
  Cognitive impairment and autism spectrum disorder
\end{itemize}

\textbf{Other organ involvement:}

\begin{itemize}
\tightlist
\item
  Cardiac rhabdomyomas (often regress spontaneously)
\item
  Renal angiomyolipomas and cysts
\item
  Pulmonary lymphangioleiomyomatosis (especially in females)
\end{itemize}

\textbf{Management:}

\begin{itemize}
\tightlist
\item
  mTOR inhibitors (everolimus, sirolimus) for SEGA or renal
  angiomyolipomas
\item
  Antiepileptic therapy
\item
  Developmental and genetic counselling
\end{itemize}

\subsection{Sturge--Weber Syndrome
(SWS)}\label{sturgeweber-syndrome-sws}

SWS is a sporadic neurocutaneous disorder caused by somatic mutations in
the \textbf{GNAQ gene}. It is characterized by vascular malformations
involving the leptomeninges and facial skin.

\textbf{Clinical features:}

\begin{itemize}
\tightlist
\item
  Facial port-wine stain (nevus flammeus) along the ophthalmic branch of
  the trigeminal nerve
\item
  Leptomeningeal angioma causing seizures, stroke-like episodes, and
  hemiparesis
\item
  Glaucoma and visual impairment
\end{itemize}

\textbf{Neuroimaging:} CT shows ``tram-track'' calcifications; MRI
reveals leptomeningeal enhancement.\\
\textbf{Management:} seizure control, ophthalmologic follow-up, and
laser therapy for port-wine stains.

\subsection{Von Hippel--Lindau Disease
(VHL)}\label{von-hippellindau-disease-vhl}

An autosomal dominant disorder caused by mutations in the \textbf{VHL
tumour suppressor gene} on chromosome 3.

\textbf{Major features:}

\begin{itemize}
\tightlist
\item
  Retinal and cerebellar hemangioblastomas
\item
  Renal cell carcinoma and pancreatic cysts
\item
  Pheochromocytoma
\end{itemize}

\textbf{Pediatric significance:} retinal angiomas may present in
adolescence.\\
\textbf{Management:} regular MRI screening and surgical or laser
intervention for tumours.

\subsection{Ataxia Telangiectasia (AT)}\label{ataxia-telangiectasia-at}

AT is an autosomal recessive disorder due to mutations in the
\textbf{ATM gene}, which regulates DNA repair.

\textbf{Key features:}

\begin{itemize}
\tightlist
\item
  Progressive cerebellar ataxia beginning in early childhood
\item
  Oculocutaneous telangiectasias (especially conjunctivae)
\item
  Immunodeficiency (low IgA and IgE)
\item
  High risk of malignancy, particularly lymphoma and leukemia
\end{itemize}

\textbf{Management:} supportive care, physiotherapy, infection
prophylaxis, and avoidance of ionizing radiation.

\subsection{Incontinentia Pigmenti
(IP)}\label{incontinentia-pigmenti-ip}

An X-linked dominant disorder lethal in males, caused by \textbf{IKBKG
gene} mutations affecting NF-κB signalling.

\textbf{Stages of skin lesions:}

\begin{enumerate}
\def\labelenumi{\arabic{enumi}.}
\tightlist
\item
  Vesicular stage (blistering rash in neonates)
\item
  Verrucous stage (wart-like lesions)
\item
  Hyperpigmented streaks along Blaschko's lines
\item
  Hypopigmented atrophic patches in adolescence/adulthood
\end{enumerate}

\textbf{Associated findings:} seizures, intellectual disability, dental
anomalies, and ocular defects.

\subsection{Hypomelanosis of Ito}\label{hypomelanosis-of-ito}

Characterized by \textbf{whorled or streaked hypopigmented lesions}
along Blaschko's lines. It represents chromosomal mosaicism rather than
a single gene defect.

\textbf{Associated features:} developmental delay, seizures, scoliosis,
and ocular anomalies.\\
Diagnosis is clinical; management is supportive.

\section{Investigations}\label{investigations-30}

Diagnostic evaluation depends on the suspected syndrome but generally
includes:

\begin{itemize}
\tightlist
\item
  \textbf{Neuroimaging (MRI brain/spine):} for intracranial lesions,
  calcifications, or tumours
\item
  \textbf{Dermatological examination:} Wood's lamp for hypopigmented
  lesions
\item
  \textbf{Ophthalmological assessment:} slit-lamp and retinal evaluation
\item
  \textbf{Genetic testing:} confirmation of specific mutations
\item
  \textbf{Renal ultrasound:} in NF1 and TSC for cysts or angiomyolipomas
\item
  \textbf{EEG:} in children with seizures
\item
  \textbf{Audiometry:} in NF2 for vestibular schwannoma detection
\end{itemize}

\section{Management Principles}\label{management-principles-2}

Management is multidisciplinary, involving paediatric neurologists,
dermatologists, ophthalmologists, geneticists, and surgeons. The
principles include:

\begin{itemize}
\tightlist
\item
  \textbf{Early recognition} through skin examination
\item
  \textbf{Regular surveillance} for neurological, renal, and ocular
  complications
\item
  \textbf{Seizure management} using appropriate antiepileptics
\item
  \textbf{Surgical intervention} for accessible tumours or disfiguring
  lesions
\item
  \textbf{Use of targeted therapies} (e.g., mTOR inhibitors in TSC)
\item
  \textbf{Genetic counselling} and screening of at-risk family members
\end{itemize}

\section{Prognosis}\label{prognosis-30}

Prognosis varies widely depending on the syndrome and severity of organ
involvement. For example:

\begin{itemize}
\tightlist
\item
  NF1 and TSC are compatible with long survival but carry risks of
  malignancy.
\item
  Sturge--Weber and AT have more guarded outcomes due to neurological
  decline.
\item
  Early diagnosis and targeted interventions have improved life
  expectancy and quality of life.
\end{itemize}

\section{Public Health Importance}\label{public-health-importance-1}

In Ghana and other low-resource settings, recognition of neurocutaneous
syndromes is often delayed due to limited access to dermatology and
genetic services. Raising awareness among healthcare providers is
essential for early identification and referral. Establishing registries
and integrating genetic counselling into paediatric services will
improve outcomes and guide family planning.

\chapter{Cerebrovascular Disease}\label{cerebrovascular-disease}

Cerebrovascular diseases in children encompass a group of disorders that
affect the blood vessels supplying the brain, leading to transient or
permanent neurological deficits. Although stroke is far less common in
children than in adults, it remains an important cause of morbidity and
mortality in paediatrics, often resulting in lifelong neurological
impairment if not promptly recognized and managed. Understanding the
anatomy of cerebral circulation, mechanisms of injury, and approach to
management is essential for clinicians caring for children.

\section{CNS Circulation}\label{cns-circulation}

The brain's blood supply is derived from two major arterial
systems---the \textbf{anterior circulation} from the \textbf{internal
carotid arteries} and the \textbf{posterior circulation} from the
\textbf{vertebral arteries}.

\begin{itemize}
\tightlist
\item
  The \textbf{internal carotid arteries} divide into the
  \textbf{anterior cerebral} and \textbf{middle cerebral arteries},
  supplying the frontal, parietal, and parts of the temporal lobes.\\
\item
  The \textbf{vertebral arteries}, arising from the subclavian arteries,
  unite to form the \textbf{basilar artery}, which then gives rise to
  the \textbf{posterior cerebral arteries} supplying the brainstem,
  cerebellum, and occipital lobes.\\
\item
  These arteries interconnect at the \textbf{Circle of Willis},
  providing collateral circulation that can partly compensate for
  occlusion or stenosis in one vascular territory.
\end{itemize}

Venous drainage occurs through the \textbf{dural venous sinuses} (such
as the superior sagittal, straight, and transverse sinuses), which
ultimately drain into the \textbf{internal jugular veins}. The integrity
of this circulation is critical to maintaining cerebral perfusion and
oxygenation.

In neonates and infants, cerebral blood flow is more vulnerable to
fluctuations in systemic perfusion due to immature autoregulatory
mechanisms, predisposing them to ischemic or hemorrhagic events under
stress conditions like hypoxia, dehydration, or infection.

\section{Definitions}\label{definitions-4}

\subsection{Stroke}\label{stroke}

\textbf{Stroke} is defined as a sudden onset of neurological deficit
resulting from a disturbance in cerebral blood flow, either due to
\textbf{ischemia} (interruption of blood supply) or \textbf{hemorrhage}
(bleeding into or around the brain), lasting more than 24 hours or
leading to death.

\subsection{Transient Ischaemic Attack
(TIA)}\label{transient-ischaemic-attack-tia}

A \textbf{Transient Ischaemic Attack (TIA)} is a brief episode of
neurological dysfunction caused by temporary cerebral ischemia,
typically lasting less than 24 hours and without permanent radiological
evidence of infarction. TIAs in children may herald an impending stroke
and warrant urgent evaluation.

\section{Classification of Stroke}\label{classification-of-stroke}

Stroke in children is classified based on the underlying pathophysiology
and anatomic distribution:

\begin{enumerate}
\def\labelenumi{\arabic{enumi}.}
\tightlist
\item
  \textbf{Ischemic Stroke}

  \begin{itemize}
  \tightlist
  \item
    \textbf{Arterial Ischemic Stroke (AIS)}: Due to obstruction of an
    artery by thrombus or embolus.
  \item
    \textbf{Cerebral Sinovenous Thrombosis (CSVT)}: Caused by thrombosis
    of cerebral veins or dural sinuses, leading to venous congestion and
    infarction.
  \end{itemize}
\item
  \textbf{Hemorrhagic Stroke}

  \begin{itemize}
  \tightlist
  \item
    \textbf{Intracerebral Hemorrhage (ICH)}: Bleeding within brain
    parenchyma.
  \item
    \textbf{Subarachnoid Hemorrhage (SAH)}: Bleeding into the
    subarachnoid space.
  \end{itemize}
\item
  \textbf{Perinatal or Neonatal Stroke}

  \begin{itemize}
  \tightlist
  \item
    Occurs between 20 weeks of gestation and 28 days of life, often
    associated with perinatal asphyxia, maternal infections, or
    coagulation disorders.
  \end{itemize}
\end{enumerate}

The distinction between these subtypes is essential for appropriate
investigation and management.

\section{Common Causes of Stroke in
Children}\label{common-causes-of-stroke-in-children}

Unlike adults, where atherosclerosis dominates, pediatric strokes have
diverse causes. These can be grouped as follows:

\begin{itemize}
\tightlist
\item
  \textbf{Cardiac causes}: Congenital heart disease (cyanotic heart
  lesions, endocarditis), cardiomyopathy, arrhythmias, and post-cardiac
  surgery emboli.\\
\item
  \textbf{Hematologic disorders}: Sickle cell disease (most common in
  sub-Saharan Africa), iron deficiency anemia, polycythemia, and
  coagulation disorders.\\
\item
  \textbf{Vascular abnormalities}: Moyamoya disease, arteriovenous
  malformations, aneurysms, fibromuscular dysplasia, and vasculitis.\\
\item
  \textbf{Infections}: Bacterial meningitis, varicella, HIV,
  tuberculosis, and malaria.\\
\item
  \textbf{Metabolic and systemic causes}: Homocystinuria, mitochondrial
  diseases, and diabetic ketoacidosis.\\
\item
  \textbf{Trauma}: Head injury causing arterial dissection or
  hemorrhage.\\
\item
  \textbf{Perinatal factors}: Birth asphyxia, maternal hypertension, and
  placental embolism.
\end{itemize}

In Ghana and similar regions, \textbf{sickle cell disease and severe
malaria} are particularly important causes of childhood stroke.

\section{Approach to the Child with
Stroke}\label{approach-to-the-child-with-stroke}

The approach involves early recognition, stabilization, identification
of the cause, and institution of appropriate therapy.

\subsection{History and Physical
Examination}\label{history-and-physical-examination}

A detailed \textbf{history} and \textbf{neurological examination} are
crucial.

\textbf{Key historical points include:}

\begin{itemize}
\tightlist
\item
  Onset and evolution of neurological symptoms (sudden or gradual)
\item
  Perinatal history in neonates
\item
  Past medical history --- notably sickle cell disease, cardiac disease,
  trauma, or infection
\item
  Family history of thrombophilia or stroke
\end{itemize}

\textbf{Symptoms and signs} may include: - Focal weakness or paralysis
(hemiparesis) - Seizures (common in neonates) - Speech difficulties
(aphasia) - Altered consciousness - Headache, vomiting - Gait
disturbances or ataxia (posterior circulation involvement) - Papilledema
in venous thrombosis

In neonates, manifestations are often subtle---poor feeding,
irritability, or asymmetric limb movements.

\section{Investigations}\label{investigations-31}

A structured approach to investigations helps confirm the diagnosis,
identify etiology, and guide treatment.

\subsection{Neuroimaging}\label{neuroimaging-1}

\begin{itemize}
\tightlist
\item
  \textbf{CT scan}: Quick and useful for detecting hemorrhage; less
  sensitive for early ischemia.\\
\item
  \textbf{MRI/MRA}: The gold standard for identifying ischemic lesions
  and vascular abnormalities.\\
\item
  \textbf{MR venography (MRV)}: For cerebral venous sinus thrombosis.\\
\item
  \textbf{Transcranial Doppler ultrasound}: Used for screening sickle
  cell disease patients at risk of stroke.
\end{itemize}

\subsection{Laboratory
Investigations}\label{laboratory-investigations-2}

\begin{itemize}
\tightlist
\item
  Full blood count: To detect anemia, polycythemia, or leukocytosis.\\
\item
  Sickle cell screening: Mandatory in all African children with
  stroke.\\
\item
  Coagulation profile: PT, aPTT, fibrinogen, D-dimers.\\
\item
  Inflammatory markers: ESR, CRP.\\
\item
  Blood cultures: If infection is suspected.\\
\item
  Serum electrolytes and glucose: To rule out metabolic causes.\\
\item
  Lumbar puncture: If infection is suspected and there is no raised
  intracranial pressure.
\end{itemize}

\subsection{Cardiac Evaluation}\label{cardiac-evaluation}

\begin{itemize}
\tightlist
\item
  \textbf{Echocardiography}: To identify congenital heart defects or
  mural thrombi.\\
\item
  \textbf{ECG}: For arrhythmias.
\end{itemize}

\chapter{Management and
Rehabilitation}\label{management-and-rehabilitation}

The goals of management are to \textbf{stabilize}, \textbf{prevent
progression}, \textbf{treat underlying causes}, and
\textbf{rehabilitate}.

\section{Acute Phase Management}\label{acute-phase-management}

\subsection{Stabilization}\label{stabilization}

\begin{itemize}
\tightlist
\item
  Ensure airway, breathing, and circulation.\\
\item
  Control seizures with anticonvulsants (e.g., phenobarbital,
  diazepam).\\
\item
  Correct hypoxia, hypoglycemia, and dehydration.\\
\item
  Manage raised intracranial pressure (elevate head, restrict fluids,
  osmotic agents if needed).
\end{itemize}

\subsection{Specific Treatment}\label{specific-treatment-1}

\begin{itemize}
\tightlist
\item
  \textbf{Ischemic stroke}:

  \begin{itemize}
  \tightlist
  \item
    Antithrombotic therapy (aspirin 3--5 mg/kg/day) once hemorrhage is
    excluded.\\
  \item
    Anticoagulation (heparin or LMWH) in venous sinus thrombosis or
    cardioembolic stroke.\\
  \item
    Exchange transfusion in sickle cell-related stroke to reduce HbS
    \textless30\%.\\
  \end{itemize}
\item
  \textbf{Hemorrhagic stroke}:

  \begin{itemize}
  \tightlist
  \item
    Control hypertension, correct coagulopathies.\\
  \item
    Neurosurgical intervention for hematoma evacuation or shunt
    insertion in hydrocephalus.
  \end{itemize}
\end{itemize}

\subsection{Infection Control}\label{infection-control}

\begin{itemize}
\tightlist
\item
  Empirical antibiotics if meningitis or sepsis is suspected.\\
\item
  Malaria treatment if smear-positive.
\end{itemize}

\section{Secondary Prevention}\label{secondary-prevention}

\begin{itemize}
\tightlist
\item
  Chronic transfusion programs in sickle cell disease to prevent
  recurrence.\\
\item
  Hydroxyurea therapy to maintain HbF and reduce stroke risk.\\
\item
  Long-term aspirin therapy for certain vasculopathies.\\
\item
  Management of cardiac or metabolic causes.
\end{itemize}

\section{Rehabilitation}\label{rehabilitation}

Early initiation of \textbf{multidisciplinary rehabilitation} is vital:

\begin{itemize}
\tightlist
\item
  \textbf{Physiotherapy} for muscle strength and mobility.\\
\item
  \textbf{Occupational therapy} for daily activity independence.\\
\item
  \textbf{Speech therapy} for language recovery.\\
\item
  \textbf{Psychological support} for emotional and cognitive
  adaptation.\\
\item
  \textbf{Educational interventions} for learning difficulties.
\end{itemize}

\chapter{Prognosis}\label{prognosis-31}

Prognosis depends on the cause, timeliness of intervention, and extent
of brain injury.

\begin{itemize}
\tightlist
\item
  \textbf{Mortality} in pediatric stroke is about 10--20\%.\\
\item
  \textbf{Neurological sequelae}, including hemiparesis, epilepsy, and
  cognitive deficits, occur in up to 60\% of survivors.\\
\item
  Recurrence is particularly high in sickle cell disease and untreated
  vascular anomalies.
\end{itemize}

\chapter{Public Health Importance}\label{public-health-importance-2}

Although less frequent than adult stroke, cerebrovascular disease in
children carries a \textbf{high burden of disability} and
\textbf{economic cost} due to long-term care needs. In sub-Saharan
Africa: - Late diagnosis and limited access to imaging delay
treatment.\\
- Sickle cell disease--related strokes are a major preventable cause.\\
- Strengthening \textbf{neonatal screening}, \textbf{malaria control},
and \textbf{access to transfusion programs} could significantly reduce
incidence.

Public health strategies should therefore include: - Early detection and
management of high-risk conditions (e.g., sickle cell disease).\\
- Training healthcare workers to recognize early symptoms.\\
- Improving neuroimaging availability and multidisciplinary care.

\chapter{Conclusion}\label{conclusion-29}

Cerebrovascular diseases in children, though uncommon, represent a
significant cause of acute neurological morbidity and long-term
disability. The underlying causes differ markedly from those in adults,
with hematologic, infectious, and congenital heart diseases
predominating in the tropics. Prompt recognition, neuroimaging, and
targeted management are key to improving outcomes. Strengthening
preventive strategies, particularly for sickle cell disease and
infectious causes, remains essential in reducing the burden of childhood
stroke in Ghana and across Africa.

\part{{Endocrinology}}

\chapter{Basics}\label{basics-2}

\section{Introduction}\label{introduction-42}

Endocrinology is the study of hormones and the glands that produce them.
In children, the endocrine system governs critical processes such as
growth, puberty, metabolism, and stress response. \textbf{Pediatric
endocrinology} focuses on disorders affecting hormone production,
secretion, and action during the developmental years.

Unlike adults, endocrine disorders in children often present with
\textbf{growth disturbances}, \textbf{delayed or precocious puberty}, or
\textbf{developmental abnormalities}, rather than classical systemic
symptoms. A strong understanding of endocrine physiology, growth
assessment, and hormonal feedback mechanisms is essential for early
recognition and management.

This essay outlines the anatomy and physiology of the endocrine system,
key hormones, principles of endocrine function testing, and an overview
of common endocrine disorders in children.

\section{Anatomy and Physiology of the Endocrine
System}\label{anatomy-and-physiology-of-the-endocrine-system}

The endocrine system consists of specialized glands that secrete
hormones directly into the bloodstream to act on distant target organs.
These hormones regulate metabolism, electrolyte balance, growth, and
reproduction.

\subsection{Major Endocrine Glands in
Children}\label{major-endocrine-glands-in-children}

\begin{longtable}[]{@{}
  >{\raggedright\arraybackslash}p{(\linewidth - 4\tabcolsep) * \real{0.1591}}
  >{\raggedright\arraybackslash}p{(\linewidth - 4\tabcolsep) * \real{0.3977}}
  >{\raggedright\arraybackslash}p{(\linewidth - 4\tabcolsep) * \real{0.4432}}@{}}
\toprule\noalign{}
\begin{minipage}[b]{\linewidth}\raggedright
Gland
\end{minipage} & \begin{minipage}[b]{\linewidth}\raggedright
Major Hormones
\end{minipage} & \begin{minipage}[b]{\linewidth}\raggedright
Primary Functions
\end{minipage} \\
\midrule\noalign{}
\endhead
\bottomrule\noalign{}
\endlastfoot
Hypothalamus & CRH, TRH, GHRH, GnRH & Regulates pituitary secretion \\
Pituitary & GH, ACTH, TSH, LH, FSH, Prolactin & Growth, metabolism,
reproduction \\
Thyroid & T3, T4, Calcitonin & Metabolism, growth, brain development \\
Parathyroid & PTH & Calcium and phosphate balance \\
Adrenal & Cortisol, Aldosterone, Androgens & Stress response,
electrolyte control \\
Pancreas & Insulin, Glucagon & Blood glucose regulation \\
Gonads & Oestrogen, Testosterone & Pubertal development \\
Pineal & Melatonin & Circadian rhythm \\
\end{longtable}

These glands communicate through a hierarchy known as the
\textbf{hypothalamic--pituitary--target gland axis}.

\section{The Hypothalamic--Pituitary
Axis}\label{the-hypothalamicpituitary-axis}

The \textbf{hypothalamus} integrates neural and hormonal signals to
regulate endocrine function. It secretes releasing or inhibiting
hormones into the hypophyseal portal system that act on the
\textbf{pituitary gland}.

The \textbf{pituitary}, often termed the ``master gland,'' releases
trophic hormones that stimulate target organs such as the thyroid,
adrenal glands, and gonads.

\subsubsection{Example of Hormonal Axes}\label{example-of-hormonal-axes}

\begin{longtable}[]{@{}
  >{\raggedright\arraybackslash}p{(\linewidth - 8\tabcolsep) * \real{0.0989}}
  >{\raggedright\arraybackslash}p{(\linewidth - 8\tabcolsep) * \real{0.2418}}
  >{\raggedright\arraybackslash}p{(\linewidth - 8\tabcolsep) * \real{0.2088}}
  >{\raggedright\arraybackslash}p{(\linewidth - 8\tabcolsep) * \real{0.1758}}
  >{\raggedright\arraybackslash}p{(\linewidth - 8\tabcolsep) * \real{0.2747}}@{}}
\toprule\noalign{}
\begin{minipage}[b]{\linewidth}\raggedright
Axis
\end{minipage} & \begin{minipage}[b]{\linewidth}\raggedright
Hypothalamic Hormone
\end{minipage} & \begin{minipage}[b]{\linewidth}\raggedright
Pituitary Hormone
\end{minipage} & \begin{minipage}[b]{\linewidth}\raggedright
Target Organ
\end{minipage} & \begin{minipage}[b]{\linewidth}\raggedright
Target Hormone
\end{minipage} \\
\midrule\noalign{}
\endhead
\bottomrule\noalign{}
\endlastfoot
Thyroid & TRH & TSH & Thyroid gland & T3, T4 \\
Adrenal & CRH & ACTH & Adrenal cortex & Cortisol \\
Gonadal & GnRH & LH, FSH & Ovaries/Testes & Oestrogen, Testosterone \\
Growth & GHRH & GH & Liver & IGF-1 \\
\end{longtable}

Feedback loops maintain hormone balance: elevated target hormones
inhibit hypothalamic and pituitary secretion, preventing overproduction.

\section{Principles of Hormone
Action}\label{principles-of-hormone-action}

Hormones exert their effects through \textbf{specific receptors} located
either on the cell membrane (peptide hormones) or within the cell
(steroid and thyroid hormones).

\subsection{Types of Hormones}\label{types-of-hormones}

\begin{longtable}[]{@{}
  >{\raggedright\arraybackslash}p{(\linewidth - 4\tabcolsep) * \real{0.3333}}
  >{\raggedright\arraybackslash}p{(\linewidth - 4\tabcolsep) * \real{0.3333}}
  >{\raggedright\arraybackslash}p{(\linewidth - 4\tabcolsep) * \real{0.3333}}@{}}
\toprule\noalign{}
\begin{minipage}[b]{\linewidth}\raggedright
Type
\end{minipage} & \begin{minipage}[b]{\linewidth}\raggedright
Examples
\end{minipage} & \begin{minipage}[b]{\linewidth}\raggedright
Mechanism of Action
\end{minipage} \\
\midrule\noalign{}
\endhead
\bottomrule\noalign{}
\endlastfoot
\textbf{Peptide hormones} & GH, insulin, ACTH & Bind to cell-surface
receptors → activate second messengers \\
\textbf{Steroid hormones} & Cortisol, oestrogen, testosterone & Enter
cells → bind nuclear receptors → modulate gene transcription \\
\textbf{Amine hormones} & T3, T4, catecholamines & Derived from tyrosine
→ act on nuclear or membrane receptors \\
\end{longtable}

Because of these mechanisms, hormone effects may be \textbf{rapid}
(e.g., insulin, catecholamines) or \textbf{delayed but sustained} (e.g.,
thyroid or steroid hormones).

\section{Growth and Development: A Central
Theme}\label{growth-and-development-a-central-theme}

\textbf{Growth} is the most visible reflection of endocrine health in
children. It results from the interplay between genetics, nutrition, and
hormones---particularly \textbf{growth hormone (GH)}, \textbf{thyroid
hormones}, \textbf{cortisol}, \textbf{sex steroids}, and
\textbf{insulin}.

\subsection{Growth Hormone Axis}\label{growth-hormone-axis}

\begin{itemize}
\tightlist
\item
  GH is secreted by the anterior pituitary in a pulsatile fashion.\\
\item
  Stimulated by GHRH and inhibited by somatostatin.\\
\item
  Acts on the liver to produce \textbf{Insulin-like Growth Factor-1
  (IGF-1)}, which mediates bone and tissue growth.
\end{itemize}

\textbf{Deficiency of GH} leads to short stature with normal body
proportions, while \textbf{excess GH} (rare in children) causes
gigantism.

\section{Assessment of Endocrine Function in
Children}\label{assessment-of-endocrine-function-in-children}

Because hormone levels fluctuate with time of day, age, and
physiological state, interpretation of results must be age-appropriate.

\subsection{Clinical Assessment}\label{clinical-assessment}

\begin{enumerate}
\def\labelenumi{\arabic{enumi}.}
\tightlist
\item
  \textbf{Detailed history}

  \begin{itemize}
  \tightlist
  \item
    Growth pattern, family history of endocrine disorders.
  \item
    Pubertal changes, appetite, and energy levels.
  \end{itemize}
\item
  \textbf{Physical examination}

  \begin{itemize}
  \tightlist
  \item
    Height and weight plotted on growth charts.
  \item
    Pubertal staging (Tanner).
  \item
    Dysmorphic features, goitre, pigmentation, or obesity.
  \end{itemize}
\end{enumerate}

\subsection{Laboratory Evaluation}\label{laboratory-evaluation-2}

\begin{itemize}
\tightlist
\item
  \textbf{Basal hormone levels:} measured at specific times (e.g.,
  morning cortisol).\\
\item
  \textbf{Dynamic tests:}

  \begin{itemize}
  \tightlist
  \item
    \emph{Stimulation tests} (for suspected deficiency): e.g., GH
    stimulation, ACTH stimulation.
  \item
    \emph{Suppression tests} (for suspected excess): e.g., dexamethasone
    suppression test.
  \end{itemize}
\item
  \textbf{Imaging:} MRI of hypothalamic-pituitary region, thyroid
  ultrasound, or adrenal CT where indicated.
\item
  \textbf{Genetic testing:} for congenital or syndromic causes.
\end{itemize}

\section{Common Endocrine Disorders in
Children}\label{common-endocrine-disorders-in-children}

\subsection{Disorders of Growth and Pituitary
Function}\label{disorders-of-growth-and-pituitary-function}

\subsection{Growth Hormone Deficiency
(GHD)}\label{growth-hormone-deficiency-ghd}

\begin{itemize}
\tightlist
\item
  Causes: congenital pituitary hypoplasia, tumours, trauma, or
  idiopathic.
\item
  Features: short stature, delayed bone age, infantile face,
  hypoglycaemia.
\item
  Diagnosis: low IGF-1, failed GH stimulation.
\item
  Treatment: recombinant GH therapy.
\end{itemize}

\subsection{Growth Hormone Excess}\label{growth-hormone-excess}

\begin{itemize}
\tightlist
\item
  Usually from a pituitary adenoma.
\item
  Causes excessive linear growth (gigantism) or acromegalic features.
\item
  Managed surgically or with somatostatin analogues.
\end{itemize}

\section{Thyroid Disorders}\label{thyroid-disorders-1}

\subsection{Congenital Hypothyroidism}\label{congenital-hypothyroidism}

\begin{itemize}
\tightlist
\item
  Caused by thyroid dysgenesis or dyshormonogenesis.
\item
  Features: prolonged neonatal jaundice, macroglossia, hypotonia,
  umbilical hernia.
\item
  Early detection via newborn screening.
\item
  Treatment: lifelong levothyroxine replacement.
\end{itemize}

\subsection{Juvenile Hypothyroidism}\label{juvenile-hypothyroidism}

\begin{itemize}
\tightlist
\item
  Often autoimmune (Hashimoto's thyroiditis).
\item
  Features: growth retardation, fatigue, dry skin, constipation.
\item
  Managed with thyroid hormone replacement.
\end{itemize}

\subsection{Hyperthyroidism}\label{hyperthyroidism}

\begin{itemize}
\tightlist
\item
  Most commonly Graves' disease (autoimmune).
\item
  Presents with weight loss, tachycardia, goitre, tremor.
\item
  Treatment includes antithyroid drugs (carbimazole), beta-blockers, or
  surgery.
\end{itemize}

\section{Adrenal Disorders}\label{adrenal-disorders}

\subsection{Congenital Adrenal Hyperplasia
(CAH)}\label{congenital-adrenal-hyperplasia-cah}

\begin{itemize}
\tightlist
\item
  Autosomal recessive enzyme defect (most often 21-hydroxylase
  deficiency).
\item
  Features: ambiguous genitalia in females, dehydration, salt wasting.
\item
  Diagnosis: elevated 17-hydroxyprogesterone.
\item
  Treatment: hydrocortisone and fludrocortisone replacement.
\end{itemize}

\subsection{Adrenal Insufficiency (Addison's
Disease)}\label{adrenal-insufficiency-addisons-disease}

\begin{itemize}
\tightlist
\item
  Causes: autoimmune, infections (TB), or congenital defects.
\item
  Features: fatigue, hyperpigmentation, hypotension, salt craving.
\item
  Treatment: hydrocortisone and mineralocorticoid replacement.
\end{itemize}

\subsection{Cushing Syndrome}\label{cushing-syndrome}

\begin{itemize}
\tightlist
\item
  From prolonged steroid therapy or adrenal tumour.
\item
  Features: obesity, moon facies, striae, growth retardation.
\item
  Managed by reducing steroid dose or surgical excision of the tumour.
\end{itemize}

\section{Disorders of Puberty}\label{disorders-of-puberty}

\subsection{Precocious Puberty}\label{precocious-puberty}

\begin{itemize}
\tightlist
\item
  Onset of secondary sexual characteristics before 8 years in girls, 9
  years in boys.
\item
  May be central (early activation of HPG axis) or peripheral (excess
  sex steroids).
\item
  Requires MRI brain and hormonal evaluation.
\item
  Managed with GnRH analogues for central forms.
\end{itemize}

\subsection{Delayed Puberty}\label{delayed-puberty}

\begin{itemize}
\tightlist
\item
  Absence of pubertal changes beyond 13 years (girls) or 14 years
  (boys).
\item
  May result from constitutional delay, hypogonadism, or chronic
  illness.
\item
  Treatment involves reassurance or sex steroid replacement when
  indicated.
\end{itemize}

\section{Disorders of Glucose
Metabolism}\label{disorders-of-glucose-metabolism}

\subsection{Type 1 Diabetes Mellitus
(T1DM)}\label{type-1-diabetes-mellitus-t1dm}

\begin{itemize}
\tightlist
\item
  Autoimmune destruction of pancreatic beta cells.
\item
  Commonest endocrine disorder in children.
\item
  Symptoms: polyuria, polydipsia, weight loss, fatigue.
\item
  Complication: diabetic ketoacidosis (DKA).
\item
  Management: insulin therapy, dietary regulation, self-monitoring.
\end{itemize}

\subsection{Type 2 Diabetes Mellitus
(T2DM)}\label{type-2-diabetes-mellitus-t2dm}

\begin{itemize}
\tightlist
\item
  Increasingly seen in obese adolescents.
\item
  Features: obesity, acanthosis nigricans, mild hyperglycaemia.
\item
  Managed by lifestyle changes, metformin, and sometimes insulin.
\end{itemize}

\subsection{Parathyroid and Calcium
Disorders}\label{parathyroid-and-calcium-disorders}

\begin{itemize}
\tightlist
\item
  \textbf{Hypocalcaemia} due to hypoparathyroidism, vitamin D
  deficiency, or pseudohypoparathyroidism → tetany, seizures, Chvostek
  and Trousseau signs.
\item
  \textbf{Hypercalcaemia} due to hyperparathyroidism or malignancy →
  polyuria, vomiting, bone pain.
\item
  Management focuses on correcting calcium and vitamin D levels.
\end{itemize}

\begin{center}\rule{0.5\linewidth}{0.5pt}\end{center}

\section{Diagnostic Principles in Pediatric
Endocrinology}\label{diagnostic-principles-in-pediatric-endocrinology}

Because hormone secretion in children changes with age and puberty,
interpretation requires \textbf{age-specific reference ranges}. For
example: - Neonates have higher cortisol and thyroid hormone levels. -
GH secretion is pulsatile; random measurements are unreliable. -
Pubertal staging is crucial for evaluating gonadal hormones.

\textbf{Dynamic testing} remains a cornerstone of diagnosis:

\begin{itemize}
\item
  \emph{Stimulation tests} (GH, ACTH, GnRH) evaluate deficiency.
\item
  \emph{Suppression tests} (dexamethasone, glucose tolerance) assess
  hormone excess.
\end{itemize}

In addition, \textbf{imaging} helps detect structural lesions, and
\textbf{genetic testing} confirms syndromic causes such as
Prader--Willi, Turner, or Kallmann syndromes.

\section{Principles of Management}\label{principles-of-management}

\subsection{Hormone Replacement}\label{hormone-replacement}

\begin{itemize}
\tightlist
\item
  Use physiological doses to mimic normal secretion (e.g.,
  hydrocortisone 3 times daily rather than dexamethasone).
\item
  Regular adjustment according to growth and puberty.
\end{itemize}

\subsection{Suppression of Hormone
Excess}\label{suppression-of-hormone-excess}

\begin{itemize}
\tightlist
\item
  Antithyroid drugs, glucocorticoid therapy for CAH, or GnRH analogues
  for precocious puberty.
\end{itemize}

\subsection{Treatment of Underlying
Causes}\label{treatment-of-underlying-causes}

\begin{itemize}
\tightlist
\item
  Surgery for tumours, antibiotics for infections, or withdrawal of
  exogenous steroids.
\end{itemize}

\subsection{Long-Term Follow-Up}\label{long-term-follow-up}

\begin{itemize}
\tightlist
\item
  Regular growth and pubertal assessment.
\item
  Monitoring for treatment side effects (e.g., growth suppression from
  steroids).
\item
  Transition planning for adult endocrinology care.
\end{itemize}

\section{Growth Monitoring and
Interpretation}\label{growth-monitoring-and-interpretation}

Growth assessment is central to paediatric endocrinology.\\
Essential tools include:

\begin{itemize}
\tightlist
\item
  \textbf{Growth charts:} plot height, weight, BMI, and head
  circumference.
\item
  \textbf{Mid-parental height:} predicts target range.
\item
  \textbf{Growth velocity:} \textless4 cm/year after age 4 suggests
  pathology.
\item
  \textbf{Bone age:} X-ray of left hand/wrist compared with standards
  (Greulich--Pyle).
\end{itemize}

Discrepancy between bone age and chronological age helps differentiate
\textbf{constitutional delay} from \textbf{endocrine causes} of short
stature.

\section{Psychosocial Aspects}\label{psychosocial-aspects}

Endocrine disorders can profoundly affect self-esteem, body image, and
school performance.\\
Children with delayed puberty or short stature may face bullying or
anxiety, while those with obesity or hirsutism may experience social
withdrawal.

Management should include:

\begin{itemize}
\tightlist
\item
  Psychological counselling.\\
\item
  Support groups and family education.\\
\item
  Encouragement of normal participation in school and sports.
\end{itemize}

\section{Summary Table}\label{summary-table-2}

\begin{longtable}[]{@{}
  >{\raggedright\arraybackslash}p{(\linewidth - 6\tabcolsep) * \real{0.0989}}
  >{\raggedright\arraybackslash}p{(\linewidth - 6\tabcolsep) * \real{0.2967}}
  >{\raggedright\arraybackslash}p{(\linewidth - 6\tabcolsep) * \real{0.2527}}
  >{\raggedright\arraybackslash}p{(\linewidth - 6\tabcolsep) * \real{0.3516}}@{}}
\toprule\noalign{}
\begin{minipage}[b]{\linewidth}\raggedright
Domain
\end{minipage} & \begin{minipage}[b]{\linewidth}\raggedright
Common Disorders
\end{minipage} & \begin{minipage}[b]{\linewidth}\raggedright
Key Hormones
\end{minipage} & \begin{minipage}[b]{\linewidth}\raggedright
Typical Features
\end{minipage} \\
\midrule\noalign{}
\endhead
\bottomrule\noalign{}
\endlastfoot
Growth & GH deficiency/excess & GH, IGF-1 & Short/tall stature \\
Thyroid & Hypo-/hyperthyroidism & T4, TSH & Growth delay, goitre \\
Adrenal & CAH, Addison's, Cushing's & Cortisol, Aldosterone &
Virilisation, fatigue, obesity \\
Puberty & Precocious, delayed & LH, FSH, sex steroids & Early/late
sexual changes \\
Glucose & T1DM, T2DM & Insulin, glucagon & Polyuria, weight change \\
Calcium & Hypo-/hyperparathyroidism & PTH, Vit D & Tetany, bone pain \\
\end{longtable}

\section{Key Takeaways}\label{key-takeaways}

\begin{itemize}
\tightlist
\item
  \textbf{Pediatric endocrinology} deals with hormonal disorders
  affecting growth, metabolism, and development.\\
\item
  Early recognition is crucial to prevent irreversible effects on growth
  and neurodevelopment.\\
\item
  \textbf{Growth failure}, \textbf{pubertal delay}, or
  \textbf{unexplained obesity} should prompt endocrine evaluation.\\
\item
  Lifelong follow-up, family education, and psychosocial support are
  integral to successful management.
\end{itemize}

\section{Suggested Reading}\label{suggested-reading}

\begin{enumerate}
\def\labelenumi{\arabic{enumi}.}
\tightlist
\item
  Nelson Textbook of Pediatrics, 22nd Edition.\\
\item
  Sperling M. A., \emph{Pediatric Endocrinology}, 5th Edition.\\
\item
  Brook CGD, Clayton PE. \emph{Clinical Pediatric Endocrinology}, 7th
  Edition.\\
\item
  WHO. \emph{Handbook on Pediatric Endocrine Disorders for Low-Resource
  Settings}, 2021.\\
\item
  ESPE Clinical Practice Guidelines, 2023.
\end{enumerate}

\chapter{Diabetes Mellitus}\label{diabetes-mellitus}

\section{Introduction}\label{introduction-43}

\textbf{Diabetes mellitus (DM)} is a chronic metabolic disorder
characterized by persistent hyperglycaemia resulting from defects in
insulin secretion, insulin action, or both.\\
In children, diabetes is one of the most common endocrine and metabolic
disorders encountered in clinical practice. The two main types seen are
\textbf{Type 1 diabetes mellitus (T1DM)} and \textbf{Type 2 diabetes
mellitus (T2DM)}, though other rare forms such as \textbf{monogenic
diabetes} and \textbf{secondary diabetes} also occur.

In Ghana and many parts of sub-Saharan Africa, the incidence of Type 1
diabetes is increasing, with children presenting frequently in diabetic
ketoacidosis (DKA). In recent years, Type 2 diabetes has emerged among
adolescents due to rising rates of obesity and sedentary lifestyles.

This chapter provides a detailed overview of diabetes mellitus in
children --- covering definitions, classification, pathophysiology,
clinical features, diagnosis, management, and complications --- with
special emphasis on practical approaches for medical students and
paediatric residents.

\section{Classification of Diabetes
Mellitus}\label{classification-of-diabetes-mellitus}

The \textbf{American Diabetes Association (ADA)} and \textbf{World
Health Organization (WHO)} classify diabetes as follows:

\begin{longtable}[]{@{}
  >{\raggedright\arraybackslash}p{(\linewidth - 4\tabcolsep) * \real{0.3333}}
  >{\raggedright\arraybackslash}p{(\linewidth - 4\tabcolsep) * \real{0.3333}}
  >{\raggedright\arraybackslash}p{(\linewidth - 4\tabcolsep) * \real{0.3333}}@{}}
\toprule\noalign{}
\begin{minipage}[b]{\linewidth}\raggedright
Category
\end{minipage} & \begin{minipage}[b]{\linewidth}\raggedright
Description
\end{minipage} & \begin{minipage}[b]{\linewidth}\raggedright
Pathophysiology
\end{minipage} \\
\midrule\noalign{}
\endhead
\bottomrule\noalign{}
\endlastfoot
\textbf{Type 1 Diabetes Mellitus (T1DM)} & Autoimmune destruction of
pancreatic β-cells leading to absolute insulin deficiency &
Immune-mediated; often associated with islet autoantibodies \\
\textbf{Type 2 Diabetes Mellitus (T2DM)} & Combination of insulin
resistance and relative insulin deficiency & Associated with obesity,
puberty, and family history \\
\textbf{Monogenic Diabetes (MODY/Neonatal DM)} & Single-gene defects
affecting β-cell function & Often autosomal dominant inheritance \\
\textbf{Secondary Diabetes} & Secondary to other conditions &
Pancreatitis, cystic fibrosis, steroid therapy, endocrine disorders \\
\end{longtable}

\section{Epidemiology}\label{epidemiology-7}

\begin{itemize}
\tightlist
\item
  \textbf{Global trends:} T1DM accounts for over 90\% of diabetes in
  children globally. The incidence varies widely, from
  \textless1/100,000 in sub-Saharan Africa to \textgreater30/100,000 in
  Europe.\\
\item
  \textbf{In Ghana:} Studies in Kumasi, Accra, and Cape Coast indicate
  increasing T1DM cases among children aged 5--15 years, often
  presenting with DKA at diagnosis.\\
\item
  \textbf{Type 2 diabetes:} Previously rare, it is now reported among
  Ghanaian adolescents, especially those with obesity and a strong
  family history.
\end{itemize}

The rise in diabetes prevalence among children is attributed to improved
awareness, urbanization, lifestyle changes, and possibly environmental
triggers.

\section{Physiology of Insulin and Glucose
Homeostasis}\label{physiology-of-insulin-and-glucose-homeostasis}

The \textbf{pancreas} plays a central role in glucose metabolism.\\
The \textbf{β-cells of the islets of Langerhans} secrete insulin, a
peptide hormone that facilitates glucose uptake by muscle and adipose
tissue and suppresses hepatic glucose production.

\subsection{Key Actions of Insulin}\label{key-actions-of-insulin}

\begin{itemize}
\tightlist
\item
  Increases glucose uptake via GLUT-4 transporters.\\
\item
  Promotes glycogen synthesis in liver and muscle.\\
\item
  Inhibits gluconeogenesis and glycogenolysis.\\
\item
  Enhances lipid and protein synthesis.
\end{itemize}

Deficiency or resistance to insulin leads to \textbf{hyperglycaemia},
\textbf{lipolysis}, \textbf{ketogenesis}, and \textbf{protein
catabolism}, which underlie the metabolic disturbances seen in diabetes.

\section{Pathophysiology}\label{pathophysiology-26}

\subsection{Type 1 Diabetes Mellitus
(T1DM)}\label{type-1-diabetes-mellitus-t1dm-1}

An \textbf{autoimmune process} targets pancreatic β-cells, resulting in
progressive loss of insulin production.

\subsubsection{Mechanism}\label{mechanism}

\begin{itemize}
\tightlist
\item
  Genetic predisposition (HLA-DR3, DR4, DQ2, DQ8).
\item
  Environmental triggers (viral infections---Coxsackie B, enterovirus,
  or toxins).
\item
  Presence of autoantibodies: GAD65, ICA, IA-2, or ZnT8.
\end{itemize}

Destruction of \textgreater80\% of β-cells leads to insulin deficiency,
hyperglycaemia, and ketone formation.

\subsubsection{Natural History}\label{natural-history-1}

\begin{enumerate}
\def\labelenumi{\arabic{enumi}.}
\tightlist
\item
  \textbf{Genetic predisposition}\\
\item
  \textbf{Autoimmunity initiation}\\
\item
  \textbf{Progressive β-cell destruction}\\
\item
  \textbf{Clinical diabetes} (symptomatic hyperglycaemia or DKA)
\end{enumerate}

\subsection{Type 2 Diabetes Mellitus
(T2DM)}\label{type-2-diabetes-mellitus-t2dm-1}

Characterized by \textbf{insulin resistance} and \textbf{relative
insulin deficiency}.

\begin{itemize}
\tightlist
\item
  Strongly linked to obesity, acanthosis nigricans, and family
  history.\\
\item
  Insulin resistance is worsened by puberty hormones, sedentary
  lifestyle, and high-calorie diets.
\end{itemize}

\subsection{Other Forms}\label{other-forms}

\begin{itemize}
\tightlist
\item
  \textbf{Monogenic diabetes:} Often presents in infancy or early
  childhood; may respond to sulfonylureas rather than insulin.\\
\item
  \textbf{Secondary diabetes:} Can follow steroid therapy, Cushing
  syndrome, or pancreatitis.
\end{itemize}

\section{Clinical Features}\label{clinical-features-25}

The presentation of diabetes in children depends on its type and stage
at diagnosis.

\subsection{Classical Symptoms of Type 1
Diabetes}\label{classical-symptoms-of-type-1-diabetes}

\begin{enumerate}
\def\labelenumi{\arabic{enumi}.}
\tightlist
\item
  \textbf{Polyuria} -- due to osmotic diuresis.\\
\item
  \textbf{Polydipsia} -- compensatory thirst.\\
\item
  \textbf{Polyphagia} -- despite weight loss.\\
\item
  \textbf{Weight loss and fatigue} -- from catabolism.\\
\item
  \textbf{Enuresis or nocturia} -- previously toilet-trained child.\\
\item
  \textbf{Recurrent infections} -- especially skin and vulvovaginal
  candidiasis.
\end{enumerate}

In many Ghanaian children, the disease is recognized \textbf{late},
often presenting as \textbf{diabetic ketoacidosis (DKA)} with vomiting,
abdominal pain, and dehydration.

\subsection{Features of Type 2
Diabetes}\label{features-of-type-2-diabetes}

\begin{itemize}
\tightlist
\item
  Often asymptomatic or mildly symptomatic.\\
\item
  Associated with \textbf{obesity}, \textbf{acanthosis nigricans},
  \textbf{hypertension}, or \textbf{dyslipidaemia}.\\
\item
  May also present with ketosis (ketosis-prone T2DM).
\end{itemize}

\section{Diabetic Ketoacidosis (DKA)}\label{diabetic-ketoacidosis-dka}

\textbf{DKA} is the most serious acute complication of diabetes in
children and remains a major cause of morbidity and mortality in Ghana.

\subsection{Pathophysiology}\label{pathophysiology-27}

Absolute or relative insulin deficiency → increased lipolysis → free
fatty acids converted to ketone bodies (acetoacetate, β-hydroxybutyrate)
→ metabolic acidosis.

\subsection{Diagnostic Criteria
(WHO/ISPAD)}\label{diagnostic-criteria-whoispad}

\begin{itemize}
\tightlist
\item
  Blood glucose \textgreater11 mmol/L\\
\item
  Venous pH \textless7.3 or bicarbonate \textless15 mmol/L\\
\item
  Presence of ketonaemia or ketonuria
\end{itemize}

\subsection{Clinical Features}\label{clinical-features-26}

\begin{itemize}
\tightlist
\item
  Dehydration\\
\item
  Kussmaul breathing (deep, laboured)\\
\item
  Fruity (acetone) breath\\
\item
  Abdominal pain, vomiting\\
\item
  Drowsiness or altered consciousness
\end{itemize}

\subsection{Complications}\label{complications-28}

\begin{itemize}
\tightlist
\item
  \textbf{Cerebral oedema} (most feared, especially in children)\\
\item
  \textbf{Shock and electrolyte imbalance}
\end{itemize}

\subsection{Management Principles}\label{management-principles-3}

\begin{enumerate}
\def\labelenumi{\arabic{enumi}.}
\tightlist
\item
  \textbf{Fluid resuscitation:} 0.9\% saline initially, then adjusted.\\
\item
  \textbf{Insulin therapy:} IV infusion 0.05--0.1 U/kg/hr after initial
  fluid resuscitation.\\
\item
  \textbf{Electrolyte correction:} especially potassium.\\
\item
  \textbf{Treatment of precipitating cause:} infection, missed insulin
  dose, etc.\\
\item
  \textbf{Close monitoring:} vital signs, glucose, electrolytes, and
  neurological status.
\end{enumerate}

\section{Diagnostic Evaluation}\label{diagnostic-evaluation-1}

\subsection{Diagnostic Criteria for Diabetes
(WHO)}\label{diagnostic-criteria-for-diabetes-who}

\begin{longtable}[]{@{}ll@{}}
\toprule\noalign{}
Test & Diagnostic Value \\
\midrule\noalign{}
\endhead
\bottomrule\noalign{}
\endlastfoot
Fasting plasma glucose & ≥7.0 mmol/L (after ≥8 hours fasting) \\
Random plasma glucose & ≥11.1 mmol/L with symptoms \\
2-hour OGTT plasma glucose & ≥11.1 mmol/L \\
HbA1c & ≥6.5\% (in standardized lab) \\
\end{longtable}

In children, diagnosis is typically based on \textbf{random glucose} and
\textbf{clinical presentation}, especially when symptomatic.

\subsection{Additional Laboratory
Tests}\label{additional-laboratory-tests}

\begin{itemize}
\tightlist
\item
  Urine dipstick for glucose and ketones\\
\item
  Serum electrolytes, urea, creatinine\\
\item
  Blood gas for acidosis (if DKA suspected)\\
\item
  HbA1c for baseline control\\
\item
  Autoantibody screening (GAD, ICA) when available\\
\item
  Lipid profile (for T2DM)
\end{itemize}

\section{Differential Diagnosis of Polyuria and
Polydipsia}\label{differential-diagnosis-of-polyuria-and-polydipsia}

\begin{longtable}[]{@{}ll@{}}
\toprule\noalign{}
Condition & Key Features \\
\midrule\noalign{}
\endhead
\bottomrule\noalign{}
\endlastfoot
Diabetes mellitus & Hyperglycaemia, glucosuria, ketonuria \\
Diabetes insipidus & Dilute urine, normal glucose \\
Psychogenic polydipsia & Normal glucose, no ketones \\
Chronic renal disease & Proteinuria, elevated creatinine \\
\end{longtable}

\section{Management of Type 1 Diabetes
Mellitus}\label{management-of-type-1-diabetes-mellitus}

The cornerstone of management is \textbf{lifelong insulin therapy} with
education, diet regulation, and psychosocial support.

\subsection{Insulin Therapy}\label{insulin-therapy}

\subsubsection{Types of Insulin}\label{types-of-insulin}

\begin{longtable}[]{@{}llll@{}}
\toprule\noalign{}
Type & Onset & Peak & Duration \\
\midrule\noalign{}
\endhead
\bottomrule\noalign{}
\endlastfoot
Rapid-acting (Lispro, Aspart) & 10--15 min & 1--2 hr & 3--5 hr \\
Short-acting (Regular) & 30--60 min & 2--4 hr & 6--8 hr \\
Intermediate (NPH) & 2--4 hr & 6--10 hr & 12--18 hr \\
Long-acting (Glargine, Detemir) & 1--2 hr & Minimal & 24 hr \\
\end{longtable}

\subsubsection{Insulin Regimens}\label{insulin-regimens}

\begin{itemize}
\tightlist
\item
  \textbf{Conventional:} Two daily injections (mixed short- and
  intermediate-acting).\\
\item
  \textbf{Basal--bolus:} Long-acting basal + rapid-acting at meals.\\
\item
  \textbf{Insulin pump therapy:} Rarely available in Ghana but ideal for
  motivated families.
\end{itemize}

\subsubsection{Dose}\label{dose}

\begin{itemize}
\tightlist
\item
  Starting total daily dose: 0.5--1.0 units/kg/day.\\
\item
  Adjust according to blood glucose trends.
\end{itemize}

\subsubsection{Monitoring}\label{monitoring-2}

\begin{itemize}
\tightlist
\item
  \textbf{Self-monitoring of blood glucose (SMBG):} before meals and at
  bedtime.\\
\item
  \textbf{HbA1c:} every 3--6 months; target \textless7.5\%.\\
\item
  \textbf{Growth and pubertal assessment} at each visit.
\end{itemize}

\subsection{Nutritional Management}\label{nutritional-management}

Dietary management aims at maintaining normoglycaemia while ensuring
adequate growth.

\subsubsection{Principles}\label{principles}

\begin{itemize}
\tightlist
\item
  Balanced diet with appropriate carbohydrate distribution (50--55\% of
  calories).\\
\item
  Encourage high-fibre complex carbohydrates; limit refined sugars.\\
\item
  Consistent meal timing, coordinated with insulin action.\\
\item
  Carbohydrate counting for insulin adjustment.\\
\item
  Avoid fasting or skipping meals.
\end{itemize}

In Ghana, diets can incorporate \textbf{local foods} --- e.g., whole
grain banku, kontomire stew, beans, plantain --- with attention to
portion control and reduced oil.

\subsection{Exercise}\label{exercise}

Regular physical activity improves insulin sensitivity and
cardiovascular health.

\textbf{Precautions:} - Monitor glucose before and after exercise.\\
- Take extra carbohydrates if prolonged activity (\textgreater60 min).\\
- Avoid vigorous exercise if glucose \textgreater14 mmol/L with ketones.

\subsection{Education and Psychosocial
Support}\label{education-and-psychosocial-support}

Education is central to successful management. Children and caregivers
must be taught:

\begin{itemize}
\tightlist
\item
  Insulin injection technique and site rotation.\\
\item
  Glucose monitoring and record keeping.\\
\item
  Recognition and management of \textbf{hypoglycaemia} and
  \textbf{DKA}.\\
\item
  Importance of adherence, diet, and exercise.
\end{itemize}

Psychosocial counselling is essential to address school integration,
stigma, and family stress. Peer support groups (e.g., Diabetes Youth
Care Ghana) are beneficial.

\section{Management of Type 2 Diabetes
Mellitus}\label{management-of-type-2-diabetes-mellitus}

Management begins with \textbf{lifestyle modification}, followed by
\textbf{pharmacotherapy} if control is inadequate.

\subsection{Lifestyle Modification}\label{lifestyle-modification}

\begin{itemize}
\tightlist
\item
  Weight reduction through diet and exercise.\\
\item
  Reduction of sugary beverages and fried foods.\\
\item
  Screen family members for diabetes and obesity.
\end{itemize}

\subsection{Pharmacological
Treatment}\label{pharmacological-treatment-1}

\begin{itemize}
\tightlist
\item
  \textbf{Metformin} is the first-line agent (500 mg once or twice
  daily, titrated as tolerated).\\
\item
  \textbf{Insulin} may be required temporarily if hyperglycaemia is
  severe or during intercurrent illness.\\
\item
  Blood pressure and lipid control are also essential.
\end{itemize}

\section{Acute and Chronic
Complications}\label{acute-and-chronic-complications}

\subsection{Acute Complications}\label{acute-complications}

\begin{longtable}[]{@{}
  >{\raggedright\arraybackslash}p{(\linewidth - 4\tabcolsep) * \real{0.3333}}
  >{\raggedright\arraybackslash}p{(\linewidth - 4\tabcolsep) * \real{0.3333}}
  >{\raggedright\arraybackslash}p{(\linewidth - 4\tabcolsep) * \real{0.3333}}@{}}
\toprule\noalign{}
\begin{minipage}[b]{\linewidth}\raggedright
Complication
\end{minipage} & \begin{minipage}[b]{\linewidth}\raggedright
Description
\end{minipage} & \begin{minipage}[b]{\linewidth}\raggedright
Prevention
\end{minipage} \\
\midrule\noalign{}
\endhead
\bottomrule\noalign{}
\endlastfoot
\textbf{Hypoglycaemia} & Glucose \textless3.9 mmol/L due to excess
insulin or missed meals & Regular meals, dose adjustment, carry glucose
snacks \\
\textbf{Diabetic Ketoacidosis (DKA)} & Life-threatening metabolic
acidosis & Early recognition and adherence \\
\textbf{Infections} & Urinary tract, skin, respiratory & Good hygiene,
immunizations \\
\end{longtable}

\subsection{Chronic Complications}\label{chronic-complications}

These are rare in well-controlled paediatric patients but may develop
after years of poor control.

\begin{longtable}[]{@{}
  >{\raggedright\arraybackslash}p{(\linewidth - 4\tabcolsep) * \real{0.1600}}
  >{\raggedright\arraybackslash}p{(\linewidth - 4\tabcolsep) * \real{0.3200}}
  >{\raggedright\arraybackslash}p{(\linewidth - 4\tabcolsep) * \real{0.5200}}@{}}
\toprule\noalign{}
\begin{minipage}[b]{\linewidth}\raggedright
System
\end{minipage} & \begin{minipage}[b]{\linewidth}\raggedright
Complication
\end{minipage} & \begin{minipage}[b]{\linewidth}\raggedright
Screening
\end{minipage} \\
\midrule\noalign{}
\endhead
\bottomrule\noalign{}
\endlastfoot
Eyes & Retinopathy & Fundoscopy from 11 years or 2 years
post-diagnosis \\
Kidneys & Microalbuminuria → nephropathy & Annual urine
albumin/creatinine ratio \\
Nerves & Peripheral neuropathy & Clinical exam annually \\
Cardiovascular & Hypertension, dyslipidaemia & BP check, lipid
profile \\
\end{longtable}

Prevention lies in \textbf{good glycaemic control (HbA1c
\textless7.5\%)}, healthy lifestyle, and regular follow-up.

\section{Follow-Up and Long-Term
Care}\label{follow-up-and-long-term-care}

\subsection{Key Components}\label{key-components}

\begin{itemize}
\tightlist
\item
  Growth and pubertal monitoring.\\
\item
  Annual screening for complications.\\
\item
  Psychosocial evaluation.\\
\item
  Immunizations: influenza, pneumococcal, hepatitis B.\\
\item
  Transition planning from paediatric to adult diabetes care.
\end{itemize}

\subsection{School Support}\label{school-support}

\begin{itemize}
\tightlist
\item
  Teachers and school nurses should understand the child's condition.\\
\item
  Allow glucose monitoring and snacks when needed.\\
\item
  Create an emergency plan for hypoglycaemia.
\end{itemize}

\section{Special Considerations in
Ghana}\label{special-considerations-in-ghana-1}

\begin{itemize}
\tightlist
\item
  \textbf{Late presentation} due to limited awareness and poor access to
  diagnostic tools.\\
\item
  \textbf{Insulin affordability} remains a challenge; reliance on
  hospital pharmacies or NHIS coverage is common.\\
\item
  \textbf{Refrigeration} issues in rural areas affect insulin storage;
  traditional clay pot coolers may help.\\
\item
  \textbf{Cultural beliefs} sometimes lead to trial of herbal remedies,
  delaying therapy.\\
\item
  \textbf{Education of healthcare workers} at peripheral centres is
  vital to improve early diagnosis and management.
\end{itemize}

\section{Emerging Trends}\label{emerging-trends}

\begin{itemize}
\tightlist
\item
  Use of \textbf{continuous glucose monitoring (CGM)} and
  \textbf{insulin pens} is increasing in urban Ghana.\\
\item
  \textbf{Tele-diabetes follow-up} and community-based screening
  programmes are expanding.\\
\item
  Research into \textbf{autoantibody prevalence} and \textbf{genetic
  profiles} in African children with diabetes is ongoing.
\end{itemize}

\section{Summary Table}\label{summary-table-3}

\begin{longtable}[]{@{}
  >{\raggedright\arraybackslash}p{(\linewidth - 6\tabcolsep) * \real{0.2500}}
  >{\raggedright\arraybackslash}p{(\linewidth - 6\tabcolsep) * \real{0.2500}}
  >{\raggedright\arraybackslash}p{(\linewidth - 6\tabcolsep) * \real{0.2500}}
  >{\raggedright\arraybackslash}p{(\linewidth - 6\tabcolsep) * \real{0.2500}}@{}}
\toprule\noalign{}
\begin{minipage}[b]{\linewidth}\raggedright
Type
\end{minipage} & \begin{minipage}[b]{\linewidth}\raggedright
Mechanism
\end{minipage} & \begin{minipage}[b]{\linewidth}\raggedright
Key Features
\end{minipage} & \begin{minipage}[b]{\linewidth}\raggedright
Management
\end{minipage} \\
\midrule\noalign{}
\endhead
\bottomrule\noalign{}
\endlastfoot
Type 1 & Autoimmune β-cell destruction → insulin deficiency & Polyuria,
weight loss, DKA & Insulin + diet + education \\
Type 2 & Insulin resistance + relative deficiency & Obesity, acanthosis
nigricans & Lifestyle + metformin ± insulin \\
Monogenic & Single gene mutation & Neonatal/childhood onset, family
history & May respond to sulfonylureas \\
Secondary & Due to other diseases/drugs & Variable & Treat underlying
cause \\
\end{longtable}

\section{Key Takeaways}\label{key-takeaways-1}

\begin{itemize}
\tightlist
\item
  Diabetes mellitus is an \textbf{increasingly common chronic condition}
  in Ghanaian children.\\
\item
  \textbf{Type 1 diabetes} remains predominant, with DKA as the main
  presenting feature.\\
\item
  \textbf{Insulin therapy}, \textbf{dietary regulation}, and
  \textbf{education} are the cornerstones of management.\\
\item
  Early recognition, regular follow-up, and psychosocial support
  significantly reduce morbidity.\\
\item
  Improving awareness and healthcare access remain critical to improving
  outcomes.
\end{itemize}

\section{Suggested Reading}\label{suggested-reading-1}

\begin{enumerate}
\def\labelenumi{\arabic{enumi}.}
\tightlist
\item
  International Society for Pediatric and Adolescent Diabetes (ISPAD)
  Clinical Practice Guidelines, 2022.\\
\item
  Nelson Textbook of Pediatrics, 22nd Edition.\\
\item
  WHO Pocket Book of Hospital Care for Children, 3rd Edition.\\
\item
  Ghana Health Service (GHS) Clinical Guidelines, 2024 Edition.\\
\item
  Amissah-Arthur MB, et al.~``Profile of childhood diabetes mellitus at
  Komfo Anokye Teaching Hospital, Kumasi.'' \emph{Ghana Medical Journal}
  2020.
\end{enumerate}

\chapter{Thyroid Disorders}\label{thyroid-disorders-2}

\section{Introduction}\label{introduction-44}

The thyroid gland plays a critical role in regulating growth,
metabolism, and neurodevelopment in children. Thyroid disorders are
among the most common endocrine problems in paediatrics and may present
with growth retardation, developmental delay, or changes in metabolic
rate.\\
Because thyroid hormones influence almost every organ system, the
clinical manifestations of their dysfunction are diverse and often
subtle, especially in infants and children.

This note provides an overview of thyroid physiology, outlines common
thyroid disorders in children, and discusses their clinical
presentation, investigation, and management.

\section{Anatomy and Physiology}\label{anatomy-and-physiology}

The thyroid gland is a butterfly-shaped organ located in the anterior
neck. It secretes the hormones \textbf{thyroxine (T4)} and
\textbf{triiodothyronine (T3)} under the control of
\textbf{thyroid-stimulating hormone (TSH)} from the anterior pituitary,
which is regulated by \textbf{thyrotropin-releasing hormone (TRH)} from
the hypothalamus.

\textbf{Functions of thyroid hormones:}

\begin{itemize}
\tightlist
\item
  Regulation of \textbf{metabolic rate} and energy expenditure.
\item
  Promotion of \textbf{growth and skeletal maturation}.
\item
  Essential for \textbf{brain development}, especially in the first 2--3
  years of life.
\item
  Maintenance of \textbf{cardiovascular, gastrointestinal, and
  neuromuscular functions}.
\end{itemize}

\section{Classification of Thyroid
Disorders}\label{classification-of-thyroid-disorders}

Thyroid diseases in children may be classified as:

\begin{longtable}[]{@{}
  >{\raggedright\arraybackslash}p{(\linewidth - 2\tabcolsep) * \real{0.5000}}
  >{\raggedright\arraybackslash}p{(\linewidth - 2\tabcolsep) * \real{0.5000}}@{}}
\toprule\noalign{}
\begin{minipage}[b]{\linewidth}\raggedright
Category
\end{minipage} & \begin{minipage}[b]{\linewidth}\raggedright
Examples
\end{minipage} \\
\midrule\noalign{}
\endhead
\bottomrule\noalign{}
\endlastfoot
\textbf{Functional abnormalities} & Hypothyroidism, Hyperthyroidism \\
\textbf{Structural abnormalities} & Goitre, Thyroid nodules, Thyroid
carcinoma \\
\textbf{Inflammatory/autoimmune disorders} & Hashimoto's thyroiditis,
Graves' disease \\
\textbf{Congenital abnormalities} & Thyroid dysgenesis,
Dyshormonogenesis, Thyroid hormone resistance \\
\end{longtable}

\section{Congenital Hypothyroidism}\label{congenital-hypothyroidism-1}

\subsection{Definition}\label{definition-23}

A condition present at birth due to a deficiency of thyroid hormone
production, leading to impaired neurodevelopment and growth if
untreated.

\subsection{Aetiology}\label{aetiology-18}

\begin{enumerate}
\def\labelenumi{\arabic{enumi}.}
\tightlist
\item
  \textbf{Thyroid dysgenesis (≈85\%)} --- agenesis, ectopia, or
  hypoplasia.
\item
  \textbf{Dyshormonogenesis (≈10\%)} --- enzyme defects in hormone
  synthesis.
\item
  \textbf{Central hypothyroidism (rare)} --- pituitary or hypothalamic
  dysfunction.
\item
  \textbf{Transient causes} --- maternal antithyroid drugs, iodine
  deficiency or excess.
\end{enumerate}

\subsection{Pathophysiology}\label{pathophysiology-28}

Lack of thyroid hormone during early infancy leads to irreversible brain
damage and developmental delay. Early detection and treatment are
therefore crucial.

\subsection{Clinical Features}\label{clinical-features-27}

Symptoms are often subtle at birth due to transplacental maternal T4.\\
Typical features include:

\begin{itemize}
\tightlist
\item
  Prolonged neonatal jaundice
\item
  Feeding difficulties and lethargy
\item
  Large fontanelle, macroglossia, umbilical hernia
\item
  Hypotonia and dry skin
\item
  Later: growth retardation, developmental delay, coarse facial
  features, hoarse cry.
\end{itemize}

\subsection{Diagnosis}\label{diagnosis-13}

\begin{itemize}
\tightlist
\item
  \textbf{Newborn screening:} Elevated TSH, low T4.
\item
  \textbf{Confirmatory tests:} Serum TSH and free T4.
\item
  \textbf{Imaging:} Thyroid ultrasound or radionuclide scan to assess
  gland structure.
\end{itemize}

\subsection{Management}\label{management-21}

\begin{itemize}
\tightlist
\item
  \textbf{Levothyroxine:} 10--15 µg/kg/day orally, initiated as soon as
  diagnosis is made (ideally within 2 weeks of birth).
\item
  \textbf{Monitoring:} TSH and free T4 every 2--4 weeks initially, then
  every 3 months after 6 months of age.
\item
  \textbf{Long-term care:} Normal growth and neurodevelopment if therapy
  is started early and adherence is maintained.
\end{itemize}

\section{Acquired Hypothyroidism}\label{acquired-hypothyroidism}

\subsection{Aetiology}\label{aetiology-19}

\begin{enumerate}
\def\labelenumi{\arabic{enumi}.}
\tightlist
\item
  \textbf{Autoimmune thyroiditis (Hashimoto's disease)} --- the most
  common cause in older children and adolescents.
\item
  Iodine deficiency or excess.
\item
  Drugs (lithium, amiodarone, antithyroid medications).
\item
  Post-irradiation or post-surgical.
\item
  Pituitary or hypothalamic dysfunction (secondary hypothyroidism).
\end{enumerate}

\subsection{Clinical Features}\label{clinical-features-28}

\begin{itemize}
\tightlist
\item
  Growth failure and delayed bone age.
\item
  Lethargy, cold intolerance, and constipation.
\item
  Weight gain, dry skin, brittle hair.
\item
  Bradycardia and delayed puberty.
\item
  Diffuse or nodular goitre (in autoimmune thyroiditis).
\end{itemize}

\subsection{Investigations}\label{investigations-32}

\begin{itemize}
\tightlist
\item
  Serum TSH (usually elevated) and free T4 (low).
\item
  Antithyroid antibodies (anti-TPO, anti-thyroglobulin) are positive in
  Hashimoto's disease.
\item
  Thyroid ultrasound --- may show heterogeneous echotexture in cases of
  autoimmune disease.
\end{itemize}

\subsection{Management}\label{management-22}

\begin{itemize}
\tightlist
\item
  \textbf{Levothyroxine replacement:} 2--4 µg/kg/day in children; adjust
  based on growth and lab monitoring.
\item
  Regular monitoring of growth velocity, TSH, and T4 levels.
\item
  Lifelong therapy in autoimmune cases; temporary therapy in transient
  hypothyroidism.
\end{itemize}

\section{Hyperthyroidism in Children}\label{hyperthyroidism-in-children}

\subsection{Overview}\label{overview-2}

Hyperthyroidism results from excessive circulating thyroid hormones,
leading to increased metabolic rate and sympathetic overactivity.\\
It is much less common in children than hypothyroidism, but is most
often due to \textbf{Graves' disease}.

\subsection{Aetiology}\label{aetiology-20}

\begin{enumerate}
\def\labelenumi{\arabic{enumi}.}
\tightlist
\item
  \textbf{Graves' disease (autoimmune)} --- antibodies stimulate TSH
  receptors → excess hormone synthesis.
\item
  Toxic multinodular goitre or adenoma.
\item
  Thyroiditis (transient thyrotoxicosis due to gland inflammation).
\item
  Iatrogenic (excessive thyroxine intake).
\end{enumerate}

\subsection{Clinical Features}\label{clinical-features-29}

\begin{itemize}
\tightlist
\item
  Weight loss despite a good appetite.
\item
  Heat intolerance, sweating, palpitations.
\item
  Tremor, irritability, hyperactivity, poor school performance.
\item
  Tachycardia, goitre, and exophthalmos (Graves' disease).
\item
  Accelerated bone maturation and growth.
\end{itemize}

\subsection{Diagnosis}\label{diagnosis-14}

\begin{itemize}
\tightlist
\item
  \textbf{Low TSH}, \textbf{elevated T3 and T4}.
\item
  \textbf{Thyroid antibodies:} positive TSH receptor antibodies in
  Graves' disease.
\item
  \textbf{Radioiodine uptake:} diffuse in Graves', focal in toxic
  adenoma, low in thyroiditis.
\end{itemize}

\subsection{Management}\label{management-23}

\subsubsection{Antithyroid Drugs}\label{antithyroid-drugs}

\begin{itemize}
\tightlist
\item
  \textbf{Carbimazole (0.5--1 mg/kg/day)} or \textbf{Methimazole} are
  first-line.
\item
  Treatment is usually continued for 12--24 months, with gradual
  tapering.
\item
  Monitor for side effects: agranulocytosis, rash, hepatotoxicity.
\end{itemize}

\subsubsection{Beta-blockers}\label{beta-blockers}

\begin{itemize}
\tightlist
\item
  \textbf{Propranolol (1--2 mg/kg/day)} for control of tachycardia,
  tremor, and anxiety.
\end{itemize}

\subsubsection{Definitive Therapy}\label{definitive-therapy}

\begin{itemize}
\tightlist
\item
  Radioiodine ablation (rarely used in children under 10).
\item
  Subtotal thyroidectomy for drug-resistant or relapsed cases.
\end{itemize}

\subsubsection{Long-Term Follow-Up}\label{long-term-follow-up-1}

\begin{itemize}
\tightlist
\item
  Regular assessment of growth, heart rate, and thyroid function.
\item
  Watch for iatrogenic hypothyroidism after definitive treatment.
\end{itemize}

\section{Autoimmune Thyroiditis (Hashimoto's
Disease)}\label{autoimmune-thyroiditis-hashimotos-disease}

\subsection{Pathophysiology}\label{pathophysiology-29}

Autoimmune destruction of the thyroid gland mediated by T-lymphocytes
and antibodies against thyroid peroxidase (TPO) and thyroglobulin.\\
It may coexist with other autoimmune disorders (e.g., type 1 diabetes,
Addison's disease).

\subsection{Clinical Features}\label{clinical-features-30}

\begin{itemize}
\tightlist
\item
  Painless, firm, diffuse goitre.
\item
  Features of hypothyroidism may develop gradually.
\item
  Occasionally presents with transient hyperthyroidism
  (``Hashitoxicosis'').
\end{itemize}

\subsection{Investigations}\label{investigations-33}

\begin{itemize}
\tightlist
\item
  Elevated TSH, low or normal T4.
\item
  Positive anti-TPO or anti-thyroglobulin antibodies.
\item
  Ultrasound: heterogeneous echotexture.
\end{itemize}

\subsection{Management}\label{management-24}

\begin{itemize}
\tightlist
\item
  Levothyroxine replacement for hypothyroid patients.
\item
  Observation for euthyroid patients with small goitres.
\item
  Screening for other autoimmune diseases (diabetes, celiac disease).
\end{itemize}

\section{Goitre in Children}\label{goitre-in-children}

\subsection{Definition}\label{definition-24}

An enlarged thyroid gland is visible or palpable in the neck.

\subsection{Causes}\label{causes-7}

\begin{longtable}[]{@{}ll@{}}
\toprule\noalign{}
Category & Examples \\
\midrule\noalign{}
\endhead
\bottomrule\noalign{}
\endlastfoot
\textbf{Physiological} & Pubertal growth spurt \\
\textbf{Iodine deficiency} & Endemic goitre \\
\textbf{Autoimmune} & Hashimoto's, Graves' \\
\textbf{Dyshormonogenesis} & Genetic enzyme defects \\
\textbf{Neoplasms} & Benign nodules, carcinoma \\
\end{longtable}

\subsection{Clinical Evaluation}\label{clinical-evaluation-1}

\begin{itemize}
\tightlist
\item
  Size, symmetry, tenderness, and presence of nodules.
\item
  Symptoms of pressure or dysfunction (dysphagia, hoarseness,
  hypo-/hyperthyroid signs).
\end{itemize}

\subsection{Investigations}\label{investigations-34}

\begin{itemize}
\tightlist
\item
  Serum TSH and free T4.
\item
  Thyroid antibodies.
\item
  Ultrasound: to distinguish diffuse vs nodular enlargement.
\item
  Fine-needle aspiration cytology (FNAC) for suspicious nodules.
\end{itemize}

\subsection{Management}\label{management-25}

\begin{itemize}
\tightlist
\item
  Treat the underlying cause.
\item
  Iodine supplementation in deficiency areas.
\item
  Levothyroxine suppressive therapy for benign goitres.
\item
  Surgical excision for compressive or suspicious nodules.
\end{itemize}

\section{Thyroid Nodules and Cancer}\label{thyroid-nodules-and-cancer}

\subsection{Overview}\label{overview-3}

Thyroid nodules are uncommon in children but carry a higher risk of
malignancy (20--25\%) than in adults.

\subsection{Risk Factors}\label{risk-factors-1}

\begin{itemize}
\tightlist
\item
  Prior head and neck irradiation.
\item
  Family history of thyroid cancer (MEN2, familial medullary carcinoma).
\item
  Chronic lymphocytic thyroiditis.
\end{itemize}

\subsection{Types}\label{types-1}

\begin{enumerate}
\def\labelenumi{\arabic{enumi}.}
\tightlist
\item
  \textbf{Papillary carcinoma} -- most common (≈70--80\%).
\item
  \textbf{Follicular carcinoma} -- second most common.
\item
  \textbf{Medullary carcinoma} -- associated with MEN2 syndromes.
\item
  \textbf{Anaplastic carcinoma} -- extremely rare in children.
\end{enumerate}

\subsection{Clinical Features}\label{clinical-features-31}

\begin{itemize}
\tightlist
\item
  Solitary firm nodule, sometimes fixed to surrounding tissue.
\item
  Cervical lymphadenopathy.
\item
  Hoarseness or dysphagia (advanced disease).
\end{itemize}

\subsection{Investigations}\label{investigations-35}

\begin{itemize}
\tightlist
\item
  \textbf{Ultrasound:} solid hypoechoic nodule with microcalcifications.
\item
  \textbf{FNAC:} gold standard for cytological diagnosis.
\item
  \textbf{Thyroid function tests:} usually normal.
\item
  \textbf{Thyroglobulin:} tumour marker in differentiated cancers.
\end{itemize}

\subsection{Management}\label{management-26}

\begin{itemize}
\tightlist
\item
  \textbf{Surgery:} near-total or total thyroidectomy.
\item
  \textbf{Radioiodine ablation} for residual tissue or metastases.
\item
  \textbf{Thyroxine suppression therapy} to prevent TSH stimulation.
\item
  \textbf{Long-term follow-up} with thyroglobulin levels and imaging.
\end{itemize}

\section{Investigations in Suspected Thyroid
Disorders}\label{investigations-in-suspected-thyroid-disorders}

\begin{longtable}[]{@{}
  >{\raggedright\arraybackslash}p{(\linewidth - 2\tabcolsep) * \real{0.3521}}
  >{\raggedright\arraybackslash}p{(\linewidth - 2\tabcolsep) * \real{0.6479}}@{}}
\toprule\noalign{}
\begin{minipage}[b]{\linewidth}\raggedright
Investigation
\end{minipage} & \begin{minipage}[b]{\linewidth}\raggedright
Purpose
\end{minipage} \\
\midrule\noalign{}
\endhead
\bottomrule\noalign{}
\endlastfoot
Serum TSH and free T4 & Assess functional status \\
Thyroid antibodies & Autoimmune thyroid disease \\
Ultrasound scan & Structure, nodules, cysts \\
Radioiodine uptake scan & Evaluate the function and nodules \\
FNAC & Cytological diagnosis \\
Newborn screening & Early detection of congenital hypothyroidism \\
\end{longtable}

\section{Key Points in Paediatric Thyroid
Disorders}\label{key-points-in-paediatric-thyroid-disorders}

\begin{itemize}
\tightlist
\item
  Early detection and treatment of \textbf{congenital hypothyroidism}
  are crucial for preventing irreversible brain damage.
\item
  \textbf{Autoimmune thyroiditis} is the most common cause of acquired
  hypothyroidism in children.
\item
  \textbf{Graves' disease} is the leading cause of paediatric
  hyperthyroidism.
\item
  \textbf{Thyroid nodules in children} should be thoroughly evaluated
  due to a higher risk of malignancy.
\item
  Long-term follow-up with growth and developmental monitoring is
  essential in all thyroid disorders.
\end{itemize}

\section{Summary Table}\label{summary-table-4}

\begin{longtable}[]{@{}
  >{\raggedright\arraybackslash}p{(\linewidth - 6\tabcolsep) * \real{0.2500}}
  >{\raggedright\arraybackslash}p{(\linewidth - 6\tabcolsep) * \real{0.2500}}
  >{\raggedright\arraybackslash}p{(\linewidth - 6\tabcolsep) * \real{0.2500}}
  >{\raggedright\arraybackslash}p{(\linewidth - 6\tabcolsep) * \real{0.2500}}@{}}
\toprule\noalign{}
\begin{minipage}[b]{\linewidth}\raggedright
Disorder
\end{minipage} & \begin{minipage}[b]{\linewidth}\raggedright
Key Features
\end{minipage} & \begin{minipage}[b]{\linewidth}\raggedright
Diagnostic Findings
\end{minipage} & \begin{minipage}[b]{\linewidth}\raggedright
Main Treatment
\end{minipage} \\
\midrule\noalign{}
\endhead
\bottomrule\noalign{}
\endlastfoot
Congenital Hypothyroidism & Prolonged jaundice, macroglossia, poor
growth & ↑TSH, ↓T4 & Levothyroxine 10--15 µg/kg/day \\
Hashimoto's Thyroiditis & Goitre, growth failure, and autoimmune markers
& ↑TSH, anti-TPO + & Levothyroxine \\
Graves' Disease & Weight loss, tachycardia, exophthalmos & ↓TSH, ↑T3/T4,
TRAb + & Carbimazole ± β-blockers \\
Goitre (Iodine Deficiency) & Diffuse neck swelling & Normal or ↑TSH &
Iodine or thyroxine \\
Thyroid Cancer & Solitary firm nodule & FNAC positive & Surgery ±
radioiodine \\
\end{longtable}

\section{Suggested Reading}\label{suggested-reading-2}

\begin{enumerate}
\def\labelenumi{\arabic{enumi}.}
\tightlist
\item
  Nelson Textbook of Paediatrics, 22nd Edition.
\item
  Sperling MA. \emph{Pediatric Endocrinology}, 5th Edition.
\item
  WHO. \emph{Guidelines for the Management of Thyroid Disorders in
  Children}.
\item
  ESPE (European Society for Paediatric Endocrinology) Clinical Practice
  Recommendations, 2023.
\end{enumerate}

\chapter{Adrenal Gland Disorders}\label{adrenal-gland-disorders}

\section{Introduction}\label{introduction-45}

The adrenal glands are essential endocrine organs responsible for the
production of glucocorticoids, mineralocorticoids, and androgens. They
play a central role in maintaining metabolism, electrolyte balance, and
the stress response.\\
In children, adrenal gland disorders are particularly important because
they can interfere with growth, puberty, and survival during
physiological stress.

This lecture provides an overview of the anatomy and physiology of the
adrenal glands, followed by a discussion of common paediatric adrenal
disorders including \textbf{adrenal insufficiency}, \textbf{congenital
adrenal hyperplasia}, \textbf{Cushing syndrome}, \textbf{adrenal
tumours}, and \textbf{disorders of adrenal medulla}.

\section{Anatomy and Physiology}\label{anatomy-and-physiology-1}

Each adrenal gland is composed of two distinct regions:

\begin{enumerate}
\def\labelenumi{\arabic{enumi}.}
\tightlist
\item
  \textbf{Adrenal cortex} --- makes up about 90\% of the gland.\\
  It has three zones:

  \begin{itemize}
  \tightlist
  \item
    \textbf{Zona glomerulosa} → secretes \textbf{aldosterone}
    (mineralocorticoid)\\
  \item
    \textbf{Zona fasciculata} → secretes \textbf{cortisol}
    (glucocorticoid)\\
  \item
    \textbf{Zona reticularis} → secretes \textbf{androgens} (DHEA and
    androstenedione)
  \end{itemize}
\item
  \textbf{Adrenal medulla} --- produces catecholamines (epinephrine and
  norepinephrine) under sympathetic nervous control.
\end{enumerate}

\textbf{Regulation:}

\begin{itemize}
\item
  \textbf{ACTH (adrenocorticotropic hormone)} from the anterior
  pituitary regulates cortisol and androgen synthesis.
\item
  \textbf{Renin-angiotensin-aldosterone system (RAAS)} primarily
  controls aldosterone secretion.
\end{itemize}

\section{Major Adrenal Hormones and Their
Actions}\label{major-adrenal-hormones-and-their-actions}

\begin{longtable}[]{@{}
  >{\raggedright\arraybackslash}p{(\linewidth - 4\tabcolsep) * \real{0.1856}}
  >{\raggedright\arraybackslash}p{(\linewidth - 4\tabcolsep) * \real{0.1959}}
  >{\raggedright\arraybackslash}p{(\linewidth - 4\tabcolsep) * \real{0.6186}}@{}}
\toprule\noalign{}
\begin{minipage}[b]{\linewidth}\raggedright
Hormone
\end{minipage} & \begin{minipage}[b]{\linewidth}\raggedright
Site of secretion
\end{minipage} & \begin{minipage}[b]{\linewidth}\raggedright
Major effects
\end{minipage} \\
\midrule\noalign{}
\endhead
\bottomrule\noalign{}
\endlastfoot
Cortisol & Zona fasciculata & Gluconeogenesis, anti-inflammatory effect,
stress response \\
Aldosterone & Zona glomerulosa & Sodium retention, potassium excretion,
water balance \\
Androgens (DHEA) & Zona reticularis & Pubic/axillary hair development \\
Catecholamines & Medulla & ``Fight or flight'' response \\
\end{longtable}

\section{Classification of Adrenal Disorders in
Children}\label{classification-of-adrenal-disorders-in-children}

Adrenal disorders can be broadly classified into:

\begin{longtable}[]{@{}
  >{\raggedright\arraybackslash}p{(\linewidth - 2\tabcolsep) * \real{0.5000}}
  >{\raggedright\arraybackslash}p{(\linewidth - 2\tabcolsep) * \real{0.5000}}@{}}
\toprule\noalign{}
\begin{minipage}[b]{\linewidth}\raggedright
Category
\end{minipage} & \begin{minipage}[b]{\linewidth}\raggedright
Examples
\end{minipage} \\
\midrule\noalign{}
\endhead
\bottomrule\noalign{}
\endlastfoot
\textbf{Hypofunction (Adrenal insufficiency)} & Primary (Addison's
disease), Secondary (ACTH deficiency), Congenital adrenal hyperplasia \\
\textbf{Hyperfunction} & Cushing syndrome, Hyperaldosteronism,
Adrenogenital syndromes \\
\textbf{Adrenal masses/tumours} & Neuroblastoma, Adrenocortical
carcinoma, Adenoma \\
\textbf{Disorders of medulla} & Pheochromocytoma \\
\end{longtable}

\section{Adrenal Insufficiency}\label{adrenal-insufficiency}

\subsection{Definition}\label{definition-25}

A condition in which the adrenal glands do not produce sufficient
quantities of corticosteroids (cortisol ± aldosterone).

\subsection{Classification}\label{classification-7}

\begin{enumerate}
\def\labelenumi{\arabic{enumi}.}
\tightlist
\item
  \textbf{Primary adrenal insufficiency (Addison's disease):}

  \begin{itemize}
  \tightlist
  \item
    Destruction or dysfunction of the adrenal cortex.
  \item
    Cortisol and aldosterone deficiency.
  \end{itemize}
\item
  \textbf{Secondary adrenal insufficiency:}

  \begin{itemize}
  \tightlist
  \item
    Due to decreased ACTH secretion from the pituitary.
  \item
    Cortisol deficiency only.
  \end{itemize}
\item
  \textbf{Tertiary adrenal insufficiency:}

  \begin{itemize}
  \tightlist
  \item
    Resulting from hypothalamic dysfunction or prolonged exogenous
    steroid therapy.
  \end{itemize}
\end{enumerate}

\subsection{Aetiology in Children}\label{aetiology-in-children}

\begin{itemize}
\tightlist
\item
  \textbf{Autoimmune adrenalitis} (most common in developed settings).\\
\item
  \textbf{Congenital adrenal hyperplasia (CAH)}.\\
\item
  \textbf{Adrenal haemorrhage or infarction} (e.g.,
  Waterhouse--Friderichsen syndrome).\\
\item
  \textbf{Infections:} Tuberculosis, CMV, fungal infections.\\
\item
  \textbf{Infiltrative diseases:} Adrenoleukodystrophy, metastatic
  neuroblastoma.\\
\item
  \textbf{Surgical or drug-induced} (ketoconazole, etomidate).
\end{itemize}

\subsection{Clinical Features}\label{clinical-features-32}

\begin{itemize}
\tightlist
\item
  Failure to thrive and weight loss\\
\item
  Fatigue, weakness, lethargy\\
\item
  Hyperpigmentation (in primary cases)\\
\item
  Hypotension, dehydration, salt craving\\
\item
  Hypoglycaemia and hyponatremia\\
\item
  Nausea, vomiting, abdominal pain
\end{itemize}

\textbf{In neonates:} prolonged jaundice, shock, or ambiguous genitalia
(in CAH).

\subsection{Investigations}\label{investigations-36}

\begin{itemize}
\tightlist
\item
  \textbf{Serum cortisol:} low morning level (\textless{} 100 nmol/L is
  suggestive).\\
\item
  \textbf{Plasma ACTH:} elevated in primary, low in secondary.\\
\item
  \textbf{Electrolytes:} hyponatremia, hyperkalemia, hypoglycemia.\\
\item
  \textbf{ACTH stimulation test (Synacthen test):}

  \begin{itemize}
  \tightlist
  \item
    Failure of cortisol to rise confirms adrenal insufficiency.\\
  \end{itemize}
\item
  \textbf{Adrenal autoantibodies} for autoimmune cause.\\
\item
  \textbf{Imaging:} CT or MRI of the adrenal glands when a structural
  cause issuspected.
\end{itemize}

\subsection{Management}\label{management-27}

\subsubsection{Acute Adrenal Crisis
(Emergency)}\label{acute-adrenal-crisis-emergency}

\textbf{Presentation:} Shock, vomiting, dehydration, hypoglycaemia.

\textbf{Management steps:}

\begin{enumerate}
\def\labelenumi{\arabic{enumi}.}
\tightlist
\item
  Immediate IV access.\\
\item
  \textbf{Hydrocortisone:} 50 mg/m² IV stat, then 50--100 mg/m²/day
  divided q6h.\\
\item
  \textbf{IV fluids:} 0.9\% saline with 5\% dextrose for rehydration and
  glucose correction.\\
\item
  Correct electrolytes.\\
\item
  Identify and treat precipitating cause (infection, steroid withdrawal,
  stress).
\end{enumerate}

\subsubsection{Chronic Replacement
Therapy}\label{chronic-replacement-therapy}

\begin{itemize}
\tightlist
\item
  \textbf{Glucocorticoid:} Hydrocortisone 8--10 mg/m²/day orally
  (divided 3 doses).\\
\item
  \textbf{Mineralocorticoid:} Fludrocortisone 0.05--0.2 mg daily.\\
\item
  \textbf{Education:} ``Sick day'' rule---double or triple steroid dose
  during illness or surgery.\\
\item
  \textbf{Monitoring:} Growth, weight, blood pressure, electrolytes.
\end{itemize}

\section{Congenital Adrenal Hyperplasia
(CAH)}\label{congenital-adrenal-hyperplasia-cah-1}

\subsection{Definition}\label{definition-26}

A group of \textbf{autosomal recessive} enzyme defects in cortisol
biosynthesis, leading to cortisol deficiency, variable aldosterone
deficiency, and androgen excess.

\subsection{Most Common Type}\label{most-common-type}

\begin{itemize}
\tightlist
\item
  \textbf{21-hydroxylase deficiency (≈90--95\%)}
\end{itemize}

\subsection{Pathophysiology}\label{pathophysiology-30}

\begin{itemize}
\tightlist
\item
  Block in cortisol synthesis → increased ACTH → adrenal hyperplasia.\\
\item
  Excess precursors diverted to androgen pathway → virilisation.\\
\item
  In salt-wasting forms → aldosterone deficiency causes hyponatremia and
  hyperkalemia.
\end{itemize}

\subsection{Classification}\label{classification-8}

\begin{longtable}[]{@{}
  >{\raggedright\arraybackslash}p{(\linewidth - 2\tabcolsep) * \real{0.3333}}
  >{\raggedright\arraybackslash}p{(\linewidth - 2\tabcolsep) * \real{0.6667}}@{}}
\toprule\noalign{}
\begin{minipage}[b]{\linewidth}\raggedright
Type
\end{minipage} & \begin{minipage}[b]{\linewidth}\raggedright
Features
\end{minipage} \\
\midrule\noalign{}
\endhead
\bottomrule\noalign{}
\endlastfoot
\textbf{Classical (Severe)} & Salt-wasting or simple virilising \\
\textbf{Non-classical (Mild)} & Partial enzyme deficiency, late onset
virilisation \\
\end{longtable}

\subsection{Clinical Features}\label{clinical-features-33}

\subsubsection{\texorpdfstring{\textbf{Salt-wasting
type}}{Salt-wasting type}}\label{salt-wasting-type}

\begin{itemize}
\tightlist
\item
  Neonatal onset (first 2 weeks).\\
\item
  Vomiting, dehydration, weight loss, shock.\\
\item
  Hyponatremia, hyperkalemia, hypoglycemia.\\
\item
  Female: ambiguous genitalia (virilisation).\\
\item
  Male: normal genitalia but presents with adrenal crisis.
\end{itemize}

\subsubsection{\texorpdfstring{\textbf{Simple virilising
type}}{Simple virilising type}}\label{simple-virilising-type}

\begin{itemize}
\tightlist
\item
  Ambiguous genitalia in females at birth.\\
\item
  Precocious puberty, accelerated bone age.\\
\item
  No salt wasting.
\end{itemize}

\subsubsection{\texorpdfstring{\textbf{Non-classical
type}}{Non-classical type}}\label{non-classical-type}

\begin{itemize}
\tightlist
\item
  Later onset (childhood/adolescence).\\
\item
  Hirsutism, acne, irregular menses in girls.\\
\item
  Early pubarche in boys.
\end{itemize}

\subsection{Diagnosis}\label{diagnosis-15}

\begin{itemize}
\tightlist
\item
  \textbf{17-hydroxyprogesterone (17-OHP):} elevated (\textgreater30
  nmol/L basal or post-ACTH).\\
\item
  \textbf{Electrolytes:} hyponatremia, hyperkalemia.\\
\item
  \textbf{Genetic testing:} CYP21A2 mutations.\\
\item
  \textbf{Ultrasound:} assess internal genitalia in virilised females.
\end{itemize}

\subsection{Management}\label{management-28}

\subsubsection{Hormone Replacement}\label{hormone-replacement-1}

\begin{itemize}
\tightlist
\item
  \textbf{Hydrocortisone:} 10--15 mg/m²/day (3 divided doses).\\
\item
  \textbf{Fludrocortisone:} 0.05--0.2 mg/day in salt-wasting forms.\\
\item
  \textbf{Sodium supplementation:} especially in neonates.
\end{itemize}

\subsubsection{Surgical Correction}\label{surgical-correction}

\begin{itemize}
\tightlist
\item
  \textbf{Genitoplasty} for virilised females (timing individualized).
\end{itemize}

\subsubsection{Long-Term Care}\label{long-term-care}

\begin{itemize}
\tightlist
\item
  Growth monitoring (avoid overtreatment leading to growth
  suppression).\\
\item
  Periodic bone age assessment.\\
\item
  Psychological and genetic counselling.\\
\item
  Lifelong follow-up in endocrinology clinic.
\end{itemize}

\section{Cushing Syndrome}\label{cushing-syndrome-1}

\subsection{Definition}\label{definition-27}

A state of \textbf{chronic glucocorticoid excess}, either due to
endogenous overproduction or exogenous steroid use.

\subsection{Aetiology}\label{aetiology-21}

\begin{longtable}[]{@{}
  >{\raggedright\arraybackslash}p{(\linewidth - 2\tabcolsep) * \real{0.5000}}
  >{\raggedright\arraybackslash}p{(\linewidth - 2\tabcolsep) * \real{0.5000}}@{}}
\toprule\noalign{}
\begin{minipage}[b]{\linewidth}\raggedright
Type
\end{minipage} & \begin{minipage}[b]{\linewidth}\raggedright
Cause
\end{minipage} \\
\midrule\noalign{}
\endhead
\bottomrule\noalign{}
\endlastfoot
\textbf{Exogenous} & Prolonged corticosteroid therapy (most common in
children) \\
\textbf{Endogenous} & ACTH-secreting pituitary adenoma (Cushing
disease), adrenal adenoma or carcinoma, ectopic ACTH secretion (rare) \\
\end{longtable}

\subsection{Clinical Features}\label{clinical-features-34}

\begin{itemize}
\tightlist
\item
  Growth failure and weight gain\\
\item
  Moon facies, truncal obesity, buffalo hump\\
\item
  Striae (purple), acne, hirsutism\\
\item
  Hypertension, glucose intolerance\\
\item
  Mood changes (depression, irritability)\\
\item
  Osteopenia or fractures\\
\item
  Delayed puberty or amenorrhea
\end{itemize}

\subsection{Investigations}\label{investigations-37}

\begin{enumerate}
\def\labelenumi{\arabic{enumi}.}
\tightlist
\item
  \textbf{Screening tests}

  \begin{itemize}
  \tightlist
  \item
    24-hour urinary free cortisol: elevated.
  \item
    Late-night salivary cortisol: loss of diurnal variation.
  \item
    Low-dose dexamethasone suppression test: failure to suppress
    cortisol.
  \end{itemize}
\item
  \textbf{Differentiation tests}

  \begin{itemize}
  \tightlist
  \item
    Plasma ACTH:

    \begin{itemize}
    \tightlist
    \item
      Low → adrenal cause\\
    \item
      Normal/high → ACTH-dependent cause.
    \end{itemize}
  \end{itemize}
\item
  \textbf{Imaging}

  \begin{itemize}
  \tightlist
  \item
    MRI pituitary (Cushing disease).\\
  \item
    CT or MRI adrenals for adenoma/carcinoma.
  \end{itemize}
\end{enumerate}

\subsection{Management}\label{management-29}

\begin{itemize}
\tightlist
\item
  \textbf{Exogenous:} Gradual tapering of steroid dose.\\
\item
  \textbf{Pituitary adenoma:} Trans-sphenoidal surgery.\\
\item
  \textbf{Adrenal tumour:} Surgical excision ± radiotherapy.\\
\item
  \textbf{Medical therapy:} Ketoconazole, metyrapone, or mitotane when
  surgery is contraindicated.
\end{itemize}

\subsection{Prognosis}\label{prognosis-32}

\begin{itemize}
\tightlist
\item
  Normal growth may resume post-treatment, but final height may be
  compromised if prolonged disease.\\
\item
  Requires careful perioperative steroid coverage to prevent adrenal
  insufficiency.
\end{itemize}

\section{Adrenal Tumours}\label{adrenal-tumours}

\subsection{Overview}\label{overview-4}

Adrenal tumours in children may be \textbf{benign (adenoma)} or
\textbf{malignant (carcinoma)}, and may secrete hormones or be
non-functioning.

\subsubsection{Adrenocortical Tumours}\label{adrenocortical-tumours}

\begin{itemize}
\tightlist
\item
  \textbf{Adenomas:} often hormonally active (Cushing, virilisation).\\
\item
  \textbf{Carcinomas:} aggressive, may secrete multiple hormones.
\end{itemize}

\textbf{Clinical features:}

\begin{itemize}
\tightlist
\item
  Rapid virilisation in girls.\\
\item
  Cushingoid features.\\
\item
  Precocious puberty in boys.\\
\item
  Abdominal mass or pain.
\end{itemize}

\textbf{Diagnosis:}

\begin{itemize}
\tightlist
\item
  Elevated serum and urinary steroid precursors.\\
\item
  Imaging (CT/MRI): large irregular mass.\\
\item
  Histopathology confirms diagnosis.
\end{itemize}

\textbf{Treatment:}

\begin{itemize}
\tightlist
\item
  Surgical excision is mainstay.\\
\item
  Chemotherapy for carcinoma (mitotane, cisplatin-based regimens).\\
\item
  Lifelong hormonal follow-up.
\end{itemize}

\subsubsection{Neuroblastoma}\label{neuroblastoma-1}

A malignant tumour arising from \textbf{neural crest cells} of the
adrenal medulla.

\textbf{Clinical features:}

\begin{itemize}
\tightlist
\item
  Abdominal mass, weight loss, hypertension.\\
\item
  Opsoclonus--myoclonus syndrome (rare paraneoplastic sign).
\end{itemize}

\textbf{Investigations:}

\begin{itemize}
\tightlist
\item
  Elevated urinary catecholamine metabolites (VMA, HVA).\\
\item
  Imaging: calcified adrenal mass.\\
\item
  Bone marrow or MIBG scan for metastases.
\end{itemize}

\textbf{Treatment:}

\begin{itemize}
\tightlist
\item
  Surgery, chemotherapy, and radiotherapy as appropriate.\\
\item
  Prognosis depends on stage and age (better in \textless1 year).
\end{itemize}

\section{Disorders of the Adrenal
Medulla}\label{disorders-of-the-adrenal-medulla}

\subsection{Pheochromocytoma}\label{pheochromocytoma}

A rare catecholamine-secreting tumour of chromaffin cells.

\subsubsection{Clinical Features}\label{clinical-features-35}

\begin{itemize}
\tightlist
\item
  Paroxysmal or sustained hypertension.\\
\item
  Headache, palpitations, sweating.\\
\item
  Pallor, tremor, anxiety.\\
\item
  May be part of \textbf{MEN2A/2B syndromes}.
\end{itemize}

\subsubsection{Investigations}\label{investigations-38}

\begin{itemize}
\tightlist
\item
  24-hour urinary \textbf{metanephrines and catecholamines}
  (diagnostic).\\
\item
  Plasma free metanephrines.\\
\item
  MRI or MIBG scan to localize tumour.
\end{itemize}

\subsubsection{Management}\label{management-30}

\begin{enumerate}
\def\labelenumi{\arabic{enumi}.}
\item
  \textbf{Preoperative preparation:}

  \begin{itemize}
  \tightlist
  \item
    Alpha-blockade (phenoxybenzamine) for 1--2 weeks.\\
  \item
    Then beta-blocker (propranolol) if tachycardia.
  \end{itemize}
\item
  \textbf{Definitive surgery:} Adrenalectomy.\\
\item
  \textbf{Postoperative monitoring} for hypotension or recurrence.
\end{enumerate}

\section{Investigation Summary Table}\label{investigation-summary-table}

\begin{longtable}[]{@{}
  >{\raggedright\arraybackslash}p{(\linewidth - 4\tabcolsep) * \real{0.2875}}
  >{\raggedright\arraybackslash}p{(\linewidth - 4\tabcolsep) * \real{0.3375}}
  >{\raggedright\arraybackslash}p{(\linewidth - 4\tabcolsep) * \real{0.3750}}@{}}
\toprule\noalign{}
\begin{minipage}[b]{\linewidth}\raggedright
Disorder
\end{minipage} & \begin{minipage}[b]{\linewidth}\raggedright
Key Test
\end{minipage} & \begin{minipage}[b]{\linewidth}\raggedright
Diagnostic Findings
\end{minipage} \\
\midrule\noalign{}
\endhead
\bottomrule\noalign{}
\endlastfoot
Adrenal insufficiency & ACTH stimulation & Low cortisol response \\
CAH & 17-hydroxyprogesterone & Elevated \\
Cushing syndrome & Dexamethasone suppression & Failure to suppress
cortisol \\
Adrenocortical tumour & Urinary steroids, imaging & Elevated
androgens/cortisol \\
Pheochromocytoma & Urinary metanephrines & Elevated catecholamines \\
\end{longtable}

\section{Long-Term Management
Considerations}\label{long-term-management-considerations}

\begin{itemize}
\tightlist
\item
  \textbf{Hormone replacement therapy} requires frequent adjustment as
  the child grows.\\
\item
  \textbf{Stress dosing} of steroids during illness or surgery is
  lifesaving.\\
\item
  \textbf{Growth monitoring} is crucial to avoid overtreatment or
  under-replacement.\\
\item
  \textbf{Psychological support} is vital for children with virilisation
  or chronic illness.\\
\item
  \textbf{Family education} on emergency management (e.g., injectable
  hydrocortisone) is essential.
\end{itemize}

\section{Summary Points}\label{summary-points}

\begin{itemize}
\tightlist
\item
  Adrenal gland disorders in children range from life-threatening
  adrenal insufficiency to hormone-secreting tumours.\\
\item
  \textbf{Congenital adrenal hyperplasia} is the most frequent inherited
  adrenal disorder.\\
\item
  \textbf{Adrenal crisis} is a paediatric emergency; treat promptly with
  IV fluids and hydrocortisone.\\
\item
  \textbf{Cushing syndrome} in children commonly results from exogenous
  steroids.\\
\item
  \textbf{Pheochromocytoma}, though rare, should be suspected in
  children with unexplained hypertension.\\
\item
  Lifelong follow-up and interdisciplinary care (paediatric
  endocrinology, surgery, psychology) improve outcomes.
\end{itemize}

\section{Suggested Reading}\label{suggested-reading-3}

\begin{enumerate}
\def\labelenumi{\arabic{enumi}.}
\tightlist
\item
  Nelson Textbook of Pediatrics, 22nd Edition.\\
\item
  Brook's \emph{Clinical Pediatric Endocrinology}, 7th Edition.\\
\item
  Speiser PW, et al.~\emph{ESPE/LWPES Consensus on CAH Management},
  \emph{J Clin Endocrinol Metab} 2018.\\
\item
  Husebye ES et al.~\emph{Diagnosis and management of primary adrenal
  insufficiency in children}, \emph{Lancet Diabetes Endocrinol}, 2021.\\
\item
  WHO. \emph{Paediatric Endocrine Disorders: A Practical Guide for
  Clinicians}, 2023.
\end{enumerate}

\chapter{Pituitary Gland Disorders}\label{pituitary-gland-disorders}

\section{Introduction}\label{introduction-46}

The pituitary gland, often called the \textbf{``master gland''}, plays a
central role in the regulation of growth, metabolism, reproduction, and
stress response. Its hormones influence nearly every endocrine organ,
including the thyroid, adrenal glands, and gonads. Disorders of the
pituitary gland, therefore, have wide-ranging systemic effects and are
of great importance in paediatric practice.

In Ghana, pituitary disorders in children are likely underdiagnosed due
to limited access to hormonal assays and imaging facilities. Many
children with growth failure or delayed puberty are labelled as
``constitutional'' cases without proper evaluation. Moreover, paediatric
endocrinology services are still emerging, and awareness among frontline
health workers remains low. Recognising pituitary disorders early can
prevent serious complications, including permanent growth failure,
adrenal crisis, or infertility.

This chapter provides an overview of pituitary anatomy and physiology,
discusses common and important pituitary disorders in childhood, and
outlines their diagnosis and management, with emphasis on
\textbf{practical approaches in resource-limited settings}.

\section{Anatomy and Physiology of the Pituitary
Gland}\label{anatomy-and-physiology-of-the-pituitary-gland}

The pituitary gland is a small, oval-shaped endocrine organ located at
the base of the brain in the \textbf{sella turcica}, just below the
hypothalamus. It is connected to the hypothalamus by the
\textbf{pituitary stalk (infundibulum)} and is divided into two main
parts:

\begin{itemize}
\tightlist
\item
  \textbf{Anterior pituitary (adenohypophysis)} --- produces six key
  hormones:

  \begin{itemize}
  \tightlist
  \item
    Growth hormone (GH)
  \item
    Adrenocorticotropic hormone (ACTH)
  \item
    Thyroid-stimulating hormone (TSH)
  \item
    Luteinizing hormone (LH)
  \item
    Follicle-stimulating hormone (FSH)
  \item
    Prolactin (PRL)
  \end{itemize}
\item
  \textbf{Posterior pituitary (neurohypophysis)} --- stores and releases
  two hypothalamic hormones:

  \begin{itemize}
  \tightlist
  \item
    Antidiuretic hormone (ADH, also called vasopressin)
  \item
    Oxytocin
  \end{itemize}
\end{itemize}

The hypothalamus regulates pituitary function through releasing and
inhibiting hormones. These hormones are secreted into the
\textbf{hypophyseal portal circulation}, linking the hypothalamus and
pituitary into an integrated \textbf{hypothalamic--pituitary axis}.

\section{Classification of Pituitary
Disorders}\label{classification-of-pituitary-disorders}

Pituitary gland disorders in children can be broadly classified as
follows:

\begin{enumerate}
\def\labelenumi{\arabic{enumi}.}
\tightlist
\item
  \textbf{Hypopituitarism} --- deficiency of one or more pituitary
  hormones
\item
  \textbf{Hyperpituitarism} --- excess secretion of one or more hormones
\item
  \textbf{Posterior pituitary disorders} --- abnormalities of ADH
  secretion (diabetes insipidus or SIADH)
\item
  \textbf{Structural lesions} --- pituitary adenomas, cysts,
  craniopharyngiomas, or infiltrative diseases
\end{enumerate}

Each disorder manifests differently, depending on which hormones are
affected and at what stage of development the dysfunction occurs.

\section{Hypopituitarism}\label{hypopituitarism}

\subsection{Definition}\label{definition-28}

Hypopituitarism refers to partial or complete deficiency of anterior
and/or posterior pituitary hormones. It may be \textbf{congenital}
(present at birth) or \textbf{acquired} (occurring later due to injury,
tumour, or infection).

\subsection{Causes}\label{causes-8}

\textbf{Congenital causes} include: - Midline developmental defects
(septo-optic dysplasia, holoprosencephaly)

\begin{itemize}
\tightlist
\item
  Genetic mutations affecting pituitary transcription factors (PROP1,
  POU1F1)
\item
  Perinatal asphyxia or trauma
\item
  Structural anomalies of the pituitary or hypothalamus on MRI
\end{itemize}

\textbf{Acquired causes} include: - Central nervous system tumours
(especially craniopharyngioma)

\begin{itemize}
\tightlist
\item
  Head trauma or irradiation
\item
  Infections such as meningitis or tuberculosis
\item
  Infiltrative diseases (e.g., Langerhans cell histiocytosis)
\end{itemize}

In Ghana, birth asphyxia, CNS infections, and head trauma are likely the
commonest causes of acquired hypopituitarism.

\subsection{Clinical Features}\label{clinical-features-36}

The clinical presentation depends on the number and severity of hormone
deficiencies:

\begin{itemize}
\tightlist
\item
  \textbf{Growth hormone deficiency (GHD):} short stature, increased fat
  mass, immature face, delayed dentition
\item
  \textbf{TSH deficiency:} secondary hypothyroidism (fatigue, poor
  growth, cold intolerance)
\item
  \textbf{ACTH deficiency:} adrenal insufficiency (hypoglycaemia,
  hypotension, fatigue)
\item
  \textbf{Gonadotropin deficiency:} delayed or absent puberty
\item
  \textbf{Prolactin deficiency:} failure of lactation in post-partum
  females (less relevant in children)
\end{itemize}

Infants with panhypopituitarism may present with \textbf{hypoglycaemia,
prolonged jaundice, micropenis, and poor feeding}. Without recognition,
mortality may occur due to adrenal crisis.

\subsection{Diagnosis}\label{diagnosis-16}

Diagnosis requires both \textbf{biochemical assessment} and
\textbf{neuroimaging}.

\begin{itemize}
\tightlist
\item
  Basal hormone levels: TSH, free T₄, cortisol, GH, IGF-1, LH/FSH,
  prolactin
\item
  Dynamic testing (e.g., insulin tolerance test, ACTH stimulation test)
  --- when available
\item
  MRI of the brain to identify structural anomalies or tumours
\end{itemize}

In Ghana, access to dynamic testing may be limited to teaching hospitals
such as KATH or KBTH. Clinicians in regional hospitals may rely on
\textbf{clinical findings}, basic thyroid function tests, and growth
parameters to guide management.

\subsection{Management}\label{management-31}

Treatment focuses on \textbf{hormone replacement} and \textbf{management
of underlying causes}.

\begin{itemize}
\tightlist
\item
  \textbf{Hydrocortisone} for ACTH deficiency (8--10 mg/m²/day in
  divided doses)
\item
  \textbf{Levothyroxine} for TSH deficiency
\item
  \textbf{Recombinant GH} for GHD (0.025--0.035 mg/kg/day
  subcutaneously)
\item
  \textbf{Sex steroids} for pubertal induction when older
\end{itemize}

In resource-limited settings, the high cost of GH therapy limits access,
and prioritization is often necessary. Multidisciplinary care involving
paediatric endocrinologists, dietitians, and psychologists yields the
best outcomes.

\section{Growth Hormone Deficiency
(GHD)}\label{growth-hormone-deficiency-ghd-1}

GHD is the most common isolated pituitary hormone deficiency in
children. It may be \textbf{idiopathic}, \textbf{genetic}, or
\textbf{secondary} to structural lesions.

\subsection{Clinical Features}\label{clinical-features-37}

Children typically present with:

\begin{itemize}
\tightlist
\item
  Short stature with normal body proportions
\item
  Delayed skeletal maturation
\item
  Chubby face and truncal obesity
\item
  Delayed dentition and puberty
\end{itemize}

In Ghana, such children are often brought late for evaluation, sometimes
after unsuccessful use of herbal remedies or growth tonics.

\subsection{Diagnosis}\label{diagnosis-17}

\begin{itemize}
\tightlist
\item
  Low serum IGF-1 and IGFBP-3 levels
\item
  GH stimulation tests (e.g., clonidine, insulin) to confirm deficiency
\item
  MRI to assess pituitary morphology
\end{itemize}

\subsection{Treatment}\label{treatment-18}

Daily subcutaneous \textbf{recombinant GH} therapy leads to excellent
catch-up growth if started early. Monitoring of growth velocity,
pubertal development, and thyroid function is essential.

Where GH therapy is unavailable, \textbf{nutritional optimisation},
\textbf{thyroid screening}, and \textbf{regular monitoring} are still
beneficial. Psychosocial support for affected children is also
essential.

\section{Hyperpituitarism}\label{hyperpituitarism}

\subsection{Definition and Causes}\label{definition-and-causes}

Hyperpituitarism refers to excessive secretion of one or more pituitary
hormones, often due to \textbf{pituitary adenomas}. In children, these
are usually \textbf{benign} but can cause significant morbidity by
compressing adjacent structures or overproducing hormones.

Common examples include:

\begin{enumerate}
\def\labelenumi{\arabic{enumi}.}
\tightlist
\item
  \textbf{GH-secreting adenoma:} gigantism
\item
  \textbf{Prolactin-secreting adenoma (prolactinoma):} galactorrhoea,
  delayed puberty
\item
  \textbf{ACTH-secreting adenoma:} Cushing's disease
\end{enumerate}

\subsection{Growth Hormone Excess
(Gigantism)}\label{growth-hormone-excess-gigantism}

\subsubsection{Clinical Features}\label{clinical-features-38}

\begin{itemize}
\tightlist
\item
  Rapid linear growth with tall stature
\item
  Coarse facial features, broad hands and feet
\item
  Headaches and visual disturbances due to tumour mass
\item
  Glucose intolerance or diabetes mellitus
\end{itemize}

\subsection{Diagnosis}\label{diagnosis-18}

\begin{itemize}
\tightlist
\item
  Elevated IGF-1 levels
\item
  Failure of GH suppression after oral glucose load
\item
  Pituitary MRI showing macroadenoma
\end{itemize}

\subsubsection{Management}\label{management-32}

\begin{itemize}
\tightlist
\item
  \textbf{Trans-sphenoidal surgery} (definitive)
\item
  \textbf{Somatostatin analogues (octreotide)} if surgery fails or is
  not available
\item
  \textbf{Radiation therapy} for residual tumour
\end{itemize}

In Ghana, collaboration between neurosurgery and endocrinologyis
essential but often limited to tertiary centres.

\subsection{Prolactinoma}\label{prolactinoma}

Prolactinomas cause hyperprolactinaemia, leading to \textbf{delayed
puberty}, \textbf{amenorrhoea}, or \textbf{galactorrhoea}.

\begin{itemize}
\tightlist
\item
  Diagnosis: elevated serum prolactin, pituitary MRI
\item
  Treatment: \textbf{dopamine agonists} (cabergoline or bromocriptine)
\item
  Surgery if medical therapy fails
\end{itemize}

\section{Posterior Pituitary
Disorders}\label{posterior-pituitary-disorders}

The posterior pituitary releases \textbf{antidiuretic hormone (ADH)},
which regulates water balance. Two major disorders may occur:

\subsection{Diabetes Insipidus (DI)}\label{diabetes-insipidus-di}

\subsubsection{Pathophysiology}\label{pathophysiology-31}

Caused by deficiency (central DI) or resistance (nephrogenic DI) to ADH.
Central DI may result from:

\begin{itemize}
\tightlist
\item
  Head trauma
\item
  Craniopharyngioma or other tumours
\item
  Post-surgical damage
\item
  CNS infections such as meningitis or tuberculosis
\end{itemize}

\subsubsection{Clinical Features}\label{clinical-features-39}

\begin{itemize}
\tightlist
\item
  Polyuria and polydipsia
\item
  Nocturia or enuresis
\item
  Dehydration and hypernatraemia
\item
  Low urine osmolality despite high plasma osmolality
\end{itemize}

\subsubsection{Diagnosis}\label{diagnosis-19}

\begin{itemize}
\tightlist
\item
  Water deprivation test (if facilities allow)
\item
  Low urine specific gravity (\textless1.005)
\item
  Response to \textbf{desmopressin} confirms central DI
\end{itemize}

\subsubsection{Management}\label{management-33}

\begin{itemize}
\tightlist
\item
  \textbf{Desmopressin (DDAVP)} intranasal or oral
\item
  Adequate hydration
\item
  Treatment of the underlying cause
\end{itemize}

Access to desmopressin can be a challenge in Ghana, requiring
coordination with major teaching hospitals or pharmacies that stock
specialised endocrinology drugs.

\subsection{Syndrome of Inappropriate ADH Secretion
(SIADH)}\label{syndrome-of-inappropriate-adh-secretion-siadh}

SIADH results from \textbf{excess ADH secretion}, leading to
\textbf{hyponatraemia} with low serum osmolality and concentrated urine.

Common causes include CNS infections, pulmonary disease, or drugs (e.g.,
carbamazepine).\\
Management involves \textbf{fluid restriction}, \textbf{salt
supplementation}, and addressing the underlying cause.

\section{Craniopharyngioma and Other Structural
Lesions}\label{craniopharyngioma-and-other-structural-lesions}

\textbf{Craniopharyngioma} is a benign but locally aggressive tumour
arising from remnants of Rathke's pouch. It is the \textbf{most common
suprasellar tumour} in children and a leading cause of acquired
hypopituitarism.

\subsection{Clinical Features}\label{clinical-features-40}

\begin{itemize}
\tightlist
\item
  Growth failure and delayed puberty
\item
  Visual impairment (bitemporal hemianopia)
\item
  Headache, vomiting due to raised intracranial pressure
\item
  Diabetes insipidus or other pituitary deficiencies
\end{itemize}

\subsection{Diagnosis}\label{diagnosis-20}

\begin{itemize}
\tightlist
\item
  MRI of the brain: cystic, calcified suprasellar mass
\item
  Endocrine evaluation for pituitary dysfunction
\end{itemize}

\subsection{Management}\label{management-34}

\begin{itemize}
\tightlist
\item
  \textbf{Surgical resection} (ideally, subtotal to preserve function)
\item
  \textbf{Postoperative hormone replacement} (hydrocortisone, thyroxine,
  GH, DDAVP)
\item
  \textbf{Radiation therapy} for residual tumour
\end{itemize}

Children require \textbf{lifelong follow-up} due to the risk of
recurrence and permanent panhypopituitarism.

In Ghana, neurosurgical expertise exists in teaching hospitals, but
postoperative endocrine support is sometimes inadequate, underscoring
the need for closer multidisciplinary collaboration.

\section{Approach to a Child with Suspected Pituitary
Disorder}\label{approach-to-a-child-with-suspected-pituitary-disorder}

A structured evaluation includes:

\begin{enumerate}
\def\labelenumi{\arabic{enumi}.}
\tightlist
\item
  \textbf{History:} growth pattern, perinatal events, head trauma, CNS
  infections, puberty timing, polyuria/polydipsia
\item
  \textbf{Examination:} height/weight, visual fields, facial features,
  pubertal staging, hydration status
\item
  \textbf{Baseline Investigations:}

  \begin{itemize}
  \tightlist
  \item
    Serum electrolytes, glucose
  \item
    Thyroid function tests
  \item
    Morning cortisol
  \item
    IGF-1 levels
  \item
    LH, FSH, prolactin
  \end{itemize}
\item
  \textbf{Imaging:} Pituitary MRI if available
\item
  \textbf{Referral:} to a paediatric endocrinologist for specialized
  testing and management
\end{enumerate}

\section{Challenges in the Ghanaian
Context}\label{challenges-in-the-ghanaian-context-1}

\begin{itemize}
\tightlist
\item
  Limited access to hormone assays (especially GH, IGF-1, cortisol)
\item
  Lack of standardised growth monitoring in many clinics
\item
  Cost and availability of hormone replacement (GH, desmopressin)
\item
  Few centres with multidisciplinary paediatric endocrine teams
\end{itemize}

Improving training in \textbf{growth assessment}, expanding
\textbf{laboratory capacity}, and integrating \textbf{endocrine
screening} into routine child health services are key to improving
diagnosis and management outcomes.

\section{Conclusion}\label{conclusion-30}

Pituitary gland disorders in children, though relatively uncommon, have
far-reaching effects on growth, development, and overall health. Early
recognition and prompt hormone replacement can transform outcomes.

In Ghana and similar settings, raising clinicians' awareness, ensuring
theavailability of key diagnostic tests, and strengthening referral
pathways are vital steps forward. Every child with unexplained growth
failure, delayed puberty, or abnormal thirst and urination deserves a
careful evaluation of pituitary function.

\section{Further Reading}\label{further-reading-2}

\begin{enumerate}
\def\labelenumi{\arabic{enumi}.}
\tightlist
\item
  Nelson Textbook of Pediatrics, 22nd Edition.
\item
  Brook CGD, Clayton PE, Brown RS. \emph{Clinical Pediatric
  Endocrinology}, 7th Edition.
\item
  Sperling MA. \emph{Pediatric Endocrinology}, 5th Edition.
\item
  Ghana Health Service. \emph{National Child Growth Monitoring
  Guidelines}, 2022.
\item
  Osei-Kwakye K, et al.~``Challenges in Diagnosing Paediatric Endocrine
  Disorders in Low-Resource Settings: The Ghanaian Perspective.''
  \emph{West African Journal of Medicine}, 2021.
\item
  WHO. \emph{Endocrine Disorders in Childhood: Diagnosis and Management
  in Resource-Limited Settings},

  \begin{enumerate}
  \def\labelenumii{\arabic{enumii}.}
  \setcounter{enumii}{2022}
  \tightlist
  \item
  \end{enumerate}
\end{enumerate}

\chapter{Calcium and Bone Metabolism}\label{calcium-and-bone-metabolism}

\section{Introduction}\label{introduction-47}

Calcium and bone metabolism form a vital part of paediatric growth and
development. Bones are not static structures; they are dynamic organs
that undergo continuous remodeling, with bone formation and resorption
occurring simultaneously. This dynamic process depends heavily on an
intricate balance between calcium homeostasis, hormonal regulation, and
adequate nutrition.

Calcium is not only essential for skeletal integrity but also for
neuromuscular function, blood coagulation, and intracellular signaling.
Disturbances in calcium or bone metabolism can lead to conditions such
as \textbf{rickets}, \textbf{osteopenia}, \textbf{hypocalcaemia}, or
\textbf{hypercalcaemia} --- all of which are relatively common in
paediatric practice in Ghana and sub-Saharan Africa.

Understanding calcium and bone metabolism is fundamental for diagnosing
and managing paediatric metabolic bone disorders, especially given the
high burden of \textbf{nutritional rickets} and \textbf{vitamin D
deficiency} observed in resource-limited settings.

\section{Overview of Bone Structure and
Function}\label{overview-of-bone-structure-and-function}

Bone serves multiple physiological roles:

\begin{itemize}
\tightlist
\item
  \textbf{Structural support and protection} for internal organs.
\item
  \textbf{Reservoir for minerals}, mainly calcium and phosphate.
\item
  \textbf{Site of haematopoiesis} within the bone marrow.
\item
  \textbf{Endocrine organ}, influencing energy metabolism through
  osteocalcin secretion.
\end{itemize}

Bone tissue consists of:

\begin{itemize}
\tightlist
\item
  \textbf{Organic matrix (osteoid):} Composed mainly of type I collagen,
  providing tensile strength.
\item
  \textbf{Inorganic mineral component:} Primarily calcium hydroxyapatite
  {[}Ca₁₀(PO₄)₆(OH)₂{]}, which gives hardness and rigidity.
\item
  \textbf{Cells:} Including osteoblasts (bone-forming), osteoclasts
  (bone-resorbing), and osteocytes (mature cells embedded in bone).
\end{itemize}

The balance between bone formation and resorption maintains bone mass
and structural integrity. This balance is regulated by both systemic
hormones and local cytokines.

\section{Calcium Homeostasis}\label{calcium-homeostasis}

\subsection{Distribution of Calcium}\label{distribution-of-calcium}

Approximately \textbf{99\%} of total body calcium is stored in the
skeleton and teeth. The remaining \textbf{1\%} is distributed in
extracellular and intracellular compartments.

\begin{longtable}[]{@{}
  >{\raggedright\arraybackslash}p{(\linewidth - 4\tabcolsep) * \real{0.2143}}
  >{\raggedright\arraybackslash}p{(\linewidth - 4\tabcolsep) * \real{0.2449}}
  >{\raggedright\arraybackslash}p{(\linewidth - 4\tabcolsep) * \real{0.5408}}@{}}
\toprule\noalign{}
\begin{minipage}[b]{\linewidth}\raggedright
Compartment
\end{minipage} & \begin{minipage}[b]{\linewidth}\raggedright
Approximate percentage
\end{minipage} & \begin{minipage}[b]{\linewidth}\raggedright
Form
\end{minipage} \\
\midrule\noalign{}
\endhead
\bottomrule\noalign{}
\endlastfoot
Bone and teeth & 99\% & Hydroxyapatite crystals \\
Extracellular fluid & 1\% & Ionized (50\%), protein-bound (40\%),
complexed (10\%) \\
\end{longtable}

\textbf{Ionized calcium} is the physiologically active form that
participates in neuromuscular and enzymatic processes.

\subsection{Normal Serum Calcium
Levels}\label{normal-serum-calcium-levels}

\begin{itemize}
\tightlist
\item
  \textbf{Total calcium:} 2.1--2.6 mmol/L
\item
  \textbf{Ionized calcium:} 1.1--1.3 mmol/L
\end{itemize}

These levels vary slightly with \textbf{age}, \textbf{protein status},
and \textbf{acid-base balance}.

\subsection{Sources and Absorption of
Calcium}\label{sources-and-absorption-of-calcium}

\subsubsection{Dietary Sources}\label{dietary-sources}

\begin{itemize}
\tightlist
\item
  \textbf{Breast milk:} Primary source in infancy, though low in
  calcium, it is highly bioavailable.
\item
  \textbf{Cow's milk, fish with bones (e.g., sardines), green leafy
  vegetables (kontomire),} and \textbf{fortified foods} are important
  dietary sources in Ghana.
\end{itemize}

\subsubsection{Absorption}\label{absorption}

\begin{itemize}
\tightlist
\item
  Occurs mainly in the \textbf{duodenum and proximal jejunum}.
\item
  Facilitated by \textbf{active transport} (vitamin D-dependent) and
  \textbf{passive diffusion}.
\item
  Factors enhancing absorption:

  \begin{itemize}
  \tightlist
  \item
    Adequate vitamin D.
  \item
    Acidic gastric pH.
  \item
    Presence of lactose in infants.
  \end{itemize}
\item
  Factors reducing absorption:

  \begin{itemize}
  \tightlist
  \item
    Phytates (in cereals), oxalates (in spinach), and high phosphate
    intake (soft drinks).
  \item
    Chronic diarrhoeal diseases or fat malabsorption.
  \end{itemize}
\end{itemize}

\section{Hormonal Regulation of Calcium and Phosphate
Metabolism}\label{hormonal-regulation-of-calcium-and-phosphate-metabolism}

Calcium balance is maintained by a \textbf{homeostatic triad} involving:

\begin{enumerate}
\def\labelenumi{\arabic{enumi}.}
\tightlist
\item
  \textbf{Parathyroid hormone (PTH)}
\item
  \textbf{Vitamin D (calcitriol)}
\item
  \textbf{Calcitonin}
\end{enumerate}

These hormones act on the \textbf{bone}, \textbf{kidney}, and
\textbf{gastrointestinal tract} to regulate calcium and phosphate
concentrations.

\subsection{Parathyroid Hormone (PTH)}\label{parathyroid-hormone-pth}

\subsubsection{Source}\label{source}

Secreted by the \textbf{chief cells} of the parathyroid glands.

\subsubsection{Stimulus}\label{stimulus}

Released in response to \textbf{low serum ionized calcium}.

\subsubsection{Actions}\label{actions}

\begin{itemize}
\tightlist
\item
  \textbf{Bone:} Stimulates osteoclast-mediated bone resorption,
  releasing calcium and phosphate.
\item
  \textbf{Kidney:}

  \begin{itemize}
  \tightlist
  \item
    Increases calcium reabsorption in distal tubules.
  \item
    Decreases phosphate reabsorption (phosphaturia).
  \item
    Enhances 1α-hydroxylase activity → increases active vitamin D
    production.
  \end{itemize}
\item
  \textbf{Intestine:} Indirectly increases calcium absorption via
  vitamin D activation.
\end{itemize}

\subsubsection{Net Effect}\label{net-effect}

Raises serum calcium and lowers serum phosphate.

\subsection{Vitamin D (Calcitriol)}\label{vitamin-d-calcitriol}

\subsubsection{Sources}\label{sources}

\begin{itemize}
\tightlist
\item
  \textbf{Endogenous synthesis:} From 7-dehydrocholesterol in the skin
  upon exposure to sunlight (UVB rays).
\item
  \textbf{Dietary intake:} From fish oil, fortified milk, eggs, and
  supplements.
\end{itemize}

\subsubsection{Metabolism}\label{metabolism}

\begin{enumerate}
\def\labelenumi{\arabic{enumi}.}
\tightlist
\item
  \textbf{Liver:} Converts cholecalciferol to \textbf{25-hydroxyvitamin
  D {[}25(OH)D{]}}.
\item
  \textbf{Kidney:} Converts 25(OH)D to \textbf{1,25-dihydroxyvitamin D
  {[}1,25(OH)₂D{]}}, the active form.
\end{enumerate}

\subsubsection{Actions}\label{actions-1}

\begin{itemize}
\tightlist
\item
  \textbf{Intestine:} Increases absorption of calcium and phosphate.
\item
  \textbf{Bone:} Promotes mineralization and bone formation.
\item
  \textbf{Kidney:} Facilitates calcium reabsorption.
\end{itemize}

\subsubsection{Net Effect}\label{net-effect-1}

Increases both serum calcium and phosphate.

\subsubsection{Ghanaian Context}\label{ghanaian-context}

Children in Ghana should theoretically have adequate vitamin D due to
abundant sunlight. However, \textbf{urbanization}, \textbf{use of
sunscreen}, \textbf{indoor lifestyles}, \textbf{dark skin pigmentation},
and \textbf{maternal deficiency during pregnancy} contribute to
suboptimal vitamin D levels in both mothers and infants. Consequently,
\textbf{nutritional rickets} remains prevalent in some communities,
particularly among exclusively breastfed infants without
supplementation.

\subsection{Calcitonin}\label{calcitonin}

\subsubsection{Source}\label{source-1}

Secreted by \textbf{parafollicular (C) cells} of the thyroid gland.

\subsubsection{Actions}\label{actions-2}

\begin{itemize}
\tightlist
\item
  Inhibits osteoclastic bone resorption.
\item
  Promotes renal calcium excretion.
\end{itemize}

\subsubsection{Net Effect}\label{net-effect-2}

Lowers serum calcium levels --- physiologically less significant in
children compared to adults.

\section{Bone Remodeling}\label{bone-remodeling}

Bone is constantly renewed through \textbf{remodeling cycles}, which
consist of:

\begin{enumerate}
\def\labelenumi{\arabic{enumi}.}
\tightlist
\item
  \textbf{Activation:} Recruitment of osteoclasts to resorption sites.
\item
  \textbf{Resorption:} Breakdown of mineral and matrix by osteoclasts.
\item
  \textbf{Reversal:} Transition phase.
\item
  \textbf{Formation:} Osteoblasts lay down new osteoid, which becomes
  mineralized.
\end{enumerate}

During \textbf{childhood and adolescence}, bone formation exceeds
resorption, resulting in net bone gain. \textbf{Peak bone mass} is
achieved around the third decade of life, after which bone loss begins
gradually.

Factors influencing bone remodeling include:

\begin{itemize}
\tightlist
\item
  \textbf{Mechanical stress:} Weight-bearing stimulates bone formation.
\item
  \textbf{Hormones:} Growth hormone, PTH, sex steroids.
\item
  \textbf{Nutrients:} Calcium, phosphate, magnesium, and vitamin D.
\item
  \textbf{Cytokines:} IL-1, TNF-α, and RANKL/OPG pathway.
\end{itemize}

\section{Disorders of Calcium and Bone Metabolism in
Children}\label{disorders-of-calcium-and-bone-metabolism-in-children}

\subsection{Hypocalcaemia}\label{hypocalcaemia}

\subsubsection{Causes}\label{causes-9}

\begin{itemize}
\tightlist
\item
  \textbf{Neonatal:}

  \begin{itemize}
  \tightlist
  \item
    Prematurity.
  \item
    Maternal diabetes.
  \item
    Birth asphyxia or sepsis.
  \item
    Hypoparathyroidism or pseudohypoparathyroidism.
  \end{itemize}
\item
  \textbf{Childhood:}

  \begin{itemize}
  \tightlist
  \item
    Vitamin D deficiency (nutritional rickets).
  \item
    Chronic renal disease.
  \item
    Hypomagnesaemia.
  \item
    Malabsorption syndromes.
  \end{itemize}
\end{itemize}

\subsubsection{Clinical Features}\label{clinical-features-41}

\begin{itemize}
\tightlist
\item
  Tetany (carpopedal spasm, laryngospasm).
\item
  Seizures.
\item
  Paresthesiae.
\item
  Chvostek's and Trousseau's signs.
\item
  Irritability or poor feeding in infants.
\end{itemize}

\subsubsection{Management}\label{management-35}

\begin{itemize}
\tightlist
\item
  Acute: IV calcium gluconate (0.5--1 mL/kg of 10\% solution).
\item
  Chronic: Oral calcium and vitamin D supplementation.
\item
  Treat underlying cause (e.g., magnesium deficiency).
\end{itemize}

\subsection{Hypercalcaemia}\label{hypercalcaemia}

\subsubsection{Causes}\label{causes-10}

\begin{itemize}
\tightlist
\item
  Iatrogenic (excess vitamin D or calcium).
\item
  Primary hyperparathyroidism (rare in children).
\item
  Malignancy (leukaemia, lymphoma).
\item
  Granulomatous diseases (sarcoidosis, tuberculosis).
\item
  Prolonged immobilization.
\end{itemize}

\subsubsection{Clinical Features}\label{clinical-features-42}

\begin{itemize}
\tightlist
\item
  Nausea, vomiting, constipation.
\item
  Polyuria, polydipsia.
\item
  Lethargy, confusion.
\item
  Renal calculi or nephrocalcinosis.
\end{itemize}

\subsubsection{Management}\label{management-36}

\begin{itemize}
\tightlist
\item
  Adequate hydration.
\item
  Loop diuretics (furosemide).
\item
  Corticosteroids for vitamin D-related causes.
\item
  Bisphosphonates in refractory cases.
\end{itemize}

\subsection{Rickets and Osteomalacia}\label{rickets-and-osteomalacia}

\subsubsection{Definition}\label{definition-29}

Defective mineralization of bone matrix, leading to soft and deformed
bones in children (rickets) and adults (osteomalacia).

\subsubsection{Causes}\label{causes-11}

\begin{itemize}
\tightlist
\item
  \textbf{Nutritional vitamin D deficiency:} Most common in Ghana.
\item
  \textbf{Calcium deficiency:} Seen in diets low in dairy or with high
  phytate content.
\item
  \textbf{Chronic kidney disease:} Renal rickets.
\item
  \textbf{Genetic:} Vitamin D--dependent or resistant rickets.
\end{itemize}

\subsubsection{Clinical Features}\label{clinical-features-43}

\begin{itemize}
\tightlist
\item
  Delayed closure of fontanelle.
\item
  Frontal bossing, rachitic rosary.
\item
  Bowed legs (genu varum) or knock knees (genu valgum).
\item
  Widened wrists and ankles.
\item
  Growth retardation.
\end{itemize}

\subsubsection{Investigations}\label{investigations-39}

\begin{itemize}
\tightlist
\item
  Low calcium and phosphate.
\item
  Elevated alkaline phosphatase.
\item
  Low 25(OH) vitamin D.
\item
  Radiographs: Cupping and fraying of metaphyses.
\end{itemize}

\subsubsection{Management}\label{management-37}

\begin{itemize}
\tightlist
\item
  Vitamin D supplementation (2,000--6,000 IU daily for 3 months, then
  maintenance).
\item
  Dietary calcium supplementation.
\item
  Adequate sunlight exposure.
\item
  Health education for caregivers.
\end{itemize}

\subsubsection{Local Note}\label{local-note}

Studies from Ghana and Nigeria have shown a \textbf{mixed calcium and
vitamin D deficiency} pattern in rickets. Exclusive reliance on sunlight
exposure without dietary correction may therefore be inadequate.

\subsection{Osteopenia and Osteoporosis of
Prematurity}\label{osteopenia-and-osteoporosis-of-prematurity}

\subsubsection{Definition}\label{definition-30}

Reduced bone mineral content in preterm infants due to inadequate
mineral accretion.

\subsubsection{Causes}\label{causes-12}

\begin{itemize}
\tightlist
\item
  Prematurity (most calcium accretion occurs in the third trimester).
\item
  Prolonged parenteral nutrition.
\item
  Chronic diuretic or steroid use.
\item
  Limited physical activity.
\end{itemize}

\subsubsection{Prevention and
Management}\label{prevention-and-management}

\begin{itemize}
\tightlist
\item
  Adequate calcium and phosphate supplementation in preterm feeds.
\item
  Use of fortified breast milk.
\item
  Gentle physical therapy.
\end{itemize}

\subsection{Renal Osteodystrophy}\label{renal-osteodystrophy}

\subsubsection{Pathophysiology}\label{pathophysiology-32}

Chronic kidney disease leads to: - Impaired phosphate excretion
(hyperphosphataemia), - Reduced calcitriol production, - Secondary
hyperparathyroidism, - Bone demineralization and deformities.

\subsubsection{Management}\label{management-38}

\begin{itemize}
\tightlist
\item
  Dietary phosphate restriction.
\item
  Phosphate binders.
\item
  Active vitamin D analogues (calcitriol).
\item
  Correction of metabolic acidosis.
\end{itemize}

\section{Laboratory Evaluation of Bone and Calcium
Metabolism}\label{laboratory-evaluation-of-bone-and-calcium-metabolism}

\begin{longtable}[]{@{}
  >{\raggedright\arraybackslash}p{(\linewidth - 4\tabcolsep) * \real{0.3333}}
  >{\raggedright\arraybackslash}p{(\linewidth - 4\tabcolsep) * \real{0.3333}}
  >{\raggedright\arraybackslash}p{(\linewidth - 4\tabcolsep) * \real{0.3333}}@{}}
\toprule\noalign{}
\begin{minipage}[b]{\linewidth}\raggedright
Test
\end{minipage} & \begin{minipage}[b]{\linewidth}\raggedright
Interpretation
\end{minipage} & \begin{minipage}[b]{\linewidth}\raggedright
Clinical Utility
\end{minipage} \\
\midrule\noalign{}
\endhead
\bottomrule\noalign{}
\endlastfoot
\textbf{Serum calcium (total/ionized)} & ↓ in hypocalcaemia, ↑ in
hypercalcaemia & Basic screen \\
\textbf{Serum phosphate} & ↓ in rickets, ↑ in renal disease & Assess
phosphate balance \\
\textbf{Alkaline phosphatase (ALP)} & Elevated in bone formation or
rickets & Marker of bone turnover \\
\textbf{Parathyroid hormone (PTH)} & High in secondary
hyperparathyroidism & Helps classify calcium disorders \\
\textbf{25(OH) vitamin D} & Reflects vitamin D stores & Deficiency
common in rickets \\
\textbf{1,25(OH)₂D} & Active form; low in renal disease & Used
selectively \\
\textbf{Urinary calcium and phosphate} & To assess renal losses & Useful
in metabolic bone diseases \\
\end{longtable}

Radiologic investigations (wrist or knee X-rays) complement biochemical
findings in rickets and other bone disorders.

\section{Nutritional Considerations and Public Health
Implications}\label{nutritional-considerations-and-public-health-implications}

In Ghana, dietary calcium intake among children is often \textbf{below
recommended levels}, particularly in rural communities where milk
consumption is limited. Staple diets based on maize, millet, and cassava
have high phytate content, reducing calcium bioavailability.

\textbf{Strategies to improve calcium and vitamin D status include:} -
Promoting exclusive breastfeeding with appropriate maternal nutrition. -
Fortification of complementary foods with calcium and vitamin D. -
Encouraging outdoor play for sunlight exposure. - Supplementation
programs for at-risk groups (infants, adolescents, pregnant women).

Public health campaigns should also emphasize the dangers of excessive
soda intake, as \textbf{phosphoric acid} in carbonated drinks can impair
calcium absorption.

\section{Summary}\label{summary-3}

\begin{itemize}
\tightlist
\item
  Calcium and phosphate metabolism is tightly regulated by \textbf{PTH},
  \textbf{vitamin D}, and \textbf{calcitonin}.
\item
  The skeleton serves as the main calcium reservoir, undergoing constant
  remodeling.
\item
  Disorders such as \textbf{rickets}, \textbf{hypocalcaemia}, and
  \textbf{renal osteodystrophy} are common in paediatric practice.
\item
  \textbf{Nutritional deficiency}, \textbf{limited sunlight exposure},
  and \textbf{chronic kidney disease} are leading causes in Ghana.
\item
  Early diagnosis, supplementation, and community education are vital
  for prevention and management.
\end{itemize}

\section{Further Reading}\label{further-reading-3}

\begin{enumerate}
\def\labelenumi{\arabic{enumi}.}
\tightlist
\item
  Thacher TD, Fischer PR. Vitamin D and calcium deficiencies in children
  living in tropical areas. \emph{Int J Vitam Nutr Res.}
  2013;83(5):292--301.\\
\item
  Oduro-Boatey C, Aryeetey R, et al.~Nutritional rickets among Ghanaian
  children: prevalence and associated dietary factors. \emph{Ghana Med
  J.} 2020;54(2):78--85.\\
\item
  Pettifor JM. Nutritional rickets: deficiency of vitamin D, calcium, or
  both? \emph{Am J Clin Nutr.} 2004;80(6 Suppl):1725S--1729S.\\
\item
  Holick MF. Vitamin D deficiency. \emph{N Engl J Med.}
  2007;357(3):266--281.\\
\item
  Weaver CM, et al.~The National Osteoporosis Foundation's position
  statement on peak bone mass development. \emph{Osteoporos Int.}
  2016;27:1281--1386.\\
\item
  WHO. \emph{Guidelines on Vitamin D Supplementation for Infants and
  Children.} Geneva: World Health Organization; 2019.\\
\item
  Ghana Health Service. \emph{Nutrition and Micronutrient Guidelines for
  Child Health in Ghana.} Accra: GHS; 2021.\\
\item
  Kliegman RM, et al.~\emph{Nelson Textbook of Pediatrics.} 21st
  ed.~Elsevier; 2020. Chapter on Calcium and Bone Metabolism.
\end{enumerate}

\chapter{Reproductive Disorders}\label{reproductive-disorders}

\section{Introduction}\label{introduction-48}

Reproductive endocrinology in children involves the study of the
development, function, and disorders of the
hypothalamic--pituitary--gonadal (HPG) axis. Unlike adults, in whom the
system operates cyclically or continuously, the paediatric HPG axis
undergoes distinct phases of activation and quiescence from fetal life
through puberty.

Reproductive disorders in children manifest as abnormalities in sexual
differentiation, pubertal timing, gonadal development, or fertility.
These conditions often present with clinical signs such as ambiguous
genitalia, delayed or precocious puberty, or menstrual irregularities in
adolescents.

In Ghana and sub-Saharan Africa, limited access to endocrine diagnostic
testing, late presentation, and sociocultural sensitivities surrounding
reproductive development pose additional challenges. Hence, an
understanding of normal reproductive physiology and early recognition of
pathological patterns is crucial for paediatricians.

\section{Physiology of the Hypothalamic--Pituitary--Gonadal
Axis}\label{physiology-of-the-hypothalamicpituitarygonadal-axis}

The HPG axis controls reproductive development and function through the
coordinated actions of:

\begin{enumerate}
\def\labelenumi{\arabic{enumi}.}
\tightlist
\item
  \textbf{Hypothalamus:} Secretes gonadotropin-releasing hormone (GnRH)
  in a pulsatile fashion.\\
\item
  \textbf{Pituitary gland:} Responds to GnRH by releasing luteinizing
  hormone (LH) and follicle-stimulating hormone (FSH).\\
\item
  \textbf{Gonads (testes and ovaries):} Produce sex steroids
  (testosterone, estrogen, progesterone) and gametes in response to LH
  and FSH.
\end{enumerate}

\subsection{Phases of Activity}\label{phases-of-activity}

\begin{itemize}
\tightlist
\item
  \textbf{Fetal life:} HPG axis active; sex differentiation occurs.
\item
  \textbf{Mini-puberty (first 6 months postnatal):} Temporary
  activation, useful for early diagnosis of some disorders.
\item
  \textbf{Childhood:} Relative quiescence of the axis.
\item
  \textbf{Puberty:} Reactivation leading to secondary sexual
  characteristics and fertility.
\end{itemize}

\section{Classification of Paediatric Reproductive
Disorders}\label{classification-of-paediatric-reproductive-disorders}

Reproductive disorders can be broadly grouped into:

\begin{enumerate}
\def\labelenumi{\arabic{enumi}.}
\tightlist
\item
  \textbf{Disorders of Sexual Differentiation (DSD)}
\item
  \textbf{Pubertal Disorders}
\item
  \textbf{Gonadal Dysfunction}
\item
  \textbf{Menstrual and Fertility Disorders}
\end{enumerate}

Each category encompasses multiple aetiologies, with some overlap.

\section{Disorders of Sexual Differentiation
(DSD)}\label{disorders-of-sexual-differentiation-dsd}

These are congenital conditions in which chromosomal, gonadal, or
anatomical sex is atypical.

\subsection{Classification}\label{classification-9}

\begin{itemize}
\tightlist
\item
  \textbf{46,XX DSD:} Often due to exposure to excess androgens (e.g.,
  congenital adrenal hyperplasia).\\
\item
  \textbf{46,XY DSD:} Results from under-virilization due to gonadal
  dysgenesis or androgen insensitivity.\\
\item
  \textbf{Sex Chromosome DSD:} Such as Turner syndrome (45,X) and
  Klinefelter syndrome (47,XXY).
\end{itemize}

\subsection{Clinical Presentation}\label{clinical-presentation-4}

\begin{itemize}
\tightlist
\item
  Ambiguous genitalia at birth\\
\item
  Inguinal or labial masses (testes)\\
\item
  Discordance between genetic and phenotypic sex\\
\item
  Failure of virilization in adolescence
\end{itemize}

\subsection{Diagnosis}\label{diagnosis-21}

\begin{itemize}
\tightlist
\item
  \textbf{Karyotyping:} Determines chromosomal sex.\\
\item
  \textbf{Hormonal assays:} Measure 17-hydroxyprogesterone,
  testosterone, LH, FSH, and AMH.\\
\item
  \textbf{Pelvic ultrasound:} Evaluates internal genital structures.\\
\item
  \textbf{Genetic testing:} Useful where available.
\end{itemize}

\subsection{Local Context}\label{local-context}

In Ghana, many cases of ambiguous genitalia present late due to cultural
stigma and inadequate neonatal screening. Early multidisciplinary
assessment (paediatrician, endocrinologist, surgeon, psychologist) is
critical for optimal outcomes and family counselling.

\subsection{Management}\label{management-39}

\begin{itemize}
\tightlist
\item
  Gender assignment based on diagnostic clarity and cultural
  sensitivity.\\
\item
  Surgical correction where indicated.\\
\item
  Hormone replacement (e.g., glucocorticoids in CAH).\\
\item
  Psychosocial support for the patient and family.
\end{itemize}

\section{Disorders of Puberty}\label{disorders-of-puberty-1}

\subsection{Definition}\label{definition-31}

Puberty is the process leading to sexual maturation and reproductive
capability. It involves the activation of the HPG axis.

\begin{itemize}
\tightlist
\item
  \textbf{Precocious puberty:} Onset of secondary sexual characteristics
  before age 8 in girls or 9 in boys.\\
\item
  \textbf{Delayed puberty:} Absence of such characteristics by age 13 in
  girls or 14 in boys.
\end{itemize}

\subsection{Precocious Puberty}\label{precocious-puberty-1}

\subsubsection{Types}\label{types-2}

\begin{itemize}
\tightlist
\item
  \textbf{Central (GnRH-dependent):} Premature activation of the HPG
  axis.
\item
  \textbf{Peripheral (GnRH-independent):} Due to excess sex steroids
  from gonadal, adrenal, or ectopic sources.
\end{itemize}

\subsubsection{Causes}\label{causes-13}

\begin{itemize}
\tightlist
\item
  Central: Idiopathic (especially in girls), CNS lesions (e.g.,
  hypothalamic hamartoma), CNS infections (common post-meningitic
  sequelae in sub-Saharan Africa).\\
\item
  Peripheral: Congenital adrenal hyperplasia, ovarian cysts or tumours,
  McCune-Albright syndrome, exogenous hormones.
\end{itemize}

\subsubsection{Clinical Features}\label{clinical-features-44}

\begin{itemize}
\tightlist
\item
  Early breast development (thelarche), pubic hair (pubarche), or menses
  in girls.\\
\item
  Testicular enlargement, penile growth, or voice deepening in boys.\\
\item
  Advanced bone age and accelerated growth.
\end{itemize}

\subsubsection{Investigations}\label{investigations-40}

\begin{itemize}
\tightlist
\item
  LH and FSH (basal and GnRH-stimulated).\\
\item
  Sex steroids (estradiol, testosterone).\\
\item
  Bone age (left hand/wrist X-ray).\\
\item
  MRI brain for CNS pathology.
\end{itemize}

\subsubsection{Management}\label{management-40}

\begin{itemize}
\tightlist
\item
  \textbf{Central:} GnRH analogues to suppress premature axis
  activation.\\
\item
  \textbf{Peripheral:} Treat underlying cause (e.g., tumour resection,
  corticosteroid replacement in CAH).\\
\item
  \textbf{Psychological support:} Essential in early-developing children
  in conservative societies.
\end{itemize}

\subsection{Delayed Puberty}\label{delayed-puberty-1}

\subsubsection{Causes}\label{causes-14}

\begin{itemize}
\tightlist
\item
  \textbf{Constitutional delay:} Most common; familial and benign.\\
\item
  \textbf{Hypogonadotropic hypogonadism:} Due to pituitary/hypothalamic
  defects (e.g., Kallmann syndrome, chronic malnutrition).\\
\item
  \textbf{Hypergonadotropic hypogonadism:} Primary gonadal failure
  (e.g., Turner, Klinefelter, mumps orchitis).
\end{itemize}

\subsubsection{Clinical Features}\label{clinical-features-45}

\begin{itemize}
\tightlist
\item
  Absence of breast or testicular development.\\
\item
  Short stature or growth failure.\\
\item
  Psychosocial distress among peers.
\end{itemize}

\subsubsection{Investigations}\label{investigations-41}

\begin{itemize}
\tightlist
\item
  LH, FSH, and sex steroid levels.\\
\item
  Karyotype where indicated.\\
\item
  MRI for central lesions.\\
\item
  Bone age assessment.
\end{itemize}

\subsubsection{Management}\label{management-41}

\begin{itemize}
\tightlist
\item
  Observation in constitutional delay.\\
\item
  Hormone replacement therapy (estrogen or testosterone).\\
\item
  Treat underlying systemic or nutritional disorders.
\end{itemize}

\subsubsection{Ghanaian Context}\label{ghanaian-context-1}

Malnutrition, chronic infections, and delayed recognition of
hypogonadism remain frequent contributors to delayed puberty. Clinical
suspicion and affordable initial hormonal testing are essential in
regional hospitals.

\section{Gonadal Dysfunction}\label{gonadal-dysfunction}

\subsection{Primary Gonadal Failure}\label{primary-gonadal-failure}

Results from intrinsic gonadal abnormalities leading to
\textbf{hypergonadotropic hypogonadism}.\\
Common causes include:

\begin{itemize}
\tightlist
\item
  Turner syndrome (45,X)
\item
  Klinefelter syndrome (47,XXY)
\item
  Chemotherapy or radiotherapy-induced damage
\item
  Autoimmune oophoritis or orchitis
\end{itemize}

\subsubsection{Clinical Features}\label{clinical-features-46}

\begin{itemize}
\tightlist
\item
  Absent or incomplete pubertal development\\
\item
  Infertility\\
\item
  Amenorrhea in females\\
\item
  Small, firm testes in males
\end{itemize}

\subsubsection{Management}\label{management-42}

\begin{itemize}
\tightlist
\item
  Hormone replacement therapy (HRT) for puberty induction and
  maintenance.\\
\item
  Fertility counselling.\\
\item
  Monitor for associated comorbidities (e.g., cardiovascular risk in
  Turner syndrome).
\end{itemize}

\subsection{Secondary Gonadal Failure}\label{secondary-gonadal-failure}

Due to hypothalamic or pituitary defects causing
\textbf{hypogonadotropic hypogonadism}.\\
Causes: CNS tumours, trauma, chronic systemic illness, or genetic
syndromes like Prader-Willi.

\section{Menstrual and Fertility
Disorders}\label{menstrual-and-fertility-disorders}

Although more relevant in late adolescence, early recognition of
abnormal menstrual patterns is important.

\#A\# Common Disorders - \textbf{Primary amenorrhea:} No menses by age
15 or within 3 years of thelarche.\\
- \textbf{Secondary amenorrhea:} Absence of menses for \textgreater3
months in a previously menstruating girl.\\
- \textbf{Oligomenorrhea:} Infrequent menses (\textgreater35 days
apart).

\#A\# Causes - Anatomic (imperforate hymen, Müllerian agenesis)\\
- Ovarian (PCOS, gonadal dysgenesis)\\
- Pituitary (hyperprolactinaemia)\\
- Hypothalamic (stress, undernutrition, excessive exercise)

\subsection{Investigations}\label{investigations-42}

\begin{itemize}
\tightlist
\item
  Serum LH, FSH, prolactin, TSH, estradiol.\\
\item
  Pelvic ultrasound.\\
\item
  Progesterone challenge test.
\end{itemize}

\subsection{Management}\label{management-43}

\begin{itemize}
\tightlist
\item
  Treat underlying cause (e.g., surgical correction, nutritional
  rehabilitation).\\
\item
  Hormonal therapy for regulation.\\
\item
  Psychological counselling.
\end{itemize}

\section{Approach to a Child with a Reproductive
Disorder}\label{approach-to-a-child-with-a-reproductive-disorder}

\begin{enumerate}
\def\labelenumi{\arabic{enumi}.}
\tightlist
\item
  \textbf{Detailed history}

  \begin{itemize}
  \tightlist
  \item
    Onset and progression of puberty\\
  \item
    Family history of delayed or precocious puberty\\
  \item
    Neonatal genital appearance, chronic illnesses\\
  \item
    Drug and toxin exposure
  \end{itemize}
\item
  \textbf{Physical examination}

  \begin{itemize}
  \tightlist
  \item
    Tanner staging\\
  \item
    Height, weight, and growth velocity\\
  \item
    Dysmorphic features or signs of chronic disease
  \end{itemize}
\item
  \textbf{Laboratory evaluation}

  \begin{itemize}
  \tightlist
  \item
    Hormonal profile (LH, FSH, estradiol/testosterone, prolactin, TSH)\\
  \item
    Karyotyping where indicated
  \end{itemize}
\item
  \textbf{Imaging}

  \begin{itemize}
  \tightlist
  \item
    Pelvic/abdominal ultrasound\\
  \item
    MRI brain/pituitary for central causes
  \end{itemize}
\item
  \textbf{Psychosocial assessment}

  \begin{itemize}
  \tightlist
  \item
    Critical in addressing stigma and self-image concerns, particularly
    in the Ghanaian context.
  \end{itemize}
\end{enumerate}

\section{Challenges in the Ghanaian and Sub-Saharan
Context}\label{challenges-in-the-ghanaian-and-sub-saharan-context}

\begin{itemize}
\tightlist
\item
  \textbf{Limited endocrine testing:} Many hospitals lack the capacity
  for detailed hormonal assays.\\
\item
  \textbf{Late presentation:} Families may delay seeking medical advice
  due to cultural stigma.\\
\item
  \textbf{Cost barriers:} HRT and genetic testing may be unaffordable
  for many patients.\\
\item
  \textbf{Need for multidisciplinary care:} Involving paediatric
  endocrinologists, surgeons, psychologists, and social workers.\\
\item
  \textbf{Education and awareness:} Community sensitization and training
  of primary care workers are key.
\end{itemize}

\section{Key Takeaways}\label{key-takeaways-2}

\begin{itemize}
\tightlist
\item
  Reproductive disorders in children arise from dysfunction of the HPG
  axis or abnormalities in sexual differentiation.\\
\item
  Early recognition and hormonal evaluation are crucial to prevent
  long-term physical and psychosocial complications.\\
\item
  In Ghana, limited diagnostic resources demand pragmatic, symptom-based
  approaches supplemented by clinical acumen.\\
\item
  Psychosocial support and culturally sensitive counselling are integral
  to care.
\end{itemize}

\section{Further Reading}\label{further-reading-4}

\begin{enumerate}
\def\labelenumi{\arabic{enumi}.}
\tightlist
\item
  Brook CGD, Clayton PE, Brown RS. \emph{Brook's Clinical Pediatric
  Endocrinology}, 8th ed. Wiley-Blackwell, 2023.\\
\item
  Sperling MA. \emph{Pediatric Endocrinology}, 5th ed.~Elsevier, 2021.\\
\item
  Osei K, et al.~``Paediatric endocrine practice in sub-Saharan Africa:
  challenges and opportunities.'' \emph{J Clin Res Pediatr Endocrinol},
  2019.\\
\item
  WHO. \emph{Sexual Maturation and Pubertal Disorders: A Guide for
  Primary Health Care Providers}, Geneva, 2018.\\
\item
  West African College of Physicians (WACP) Curriculum for Paediatric
  Endocrinology, 2024.
\end{enumerate}

\chapter{Growth Disorders}\label{growth-disorders}

\section{Introduction}\label{introduction-49}

Growth is one of the most visible and sensitive indicators of a child's
health and overall well-being. In paediatrics, monitoring growth is an
essential part of clinical care, as deviations from normal patterns
often signal underlying disease, malnutrition, or hormonal imbalance.

A \textbf{growth disorder} refers to any condition that results in
abnormal stature or growth velocity compared with reference standards
for age and sex. These disorders may present as \textbf{short stature},
\textbf{tall stature}, or \textbf{abnormal growth velocity}, and they
can have \textbf{endocrine}, \textbf{genetic}, \textbf{systemic}, or
\textbf{nutritional} causes.

In Ghana and much of sub-Saharan Africa, poor nutrition, chronic
infections, and delayed recognition of endocrine disorders remain major
contributors to abnormal growth. The limited availability of specialized
diagnostic tests often necessitates a pragmatic, clinically guided
approach.

\section{Normal Growth Physiology}\label{normal-growth-physiology}

Normal growth is influenced by the complex interplay between genetic,
nutritional, hormonal, and environmental factors. The main hormones
involved are:

\begin{enumerate}
\def\labelenumi{\arabic{enumi}.}
\tightlist
\item
  \textbf{Growth Hormone (GH):} Secreted by the anterior pituitary in a
  pulsatile manner, stimulating hepatic production of insulin-like
  growth factor 1 (IGF-1).
\item
  \textbf{IGF-1 and IGFBP-3:} Mediate the growth-promoting effects of GH
  at the tissue level.
\item
  \textbf{Thyroid hormones:} Essential for normal bone maturation and
  growth velocity.
\item
  \textbf{Sex steroids (estrogen and testosterone):} Promote pubertal
  growth spurt and epiphyseal closure.
\item
  \textbf{Cortisol:} In excess, inhibits growth.
\item
  \textbf{Insulin:} Promotes growth through anabolic effects.
\end{enumerate}

\subsubsection{Phases of Growth}\label{phases-of-growth}

\begin{enumerate}
\def\labelenumi{\arabic{enumi}.}
\tightlist
\item
  \textbf{Infancy (0--2 years):} Rapid growth, largely
  nutrition-dependent.
\item
  \textbf{Childhood (2 years--puberty):} Steady growth, GH-dependent.
\item
  \textbf{Pubertal phase:} Growth acceleration due to synergistic
  effects of GH and sex steroids.
\end{enumerate}

\section{Assessment of Growth}\label{assessment-of-growth}

A systematic approach to growth assessment is essential for early
recognition of abnormalities.

\subsection{Measurement Techniques}\label{measurement-techniques}

\begin{itemize}
\item
  \textbf{Height:} Using a stadiometer (standing) or length board
  (infants).
\item
  \textbf{Weight:} Calibrated scale.
\item
  \textbf{Head circumference:} For children \textless5 years.
\item
  \textbf{Mid-parental height (MPH):} Estimate of genetic potential.

  {[} \text{MPH (boys)} =
  \frac{(\text{father's height} + \text{mother's height} + 13)}{2} {]}
  {[} \text{MPH (girls)} =
  \frac{(\text{father's height} + \text{mother's height} - 13)}{2} {]}
\end{itemize}

\subsection{Growth Charts}\label{growth-charts}

\begin{itemize}
\tightlist
\item
  Use WHO or CDC growth charts.
\item
  Plot serial measurements to determine \textbf{growth velocity}.
\item
  Identify \textbf{crossing of centiles}, which is more significant than
  a single low value.
\end{itemize}

\subsection{Bone Age Assessment}\label{bone-age-assessment}

\begin{itemize}
\tightlist
\item
  X-ray of the left hand and wrist compared to Greulich and Pyle
  standards.
\item
  Bone age lags behind chronological age in endocrine causes, but is
  usually normal in familial short stature.
\end{itemize}

\subsection{Pubertal Assessment}\label{pubertal-assessment}

\begin{itemize}
\tightlist
\item
  Tanner staging for breast, genital, and pubic hair development.
\item
  Pubertal timing provides clues to underlying endocrine abnormalities.
\end{itemize}

\section{Classification of Growth
Disorders}\label{classification-of-growth-disorders}

Growth disorders can be broadly classified as:

\begin{enumerate}
\def\labelenumi{\arabic{enumi}.}
\tightlist
\item
  \textbf{Short stature:} Height below the 3rd percentile or
  \textgreater2 standard deviations (SD) below mean for age and sex.\\
\item
  \textbf{Tall stature:} Height above the 97th percentile or
  \textgreater2 SD above mean.\\
\item
  \textbf{Abnormal growth velocity:} Deviation from expected rate for
  age.
\end{enumerate}

This chapter focuses on \textbf{short stature}, the commonest concern in
paediatric endocrinology.

\section{Short Stature}\label{short-stature}

\subsection{Aetiological
Classification}\label{aetiological-classification}

\begin{longtable}[]{@{}
  >{\raggedright\arraybackslash}p{(\linewidth - 4\tabcolsep) * \real{0.2500}}
  >{\raggedright\arraybackslash}p{(\linewidth - 4\tabcolsep) * \real{0.4200}}
  >{\raggedright\arraybackslash}p{(\linewidth - 4\tabcolsep) * \real{0.3300}}@{}}
\toprule\noalign{}
\begin{minipage}[b]{\linewidth}\raggedright
Category
\end{minipage} & \begin{minipage}[b]{\linewidth}\raggedright
Example Disorders
\end{minipage} & \begin{minipage}[b]{\linewidth}\raggedright
Key Features
\end{minipage} \\
\midrule\noalign{}
\endhead
\bottomrule\noalign{}
\endlastfoot
\textbf{Normal variants} & Familial short stature, constitutional growth
delay & Normal bone age or delayed bone age with normal growth
velocity \\
\textbf{Endocrine causes} & Growth hormone deficiency, hypothyroidism,
Cushing's syndrome & Decreased growth velocity, delayed bone age \\
\textbf{Chronic systemic disease} & Renal, cardiac, hepatic disease,
malnutrition & Poor weight gain precedes height faltering \\
\textbf{Genetic and skeletal disorders} & Turner syndrome,
achondroplasia & Dysmorphic features, disproportionate body segments \\
\textbf{Psychosocial deprivation} & Neglect, chronic stress & Variable
catch-up growth when environment improves \\
\end{longtable}

\section{Endocrine Causes of Short
Stature}\label{endocrine-causes-of-short-stature}

\subsection{Growth Hormone Deficiency
(GHD)}\label{growth-hormone-deficiency-ghd-2}

\subsubsection{Definition}\label{definition-32}

Deficiency of GH secretion, either isolated or as part of multiple
pituitary hormone deficiencies.

\subsubsection{Causes}\label{causes-15}

\begin{itemize}
\tightlist
\item
  \textbf{Congenital:} Pituitary hypoplasia, midline defects, genetic
  mutations.
\item
  \textbf{Acquired:} Brain tumours (craniopharyngioma), head trauma, CNS
  infections (meningitis, tuberculosis), irradiation.
\end{itemize}

\subsubsection{Clinical Features}\label{clinical-features-47}

\begin{itemize}
\tightlist
\item
  Normal birth size.
\item
  Progressive deviation from growth curve after 6 months.
\item
  Chubby face, truncal adiposity.
\item
  Delayed dentition, bone age, and puberty.
\item
  Often normal intelligence.
\end{itemize}

\subsubsection{Investigations}\label{investigations-43}

\begin{itemize}
\tightlist
\item
  \textbf{Serum IGF-1 and IGFBP-3:} Screening tests (low in GHD).
\item
  \textbf{GH stimulation tests:} Using insulin, clonidine, or glucagon
  (confirmatory).
\item
  \textbf{MRI brain:} Evaluate pituitary and hypothalamus.
\item
  \textbf{Other pituitary hormones:} TSH, ACTH, gonadotropins.
\end{itemize}

\subsubsection{Management}\label{management-44}

\begin{itemize}
\tightlist
\item
  \textbf{Recombinant GH therapy} (0.03--0.05 mg/kg/day SC).
\item
  Monitor growth velocity and bone age.
\item
  Address underlying cause if secondary.
\end{itemize}

\subsubsection{Ghanaian Context}\label{ghanaian-context-2}

GH assays and stimulation tests are limited to a few tertiary centres
(e.g., KATH, KBTH). Empirical diagnosis is sometimes supported by
clinical features and low IGF-1, if available.

\subsection{Hypothyroidism}\label{hypothyroidism}

\subsubsection{Pathophysiology}\label{pathophysiology-33}

Deficiency of thyroid hormones impairs bone maturation, metabolism, and
growth.

\subsubsection{Clinical Features}\label{clinical-features-48}

\begin{itemize}
\tightlist
\item
  Growth failure with delayed bone age.
\item
  Puffy face, coarse hair, dry skin.
\item
  Bradycardia, constipation, lethargy.
\item
  Delayed puberty.
\end{itemize}

\subsubsection{Diagnosis}\label{diagnosis-22}

\begin{itemize}
\tightlist
\item
  Low free T4, high TSH (primary hypothyroidism).
\item
  Low T4, low TSH (secondary hypothyroidism).
\end{itemize}

\subsubsection{Management}\label{management-45}

\begin{itemize}
\tightlist
\item
  Lifelong \textbf{levothyroxine replacement} (10--15 µg/kg/day).
\item
  Monitor TSH and growth response.
\end{itemize}

\subsubsection{Local Note}\label{local-note-1}

In Ghana, neonatal screening for congenital hypothyroidism is not yet
universal, so cases may present late with severe growth retardation and
cognitive impairment.

\subsection{Cushing's Syndrome}\label{cushings-syndrome}

\subsubsection{Definition}\label{definition-33}

Chronic exposure to excess glucocorticoids (endogenous or exogenous).

\subsubsection{Clinical Features}\label{clinical-features-49}

\begin{itemize}
\tightlist
\item
  Growth failure with weight gain (hallmark).
\item
  Moon face, truncal obesity, striae, hypertension.
\item
  Proximal muscle weakness.
\end{itemize}

\subsubsection{Causes}\label{causes-16}

\begin{itemize}
\tightlist
\item
  Prolonged steroid therapy (commonest in Ghana).
\item
  Adrenal or pituitary tumours (rare).
\end{itemize}

\subsubsection{Management}\label{management-46}

\begin{itemize}
\tightlist
\item
  Taper and discontinue exogenous steroids where possible.
\item
  Treat underlying pathology surgically or medically.
\end{itemize}

\section{Non-Endocrine Causes}\label{non-endocrine-causes}

\subsection{Chronic Systemic Illness}\label{chronic-systemic-illness}

Chronic diseases such as renal failure, cyanotic congenital heart
disease, inflammatory bowel disease, and sickle cell anaemia impair
growth.

\subsubsection{Mechanisms}\label{mechanisms-1}

\begin{itemize}
\tightlist
\item
  Nutritional deficits.
\item
  Increased energy demands.
\item
  Cytokine-mediated suppression of GH/IGF axis.
\end{itemize}

\subsubsection{Management}\label{management-47}

\begin{itemize}
\tightlist
\item
  Optimize control of the primary disease.
\item
  Nutritional rehabilitation.
\end{itemize}

\subsection{Malnutrition}\label{malnutrition}

A leading cause of growth retardation in Ghana.\\
- \textbf{Kwashiorkor and marasmus} lead to stunting and wasting.\\
- Linear growth improves only after prolonged nutritional
rehabilitation.\\
- Differentiation from endocrine short stature: \textbf{low
weight-for-height} and \textbf{normal bone age}.

\section{Genetic and Skeletal
Dysplasias}\label{genetic-and-skeletal-dysplasias}

\subsubsection{Turner Syndrome (45,X)}\label{turner-syndrome-45x}

\begin{itemize}
\tightlist
\item
  Short stature, webbed neck, shield chest.
\item
  Gonadal dysgenesis (streak ovaries).
\item
  Coarctation of aorta, renal anomalies.
\end{itemize}

\textbf{Diagnosis:} Karyotype confirmation.\\
\textbf{Management:} GH therapy, estrogen replacement, cardiac and renal
monitoring.

\subsubsection{Achondroplasia}\label{achondroplasia}

\begin{itemize}
\tightlist
\item
  Disproportionate short stature.
\item
  Short limbs, large head, normal intelligence.
\item
  Autosomal dominant FGFR3 mutation.
\end{itemize}

\section{Tall Stature}\label{tall-stature}

Although less common, tall stature may also warrant evaluation.

\subsection{Causes}\label{causes-17}

\begin{enumerate}
\def\labelenumi{\arabic{enumi}.}
\tightlist
\item
  \textbf{Familial tall stature:} Normal growth velocity, normal bone
  age.
\item
  \textbf{Endocrine:} GH excess (gigantism), precocious puberty,
  hyperthyroidism.
\item
  \textbf{Genetic:} Marfan syndrome, Klinefelter syndrome.
\item
  \textbf{Obesity:} Early growth acceleration with premature epiphyseal
  closure.
\end{enumerate}

\subsubsection{Investigations}\label{investigations-44}

\begin{itemize}
\tightlist
\item
  Bone age.
\item
  GH and IGF-1 levels.
\item
  Karyotype (for Klinefelter's).
\item
  Echocardiography (for Marfan's).
\end{itemize}

\subsubsection{Management}\label{management-48}

\begin{itemize}
\tightlist
\item
  Treat underlying endocrine cause.
\item
  Psychosocial support for body image issues.
\end{itemize}

\section{Growth Hormone Therapy: Principles and
Monitoring}\label{growth-hormone-therapy-principles-and-monitoring}

\subsubsection{Indications}\label{indications}

\begin{itemize}
\tightlist
\item
  GH deficiency.
\item
  Turner syndrome.
\item
  Chronic renal failure.
\item
  Small for gestational age without catch-up by age 2 years.
\item
  Idiopathic short stature (in selected cases).
\end{itemize}

\subsubsection{Administration}\label{administration}

\begin{itemize}
\tightlist
\item
  Daily subcutaneous injection at night.
\item
  Dose: 0.03--0.05 mg/kg/day.
\end{itemize}

\subsubsection{Monitoring}\label{monitoring-3}

\begin{itemize}
\tightlist
\item
  Growth velocity (every 6 months).
\item
  Bone age (yearly).
\item
  IGF-1 levels (to guide dosing).
\item
  Adverse effects: pseudotumor cerebri, slipped capital femoral
  epiphysis, glucose intolerance.
\end{itemize}

\subsubsection{Cost and Accessibility in
Ghana}\label{cost-and-accessibility-in-ghana}

GH therapy is expensive (≈GHS 1500--2500/month) and rarely covered by
national health insurance, limiting access to selected tertiary
hospitals. Efforts are ongoing to improve subsidization for paediatric
endocrine conditions.

\section{Approach to a Child with Short
Stature}\label{approach-to-a-child-with-short-stature}

\begin{enumerate}
\def\labelenumi{\arabic{enumi}.}
\tightlist
\item
  \textbf{Confirm true short stature} (plot on chart).
\item
  \textbf{Assess growth velocity} (serial measurements).
\item
  \textbf{Evaluate mid-parental height}.
\item
  \textbf{Assess for dysmorphism or disproportion}.
\item
  \textbf{Investigate} based on findings:
\end{enumerate}

\begin{longtable}[]{@{}
  >{\raggedright\arraybackslash}p{(\linewidth - 4\tabcolsep) * \real{0.1392}}
  >{\raggedright\arraybackslash}p{(\linewidth - 4\tabcolsep) * \real{0.4304}}
  >{\raggedright\arraybackslash}p{(\linewidth - 4\tabcolsep) * \real{0.4304}}@{}}
\toprule\noalign{}
\begin{minipage}[b]{\linewidth}\raggedright
Step
\end{minipage} & \begin{minipage}[b]{\linewidth}\raggedright
Key Tests
\end{minipage} & \begin{minipage}[b]{\linewidth}\raggedright
Purpose
\end{minipage} \\
\midrule\noalign{}
\endhead
\bottomrule\noalign{}
\endlastfoot
Screening & CBC, ESR, renal \& liver function & Rule out systemic
disease \\
Endocrine & TSH, free T4, IGF-1 & Identify hypothyroidism or GHD \\
Bone Age & X-ray left hand & Assess growth potential \\
Genetic & Karyotype (girls) & Detect Turner syndrome \\
Imaging & MRI brain & Assess pituitary or hypothalamus \\
\end{longtable}

\section{Psychosocial and Public Health
Aspects}\label{psychosocial-and-public-health-aspects}

\begin{itemize}
\tightlist
\item
  \textbf{Psychological impact:} Children with growth disorders often
  face teasing, low self-esteem, and academic difficulties.
\item
  \textbf{Parental counselling:} Important to distinguish between benign
  and pathological causes.
\item
  \textbf{Community education:} Emphasize routine growth monitoring at
  child welfare clinics.
\item
  \textbf{Policy direction:} Advocacy for national newborn screening and
  early endocrine referral networks.
\end{itemize}

\section{Key Takeaways}\label{key-takeaways-3}

\begin{itemize}
\tightlist
\item
  Growth disorders are common indicators of underlying systemic or
  endocrine disease.\\
\item
  Accurate measurement and serial growth plotting are crucial diagnostic
  steps.\\
\item
  Endocrine causes typically have delayed bone age with preserved
  weight.\\
\item
  In Ghana, nutritional and chronic disease-related growth failure
  remain prevalent, but endocrine causes should be considered early.\\
\item
  Early referral to specialized centres can improve long-term outcomes.
\end{itemize}

\section{Further Reading}\label{further-reading-5}

\begin{enumerate}
\def\labelenumi{\arabic{enumi}.}
\tightlist
\item
  Sperling MA. \emph{Pediatric Endocrinology}, 5th ed.~Elsevier, 2021.
\item
  Brook CGD, Clayton PE, Brown RS. \emph{Brook's Clinical Pediatric
  Endocrinology}, 8th ed. Wiley-Blackwell, 2023.
\item
  De Onis M et al.~\emph{WHO Child Growth Standards: Methods and
  Development}. Geneva: WHO, 2006.
\item
  Osei K, et al.~``Endocrine causes of short stature in sub-Saharan
  Africa: a diagnostic challenge.'' \emph{Ghana Med J}, 2018.
\item
  West African College of Physicians (WACP). \emph{Curriculum for
  Paediatric Endocrinology}, 2024.
\end{enumerate}

\part{{Haematology}}

\chapter{Basics}\label{basics-3}

\section{Introduction}\label{introduction-50}

Pediatric haematology requires an understanding of how the blood and
blood-forming organs develop anatomically, how they function
physiologically, and how disease processes disrupt these systems.
Children differ significantly from adults in their hematologic profiles
due to developmental changes, increased metabolic demands, and perinatal
transitions. In Ghana and Sub-Saharan Africa, the clinical spectrum is
strongly influenced by infectious diseases (malaria, HIV, TB),
nutritional deficiencies, and inherited disorders such as sickle cell
disease (SCD).

This chapter provides an integrated review of \textbf{anatomy},
\textbf{physiology}, and \textbf{pathology} of the hematopoietic system,
with emphasis on conditions most relevant to children in the region.

\section{Anatomy of the Hematologic
System}\label{anatomy-of-the-hematologic-system}

The hematologic system includes the \textbf{bone marrow}, \textbf{blood
cells}, \textbf{lymphoid organs}, the \textbf{reticuloendothelial
system}, vascular endothelium, and plasma proteins.

\subsection{Bone marrow}\label{bone-marrow}

\begin{itemize}
\tightlist
\item
  \textbf{Sites of hematopoiesis}

  \begin{itemize}
  \tightlist
  \item
    \emph{Fetal life:} yolk sac → liver \& spleen → bone marrow (by
    third trimester).
  \item
    \emph{Postnatal:} whole skeleton initially; becomes restricted to
    the axial skeleton (sternum, ribs, pelvis, vertebrae) during
    adolescence.
  \end{itemize}
\item
  \textbf{Cellularity}

  \begin{itemize}
  \tightlist
  \item
    Neonates and infants: very high cellularity
    (\textasciitilde80--90\%); gradually becomes fattier with age.
  \end{itemize}
\item
  \textbf{Components}

  \begin{itemize}
  \tightlist
  \item
    Hematopoietic stem cells (HSCs), stromal cells (fibroblasts,
    adipocytes), sinusoids, and supporting extracellular matrix.
  \end{itemize}
\end{itemize}

\textbf{Clinical relevance:} High marrow activity in infants enables
robust erythropoiesis but also means bone marrow failure syndromes
present early and dramatically. Bone marrow aspiration is essential when
leukaemia or aplasia is suspected.

\subsection{Lymphoid organs}\label{lymphoid-organs}

\begin{itemize}
\tightlist
\item
  \textbf{Thymus}

  \begin{itemize}
  \tightlist
  \item
    Large and active in childhood; required for T-cell maturation.
  \item
    Involution begins in adolescence.
  \item
    Absent or underdeveloped thymus → severe T-cell deficiency (e.g.,
    DiGeorge syndrome).
  \end{itemize}
\item
  \textbf{Spleen}

  \begin{itemize}
  \tightlist
  \item
    Filters aged/damaged RBCs; participates in immune responses and can
    host extramedullary hematopoiesis.
  \item
    Vulnerable to sequestration in SCD; frequently enlarged in
    malaria-endemic settings.
  \end{itemize}
\item
  \textbf{Lymph nodes}

  \begin{itemize}
  \tightlist
  \item
    More reactive and often enlarged in children due to frequent antigen
    exposure.
  \end{itemize}
\end{itemize}

\section{Physiology of the Hematologic
System}\label{physiology-of-the-hematologic-system}

\subsection{Hematopoiesis}\label{hematopoiesis}

\textbf{Definition:} production of blood cells from multipotent HSCs.

\textbf{Developmental stages}

\begin{enumerate}
\def\labelenumi{\arabic{enumi}.}
\tightlist
\item
  Yolk sac hematopoiesis (early gestation)
\item
  Foetal liver (mid to late gestation) --- primary site until birth
\item
  Bone marrow (dominant after birth)
\end{enumerate}

\textbf{Paediatric differences}

\begin{itemize}
\tightlist
\item
  Infants have higher baseline erythropoietic activity.
\item
  Lymphopoiesis is intensified during early childhood for immune
  maturation.
\end{itemize}

\subsection{Red blood cell physiology}\label{red-blood-cell-physiology}

\begin{itemize}
\tightlist
\item
  \textbf{Haemoglobin switching:} HbF (α₂γ₂) predominates at birth
  (≈70--80\%). Switch to HbA (α₂β₂) occurs over the first 6 months.
\item
  \textbf{Physiologic} anaemia of infancy: Postnatal fall in
  erythropoietin causes a nadir in haemoglobin at \textasciitilde6--8
  weeks; usually mild and self-limiting.
\item
  \textbf{Oxygen affinity:} HbF has higher O₂ affinity than HbA ---
  clinically significant for newborn adaptation.
\end{itemize}

\textbf{Clinical links}

\begin{itemize}
\item
  Hemoglobinopathies (e.g., SCD) typically become clinically significant
  after HbF declines.
\item
  Nutritional deficiencies (iron, folate, B12) rapidly impair RBC
  production due to high turnover.
\end{itemize}

\subsection{White blood cell
physiology}\label{white-blood-cell-physiology}

\begin{itemize}
\tightlist
\item
  \textbf{WBC composition:} Infants often have relative lymphocytosis;
  neutrophil counts are lower in early life than in older children and
  adults.
\item
  \textbf{Immune maturation:} Neonates and young infants have immature
  innate and adaptive immune responses.
\end{itemize}

\textbf{Clinical links}

\begin{itemize}
\item
  Distinguish physiologic lymphocytosis from pathologic causes (e.g.,
  lymphoid leukaemia).
\item
  Severe neutropenia increases the risk of fulminant bacterial
  infections.
\end{itemize}

\subsection{Platelet and coagulation
physiology}\label{platelet-and-coagulation-physiology}

\begin{itemize}
\tightlist
\item
  \textbf{Platelet counts} in neonates are broadly similar to those of
  adults.
\item
  \textbf{Neonatal factor levels:} Lower vitamin K stores and reduced
  levels of some vitamin K-dependent clotting factors (II, VII, IX, X)
  increase bleeding risk.
\item
  \textbf{Vitamin K prophylaxis} at birth is crucial to prevent Vitamin
  K Deficiency Bleeding (VKDB).
\end{itemize}

\section{Pathology in Pediatric
Haematology}\label{pathology-in-pediatric-haematology}

Hematologic pathology in children can be classified by the affected cell
line or process: production, maturation, destruction, and hemostasis.

\subsection{Disorders of red blood
cells}\label{disorders-of-red-blood-cells}

\subsubsection{Anaemia (physiologic
classification)}\label{anaemia-physiologic-classification}

\begin{itemize}
\tightlist
\item
  \textbf{Production failure:} iron deficiency, folate/B12 deficiency,
  bone marrow failure (aplastic anaemia).
\item
  \textbf{Destruction (hemolysis):} sickle cell disease, G6PD
  deficiency, malaria-related hemolysis, and immune hemolytic anaemia.
\item
  \textbf{Blood loss:} trauma, gastrointestinal blood loss (e.g.,
  hookworm), peripartum haemorrhage.
\end{itemize}

\textbf{Local epidemiology (Ghana \& region):}

\begin{itemize}
\item
  Severe anaemia in children \textless5 commonly results from malaria +
  iron deficiency.
\item
  Sickle cell disease is a major contributor to hemolytic anaemia and
  childhood morbidity.
\end{itemize}

\subsubsection{Hemoglobinopathies}\label{hemoglobinopathies}

\begin{itemize}
\tightlist
\item
  \textbf{Sickle cell disease (HbSS)}: vaso-occlusive crises, acute
  chest syndrome, splenic sequestration, stroke risk.
\item
  \textbf{Thalassemias:} less frequent in West Africa than in the
  Mediterranean/SE Asia, but should be considered in microcytic anaemia
  unresponsive to iron.
\end{itemize}

\subsubsection{Polycythemia}\label{polycythemia}

\begin{itemize}
\tightlist
\item
  Neonatal polycythemia from delayed cord clamping, maternal diabetes,
  or twin-twin transfusion syndrome.
\end{itemize}

\subsection{Disorders of white blood
cells}\label{disorders-of-white-blood-cells}

\begin{itemize}
\tightlist
\item
  \textbf{Neutropenia:} viral causes, drug-induced (e.g.,
  chloramphenicol), congenital forms.
\item
  \textbf{Leukocytosis \& blasts:} often reactive, but a high blast
  percentage suggests leukaemia (ALL is the most common childhood
  leukaemia).
\item
  \textbf{Immunodeficiencies:} primary (SCID, DiGeorge) and secondary
  (malnutrition, HIV).
\end{itemize}

\subsection{Platelet and coagulation
disorders}\label{platelet-and-coagulation-disorders}

\begin{itemize}
\tightlist
\item
  \textbf{Thrombocytopenia:} ITP (post-infectious), dengue (epidemic
  settings), marrow infiltration (leukaemia).
\item
  \textbf{Coagulation defects:} haemophilia A/B, von Willebrand disease,
  VKDB, and DIC (seen in severe sepsis and complicated malaria).
\end{itemize}

\subsection{Reticuloendothelial and lymphoid
pathology}\label{reticuloendothelial-and-lymphoid-pathology}

\begin{itemize}
\tightlist
\item
  \textbf{Splenomegaly:} chronic malaria, SCD (early enlargement; later
  autosplenectomy), portal hypertension, malignancy.
\item
  \textbf{Lymphadenopathy:} reactive infections, TB lymphadenitis,
  lymphomas.
\end{itemize}

\section{4 Integrating Anatomy, Physiology and Pathology --- Clinical
Perspectives}\label{integrating-anatomy-physiology-and-pathology-clinical-perspectives}

\begin{longtable}[]{@{}
  >{\raggedright\arraybackslash}p{(\linewidth - 6\tabcolsep) * \real{0.2245}}
  >{\raggedleft\arraybackslash}p{(\linewidth - 6\tabcolsep) * \real{0.2857}}
  >{\raggedright\arraybackslash}p{(\linewidth - 6\tabcolsep) * \real{0.2449}}
  >{\raggedright\arraybackslash}p{(\linewidth - 6\tabcolsep) * \real{0.2449}}@{}}
\toprule\noalign{}
\begin{minipage}[b]{\linewidth}\raggedright
Component
\end{minipage} & \begin{minipage}[b]{\linewidth}\raggedleft
Normal (Anatomy/Physiology)
\end{minipage} & \begin{minipage}[b]{\linewidth}\raggedright
Pathology
\end{minipage} & \begin{minipage}[b]{\linewidth}\raggedright
Pediatric/Local Relevance
\end{minipage} \\
\midrule\noalign{}
\endhead
\bottomrule\noalign{}
\endlastfoot
Bone marrow & High cellularity in infancy & Aplastic anaemia, leukaemia
& Early pancytopenia is often dramatic \\
HbF → HbA switch & HbF is dominant at birth & SCD symptoms as HbF falls
& Hydroxyurea raises HbF and reduces crises \\
Thymus & Central to T-cell maturation & Thymic aplasia → SCID &
Recurrent severe infections in infancy \\
Spleen & Filters RBCs, immune role & Sequestration, autosplenectomy &
Major role in SCD \& malaria pathology \\
Vitamin K stores & Low in neonates & VKDB & Important in out-of-facility
births \\
Immune maturation & Lymphocyte predominance early & Leukocyte disorders
& Distinguish physiologic vs pathologic counts \\
\end{longtable}

\section{Practical Approach to Common
Presentations}\label{practical-approach-to-common-presentations}

\subsection{Child with pallor / suspected
anaemia}\label{child-with-pallor-suspected-anaemia}

\begin{enumerate}
\def\labelenumi{\arabic{enumi}.}
\tightlist
\item
  History: onset, feeding, bleeding, family history
  (hemoglobinopathies), recent infections, drug exposure.
\item
  Examination: pallor distribution, jaundice, splenomegaly, and signs of
  heart failure.
\item
  Initial investigations:

  \begin{itemize}
  \tightlist
  \item
    Complete blood count (FBC) with indices
  \item
    Reticulocyte count
  \item
    Peripheral blood film (morphology; malaria parasites on thick/thin
    films)
  \item
    Malaria RDT where microscopy is unavailable
  \item
    Ferritin/iron studies, if available
  \item
    G6PD assay if hemolysis is suspected
  \end{itemize}
\item
  Management principles:

  \begin{itemize}
  \tightlist
  \item
    Treat underlying cause (e.g., antimalarials, iron for deficiency).
  \item
    Transfusion for severe or symptomatic anaemia (local transfusion
    guidelines).
  \end{itemize}
\end{enumerate}

\subsection{Child with
bleeding/bruising}\label{child-with-bleedingbruising}

\begin{enumerate}
\def\labelenumi{\arabic{enumi}.}
\tightlist
\item
  Obtain platelet count, PT/INR, and aPTT.
\item
  Consider ITP, haemophilia, vWD, and DIC.
\item
  Manage bleeding (local measures, tranexamic acid where indicated) and
  refer for factor assays or specialist care.
\end{enumerate}

\subsection{5.3 Suspected leukaemia}\label{suspected-leukaemia}

\begin{enumerate}
\def\labelenumi{\arabic{enumi}.}
\tightlist
\item
  Red flags: persistent fever, bone pain, bruising, hepatosplenomegaly,
  abnormal white cell differential with blasts.
\item
  Urgent FBC, blood film and prompt referral to a paediatric oncology
  unit for bone marrow aspirate and staging.
\end{enumerate}

\section{Management Principles}\label{management-principles-4}

\begin{itemize}
\tightlist
\item
  \textbf{Supportive care:} hydration, treating infections, nutritional
  rehabilitation, and pain control.
\item
  \textbf{Transfusion therapy:} indicated for severe symptomatic
  anaemia, acute blood loss, or SCD complications (exchange transfusion
  for stroke/ACS).
\item
  \textbf{Disease-specific therapies:} iron replacement, hydroxyurea for
  SCD, factor replacement for haemophilia, chemotherapy for
  malignancies.
\item
  \textbf{Prevention:} newborn screening for SCD, universal vitamin K
  prophylaxis, routine immunisations, insecticide-treated nets and
  malaria chemoprophylaxis where indicated, and routine deworming.
\end{itemize}

\section{Summary}\label{summary-4}

A working knowledge of the \textbf{anatomy} (bone marrow, thymus,
spleen), \textbf{physiology} (developmental hematopoiesis, haemoglobin
switching, immune maturation), and pathology (anaemia,
hemoglobinopathies, leukaemias, bleeding disorders) is essential for
pediatric practice. In Ghana and across Sub-Saharan Africa, high-burden
diseases such as malaria, SCD, and nutritional deficiencies shape
clinical priorities. Effective care depends on early recognition,
appropriate investigation (FBC, blood film, reticulocyte count,
coagulation studies), timely supportive measures, and referral for
specialised therapies.

\section{Further reading}\label{further-reading-6}

\begin{enumerate}
\def\labelenumi{\arabic{enumi}.}
\tightlist
\item
  Kliegman RM, Stanton B, St.~Geme JW III, Schor NF, Behrman RE.
  \emph{Nelson Textbook of Paediatrics}.
\item
  Hoffbrand AV, Moss PAH. \emph{Essential Haematology}.
\item
  World Health Organisation. \emph{Sickle Cell Disease: A Strategy for
  the WHO African Region}.
\item
  Ghana Health Service / Ministry of Health -- \emph{Standard Treatment
  Guidelines}.
\end{enumerate}

\chapter{Sickle Cell Disease}\label{sickle-cell-disease-1}

\section{Introduction}\label{introduction-51}

Sickle Cell Disease (SCD) represents one of the most important inherited
disorders of haemoglobin globally and in Ghana, where it poses a
significant public health challenge. It remains a major cause of
childhood morbidity and mortality across sub-Saharan Africa, yet it is
largely preventable through newborn screening and manageable with timely
interventions. Understanding its genetics, pathophysiology, clinical
presentation, and management principles is therefore crucial for every
medical student and health professional involved in paediatric care.

Sickle Cell Disease encompasses a group of inherited haemoglobinopathies
characterized by the presence of abnormal haemoglobin S (HbS) within red
blood cells. When deoxygenated, HbS polymerizes, causing the red cells
to assume a rigid, sickle-like shape. These sickled cells have a
shortened lifespan and tend to obstruct small blood vessels, resulting
in chronic haemolytic anaemia, episodic vaso-occlusive crises, and
progressive multi-organ damage.

The disease is inherited in an autosomal recessive pattern and primarily
affects populations from malaria-endemic regions, where the sickle cell
trait confers partial protection against severe \emph{Plasmodium
falciparum} infection. Consequently, the prevalence of the sickle gene
is highest in sub-Saharan Africa, the Middle East, and parts of India.

In Ghana, it is estimated that about \textbf{2\% of newborns} are
affected by SCD annually, and up to \textbf{30\% of the population}
carry the sickle trait (HbAS).

\section{Genetics and Molecular
Basis}\label{genetics-and-molecular-basis}

Normal adult haemoglobin (HbA) consists of two alpha (α) and two beta
(β) chains. In SCD, a \textbf{single-point mutation} occurs in the
β-globin gene on chromosome 11, substituting valine for glutamic acid at
position 6 of the β-chain. This substitution profoundly alters
haemoglobin solubility.

When deoxygenated, HbS molecules polymerize into long, rigid fibres,
distorting the red blood cells into the characteristic ``sickle'' shape.
These cells are less deformable, leading to \textbf{vaso-occlusion} and
\textbf{premature haemolysis}.

Common genetic variants of SCD include:

\begin{itemize}
\tightlist
\item
  \textbf{HbSS (Sickle Cell Anaemia)} -- homozygous state; most severe.
\item
  \textbf{HbSC disease} -- compound heterozygote for HbS and HbC.
\item
  \textbf{Sickle β-thalassaemia} -- combination of HbS with
  β-thalassaemia mutation (can be mild or severe).
\end{itemize}

\section{Epidemiology}\label{epidemiology-8}

Globally, about \textbf{300,000--400,000 infants} are born each year
with SCD. Sub-Saharan Africa accounts for more than 75\% of these
births. Without adequate care, most affected children in low-resource
settings die before their fifth birthday from infections or severe
anaemia.

In Ghana, SCD is among the top causes of paediatric hospital admissions.
The institution of \textbf{newborn screening programs} and
\textbf{comprehensive sickle cell clinics}, especially in teaching
hospitals, has significantly improved survival. Early diagnosis through
neonatal screening, coupled with penicillin prophylaxis and
immunization, can reduce childhood mortality by up to 70\%.

\section{Pathophysiology}\label{pathophysiology-34}

The clinical manifestations of SCD arise primarily from two interrelated
mechanisms: \textbf{haemolysis} and \textbf{vaso-occlusion}.

\subsection{Haemolysis}\label{haemolysis}

Sickled red cells are fragile and have a lifespan of only 10--20 days
(compared to 120 days for normal RBCs). Their destruction leads to
chronic anaemia, jaundice, and compensatory bone marrow expansion. Free
haemoglobin released during haemolysis also scavenges nitric oxide,
promoting vasoconstriction and endothelial dysfunction.

\subsection{Vaso-occlusion}\label{vaso-occlusion}

Rigid sickled cells adhere to vascular endothelium and obstruct small
vessels, resulting in tissue ischemia, infarction, and pain crises.
Repeated episodes cause chronic organ damage, particularly in the
spleen, kidneys, and bones.

Contributing factors include dehydration, infection, hypoxia, acidosis,
and cold exposure --- all of which increase sickling tendency.

\subsection{Other Pathophysiological
Effects}\label{other-pathophysiological-effects}

\begin{itemize}
\tightlist
\item
  \textbf{Splenic dysfunction} predisposes to overwhelming infections
  with encapsulated organisms (e.g., \emph{Streptococcus pneumoniae}).\\
\item
  \textbf{Bone marrow hyperactivity} can cause skeletal deformities and
  extramedullary haematopoiesis.\\
\item
  \textbf{Endothelial activation} and chronic inflammation contribute to
  long-term vasculopathy and pulmonary hypertension.
\end{itemize}

\section{Clinical Features}\label{clinical-features-50}

The disease spectrum varies widely depending on genotype and
environmental influences.

\subsection{Infancy and Early
Childhood}\label{infancy-and-early-childhood}

\begin{itemize}
\tightlist
\item
  Usually asymptomatic until 4--6 months when fetal haemoglobin (HbF)
  declines.\\
\item
  \textbf{Dactylitis (hand-foot syndrome):} Painful swelling of hands
  and feet, often the first manifestation.\\
\item
  \textbf{Severe anaemia} leading to pallor, jaundice, and poor
  growth.\\
\item
  \textbf{Increased susceptibility to infections}, particularly
  pneumonia and sepsis.
\end{itemize}

\subsection{Older Children and
Adolescents}\label{older-children-and-adolescents-1}

Common features include:

\begin{itemize}
\tightlist
\item
  Recurrent painful crises affecting bones, chest, or abdomen.\\
\item
  Chronic anaemia with scleral icterus and gallstones.\\
\item
  Growth retardation and delayed puberty.\\
\item
  Splenic atrophy (autosplenectomy) by age 6--8 years.\\
\item
  Leg ulcers and avascular necrosis of the femoral head.\\
\item
  Neurological events such as stroke or seizures.
\end{itemize}

\subsection{Major Clinical Syndromes}\label{major-clinical-syndromes}

\begin{enumerate}
\def\labelenumi{\arabic{enumi}.}
\tightlist
\item
  \textbf{Vaso-occlusive (Painful) Crisis:} Most frequent; triggered by
  dehydration, cold, or infection.\\
\item
  \textbf{Sequestration Crisis:} Sudden pooling of blood in spleen or
  liver leading to shock; more common in infants and young children.\\
\item
  \textbf{Aplastic Crisis:} Transient bone marrow suppression, often due
  to parvovirus B19 infection.\\
\item
  \textbf{Haemolytic Crisis:} Acute exacerbation of haemolysis with
  jaundice and falling haematocrit.\\
\item
  \textbf{Acute Chest Syndrome:} Characterized by chest pain, fever,
  hypoxia, and pulmonary infiltrates --- a leading cause of mortality.
\end{enumerate}

\section{Differential Diagnosis}\label{differential-diagnosis-24}

\begin{itemize}
\tightlist
\item
  Other causes of chronic haemolytic anaemia: thalassaemia, hereditary
  spherocytosis, G6PD deficiency.\\
\item
  Recurrent bone pain from osteomyelitis.\\
\item
  Malaria with anaemia.\\
\item
  Leukaemia or aplastic anaemia.
\end{itemize}

\section{Investigations}\label{investigations-45}

\subsection{Screening and Diagnostic
Tests}\label{screening-and-diagnostic-tests}

\begin{itemize}
\tightlist
\item
  \textbf{Sickle cell solubility test:} Rapid screening method.\\
\item
  \textbf{Haemoglobin electrophoresis:} Gold standard for definitive
  diagnosis and genotyping (HbSS, HbSC, HbSβ-thalassaemia).\\
\item
  \textbf{High-performance liquid chromatography (HPLC):} Provides
  precise quantification of haemoglobin fractions.\\
\item
  \textbf{DNA analysis:} Used for prenatal diagnosis.
\end{itemize}

\subsection{Routine Monitoring}\label{routine-monitoring}

\begin{itemize}
\tightlist
\item
  Full blood count (moderate anaemia, high reticulocyte count).\\
\item
  Liver function tests (for bilirubin and transaminase levels).\\
\item
  Renal function (creatinine, urinalysis for proteinuria).\\
\item
  Chest X-ray and echocardiogram in chronic cases.\\
\item
  Transcranial Doppler ultrasound annually in children with HbSS (to
  screen for stroke risk).
\end{itemize}

\section{Management}\label{management-49}

Management of SCD is comprehensive and lifelong, involving prevention of
crises, prompt treatment of complications, and psychosocial support.

\subsection{Health Maintenance and Preventive
Care}\label{health-maintenance-and-preventive-care}

\begin{itemize}
\tightlist
\item
  \textbf{Newborn screening:} Enables early identification and
  initiation of prophylaxis.\\
\item
  \textbf{Health education:} Parents should be educated about hydration,
  nutrition, and avoiding triggers (cold, stress, infection).\\
\item
  \textbf{Prophylactic antibiotics:} Oral penicillin V (125 mg twice
  daily \textless3 years; 250 mg twice daily \textgreater3 years) from
  diagnosis until at least age 5.\\
\item
  \textbf{Vaccinations:} Pneumococcal, \emph{Haemophilus influenzae}
  type b, meningococcal, hepatitis B, and annual influenza vaccines.\\
\item
  \textbf{Folic acid supplementation:} To support red cell production.
\end{itemize}

\subsection{Management of Acute
Crises}\label{management-of-acute-crises}

\subsubsection{Painful Crisis}\label{painful-crisis}

\begin{itemize}
\tightlist
\item
  Assess severity and exclude infection or acute chest syndrome.\\
\item
  Adequate hydration (oral or IV fluids).\\
\item
  Analgesia: stepwise approach using paracetamol, NSAIDs, and opioids
  for severe pain.\\
\item
  Oxygen if hypoxic; treat precipitating factors.
\end{itemize}

\subsubsection{Sequestration Crisis}\label{sequestration-crisis}

\begin{itemize}
\tightlist
\item
  Rapid restoration of blood volume with cautious transfusion.\\
\item
  Monitor for recurrence; splenectomy may be indicated after repeated
  episodes.
\end{itemize}

\subsubsection{Aplastic Crisis}\label{aplastic-crisis}

\begin{itemize}
\tightlist
\item
  Supportive care with transfusions; isolation if parvovirus suspected.
\end{itemize}

\subsubsection{Acute Chest Syndrome}\label{acute-chest-syndrome}

\begin{itemize}
\tightlist
\item
  Urgent treatment with antibiotics, oxygen therapy, analgesia, and
  blood transfusion (exchange transfusion in severe cases).
\end{itemize}

\subsection{Chronic Management}\label{chronic-management-1}

\subsubsection{Hydroxyurea Therapy}\label{hydroxyurea-therapy}

\begin{itemize}
\tightlist
\item
  Increases fetal haemoglobin (HbF) levels, reducing frequency of pain
  crises and acute chest syndrome.\\
\item
  Typical dose: 15--35 mg/kg/day with close monitoring for
  myelosuppression.
\end{itemize}

\subsubsection{Blood Transfusions}\label{blood-transfusions}

\begin{itemize}
\tightlist
\item
  Used for severe anaemia, stroke prevention, or perioperative
  optimization.\\
\item
  Chronic transfusion programs aim to maintain HbS \textless30\%.\\
\item
  Risk of iron overload; managed with chelation (deferasirox or
  deferoxamine).
\end{itemize}

\subsubsection{Stem Cell
Transplantation}\label{stem-cell-transplantation-1}

\begin{itemize}
\tightlist
\item
  The only curative treatment.\\
\item
  Best outcomes when performed early from an HLA-matched sibling
  donor.\\
\item
  Cost and donor availability remain limiting factors in Africa.
\end{itemize}

\subsection{Management of
Complications}\label{management-of-complications-2}

\begin{itemize}
\tightlist
\item
  \textbf{Stroke:} Urgent transfusion and long-term prevention via
  chronic transfusions or hydroxyurea.\\
\item
  \textbf{Renal disease:} ACE inhibitors for proteinuria.\\
\item
  \textbf{Pulmonary hypertension:} Oxygen and hydroxyurea; avoid
  hypoxia.\\
\item
  \textbf{Gallstones:} Cholecystectomy if symptomatic.\\
\item
  \textbf{Leg ulcers:} Local wound care and infection control.
\end{itemize}

\subsection{Psychosocial and Educational
Support}\label{psychosocial-and-educational-support}

Chronic illness impacts schooling, family dynamics, and mental health.
Counselling and peer support programs are vital. Transition to adult
care should be planned from adolescence.

\section{Complications}\label{complications-29}

\begin{longtable}[]{@{}ll@{}}
\toprule\noalign{}
System & Major Complications \\
\midrule\noalign{}
\endhead
\bottomrule\noalign{}
\endlastfoot
Haematologic & Severe anaemia, aplastic crisis \\
Cardiovascular & Cardiomegaly, high-output failure \\
Respiratory & Acute chest syndrome, pulmonary hypertension \\
Neurological & Stroke, seizures \\
Renal & Hematuria, hyposthenuria, renal failure \\
Musculoskeletal & Avascular necrosis, chronic leg ulcers \\
Hepatobiliary & Gallstones, hepatic sequestration \\
Growth & Delayed puberty, short stature \\
\end{longtable}

\section{Prevention}\label{prevention-18}

\begin{enumerate}
\def\labelenumi{\arabic{enumi}.}
\tightlist
\item
  \textbf{Public Health Interventions}

  \begin{itemize}
  \tightlist
  \item
    Premarital and antenatal genetic counselling.\\
  \item
    Neonatal screening programs nationwide.\\
  \item
    Education on carrier status and family planning.
  \end{itemize}
\item
  \textbf{Infection Prevention}

  \begin{itemize}
  \tightlist
  \item
    Routine immunizations.\\
  \item
    Prophylactic antibiotics and malaria prevention.
  \end{itemize}
\item
  \textbf{Nutritional Support}

  \begin{itemize}
  \tightlist
  \item
    Balanced diet with adequate folate and hydration.\\
  \item
    Avoidance of triggers such as cold, dehydration, and strenuous
    activity.
  \end{itemize}
\end{enumerate}

\section{Prognosis}\label{prognosis-33}

With comprehensive care, children with SCD can now survive into
adulthood with reasonable quality of life. Prognosis depends on genotype
(HbSS being most severe), frequency of complications, and access to
medical care.\\
Early diagnosis, hydroxyurea use, and stroke prevention have
significantly improved survival rates.

In resource-limited settings like Ghana, the main challenge remains
\textbf{late diagnosis and inadequate access} to specialized care.
Expansion of community-based sickle cell programs, health education, and
government support are essential for improving outcomes.

\section{Conclusion}\label{conclusion-31}

Sickle Cell Disease in children remains a major paediatric concern in
Ghana and across Africa. Its high prevalence, severe complications, and
lifelong impact demand a holistic approach integrating early detection,
preventive care, effective crisis management, and psychosocial support.
With increased awareness, improved access to diagnostics and hydroxyurea
therapy, and the scaling up of neonatal screening, the burden of SCD can
be significantly reduced, transforming the outlook for affected children
and their families.

\chapter{Other Hemoglobinopathies}\label{other-hemoglobinopathies}

\section{Introduction}\label{introduction-52}

Hemoglobinopathies are a diverse group of inherited disorders affecting
the structure or production of the globin chains of haemoglobin. While
sickle cell disease is the most prevalent and clinically significant
hemoglobinopathy in Ghana and West Africa, other disorders---such as
thalassemias, unstable haemoglobins, and quantitative or structural
variants---also contribute substantially to childhood morbidity. These
conditions, though less frequently recognised, are important causes of
chronic anaemia, transfusion dependence, and complications such as
gallstones, iron overload, and growth failure.

Hemoglobinopathies arise from mutations in the α-globin or β-globin
genes, leading to either defective haemoglobin synthesis (thalassemias)
or theproduction of abnormal haemoglobin variants. Understanding their
pathophysiology is crucial for appropriate diagnosis and management,
particularly in resource-limited settings where newborn screening may
not yet be widespread, and where chronic haemoglobin disorders may be
misdiagnosed as nutritional anaemia or sickle cell disease.

This chapter provides an overview of non-sickle cell hemoglobinopathies
relevant to paediatric practice in Ghana, reviews their clinical
features and diagnostic approaches, and outlines principles of
management.

\section{Anatomy and Physiology of
Haemoglobin}\label{anatomy-and-physiology-of-haemoglobin}

Haemoglobin is a tetrameric protein composed of two pairs of globin
chains (α and β in HbA) and four heme groups. Key components include:

\begin{itemize}
\tightlist
\item
  \textbf{HbA (α₂β₂)}: predominant adult haemoglobin\\
\item
  \textbf{HbA₂ (α₂δ₂)}: minor adult haemoglobin\\
\item
  \textbf{HbF (α₂γ₂)}: predominant fetal haemoglobin
\end{itemize}

Globin chain production is genetically regulated:

\begin{itemize}
\tightlist
\item
  \textbf{α-globin genes}: four genes located on chromosome 16\\
\item
  \textbf{β-globin gene cluster}: located on chromosome 11, includes δ,
  γ, and β genes
\end{itemize}

Disorders affecting these genes result in imbalances in haemoglobin
chain production or thesynthesis of structurally abnormal haemoglobin
molecules.

\section{Classification of Hemoglobinopathies (excluding Sickle Cell
Disease)}\label{classification-of-hemoglobinopathies-excluding-sickle-cell-disease}

\begin{enumerate}
\def\labelenumi{\arabic{enumi}.}
\item
  \textbf{Thalassemias (quantitative disorders)}

  \begin{itemize}
  \tightlist
  \item
    α-thalassemia\\
  \item
    β-thalassemia\\
  \item
    δβ-thalassemia and hereditary persistence of fetal haemoglobin
    (HPFH)
  \end{itemize}
\item
  \textbf{Structural variants (qualitative disorders)}

  \begin{itemize}
  \tightlist
  \item
    Haemoglobin C (HbC)\\
  \item
    Haemoglobin E (HbE)\\
  \item
    Haemoglobin D and others
  \end{itemize}
\item
  \textbf{Unstable hemoglobins}
\item
  \textbf{Methemoglobinemia and related hemoglobin variants}
\end{enumerate}

While many are asymptomatic carriers, homozygous or compound
heterozygous states may cause significant disease.

\section{α-Thalassemia}\label{ux3b1-thalassemia}

\subsection{Pathophysiology}\label{pathophysiology-35}

α-Thalassemia results from the deletion or mutation of one or more of
the four α-globin genes. The clinical severity corresponds to the number
of genes affected:

\begin{itemize}
\tightlist
\item
  \textbf{One-gene deletion (α⁺ trait)}: silent carrier\\
\item
  \textbf{Two-gene deletion (α-thalassemia trait)}: mild microcytic
  anaemia\\
\item
  \textbf{Three-gene deletion (HbH disease)}: moderate to severe chronic
  hemolytic anaemia\\
\item
  \textbf{Four-gene deletion (Hb Bart's hydrops fetalis)}: incompatible
  with life
\end{itemize}

In West Africa, single-gene deletions are more common, contributing to
high rates of microcytosis often misinterpreted as iron deficiency
anaemia.

\subsection{Clinical Features}\label{clinical-features-51}

\begin{itemize}
\tightlist
\item
  Silent carriers: asymptomatic\\
\item
  α-thalassemia trait: mild anaemia, microcytosis, borderline Hb\\
\item
  \textbf{HbH disease}:

  \begin{itemize}
  \tightlist
  \item
    Chronic hemolysis\\
  \item
    Jaundice\\
  \item
    Splenomegaly\\
  \item
    Episodes of hemolytic crises, especially with infections or
    oxidative stress\\
  \item
    Growth faltering
  \end{itemize}
\end{itemize}

\subsection{Diagnosis}\label{diagnosis-23}

\begin{itemize}
\tightlist
\item
  CBC: microcytosis, hypochromia, elevated RBC count\\
\item
  Peripheral smear: target cells, inclusion bodies (HbH)\\
\item
  Hb electrophoresis: usually normal except in HbH disease (presence of
  HbH or Hb Bart's)\\
\item
  DNA testing: confirms deletions (limited availability in Ghana)
\end{itemize}

\subsection{Management}\label{management-50}

\begin{itemize}
\tightlist
\item
  Folic acid supplementation\\
\item
  Avoid oxidative drugs (e.g., sulphonamides)\\
\item
  Transfusions during acute crises\\
\item
  Splenectomy may be considered in severe HbH disease (with appropriate
  vaccination)\\
\item
  Genetic counselling is necessary for families
\end{itemize}

\section{β-Thalassemia}\label{ux3b2-thalassemia}

\subsection{Pathophysiology}\label{pathophysiology-36}

β-Thalassemia results from point mutations affecting β-globin gene
expression. It is less common in West Africa but occurs sporadically due
to population migration. β-Thalassemia leads to reduced (β⁺) or absent
(β⁰) β-chain production, causing α-chain excess, ineffective
erythropoiesis, and hemolysis.

\subsection{Types}\label{types-3}

\begin{itemize}
\tightlist
\item
  \textbf{β-thalassemia minor (trait)}: heterozygous carriers\\
\item
  \textbf{β-thalassemia intermedia}: moderate disease\\
\item
  \textbf{β-thalassemia major (Cooley's anaemia)}: severe,
  transfusion-dependent from infancy
\end{itemize}

\subsection{Clinical Features}\label{clinical-features-52}

\subsubsection{β-Thalassemia Minor}\label{ux3b2-thalassemia-minor}

\begin{itemize}
\tightlist
\item
  Mild anemia\\
\item
  Microcytosis\\
\item
  Asymptomatic
\end{itemize}

\subsubsection{β-Thalassemia Major}\label{ux3b2-thalassemia-major}

\begin{itemize}
\tightlist
\item
  Severe anaemia appearing at 3--6 months\\
\item
  Failure to thrive\\
\item
  Jaundice\\
\item
  Hepatosplenomegaly\\
\item
  Bone changes (frontal bossing, maxillary prominence)\\
\item
  Recurrent infections\\
\item
  Gallstones\\
\item
  Cardiac failure and endocrine dysfunction due to iron overload
\end{itemize}

\subsection{Diagnosis}\label{diagnosis-24}

\begin{itemize}
\tightlist
\item
  CBC: severe microcytic hypochromic anaemia\\
\item
  Electrophoresis: low HbA, elevated HbF and HbA₂\\
\item
  Serum ferritin: elevated with transfusions\\
\item
  Radiology: bone changes in poorly transfused children
\end{itemize}

\subsection{Management}\label{management-51}

\begin{itemize}
\tightlist
\item
  Regular transfusion regimen to maintain Hb \textgreater{} 9--10 g/dL\\
\item
  \textbf{Iron chelation therapy} (deferoxamine, deferiprone,
  deferasirox)\\
\item
  Endocrine screening (thyroid, diabetes, puberty disorders)\\
\item
  Splenectomy for hypersplenism\\
\item
  \textbf{Curative therapy}: bone marrow transplantation --- limited
  availability in West Africa\\
\item
  Genetic counselling
\end{itemize}

In Ghana, limited access to electrophoresis and transfusion challenges
often delay diagnosis.

\section{Hemoglobin C (HbC) Disease}\label{hemoglobin-c-hbc-disease}

\subsection{Epidemiology}\label{epidemiology-9}

HbC is common in West Africa, including Ghana. The HbC gene frequency is
highest in northern Ghana and Burkina Faso.

\subsection{Pathophysiology}\label{pathophysiology-37}

A substitution of lysine for glutamic acid at position six on the
β-globin chain. HbC tends to crystallise within red cells, leading to
reduced RBC lifespan.

\subsection{Clinical Forms}\label{clinical-forms}

\begin{itemize}
\tightlist
\item
  \textbf{HbAC (trait)}: asymptomatic\\
\item
  \textbf{HbCC (homozygous)}:

  \begin{itemize}
  \tightlist
  \item
    Mild to moderate haemolytic anaemia\\
  \item
    Splenomegaly\\
  \item
    Intermittent jaundice\\
  \item
    Pigment gallstones
  \end{itemize}
\end{itemize}

\subsection{Diagnosis}\label{diagnosis-25}

\begin{itemize}
\tightlist
\item
  Electrophoresis: predominant HbC\\
\item
  Blood film: target cells, crystals
\end{itemize}

\subsection{Management}\label{management-52}

\begin{itemize}
\tightlist
\item
  Generally benign condition\\
\item
  Supportive care\\
\item
  Monitor for gallstones\\
\item
  Folic acid supplementation
\end{itemize}

\section{Haemoglobin E (HbE)}\label{haemoglobin-e-hbe}

\subsection{Epidemiology}\label{epidemiology-10}

Common in Southeast Asia; rare in Ghana but seen in children of
immigrant families or mixed ancestry.

\subsection{Clinical Features}\label{clinical-features-53}

\begin{itemize}
\tightlist
\item
  HbE trait: asymptomatic\\
\item
  HbE disease: mild anaemia\\
\item
  HbE/β-thalassemia: significant anaemia, similar to β-thalassemia
  intermedia
\end{itemize}

\subsection{Diagnosis and Management}\label{diagnosis-and-management-2}

\begin{itemize}
\tightlist
\item
  Electrophoresis shows the HbE band.\\
\item
  Management mirrors β-thalassemia intermedia (occasional transfusion,
  iron monitoring)
\end{itemize}

\section{δβ-Thalassemia and HPFH}\label{ux3b4ux3b2-thalassemia-and-hpfh}

\subsection{δβ-Thalassemia}\label{ux3b4ux3b2-thalassemia}

\begin{itemize}
\tightlist
\item
  Reduced δ and β chain production\\
\item
  Elevated HbF\\
\item
  Moderate anemia
\end{itemize}

\subsection{Hereditary Persistence of Fetal Haemoglobin
(HPFH)}\label{hereditary-persistence-of-fetal-haemoglobin-hpfh}

\begin{itemize}
\tightlist
\item
  Benign persistence of HbF into adulthood\\
\item
  Often asymptomatic\\
\item
  Important in genetic counselling due to interaction with other
  variants
\end{itemize}

\section{Unstable Hemoglobins}\label{unstable-hemoglobins}

These are rare variants with reduced molecular stability leading to
chronic hemolysis.

\subsection{Features}\label{features}

\begin{itemize}
\tightlist
\item
  Chronic hemolytic anaemia\\
\item
  Heinz bodies on staining\\
\item
  Episodes precipitated by illness or oxidant drugs
\end{itemize}

\section{Methemoglobinemia}\label{methemoglobinemia}

Specific haemoglobin variants predispose to methaemoglobinemia,
presenting with cyanosis unresponsive to oxygen therapy.

\subsection{Clinical Features}\label{clinical-features-54}

\begin{itemize}
\tightlist
\item
  ``Chocolate-brown'' blood\\
\item
  Cyanosis\\
\item
  Normal PaO₂ despite low saturation
\end{itemize}

\subsection{Management}\label{management-53}

\begin{itemize}
\tightlist
\item
  Methylene blue for acquired forms\\
\item
  Congenital forms may require long-term monitoring.
\end{itemize}

\section{Clinical Presentation of
Hemoglobinopathies}\label{clinical-presentation-of-hemoglobinopathies}

Children may present with:

\begin{itemize}
\tightlist
\item
  Chronic anemia\\
\item
  Jaundice\\
\item
  Poor growth\\
\item
  Splenomegaly\\
\item
  Bone pain (in thalassemia)\\
\item
  Recurrent infections\\
\item
  Gallstones\\
\item
  Transfusion dependence
\end{itemize}

In Ghana, such features are often mistakenly attributed to sickle cell
disease, malaria, or iron deficiency, highlighting the need for improved
diagnostics.

\section{Differential Diagnosis}\label{differential-diagnosis-25}

\begin{itemize}
\tightlist
\item
  Iron deficiency anaemia\\
\item
  Sickle cell syndromes\\
\item
  Chronic infection (e.g., TB, HIV)\\
\item
  Hemolytic anemias (G6PD deficiency)
\end{itemize}

\section{Investigations}\label{investigations-46}

\begin{itemize}
\tightlist
\item
  CBC and blood film\\
\item
  Reticulocyte count\\
\item
  Hb electrophoresis or HPLC\\
\item
  Serum ferritin\\
\item
  Liver function tests\\
\item
  Ultrasound (hepatosplenomegaly, gallstones)
\end{itemize}

\section{Management Principles}\label{management-principles-5}

\begin{enumerate}
\def\labelenumi{\arabic{enumi}.}
\tightlist
\item
  \textbf{Accurate diagnosis}\\
\item
  \textbf{Regular follow-up}\\
\item
  \textbf{Transfusion support for moderate/severe disease}\\
\item
  \textbf{Iron overload monitoring and chelation}\\
\item
  \textbf{Nutritional support (folate, vitamin D assessment)}\\
\item
  \textbf{Splenectomy in selected cases}\\
\item
  \textbf{Vaccinations}

  \begin{itemize}
  \tightlist
  \item
    Pneumococcal\\
  \item
    Meningococcal\\
  \item
    Hib\\
  \end{itemize}
\item
  \textbf{Family and genetic counselling}
\end{enumerate}

\section{Complications}\label{complications-30}

\begin{itemize}
\tightlist
\item
  Gallstones\\
\item
  Iron overload (endocrine, cardiac, hepatic complications)\\
\item
  Growth failure\\
\item
  Splenomegaly\\
\item
  Osteopenia\\
\item
  Cardiac failure (especially in β-thalassemia major)
\end{itemize}

\section{Local Context: Ghana and West
Africa}\label{local-context-ghana-and-west-africa}

\begin{itemize}
\tightlist
\item
  Haemoglobin C trait is common in northern Ghana.\\
\item
  α-thalassemia trait contributes significantly to microcytosis in
  Ghanaian children and is often confused with iron deficiency.\\
\item
  Limited access to electrophoresis, DNA analysis, and HPLC leads to
  underdiagnosis.\\
\item
  Blood transfusion availability can be inconsistent, complicating
  management for thalassemia major.\\
\item
  Iron chelation therapy is expensive and often unavailable.\\
\item
  Newborn screening programs in Ghana currently focus mainly on sickle
  cell disease, leaving other hemoglobinopathies undetected.
\end{itemize}

\section{Key Points for Clinical
Practice}\label{key-points-for-clinical-practice}

\begin{itemize}
\tightlist
\item
  Always evaluate microcytosis that does not respond to iron therapy for
  thalassemias.\\
\item
  Consider hemoglobinopathy in any child with chronic hemolytic anaemia
  without sickling crises.\\
\item
  HbC disease may cause gallstones at a young age---consider abdominal
  ultrasound in symptomatic children.\\
\item
  Ensure early transfusion and chelation for suspected thalassemia
  major.\\
\item
  Provide comprehensive counselling to families, including future
  reproductive implications.
\end{itemize}

\section{Further Reading}\label{further-reading-7}

\begin{enumerate}
\def\labelenumi{\arabic{enumi}.}
\tightlist
\item
  Weatherall DJ, Clegg JB. \emph{The Thalassaemia Syndromes}. Oxford
  University Press.\\
\item
  Hoffbrand AV, Higgs DR, Keeling D, Mehta AB. \emph{Postgraduate
  Haematology}. Wiley-Blackwell.\\
\item
  Ohene-Frempong K. Hemoglobinopathies in Africa. \emph{Pediatric
  Clinics of North America}.\\
\item
  Ministry of Health Ghana \& Ghana Health Service. \emph{National
  Newborn Screening Guidelines}.\\
\item
  Modell B, Darlison M. Global epidemiology of hemoglobin disorders.
  \emph{Bulletin of the WHO}.
\end{enumerate}

\chapter{Anemia}\label{anemia}

\section{Introduction}\label{introduction-53}

Anaemia is one of the most common haematological disorders in children
worldwide, particularly in low- and middle-income countries. It is a
major public health problem that contributes to growth retardation,
impaired cognitive development, increased susceptibility to infection,
and elevated morbidity and mortality. In Ghana and much of sub-Saharan
Africa, anaemia frequently results from nutritional deficiencies,
infections (such as malaria and helminthiasis), and genetic haemoglobin
disorders.

\section{Definition}\label{definition-34}

\textbf{Anaemia} is defined as a reduction in the concentration of
haemoglobin (Hb) in the blood below the normal range for age, sex, and
altitude, resulting in decreased oxygen-carrying capacity.

\subsection{WHO Haemoglobin Cut-offs for Anaemia in
Children}\label{who-haemoglobin-cut-offs-for-anaemia-in-children}

\begin{longtable}[]{@{}ll@{}}
\toprule\noalign{}
Age group & Hb (g/dL) threshold \\
\midrule\noalign{}
\endhead
\bottomrule\noalign{}
\endlastfoot
6--59 months & \textless{} 11.0 \\
5--11 years & \textless{} 11.5 \\
12--14 years & \textless{} 12.0 \\
≥15 years (females) & \textless{} 12.0 \\
≥15 years (males) & \textless{} 13.0 \\
\end{longtable}

\subsection{Classification by Severity (6 months--5
years)}\label{classification-by-severity-6-months5-years}

\begin{longtable}[]{@{}ll@{}}
\toprule\noalign{}
Severity & Hb (g/dL) \\
\midrule\noalign{}
\endhead
\bottomrule\noalign{}
\endlastfoot
Mild & 10--10.9 \\
Moderate & 7--9.9 \\
Severe & \textless{} 7 \\
\end{longtable}

\section{Classification}\label{classification-10}

Anaemia can be classified in several ways:

\subsection{1. Based on Pathophysiology}\label{based-on-pathophysiology}

\begin{itemize}
\tightlist
\item
  \textbf{Decreased production of red blood cells} (e.g., iron
  deficiency, aplastic anaemia, bone marrow suppression).
\item
  \textbf{Increased destruction of red blood cells (haemolysis)} (e.g.,
  sickle cell disease, G6PD deficiency, malaria).
\item
  \textbf{Blood loss} (acute or chronic haemorrhage, intestinal
  parasites).
\end{itemize}

\section{2. Based on Red Cell Indices (Morphological
Classification)}\label{based-on-red-cell-indices-morphological-classification}

\begin{longtable}[]{@{}
  >{\raggedright\arraybackslash}p{(\linewidth - 4\tabcolsep) * \real{0.2959}}
  >{\raggedright\arraybackslash}p{(\linewidth - 4\tabcolsep) * \real{0.1020}}
  >{\raggedright\arraybackslash}p{(\linewidth - 4\tabcolsep) * \real{0.6020}}@{}}
\toprule\noalign{}
\begin{minipage}[b]{\linewidth}\raggedright
Type
\end{minipage} & \begin{minipage}[b]{\linewidth}\raggedright
MCV (fL)
\end{minipage} & \begin{minipage}[b]{\linewidth}\raggedright
Common Causes
\end{minipage} \\
\midrule\noalign{}
\endhead
\bottomrule\noalign{}
\endlastfoot
\textbf{Microcytic hypochromic} & \textless{} 80 & Iron deficiency,
thalassaemia, anaemia of chronic disease \\
\textbf{Normocytic normochromic} & 80--100 & Acute blood loss,
haemolysis, chronic disease \\
\textbf{Macrocytic} & \textgreater{} 100 & Folate or vitamin B12
deficiency, hypothyroidism \\
\end{longtable}

\chapter{Epidemiology}\label{epidemiology-11}

\begin{itemize}
\tightlist
\item
  \textbf{Global prevalence:} About 40\% of children under 5 years are
  anaemic (WHO 2023).
\item
  \textbf{Sub-Saharan Africa:} Up to 60\% prevalence in under-fives.
\item
  \textbf{Ghana:} Nationally, \textasciitilde45\% of children under five
  are anaemic (GDHS 2022).
\end{itemize}

Anaemia contributes to impaired learning ability, stunted growth, and
increased risk of death from infectious diseases.

\section{Aetiology}\label{aetiology-22}

Anaemia in children is multifactorial and may be grouped under three
broad mechanisms:

\subsection{1. Decreased RBC Production}\label{decreased-rbc-production}

\begin{itemize}
\tightlist
\item
  \textbf{Iron deficiency anaemia (IDA):} Most common cause globally.
\item
  \textbf{Folate or vitamin B12 deficiency:} Due to poor intake or
  malabsorption.
\item
  \textbf{Chronic disease:} Reduced erythropoietin response and iron
  sequestration.
\item
  \textbf{Bone marrow suppression:} Viral infections (parvovirus B19),
  drugs, or malignancy.
\end{itemize}

\subsection{2. Increased RBC Destruction
(Haemolysis)}\label{increased-rbc-destruction-haemolysis}

\begin{itemize}
\tightlist
\item
  \textbf{Inherited haemoglobinopathies:} Sickle cell disease,
  thalassaemia.
\item
  \textbf{Red cell enzyme defects:} G6PD deficiency.
\item
  \textbf{Immune-mediated haemolysis:} Autoimmune haemolytic anaemia.
\item
  \textbf{Infections:} Malaria, sepsis.
\end{itemize}

\subsection{3. Blood Loss}\label{blood-loss}

\begin{itemize}
\tightlist
\item
  \textbf{Parasitic infestations:} Hookworm, schistosomiasis.
\item
  \textbf{Gastrointestinal bleeding:} Meckel's diverticulum, ulcers.
\item
  \textbf{Trauma or surgery.}
\end{itemize}

\section{Pathophysiology}\label{pathophysiology-38}

Haemoglobin synthesis requires: - \textbf{Iron} --- essential component
of haem. - \textbf{Globin chains} --- produced by bone marrow. -
\textbf{Vitamin B12 and folate} --- necessary for DNA synthesis.

Deficiency of any of these components leads to impaired erythropoiesis
and reduced Hb concentration.

In anaemia, tissue hypoxia develops, stimulating increased cardiac
output and erythropoietin release. Chronic anaemia leads to compensatory
mechanisms such as: - Increased 2,3-DPG in RBCs (enhances oxygen
delivery). - Reticulocytosis in haemolytic anaemias.

\section{Clinical Features}\label{clinical-features-55}

\subsection{General Symptoms}\label{general-symptoms}

\begin{itemize}
\tightlist
\item
  Fatigue and weakness.
\item
  Pallor (conjunctival, palmar, mucosal).
\item
  Dizziness, headache.
\item
  Tachycardia, dyspnoea on exertion.
\item
  Poor concentration and irritability.
\end{itemize}

\subsection{Signs}\label{signs}

\begin{itemize}
\tightlist
\item
  Pallor (best seen in conjunctiva, tongue, palms).
\item
  Systolic flow murmur.
\item
  Bounding pulse.
\item
  Signs of cardiac failure in severe cases.
\end{itemize}

\subsection{Specific Features Based on
Cause}\label{specific-features-based-on-cause}

\begin{longtable}[]{@{}
  >{\raggedright\arraybackslash}p{(\linewidth - 2\tabcolsep) * \real{0.2805}}
  >{\raggedright\arraybackslash}p{(\linewidth - 2\tabcolsep) * \real{0.7195}}@{}}
\toprule\noalign{}
\begin{minipage}[b]{\linewidth}\raggedright
Cause
\end{minipage} & \begin{minipage}[b]{\linewidth}\raggedright
Additional Features
\end{minipage} \\
\midrule\noalign{}
\endhead
\bottomrule\noalign{}
\endlastfoot
Iron deficiency & Koilonychia, angular stomatitis, pica \\
Folate/B12 deficiency & Glossitis, hyperpigmentation, neurological
deficits (B12) \\
Haemolytic anaemia & Jaundice, splenomegaly, dark urine \\
Sickle cell disease & Painful crises, leg ulcers, hepatosplenomegaly \\
Malaria & Fever, splenomegaly, hepatomegaly \\
\end{longtable}

\section{Evaluation of a Child with
Anaemia}\label{evaluation-of-a-child-with-anaemia}

\subsection{1. History}\label{history-5}

\begin{itemize}
\tightlist
\item
  Dietary intake (iron-rich foods, cow's milk overuse).
\item
  Chronic illnesses or infections.
\item
  Drug history (antimalarials, sulpha drugs).
\item
  Family history of haemoglobinopathies.
\item
  Stool characteristics (for parasitic or GI blood loss).
\end{itemize}

\subsection{2. Physical Examination}\label{physical-examination-3}

\begin{itemize}
\tightlist
\item
  Pallor and jaundice.
\item
  Growth assessment.
\item
  Splenomegaly or hepatomegaly.
\item
  Lymphadenopathy.
\item
  Cardiac signs (murmurs, gallop rhythm).
\end{itemize}

\subsection{3. Laboratory
Investigations}\label{laboratory-investigations-3}

\begin{longtable}[]{@{}
  >{\raggedright\arraybackslash}p{(\linewidth - 2\tabcolsep) * \real{0.4070}}
  >{\raggedright\arraybackslash}p{(\linewidth - 2\tabcolsep) * \real{0.5930}}@{}}
\toprule\noalign{}
\begin{minipage}[b]{\linewidth}\raggedright
Test
\end{minipage} & \begin{minipage}[b]{\linewidth}\raggedright
Purpose
\end{minipage} \\
\midrule\noalign{}
\endhead
\bottomrule\noalign{}
\endlastfoot
Full blood count (FBC) & Hb, RBC indices (MCV, MCH, MCHC) \\
Peripheral blood film & Morphology and clues to cause \\
Reticulocyte count & Distinguishes decreased production
vs.~haemolysis \\
Serum ferritin, iron, TIBC & Assess iron status \\
Serum folate, vitamin B12 & For macrocytic anaemia \\
Malaria test (RDT or smear) & Detect malaria parasitaemia \\
Stool examination & Detect ova, parasites, occult blood \\
Haemoglobin electrophoresis & Diagnose sickle cell or thalassaemia \\
Direct antiglobulin test (Coombs) & Immune haemolysis \\
\end{longtable}

\section{Iron Deficiency Anaemia
(IDA)}\label{iron-deficiency-anaemia-ida}

\subsection{Epidemiology}\label{epidemiology-12}

IDA accounts for more than half of anaemia cases in African children.

\subsection{Aetiology}\label{aetiology-23}

\begin{itemize}
\tightlist
\item
  Low dietary intake (mainly cereal-based diets, little meat).
\item
  Poor absorption (intestinal infections, celiac disease).
\item
  Chronic blood loss (hookworm, schistosomiasis, menstruation in
  adolescents).
\item
  Increased requirements (rapid growth, infection recovery).
\end{itemize}

\subsection{Pathophysiology}\label{pathophysiology-39}

Iron depletion progresses through three stages: 1. \textbf{Iron
depletion:} Reduced ferritin stores. 2. \textbf{Iron-deficient
erythropoiesis:} Low transferrin saturation. 3. \textbf{Iron deficiency
anaemia:} Low Hb and microcytosis.

\subsection{Clinical Features}\label{clinical-features-56}

\begin{itemize}
\tightlist
\item
  Pallor, fatigue, irritability.
\item
  Koilonychia (spoon nails).
\item
  Pica (craving for non-food substances).
\item
  Glossitis and angular cheilitis.
\end{itemize}

\subsection{Laboratory Findings}\label{laboratory-findings}

\begin{itemize}
\tightlist
\item
  Low Hb, MCV, MCH.
\item
  Low serum ferritin and iron.
\item
  High total iron-binding capacity (TIBC).
\item
  Hypochromic microcytic RBCs.
\end{itemize}

\subsection{Management}\label{management-54}

\begin{enumerate}
\def\labelenumi{\arabic{enumi}.}
\tightlist
\item
  \textbf{Treat underlying cause} (e.g., worms, dietary deficiency).
\item
  \textbf{Iron supplementation:}

  \begin{itemize}
  \tightlist
  \item
    Elemental iron 3--6 mg/kg/day orally for 3 months after Hb
    normalization.
  \item
    Ferrous sulphate preferred; give with vitamin C-rich juice.
  \end{itemize}
\item
  \textbf{Dietary advice:} Encourage iron-rich foods (meat, beans, green
  leafy vegetables).
\item
  \textbf{Prevent reinfection:} Deworming, malaria prevention.
\end{enumerate}

\chapter{Haemolytic Anaemias}\label{haemolytic-anaemias}

\section{Sickle Cell Disease (SCD)}\label{sickle-cell-disease-scd}

\begin{itemize}
\tightlist
\item
  \textbf{Pathophysiology:} Substitution of valine for glutamic acid at
  the 6th position of the β-globin chain causes HbS polymerization when
  deoxygenated.
\item
  \textbf{Complications:} Vaso-occlusive crises, anaemia, infections,
  stroke.
\item
  \textbf{Management:}

  \begin{itemize}
  \tightlist
  \item
    Folic acid 5 mg/day.
  \item
    Penicillin prophylaxis (until age 5).
  \item
    Vaccinations (pneumococcal, Hib, meningococcal).
  \item
    Hydroxyurea to reduce crises.
  \item
    Transfusion for severe anaemia or stroke.
  \end{itemize}
\end{itemize}

\subsection{G6PD Deficiency}\label{g6pd-deficiency}

\begin{itemize}
\tightlist
\item
  \textbf{Mechanism:} Lack of G6PD enzyme → inability to handle
  oxidative stress → RBC lysis.
\item
  \textbf{Triggers:} Certain drugs (sulpha, antimalarials), fava beans,
  infections.
\item
  \textbf{Features:} Jaundice, pallor, dark urine.
\item
  \textbf{Management:} Avoid triggers, treat infections, supportive
  transfusion if severe.
\end{itemize}

\section{Anaemia due to Infections}\label{anaemia-due-to-infections}

\subsection{Malaria}\label{malaria}

\begin{itemize}
\tightlist
\item
  Common cause of acute severe anaemia in endemic regions.
\item
  Mechanisms: RBC destruction, bone marrow suppression, splenic
  sequestration.
\item
  \textbf{Management:}

  \begin{itemize}
  \tightlist
  \item
    Artemisinin-based combination therapy.
  \item
    Blood transfusion for Hb \textless5 g/dL or symptomatic severe
    anaemia.
  \end{itemize}
\end{itemize}

\subsection{Hookworm and
Schistosomiasis}\label{hookworm-and-schistosomiasis}

\begin{itemize}
\tightlist
\item
  Chronic intestinal blood loss.
\item
  \textbf{Prevention:} Periodic deworming, sanitation improvement.
\end{itemize}

\section{Megaloblastic Anaemia}\label{megaloblastic-anaemia}

\subsection{Causes}\label{causes-18}

\begin{itemize}
\tightlist
\item
  Folate or vitamin B12 deficiency.
\end{itemize}

\subsection{Features}\label{features-1}

\begin{itemize}
\tightlist
\item
  Pallor, glossitis, hyperpigmentation.
\item
  Neurological signs (B12 deficiency only).
\end{itemize}

\subsection{Laboratory Findings}\label{laboratory-findings-1}

\begin{itemize}
\tightlist
\item
  Macrocytosis (high MCV), hypersegmented neutrophils.
\end{itemize}

\subsection{Management}\label{management-55}

\begin{itemize}
\tightlist
\item
  Oral folic acid 1--5 mg/day.
\item
  Vitamin B12 1000 µg IM weekly × 6, then monthly if deficient.
\end{itemize}

\section{Aplastic Anaemia}\label{aplastic-anaemia}

\subsection{Definition}\label{definition-35}

Bone marrow failure leading to pancytopenia.

\subsection{Causes}\label{causes-19}

\begin{itemize}
\tightlist
\item
  Idiopathic, viral (hepatitis, EBV, parvovirus), drugs
  (chloramphenicol), radiation.
\end{itemize}

\subsection{Features}\label{features-2}

\begin{itemize}
\tightlist
\item
  Pallor, bleeding (petechiae), infections.
\end{itemize}

\subsection{Management}\label{management-56}

\begin{itemize}
\tightlist
\item
  Stop offending agent, supportive transfusion, bone marrow transplant
  if severe.
\end{itemize}

\section{Severe Anaemia --- Emergency
Management}\label{severe-anaemia-emergency-management}

\subsubsection{Indications for Blood
Transfusion}\label{indications-for-blood-transfusion}

\begin{itemize}
\tightlist
\item
  Hb \textless{} 4 g/dL regardless of symptoms.
\item
  Hb \textless{} 6 g/dL with respiratory distress, heart failure, or
  shock.
\end{itemize}

\subsubsection{Protocol}\label{protocol}

\begin{itemize}
\tightlist
\item
  Packed red cells 10 mL/kg over 3--4 hours (max 15 mL/kg).
\item
  Furosemide 1 mg/kg IV if signs of overload.
\end{itemize}

\section{Prevention of Anaemia}\label{prevention-of-anaemia}

\begin{enumerate}
\def\labelenumi{\arabic{enumi}.}
\tightlist
\item
  \textbf{Exclusive breastfeeding} for 6 months.
\item
  \textbf{Complementary feeding} with iron-rich foods.
\item
  \textbf{Routine deworming} every 6 months after 1 year.
\item
  \textbf{Malaria control:} ITNs, chemoprevention.
\item
  \textbf{Iron and folate supplementation} in high-risk groups.
\item
  \textbf{Control of haemoglobinopathies:} Premarital screening, genetic
  counselling.
\end{enumerate}

\section{Public Health Approaches}\label{public-health-approaches}

\begin{itemize}
\tightlist
\item
  \textbf{Integrated Management of Childhood Illness (IMCI):} includes
  anaemia screening and management.
\item
  \textbf{School health programs:} periodic deworming and iron
  supplementation.
\item
  \textbf{Maternal health interventions:} prevent maternal anaemia.
\item
  \textbf{Nutrition education:} balanced diet promotion.
\end{itemize}

\section{Prognosis}\label{prognosis-34}

\begin{itemize}
\tightlist
\item
  Depends on cause and severity.
\item
  Nutritional anaemias respond well to treatment.
\item
  Haemolytic anaemias and marrow failure syndromes require specialized
  care.
\item
  Chronic anaemia impairs growth and learning outcomes.
\end{itemize}

\section{Key Takeaways}\label{key-takeaways-4}

\begin{itemize}
\tightlist
\item
  Anaemia in children is common, preventable, and treatable.
\item
  Iron deficiency and malaria are leading causes in sub-Saharan Africa.
\item
  Classification and morphological assessment guide diagnosis.
\item
  Management requires identifying underlying causes and addressing
  nutrition and infections.
\item
  Prevention strategies are essential at individual and public health
  levels.
\end{itemize}

\section{Suggested References}\label{suggested-references}

\begin{enumerate}
\def\labelenumi{\arabic{enumi}.}
\tightlist
\item
  World Health Organization. \emph{Haemoglobin concentrations for the
  diagnosis of anaemia and assessment of severity.}
  WHO/NMH/NHD/MNM/11.1, 2023.
\item
  Ghana Health Service. \emph{Child Health Record Book and Treatment
  Guidelines}, 2022.
\item
  Osungbade KO, Oladunjoye AO. Preventive treatments of iron deficiency
  anaemia in children in developing countries: What works? \emph{Afr
  Health Sci}. 2021;21(3):1120--1130.
\item
  Ministry of Health, Ghana. \emph{Standard Treatment Guidelines}, 2022.
\item
  Kassebaum NJ et al.~The global burden of anaemia. \emph{Lancet
  Haematol}. 2020;7(9):e690--e701.
\end{enumerate}

\chapter{Bleeding Disorders}\label{bleeding-disorders}

\section{\texorpdfstring{\textbf{Introduction}}{Introduction}}\label{introduction-54}

Bleeding disorders in children represent a unique and complex group of
hematologic conditions that pose significant diagnostic, therapeutic,
and psychosocial challenges. From inherited conditions like haemophilia
and von Willebrand disease to acquired disorders related to liver
disease, immune dysfunction, or malignancy, the spectrum of bleeding
disorders in the pediatric population is broad and often
under-recognised, especially in low-resource settings.

Early identification and appropriate management of bleeding disorders
are critical in preventing life-threatening complications, ensuring
optimal physical development, and maintaining quality of life. The
pediatric context necessitates a holistic, family-centred approach that
considers not only the child's medical needs but also the psychosocial
impact on caregivers and families.

This section aims to provide a comprehensive yet practical guide to the
diagnosis, evaluation, and management of bleeding disorders in children.
It integrates current scientific knowledge with clinical experience and
real-world considerations across diverse healthcare settings.

Furthermore, special emphasis is placed on challenges specific to
resource-limited settings, advances in gene therapy and personalised
medicine, and the role of education and advocacy in improving outcomes.

\section{\texorpdfstring{\textbf{Hemostasis}
~}{Hemostasis ~}}\label{hemostasis}

The
term~\href{https://www.osmosis.org/learn/Thrombocytopenia:_Nursing}{hemostasis}~is
coined from ``hem-'', which means ``blood'', and ``-stasis'', which
means ``to stop.''~Hence, hemostasis is a physiological process by which
bleeding is stopped at the site of blood vessel injury. It represents an
interplay among blood vessel constriction, platelet plug formation, and
blood coagulation through fibrin clot formation. Hemostasis is essential
for maintaining vascular integrity and preventing haemorrhage. Blood
clot formation at the wrong time or site within a blood vessel is termed
\textbf{Thrombosis,} and this is a pathological condition.

The normal hemostatic process requires a balance between procoagulants
and anticoagulants. An imbalance can lead to either excessive bleeding
or unwanted clot formation.

\begin{figure}[H]

{\centering \pandocbounded{\includegraphics[keepaspectratio]{images/heam-coag-pathway.png}}

}

\caption{Haemostasis: Balance between anticoagulants and procoagulants}

\end{figure}%

\subsection{\texorpdfstring{\textbf{Components of
Hemostasis}}{Components of Hemostasis}}\label{components-of-hemostasis}

There are three (3) main components of haemostasis.

\begin{enumerate}
\def\labelenumi{\arabic{enumi}.}
\tightlist
\item
  Vascular constriction and platelet constriction form the primary
  haemostasis and lead to the formation of the platelet plug.
\item
  The coagulation cascade initiates secondary haemostasis and leads to
  the formation of a fibrin clot.
\item
  The last phase is fibrinolysis, which leads to clot breakdown and
  reestablishment of vessel patency.
\end{enumerate}

\begin{center}
\pandocbounded{\includegraphics[keepaspectratio]{images/haem-sec-coag-pathway.png}}
\end{center}

\subsection{Coagulation Pathway (Secondary
hemostasis)}\label{coagulation-pathway-secondary-hemostasis}

The coagulation pathway is a sequence of enzymatic conversions of
proenzymes into activated forms. It consists of an intrinsic, extrinsic
and a common pathway.

It is worth noting that the traditional model of this cascade, with its
distinct intrinsic and extrinsic pathways merging into a single
pathway,~is an oversimplification. In reality, that is in vivo, these
pathways likely activate each other simultaneously, creating a more
complex and interconnected system.

\begin{figure}[H]

{\centering \pandocbounded{\includegraphics[keepaspectratio]{images/haem-invitro-coag-pathway.png}}

}

\caption{Coagulation cascade in vitro}

\end{figure}%%
\begin{figure}[H]

{\centering \includegraphics[width=6.25in,height=\textheight,keepaspectratio]{images/haem-clotting-factors.jpg}

}

\caption{Numerical designation of clotting factors and their Alternative
names}

\end{figure}%

\section{\texorpdfstring{\textbf{Bleeding
Disorders}}{Bleeding Disorders}}\label{bleeding-disorders-1}

Bleeding disorders, also known as bleeding diathesis, are various
conditions characterised by abnormalities in hemostasis that lead to
abnormal bleeding. It mainly results from an imbalance in hemostasis,
with the balance tilting towards less procoagulants or increased
anti-coagulant or fibrinolytic activity.

\textbf{History}

Areas of emphasis.

\begin{itemize}
\tightlist
\item
  Age of onset (helps distinguish between congenital and acquired
  disease)
\item
  Bleed (spontaneous or post-haemostatic challenge (HC) i.e., Trauma,
  surgery)
\item
  *Location of bleed: umbilical stump bleed, bleeding post circumcision,
  epistaxis, gum bleed, haemarthrosis, menorrhagia, post-partum
  haemorrhage.
\item
  Transfusion history, including when the patient was last transfused.
\item
  Frequency and pattern of bleed (one site or multiple sites)
\item
  Drug history including herbs (aspirin, clopidogrel, tirofiban, etc)
\item
  Menstrual and obstetric history
\item
  Neonate: vit K injection
\item
  History of haemostatic challenge (surgery, dental work, injury)
\item
  Haemostatic measures given
\item
  Family history of abnormal bleeding (gender of affected relatives)
\item
  Other comorbidities: malignancy, liver disease, renal disease, etc
\item
  Inconsistent history (non-accidental injury)
\end{itemize}

*May suggest the part of the haemostatic system affected; primary
haemostasis often experiences mucocutaneous bleeding, and secondary
haemostasis has deep tissue bleeds, such as muscle and joint bleeds

NB: At the end of history, these questions must be ``why is my patient
bleeding''? Will my patient bleed after an HC? Could other family
members be at risk of bleeding?

Any objective way to assess bleeding is with the BAT score (bleeding
assessment tool), e.g., the ISTH BAT.

\textbf{Physical examination}

Look out for these clinical features when examining the patient.

Dysmorphic features

Skin and mucosa (oral): petechiae, purpura, ecchymosis

Joint swelling and deformities

Significant lymphadenopathy

Organomegaly

Presence of haemangioma (Kasabach-Merritt syndrome)

\begin{figure}[H]

{\centering \pandocbounded{\includegraphics[keepaspectratio]{images/haem-petechae.png}}

}

\caption{Petechaie and Prupura}

\end{figure}%%
\begin{figure}[H]

{\centering \pandocbounded{\includegraphics[keepaspectratio]{images/heam-hemarthrosis.png}}

}

\caption{Hemarthrosis}

\end{figure}%

\subsection{Screening tests}\label{screening-tests}

\begin{itemize}
\tightlist
\item
  CBC: level of haemoglobin, Total WBC, differential WBC, Platelets
\item
  Liver function test
\item
  Coagulation tests: Prothrombin time (PT), Activated partial
  thromboplastin time (aPTT), fibrinogen assay, thrombin time (TT),
  clotting factor assays, platelet count
\item
  Specific tests: vWF antigen and vWF-Ristocetin cofactor activity,
  Platelet function analyser (not recommended)
\end{itemize}

\begin{figure}[H]

{\centering \pandocbounded{\includegraphics[keepaspectratio]{images/heam-screening-tests.png}}

}

\caption{Illustration of the screening tests}

\end{figure}%

\subsection{Screening tests and
differentials}\label{screening-tests-and-differentials}

\ul{\textbf{Prolonged PT}}

\begin{itemize}
\tightlist
\item
  Vit K def
\item
  Anticoagulants eg warfarin
\item
  FVII def
\item
  FVII inhibitor
\item
  Liver dx
\end{itemize}

\ul{\textbf{Prolonged aPTT}}

\begin{itemize}
\tightlist
\item
  vWD
\item
  FVIII, FIX~ and XI deficiency
\item
  Lupus anticoagulants
\item
  Heparin
\item
  FXII, HMWK, PK def
\end{itemize}

\ul{\textbf{Prolonged PT and aPTT}}

\begin{itemize}
\tightlist
\item
  Vit K def
\item
  F II, V~ and IX def
\item
  Fibrinogen def
\item
  Liver dx
\end{itemize}

\ul{\textbf{Prolonged PT and aPTT + low platelets}}

\begin{itemize}
\tightlist
\item
  DIC
\item
  Liver disease
\end{itemize}

\ul{\textbf{Normal PT , aPTT and Platelet}}

\begin{itemize}
\tightlist
\item
  vWD
\item
  FXIII def
\item
  Platelet function defect
\item
  Vascular disorders
\end{itemize}

\begin{figure}[H]

{\centering \includegraphics[width=6.25in,height=\textheight,keepaspectratio]{images/haem-prim-sec-hemostasis.png}

}

\caption{Primary and Secondary Hemostasis Disorders}

\end{figure}%

\chapter{Transfusion}\label{transfusion}

\section{Introduction}\label{introduction-55}

Blood transfusion is a critical, lifesaving intervention in paediatric
practice. In sub-Saharan Africa, including Ghana, transfusion plays a
vital role in the management of severe anaemia, neonatal conditions,
trauma, malignancies, and surgical emergencies. However, transfusion
also carries significant risks such as transfusion reactions,
infections, circulatory overload, and alloimmunization.

Paediatric transfusion differs from adult transfusion in several key
ways: children's blood volumes are smaller, immune systems are less
mature, and many transfusion indications arise from conditions unique to
childhood (e.g., severe malaria anaemia, haemoglobinopathies).
Therefore, understanding transfusion principles tailored to paediatrics
is essential for safe and effective care.

This chapter provides a comprehensive review of paediatric transfusion
medicine---including physiology, indications, product selection, dosing,
safety, and Ghana-specific considerations. It is intended for medical
students, residents, and practising paediatricians preparing for ward
work and examinations.

\section{Physiology of Blood and Components in
Children}\label{physiology-of-blood-and-components-in-children}

Children have different blood volumes and haematological characteristics
compared with adults.

\subsection{Blood Volume by Age}\label{blood-volume-by-age}

\begin{longtable}[]{@{}ll@{}}
\toprule\noalign{}
Age Group & Approximate Blood Volume \\
\midrule\noalign{}
\endhead
\bottomrule\noalign{}
\endlastfoot
Preterm neonate & 90--100 mL/kg \\
Term neonate & 80--90 mL/kg \\
Infant (1--12 months) & 75--80 mL/kg \\
Child & 70--75 mL/kg \\
Adolescent & 65--70 mL/kg \\
\end{longtable}

The smaller the blood volume, the greater the physiological impact of
even minor blood loss. A loss of 20 mL of blood is negligible in an
adult but clinically relevant in a neonate.

\subsection{Developmental Haematology}\label{developmental-haematology}

\begin{itemize}
\tightlist
\item
  \textbf{Neonatal haemoglobin} is predominantly fetal haemoglobin
  (HbF), with high oxygen affinity.
\item
  \textbf{Physiological anaemia} occurs at 6--12 weeks due to decline in
  erythropoietin.
\item
  \textbf{Iron stores} deplete early in life, making infants more
  vulnerable to iron deficiency.
\item
  \textbf{Immune system immaturity} increases susceptibility to
  transfusion-transmitted pathogens and alloimmunization.
\end{itemize}

\section{Indications for Paediatric
Transfusion}\label{indications-for-paediatric-transfusion}

Transfusion decisions must balance expected benefits against risks.
Indications can be grouped into:

\begin{enumerate}
\def\labelenumi{\arabic{enumi}.}
\tightlist
\item
  \textbf{Red Cell Transfusion}
\item
  \textbf{Platelet Transfusion}
\item
  \textbf{Plasma (FFP) Transfusion}
\item
  \textbf{Cryoprecipitate Transfusion}
\item
  \textbf{Whole Blood (rare, selected contexts)}
\end{enumerate}

\subsection{Indications for Red Cell
Transfusion}\label{indications-for-red-cell-transfusion}

\subsubsection{Acute Severe Anaemia}\label{acute-severe-anaemia}

Common causes in Ghana:

\begin{itemize}
\tightlist
\item
  Severe malaria
\item
  Sickle cell disease crises
\item
  Acute haemolysis (G6PD deficiency)
\item
  Acute bleeding (trauma, obstetric haemorrhage in adolescents)
\item
  Severe sepsis
\end{itemize}

\textbf{Thresholds for transfusion (common paediatric practice):}

\begin{itemize}
\tightlist
\item
  Hb \textless{} 4 g/dL --- transfuse immediately.
\item
  Hb 4--6 g/dL --- transfuse if symptomatic (shock, respiratory
  distress, heart failure).
\item
  Hb 6--10 g/dL --- consider only in special situations such as:

  \begin{itemize}
  \tightlist
  \item
    ongoing blood loss\\
  \item
    cardiopulmonary disease\\
  \item
    severe infection with compromised oxygen delivery\\
  \end{itemize}
\item
  Hb \textgreater{} 10 g/dL --- rarely indicated.
\end{itemize}

\subsubsection{Chronic Anaemia}\label{chronic-anaemia}

Conditions:

\begin{itemize}
\tightlist
\item
  Thalassemia major\\
\item
  Sickle cell disease (selected indications; not routine)\\
\item
  Bone marrow failure syndromes\\
\item
  Chronic renal disease
\end{itemize}

Transfusions are individualized, often part of long-term management.

\subsubsection{Perioperative
Transfusion}\label{perioperative-transfusion}

Indications:

\begin{itemize}
\tightlist
\item
  Anticipated significant blood loss
\item
  Preoperative correction of Hb \textless{} 8 g/dL in major surgery
\item
  Intraoperative haemodynamic instability due to blood loss
\end{itemize}

\subsubsection{Neonatal Transfusion
Indications}\label{neonatal-transfusion-indications}

\begin{itemize}
\tightlist
\item
  Symptomatic anaemia\\
\item
  Anaemia of prematurity with severe cardiorespiratory compromise\\
\item
  Significant iatrogenic blood loss in NICUs\\
\item
  Exchange transfusion for severe jaundice (RBC + plasma)
\end{itemize}

\section{Blood Components and Their
Uses}\label{blood-components-and-their-uses}

Modern transfusion practice prefers \textbf{component therapy} rather
than whole blood.

\subsection{Packed Red Blood Cells
(PRBCs)}\label{packed-red-blood-cells-prbcs}

\begin{itemize}
\tightlist
\item
  Indicated for anaemia with reduced oxygen-carrying capacity.
\item
  Haematocrit: 50--70\%.
\end{itemize}

\subsection{Whole Blood}\label{whole-blood}

\begin{itemize}
\tightlist
\item
  Used in Ghana in select situations such as severe acute haemorrhage in
  resource-constrained facilities.
\item
  Haematocrit \textasciitilde40\%.
\item
  Higher risk of volume overload.
\end{itemize}

\subsection{Platelets}\label{platelets}

Indicated for:

\begin{itemize}
\tightlist
\item
  Active bleeding with thrombocytopenia
\item
  Prophylaxis in very low platelet counts (\textless10 × 10⁹/L)
\item
  Platelet dysfunction (e.g., uremia)
\end{itemize}

\subsection{Fresh Frozen Plasma (FFP)}\label{fresh-frozen-plasma-ffp}

Indications: - Bleeding due to coagulation factor deficiency - DIC -
Warfarin reversal in adolescents

Not for:

\begin{itemize}
\tightlist
\item
  Volume expansion
\item
  Simple anaemia
\end{itemize}

\subsection{Cryoprecipitate}\label{cryoprecipitate}

Contains fibrinogen, von Willebrand factor, factor VIII.

Indications: - Hypofibrinogenemia (\textless1 g/L) - Disseminated
intravascular coagulation - Certain bleeding disorders

\section{Dosing of Blood Components in
Children}\label{dosing-of-blood-components-in-children}

Paediatric transfusion doses are weight-based.

\subsubsection{Red Blood Cells}\label{red-blood-cells}

\begin{itemize}
\tightlist
\item
  \textbf{Dose:} 10--15 mL/kg (PRBC)
\item
  \textbf{Expected Hb rise:} 1--2 g/dL per 10 mL/kg
\item
  Neonates: 10--15 mL/kg over 2--4 hours
\end{itemize}

\subsubsection{Whole Blood}\label{whole-blood-1}

\begin{itemize}
\tightlist
\item
  20 mL/kg (for acute blood loss)
\end{itemize}

\subsubsection{Platelets}\label{platelets-1}

\begin{itemize}
\tightlist
\item
  \textbf{Dose:} 10--15 mL/kg
\end{itemize}

or

\begin{itemize}
\tightlist
\item
  1 unit per 10 kg body weight
\end{itemize}

Expected rise: 20--40 × 10⁹/L

\subsubsection{FFP}\label{ffp}

\begin{itemize}
\tightlist
\item
  10--15 mL/kg
\end{itemize}

\subsubsection{Cryoprecipitate}\label{cryoprecipitate-1}

\begin{itemize}
\tightlist
\item
  1 unit per 5 kg body weight
\end{itemize}

or

\begin{itemize}
\tightlist
\item
  5--10 mL/kg
\end{itemize}

\section{Special Considerations in Neonatal
Transfusion}\label{special-considerations-in-neonatal-transfusion}

Neonates have unique physiology requiring careful approach.

\subsection{Key Principles}\label{key-principles}

\begin{itemize}
\tightlist
\item
  Use \textbf{CMV-reduced}, \textbf{irradiated}, and \textbf{fresh
  (\textless5 days old)} blood where available.
\item
  Avoid potassium accumulation by not using blood stored too long.
\item
  Transfuse slowly: \textbf{5 mL/kg/hr} unless urgent.
\end{itemize}

\subsection{Exchange Transfusion}\label{exchange-transfusion}

Indications:

\begin{itemize}
\tightlist
\item
  Severe hyperbilirubinemia
\item
  Haemolytic disease of the newborn
\end{itemize}

Blood requirements: - Crossmatch with mother's blood - O-negative,
antigen-compatible PRBCs reconstituted with plasma - Haematocrit
40--50\%

\section{Transfusion Decision-Making: Thresholds and Clinical
Judgment}\label{transfusion-decision-making-thresholds-and-clinical-judgment}

In resource-limited settings like Ghana, clinical judgment is especially
important.

\subsection{Clinical Factors Supporting
Transfusion:}\label{clinical-factors-supporting-transfusion}

\begin{itemize}
\tightlist
\item
  Tachycardia
\item
  Respiratory distress
\item
  Signs of cardiac failure
\item
  Lethargy or altered consciousness
\item
  Hypoxia (SpO2 \textless{} 92\%)
\item
  Shock
\end{itemize}

\subsection{Laboratory Red Flags:}\label{laboratory-red-flags}

\begin{itemize}
\tightlist
\item
  Hb \textless{} 4 g/dL (emergency)
\item
  Hb \textless{} 6 g/dL with symptoms
\item
  Rising lactate
\item
  Severe thrombocytopenia (\textless10 × 10⁹/L)
\end{itemize}

\subsection{When Not to Transfuse:}\label{when-not-to-transfuse}

\begin{itemize}
\tightlist
\item
  Mild asymptomatic anaemia
\item
  Anaemia that is nutritional and not severe
\item
  Iron deficiency responsive to oral supplementation
\item
  Fever alone without evidence of haemodynamic compromise
\end{itemize}

\section{Risks and Complications of
Transfusion}\label{risks-and-complications-of-transfusion}

Transfusion is not without danger. Vigilance is mandatory.

\subsection{Acute Transfusion
Reactions}\label{acute-transfusion-reactions}

\begin{itemize}
\tightlist
\item
  \textbf{Febrile non-haemolytic reaction}
\item
  \textbf{Allergic reactions} (urticaria to anaphylaxis)
\item
  \textbf{Acute haemolytic reaction} (ABO mismatch)
\item
  \textbf{Transfusion-associated circulatory overload (TACO)}
\item
  \textbf{Transfusion-related acute lung injury (TRALI)}
\end{itemize}

\subsection{Delayed Reactions}\label{delayed-reactions}

\begin{itemize}
\tightlist
\item
  Delayed haemolytic transfusion reaction
\item
  Transfusion-transmitted infections
\item
  Alloimmunization (especially in sickle cell patients)
\item
  Iron overload (chronic transfusers)
\end{itemize}

\subsection{Transfusion-Transmitted Infections
(TTIs)}\label{transfusion-transmitted-infections-ttis}

In Ghana, screening is mandatory for:

\begin{itemize}
\tightlist
\item
  HIV
\item
  Hepatitis B
\item
  Hepatitis C
\item
  Syphilis
\end{itemize}

Malaria transmission risk exists despite screening.

\subsection{Iron Overload}\label{iron-overload}

Occurs in:

\begin{itemize}
\tightlist
\item
  Thalassemia major
\item
  Chronic transfusion protocols\\
  \textbf{Management:} Iron chelation with deferasirox or deferoxamine.
\end{itemize}

\section{Transfusion Safety and
Protocols}\label{transfusion-safety-and-protocols}

Safe transfusion practice involves multiple layers of protection.

\subsection{Pre-Transfusion Steps}\label{pre-transfusion-steps}

\begin{itemize}
\tightlist
\item
  Confirm indication.
\item
  Check baseline vitals.
\item
  Obtain informed consent.
\item
  Crossmatch blood (patient name, age, hospital number).
\item
  Use appropriate component.
\end{itemize}

\subsection{Bedside Checks}\label{bedside-checks}

Always perform the \textbf{three Rs}:

\begin{enumerate}
\def\labelenumi{\arabic{enumi}.}
\tightlist
\item
  \textbf{Right patient}\\
\item
  \textbf{Right blood}\\
\item
  \textbf{Right time}
\end{enumerate}

Any mismatch can be fatal.

\subsection{3. Monitoring During
Transfusion}\label{monitoring-during-transfusion}

\begin{longtable}[]{@{}
  >{\raggedright\arraybackslash}p{(\linewidth - 2\tabcolsep) * \real{0.2329}}
  >{\raggedright\arraybackslash}p{(\linewidth - 2\tabcolsep) * \real{0.7671}}@{}}
\toprule\noalign{}
\begin{minipage}[b]{\linewidth}\raggedright
Time
\end{minipage} & \begin{minipage}[b]{\linewidth}\raggedright
Action
\end{minipage} \\
\midrule\noalign{}
\endhead
\bottomrule\noalign{}
\endlastfoot
Start & Observe for first 15 minutes; vital signs every 15 min \\
Mid-transfusion & Vitals every 30 min \\
End & Vitals at completion \\
Post & Monitor for late reactions \\
\end{longtable}

Observe for:

\begin{itemize}
\tightlist
\item
  Fever\\
\item
  Rigors\\
\item
  Rash\\
\item
  Difficulty breathing\\
\item
  Hypotension\\
\item
  Haemoglobinuria
\end{itemize}

\subsection{Management of Transfusion
Reactions}\label{management-of-transfusion-reactions}

\begin{itemize}
\tightlist
\item
  Stop transfusion immediately.
\item
  Maintain IV line with saline.
\item
  Check vitals and notify senior/clincians.
\item
  Send blood bag and samples for investigation.
\item
  Treat according to reaction type (antihistamine, adrenaline, oxygen
  support etc.).
\end{itemize}

\section{Transfusion in Special
Groups}\label{transfusion-in-special-groups}

\subsection{Children With Severe
Malaria}\label{children-with-severe-malaria}

Common in Ghana.

Key points:

\begin{itemize}
\tightlist
\item
  Hb \textless{} 6 g/dL with respiratory distress → urgent transfusion.
\item
  Monitor for fluid overload.
\item
  Treat malaria concurrently with artesunate.
\end{itemize}

\subsection{Sickle Cell Disease}\label{sickle-cell-disease-2}

\begin{itemize}
\tightlist
\item
  Transfuse only for specific indications:

  \begin{itemize}
  \tightlist
  \item
    Acute chest syndrome\\
  \item
    Stroke\\
  \item
    Severe anaemia\\
  \item
    Preoperative optimization
  \end{itemize}
\item
  Use \textbf{HbS-negative}, matched blood if available.\\
\item
  Avoid unnecessary transfusions to prevent alloimmunization.
\end{itemize}

\subsection{Oncology Patients}\label{oncology-patients}

\begin{itemize}
\tightlist
\item
  Require recurrent transfusions (platelets + RBCs).
\item
  High alloimmunization risk.
\item
  Need irradiated components if receiving chemotherapy causing
  immunosuppression.
\end{itemize}

\section{Blood Banking and Transfusion Services in
Ghana}\label{blood-banking-and-transfusion-services-in-ghana}

The National Blood Service (NBS) Ghana coordinates:

\begin{itemize}
\tightlist
\item
  Blood donor recruitment
\item
  Testing and processing
\item
  Distribution to health facilities
\end{itemize}

\subsection{Challenges:}\label{challenges}

\begin{itemize}
\tightlist
\item
  Heavy reliance on \textbf{replacement donors} rather than voluntary
  donors
\item
  Regular shortages necessitating the use of whole blood
\item
  Limited availability of specialized products such as irradiated blood
\item
  Challenges in cold chain maintenance in rural hospitals
\item
  High prevalence of TTIs in general population increasing risk of
  transfusion-transmitted infections
\end{itemize}

\subsection{Strengths:}\label{strengths}

\begin{itemize}
\tightlist
\item
  National standardized screening procedures
\item
  Expanded donor mobilization programmes
\item
  Increasing availability of PRBCs and FFP in tertiary centres
\end{itemize}

Improving Ghana's transfusion services requires enhanced voluntary
donation and strengthened hospital transfusion committees.

\section{Clinical Approach: How to Manage a Child Requiring
Transfusion}\label{clinical-approach-how-to-manage-a-child-requiring-transfusion}

\textbf{Step 1: Confirm Indication}

Ask: ``Will this transfusion save the child's life or significantly
improve outcome?''

\textbf{Step 2: Evaluate Urgency}

\begin{itemize}
\tightlist
\item
  Emergency (Hb \textless{} 4 g/dL or shock)\\
\item
  Semi-urgent\\
\item
  Elective
\end{itemize}

\textbf{Step 3: Choose the Right Product}

\begin{itemize}
\tightlist
\item
  Anaemia → PRBC\\
\item
  Bleeding due to coagulopathy → FFP\\
\item
  Severe thrombocytopenia → Platelets\\
\item
  Hypofibrinogenemia → Cryoprecipitate
\end{itemize}

\textbf{Step 4: Calculate Dose}

Weight-based dosing.

\textbf{Step 5: Bedside Safety Checks}

Identity, unit number, compatibility.

\textbf{Step 6: Monitor and Document}

Before, during, and after the transfusion.

\textbf{Step 7: Reassess}

A post-transfusion Hb should be checked only when clinically necessary.

\section{Key Points for Exams and Clinical
Practice}\label{key-points-for-exams-and-clinical-practice}

\begin{itemize}
\tightlist
\item
  PRBC dose: \textbf{10--15 mL/kg} raises Hb by 1--2 g/dL.\\
\item
  Whole blood for \textbf{massive haemorrhage}: 20 mL/kg.\\
\item
  Platelets indicated when \textbf{platelets \textless10 × 10⁹/L}
  without bleeding.\\
\item
  Never use FFP for \textbf{nutritional anaemia} or \textbf{volume
  expansion}.\\
\item
  Neonates require \textbf{fresh}, \textbf{CMV-reduced},
  \textbf{irradiated} blood.\\
\item
  In Ghana, severe malaria is the leading cause of paediatric
  transfusion.\\
\item
  Monitor closely for TACO and TRALI.\\
\item
  Document all transfusion reactions and report to hospital transfusion
  committee.
\end{itemize}

\section{Further Reading}\label{further-reading-8}

\begin{enumerate}
\def\labelenumi{\arabic{enumi}.}
\tightlist
\item
  WHO. \emph{Guidelines on the Use of Blood and Blood Products}. Geneva:
  World Health Organization.\\
\item
  Klein HG, Anstee DJ. \emph{Mollison's Blood Transfusion in Clinical
  Medicine}, 13th ed.\\
\item
  West African College of Physicians (WACP). \emph{Curriculum for
  Paediatrics}.\\
\item
  National Blood Service, Ghana. \emph{Guidelines for Clinical
  Transfusion Practice}.\\
\item
  Roback JD et al.~\emph{Technical Manual of the American Association of
  Blood Banks (AABB)}, 20th ed.\\
\item
  Bates I et al.~``Transfusion in Sub-Saharan Africa: Challenges and
  Solutions.'' \emph{Lancet.}
\end{enumerate}

\chapter{Bone Marrow Failure
Syndormes}\label{bone-marrow-failure-syndormes}

\section{Introduction}\label{introduction-56}

Bone marrow failure syndromes (BMFS) refer to a heterogeneous group of
disorders in which the bone marrow is unable to produce an adequate
number of one or more blood cell lines. These disorders may affect red
blood cells, white blood cells, and/or platelets, leading respectively
to anemia, leukopenia, and thrombocytopenia. Clinically, they present
with fatigue, recurrent infections, mucosal bleeding, and growth
impairment in children.

BMFS may be \textbf{congenital (inherited)} or \textbf{acquired}, and
early recognition is essential for timely management. In sub-Saharan
Africa, including Ghana, diagnosis is often delayed due to limited
access to specialized laboratory investigations such as bone marrow
cytogenetics, flow cytometry, telomere length testing, and genetic
sequencing. As a result, these disorders may be misdiagnosed as
nutritional anemia, malaria, leukemia, or sepsis.

This chapter provides a comprehensive review of both congenital and
acquired bone marrow failure syndromes in children, with a focus on
clinical features, diagnostic challenges, and management approaches
applicable within Ghana and similar resource-limited settings.

\section{Normal Structure and Physiology of Bone
Marrow}\label{normal-structure-and-physiology-of-bone-marrow}

Bone marrow is responsible for hematopoiesis --- the production of red
cells, white cells, and platelets. Key components include:

\begin{itemize}
\tightlist
\item
  \textbf{Hematopoietic stem cells (HSCs)}: pluripotent cells capable of
  self-renewal and differentiation.
\item
  \textbf{Stromal cells}: including fibroblasts, adipocytes, endothelial
  cells, and macrophages that support HSC survival.
\item
  \textbf{Growth factors and cytokines}: such as erythropoietin, G-CSF,
  and thrombopoietin.
\end{itemize}

Any insult to stem cells, stromal support, or regulatory cytokines can
lead to bone marrow failure.

\section{Classification of Bone Marrow Failure
Syndromes}\label{classification-of-bone-marrow-failure-syndromes}

BMFS can be broadly categorized as:

\subsection{\texorpdfstring{\textbf{Congenital (Inherited) Bone Marrow
Failure Syndromes
(IBMFS)}}{Congenital (Inherited) Bone Marrow Failure Syndromes (IBMFS)}}\label{congenital-inherited-bone-marrow-failure-syndromes-ibmfs}

These include: - Fanconi anaemia\\
- Diamond--Blackfan anaemia\\
- Shwachman--Diamond syndrome\\
- Dyskeratosis congenita (telomeropathies)\\
- Congenital amegakaryocytic thrombocytopenia\\
- Thrombocytopenia with absent radius (TAR) syndrome\\
- Congenital neutropenias (e.g., Kostmann syndrome)

\subsection{\texorpdfstring{\textbf{Acquired Bone Marrow Failure
Syndromes}}{Acquired Bone Marrow Failure Syndromes}}\label{acquired-bone-marrow-failure-syndromes}

Including: - Aplastic anaemia\\
- Transient erythroblastopenia of childhood (TEC)\\
- Drug-induced bone marrow suppression\\
- Viral-associated marrow suppression (e.g., Parvovirus B19, HIV)\\
- Nutritional (e.g., severe malnutrition with marrow hypoplasia)\\
- Immune-mediated marrow aplasia

\section{Congenital Bone Marrow Failure
Syndromes}\label{congenital-bone-marrow-failure-syndromes}

\subsection{Fanconi Anaemia (FA)}\label{fanconi-anaemia-fa}

\subsubsection{Overview and
Pathophysiology}\label{overview-and-pathophysiology}

Fanconi anaemia is the most common inherited cause of aplastic anemia in
children. It is an autosomal recessive disorder characterized by
impaired DNA repair, leading to chromosomal instability and progressive
marrow failure.

\subsubsection{Clinical Features}\label{clinical-features-57}

FA presents with a wide spectrum of abnormalities: -
\textbf{Hematologic}: pancytopenia, usually presenting between ages
5--10 years. - \textbf{Physical anomalies}: - Short stature\\
- Absent or hypoplastic thumbs\\
- Radial ray defects\\
- Skin hyperpigmentation or café-au-lait spots\\
- Renal anomalies\\
- Microcephaly - \textbf{Endocrine features}: hypothyroidism, glucose
intolerance, delayed puberty. - \textbf{Malignancy risks}: - Acute
myeloid leukemia\\
- Squamous cell carcinoma of head/neck and genital tract

\subsubsection{Diagnosis}\label{diagnosis-26}

\begin{itemize}
\tightlist
\item
  CBC: pancytopenia\\
\item
  Bone marrow: hypocellular\\
\item
  \textbf{Chromosomal breakage test} using diepoxybutane (DEB) or
  mitomycin C is diagnostic\\
\item
  Genetic testing (if available)\\
\item
  Screening for associated anomalies (renal ultrasound, endocrine
  evaluation)
\end{itemize}

\subsubsection{Management}\label{management-57}

\begin{itemize}
\tightlist
\item
  Supportive care: transfusions, infection control\\
\item
  Androgens (e.g., oxymetholone) may improve counts temporarily\\
\item
  Hematopoietic stem cell transplantation (HSCT) is curative\\
\item
  Cancer surveillance\\
\item
  Genetic counselling for families
\end{itemize}

\subsubsection{Ghana Context}\label{ghana-context}

FA is likely underdiagnosed. Limited access to DEB testing hampers
confirmation, often leading to misclassification as idiopathic aplastic
anemia.

\subsection{Diamond--Blackfan Anaemia
(DBA)}\label{diamondblackfan-anaemia-dba}

\subsubsection{Overview}\label{overview-5}

DBA is a congenital pure red cell aplasia presenting in infancy due to
mutations affecting ribosomal protein synthesis.

\subsubsection{Clinical Features}\label{clinical-features-58}

\begin{itemize}
\tightlist
\item
  Onset before 1 year of age\\
\item
  Severe macrocytic anemia\\
\item
  Reticulocytopenia\\
\item
  Physical anomalies in 30--50\%:

  \begin{itemize}
  \tightlist
  \item
    Craniofacial abnormalities\\
  \item
    Thumb anomalies\\
  \item
    Short stature\\
  \item
    Cardiac defects
  \end{itemize}
\end{itemize}

\subsubsection{Diagnosis}\label{diagnosis-27}

\begin{itemize}
\tightlist
\item
  CBC: macrocytic anemia with low reticulocytes\\
\item
  Bone marrow: absent or decreased erythroid precursors\\
\item
  Elevated erythropoietin levels\\
\item
  Genetic testing (RPS19 mutations common)
\end{itemize}

\subsubsection{Management}\label{management-58}

\begin{itemize}
\tightlist
\item
  Prednisolone (first-line therapy)\\
\item
  Chronic transfusion therapy for steroid-resistant patients\\
\item
  Iron chelation if transfusion-dependent\\
\item
  HSCT is curative\\
\item
  Nutritional and endocrine monitoring
\end{itemize}

\subsection{Shwachman--Diamond Syndrome
(SDS)}\label{shwachmandiamond-syndrome-sds}

\subsubsection{Overview}\label{overview-6}

Autosomal recessive disorder characterized by: - Bone marrow failure
(neutropenia common)\\
- Exocrine pancreatic insufficiency\\
- Skeletal abnormalities

\subsubsection{Clinical Features}\label{clinical-features-59}

\begin{itemize}
\tightlist
\item
  Failure to thrive\\
\item
  Recurrent infections\\
\item
  Steatorrhea due to pancreatic insufficiency\\
\item
  Variable cytopenias
\end{itemize}

\subsubsection{Diagnosis}\label{diagnosis-28}

\begin{itemize}
\tightlist
\item
  CBC showing neutropenia\\
\item
  Low fecal elastase (pancreatic insufficiency)\\
\item
  Bone marrow with hypocellularity\\
\item
  Genetic testing (SBDS mutations)
\end{itemize}

\subsubsection{Management}\label{management-59}

\begin{itemize}
\tightlist
\item
  Pancreatic enzyme replacement therapy\\
\item
  G-CSF for severe neutropenia\\
\item
  HSCT for marrow failure\\
\item
  Monitor for malignant transformation (risk of AML)
\end{itemize}

\subsection{Dyskeratosis Congenita
(DC)}\label{dyskeratosis-congenita-dc}

\subsubsection{Overview}\label{overview-7}

A telomere biology disorder with triad: - Reticular skin pigmentation\\
- Nail dystrophy\\
- Oral leukoplakia

\subsubsection{Clinical Features}\label{clinical-features-60}

\begin{itemize}
\tightlist
\item
  Progressive marrow failure\\
\item
  Pulmonary fibrosis\\
\item
  Liver disease\\
\item
  Increased cancer risk
\end{itemize}

\subsubsection{Diagnosis}\label{diagnosis-29}

\begin{itemize}
\tightlist
\item
  Telomere length assays\\
\item
  Bone marrow: hypocellularity\\
\item
  Genetic testing (TERT, TERC genes)
\end{itemize}

\subsubsection{Management}\label{management-60}

\begin{itemize}
\tightlist
\item
  Supportive care\\
\item
  HSCT for marrow failure\\
\item
  Monitoring for liver and lung complications
\end{itemize}

\subsection{Congenital Amegakaryocytic Thrombocytopenia
(CAMT)}\label{congenital-amegakaryocytic-thrombocytopenia-camt}

\subsubsection{Features}\label{features-3}

\begin{itemize}
\tightlist
\item
  Severe thrombocytopenia early in infancy\\
\item
  Absent megakaryocytes in marrow\\
\item
  Mutations in thrombopoietin receptor (c-MPL gene)\\
\item
  Progresses to pancytopenia
\end{itemize}

\subsubsection{Management}\label{management-61}

\begin{itemize}
\tightlist
\item
  Platelet transfusions\\
\item
  HSCT is curative
\end{itemize}

\subsection{TAR Syndrome (Thrombocytopenia with Absent
Radius)}\label{tar-syndrome-thrombocytopenia-with-absent-radius}

\subsubsection{Features}\label{features-4}

\begin{itemize}
\tightlist
\item
  Normal thumbs (unlike FA)\\
\item
  Severe thrombocytopenia in infancy\\
\item
  Cow's milk intolerance
\end{itemize}

\subsubsection{Management}\label{management-62}

\begin{itemize}
\tightlist
\item
  Supportive care\\
\item
  Platelet transfusions\\
\item
  Nutritional management\\
\item
  Usually improves with age
\end{itemize}

\subsection{Congenital Neutropenias (including Kostmann
syndrome)}\label{congenital-neutropenias-including-kostmann-syndrome}

\subsubsection{Features}\label{features-5}

\begin{itemize}
\tightlist
\item
  Severe neutropenia \textless{} 500/µL\\
\item
  Recurrent bacterial infections\\
\item
  Bone marrow with maturation arrest of granulocytes
\end{itemize}

\subsubsection{Management}\label{management-63}

\begin{itemize}
\tightlist
\item
  G-CSF is highly effective\\
\item
  HSCT if refractory or transformation to MDS/AML occurs
\end{itemize}

\section{Acquired Bone Marrow Failure
Syndromes}\label{acquired-bone-marrow-failure-syndromes-1}

\subsection{Aplastic Anaemia}\label{aplastic-anaemia-1}

\subsubsection{Overview}\label{overview-8}

Aplastic anaemia is characterized by peripheral pancytopenia and
hypocellular bone marrow due to immune-mediated destruction of stem
cells.

\subsubsection{Causes}\label{causes-20}

\begin{itemize}
\tightlist
\item
  Idiopathic (most common)\\
\item
  Drugs: chloramphenicol, anticonvulsants, NSAIDs\\
\item
  Viruses: hepatitis, EBV, HIV\\
\item
  Toxins: benzene\\
\item
  Autoimmune disorders
\end{itemize}

\subsubsection{Clinical Features}\label{clinical-features-61}

\begin{itemize}
\tightlist
\item
  Fatigue (anemia)\\
\item
  Infections (neutropenia)\\
\item
  Bleeding/bruising (thrombocytopenia)\\
\item
  No physical anomalies (helps distinguish from congenital causes)
\end{itemize}

\subsubsection{Diagnosis}\label{diagnosis-30}

\begin{itemize}
\tightlist
\item
  Pancytopenia\\
\item
  Bone marrow biopsy: hypocellularity without fibrosis or malignancy
\end{itemize}

\subsubsection{Management}\label{management-64}

\begin{itemize}
\tightlist
\item
  Supportive care\\
\item
  Immunosuppressive therapy (ATG + cyclosporine)\\
\item
  HSCT (curative)
\end{itemize}

\subsubsection{Ghana Context}\label{ghana-context-1}

Availability of ATG and cyclosporine may be limited; supportive care and
referral are essential.

\subsection{Transient Erythroblastopenia of Childhood
(TEC)}\label{transient-erythroblastopenia-of-childhood-tec}

\subsubsection{Overview}\label{overview-9}

Acquired, temporary suppression of erythropoiesis, often post-viral.

\subsubsection{Features}\label{features-6}

\begin{itemize}
\tightlist
\item
  Occurs in children 6 months--6 years\\
\item
  Normocytic anemia\\
\item
  Reticulocytopenia\\
\item
  Well child apart from pallor
\end{itemize}

\subsubsection{Management}\label{management-65}

\begin{itemize}
\tightlist
\item
  Observation\\
\item
  Transfusion rarely needed\\
\item
  Recovery in 1--2 months
\end{itemize}

\subsection{Parvovirus B19--Associated Aplastic
Crisis}\label{parvovirus-b19associated-aplastic-crisis}

Especially severe in children with chronic hemolytic anaemias (e.g.,
HbC, thalassemia, sickle cell).

\subsubsection{Features}\label{features-7}

\begin{itemize}
\tightlist
\item
  Sudden drop in Hb\\
\item
  Reticulocytopenia\\
\item
  Fever
\end{itemize}

\subsubsection{Management}\label{management-66}

\begin{itemize}
\tightlist
\item
  Transfusion support\\
\item
  Treat underlying infection\\
\item
  Immunoglobulin therapy in immunodeficiency
\end{itemize}

\subsection{Drug-Induced Bone Marrow
Suppression}\label{drug-induced-bone-marrow-suppression}

Common culprits: - Sulphonamides\\
- Chloramphenicol\\
- Anticonvulsants\\
- Cytotoxic chemotherapy

\subsubsection{Management}\label{management-67}

\begin{itemize}
\tightlist
\item
  Stop offending drug\\
\item
  Supportive care\\
\item
  G-CSF or EPO in selected cases
\end{itemize}

\subsection{Nutritional Marrow
Failure}\label{nutritional-marrow-failure}

Severe malnutrition can lead to hypoplastic marrow with cytopenias.

Management focuses on nutritional rehabilitation.

\section{Diagnosis of Bone Marrow Failure
Syndromes}\label{diagnosis-of-bone-marrow-failure-syndromes}

Key investigations include: - CBC with differential\\
- Reticulocyte count\\
- Bone marrow aspiration and biopsy (cellularity, morphology)\\
- Cytogenetics\\
- Genetic testing (if available)\\
- Telomere length testing\\
- Screening for infections (HIV, hepatitis, parvovirus B19)

Where such investigations are unavailable, careful clinical evaluation
and referral are essential.

\section{Management Principles}\label{management-principles-6}

\begin{enumerate}
\def\labelenumi{\arabic{enumi}.}
\tightlist
\item
  \textbf{Identify and treat reversible causes}\\
\item
  \textbf{Provide supportive care}

  \begin{itemize}
  \tightlist
  \item
    Transfusions\\
  \item
    Infection prophylaxis\\
  \item
    Growth and endocrine monitoring\\
  \end{itemize}
\item
  \textbf{Consider HSCT early in congenital syndromes}\\
\item
  \textbf{Provide family counselling}\\
\item
  \textbf{Monitor for long-term complications (e.g., AML, MDS)}\\
\item
  \textbf{Vaccination and infection prevention strategies}
\end{enumerate}

\section{Local Context: Ghana and Sub-Saharan
Africa}\label{local-context-ghana-and-sub-saharan-africa}

\begin{itemize}
\tightlist
\item
  Diagnostic gaps are significant; few centres perform bone marrow
  biopsies routinely.\\
\item
  Congenital syndromes may be under-recognised and misdiagnosed as
  aplastic anaemia or malnutrition.\\
\item
  HSCT is not widely available; most children rely on supportive care.\\
\item
  Parvovirus B19--induced crises are common due to high background
  prevalence.\\
\item
  Drug-induced marrow failure is an important consideration due to
  widespread use of chloramphenicol and sulphonamides in some settings.
\end{itemize}

\section{Key Points for Clinical
Practice}\label{key-points-for-clinical-practice-1}

\begin{itemize}
\tightlist
\item
  Always distinguish congenital from acquired causes through careful
  history and examination.\\
\item
  Congenital anomalies (thumb abnormalities, skin changes) are important
  diagnostic clues.\\
\item
  HSCT is the only curative option for many congenital syndromes.\\
\item
  Avoid unnecessary transfusions unless clinically indicated.\\
\item
  Maintain high suspicion for viral causes in acute marrow suppression.
\end{itemize}

\section{Further Reading}\label{further-reading-9}

\begin{enumerate}
\def\labelenumi{\arabic{enumi}.}
\tightlist
\item
  Alter BP. Inherited bone marrow failure syndromes. \emph{Hematology Am
  Soc Hematol Educ Program}.\\
\item
  Young NS. Aplastic anemia. \emph{N Engl J Med}.\\
\item
  Savage SA, Bertuch AA. The genetics and clinical manifestations of
  telomere biology disorders.\\
\item
  Hoffbrand AV, Higgs DR. \emph{Postgraduate Haematology}.\\
\item
  Ghana Health Service. \emph{Guidelines for Paediatric Haematology and
  Oncology}.
\end{enumerate}

\part{{Gastroenterology}}

\chapter{Nutrition}\label{nutrition}

\begin{center}\rule{0.5\linewidth}{0.5pt}\end{center}

\section{Introduction}\label{introduction-57}

Adequate nutrition is fundamental to the growth, development, and
survival of children. Optimal nutrition during infancy and childhood
supports physical growth, brain development, immune function, and
overall well-being. Malnutrition --- encompassing both undernutrition
and overnutrition --- remains a major public health concern in low- and
middle-income countries, including Ghana.

Child nutrition must be understood within the context of developmental
stages, dietary requirements, feeding practices, and common nutritional
disorders. The health worker's role is to promote appropriate feeding,
prevent malnutrition, and manage nutritional deficiencies promptly.

\section{Objectives}\label{objectives}

At the end of this session, learners should be able to: 1. Describe the
principles of nutrition and growth in children. 2. Outline nutritional
requirements at different ages. 3. Discuss breastfeeding and
complementary feeding practices. 4. Recognize and manage common
nutritional disorders. 5. Explain preventive strategies for
malnutrition.

\section{Overview of Nutrition in
Children}\label{overview-of-nutrition-in-children}

Nutrition refers to the intake and utilization of nutrients to maintain
health, support growth, and enable bodily functions. It involves
\textbf{macronutrients} (carbohydrates, proteins, fats) and
\textbf{micronutrients} (vitamins, minerals, and trace elements).

Children have higher nutrient requirements per kilogram of body weight
than adults due to rapid growth and development, particularly in the
first two years of life.

\section{Components of a Balanced
Diet}\label{components-of-a-balanced-diet}

A balanced diet provides adequate amounts of all essential nutrients.
The key components are:

\begin{longtable}[]{@{}
  >{\raggedright\arraybackslash}p{(\linewidth - 4\tabcolsep) * \real{0.3333}}
  >{\raggedright\arraybackslash}p{(\linewidth - 4\tabcolsep) * \real{0.3333}}
  >{\raggedright\arraybackslash}p{(\linewidth - 4\tabcolsep) * \real{0.3333}}@{}}
\toprule\noalign{}
\begin{minipage}[b]{\linewidth}\raggedright
Nutrient
\end{minipage} & \begin{minipage}[b]{\linewidth}\raggedright
Function
\end{minipage} & \begin{minipage}[b]{\linewidth}\raggedright
Major Sources
\end{minipage} \\
\midrule\noalign{}
\endhead
\bottomrule\noalign{}
\endlastfoot
Carbohydrates & Energy source & Cereals, grains, fruits, tubers \\
Proteins & Growth and tissue repair & Legumes, milk, meat, fish, eggs \\
Fats & Energy storage, cell membranes, absorption of fat-soluble
vitamins & Oils, nuts, dairy, meat \\
Vitamins & Metabolic functions & Fruits, vegetables, fortified foods \\
Minerals & Bone health, enzyme function & Milk, fish, leafy vegetables,
iodized salt \\
Water & Solvent, thermoregulation & Drinking water, fruits, soups \\
\end{longtable}

\section{Nutritional Requirements by
Age}\label{nutritional-requirements-by-age}

\subsection{Neonates and Infants (0--6
months)}\label{neonates-and-infants-06-months}

\begin{itemize}
\tightlist
\item
  \textbf{Exclusive breastfeeding} is recommended for the first 6
  months.
\item
  Breast milk provides all nutrients except vitamin D (and sometimes
  vitamin K).
\item
  Breastfeeding should be on demand, day and night.
\end{itemize}

\subsection{Infants (6--24 months)}\label{infants-624-months}

\begin{itemize}
\tightlist
\item
  \textbf{Complementary feeding} begins at 6 months while continuing
  breastfeeding up to 2 years or beyond.
\item
  Start with pureed or mashed foods, progressing to family foods.
\item
  Energy-dense meals: 2--3 times daily at 6--8 months, 3--4 times after
  9 months, plus snacks.
\end{itemize}

\section{Preschool Children (2--5
years)}\label{preschool-children-25-years}

\begin{itemize}
\tightlist
\item
  Require a variety of foods from all groups.
\item
  Encourage family meals, adequate protein, fruits, and vegetables.
\item
  Avoid excessive sugary snacks and drinks.
\end{itemize}

\subsection{School-Age Children (6--12
years)}\label{school-age-children-612-years}

\begin{itemize}
\tightlist
\item
  Nutrition supports steady growth and school performance.
\item
  Meals should be balanced with sufficient carbohydrates, proteins, and
  iron.
\item
  Promote hygiene and safe water intake.
\end{itemize}

\subsection{Adolescents (10--19 years)}\label{adolescents-1019-years}

\begin{itemize}
\tightlist
\item
  Period of rapid growth and hormonal changes.
\item
  Increased need for calories, protein, calcium, iron, and vitamins.
\item
  Girls require additional iron due to menstruation.
\end{itemize}

\section{Breastfeeding}\label{breastfeeding}

\subsection{Importance of
Breastfeeding}\label{importance-of-breastfeeding}

\begin{itemize}
\tightlist
\item
  Provides ideal nutrition and antibodies.
\item
  Reduces risk of diarrhoea, pneumonia, and sudden infant death
  syndrome.
\item
  Promotes bonding and optimal neurodevelopment.
\end{itemize}

\subsection{Composition of Breast
Milk}\label{composition-of-breast-milk}

\begin{longtable}[]{@{}
  >{\raggedright\arraybackslash}p{(\linewidth - 2\tabcolsep) * \real{0.3671}}
  >{\raggedright\arraybackslash}p{(\linewidth - 2\tabcolsep) * \real{0.6329}}@{}}
\toprule\noalign{}
\begin{minipage}[b]{\linewidth}\raggedright
Component
\end{minipage} & \begin{minipage}[b]{\linewidth}\raggedright
Function
\end{minipage} \\
\midrule\noalign{}
\endhead
\bottomrule\noalign{}
\endlastfoot
Lactose & Main carbohydrate; energy and promotes gut flora \\
Whey and casein & Proteins aiding digestion and immunity \\
Lipids & Energy, essential fatty acids, brain development \\
Vitamins and minerals & Nutritional balance \\
Antibodies and immune cells & Protection against infections \\
\end{longtable}

\subsection{Stages of Breast Milk}\label{stages-of-breast-milk}

\begin{itemize}
\tightlist
\item
  \textbf{Colostrum:} Yellow, thick milk secreted in first 3--5 days;
  rich in antibodies.
\item
  \textbf{Transitional milk:} 5--14 days postpartum.
\item
  \textbf{Mature milk:} From 2 weeks onwards.
\end{itemize}

\subsection{Common Breastfeeding
Challenges}\label{common-breastfeeding-challenges}

\begin{itemize}
\tightlist
\item
  Poor latch or positioning.
\item
  Cracked nipples, mastitis.
\item
  Perceived low milk supply.
\item
  Management: counselling, proper technique, frequent feeding,
  expressing milk if needed.
\end{itemize}

\section{Complementary Feeding}\label{complementary-feeding}

Complementary feeding is the process of introducing solid and semi-solid
foods alongside breast milk from 6 months of age.

\subsubsection{Principles}\label{principles-1}

\begin{enumerate}
\def\labelenumi{\arabic{enumi}.}
\tightlist
\item
  Timely introduction at 6 months.
\item
  Adequate --- providing sufficient energy, protein, and micronutrients.
\item
  Safe --- hygienic preparation and storage.
\item
  Appropriate --- culturally acceptable, age-appropriate texture.
\end{enumerate}

\subsubsection{Common Complementary Foods in
Ghana}\label{common-complementary-foods-in-ghana}

\begin{itemize}
\tightlist
\item
  Mashed yam, rice, or maize porridge with beans, fish, or groundnuts.
\item
  Fortified cereals, fruits, and vegetables.
\end{itemize}

\subsubsection{Feeding Frequency}\label{feeding-frequency}

\begin{longtable}[]{@{}llll@{}}
\toprule\noalign{}
Age (months) & Breastfeeds & Meals/day & Snacks \\
\midrule\noalign{}
\endhead
\bottomrule\noalign{}
\endlastfoot
6--8 & On demand & 2--3 & 1--2 \\
9--11 & On demand & 3--4 & 1--2 \\
12--24 & On demand & 3--4 & 1--2 \\
\end{longtable}

\section{Nutritional Assessment}\label{nutritional-assessment}

A comprehensive nutritional assessment includes \textbf{anthropometry,
dietary history, and clinical examination.}

\subsection{Anthropometric Measures}\label{anthropometric-measures}

\begin{longtable}[]{@{}
  >{\raggedright\arraybackslash}p{(\linewidth - 4\tabcolsep) * \real{0.2949}}
  >{\raggedright\arraybackslash}p{(\linewidth - 4\tabcolsep) * \real{0.3333}}
  >{\raggedright\arraybackslash}p{(\linewidth - 4\tabcolsep) * \real{0.3718}}@{}}
\toprule\noalign{}
\begin{minipage}[b]{\linewidth}\raggedright
Measure
\end{minipage} & \begin{minipage}[b]{\linewidth}\raggedright
Purpose
\end{minipage} & \begin{minipage}[b]{\linewidth}\raggedright
Indicators of Malnutrition
\end{minipage} \\
\midrule\noalign{}
\endhead
\bottomrule\noalign{}
\endlastfoot
Weight-for-age & Growth monitoring & Underweight \\
Height/length-for-age & Chronic malnutrition & Stunting \\
Weight-for-height & Acute malnutrition & Wasting \\
MUAC & Screening in 6--59 months & \textless12.5 cm indicates wasting \\
\end{longtable}

\subsection{Clinical Assessment}\label{clinical-assessment-1}

\begin{itemize}
\tightlist
\item
  Look for signs of deficiency: pallor, oedema, hair changes, skin
  lesions, angular stomatitis, goitre, rickets.
\end{itemize}

\subsection{Dietary Assessment}\label{dietary-assessment}

\begin{itemize}
\tightlist
\item
  24-hour recall, food frequency questionnaire.
\item
  Identify inadequate intake or inappropriate feeding practices.
\end{itemize}

\section{Common Nutritional
Disorders}\label{common-nutritional-disorders}

\subsection{1. Protein--Energy Malnutrition
(PEM)}\label{proteinenergy-malnutrition-pem}

\subsubsection{Types}\label{types-4}

\begin{itemize}
\tightlist
\item
  \textbf{Marasmus:} severe wasting, due to calorie deficiency.
\item
  \textbf{Kwashiorkor:} protein deficiency with oedema.
\end{itemize}

\subsubsection{Clinical Features}\label{clinical-features-62}

\begin{longtable}[]{@{}ll@{}}
\toprule\noalign{}
Marasmus & Kwashiorkor \\
\midrule\noalign{}
\endhead
\bottomrule\noalign{}
\endlastfoot
Severe wasting & Oedema (especially feet, face) \\
Alert but irritable & Apathetic, moon face \\
No oedema & Dermatosis, sparse hair \\
\end{longtable}

\subsubsection{Management}\label{management-68}

\begin{enumerate}
\def\labelenumi{\arabic{enumi}.}
\tightlist
\item
  \textbf{Stabilization phase (1--2 days):} Treat hypoglycaemia,
  hypothermia, dehydration, infections.
\item
  \textbf{Transition phase (2--7 days):} Gradually increase feeding.
\item
  \textbf{Rehabilitation phase:} Catch-up growth using F-100 or locally
  fortified diets.
\item
  Provide micronutrient supplements (vitamin A, folate, zinc).
\end{enumerate}

\subsection{2. Micronutrient
Deficiencies}\label{micronutrient-deficiencies}

\subsubsection{a) Iron Deficiency
Anaemia}\label{a-iron-deficiency-anaemia}

\begin{itemize}
\tightlist
\item
  Most common deficiency globally.
\item
  Causes: inadequate intake, blood loss, infections (hookworm, malaria).
\item
  Features: pallor, lethargy, poor cognition.
\item
  Management: iron supplementation, dietary education.
\end{itemize}

\subsubsection{b) Vitamin A Deficiency}\label{b-vitamin-a-deficiency}

\begin{itemize}
\tightlist
\item
  Causes xerophthalmia, night blindness, increased infection risk.
\item
  Prevention: vitamin A supplementation every 6 months (per national
  guidelines), foods like mango, palm oil, liver.
\end{itemize}

\subsubsection{c) Iodine Deficiency}\label{c-iodine-deficiency}

\begin{itemize}
\tightlist
\item
  Causes goitre, cretinism, developmental delay.
\item
  Prevention: universal salt iodization.
\end{itemize}

\subsubsection{d) Zinc Deficiency}\label{d-zinc-deficiency}

\begin{itemize}
\tightlist
\item
  Causes growth retardation, diarrhoea, impaired immunity.
\item
  Prevention: dietary zinc and supplementation in diarrhoeal disease.
\end{itemize}

\subsubsection{e) Vitamin D Deficiency}\label{e-vitamin-d-deficiency}

\begin{itemize}
\tightlist
\item
  Causes rickets in children.
\item
  Prevention: sunlight exposure, fortified foods.
\end{itemize}

\section{3. Overnutrition and Childhood
Obesity}\label{overnutrition-and-childhood-obesity}

\begin{itemize}
\tightlist
\item
  Increasingly common with urbanization and sedentary lifestyles.
\item
  Associated with diabetes, hypertension, dyslipidaemia.
\item
  Prevention: promote physical activity, reduce sugary and processed
  foods.
\end{itemize}

\section{Prevention of Malnutrition}\label{prevention-of-malnutrition}

\begin{enumerate}
\def\labelenumi{\arabic{enumi}.}
\tightlist
\item
  \textbf{Promote exclusive breastfeeding and appropriate complementary
  feeding.}
\item
  \textbf{Routine growth monitoring and promotion (GMP).}
\item
  \textbf{Micronutrient supplementation and fortification programs.}
\item
  \textbf{Nutrition education for caregivers.}
\item
  \textbf{Control of infections (immunization, deworming, malaria
  prevention).}
\item
  \textbf{Food security and poverty reduction initiatives.}
\end{enumerate}

\section{Nutrition in Special
Conditions}\label{nutrition-in-special-conditions}

\subsection{Preterm Infants}\label{preterm-infants}

\begin{itemize}
\tightlist
\item
  Require higher protein and energy intake.
\item
  Use fortified breast milk or preterm formulas.
\end{itemize}

\subsection{Children with Chronic
Illness}\label{children-with-chronic-illness}

\begin{itemize}
\tightlist
\item
  Monitor growth closely (e.g., in congenital heart disease, HIV, cystic
  fibrosis).
\item
  Supplement calories and micronutrients.
\end{itemize}

\subsection{HIV-Infected Children}\label{hiv-infected-children}

\begin{itemize}
\tightlist
\item
  Require increased caloric intake (10--30\% higher).
\item
  Nutritional counselling and food support are vital.
\end{itemize}

\section{Nutritional Counselling and Health
Promotion}\label{nutritional-counselling-and-health-promotion}

\begin{itemize}
\tightlist
\item
  Educate caregivers on balanced diets using local foods.
\item
  Encourage family meals and regular feeding schedules.
\item
  Monitor and discuss growth charts during child welfare visits.
\item
  Address myths and misconceptions around feeding (e.g., withholding
  protein in diarrhoea).
\end{itemize}

\section{Key Takeaways}\label{key-takeaways-5}

\begin{itemize}
\tightlist
\item
  Nutrition is critical for child survival, growth, and development.
\item
  Exclusive breastfeeding for 6 months, followed by appropriate
  complementary feeding, is essential.
\item
  Malnutrition may present as undernutrition or overnutrition.
\item
  Prevention strategies include education, supplementation, and
  infection control.
\item
  Health professionals must advocate for and monitor good nutrition in
  all children.
\end{itemize}

\section{Suggested References}\label{suggested-references-1}

\begin{enumerate}
\def\labelenumi{\arabic{enumi}.}
\tightlist
\item
  World Health Organization. \emph{Infant and Young Child Feeding: Model
  Chapter for Textbooks for Medical Students and Allied Health
  Professionals}. WHO, 2009.
\item
  Ghana Health Service. \emph{Infant and Young Child Feeding Policy and
  Strategy}, 2021.
\item
  UNICEF. \emph{The State of the World's Children: Nutrition Edition},
  2023.
\item
  World Health Organization. \emph{Pocket Book of Hospital Care for
  Children}, 3rd Edition, 2023.
\item
  Black RE et al.~Maternal and child undernutrition and overweight in
  low-income and middle-income countries. \emph{Lancet}.
  2021;397(10283):138--154.
\end{enumerate}

\chapter{Malnutrition}\label{malnutrition-1}

\section{Introduction}\label{introduction-58}

Malnutrition remains one of the most significant health challenges
affecting children globally, especially in low- and middle-income
countries such as Ghana. It contributes substantially to morbidity and
mortality, being an underlying factor in nearly half of all under-five
deaths. Malnutrition includes both \textbf{undernutrition} (deficiencies
in energy, protein, or micronutrients) and \textbf{overnutrition}
(overweight and obesity).

Childhood malnutrition results from an interplay between inadequate
dietary intake, infections, and social determinants such as poverty and
poor sanitation. It affects physical growth, cognitive development, and
immune competence, leading to long-term consequences in adulthood.

\section{Definition}\label{definition-36}

\textbf{Malnutrition} is a pathological state resulting from an
imbalance between nutrient intake and the body's requirements. It may
result in:

\begin{enumerate}
\def\labelenumi{\arabic{enumi}.}
\tightlist
\item
  \textbf{Undernutrition} --- deficiency of energy, protein, or
  micronutrients.
\item
  \textbf{Overnutrition} --- excess intake of energy and nutrients.
\end{enumerate}

In children, \textbf{protein--energy malnutrition (PEM)} is the most
severe form of undernutrition.

\section{Classification}\label{classification-11}

\subsection{1. Based on Type}\label{based-on-type}

\begin{itemize}
\tightlist
\item
  \textbf{Undernutrition:} Wasting, stunting, underweight.
\item
  \textbf{Overnutrition:} Overweight, obesity.
\item
  \textbf{Micronutrient deficiencies:} e.g., vitamin A, iron, iodine,
  zinc.
\end{itemize}

\subsection{2. Based on Clinical Form
(PEM)}\label{based-on-clinical-form-pem}

\begin{longtable}[]{@{}
  >{\raggedright\arraybackslash}p{(\linewidth - 4\tabcolsep) * \real{0.2476}}
  >{\raggedright\arraybackslash}p{(\linewidth - 4\tabcolsep) * \real{0.4381}}
  >{\raggedright\arraybackslash}p{(\linewidth - 4\tabcolsep) * \real{0.3143}}@{}}
\toprule\noalign{}
\begin{minipage}[b]{\linewidth}\raggedright
Type
\end{minipage} & \begin{minipage}[b]{\linewidth}\raggedright
Description
\end{minipage} & \begin{minipage}[b]{\linewidth}\raggedright
Key Features
\end{minipage} \\
\midrule\noalign{}
\endhead
\bottomrule\noalign{}
\endlastfoot
\textbf{Marasmus} & Deficiency of energy (calories) & Severe wasting, no
oedema \\
\textbf{Kwashiorkor} & Deficiency of protein with adequate calories &
Oedema, dermatosis, fatty liver \\
\textbf{Marasmic--Kwashiorkor} & Combined energy and protein deficiency
& Wasting plus oedema \\
\end{longtable}

\subsection{3. Based on Anthropometric Indices
(WHO)}\label{based-on-anthropometric-indices-who}

\begin{longtable}[]{@{}
  >{\raggedright\arraybackslash}p{(\linewidth - 4\tabcolsep) * \real{0.3049}}
  >{\raggedright\arraybackslash}p{(\linewidth - 4\tabcolsep) * \real{0.2927}}
  >{\raggedright\arraybackslash}p{(\linewidth - 4\tabcolsep) * \real{0.4024}}@{}}
\toprule\noalign{}
\begin{minipage}[b]{\linewidth}\raggedright
Indicator
\end{minipage} & \begin{minipage}[b]{\linewidth}\raggedright
Definition
\end{minipage} & \begin{minipage}[b]{\linewidth}\raggedright
Category
\end{minipage} \\
\midrule\noalign{}
\endhead
\bottomrule\noalign{}
\endlastfoot
Weight-for-height (WFH) & \textless{} -2 SD of WHO median & Wasting
(acute malnutrition) \\
Height-for-age (HFA) & \textless{} -2 SD & Stunting (chronic
malnutrition) \\
Weight-for-age (WFA) & \textless{} -2 SD & Underweight \\
BMI-for-age & \textgreater{} +2 SD & Overweight \\
BMI-for-age & \textgreater+3 SD & Obesity \\
\end{longtable}

\section{Epidemiology}\label{epidemiology-13}

\begin{itemize}
\tightlist
\item
  \textbf{Global burden:} About 149 million children under 5 are
  stunted, 45 million wasted, and 37 million overweight (UNICEF 2023).
\item
  \textbf{Sub-Saharan Africa:} Persistent high rates of undernutrition;
  emerging overweight trends in urban areas.
\item
  \textbf{Ghana:} Stunting \textasciitilde17\%, wasting
  \textasciitilde6\%, overweight \textasciitilde3\% (GDHS 2022).
\end{itemize}

Malnutrition often clusters with poverty, low maternal education, poor
sanitation, and infectious diseases such as diarrhoea and malaria.

\section{Aetiology}\label{aetiology-24}

The causes of malnutrition are multifactorial and interrelated. The
UNICEF conceptual framework divides them into \textbf{immediate},
\textbf{underlying}, and \textbf{basic} causes.

\subsection{1. Immediate Causes}\label{immediate-causes}

\begin{itemize}
\tightlist
\item
  \textbf{Inadequate dietary intake} --- insufficient energy, protein,
  or micronutrients.
\item
  \textbf{Disease burden} --- infections increase nutrient losses and
  metabolic demands.
\end{itemize}

\subsection{2. Underlying Causes}\label{underlying-causes}

\begin{itemize}
\tightlist
\item
  \textbf{Household food insecurity} --- lack of access to adequate
  food.
\item
  \textbf{Poor care and feeding practices} --- inappropriate
  breastfeeding or complementary feeding.
\item
  \textbf{Unhealthy environment and inadequate health services} ---
  infections and lack of preventive care.
\end{itemize}

\subsection{3. Basic Causes}\label{basic-causes}

\begin{itemize}
\tightlist
\item
  Poverty, low education, unemployment.
\item
  Political instability, poor governance, and social inequality.
\end{itemize}

\section{Pathophysiology}\label{pathophysiology-40}

Malnutrition disrupts multiple physiological systems:

\begin{enumerate}
\def\labelenumi{\arabic{enumi}.}
\tightlist
\item
  \textbf{Energy deficiency} → muscle wasting, loss of subcutaneous fat.
\item
  \textbf{Protein deficiency} → impaired synthesis of enzymes, albumin,
  and immune factors.
\item
  \textbf{Micronutrient deficiency} → anaemia, impaired growth,
  increased susceptibility to infection.
\item
  \textbf{Metabolic adaptation:} Reduced basal metabolic rate and
  altered hormonal responses (insulin, cortisol).
\item
  \textbf{Immune dysfunction:} Atrophy of lymphoid tissue leading to
  immunosuppression.
\end{enumerate}

\subsubsection{In Kwashiorkor:}\label{in-kwashiorkor}

\begin{itemize}
\tightlist
\item
  Low plasma albumin → oedema.
\item
  Fatty liver due to defective lipoprotein synthesis.
\item
  Oxidative stress and free radical damage contribute to cellular
  injury.
\end{itemize}

\section{Clinical Features}\label{clinical-features-63}

\subsection{General Presentation}\label{general-presentation-1}

\begin{itemize}
\tightlist
\item
  Poor growth and weight loss.
\item
  Wasting or oedema.
\item
  Recurrent infections (respiratory, diarrhoeal).
\item
  Lethargy, irritability, apathy.
\item
  Developmental delay.
\end{itemize}

\subsection{A. Marasmus}\label{a.-marasmus}

\begin{itemize}
\tightlist
\item
  Severe wasting, ``skin and bones'' appearance.
\item
  Loss of subcutaneous fat and muscle.
\item
  Alert but irritable.
\item
  No oedema.
\item
  Wrinkled skin, sunken eyes, sparse dry hair.
\end{itemize}

\subsection{B. Kwashiorkor}\label{b.-kwashiorkor}

\begin{itemize}
\tightlist
\item
  Generalized or dependent oedema (starting from feet).
\item
  Moon face, flaky paint dermatosis.
\item
  Sparse, easily pluckable hair with flag sign.
\item
  Enlarged fatty liver.
\item
  Apathy, poor appetite.
\item
  Anaemia and susceptibility to infection.
\end{itemize}

\subsection{C. Marasmic--Kwashiorkor}\label{c.-marasmickwashiorkor}

\begin{itemize}
\tightlist
\item
  Features of both wasting and oedema.
\end{itemize}

\section{Complications of Severe Acute Malnutrition
(SAM)}\label{complications-of-severe-acute-malnutrition-sam}

\begin{enumerate}
\def\labelenumi{\arabic{enumi}.}
\tightlist
\item
  \textbf{Hypoglycaemia}
\item
  \textbf{Hypothermia}
\item
  \textbf{Dehydration} (often masked by oedema)
\item
  \textbf{Electrolyte imbalance} (low potassium, magnesium)
\item
  \textbf{Infections} (often with subtle signs)
\item
  \textbf{Anaemia}
\item
  \textbf{Heart failure}
\end{enumerate}

\section{Assessment and Diagnosis}\label{assessment-and-diagnosis}

\subsection{1. Anthropometric
Measurements}\label{anthropometric-measurements-1}

\begin{longtable}[]{@{}
  >{\raggedright\arraybackslash}p{(\linewidth - 4\tabcolsep) * \real{0.2222}}
  >{\raggedright\arraybackslash}p{(\linewidth - 4\tabcolsep) * \real{0.2778}}
  >{\raggedright\arraybackslash}p{(\linewidth - 4\tabcolsep) * \real{0.5000}}@{}}
\toprule\noalign{}
\begin{minipage}[b]{\linewidth}\raggedright
Indicator
\end{minipage} & \begin{minipage}[b]{\linewidth}\raggedright
Tool
\end{minipage} & \begin{minipage}[b]{\linewidth}\raggedright
Interpretation
\end{minipage} \\
\midrule\noalign{}
\endhead
\bottomrule\noalign{}
\endlastfoot
Weight-for-height & Salter scale, WHO chart & \textless{} -3 SD = severe
wasting \\
MUAC (6--59 months) & MUAC tape & \textless{} 11.5 cm = SAM \\
Oedema & Clinical exam & Presence indicates SAM regardless of weight \\
\end{longtable}

\subsection{2. Laboratory
Investigations}\label{laboratory-investigations-4}

\begin{itemize}
\tightlist
\item
  \textbf{Blood glucose:} detect hypoglycaemia.
\item
  \textbf{Haemoglobin:} assess for anaemia.
\item
  \textbf{Electrolytes:} Na⁺, K⁺, Mg²⁺.
\item
  \textbf{Urinalysis:} infection or proteinuria.
\item
  \textbf{Malaria test:} especially in endemic areas.
\item
  \textbf{Stool examination:} parasites.
\item
  \textbf{HIV testing:} as indicated.
\end{itemize}

\section{Management of Severe Acute Malnutrition
(SAM)}\label{management-of-severe-acute-malnutrition-sam}

Management is guided by WHO protocols and Ghana Health Service
guidelines.

\subsection{Objectives}\label{objectives-1}

\begin{itemize}
\tightlist
\item
  Treat or prevent complications.
\item
  Correct nutritional deficiencies.
\item
  Achieve catch-up growth.
\item
  Promote caregiver education.
\end{itemize}

\subsection{Phases of Management}\label{phases-of-management}

\subsubsection{\texorpdfstring{1. \textbf{Stabilization Phase (Days
1--2)}}{1. Stabilization Phase (Days 1--2)}}\label{stabilization-phase-days-12}

\begin{itemize}
\tightlist
\item
  Manage life-threatening problems.
\item
  Avoid excess protein and sodium.
\item
  Feed with \textbf{F-75} (75 kcal/100 mL, 0.9 g protein/100 mL) every
  2--3 hours.
\item
  Prevent hypoglycaemia and hypothermia.
\end{itemize}

\paragraph{a) Hypoglycaemia}\label{a-hypoglycaemia}

\begin{itemize}
\tightlist
\item
  Feed immediately.
\item
  If unable to feed: 10\% dextrose 5 mL/kg IV or via NG tube.
\end{itemize}

\paragraph{b) Hypothermia}\label{b-hypothermia}

\begin{itemize}
\tightlist
\item
  Keep child warm (skin-to-skin, warm room, clothing).
\end{itemize}

\paragraph{c) Dehydration}\label{c-dehydration}

\begin{itemize}
\tightlist
\item
  Use \textbf{ReSoMal} (Rehydration Solution for Malnutrition) orally or
  via NG.
\item
  Avoid standard ORS due to high sodium.
\end{itemize}

\paragraph{d) Infections}\label{d-infections}

\begin{itemize}
\tightlist
\item
  Empiric antibiotics (ampicillin + gentamicin, or ceftriaxone if
  indicated).
\end{itemize}

\paragraph{e) Micronutrient
supplementation}\label{e-micronutrient-supplementation}

\begin{itemize}
\tightlist
\item
  Vitamin A (not if given within last month), multivitamins, folic acid,
  zinc, copper, magnesium.
\end{itemize}

\subsubsection{\texorpdfstring{2. \textbf{Transition Phase (Days
2--7)}}{2. Transition Phase (Days 2--7)}}\label{transition-phase-days-27}

\begin{itemize}
\tightlist
\item
  Introduce higher-calorie feeds: \textbf{F-100} (100 kcal/100 mL, 2.9 g
  protein/100 mL).
\item
  Gradually increase intake to meet energy needs.
\end{itemize}

\subsubsection{\texorpdfstring{3. \textbf{Rehabilitation Phase (Day 7
onwards)}}{3. Rehabilitation Phase (Day 7 onwards)}}\label{rehabilitation-phase-day-7-onwards}

\begin{itemize}
\tightlist
\item
  Catch-up growth with F-100 or locally fortified feeds.
\item
  Encourage play and stimulation.
\item
  Treat underlying causes (e.g., poor feeding, infection).
\end{itemize}

\subsection{Discharge Criteria}\label{discharge-criteria}

\begin{itemize}
\tightlist
\item
  Weight-for-height \textgreater{} -2 SD for 2 consecutive weeks.
\item
  No oedema for at least 2 weeks.
\item
  Good appetite and clinical recovery.
\end{itemize}

\section{Management of Moderate Acute Malnutrition
(MAM)}\label{management-of-moderate-acute-malnutrition-mam}

\begin{itemize}
\tightlist
\item
  Managed as outpatient.
\item
  Supplementary feeding with ready-to-use foods (RUFs).
\item
  Nutrition counselling for caregivers.
\item
  Monitor growth weekly.
\end{itemize}

\section{Management of Chronic Malnutrition
(Stunting)}\label{management-of-chronic-malnutrition-stunting}

\begin{itemize}
\tightlist
\item
  Address long-term inadequate nutrition.
\item
  Promote maternal nutrition and antenatal care.
\item
  Early initiation of breastfeeding.
\item
  Timely complementary feeding and infection control.
\end{itemize}

\section{Management of Micronutrient
Deficiencies}\label{management-of-micronutrient-deficiencies}

\subsection{1. Iron Deficiency Anaemia}\label{iron-deficiency-anaemia}

\begin{itemize}
\tightlist
\item
  Oral iron (3--6 mg/kg/day elemental iron) for 3 months.
\item
  Treat underlying infection (malaria, hookworm).
\end{itemize}

\subsection{2. Vitamin A Deficiency}\label{vitamin-a-deficiency}

\begin{itemize}
\tightlist
\item
  200,000 IU single dose (\textgreater12 months), 100,000 IU (6--12
  months), 50,000 IU (\textless6 months).
\item
  Encourage intake of carotene-rich foods (mango, palm oil, liver).
\end{itemize}

\subsection{3. Iodine Deficiency}\label{iodine-deficiency}

\begin{itemize}
\tightlist
\item
  Universal salt iodization.
\end{itemize}

\subsection{4. Zinc Deficiency}\label{zinc-deficiency}

\begin{itemize}
\tightlist
\item
  Zinc supplementation (10--20 mg/day) during diarrhoeal illness.
\end{itemize}

\section{Prevention of Malnutrition}\label{prevention-of-malnutrition-1}

\begin{enumerate}
\def\labelenumi{\arabic{enumi}.}
\tightlist
\item
  \textbf{Exclusive breastfeeding} for the first 6 months.
\item
  \textbf{Timely and adequate complementary feeding} from 6 months.
\item
  \textbf{Micronutrient supplementation and fortification} (iron,
  vitamin A, zinc, iodine).
\item
  \textbf{Routine immunization} to reduce infections.
\item
  \textbf{Nutrition education} for caregivers.
\item
  \textbf{Deworming} every 6 months after age 1.
\item
  \textbf{Food security interventions} at household and community level.
\end{enumerate}

\section{Nutrition Rehabilitation Centres
(NRCs)}\label{nutrition-rehabilitation-centres-nrcs}

These provide facility-based care for children with SAM who cannot be
managed as outpatients. Activities include: - Stabilization and
rehabilitation phases. - Nutrition education for caregivers. -
Psychosocial stimulation. - Growth monitoring and follow-up.

\section{Community Management of Acute Malnutrition
(CMAM)}\label{community-management-of-acute-malnutrition-cmam}

The CMAM strategy integrates facility and community-level interventions:
- \textbf{Screening:} by MUAC or oedema. - \textbf{Outpatient
therapeutic care:} for uncomplicated SAM using RUTF (e.g., Plumpy'Nut).
- \textbf{Inpatient care:} for complicated SAM. - \textbf{Community
outreach:} to promote early detection and referral.

\section{Overnutrition and Childhood
Obesity}\label{overnutrition-and-childhood-obesity-1}

\subsection{Definition}\label{definition-37}

Excessive fat accumulation due to imbalance between energy intake and
expenditure.

\subsection{Risk Factors}\label{risk-factors-2}

\begin{itemize}
\tightlist
\item
  High-calorie, low-nutrient diets.
\item
  Sedentary lifestyle.
\item
  Urbanization.
\item
  Genetic predisposition.
\end{itemize}

\subsection{Complications}\label{complications-31}

\begin{itemize}
\tightlist
\item
  Type 2 diabetes mellitus.
\item
  Hypertension, dyslipidaemia.
\item
  Non-alcoholic fatty liver disease.
\item
  Orthopaedic and psychosocial problems.
\end{itemize}

\subsection{Management}\label{management-69}

\begin{itemize}
\tightlist
\item
  Lifestyle modification: healthy diet, regular physical activity.
\item
  Family-based behavioural interventions.
\item
  Avoid restrictive diets in growing children.
\end{itemize}

\section{Prognosis}\label{prognosis-35}

\begin{itemize}
\tightlist
\item
  \textbf{Mild to moderate malnutrition:} reversible with appropriate
  intervention.
\item
  \textbf{Severe malnutrition:} carries a mortality rate of 10--30\% if
  untreated.
\item
  Early detection and management significantly improve outcomes.
\item
  Stunting has long-term impacts on cognition, school performance, and
  adult productivity.
\end{itemize}

\section{Public Health and Policy
Considerations}\label{public-health-and-policy-considerations}

\begin{itemize}
\tightlist
\item
  \textbf{Integrated management of childhood illness (IMCI)} includes
  nutrition counselling.
\item
  \textbf{Ghana Health Service programs:} CMAM, growth monitoring,
  vitamin A supplementation, and infant feeding promotion.
\item
  \textbf{Sustainable Development Goals (SDG 2):} End all forms of
  malnutrition by 2030.
\end{itemize}

\section{Key Takeaways}\label{key-takeaways-6}

\begin{itemize}
\tightlist
\item
  Malnutrition encompasses both undernutrition and overnutrition.
\item
  Protein--energy malnutrition (marasmus and kwashiorkor) is
  preventable.
\item
  Early recognition, stabilization, and rehabilitation are critical in
  SAM management.
\item
  Micronutrient supplementation and infection control are essential.
\item
  Prevention through improved feeding practices and socioeconomic
  development remains the cornerstone.
\end{itemize}

\section{Suggested References}\label{suggested-references-2}

\begin{enumerate}
\def\labelenumi{\arabic{enumi}.}
\tightlist
\item
  World Health Organization. \emph{Pocket Book of Hospital Care for
  Children}, 3rd Edition, 2023.
\item
  Ghana Health Service. \emph{Integrated Management of Acute
  Malnutrition Guidelines}, 2021.
\item
  UNICEF/WHO/World Bank. \emph{Levels and Trends in Child Malnutrition},
  2023.
\item
  Black RE, Victora CG, Walker SP et al.~Maternal and child
  undernutrition and overweight in low-income and middle-income
  countries. \emph{Lancet}. 2021;397(10283):138--154.
\item
  Ministry of Health, Ghana. \emph{Standard Treatment Guidelines}, 2022.
\end{enumerate}

\chapter{Liver Disorders}\label{liver-disorders}

\section{Introduction}\label{introduction-59}

Liver diseases in children represent a significant component of
paediatric morbidity and mortality worldwide, including in Ghana. The
liver plays a central role in metabolism, detoxification, bile
production, and immunity. Pediatric liver diseases range from transient
biochemical abnormalities to severe chronic liver failure. Understanding
the pathophysiology, clinical presentation, diagnosis, and management is
essential for early intervention and improved outcomes.

\section{Anatomy and Physiology of the Pediatric
Liver}\label{anatomy-and-physiology-of-the-pediatric-liver}

The liver is the largest solid organ in the body, comprising about 5\%
of the body weight in neonates. It performs vital functions, including:

\begin{itemize}
\tightlist
\item
  Carbohydrate, protein, and lipid metabolism
\item
  Synthesis of coagulation factors and plasma proteins
\item
  Bile production and excretion
\item
  Detoxification of drugs and toxins
\item
  Immune surveillance through Kupffer cells
\end{itemize}

Due to its wide range of functions, liver dysfunction can present in
multiple ways, ranging from jaundice to coagulopathy, growth failure, or
encephalopathy.

\section{Classification of Pediatric Liver
Diseases}\label{classification-of-pediatric-liver-diseases}

Pediatric liver diseases can be broadly classified into:

\begin{enumerate}
\def\labelenumi{\arabic{enumi}.}
\tightlist
\item
  Congenital/Genetic Disorders
\item
  Infectious Causes
\item
  Autoimmune Liver Diseases
\item
  Metabolic Liver Diseases
\item
  Toxic and Drug-Induced Hepatopathies
\item
  Cholestatic Liver Diseases
\item
  Liver Tumors
\end{enumerate}

Each category encompasses specific diseases with unique features,
although overlaps may exist.

Neonatal Cholestasis

\textbf{Definition}: Neonatal cholestasis is defined as prolonged
conjugated hyperbilirubinemia (\textgreater1 mg/dL if total bilirubin is
\textless5 mg/dL or \textgreater20\% of total bilirubin if \textgreater5
mg/dL) lasting beyond 14 days of life in term infants.

\textbf{Common Causes:}

\begin{itemize}
\tightlist
\item
  Biliary atresia
\item
  Neonatal hepatitis
\item
  Metabolic diseases (e.g., galactosemia, alpha-1 antitrypsin
  deficiency)
\item
  Infections (e.g., TORCH infections)
\end{itemize}

\subsection{Biliary Atresia}\label{biliary-atresia}

\begin{itemize}
\tightlist
\item
  \textbf{Pathology}: Progressive fibro-obliterative disease of the bile
  ducts.
\item
  \textbf{Presentation}: Jaundice, pale stools, dark urine,
  hepatomegaly.
\item
  \textbf{Diagnosis}: Ultrasound, hepatobiliary iminodiacetic acid
  (HIDA) scan, liver biopsy.
\item
  \textbf{Treatment}: Kasai portoenterostomy; liver transplant in
  advanced cases.
\end{itemize}

\subsection{Neonatal Hepatitis}\label{neonatal-hepatitis}

\begin{itemize}
\tightlist
\item
  May be idiopathic or secondary to infections.
\item
  Histology shows giant cell transformation.
\item
  Management includes supportive care and treating underlying causes.
\end{itemize}

Acute Hepatitis

Acute hepatitis in children can be viral, drug-induced, or autoimmune.

\textbf{Viral Hepatitis}

\begin{itemize}
\tightlist
\item
  \textbf{Hepatitis A virus (HAV)}: Feco-oral transmission,
  self-limiting.
\item
  \textbf{Hepatitis B virus (HBV)}: Perinatal transmission common in
  Ghana; can lead to chronic infection.
\item
  \textbf{Hepatitis C virus (HCV)}: Less common; vertical transmission
  possible.
\item
  \textbf{Other viruses}: Epstein-Barr virus, cytomegalovirus.
\end{itemize}

\textbf{Clinical Features}

\begin{itemize}
\tightlist
\item
  Jaundice
\item
  Anorexia
\item
  Vomiting
\item
  Fever
\item
  Tender hepatomegaly
\end{itemize}

\textbf{Diagnosis}

\begin{itemize}
\tightlist
\item
  Liver function tests: Elevated AST, ALT, bilirubin
\item
  Serological markers: HBsAg, anti-HAV IgM, anti-HCV
\end{itemize}

\textbf{Management}

\begin{itemize}
\tightlist
\item
  Supportive care in most cases
\item
  Antivirals in selected chronic HBV/HCV cases
\item
  Immunization for prevention (e.g., hepatitis B vaccine)
\end{itemize}

Chronic Hepatitis

Persistent liver inflammation \textgreater6 months is classified as
chronic hepatitis.

\textbf{Causes}

\begin{itemize}
\tightlist
\item
  Chronic HBV/HCV
\item
  Autoimmune hepatitis
\item
  Metabolic disorders (e.g., Wilson's disease)
\end{itemize}

\subsection{Autoimmune Hepatitis (AIH)}\label{autoimmune-hepatitis-aih}

\begin{itemize}
\tightlist
\item
  More common in adolescents
\item
  Autoantibodies: ANA, SMA, LKM-1
\item
  Associated with other autoimmune diseases
\item
  Requires immunosuppressive therapy (e.g., prednisolone and
  azathioprine)
\end{itemize}

\subsection{Wilson's Disease}\label{wilsons-disease}

\begin{itemize}
\tightlist
\item
  Autosomal recessive disorder of copper metabolism
\item
  Presents with liver dysfunction, neuropsychiatric symptoms
\item
  Low ceruloplasmin, elevated 24-hour urinary copper
\item
  Treatment: Zinc, chelating agents (penicillamine)
\end{itemize}

Metabolic Liver Diseases

These are inherited disorders affecting liver metabolism.

\textbf{Examples}

\begin{itemize}
\tightlist
\item
  \textbf{Galactosemia}: Deficiency of galactose-1-phosphate uridyl
  transferase; leads to liver failure, cataracts, E. coli sepsis.
\item
  \textbf{Hereditary fructose intolerance}
\item
  \textbf{Tyrosinemia type 1}: Presents with hepatomegaly, coagulopathy;
  managed with nitisinone and diet.
\end{itemize}

These conditions are often diagnosed early due to symptoms like
hypoglycemia, hepatomegaly, and failure to thrive.

\section{Liver Failure}\label{liver-failure}

\textbf{Acute Liver Failure (ALF)}

Defined by rapid deterioration in liver function with encephalopathy and
coagulopathy in a previously healthy child.

\textbf{Causes}

\begin{itemize}
\tightlist
\item
  Viral hepatitis (especially HBV)
\item
  Drugs (e.g., paracetamol overdose)
\item
  Metabolic diseases
\item
  Autoimmune hepatitis
\end{itemize}

\textbf{Clinical Features}

\begin{itemize}
\tightlist
\item
  Jaundice
\item
  Confusion or irritability
\item
  Bleeding tendencies
\item
  Ascites
\end{itemize}

\textbf{Management}

\begin{itemize}
\tightlist
\item
  Supportive care in ICU
\item
  Monitoring for cerebral edema, hypoglycemia
\item
  Specific treatment depending on etiology (e.g., N-acetylcysteine for
  paracetamol poisoning)
\item
  Liver transplant if unresponsive to therapy
\end{itemize}

\section{Cirrhosis and Portal
Hypertension}\label{cirrhosis-and-portal-hypertension}

Cirrhosis refers to end-stage liver disease with fibrosis and
regenerative nodules. In children, it can arise from:

\begin{itemize}
\tightlist
\item
  Biliary atresia
\item
  Chronic viral hepatitis
\item
  Metabolic liver disease
\item
  Autoimmune hepatitis
\end{itemize}

\textbf{Portal Hypertension}

Leads to complications such as:

\begin{itemize}
\tightlist
\item
  Splenomegaly
\item
  Esophageal varices
\item
  Ascites
\end{itemize}

\textbf{Management}

\begin{itemize}
\tightlist
\item
  Beta-blockers to reduce portal pressure
\item
  Endoscopic variceal ligation
\item
  Diuretics for ascites
\item
  Consideration for liver transplant
\end{itemize}

\section{Liver Tumors}\label{liver-tumors}

Pediatric liver tumors are rare but important.

\textbf{Hepatoblastoma}

\begin{itemize}
\tightlist
\item
  Most common malignant liver tumor in children \textless3 years
\item
  Presents with abdominal mass, weight loss
\item
  Alpha-fetoprotein (AFP) often elevated
\item
  Treatment includes surgery and chemotherapy
\end{itemize}

\textbf{Hepatocellular Carcinoma (HCC)}

\begin{itemize}
\tightlist
\item
  More common in older children and those with chronic HBV
\item
  Often presents late with poor prognosis
\end{itemize}

\section{Drug and Toxin-Induced Liver
Injury}\label{drug-and-toxin-induced-liver-injury}

Numerous drugs and herbal preparations used in Ghana have hepatotoxic
potential.

\textbf{Common agents}

\begin{itemize}
\tightlist
\item
  Paracetamol overdose: Can cause acute liver failure.
\item
  Antituberculous therapy (e.g., isoniazid, rifampicin)
\item
  Herbal medications: Unregulated products may lead to hepatotoxicity.
\end{itemize}

\textbf{Prevention and Management}

\begin{itemize}
\tightlist
\item
  Educating caregivers on safe medication use
\item
  Early recognition of hepatotoxicity
\item
  Withdrawal of offending agent
\item
  Use of antidotes where available (e.g., NAC for paracetamol)
\end{itemize}

\section{Liver Transplantation in
Children}\label{liver-transplantation-in-children}

In end-stage liver disease or irreversible acute liver failure, liver
transplantation may be lifesaving.

\textbf{Indications}

\begin{itemize}
\tightlist
\item
  Biliary atresia unresponsive to surgery
\item
  Metabolic liver disease
\item
  Acute liver failure not responsive to treatment
\item
  Liver tumors not resectable
\end{itemize}

Although liver transplantation is not widely available in Ghana, there
are ongoing efforts to improve access through regional collaborations.

\section{Clinical Evaluation of a Child with Suspected Liver
Disease}\label{clinical-evaluation-of-a-child-with-suspected-liver-disease}

\subsection{History}\label{history-6}

\begin{itemize}
\tightlist
\item
  Duration and progression of jaundice
\item
  Stool and urine color
\item
  Family history of liver or metabolic diseases
\item
  Drug and herbal intake
\item
  Birth and feeding history
\end{itemize}

\subsection{Physical Examination}\label{physical-examination-4}

\begin{itemize}
\tightlist
\item
  Jaundice
\item
  Hepatosplenomegaly
\item
  Ascites
\item
  Encephalopathy
\item
  Growth assessment
\end{itemize}

\subsection{Laboratory Tests}\label{laboratory-tests-1}

\begin{itemize}
\tightlist
\item
  Liver function tests: AST, ALT, ALP, GGT, bilirubin
\item
  Coagulation profile: PT, INR
\item
  Serum albumin
\item
  Viral markers
\item
  Autoimmune profile
\item
  Metabolic screens
\end{itemize}

\subsection{Imaging}\label{imaging-3}

\begin{itemize}
\tightlist
\item
  Abdominal ultrasound
\item
  Doppler studies of hepatic vasculature
\item
  CT or MRI in selected cases
\end{itemize}

\subsection{Liver Biopsy}\label{liver-biopsy}

\begin{itemize}
\tightlist
\item
  Helpful in diagnosing chronic hepatitis, metabolic and autoimmune
  diseases
\end{itemize}

\section{Prevention and Public Health
Implications}\label{prevention-and-public-health-implications}

In Ghana, a major proportion of the liver disease burden is preventable
through:

\begin{itemize}
\tightlist
\item
  \textbf{Universal hepatitis B vaccination} at birth
\item
  \textbf{Screening pregnant women} for HBV
\item
  \textbf{Avoiding unnecessary use of hepatotoxic drugs and herbs}
\item
  \textbf{Education on nutrition and infection prevention}
\item
  \textbf{Early detection and referral} of children with jaundice,
  hepatomegaly, or growth faltering
\end{itemize}

\section{Conclusion}\label{conclusion-32}

Liver diseases in children vary in cause and symptoms. A high level of
suspicion, early diagnosis, and prompt treatment are vital for better
outcomes. For medical students and healthcare providers in Ghana,
understanding these conditions---including the role of local infections
and cultural practices---is important for effective care and preventing
long-term problems. Focusing on vaccination, health education, and
improving diagnostic tools is a key national priority for managing
paediatric liver diseases.

\chapter{Prolonged Jaundice}\label{prolonged-jaundice}

\section{Introduction}\label{introduction-60}

Jaundice is one of the most common clinical findings in the neonatal
period, affecting approximately 60\% of term and 80\% of preterm
newborns. In most cases, neonatal jaundice is physiological and resolves
within the first week of life. However, when jaundice persists beyond
the expected period, it is termed \textbf{prolonged jaundice}, and
warrants thorough evaluation to exclude underlying pathology such as
biliary atresia, hypothyroidism, or hemolytic disease.

Early recognition and appropriate management of prolonged jaundice are
essential to prevent complications such as cholestasis, kernicterus, and
long-term hepatic injury.

\section{Definition}\label{definition-38}

\textbf{Prolonged (or persistent) neonatal jaundice} is defined as
\textbf{jaundice persisting beyond:} - \textbf{14 days} in term infants,
or - \textbf{21 days} in preterm infants.

The key to evaluating prolonged jaundice lies in distinguishing between
\textbf{conjugated (direct)} and \textbf{unconjugated (indirect)}
hyperbilirubinaemia.

\begin{itemize}
\tightlist
\item
  \textbf{Unconjugated hyperbilirubinaemia:} often benign (e.g., breast
  milk jaundice, haemolysis), but may occasionally indicate pathological
  processes.
\item
  \textbf{Conjugated hyperbilirubinaemia:} always pathological and
  indicative of hepatobiliary disease or systemic disorder.
\end{itemize}

\section{Bilirubin Metabolism: Brief
Review}\label{bilirubin-metabolism-brief-review}

Bilirubin is a breakdown product of haem derived from senescent red
blood cells.

\begin{enumerate}
\def\labelenumi{\arabic{enumi}.}
\tightlist
\item
  \textbf{Production:} Haem is converted to biliverdin, then to
  unconjugated bilirubin in the reticuloendothelial system.
\item
  \textbf{Transport:} Unconjugated bilirubin binds to albumin for
  transport to the liver.
\item
  \textbf{Hepatic uptake and conjugation:} Hepatocytes conjugate
  bilirubin with glucuronic acid via the enzyme \textbf{UDP-glucuronyl
  transferase}, forming conjugated (water-soluble) bilirubin.
\item
  \textbf{Excretion:} Conjugated bilirubin is excreted in bile into the
  intestines, converted to urobilinogen and stercobilin, and eliminated
  in faeces. A portion is reabsorbed via the enterohepatic circulation.
\end{enumerate}

Disruption at any step can result in elevated serum bilirubin and
jaundice.

\section{Classification}\label{classification-12}

Prolonged jaundice can be broadly divided into:

\begin{enumerate}
\def\labelenumi{\arabic{enumi}.}
\tightlist
\item
  \textbf{Predominantly unconjugated (indirect) hyperbilirubinaemia}
\item
  \textbf{Predominantly conjugated (direct) hyperbilirubinaemia
  (cholestatic jaundice)}
\end{enumerate}

The distinction is made using serum bilirubin fractionation: -
Conjugated bilirubin \textgreater{} 20\% of total, or \textgreater{} 34
µmol/L (2 mg/dL) is \textbf{pathological}.

\section{Causes of Prolonged
Jaundice}\label{causes-of-prolonged-jaundice}

\subsection{A. Predominantly Unconjugated
Hyperbilirubinaemia}\label{a.-predominantly-unconjugated-hyperbilirubinaemia}

\subsubsection{\texorpdfstring{1. \textbf{Breast Milk
Jaundice}}{1. Breast Milk Jaundice}}\label{breast-milk-jaundice}

\begin{itemize}
\tightlist
\item
  Occurs in healthy, exclusively breastfed infants.
\item
  Typically appears after the first week and may persist for up to 6
  weeks.
\item
  Mechanism: presence of β-glucuronidase and other substances in breast
  milk that increase enterohepatic circulation of bilirubin.
\item
  Infant is otherwise healthy with normal weight gain and stool/urine
  colour.
\end{itemize}

\subsubsection{\texorpdfstring{2. \textbf{Breastfeeding (Lactation
Failure)
Jaundice}}{2. Breastfeeding (Lactation Failure) Jaundice}}\label{breastfeeding-lactation-failure-jaundice}

\begin{itemize}
\tightlist
\item
  Seen in the first week due to inadequate milk intake and dehydration.
\item
  Leads to increased enterohepatic circulation and elevated unconjugated
  bilirubin.
\end{itemize}

\subsubsection{\texorpdfstring{3. \textbf{Haemolytic
Disorders}}{3. Haemolytic Disorders}}\label{haemolytic-disorders}

\begin{itemize}
\tightlist
\item
  \textbf{ABO or Rh incompatibility}
\item
  \textbf{G6PD deficiency} (common in Ghana and other African settings)
\item
  \textbf{Hereditary spherocytosis, pyruvate kinase deficiency}
\item
  These conditions cause increased haemolysis and bilirubin production.
\end{itemize}

\subsubsection{\texorpdfstring{4.
\textbf{Hypothyroidism}}{4. Hypothyroidism}}\label{hypothyroidism-1}

\begin{itemize}
\tightlist
\item
  Reduced hepatic conjugation and decreased gut motility leading to
  prolonged unconjugated jaundice.
\end{itemize}

\subsubsection{\texorpdfstring{5. \textbf{Crigler--Najjar and Gilbert
Syndromes}}{5. Crigler--Najjar and Gilbert Syndromes}}\label{criglernajjar-and-gilbert-syndromes}

\begin{itemize}
\tightlist
\item
  Rare genetic disorders of bilirubin conjugation.

  \begin{itemize}
  \tightlist
  \item
    \emph{Crigler--Najjar Type I:} complete absence of glucuronyl
    transferase (severe).
  \item
    \emph{Crigler--Najjar Type II} and \emph{Gilbert syndrome:} partial
    deficiency.
  \end{itemize}
\end{itemize}

\subsubsection{\texorpdfstring{6.
\textbf{Prematurity}}{6. Prematurity}}\label{prematurity-1}

\begin{itemize}
\tightlist
\item
  Immature hepatic conjugation enzyme systems.
\end{itemize}

\subsection{B. Predominantly Conjugated Hyperbilirubinaemia (Cholestatic
Jaundice)}\label{b.-predominantly-conjugated-hyperbilirubinaemia-cholestatic-jaundice}

Conjugated bilirubin is water-soluble; its presence in serum is always
abnormal and indicates hepatocellular or obstructive disease.

\subsubsection{\texorpdfstring{1. \textbf{Biliary
Atresia}}{1. Biliary Atresia}}\label{biliary-atresia-1}

\begin{itemize}
\tightlist
\item
  Progressive, idiopathic obliteration of extrahepatic bile ducts.
\item
  Presents with jaundice after 2 weeks, pale (acholic) stools, dark
  urine, and hepatomegaly.
\item
  Requires urgent surgical intervention (Kasai portoenterostomy).
\end{itemize}

\subsubsection{\texorpdfstring{2. \textbf{Neonatal Hepatitis (Idiopathic
or
Infectious)}}{2. Neonatal Hepatitis (Idiopathic or Infectious)}}\label{neonatal-hepatitis-idiopathic-or-infectious}

\begin{itemize}
\tightlist
\item
  Inflammation and dysfunction of hepatocytes.
\item
  Causes include viral infections (CMV, rubella, hepatitis B,
  enteroviruses) and bacterial sepsis.
\end{itemize}

\subsubsection{\texorpdfstring{3. \textbf{Metabolic
Disorders}}{3. Metabolic Disorders}}\label{metabolic-disorders}

\begin{itemize}
\tightlist
\item
  \textbf{Galactosaemia:} deficiency of galactose-1-phosphate uridyl
  transferase leading to liver dysfunction, vomiting, and hypoglycaemia.
\item
  \textbf{Tyrosinaemia type I, α1-antitrypsin deficiency:} cause
  hepatocellular damage and cholestasis.
\end{itemize}

\subsubsection{\texorpdfstring{4. \textbf{Endocrine
Disorders}}{4. Endocrine Disorders}}\label{endocrine-disorders}

\begin{itemize}
\tightlist
\item
  \textbf{Hypopituitarism, hypothyroidism} may present with prolonged
  conjugated jaundice.
\end{itemize}

\subsubsection{\texorpdfstring{5. \textbf{Parenteral
Nutrition--Associated
Cholestasis}}{5. Parenteral Nutrition--Associated Cholestasis}}\label{parenteral-nutritionassociated-cholestasis}

\begin{itemize}
\tightlist
\item
  Seen in preterm infants on prolonged parenteral nutrition.
\end{itemize}

\subsubsection{\texorpdfstring{6. \textbf{Genetic or Structural
Disorders}}{6. Genetic or Structural Disorders}}\label{genetic-or-structural-disorders}

\begin{itemize}
\tightlist
\item
  \textbf{Alagille syndrome:} paucity of intrahepatic bile ducts,
  associated with cardiac and facial anomalies.
\item
  \textbf{Choledochal cysts:} congenital dilatation of the bile ducts
  causing obstruction.
\end{itemize}

\section{Clinical Features}\label{clinical-features-64}

\subsection{A. History}\label{a.-history}

\begin{itemize}
\tightlist
\item
  Onset and duration of jaundice
\item
  Feeding history: breast or formula, adequacy of feeds
\item
  Colour of stool (pale/acholic vs normal yellow)
\item
  Colour of urine (dark vs normal)
\item
  Growth pattern and weight gain
\item
  Family history of haemolytic disorders or liver disease
\item
  Maternal infections during pregnancy (e.g., hepatitis, TORCH)
\item
  Perinatal history (asphyxia, sepsis, prematurity)
\end{itemize}

\subsection{B. Physical Examination}\label{b.-physical-examination}

\begin{itemize}
\tightlist
\item
  \textbf{General appearance:} activity level, growth, nutritional
  status
\item
  \textbf{Colour:} extent of jaundice (blanching under natural light)
\item
  \textbf{Stool and urine:} observe directly if possible
\item
  \textbf{Hepatomegaly:} common in biliary atresia and hepatitis
\item
  \textbf{Splenomegaly:} may indicate haemolysis or portal hypertension
\item
  \textbf{Dysmorphic features:} suggest syndromic or metabolic disorders
\item
  \textbf{Signs of hypothyroidism:} macroglossia, umbilical hernia,
  hypotonia
\item
  \textbf{Other:} ascites, petechiae (suggests liver failure)
\end{itemize}

\section{Investigations}\label{investigations-47}

The goal is to differentiate between conjugated and unconjugated
hyperbilirubinaemia and identify underlying causes.

\subsection{Initial Screening}\label{initial-screening}

\begin{enumerate}
\def\labelenumi{\arabic{enumi}.}
\tightlist
\item
  \textbf{Total and direct (conjugated) serum bilirubin:} essential
  first step.

  \begin{itemize}
  \tightlist
  \item
    Conjugated \textgreater{} 34 µmol/L or \textgreater20\% of total =
    cholestasis.
  \end{itemize}
\item
  \textbf{Full blood count, reticulocyte count:} anaemia and
  reticulocytosis suggest haemolysis.
\item
  \textbf{Blood group and Coombs test:} for ABO/Rh incompatibility.
\item
  \textbf{Peripheral smear:} to detect spherocytes or other RBC
  abnormalities.
\item
  \textbf{Thyroid function tests (TFTs):} rule out hypothyroidism.
\item
  \textbf{G6PD screening:} especially in African and Asian infants.
\item
  \textbf{Liver function tests (ALT, AST, ALP, GGT, albumin, PT/INR):}
  assess hepatocellular or cholestatic pattern.
\end{enumerate}

\subsection{Additional/Targeted Tests}\label{additionaltargeted-tests}

\begin{itemize}
\tightlist
\item
  \textbf{Urine analysis:} bilirubin (present in conjugated jaundice),
  reducing substances (galactosaemia).
\item
  \textbf{Viral studies:} TORCH, hepatitis panel, CMV PCR.
\item
  \textbf{Metabolic screening:} galactose-1-phosphate, amino acid
  profile.
\item
  \textbf{Ultrasound abdomen:} evaluate biliary tree and liver
  architecture.
\item
  \textbf{Hepatobiliary scintigraphy (HIDA scan):} assess bile
  excretion; non-visualization of intestines after tracer administration
  suggests biliary atresia.
\item
  \textbf{Liver biopsy:} gold standard for differentiating biliary
  atresia from neonatal hepatitis.
\end{itemize}

\section{Differential Diagnosis}\label{differential-diagnosis-26}

\begin{itemize}
\tightlist
\item
  Physiological jaundice (if prolonged due to mild immaturity)
\item
  Breast milk jaundice
\item
  Biliary atresia
\item
  Neonatal hepatitis
\item
  Hypothyroidism
\item
  G6PD deficiency
\item
  Sepsis
\item
  Galactosaemia or other metabolic disorders
\end{itemize}

\section{Management}\label{management-70}

Management depends on the underlying cause and the type of
hyperbilirubinaemia.

\section{A. General Principles}\label{a.-general-principles}

\begin{enumerate}
\def\labelenumi{\arabic{enumi}.}
\tightlist
\item
  \textbf{Identify the cause early.}
\item
  \textbf{Differentiate conjugated from unconjugated jaundice.}
\item
  \textbf{Prevent bilirubin encephalopathy in unconjugated
  hyperbilirubinaemia.}
\item
  \textbf{Provide nutritional and supportive care.}
\end{enumerate}

\subsection{B. Management of Unconjugated Prolonged
Jaundice}\label{b.-management-of-unconjugated-prolonged-jaundice}

\subsubsection{\texorpdfstring{1. \textbf{Breast Milk
Jaundice}}{1. Breast Milk Jaundice}}\label{breast-milk-jaundice-1}

\begin{itemize}
\tightlist
\item
  Continue breastfeeding; it is benign.
\item
  If bilirubin is significantly elevated (\textgreater250 µmol/L), a
  temporary 24-hour interruption with formula may cause rapid decline.
\item
  No need for phototherapy unless bilirubin levels approach treatment
  thresholds.
\end{itemize}

\subsubsection{\texorpdfstring{2. \textbf{Breastfeeding/Lactation
Failure
Jaundice}}{2. Breastfeeding/Lactation Failure Jaundice}}\label{breastfeedinglactation-failure-jaundice}

\begin{itemize}
\tightlist
\item
  Encourage frequent and effective breastfeeding (8--12 times/day).
\item
  Manage dehydration if present.
\end{itemize}

\subsubsection{\texorpdfstring{3. \textbf{Haemolytic
Causes}}{3. Haemolytic Causes}}\label{haemolytic-causes}

\begin{itemize}
\tightlist
\item
  Manage underlying haemolysis:

  \begin{itemize}
  \tightlist
  \item
    Exchange transfusion if bilirubin at neurotoxic levels.
  \item
    Avoid oxidative drugs in G6PD deficiency.
  \end{itemize}
\end{itemize}

\subsubsection{\texorpdfstring{4.
\textbf{Hypothyroidism}}{4. Hypothyroidism}}\label{hypothyroidism-2}

\begin{itemize}
\tightlist
\item
  Treat with \textbf{levothyroxine} once confirmed.
\end{itemize}

\subsubsection{\texorpdfstring{5. \textbf{Crigler--Najjar
Syndrome}}{5. Crigler--Najjar Syndrome}}\label{criglernajjar-syndrome}

\begin{itemize}
\tightlist
\item
  Type I: lifelong phototherapy, liver transplantation may be curative.
\item
  Type II: responds to \textbf{phenobarbital}.
\end{itemize}

\subsection{C. Management of Conjugated (Cholestatic)
Jaundice}\label{c.-management-of-conjugated-cholestatic-jaundice}

\subsubsection{\texorpdfstring{1. \textbf{Biliary
Atresia}}{1. Biliary Atresia}}\label{biliary-atresia-2}

\begin{itemize}
\tightlist
\item
  Early diagnosis is critical (surgery before 8 weeks improves
  prognosis).
\item
  \textbf{Kasai portoenterostomy} is the initial procedure.
\item
  Postoperative management includes antibiotics, fat-soluble vitamins
  (A, D, E, K), and nutritional support.
\end{itemize}

\subsubsection{\texorpdfstring{2. \textbf{Neonatal
Hepatitis}}{2. Neonatal Hepatitis}}\label{neonatal-hepatitis-1}

\begin{itemize}
\tightlist
\item
  Supportive therapy with adequate nutrition, vitamin supplementation,
  and management of complications.
\item
  Antiviral therapy if specific infection identified.
\end{itemize}

\subsubsection{\texorpdfstring{3. \textbf{Metabolic
Disorders}}{3. Metabolic Disorders}}\label{metabolic-disorders-1}

\begin{itemize}
\tightlist
\item
  \textbf{Galactosaemia:} eliminate galactose from the diet immediately.
\item
  \textbf{Tyrosinaemia:} low tyrosine and phenylalanine diet, nitisinone
  therapy.
\item
  \textbf{α1-Antitrypsin deficiency:} supportive care; avoid hepatotoxic
  drugs.
\end{itemize}

\subsubsection{\texorpdfstring{4. \textbf{Endocrine
Disorders}}{4. Endocrine Disorders}}\label{endocrine-disorders-1}

\begin{itemize}
\tightlist
\item
  Replacement therapy for hypothyroidism or hypopituitarism.
\end{itemize}

\subsection{\texorpdfstring{5. \textbf{Parenteral Nutrition--Associated
Cholestasis}}{5. Parenteral Nutrition--Associated Cholestasis}}\label{parenteral-nutritionassociated-cholestasis-1}

\begin{itemize}
\tightlist
\item
  Reduce or discontinue parenteral nutrition; promote enteral feeding.
\item
  Use ursodeoxycholic acid to improve bile flow.
\end{itemize}

\subsubsection{\texorpdfstring{6. \textbf{Symptomatic
Management}}{6. Symptomatic Management}}\label{symptomatic-management}

\begin{itemize}
\tightlist
\item
  \textbf{Fat-soluble vitamins:} due to malabsorption in cholestasis.
\item
  \textbf{Medium-chain triglyceride (MCT)-based feeds:} improve calorie
  intake.
\item
  \textbf{Pruritus relief:} cholestyramine, ursodeoxycholic acid.
\end{itemize}

\section{Complications}\label{complications-32}

\begin{itemize}
\tightlist
\item
  \textbf{Kernicterus (bilirubin encephalopathy):} from severe
  unconjugated hyperbilirubinaemia.
\item
  \textbf{Malnutrition:} from fat malabsorption in cholestasis.
\item
  \textbf{Coagulopathy:} due to vitamin K deficiency.
\item
  \textbf{Liver failure:} in progressive hepatobiliary disease.
\item
  \textbf{Portal hypertension and cirrhosis:} in biliary atresia and
  chronic hepatitis.
\end{itemize}

\section{Prognosis}\label{prognosis-36}

\begin{itemize}
\tightlist
\item
  \textbf{Unconjugated causes} (breast milk jaundice, mild haemolysis)
  have excellent prognosis with supportive care.
\item
  \textbf{Biliary atresia:} prognosis depends on early detection and
  successful surgery. Delayed intervention leads to cirrhosis and need
  for liver transplantation.
\item
  \textbf{Metabolic and genetic disorders:} outcome depends on prompt
  diagnosis and disease-specific therapy.
\end{itemize}

\section{Prevention}\label{prevention-19}

\begin{itemize}
\tightlist
\item
  Promote exclusive breastfeeding and early postnatal follow-up.
\item
  Routine newborn screening for G6PD deficiency and congenital
  hypothyroidism.
\item
  Early recognition of pale stools and dark urine by caregivers.
\item
  Prompt referral of any infant jaundiced beyond 2 weeks for evaluation.
\end{itemize}

\section{Approach Summary for Prolonged
Jaundice}\label{approach-summary-for-prolonged-jaundice}

\begin{enumerate}
\def\labelenumi{\arabic{enumi}.}
\tightlist
\item
  \textbf{Assess duration and severity.}
\item
  \textbf{Fractionate bilirubin.}
\item
  \textbf{Conjugated \textgreater{} 20\% or \textgreater34 µmol/L →
  evaluate for cholestasis.}
\item
  \textbf{Check stool colour and urine colour.}
\item
  \textbf{Investigate systematically (haemolysis, thyroid, infection,
  metabolic, anatomical).}
\item
  \textbf{Manage cause-specific and provide supportive care.}
\item
  \textbf{Refer early for suspected biliary atresia.}
\end{enumerate}

\section{Key Takeaways}\label{key-takeaways-7}

\begin{itemize}
\tightlist
\item
  Any infant jaundiced beyond 2 weeks must be evaluated for pathological
  causes.
\item
  Always determine if the bilirubin is conjugated or unconjugated.
\item
  Conjugated hyperbilirubinaemia is never physiological.
\item
  Early diagnosis of biliary atresia (\textless8 weeks) markedly
  improves outcomes.
\item
  Supportive care (nutrition, vitamins, prevention of complications) is
  crucial.
\end{itemize}

\section{Suggested References}\label{suggested-references-3}

\begin{enumerate}
\def\labelenumi{\arabic{enumi}.}
\tightlist
\item
  World Health Organization. \emph{Pocket Book of Hospital Care for
  Children}, 3rd Edition, 2023.
\item
  Ghana Health Service. \emph{Standard Treatment Guidelines}, 2022.
\item
  Balistreri WF, Bezerra JA. Neonatal cholestasis. \emph{J Pediatr}.
  2021;229:8--17.
\item
  Fawaz R et al.~Guideline for the evaluation of cholestatic jaundice in
  infants. \emph{J Pediatr Gastroenterol Nutr}. 2017;64(1):154--168.
\item
  Moyer V, Freese DK, Whitington PF, et al.~Guideline for the evaluation
  of cholestatic jaundice in infants. \emph{J Pediatr Gastroenterol
  Nutr.} 2004;39(2):115--128.
\end{enumerate}

\chapter{Diarrhoea Diseases}\label{diarrhoea-diseases}

\section{Introduction}\label{introduction-61}

Diarrhoeal diseases remain one of the most common causes of morbidity
and mortality among children worldwide, particularly in low- and
middle-income countries. In sub-Saharan Africa, including Ghana,
diarrhoea is a leading cause of outpatient visits, hospital admissions,
and deaths among children under five years. The World Health
Organization (WHO) estimates that children under five experience an
average of three episodes of diarrhoea per year, contributing to
significant nutritional deficits and growth faltering.

\textbf{Definition:} Diarrhoea is defined as the passage of
\textbf{three or more loose or watery stools within 24 hours}, or a
stool consistency that is looser than normal for the individual. In
infants on breast milk, frequent passage of soft stools may be normal
and should not be mistaken for diarrhoea unless the stools are unusually
watery and accompanied by other symptoms such as dehydration or fever.

\section{Classification}\label{classification-13}

Diarrhoeal diseases are broadly classified based on \textbf{duration}
and \textbf{pathophysiology}.

\subsection{By Duration}\label{by-duration}

\begin{itemize}
\tightlist
\item
  \textbf{Acute Diarrhoea:} Lasts less than 14 days. Usually due to
  infectious causes (viral, bacterial, or parasitic).
\item
  \textbf{Persistent Diarrhoea:} Lasts 14 days or longer. May follow an
  acute infectious episode and often associated with malnutrition,
  secondary lactose intolerance, or underlying enteric pathology.
\item
  \textbf{Chronic Diarrhoea:} Lasts more than 30 days. Typically due to
  non-infectious causes such as malabsorption, inflammatory bowel
  disease, or congenital disorders.
\end{itemize}

\subsection{By Clinical Presentation}\label{by-clinical-presentation}

\begin{itemize}
\tightlist
\item
  \textbf{Acute Watery Diarrhoea:} Sudden onset, stools watery without
  visible blood. Common causes: \emph{Rotavirus, Norovirus, Vibrio
  cholerae, ETEC}.
\item
  \textbf{Acute Bloody Diarrhoea (Dysentery):} Presence of visible blood
  or mucus; indicates mucosal invasion. Common causes: \emph{Shigella,
  Entamoeba histolytica, Salmonella, Campylobacter}.
\item
  \textbf{Persistent Diarrhoea:} Often due to prolonged infection,
  malnutrition, or mucosal damage.
\item
  \textbf{Diarrhoea with Severe Malnutrition:} Often chronic and
  complicated by electrolyte imbalance and secondary infections.
\end{itemize}

\section{Epidemiology}\label{epidemiology-14}

In Ghana, diarrhoeal diseases are among the top five causes of
outpatient visits in children under five. Peak incidence is between 6
months and 2 years (weaning period). Seasonality is common --- cases
often increase during the rainy season due to contamination of water
sources. Contributing factors include poor sanitation, inadequate hand
hygiene, unsafe drinking water, and improper food handling.

\section{Aetiology}\label{aetiology-25}

\subsection{Infectious Causes}\label{infectious-causes}

\subsubsection{Viral}\label{viral}

\begin{itemize}
\tightlist
\item
  \textbf{Rotavirus:} Leading cause in children under 2 years; profuse
  watery diarrhoea, vomiting, low-grade fever.
\item
  \textbf{Norovirus:} Causes outbreaks across all ages.
\item
  \textbf{Adenovirus (types 40, 41):} Prolonged watery diarrhoea.
\item
  \textbf{Astrovirus:} Generally mild illness.
\end{itemize}

\subsubsection{Bacterial}\label{bacterial}

\begin{itemize}
\tightlist
\item
  \textbf{ETEC (Enterotoxigenic E. coli):} Common in travellers and
  children; toxin-mediated secretory diarrhoea.
\item
  \textbf{EPEC (Enteropathogenic E. coli):} Infantile diarrhoea.
\item
  \textbf{Shigella spp.:} Dysentery with fever, tenesmus, bloody stools.
\item
  \textbf{Salmonella (non-typhoidal):} Food-borne; may cause invasive
  disease in infants.
\item
  \textbf{Campylobacter jejuni:} Associated with poultry; can be
  dysenteric.
\item
  \textbf{Vibrio cholerae:} Causes profuse watery diarrhoea
  (``rice-water stools'').
\item
  \textbf{Clostridioides difficile:} Often post-antibiotic.
\end{itemize}

\subsubsection{Parasitic}\label{parasitic}

\begin{itemize}
\tightlist
\item
  \textbf{Giardia lamblia:} Chronic/intermittent diarrhoea,
  malabsorption, steatorrhoea.
\item
  \textbf{Entamoeba histolytica:} Amoebic dysentery; may complicate with
  liver abscess.
\item
  \textbf{Cryptosporidium parvum:} Important in immunocompromised
  children (e.g., HIV).
\item
  \textbf{Isospora belli, Cyclospora:} Prolonged diarrhoea in
  immunosuppressed patients.
\end{itemize}

\subsection{Non-infectious Causes}\label{non-infectious-causes}

\begin{itemize}
\tightlist
\item
  Food intolerances (e.g., lactose intolerance)
\item
  Celiac disease
\item
  Inflammatory bowel disease (IBD)
\item
  Irritable bowel syndrome (IBS)
\item
  Congenital enzyme deficiencies
\item
  Medications (antibiotics, laxatives)
\end{itemize}

\section{Pathophysiology}\label{pathophysiology-41}

Diarrhoea results from disturbed intestinal fluid and electrolyte
transport. Key mechanisms:

\begin{itemize}
\tightlist
\item
  \textbf{Osmotic diarrhoea:} Non-absorbable solutes in the lumen draw
  water (e.g., lactose intolerance). Typically stops with fasting.
\item
  \textbf{Secretory diarrhoea:} Increased electrolyte and water
  secretion (e.g., cholera, ETEC). Persists despite fasting.
\item
  \textbf{Exudative diarrhoea:} Mucosal inflammation causes loss of
  blood, mucus, and proteins (e.g., Shigella, E. histolytica).
\item
  \textbf{Motility-related diarrhoea:} Rapid transit reduces absorption
  time (e.g., post-surgical states).
\item
  \textbf{Mixed mechanisms:} Especially in persistent diarrhoea where
  infection and malnutrition combine.
\end{itemize}

\section{Clinical Features}\label{clinical-features-65}

\subsection{Symptoms}\label{symptoms}

\begin{itemize}
\tightlist
\item
  Frequent loose or watery stools (± blood)
\item
  Vomiting (common in viral aetiologies)
\item
  Fever
\item
  Abdominal cramps or pain
\item
  Tenesmus (with dysentery)
\item
  Reduced appetite
\item
  Signs of dehydration: thirst, dry mucous membranes, decreased urine
  output, sunken eyes
\end{itemize}

\subsection{Signs of Dehydration (WHO
classification)}\label{signs-of-dehydration-who-classification}

\begin{itemize}
\tightlist
\item
  \textbf{No dehydration:} alert, normal eyes, drinks normally, skin
  pinch goes back quickly.
\item
  \textbf{Some dehydration:} restless/irritable, sunken eyes, thirsty,
  skin pinch goes back slowly.
\item
  \textbf{Severe dehydration:} lethargic/unconscious, very sunken eyes,
  unable to drink, skin pinch goes back very slowly (\textgreater2
  seconds).
\end{itemize}

\section{Assessment and Diagnosis}\label{assessment-and-diagnosis-1}

\subsection{History}\label{history-7}

\begin{itemize}
\tightlist
\item
  Onset, duration, stool frequency/character, presence of blood or mucus
\item
  Associated vomiting, fever, abdominal pain
\item
  Recent antibiotic use, recent food or water exposures, sick contacts
\item
  Breastfeeding/weaning history
\item
  Immunization status (rotavirus vaccine)
\item
  Underlying conditions (HIV, malnutrition)
\end{itemize}

\subsection{Physical Examination}\label{physical-examination-5}

\begin{itemize}
\tightlist
\item
  Vital signs, hydration assessment (capillary refill, pulse, blood
  pressure if indicated)
\item
  Weight and comparison with prior measurements
\item
  Nutritional assessment for wasting/edema
\item
  Abdominal examination for tenderness, distension, palpable masses
\item
  Perianal and perineal inspection for skin irritation or lesions
\end{itemize}

\subsection{Laboratory
Investigations}\label{laboratory-investigations-5}

(Not required for uncomplicated mild diarrhoea)

\begin{itemize}
\tightlist
\item
  \textbf{Stool microscopy, culture and sensitivity:} Indicated for
  persistent diarrhoea, dysentery, or outbreak investigation.
\item
  \textbf{Stool ova and parasites:} If parasitic infection suspected.
\item
  \textbf{Stool for reducing substances:} If lactose
  intolerance/malabsorption suspected.
\item
  \textbf{Electrolytes and renal function:} In severe dehydration or
  when IV fluids are given.
\item
  \textbf{HIV testing:} For chronic/persistent diarrhoea or risk
  factors.
\end{itemize}

\section{Differential Diagnosis}\label{differential-diagnosis-27}

\begin{itemize}
\tightlist
\item
  Acute appendicitis (may present with diarrhoea early)
\item
  Urinary tract infection presenting with fever and diarrhoea
\item
  Intussusception (bloody stools, abdominal pain, vomiting)
\item
  Malabsorption syndromes
\item
  Irritable bowel syndrome
\item
  Systemic sepsis with diarrhoea
\end{itemize}

\section{Management}\label{management-71}

\subsection{General Principles}\label{general-principles-4}

\begin{enumerate}
\def\labelenumi{\arabic{enumi}.}
\tightlist
\item
  Assess and correct dehydration.
\item
  Maintain nutrition and early refeeding.
\item
  Treat specific causes when indicated.
\item
  Prevent complications and recurrence.
\end{enumerate}

\subsection{WHO Fluid Management
Plans}\label{who-fluid-management-plans}

\begin{itemize}
\tightlist
\item
  \textbf{Plan A (No dehydration):} Home care with extra fluids and
  continued feeding.
\item
  \textbf{Plan B (Some dehydration):} ORS 75 mL/kg over 4 hours.
\item
  \textbf{Plan C (Severe dehydration):} Immediate IV rehydration
  (Ringer's lactate or normal saline).
\end{itemize}

\subsubsection{Plan A --- No Dehydration}\label{plan-a-no-dehydration}

\begin{itemize}
\tightlist
\item
  Continue breastfeeding and normal feeding. Provide extra fluids:

  \begin{itemize}
  \tightlist
  \item
    Less than 2 years: 50--100 mL after each loose stool
  \item
    2--10 years: 100--200 mL after each loose stool
  \item
    Less than 10 years: as much as desired
  \end{itemize}
\item
  Give zinc supplementation (10 mg/day if \textless6 months; 20 mg/day
  if ≥6 months) for 10--14 days.
\item
  Educate caregivers about danger signs.
\end{itemize}

\subsubsection{Plan B --- Some
Dehydration}\label{plan-b-some-dehydration}

\begin{itemize}
\tightlist
\item
  Give ORS 75 mL/kg over 4 hours and reassess. If improved, continue
  feeding; if not, repeat Plan B or escalate to Plan C.
\item
  Continue breastfeeding and frequent small feeds.
\end{itemize}

\subsubsection{Plan C --- Severe
Dehydration}\label{plan-c-severe-dehydration}

\begin{itemize}
\tightlist
\item
  \textbf{IV fluid therapy:}

  \begin{itemize}
  \tightlist
  \item
    Infants (\textless12 months): 30 mL/kg in first 1 hour, then 70
    mL/kg over next 5 hours (Ringer's lactate preferred).
  \item
    Children (\textgreater12 months): 30 mL/kg in first 30 minutes, then
    70 mL/kg over next 2.5 hours.
  \end{itemize}
\item
  Monitor for signs of fluid overload; reassess frequently.
\item
  If IV access cannot be established, use nasogastric rehydration with
  ORS.
\end{itemize}

\subsection{Nutritional Management}\label{nutritional-management-1}

\begin{itemize}
\tightlist
\item
  \textbf{Continue breastfeeding} throughout the illness.
\item
  Do not withhold food; resume age-appropriate feeding as soon as
  rehydration is achieved.
\item
  Use energy-dense, micronutrient-rich foods during recovery and give
  extra meals for 2 weeks post-illness.
\item
  Avoid high-sugar drinks and undiluted fruit juices.
\end{itemize}

\subsection{Specific Therapy}\label{specific-therapy-1}

\begin{itemize}
\tightlist
\item
  \textbf{Antibiotics} are not routine for acute watery diarrhoea.
  Indications include:

  \begin{itemize}
  \tightlist
  \item
    Dysentery (suspected Shigella)
  \item
    Confirmed cholera with severe dehydration
  \item
    Laboratory-confirmed bacterial infections
  \item
    Immunocompromised patients with invasive bacterial disease
  \end{itemize}
\end{itemize}

\textbf{Common choices (local guidelines may vary):} - \emph{Shigella:}
Ciprofloxacin 15 mg/kg orally twice daily for 3 days (check local
resistance patterns). - \emph{Cholera:} Single dose doxycycline
(adults/older children) or azithromycin 10 mg/kg single dose where
doxycycline contraindicated. - \emph{Giardiasis/Amebiasis:}
Metronidazole (observe paediatric dosing recommendations).

\textbf{Avoid antimotility agents (e.g., loperamide) in young children.}

\subsection{Zinc Supplementation}\label{zinc-supplementation}

\begin{itemize}
\tightlist
\item
  Reduces severity and duration, and prevents recurrence. Dose:

  \begin{itemize}
  \tightlist
  \item
    \textless6 months: 10 mg/day for 10--14 days
  \item
    ≥6 months: 20 mg/day for 10--14 days
  \end{itemize}
\end{itemize}

\section{Management of Persistent and Chronic
Diarrhoea}\label{management-of-persistent-and-chronic-diarrhoea}

\begin{itemize}
\tightlist
\item
  Investigate and treat underlying causes: persistent infection,
  malabsorption, cow's milk protein allergy, post-infectious lactose
  intolerance.
\item
  Nutritional rehabilitation with low-lactose or lactose-free feeds if
  indicated.
\item
  Replace micronutrients (zinc, vitamin A, folate) and correct anaemia
  when present.
\item
  Consider referral for specialised investigations (endoscopy, biopsy,
  sweat test, advanced imaging) if diarrhoea persists beyond 4 weeks.
\end{itemize}

\section{Complications}\label{complications-33}

\begin{itemize}
\tightlist
\item
  Dehydration and hypovolaemic shock
\item
  Electrolyte disturbances (hypokalaemia, hyponatraemia)
\item
  Metabolic acidosis
\item
  Malnutrition and stunting
\item
  Secondary lactose intolerance
\item
  Acute kidney injury in severe cases
\item
  Sepsis, especially with invasive bacterial pathogens
\end{itemize}

\section{Prevention and Control}\label{prevention-and-control}

\subsection{Household measures}\label{household-measures}

\begin{itemize}
\tightlist
\item
  Exclusive breastfeeding for the first 6 months of life
\item
  Handwashing with soap after defecation and before food preparation
\item
  Safe drinking water (boiling, chlorination, filtration)
\item
  Proper sanitation and safe disposal of faeces
\item
  Hygienic food preparation and storage
\item
  Timely use of ORS and zinc in episodes of diarrhoea
\end{itemize}

\subsection{Public health measures}\label{public-health-measures-1}

\begin{itemize}
\tightlist
\item
  \textbf{Rotavirus vaccination} included in Ghana's EPI since 2012 ---
  major impact on severe rotaviral disease.
\item
  Cholera surveillance and targeted vaccination in outbreaks and endemic
  areas.
\item
  Strengthening WASH (Water, Sanitation and Hygiene) infrastructure and
  behaviour-change programmes.
\end{itemize}

\section{Prognosis}\label{prognosis-37}

Most children recover fully with early recognition and appropriate
management. Recurrent or persistent diarrhoea is a major contributor to
malnutrition, stunting, and impaired cognitive development.

\section{Key Clinical Points}\label{key-clinical-points}

\begin{itemize}
\tightlist
\item
  Accurate assessment of dehydration is critical and guides management.
\item
  ORS and zinc are cornerstone therapies for most childhood diarrhoeas.
\item
  Continue feeding and especially breastfeeding; do not withhold
  nutrition.
\item
  Reserve antibiotics for indicated cases and follow local resistance
  patterns.
\item
  Prevention through vaccination (rotavirus), breastfeeding, and WASH
  interventions is essential.
\end{itemize}

\section{Suggested Further Reading}\label{suggested-further-reading}

\begin{itemize}
\tightlist
\item
  World Health Organization. \emph{Pocket Book of Hospital Care for
  Children}, 3rd Edition (most recent).
\item
  Ghana Health Service. \emph{Standard Treatment Guidelines} (latest
  edition).
\item
  UNICEF/WHO. \emph{Integrated Management of Childhood Illness (IMCI)
  Chart Booklet}.
\end{itemize}

\chapter{Common Vitamin-Related
Pathologies}\label{common-vitamin-related-pathologies}

\section{Introduction}\label{introduction-62}

Vitamins are essential organic compounds required in small quantities
for normal growth, metabolism, and physiological functions. Although
they do not provide energy, they are vital cofactors in numerous
biochemical reactions. Children, because of their rapid growth and
development, are particularly susceptible to vitamin deficiencies and
imbalances. In resource-limited settings, these deficiencies remain a
significant cause of morbidity and mortality, often coexisting with
protein-energy malnutrition and infections.

This lecture note reviews the physiology, deficiency states, and
clinical manifestations of common vitamin-related pathologies
encountered in paediatric practice.

\section{Classification of Vitamins}\label{classification-of-vitamins}

Vitamins are broadly classified into \textbf{fat-soluble} and
\textbf{water-soluble} groups based on their solubility and
physiological characteristics.

\begin{longtable}[]{@{}ll@{}}
\toprule\noalign{}
Fat-soluble & Water-soluble \\
\midrule\noalign{}
\endhead
\bottomrule\noalign{}
\endlastfoot
Vitamin A (Retinoids) & Vitamin B1 (Thiamine) \\
Vitamin D (Calciferols) & Vitamin B2 (Riboflavin) \\
Vitamin E (Tocopherols) & Vitamin B3 (Niacin) \\
Vitamin K (Phylloquinone and Menaquinone) & Vitamin B6 (Pyridoxine) \\
& Vitamin B12 (Cobalamin) \\
& Folic Acid (Vitamin B9) \\
& Vitamin C (Ascorbic Acid) \\
\end{longtable}

\textbf{Fat-soluble vitamins} are absorbed along with dietary fat and
stored in the liver and adipose tissue; hence, deficiencies develop
slowly. Conversely, \textbf{water-soluble vitamins} are not stored
significantly and require continuous dietary intake.

\section{Vitamin A Deficiency}\label{vitamin-a-deficiency-1}

\subsection{Physiology}\label{physiology}

Vitamin A is essential for vision, epithelial integrity, immune
function, and growth. It is obtained as preformed retinol from animal
sources (e.g., liver, milk, eggs) or as provitamin A carotenoids from
plant sources (e.g., carrots, leafy greens).

\subsection{Pathophysiology and
Deficiency}\label{pathophysiology-and-deficiency}

Deficiency arises from inadequate intake, malabsorption (as in cystic
fibrosis or cholestatic liver disease), or increased requirements during
infection and growth.

\subsection{Clinical Features}\label{clinical-features-66}

\begin{itemize}
\tightlist
\item
  \textbf{Ocular manifestations (Xerophthalmia):}

  \begin{itemize}
  \tightlist
  \item
    Night blindness (earliest symptom)
  \item
    Conjunctival xerosis
  \item
    Bitot's spots
  \item
    Corneal xerosis, keratomalacia → blindness
  \end{itemize}
\item
  \textbf{Systemic manifestations:}

  \begin{itemize}
  \tightlist
  \item
    Growth retardation
  \item
    Increased susceptibility to infections (especially measles and
    diarrhoea)
  \item
    Follicular hyperkeratosis of the skin
  \end{itemize}
\end{itemize}

\subsection{Prevention and
Management}\label{prevention-and-management-1}

\begin{itemize}
\tightlist
\item
  \textbf{Dietary diversification:} inclusion of animal sources and
  carotene-rich foods.
\item
  \textbf{Supplementation:} WHO recommends 100,000 IU (infants 6--11
  months) or 200,000 IU (children 12--59 months) every 4--6 months in
  endemic regions.
\item
  \textbf{Treatment of severe cases:} high-dose vitamin A therapy
  (200,000 IU orally on days 1, 2, and 14).
\end{itemize}

\section{Vitamin D Deficiency
(Rickets)}\label{vitamin-d-deficiency-rickets}

\subsection{Physiology}\label{physiology-1}

Vitamin D regulates calcium and phosphate metabolism, promoting bone
mineralization. It is synthesized in the skin upon sunlight exposure and
obtained from dietary sources such as fortified milk, fish oils, and
eggs.

\subsection{Pathophysiology}\label{pathophysiology-42}

Deficiency results in defective bone mineralization, leading to
\textbf{rickets} in children and \textbf{osteomalacia} in adults.

\subsection{Clinical Features}\label{clinical-features-67}

\begin{itemize}
\tightlist
\item
  Skeletal abnormalities:

  \begin{itemize}
  \tightlist
  \item
    Craniotabes, frontal bossing, and delayed closure of fontanelles
  \item
    Rachitic rosary (prominent costochondral junctions)
  \item
    Wrist and ankle widening
  \item
    Bowed legs (genu varum) or knock knees (genu valgum)
  \end{itemize}
\item
  Delayed motor milestones
\item
  Hypocalcaemic symptoms: tetany, seizures
\end{itemize}

\subsection{Investigations}\label{investigations-48}

\begin{itemize}
\tightlist
\item
  Low serum calcium and phosphate
\item
  Elevated alkaline phosphatase
\item
  Radiographic features: cupping, fraying, and widening of metaphyses
\end{itemize}

\subsection{Management}\label{management-72}

\begin{itemize}
\tightlist
\item
  \textbf{Vitamin D supplementation:} 2,000 IU daily for 3 months,
  followed by maintenance of 400 IU daily.
\item
  \textbf{Calcium supplementation:} as deficiency often coexists.
\item
  \textbf{Sunlight exposure:} at least 30 minutes weekly on face and
  limbs.
\end{itemize}

\section{Vitamin E Deficiency}\label{vitamin-e-deficiency}

\subsection{Physiology}\label{physiology-2}

Vitamin E acts as an antioxidant, protecting cell membranes from
oxidative damage.

\subsection{Clinical Manifestations}\label{clinical-manifestations-1}

\begin{itemize}
\tightlist
\item
  Haemolytic anaemia in preterm infants
\item
  Neurological manifestations: ataxia, hyporeflexia, and peripheral
  neuropathy
\item
  Retinopathy
\end{itemize}

\subsection{Management}\label{management-73}

Oral vitamin E (10--25 IU/kg/day) for deficiency states and prevention
in premature infants.

\section{Vitamin K Deficiency}\label{vitamin-k-deficiency}

\subsection{Overview}\label{overview-10}

Vitamin K is necessary for synthesis of clotting factors II, VII, IX,
and X. It is obtained from green vegetables and synthesized by
intestinal bacteria.

\subsection{Clinical Manifestations}\label{clinical-manifestations-2}

\begin{itemize}
\tightlist
\item
  \textbf{Haemorrhagic disease of the newborn (HDN):}

  \begin{itemize}
  \tightlist
  \item
    Early (within 24 hours): due to maternal drug interference (e.g.,
    anticonvulsants)
  \item
    Classic (2--7 days): due to low stores and inadequate milk supply
  \item
    Late (2--12 weeks): seen in exclusively breastfed infants without
    prophylaxis
  \end{itemize}
\item
  Bleeding from mucosal sites, gastrointestinal tract, or intracranial
  haemorrhage
\end{itemize}

\subsection{Prevention and
Management}\label{prevention-and-management-2}

\begin{itemize}
\tightlist
\item
  \textbf{Prophylaxis:} 1 mg vitamin K1 intramuscularly at birth
\item
  \textbf{Treatment:} 1--5 mg vitamin K1 parenterally, plus transfusion
  for severe bleeding
\end{itemize}

\section{Vitamin B Complex
Deficiencies}\label{vitamin-b-complex-deficiencies}

\subsection{Thiamine (Vitamin B1)}\label{thiamine-vitamin-b1}

\begin{itemize}
\tightlist
\item
  \textbf{Function:} Coenzyme in carbohydrate metabolism.
\item
  \textbf{Deficiency (Beriberi):}

  \begin{itemize}
  \tightlist
  \item
    \emph{Infantile form:} heart failure, tachycardia, vomiting,
    aphonia.
  \item
    \emph{Adult form:} peripheral neuropathy, Wernicke's encephalopathy.
  \end{itemize}
\item
  \textbf{Management:} Thiamine 10--25 mg daily.
\end{itemize}

\subsection{Riboflavin (Vitamin B2)}\label{riboflavin-vitamin-b2}

\begin{itemize}
\tightlist
\item
  \textbf{Function:} Oxidation-reduction reactions.
\item
  \textbf{Deficiency:} angular stomatitis, cheilosis, glossitis,
  seborrheic dermatitis.
\item
  \textbf{Treatment:} Riboflavin 5--10 mg daily.
\end{itemize}

\subsection{Niacin (Vitamin B3)}\label{niacin-vitamin-b3}

\begin{itemize}
\tightlist
\item
  \textbf{Function:} Component of NAD/NADP.
\item
  \textbf{Deficiency (Pellagra):} dermatitis, diarrhoea, dementia.
\item
  \textbf{Treatment:} Nicotinamide 50--300 mg/day.
\end{itemize}

\subsection{Pyridoxine (Vitamin B6)}\label{pyridoxine-vitamin-b6}

\begin{itemize}
\tightlist
\item
  \textbf{Deficiency:} seizures, anaemia, glossitis, peripheral
  neuropathy.
\item
  \textbf{Management:} Pyridoxine 5--10 mg/day; higher doses in
  drug-induced cases (e.g., isoniazid).
\end{itemize}

\subsection{Folic Acid (Vitamin B9)}\label{folic-acid-vitamin-b9}

\begin{itemize}
\tightlist
\item
  \textbf{Function:} DNA synthesis and red cell maturation.
\item
  \textbf{Deficiency:} megaloblastic anaemia, neural tube defects in
  foetus.
\item
  \textbf{Treatment:} Folic acid 1--5 mg daily.
\end{itemize}

\subsection{Vitamin B12 (Cobalamin)}\label{vitamin-b12-cobalamin}

\begin{itemize}
\tightlist
\item
  \textbf{Function:} DNA synthesis, myelin formation.
\item
  \textbf{Deficiency:} megaloblastic anaemia, neurological deficits.
\item
  \textbf{Causes:} vegan diet, intrinsic factor deficiency, ileal
  malabsorption.
\item
  \textbf{Treatment:} Hydroxocobalamin 1 mg IM monthly after initial
  correction.
\end{itemize}

\section{Vitamin C Deficiency
(Scurvy)}\label{vitamin-c-deficiency-scurvy}

\subsection{Physiology}\label{physiology-3}

Vitamin C is essential for collagen synthesis, iron absorption, and
immune defence.

\subsection{Deficiency}\label{deficiency}

Occurs in children fed boiled milk, tea, or refined diets lacking fruits
and vegetables.

\subsection{Clinical Features}\label{clinical-features-68}

\begin{itemize}
\tightlist
\item
  Petechiae, purpura, and bleeding gums
\item
  Swollen, tender joints
\item
  Poor wound healing
\item
  Anaemia
\item
  Radiologic features: ``white line of Fraenkel'', ``Pelkan's spurs''
\end{itemize}

\subsection{Management}\label{management-74}

Oral vitamin C 100--300 mg/day for 1--2 weeks followed by dietary
correction.

\section{Summary Table}\label{summary-table-5}

\begin{longtable}[]{@{}
  >{\raggedright\arraybackslash}p{(\linewidth - 6\tabcolsep) * \real{0.1429}}
  >{\raggedright\arraybackslash}p{(\linewidth - 6\tabcolsep) * \real{0.2286}}
  >{\raggedright\arraybackslash}p{(\linewidth - 6\tabcolsep) * \real{0.2857}}
  >{\raggedright\arraybackslash}p{(\linewidth - 6\tabcolsep) * \real{0.3429}}@{}}
\toprule\noalign{}
\begin{minipage}[b]{\linewidth}\raggedright
Vitamin
\end{minipage} & \begin{minipage}[b]{\linewidth}\raggedright
Major Function
\end{minipage} & \begin{minipage}[b]{\linewidth}\raggedright
Deficiency Disease
\end{minipage} & \begin{minipage}[b]{\linewidth}\raggedright
Key Clinical Features
\end{minipage} \\
\midrule\noalign{}
\endhead
\bottomrule\noalign{}
\endlastfoot
A & Vision, epithelial integrity & Xerophthalmia & Night blindness,
Bitot's spots \\
D & Calcium metabolism & Rickets & Bone deformities, delayed
milestones \\
E & Antioxidant & Neuropathy, anaemia & Ataxia, haemolysis \\
K & Coagulation & HDN & Bleeding tendency \\
B1 & Energy metabolism & Beriberi & Heart failure, neuropathy \\
B2 & Redox reactions & Ariboflavinosis & Cheilosis, glossitis \\
B3 & NAD/NADP synthesis & Pellagra & Dermatitis, diarrhoea, dementia \\
B6 & Amino acid metabolism & Pyridoxine deficiency & Seizures,
anaemia \\
B9 & DNA synthesis & Megaloblastic anaemia & Anaemia, neural tube
defects \\
B12 & DNA synthesis & Pernicious anaemia & Anaemia, neuropathy \\
C & Collagen formation & Scurvy & Bleeding gums, bone pain \\
\end{longtable}

\section{Key Points}\label{key-points-1}

\begin{itemize}
\tightlist
\item
  Vitamin deficiencies often coexist with general malnutrition.
\item
  Preventive supplementation programs are cost-effective and lifesaving.
\item
  Recognition of subtle clinical features prevents irreversible
  complications such as blindness (Vitamin A) or neurological damage
  (B12).
\item
  Balanced diets rich in fruits, vegetables, whole grains, and animal
  products are the cornerstone of prevention.
\end{itemize}

\section{Suggested Reading}\label{suggested-reading-4}

\begin{enumerate}
\def\labelenumi{\arabic{enumi}.}
\tightlist
\item
  WHO. \emph{Guideline: Vitamin A Supplementation in Infants and
  Children 6--59 Months of Age}. 2011.\\
\item
  Nelson Textbook of Pediatrics, 22nd Edition.\\
\item
  UNICEF/WHO. \emph{Micronutrient Deficiencies in Children: Global
  Report}, 2022.\\
\item
  Gleason, G. R., \& Scrimshaw, N. S. \emph{Clinical Nutrition of the
  Young Child}. WHO, 2018.
\end{enumerate}

\part{{Dermatology}}

\chapter{Bacterial Skin Infections}\label{bacterial-skin-infections}

\section{Introduction}\label{introduction-63}

The skin is sterile at delivery and becomes colonised shortly after. The
normal skin of healthy infants and children is resistant to invasion by
most bacteria because the cutaneous surface serves as a dry mechanical
barrier.

\begin{itemize}
\item
  The \textbf{\emph{transient flora}} consists of multiple organisms
  that are deposited on the skin from the environment presumably, they
  do not proliferate and are removed easily by washing or scrubbing the
  affected area.
\item
  The \textbf{\emph{resident flora}} consists of a smaller number of
  organisms which are found regularly in appreciable numbers on the skin
  of normal individuals, multiply on the skin, form stable communities
  on the cutaneous surface, and are not easily dislodged.
\item
  \textbf{\emph{Pathogenic bacteria}}, not ordinarily a regular part of
  this flora, persist on the skin if there is a continuous replacement
  from some internal or external source or if the integrity of the skin
  is disrupted by injury or disease.
\end{itemize}

\emph{Staphylococcus aureus} forms part of the normal human flora. 10\%
to 30\% of individuals are nasal carriers of \emph{Staphylococcus
aureus} and 70\% to 90\% are transient carriers.

\section{Impetigo}\label{impetigo}

\textbf{\emph{Question}}:

A 4-year-old male child presents to the clinic with a one-week history
of a fluid-filled eruption at the chin which easily ruptures and spreads
to other areas. Mother complains his 2-year-old sister has also
developed similar lesions 2 days ago.

\begin{enumerate}
\def\labelenumi{\alph{enumi}.}
\tightlist
\item
  What is your most likely diagnosis?
\item
  How will you manage this family?
\end{enumerate}

\subsection{Definition \&
incidence/prevalence}\label{definition-incidenceprevalence}

Impetigo is a common, contagious superficial skin infection. It is seen
in all age groups, most common in infants and children. It can involve
any body surface but occurs most often on the exposed parts of the body,
especially the face, hands, neck and extremities.

\subsection{Aetiology}\label{aetiology-26}

It is caused by streptococci, staphylococci or both.

\subsection{Pathogenesis}\label{pathogenesis-6}

There are two classic forms

\begin{enumerate}
\def\labelenumi{\arabic{enumi}.}
\tightlist
\item
  \textbf{\emph{Non-bullous impetigo}} -- accounts for \textgreater70\%
  of cases. It is caused by S. aureus or S. pyogenes. Spread by direct
  contact or through fomites. It begins with a 1 to 2mm erythematous
  papule or pustule that soon develops into a thin-roofed vesicle or
  bulla surrounded by a narrow rim of erythema. The vesicle ruptures
  easily with the release of a thin, cloudy, yellow fluid that dries,
  forming a honey-coloured crust, the hallmark of non-bullous impetigo.
  S. pyogenes may cause post-streptococcal glomerulonephritis if the
  nephrogenic strain is involved. This is shown in
  Figure~\ref{fig-derm-non-bulous-impetigo}
\item
  \textbf{\emph{Bullous impetigo}} - This is always caused by
  Staphylococcus aureus. Presents as flaccid, thin-walled bullae or more
  commonly, tender, shallow erosions surrounded by a remnant of the
  blister roof. High reservoir sites include nasal carriage of
  asymptomatic persons and perineum. This is shown in
  Figure~\ref{fig-derm-bullous-impetigo}
\end{enumerate}

\begin{figure}

\centering{

\includegraphics[width=3.125in,height=\textheight,keepaspectratio]{images/derm-non-bullous-impetigo.jpg}

}

\caption{\label{fig-derm-non-bulous-impetigo}Non-Bullous Impetigo in a
Child}

\end{figure}%

\begin{figure}

\centering{

\pandocbounded{\includegraphics[keepaspectratio]{images/derm-bullous-impetigo.jpg}}

}

\caption{\label{fig-derm-bullous-impetigo}Bullous Impetigo in a child}

\end{figure}%

\subsection{Investigations}\label{investigations-49}

Diagnosis is usually made clinically, however, aspirate of the fluid
from the lesion can be sent for culture and sensitivity pattern in
extensive cases.

\subsection{Treatment}\label{treatment-19}

\begin{enumerate}
\def\labelenumi{\arabic{enumi}.}
\tightlist
\item
  Local skin care: cleansing, removal of crust
\item
  Mild and localised disease- topical antibiotics e.g.~Mupirocin
\item
  In severe or more widespread diseases -- oral antibiotics, e.g.:
  amoxicillin plus clavulanic acid, erythromycin
\end{enumerate}

\textbf{Note}: Staph eradication can be considered in people with
recurrent impetigo.

\section{Ecthyma}\label{ecthyma}

Ecthyma is a deep or ulcerative type of non-bullous impetigo. Commonly
seen on the lower extremities and buttocks of children. Lesions are
painful and heal slowly with scar formation. The initial lesion is a
vesiculopustular with an erythematous base and firmly adherent crust.
Removal of the crust reveals an underlying saucer-shaped ulcer and
raised margin.

\begin{figure}

\centering{

\includegraphics[width=3.64583in,height=\textheight,keepaspectratio]{images/derm-ecthyma.jpg}

}

\caption{\label{fig-derm-ecthyma}Ecthyma in a child}

\end{figure}%

\subsection{Treatment}\label{treatment-20}

This is with warm compresses and an appropriate systemic antibiotic.

\subsection{Complications}\label{complications-34}

\begin{itemize}
\tightlist
\item
  Cellulitis
\item
  Acute post-streptococcal glomerulonephritis in 5\% of cases of
  non-bullous impetigo by S. pyogenes (serotypes 1, 4,12, 25 and 49)
\end{itemize}

\subsection{Prognosis}\label{prognosis-38}

Impetigo heals well without scarring

\subsection{Differential diagnosis}\label{differential-diagnosis-28}

The most common differentials to consider include insect bite,
eczematous dermatoses and herpes simplex viral infection (HSV) in
non-bullous impetigo and thermal burns, bullous insect bite reactions
and HSV in bullous impetigo.

\chapter{Fungal Skin Infections}\label{fungal-skin-infections}

\chapter{Alopecia}\label{alopecia}

\chapter{Eczema}\label{eczema}

\chapter{Dermatitis}\label{dermatitis}

\chapter{Scabies}\label{scabies}

\section{Introduction}\label{introduction-64}

7 years 7-year-old male child presents to the outpatient unit with a
two-month history of an itchy rash. According to the mother, the rash is
recurrent.

\begin{enumerate}
\def\labelenumi{\Roman{enumi}.}
\tightlist
\item
  What relevant questions will you ask?
\item
  What is your diagnosis?
\item
  Outline your management plan for this patient
\end{enumerate}

\section{Definition \&
incidence/prevalence}\label{definition-incidenceprevalence-1}

It's a common pruritic skin infection, caused by an infestation of the
mite Sarcoptes scabiei var. hominis, which is an obligate parasite. The
pruritus associated with scabies is usually severe and more especially
at night.

\section{Epidemiology}\label{epidemiology-15}

It's a worldwide problem, with a significant public health burden that
is, it is very contagious. Prevalence is higher in children and those
who are sexually active. Environmental factors that hastens its spread
include:

\begin{enumerate}
\def\labelenumi{\arabic{enumi}.}
\tightlist
\item
  Overcrowding
\item
  Delayed treatment of primary cases
\item
  Lack of public awareness of the condition
\end{enumerate}

Transmission is by direct contact with infected persons or fomites
(beddings, clothing). The crusted scabies (formerly known as the
Norwegian scabies) are usually found in immunocompromised and
incapacitated individuals.

\section{Pathogenesis}\label{pathogenesis-7}

The incubation period can range from days to months. The whole life
cycle of the mite is between 30 to 60 days. It can live approximately 3
days or fewer of the human host but up to 7 days if from crusted
scabies. The female mite lays between 1 to 3 eggs a day, which takes
approximately 10 days to mature. In first-time infestations, it usually
takes 2 to 6 weeks before the host's immune system becomes sensitized to
the mite or its by-products, resulting in pruritus and cutaneous
lesions.

\section{Clinical presentation}\label{clinical-presentation-5}

Patients will present with symmetrical cutaneous lesions with intense
pruritus which is accentuated at night.~ Cutaneous lesions are usually
small erythematous papules with variable degrees of excoriations. Other
times, lesions may be vesicular, indurated nodules, eczematous
dermatitis and secondary bacterial infections. Burrows, which represent
tunnels a female mite excavates while laying eggs are pathognomonic for
scabies. Note: The distribution forms the basis of the clinical
diagnosis. Areas to look out for lesions include:

\begin{itemize}
\tightlist
\item
  the interdigital webbing of the hands
\item
  flexural aspects of the wrists
\item
  axillae
\item
  posterior auricular area
\item
  waist (including the umbilicus)
\item
  ankles
\item
  feet
\item
  buttocks
\item
  In men, check the penile and scrotum
\item
  In women, the areolae, nipples and vulvar area
\item
  In infants, the elderly and immunocompromised hosts, all skin surfaces
  are susceptible, including the scalp and face
\end{itemize}

Figure~\ref{fig-derm-scabies-interdigital},
Figure~\ref{fig-derm-scabies-hands},
Figure~\ref{fig-derm-scabies-abdomen},
Figure~\ref{fig-derm-scabies-wrist} show papules and some excoriations
at the typical distribution sites of scabetic lesions (interdigital web
spaces, along the phalanges, around umbilicus and wrist).
Figure~\ref{fig-derm-scabies-buttocks} shows crusting in the
intergluteal cleft, and Figure~\ref{fig-derm-scabies-genitals} depicts
papules and some crusting over the penile glans as well as scabetic
granulomas (nodules) on the scrotum of a child.

\begin{figure}

\centering{

\pandocbounded{\includegraphics[keepaspectratio]{images/derm-scabies-abdomen.jpg}}

}

\caption{\label{fig-derm-scabies-abdomen}Scabies showing lesion on the
abdomen}

\end{figure}%

\begin{figure}

\centering{

\pandocbounded{\includegraphics[keepaspectratio]{images/derm-scabies-buttocks.jpg}}

}

\caption{\label{fig-derm-scabies-buttocks}Scabies showing lesions on the
buttocks}

\end{figure}%

\begin{figure}

\centering{

\pandocbounded{\includegraphics[keepaspectratio]{images/derm-scabies-genitals.jpg}}

}

\caption{\label{fig-derm-scabies-genitals}Scabies showing lesions on the
genitals}

\end{figure}%

\begin{figure}

\centering{

\pandocbounded{\includegraphics[keepaspectratio]{images/derm-scabies-hands.jpg}}

}

\caption{\label{fig-derm-scabies-hands}Scabies showing lesions on the
hands}

\end{figure}%

\begin{figure}

\centering{

\pandocbounded{\includegraphics[keepaspectratio]{images/derm-scabies-interdigital.jpg}}

}

\caption{\label{fig-derm-scabies-interdigital}Scabies showing lesions in
the interdigital spaces}

\end{figure}%

\begin{figure}

\centering{

\pandocbounded{\includegraphics[keepaspectratio]{images/derm-scabies-wrist.jpg}}

}

\caption{\label{fig-derm-scabies-wrist}Scabies showing lesion on the
wrist}

\end{figure}%

\section{Investigations}\label{investigations-50}

Confirmation of scabies can be done by light microscopy of mineral oil
preparation of skin scrapings from an infested area to observe either
the mite, its eggs and or scybala (faeces). The skin scrappings can be
obtained by using a scalpel blade. Microscopic examination of adhesive
tape can also be done. Skin biopsy is rarely performed

\section{Treatment}\label{treatment-21}

Two topical treatments, one week apart of a prescribed anti-scabeitic to
be applied from the neck down. Leave on for 24 hours and wash after
that. Special attention is to be given to application in the
interdigital web spaces, umbilicus, genital and gluteal cleft. Treat all
close contacts at the same time. All bedding and clothing are to be
washed in hot water, sun dry and ironed to be safe for reuse. Available
anti-scabeitics in Ghana are permethrin 5\% cream and benzyl benzoate
25\%. Oral Ivermectin is reserved for treating crusted scabies

\section{Complications}\label{complications-35}

They are generally mild, including secondary bacterial infections, a
breach in the skin integrity, pain and reduced function due to pain.

\section{Prognosis}\label{prognosis-39}

If treated correctly, has a very good prognosis

\section{Differential diagnosis}\label{differential-diagnosis-29}

In the absence of a burrow and dermoscopically identifying the mite, the
differential diagnosis is quite broad including atopic dermatitis,
contact dermatitis, seborrhoiec dermatitis, arthropod bites, pyoderma
and bullous pemphigoid. In infants, severe infestation can resemble
Langerhans cell histiocytosis.

\part{{Home Accidents}}

\chapter{Iron Poisoning}\label{iron-poisoning}

\chapter{Paracetamol Poisoning}\label{paracetamol-poisoning}

\section{Introduction}\label{introduction-65}

Paracetamol (also known as acetaminophen) is one of the most commonly
used over-the-counter medications for fever and pain in children. While
generally safe at therapeutic doses, \textbf{paracetamol poisoning} can
lead to \textbf{life-threatening hepatotoxicity} if taken in excessive
amounts.

In Ghana and similar settings, where self-medication and delayed
health-seeking behaviors are common, cases of \textbf{accidental or
intentional ingestion} are frequently encountered. This makes it
essential for every medical student to understand the presentation,
pathophysiology, diagnosis, and management of paracetamol poisoning.

\section{Epidemiology}\label{epidemiology-16}

\begin{itemize}
\tightlist
\item
  Common cause of \textbf{drug overdose in children} globally.
\item
  In children under 6, most cases are \textbf{accidental}.
\item
  In adolescents, intentional ingestion may indicate \textbf{suicidal
  ideation}.
\item
  Easy availability and parental unawareness of correct dosing
  contribute to risk.
\item
  Liquid formulations (e.g., syrups) and tablets are common sources.
\end{itemize}

\section{Toxic Dose}\label{toxic-dose}

\textbf{Therapeutic dose:}

\begin{itemize}
\tightlist
\item
  \textbf{10--15 mg/kg per dose}, up to \textbf{60 mg/kg/day} in
  children.
\end{itemize}

\textbf{Toxic dose:}

\begin{itemize}
\tightlist
\item
  \textbf{Single ingestion of ≥150 mg/kg} in children is considered
  \textbf{potentially toxic}.
\item
  In neonates and infants, even \textbf{lower doses may cause toxicity}
  due to immature liver metabolism.
\end{itemize}

\section{Pathophysiology}\label{pathophysiology-43}

Paracetamol is primarily metabolized in the liver by

\begin{enumerate}
\def\labelenumi{\arabic{enumi}.}
\tightlist
\item
  \textbf{Glucuronidation} and \textbf{sulfation} → Non-toxic
  metabolites (90\%)
\item
  \textbf{Cytochrome P450 enzyme (CYP2E1)} → Minor pathway forms a
  \textbf{toxic metabolite} called \textbf{NAPQI
  (N-acetyl-p-benzoquinone imine)}
\end{enumerate}

\textbf{In therapeutic doses}

\begin{itemize}
\tightlist
\item
  NAPQI is rapidly detoxified by \textbf{glutathione}
\end{itemize}

\textbf{In overdose:}

\begin{itemize}
\tightlist
\item
  Glutathione stores are \textbf{depleted}
\item
  NAPQI accumulates → \textbf{Hepatocellular damage}, centrilobular
  necrosis
\item
  Severe cases can lead to \textbf{acute liver failure},
  \textbf{coagulopathy}, and \textbf{death}
\end{itemize}

\section{Clinical Features}\label{clinical-features-69}

Paracetamol toxicity has a \textbf{four-phase clinical progression}:

\textbf{Phase I (0--24 hours) -- \emph{Asymptomatic or mild GI
symptoms}}

\begin{itemize}
\tightlist
\item
  Nausea, vomiting
\item
  Anorexia
\item
  Pallor, lethargy
\item
  Most children appear \textbf{well}
\end{itemize}

\textbf{Phase II (24--72 hours) -- \emph{Hepatic injury begins}}

\begin{itemize}
\tightlist
\item
  Right upper quadrant pain or tenderness
\item
  Elevated liver enzymes (ALT, AST)
\item
  Prolonged prothrombin time (PT)
\item
  Oliguria (renal involvement)
\end{itemize}

\textbf{Phase III (72--96 hours) -- \emph{Maximum hepatotoxicity}}

\begin{itemize}
\tightlist
\item
  Jaundice
\item
  Hepatic encephalopathy
\item
  Bleeding (coagulopathy)
\item
  Hypoglycemia
\item
  Acute kidney injury
\item
  Multi-organ failure
\end{itemize}

\textbf{Phase IV (4--14 days) -- \emph{Recovery or death}}

\begin{itemize}
\tightlist
\item
  Complete recovery in survivors
\item
  Liver regeneration may take weeks
\end{itemize}

\section{Risk Factors for Severe
Poisoning}\label{risk-factors-for-severe-poisoning}

\begin{itemize}
\tightlist
\item
  \textbf{Delayed presentation (\textgreater8 hours)}
\item
  \textbf{Repeated supratherapeutic dosing}
\item
  Malnutrition (reduced glutathione reserves)
\item
  Pre-existing liver disease (e.g., hepatitis B)
\item
  Concomitant use of enzyme inducers (e.g., anti-TB drugs)
\end{itemize}

\section{Diagnosis}\label{diagnosis-31}

\textbf{Clinical history:}

\begin{itemize}
\tightlist
\item
  Time, dose, and formulation of paracetamol ingested
\item
  Number of tablets or volume of syrup
\item
  Co-ingestants (e.g., alcohol, antihistamines)
\item
  Symptoms since ingestion
\end{itemize}

\textbf{Physical examination}

\begin{itemize}
\tightlist
\item
  Vital signs, signs of hepatic dysfunction
\item
  Abdominal tenderness (RUQ)
\item
  Jaundice, mental status (encephalopathy)
\end{itemize}

\textbf{Laboratory investigations:}

\begin{itemize}
\tightlist
\item
  \textbf{Serum paracetamol level} (at 4 hours post-ingestion)
\item
  \textbf{Liver function tests (LFTs)}: ALT, AST, bilirubin
\item
  \textbf{Prothrombin time / INR}
\item
  \textbf{Serum creatinine and urea} (renal function)
\item
  \textbf{Blood glucose}
\item
  \textbf{Electrolytes}
\item
  \textbf{Arterial blood gas} (if acidosis suspected)
\end{itemize}

\section{Rumack-Matthew Nomogram}\label{rumack-matthew-nomogram}

Used to interpret \textbf{serum paracetamol levels} and determine need
for \textbf{antidote}.

\begin{itemize}
\tightlist
\item
  Plot serum paracetamol level against time since ingestion (only valid
  for \textbf{single acute ingestions})
\item
  A level \textbf{above the ``treatment line''} (150 µg/mL at 4 hours)
  indicates need for \textbf{N-acetylcysteine (NAC)}
\end{itemize}

Not applicable for

\begin{itemize}
\tightlist
\item
  Ingestion \textless4 hours or \textgreater24 hours ago
\item
  Repeated supratherapeutic ingestion
\item
  Extended-release formulations
\end{itemize}

\section{Management}\label{management-75}

\subsection{Initial Stabilization}\label{initial-stabilization-2}

\begin{enumerate}
\def\labelenumi{\arabic{enumi}.}
\tightlist
\item
  \textbf{Airway, Breathing, Circulation (ABCs)}
\item
  \textbf{Vital signs monitoring}
\item
  \textbf{IV access}
\item
  \textbf{Activated charcoal}:

  \begin{itemize}
  \tightlist
  \item
    Indicated if child presents \textbf{within 1 hour} of ingestion
  \item
    Dose: 1 g/kg (maximum 50 g)
  \item
    Only if \textbf{airway is protected}
  \end{itemize}
\end{enumerate}

\section{Antidote -- N-Acetylcysteine
(NAC)}\label{antidote-n-acetylcysteine-nac}

\textbf{Mechanism:}

\begin{itemize}
\tightlist
\item
  Replenishes \textbf{glutathione stores}
\item
  Enhances non-toxic metabolism of NAPQI
\end{itemize}

\textbf{Indications:}

\begin{itemize}
\tightlist
\item
  Serum level above treatment line on nomogram
\item
  Unknown ingestion time + raised LFTs
\item
  Clinical signs of hepatotoxicity
\item
  Suspected ingestion \textgreater150 mg/kg
\end{itemize}

\textbf{NAC Administration}

\textbf{IV Route (preferred in children):}

\begin{itemize}
\tightlist
\item
  \textbf{Loading dose}: 150 mg/kg over 1 hour
\item
  \textbf{Then}: 50 mg/kg over 4 hours
\item
  \textbf{Then}: 100 mg/kg over 16 hours (total = 300 mg/kg over 21
  hours)
\end{itemize}

\textbf{Oral Route (if IV not available):}

\begin{itemize}
\tightlist
\item
  140 mg/kg loading dose
\item
  Then 70 mg/kg every 4 hours × 17 doses (total = 1330 mg/kg over 72
  hours)
\item
  Unpalatable and may induce vomiting
\end{itemize}

\section{Supportive Care}\label{supportive-care-3}

\begin{itemize}
\tightlist
\item
  \textbf{IV fluids}: for dehydration or shock
\item
  \textbf{Glucose}: to prevent/treat hypoglycemia
\item
  \textbf{Vitamin K} or FFP: for coagulopathy
\item
  \textbf{Dialysis}: for renal failure or severe acidosis
\item
  \textbf{Liver transplant}: in fulminant hepatic failure (not readily
  available in Ghana)
\end{itemize}

\section{Monitoring}\label{monitoring-4}

\begin{itemize}
\tightlist
\item
  \textbf{Liver enzymes} every 12--24 hours
\item
  \textbf{INR}, \textbf{glucose}, \textbf{renal function}
\item
  Mental status (for encephalopathy)
\item
  Continue NAC until clinical and biochemical improvement
\end{itemize}

\section{Disposition}\label{disposition}

\begin{longtable}[]{@{}
  >{\raggedright\arraybackslash}p{(\linewidth - 2\tabcolsep) * \real{0.4815}}
  >{\raggedright\arraybackslash}p{(\linewidth - 2\tabcolsep) * \real{0.5185}}@{}}
\toprule\noalign{}
\endhead
\bottomrule\noalign{}
\endlastfoot
\textbf{Scenario} & \textbf{Action} \\
Ingestion \textless150 mg/kg, asymptomatic & Observe for 4--6 hours,
discharge if well \\
Ingestion \textgreater150 mg/kg, \textless8 hours ago & Begin NAC \\
Ingestion \textgreater24 hours ago, symptomatic & Treat as hepatic
injury, give NAC \\
Intentional overdose & Admit, psychiatric evaluation \\
\end{longtable}

\section{Prevention Strategies}\label{prevention-strategies}

\begin{itemize}
\tightlist
\item
  \textbf{Public education}: About correct pediatric dosing
\item
  \textbf{Child-proof containers}
\item
  \textbf{Proper labeling} of medications
\item
  \textbf{Avoid overlapping medications} with paracetamol content (e.g.,
  cold and pain meds)
\item
  Educate parents to \textbf{seek medical attention early} after
  overdose
\end{itemize}

\section{Special Considerations in
Ghana}\label{special-considerations-in-ghana-2}

\begin{itemize}
\tightlist
\item
  Limited access to \textbf{serum paracetamol assays}: clinical judgment
  and reported dose guide treatment
\item
  NAC may not always be readily available -- advocate for stocking in
  \textbf{district and regional hospitals}
\item
  \textbf{Traditional medicines} may contain unknown paracetamol content
  -- careful history is important
\item
  Early transfer to \textbf{higher-level facilities} for severe cases
\item
  Community education on \textbf{risks of overmedication and
  self-medication}
\end{itemize}

\section{Case Scenario}\label{case-scenario-1}

\textbf{Case 1: 3-year-old girl}

\textbf{Presentation:}

\begin{itemize}
\tightlist
\item
  Mother reports child accidentally drank 10 teaspoons of paracetamol
  syrup (120 mg/5 ml) ≈ 1200 mg
\item
  Weight: 12 kg → Toxic dose = 1800 mg (150 mg/kg)
\item
  Child appears well; ingestion occurred 2 hours ago
\end{itemize}

\textbf{Action:}

\begin{itemize}
\tightlist
\item
  Calculate dose: 1200 mg ÷ 12 kg = 100 mg/kg
\item
  Below toxic threshold → Observe for 4--6 hours
\item
  No NAC needed
\item
  Educate mother on correct dosing
\end{itemize}

\textbf{Case 2: 13-year-old girl (suicidal ingestion)}

\textbf{Presentation:}

\begin{itemize}
\tightlist
\item
  Took 20 x 500 mg tablets (10,000 mg) 4 hours ago
\item
  Weight: 40 kg → Dose = 250 mg/kg (toxic)
\item
  Nausea, abdominal pain
\item
  Start IV NAC
\item
  Check LFTs, INR, glucose
\item
  Admit and arrange psychiatric evaluation
\end{itemize}

\section{Summary Table}\label{summary-table-6}

\begin{longtable}[]{@{}ll@{}}
\toprule\noalign{}
\endhead
\bottomrule\noalign{}
\endlastfoot
\textbf{Parameter} & \textbf{Value} \\
Safe dose & 10--15 mg/kg/dose \\
Maximum daily dose & 60 mg/kg/day \\
Toxic dose & \textgreater150 mg/kg (single ingestion) \\
Antidote & N-acetylcysteine (NAC) \\
NAC IV dose & 300 mg/kg over 21 hours \\
Serum level interpretation & Rumack-Matthew nomogram \\
Onset of liver injury & 24--72 hours post-ingestion \\
Outcome & Excellent if treated early \\
\end{longtable}

\section{Conclusion}\label{conclusion-33}

Paracetamol poisoning is \textbf{common, preventable}, and
\textbf{potentially fatal} if not recognized and treated early. Prompt
administration of \textbf{N-acetylcysteine} can prevent liver failure
even in significant overdoses. In Ghana, clinical assessment remains the
cornerstone due to limited laboratory resources. Medical students must
be vigilant, advocate for early treatment, and educate caregivers on
safe medication practices.

\chapter{Insecticide Poisoning}\label{insecticide-poisoning}

\chapter{Caustic Soda Ingestion}\label{caustic-soda-ingestion}

\chapter{Hydrocarbons}\label{hydrocarbons}

\chapter{Electrical Injuries}\label{electrical-injuries}

\chapter{Drowning}\label{drowning}

\part{{Animal Bites}}

\chapter{Snake Bites}\label{snake-bites}

\section{Introduction}\label{introduction-66}

Snake bites are a significant public health concern in many parts of the
world, particularly in rural, tropical, and subtropical regions. Ghana,
with its mix of rural populations and agriculture-based livelihoods, is
home to both venomous and non-venomous snakes. Children are particularly
vulnerable due to their small body size, curious nature, and increased
exposure during play or farm-related activities. In children,
envenomation may have more severe effects compared to adults,
necessitating prompt and effective clinical management.

This note aims to provide medical students in Ghana with a comprehensive
overview of snake bites in children, covering epidemiology,
pathophysiology, clinical features, diagnosis, management,
complications, and preventive strategies.

\section{Epidemiology}\label{epidemiology-17}

\begin{itemize}
\tightlist
\item
  \textbf{Global Burden:} The World Health Organization (WHO) estimates
  over 5 million snake bites per year globally, leading to over 100,000
  deaths and 400,000 amputations or other permanent disabilities.
\item
  \textbf{Africa and Ghana:} Sub-Saharan Africa accounts for a
  significant portion of these bites. In Ghana, rural areas, including
  the Northern, Upper East, Volta, Ashanti, and Brong-Ahafo regions,
  report high incidences of snakebites.
\item
  \textbf{Age Distribution:} Children under 15 years are at high risk,
  especially boys who may assist in farming or play outdoors.
\end{itemize}

\section{Common Venomous Snakes in
Ghana}\label{common-venomous-snakes-in-ghana}

In Ghana, the major venomous snakes include:

\begin{longtable}[]{@{}
  >{\raggedright\arraybackslash}p{(\linewidth - 6\tabcolsep) * \real{0.2500}}
  >{\raggedright\arraybackslash}p{(\linewidth - 6\tabcolsep) * \real{0.2500}}
  >{\raggedright\arraybackslash}p{(\linewidth - 6\tabcolsep) * \real{0.2500}}
  >{\raggedright\arraybackslash}p{(\linewidth - 6\tabcolsep) * \real{0.2500}}@{}}
\toprule\noalign{}
\endhead
\bottomrule\noalign{}
\endlastfoot
\textbf{Snake Family} & \textbf{Example} & \textbf{Type of Venom} &
\textbf{Effects} \\
\textbf{Elapidae} & Cobra (e.g., \emph{Naja nigricollis}) & Neurotoxic &
Paralysis, respiratory failure \\
& Mamba (\emph{Dendroaspis} spp.) & Neurotoxic & Rapid-onset
paralysis \\
\textbf{Viperidae} & Puff Adder (\emph{Bitis arietans}) & Cytotoxic and
Hemotoxic & Swelling, bleeding, necrosis \\
& Gaboon Viper (\emph{Bitis gabonica}) & Hemotoxi & Severe local
effects, shock \\
\end{longtable}

Non-venomous snakes also exist and are often mistaken for dangerous
species, leading to unnecessary anxiety and medical attention.

\section{Pathophysiology of Snake
Envenomation}\label{pathophysiology-of-snake-envenomation}

\textbf{Types of Venom}

\begin{enumerate}
\def\labelenumi{\arabic{enumi}.}
\tightlist
\item
  \textbf{Neurotoxic Venom (e.g., mambas, cobras):}

  \begin{itemize}
  \item
    Acts on neuromuscular junctions
  \item
    Leads to paralysis and respiratory arrest
  \end{itemize}
\item
  \textbf{Cytotoxic Venom (e.g., puff adder):}

  \begin{itemize}
  \tightlist
  \item
    Causes tissue destruction and necrosis
  \item
    Leads to local swelling, blistering, and potential amputation
  \end{itemize}
\item
  \textbf{Hemotoxic Venom (e.g., vipers)}

  \begin{itemize}
  \tightlist
  \item
    Disrupts blood clotting
  \item
    Causes internal bleeding, hypotension, and shoc
  \end{itemize}
\end{enumerate}

\textbf{Factors Influencing Severity in Children}

\begin{itemize}
\tightlist
\item
  Smaller body mass increases the venom-to-weight ratio
\item
  Delayed presentation to healthcare
\item
  Inappropriate first aid (e.g., tourniquets)
\end{itemize}

\section{Clinical Features}\label{clinical-features-70}

\textbf{Local Effects}

\begin{itemize}
\tightlist
\item
  Pain and swelling at the bite site
\item
  Bruising and blister formation
\item
  Tissue necrosis (especially with cytotoxic venom)
\item
  Fang marks (may be single or double puncture wounds)
\end{itemize}

\textbf{Systemic Effects}

\textbf{Neurotoxic Bites:}

\begin{itemize}
\tightlist
\item
  Ptosis (drooping eyelids)
\item
  Dysphagia (difficulty swallowing)
\item
  Respiratory distress
\item
  Flaccid paralysis
\end{itemize}

\textbf{Hemotoxic Bites:}

\begin{itemize}
\tightlist
\item
  Bleeding from gums, nose, or bite site
\item
  Hematuria
\item
  Hypotension and shock
\item
  Disseminated intravascular coagulation (DIC)
\end{itemize}

\textbf{Other Signs:}

\begin{itemize}
\tightlist
\item
  Fever
\item
  Vomiting
\item
  Abdominal pain
\item
  Shock
\end{itemize}

\section{First Aid and Pre-Hospital
Management}\label{first-aid-and-pre-hospital-management}

\textbf{Do's:}

\begin{itemize}
\tightlist
\item
  Keep the child calm and still (to slow the venom spread)
\item
  Immobilize the bitten limb using a splint
\item
  Remove any tight clothing or jewelry near the bite
\item
  Transport to a healthcare facility quickly
\end{itemize}

\textbf{Don'ts}

\begin{itemize}
\tightlist
\item
  Do not apply a tourniquet
\item
  Do not suck the venom
\item
  Do not cut the wound
\item
  Do not apply traditional medicines
\end{itemize}

\section{Hospital Evaluation and
Diagnosis}\label{hospital-evaluation-and-diagnosis}

History\textbf{:}

\begin{itemize}
\tightlist
\item
  Time and place of bite
\item
  Activity at the time of bite
\item
  Description or photo of the snake (if available)
\item
  Symptoms since the bite
\end{itemize}

Physical Examination\textbf{:}

\begin{itemize}
\tightlist
\item
  Vital signs: hypotension, tachypnea, hypoxia
\item
  Site inspection: swelling, necrosis, fang marks
\item
  Neurologic assessment: cranial nerves, motor strength
\item
  Bleeding manifestations
\end{itemize}

Laboratory Investigations\textbf{:}

\begin{itemize}
\tightlist
\item
  Full blood count (for anemia, leukocytosis, thrombocytopenia)
\item
  Clotting profile (PT, aPTT, 20-minute whole blood clotting test)
\item
  Renal function tests (serum urea, creatinine)
\item
  Urinalysis (hematuria, myoglobinuria)
\item
  Crossmatch for transfusion if necessary
\end{itemize}

\textbf{20-Minute Whole Blood Clotting Test (20WBCT):}

\begin{itemize}
\tightlist
\item
  Simple bedside test using a clean, dry glass tube
\item
  Failure to clot within 20 minutes suggests coagulopathy (common in
  viper bites)
\end{itemize}

\section{Antivenom Therapy}\label{antivenom-therapy}

\textbf{Indications:}

\begin{itemize}
\tightlist
\item
  Rapidly progressive swelling
\item
  Systemic signs: neurotoxicity, coagulopathy, hypotension
\item
  Evidence of hemolysis or bleeding
\item
  Children with severe pain or systemic deterioration
\end{itemize}

\textbf{Types of Antivenom:}

\begin{itemize}
\tightlist
\item
  \textbf{Polyvalent antivenoms} are commonly used in Ghana and are
  effective against several species.
\item
  Supplied by institutions like the Ministry of Health and the WHO.
\end{itemize}

\textbf{Administrations}

\begin{itemize}
\tightlist
\item
  Test for hypersensitivity (some centers do not recommend skin testing)
\item
  Administer IV over 30-60 minutes
\item
  Monitor for anaphylaxis (rash, bronchospasm, hypotension)
\end{itemize}

\textbf{Side Effects:}

\begin{itemize}
\tightlist
\item
  Early reactions: urticaria, itching, anaphylaxis
\item
  Late reactions: serum sickness (fever, rash, arthritis)
\end{itemize}

\section{Supportive Management}\label{supportive-management-1}

\begin{itemize}
\tightlist
\item
  \textbf{Airway and Breathing:} Intubation and ventilation if
  neurotoxic paralysis occurs
\item
  \textbf{Circulation:} IV fluids for shock, blood transfusions for
  anemia or coagulopathy
\item
  \textbf{Pain Control:} Paracetamol; avoid NSAIDs due to bleeding risk
\item
  \textbf{Tetanus Prophylaxis}
\item
  \textbf{Antibiotics:} Only if signs of secondary infection; snakebite
  wounds are generally not sterile
\item
  \textbf{Wound Care:} Debridement if necrosis develops; monitor for
  compartment syndrome
\end{itemize}

\section{Complications}\label{complications-36}

\begin{itemize}
\item
  \textbf{Acute Kidney Injury (AKI):} Hemoglobinuria or hypotension may
  lead to renal damage
\item
  \textbf{Compartment Syndrome:} Due to excessive swelling; requires
  surgical fasciotomy
\item
  \textbf{Limb Loss:} From severe necrosis or gangrene
\item
  \textbf{Shock:} From venom or sepsis
\item
  \textbf{Chronic Sequelae:}

  \begin{itemize}
  \tightlist
  \item
    Disfigurement
  \item
    Reduced limb function
  \item
    Psychosocial issues
  \end{itemize}
\end{itemize}

\section{Prognosis}\label{prognosis-40}

\begin{itemize}
\tightlist
\item
  Prognosis depends on:

  \begin{itemize}
  \tightlist
  \item
    Type of snake
  \item
    Time to hospital presentation
  \item
    Child's nutritional status and comorbidities
  \end{itemize}
\item
  Early intervention greatly improves outcomes.
\item
  Mortality rates can be significantly reduced with appropriate
  antivenom and supportive care.
\end{itemize}

\section{Prevention Strategies}\label{prevention-strategies-1}

\textbf{Community Education:}

\begin{itemize}
\tightlist
\item
  Teach children to avoid snake-infested areas
\item
  Use of protective clothing (boots, gloves)
\item
  Awareness about seeking early medical care
\end{itemize}

\textbf{Environmental Measures:}

\begin{itemize}
\tightlist
\item
  Keep surroundings clear of bushes and rodents (which attract snakes)
\item
  Use of mosquito nets (many bites occur at night)
\end{itemize}

\textbf{Government and Health Policy}

\begin{itemize}
\tightlist
\item
  Ensure a consistent supply of antivenoms
\item
  Train rural healthcare workers in snakebite management
\item
  Integrate snakebite education in school curricula
\end{itemize}

\section{Special Considerations in
Ghana}\label{special-considerations-in-ghana-3}

\begin{itemize}
\tightlist
\item
  Traditional beliefs often delay hospital treatment
\item
  Transportation difficulties in rural areas
\item
  Cost of antivenom, though often subsidized, remains a barrier
\item
  Lack of access to ventilators in some facilities hinders care for
  neurotoxic bites
\end{itemize}

\section{Case Study (Example)}\label{case-study-example}

\textbf{Patient:} 7-year-old boy from Brong-Ahafo\\
\textbf{Presentation:} Bitten on the left foot while walking through the
grass to school\\
\textbf{Symptoms:} Pain, swelling up to the knee, bleeding from the
gums\\
\textbf{Findings:} Fang marks present, 20WBCT abnormal, hematuria\\
\textbf{Management:}

\begin{itemize}
\item
  IV fluids
\item
  Polyvalent antivenom (2 vials)
\item
  Paracetamol for pain
\item
  Close monitoring in the pediatric ward

  \textbf{Outcome:} Swelling reduced by day 3, discharged on day 5 with
  normal clotting time
\end{itemize}

\section{Summary}\label{summary-5}

\begin{itemize}
\tightlist
\item
  Snake bites are a medical emergency in children, especially in rural
  Ghana.
\item
  Prompt immobilization and transportation to a health facility are
  critical.
\item
  Antivenom is the cornerstone of treatment for venomous bites.
\item
  Supportive care and complication management improve survival and
  reduce the risk of disability.
\item
  Preventive education and community engagement are essential.
\end{itemize}

\section{Key Points}\label{key-points-2}

\begin{enumerate}
\def\labelenumi{\arabic{enumi}.}
\tightlist
\item
  Always suspect a venomous bite in a symptomatic child from an endemic
  area.
\item
  Do not delay administering antivenom if systemic signs or rapid local
  progression are present.
\item
  Monitor vital signs and watch for early and late antivenom reactions.
\item
  Avoid outdated or potentially harmful first-aid practices, such as
  using tourniquets or making incisions.
\item
  Educate families and communities about snakebite prevention, early
  care, and treatment.
\end{enumerate}

\chapter{Dog Bites}\label{dog-bites}

\chapter{Scorpion Stings}\label{scorpion-stings}

\part{{Common Syndrome}}

\chapter{Down Syndrome}\label{down-syndrome}

\section{Introduction}\label{introduction-67}

Down syndrome, also referred to as Trisomy 21, is the most common
chromosomal disorder, characterized by a range of physical,
developmental, and cognitive abnormalities. First described by John
Langdon Down in 1866, it occurs due to an extra copy of chromosome 21,
which results in overexpression of the genes on this chromosome.
Affecting approximately 1 in 700 live births worldwide, Down syndrome is
seen across all racial, ethnic, and socioeconomic groups. This condition
presents with varying degrees of intellectual disability, distinctive
physical features, and a predisposition to certain medical conditions.
Advances in prenatal screening, early interventions, and supportive care
have significantly improved the quality of life and life expectancy for
individuals with Down syndrome.

\section{\texorpdfstring{\textbf{Definition}}{Definition}}\label{definition-39}

Down syndrome is a congenital chromosomal disorder caused by trisomy 21,
resulting from an extra chromosome 21 in whole or part. It is a
multisystem condition affecting physical development, cognitive
abilities, and susceptibility to specific medical disorders. Based on
the genetic cause, Down syndrome is classified into three types:

\begin{enumerate}
\def\labelenumi{\arabic{enumi}.}
\item
  Trisomy 21 (95\%): A result of nondisjunction during meiosis, leading
  to an extra chromosome 21 in all cells.
\item
  Translocation (4\%): A structural rearrangement where a part of
  chromosome 21 is attached to another chromosome, often chromosome 14.
\item
  Mosaicism (1\%): A mixture of normal and trisomic cells due to a
  post-zygotic nondisjunction event.
\end{enumerate}

\section{\texorpdfstring{\textbf{Genetics}}{Genetics}}\label{genetics}

The extra genetic material from chromosome 21 disrupts normal
development and leads to the phenotypic features of Down syndrome.
Chromosome 21 is one of the smallest autosomes and contains around
200--300 genes. Overexpression of key genes contributes to the
syndrome's clinical manifestations. These include:

\begin{enumerate}
\def\labelenumi{\arabic{enumi}.}
\item
  DYRK1A: Implicated in cognitive impairment.
\item
  APP (Amyloid Precursor Protein): Associated with early-onset
  Alzheimer's disease.
\item
  SOD1 (Superoxide Dismutase 1): Contributes to oxidative stress.
\item
  ETS2: May influence craniofacial development.
\end{enumerate}

Advanced maternal age is a significant risk factor due to increased
nondisjunction during oocyte maturation. The risk of Down syndrome
increases with maternal age, from about 1 in 1,500 at age 20 to 1 in 100
at age 40.

\section{\texorpdfstring{\textbf{Clinical
Features}}{Clinical Features}}\label{clinical-features-71}

Individuals with Down syndrome exhibit a characteristic phenotype along
with a range of medical and developmental complications. These include:

\begin{enumerate}
\def\labelenumi{\arabic{enumi}.}
\item
  Craniofacial Features:

  \begin{itemize}
  \item
    Flat facial profile and nasal bridge
  \item
    Brachycephaly (short and broad skull)
  \item
    Upward-slanting palpebral fissures
  \item
    Epicanthal folds
  \item
    Low-set, small ears
  \item
    Protruding tongue and small oral cavity
  \end{itemize}
\item
  Musculoskeletal Features:

  \begin{itemize}
  \item
    Hypotonia (generalized low muscle tone)
  \item
    Short stature
  \item
    Joint hypermobility
  \item
    Single transverse palmar crease (Simian crease)
  \item
    Clinodactyly (inward curvature of the fifth finger)
  \end{itemize}
\item
  Neurological and Cognitive Features:

  \begin{itemize}
  \item
    Mild to moderate intellectual disability (average IQ: 50--70)
  \item
    Delayed developmental milestones (e.g., sitting, walking, and
    speech)
  \item
    Early-onset Alzheimer's disease (in 50\% by their 50s)
  \end{itemize}
\item
  Congenital Anomalies:

  \begin{itemize}
  \item
    Cardiac: Approximately 50\% have congenital heart defects, primarily
    atrioventricular septal defects (AVSD) and ventricular septal
    defects (VSD).
  \item
    Gastrointestinal: Duodenal atresia, Hirschsprung disease, and
    annular pancreas are common
  \end{itemize}
\item
  Medical Comorbidities:

  \begin{itemize}
  \item
    Endocrine: Hypothyroidism is frequently observed.
  \item
    Hematological: Increased risk of leukemia, particularly acute
    lymphoblastic leukemia (ALL) and acute megakaryoblastic leukemia
    (AMKL).
  \item
    Immunological: Impaired immune responses lead to recurrent
    infections.
  \item
    Respiratory: Obstructive sleep apnea due to airway abnormalities.
  \item
    Ophthalmic: Cataracts, refractive errors, and strabismus.
  \item
    Auditory: Conductive and sensorineural hearing loss.
  \end{itemize}
\end{enumerate}

\section{\texorpdfstring{\textbf{Investigations}}{Investigations}}\label{investigations-51}

Diagnosis of Down syndrome can be made prenatally or postnatally using a
combination of clinical evaluation and laboratory tests.

\begin{enumerate}
\def\labelenumi{\arabic{enumi}.}
\item
  Prenatal Screening:

  \begin{itemize}
  \item
    Non-Invasive Methods:

    \begin{itemize}
    \tightlist
    \item
      First-Trimester Screening: Combines maternal serum markers (β-hCG
      and PAPP-A) with nuchal translucency measurement via ultrasound.
    \item
      Second-Trimester Screening: Measures serum markers like AFP,
      β-hCG, estriol, and inhibin-A (quadruple test).
    \item
      Non-Invasive Prenatal Testing (NIPT): Analyzes cell-free fetal DNA
      in maternal blood with high sensitivity and specificity.
    \end{itemize}
  \item
    Invasive Diagnostic Methods:

    \begin{itemize}
    \tightlist
    \item
      Chorionic Villus Sampling (CVS): Performed at 10--13 weeks of
      gestation.
    \item
      Amniocentesis: Performed after 15 weeks of gestation.
    \item
      Cordocentesis: Used for advanced gestational age
    \end{itemize}
  \end{itemize}
\item
  Postnatal Diagnosis:

  \begin{itemize}
  \tightlist
  \item
    Clinical Evaluation: Characteristic physical features and hypotonia
    are noted.
  \item
    Cytogenetic Testing: Karyotyping confirms the diagnosis by
    identifying trisomy 21. Fluorescence in situ hybridization (FISH) or
    chromosomal microarray analysis provides rapid results.
  \end{itemize}
\item
  Additional Investigations:

  \begin{itemize}
  \item
    Echocardiogram to detect congenital heart defects.
  \item
    Thyroid function tests (TSH and free T4) to screen for
    hypothyroidism.
  \item
    Complete blood count (CBC) for hematologic abnormalities.
  \item
    Hearing and vision assessments.
  \item
    Polysomnography for suspected obstructive sleep apnea.
  \end{itemize}
\end{enumerate}

\section{\texorpdfstring{\textbf{Treatment}}{Treatment}}\label{treatment-22}

There is no cure for Down syndrome, but early intervention, medical
management, and supportive care can significantly improve outcomes.

\begin{enumerate}
\def\labelenumi{\arabic{enumi}.}
\item
  Developmental Interventions:

  \begin{itemize}
  \item
    Early therapy programs, including speech, occupational, and physical
    therapy.
  \item
    Special education tailored to cognitive abilities and needs.
  \end{itemize}
\item
  Medical Management:

  \begin{itemize}
  \item
    Surgical correction of congenital anomalies (e.g., AVSD or duodenal
    atresia).
  \item
    Thyroid hormone replacement for hypothyroidism.
  \item
    Antibiotics for recurrent infections.
  \item
    Management of obstructive sleep apnea with CPAP or
    adenotonsillectomy.
  \item
    Regular follow-up for screening and management of complications like
    leukemia or Alzheimer's disease.
  \end{itemize}
\item
  Psychosocial Support:

  \begin{itemize}
  \item
    Enabling social inclusion through community programs.
  \item
    Support groups for families and caregivers.
  \item
    Promoting independent living skills in adulthood
  \end{itemize}
\end{enumerate}

\section{\texorpdfstring{\textbf{Counselling}}{Counselling}}\label{counselling}

Comprehensive genetic and psychosocial counseling is essential for
families of individuals with Down syndrome.

\begin{enumerate}
\def\labelenumi{\arabic{enumi}.}
\tightlist
\item
  Prenatal Counselling:

  \begin{itemize}
  \tightlist
  \item
    Educate expectant parents about the condition, available diagnostic
    methods, and outcomes.
  \item
    Provide psychological support to help them make informed decisions
    regarding pregnancy continuation or termination.
  \end{itemize}
\item
  Postnatal Counseling:

  \begin{itemize}
  \tightlist
  \item
    Offer guidance on medical care, developmental milestones, and
    prognosis.
  \item
    Address the emotional and social concerns of the family.
  \item
    Encourage joining support groups and advocacy networks.
  \end{itemize}
\item
  Genetic Counseling:

  \begin{itemize}
  \tightlist
  \item
    For translocation Down syndrome, parents should undergo karyotyping
    to determine if one is a carrier of a balanced translocation, which
    increases the recurrence risk in future pregnancies.
  \end{itemize}
\end{enumerate}

\section{\texorpdfstring{\textbf{Conclusion}}{Conclusion}}\label{conclusion-34}

Down syndrome is a complex chromosomal disorder requiring a
multidisciplinary approach to diagnosis, treatment, and management.
Advances in prenatal screening and medical care have improved the life
expectancy and quality of life for individuals with Down syndrome.
However, ongoing research into the genetic and molecular basis of the
condition holds promise for novel therapeutic interventions. Holistic
care, including medical, developmental, and psychosocial support, is
crucial in empowering individuals with Down syndrome to lead fulfilling
lives.

\section{Review Schedule}\label{review-schedule}

\begin{longtable}[]{@{}
  >{\raggedright\arraybackslash}p{(\linewidth - 4\tabcolsep) * \real{0.3300}}
  >{\raggedright\arraybackslash}p{(\linewidth - 4\tabcolsep) * \real{0.3300}}
  >{\raggedright\arraybackslash}p{(\linewidth - 4\tabcolsep) * \real{0.3300}}@{}}
\toprule\noalign{}
\endhead
\bottomrule\noalign{}
\endlastfoot
\textbf{Age:} \emph{Focus Areas} & \textbf{Assessments \& Follow-Up} &
\textbf{Referral/Intervention} \\
\textbf{Birth -- 1 Month:} \emph{Initial Screening} &
\begin{minipage}[t]{\linewidth}\raggedright
\begin{itemize}
\tightlist
\item
  Confirm karyotype
\item
  Physical exam for congenital anomalies
\item
  Cardiac exam (Echo)
\item
  Hearing screen- Vision screen- Thyroid (TSH/T4)
\item
  Feeding and weight monitoring
\end{itemize}
\end{minipage} & \begin{minipage}[t]{\linewidth}\raggedright
\begin{itemize}
\tightlist
\item
  Refer to cardiology if CHD is suspected
\item
  Audiology
\item
  Endocrine if thyroid abnormal
\item
  Genetic counselling
\end{itemize}
\end{minipage} \\
\textbf{1 -- 6 Months:} Early Development &
\begin{minipage}[t]{\linewidth}\raggedright
\begin{itemize}
\tightlist
\item
  Growth (weight, length, head circumference)
\item
  Neurodevelopment
\item
  ENT exam
\item
  Repeat hearing if needed
\item
  Monitor for feeding issues
\end{itemize}
\end{minipage} & \begin{minipage}[t]{\linewidth}\raggedright
\begin{itemize}
\tightlist
\item
  Physiotherapy
\item
  Feeding team
\item
  Early intervention programs
\end{itemize}
\end{minipage} \\
\textbf{6 -- 12 Months:} Development \& Screening &
\begin{minipage}[t]{\linewidth}\raggedright
\begin{itemize}
\tightlist
\item
  Developmental milestone check
\item
  ENT/hearing follow-up
\item
  Vision assessment
\item
  Thyroid screen at 6 months
\item
  Monitor for constipation, reflux
\end{itemize}
\end{minipage} & \begin{minipage}[t]{\linewidth}\raggedright
\begin{itemize}
\tightlist
\item
  Refer to ophthalmology if strabismus or visual issues
\item
  OT/speech therapy
\end{itemize}
\end{minipage} \\
\textbf{1 -- 2 Years:} Cognitive \& Physical Growth &
\begin{minipage}[t]{\linewidth}\raggedright
\begin{itemize}
\tightlist
\item
  Annual thyroid (TSH/T4)
\item
  Hearing test every 6 months
\item
  Vision annually
\item
  CBC if not yet done (risk of leukemia)
\item
  Monitor sleep apnea
\end{itemize}
\end{minipage} & \begin{minipage}[t]{\linewidth}\raggedright
\begin{itemize}
\tightlist
\item
  ENT for sleep apnea
\item
  Nutrition/dietitian
\item
  Developmental therapist
\end{itemize}
\end{minipage} \\
\textbf{2 -- 5 Years:} School Readiness &
\begin{minipage}[t]{\linewidth}\raggedright
\begin{itemize}
\tightlist
\item
  Annual thyroid
\item
  Hearing test annually
\item
  Vision annually
\item
  Growth charts for DS
\item
  Screen for ASD/ADHD
\item
  Dental exam every 6 months
\item
  Behavioral assessment
\end{itemize}
\end{minipage} & \begin{minipage}[t]{\linewidth}\raggedright
\begin{itemize}
\tightlist
\item
  Enroll in an inclusive preschool
\item
  Speech \& language therapy
\item
  Audiologist, ENT, dentist
\end{itemize}
\end{minipage} \\
\textbf{5 -- 10 Years:} Academic, Social, and Health Monitoring &
\begin{minipage}[t]{\linewidth}\raggedright
\begin{itemize}
\tightlist
\item
  Thyroid annually
\item
  Vision \& hearing annually
\item
  Dental care every 6 months
\item
  Check for atlantoaxial instability (C-spine X-ray before sports)
\item
  Behaviour/school performance
\item
  Sleep \& snoring screening
\end{itemize}
\end{minipage} & \begin{minipage}[t]{\linewidth}\raggedright
\begin{itemize}
\tightlist
\item
  Psychologist or special educator
\item
  Sleep study if suspected OSA
\item
  Orthopedic review for signs of instability
\end{itemize}
\end{minipage} \\
\textbf{10 -- 13 Years:} Puberty \& Social Development &
\begin{minipage}[t]{\linewidth}\raggedright
\begin{itemize}
\tightlist
\item
  Annual thyroid
\item
  Vision, hearing and , dental
\item
  Growth and puberty assessment
\item
  Discuss menstrual hygiene (girls)
\item
  Sexual education (age-appropriate)- Screen for depression/anxiety
\end{itemize}
\end{minipage} & \begin{minipage}[t]{\linewidth}\raggedright
\begin{itemize}
\tightlist
\item
  Adolescent counselor
\item
  Gynecology referral if needed
\end{itemize}
\end{minipage} \\
\textbf{13 -- 16 Years:} Transition Planning &
\begin{minipage}[t]{\linewidth}\raggedright
\begin{itemize}
\tightlist
\item
  Continue yearly thyroid, vision, and hearing
\item
  Discuss vocational goals
\item
  Encourage independence \& self-care
\item
  Monitor for obesity and diabetes signs.
\item
  Behavioral and mental health check
\end{itemize}
\end{minipage} & \begin{minipage}[t]{\linewidth}\raggedright
\begin{itemize}
\tightlist
\item
  Social services
\item
  Vocational/skills training
\item
  Mental health services
\end{itemize}
\end{minipage} \\
\end{longtable}

\begin{longtable}[]{@{}l@{}}
\toprule\noalign{}
\endhead
\bottomrule\noalign{}
\endlastfoot
 \\
\end{longtable}

\chapter{Edward's Syndrome}\label{edwards-syndrome}

\section{\texorpdfstring{\textbf{Introduction}}{Introduction}}\label{introduction-68}

Edward syndrome, or Trisomy 18, is the second most common autosomal
trisomy after Down syndrome, associated with a high rate of perinatal
mortality and profound developmental abnormalities. First described by
John Hilton Edward in 1960, the condition arises due to an extra copy of
chromosome 18. The estimated prevalence is approximately 1 in 5,000 live
births, but the actual incidence is higher due to early embryonic and
fetal losses. Edward syndrome has a significant impact on physical,
cognitive, and systemic development, with most affected infants
demonstrating severe disability and reduced survival.

\section{\texorpdfstring{\textbf{Definition}}{Definition}}\label{definition-40}

Edward syndrome is a chromosomal disorder caused by the presence of an
extra copy of chromosome 18, resulting in multisystem abnormalities. The
condition is characterized by intrauterine growth restriction (IUGR),
distinctive craniofacial and limb features, and major organ defects.
Based on the cytogenetic findings, Edward syndrome is classified into:

\begin{enumerate}
\def\labelenumi{\arabic{enumi}.}
\tightlist
\item
  \textbf{Full Trisomy 18 (90\%)}: The extra chromosome 18 in all cells
  due to non-disjunction during meiosis.
\item
  \textbf{Mosaic Trisomy 18 (5--10\%)}: Some cells have a normal
  karyotype, while others exhibit trisomy 18 due to post-zygotic
  non-disjunction.
\item
  \textbf{Partial Trisomy 18 (\textless1\%)}: A part of chromosome 18 is
  duplicated and attached to another chromosome.
\end{enumerate}

\section{\texorpdfstring{\textbf{Genetics}}{Genetics}}\label{genetics-1}

The underlying cause of Edward syndrome is the presence of three copies
of chromosome 18, which leads to the overexpression of genes located on
this chromosome. Chromosome 18 contains approximately 500--600 genes,
many involved in critical developmental pathways.

\begin{enumerate}
\def\labelenumi{\arabic{enumi}.}
\tightlist
\item
  \textbf{Mechanism of Trisomy 18}:

  \begin{itemize}
  \tightlist
  \item
    The most common cause is \textbf{meiotic nondisjunction},
    particularly in maternal gametes.
  \item
    Advanced maternal age is a significant risk factor, as with other
    trisomies.
  \end{itemize}
\item
  \textbf{Mosaicism} occurs when nondisjunction occurs after
  fertilization, resulting in a mixture of normal and trisomic cells.
  Mosaic cases often have milder phenotypes than full trisomy.
\item
  \textbf{Inheritance}:

  \begin{itemize}
  \tightlist
  \item
    Most cases are sporadic and not inherited.
  \item
    Rarely, partial trisomy may result from a balanced translocation in
    one parent.
  \end{itemize}
\end{enumerate}

\section{\texorpdfstring{\textbf{Clinical
Features}}{Clinical Features}}\label{clinical-features-72}

Edward syndrome presents a constellation of physical, neurological, and
systemic abnormalities, which are often apparent prenatally or at birth.

\begin{enumerate}
\def\labelenumi{\arabic{enumi}.}
\tightlist
\item
  \textbf{Prenatal Features}:

  \begin{itemize}
  \tightlist
  \item
    Severe intrauterine growth restriction (IUGR)
  \item
    Polyhydramnios or oligohydramnios
  \item
    Structural anomalies detectable on fetal ultrasound (e.g., cardiac
    defects, clenched hands, and overlapping fingers)
  \end{itemize}
\item
  \textbf{Craniofacial Features}:

  \begin{itemize}
  \tightlist
  \item
    Microcephaly
  \item
    Prominent occiput
  \item
    Low-set, malformed ears
  \item
    Micrognathia (undersized jaw)
  \item
    Cleft lip and/or palate (less common)
  \end{itemize}
\item
  \textbf{Limb and Skeletal Abnormalities}:

  \begin{itemize}
  \tightlist
  \item
    Clenched hands with overlapping fingers
  \item
    Rocker-bottom feet
  \item
    Hypoplastic nails
  \item
    Short sternum
  \end{itemize}
\item
  \textbf{Neurological Features}:

  \begin{itemize}
  \tightlist
  \item
    Severe intellectual disability
  \item
    Hypotonia or hypertonia
  \item
    Seizures
  \end{itemize}
\item
  \textbf{Cardiac Abnormalities}:

  \begin{itemize}
  \tightlist
  \item
    Present in 90--95\% of cases
  \item
    The most common defects are \href{cvs-vsd.qmd}{ventricular septal
    defects}, \href{cvs-asd.qmd}{atrial septal defects}, and
    \href{cvs-pda.qmd}{patent ductus arteriosus}.
  \end{itemize}
\item
  \textbf{Other Systemic Abnormalities}:

  \begin{itemize}
  \tightlist
  \item
    \textbf{Gastrointestinal}: Omphalocele, esophageal atresia, or
    malrotation.
  \item
    \textbf{Renal}: Horseshoe kidney, hydronephrosis, or renal agenesis.
  \item
    \textbf{Respiratory}: Hypoplastic lungs in some cases.
  \end{itemize}
\item
  \textbf{Postnatal Features}:

  \begin{itemize}
  \tightlist
  \item
    Failure to thrive due to feeding difficulties.
  \item
    Persistent respiratory infections.
  \item
    Limited spontaneous movement.
  \end{itemize}
\end{enumerate}

\section{\texorpdfstring{\textbf{Investigations}}{Investigations}}\label{investigations-52}

A combination of prenatal and postnatal diagnostic tools confirms Edward
syndrome and identifies associated anomalies.

\begin{enumerate}
\def\labelenumi{\arabic{enumi}.}
\tightlist
\item
  \textbf{Prenatal Investigations}:

  \begin{itemize}
  \tightlist
  \item
    \textbf{Ultrasound Findings}:

    \begin{itemize}
    \tightlist
    \item
      IUGR
    \item
      Structural anomalies (e.g., cardiac defects, clenched hands,
      omphalocele)
    \end{itemize}
  \item
    \textbf{Maternal Serum Screening}:

    \begin{itemize}
    \tightlist
    \item
      Low alpha-fetoprotein (AFP) levels, unconjugated estriol (uE3),
      and free β-hCG.
    \end{itemize}
  \item
    \textbf{Non-Invasive Prenatal Testing (NIPT)}:

    \begin{itemize}
    \tightlist
    \item
      Detects fetal cell-free DNA in maternal blood with high
      sensitivity and specificity for trisomy 18
    \end{itemize}
  \item
    \textbf{Invasive Diagnostic Testing}:

    \begin{itemize}
    \tightlist
    \item
      \textbf{Amniocentesis} (after 15 weeks): Definitive karyotyping to
      confirm trisomy 18.
    \item
      \textbf{Chorionic Villus Sampling (CVS)} (10--13 weeks): Early
      diagnostic confirmation
    \end{itemize}
  \end{itemize}
\item
  \textbf{Postnatal Investigations}:

  \begin{itemize}
  \tightlist
  \item
    \textbf{Cytogenetic Testing}:

    \begin{itemize}
    \tightlist
    \item
      Karyotyping remains the gold standard for diagnosis.
    \item
      Fluorescence in situ hybridization (FISH) or chromosomal
      microarray analysis for rapid diagnosis.
    \end{itemize}
  \item
    \textbf{Echocardiogram}:

    \begin{itemize}
    \tightlist
    \item
      Essential to identify and evaluate congenital heart defects.
    \end{itemize}
  \item
    \textbf{Renal Ultrasound}:

    \begin{itemize}
    \tightlist
    \item
      Detects structural anomalies like horseshoe kidney or
      hydronephrosis.
    \end{itemize}
  \item
    \textbf{Other Tests}:

    \begin{itemize}
    \tightlist
    \item
      Brain imaging (MRI or CT) for structural abnormalities.
    \item
      Hearing and vision screening.
    \end{itemize}
  \end{itemize}
\end{enumerate}

\section{\texorpdfstring{\textbf{Treatment}}{Treatment}}\label{treatment-23}

Edward syndrome does not have a curative treatment, and management
focuses on supportive and palliative care to improve the quality of
life. Treatment depends on the severity of symptoms and associated
anomalies.

\begin{enumerate}
\def\labelenumi{\arabic{enumi}.}
\tightlist
\item
  \textbf{Neonatal Care}: Support for feeding difficulties, often
  requiring nasogastric tube feeding. Management of respiratory distress
  with oxygen therapy or mechanical ventilation.
\item
  \textbf{Cardiac Care}: Severe cardiac defects may require surgical
  intervention, but many families opt for conservative management due to
  the poor prognosis.
\item
  \textbf{Surgical Management}: Repairs for gastrointestinal anomalies
  like esophageal atresia or omphalocele if compatible with survival.
\item
  \textbf{Developmental Support}: Physical, occupational, and speech
  therapy can be introduced to optimize motor skills and communication,
  although progress is often limited.
\item
  \textbf{Palliative Care}: Comfort-focused care is essential for
  managing pain, feeding issues, and respiratory infections.

  \begin{itemize}
  \tightlist
  \item
    Hospice care is often recommended for severe cases.
  \end{itemize}
\end{enumerate}

\section{\texorpdfstring{\textbf{Counselling}}{Counselling}}\label{counselling-1}

Counseling is vital to managing Edward syndrome and providing support to
families before and after diagnosis.

\begin{enumerate}
\def\labelenumi{\arabic{enumi}.}
\tightlist
\item
  \textbf{Prenatal Counselling}:

  \begin{itemize}
  \tightlist
  \item
    Inform parents about the diagnosis, prognosis, and potential
    outcomes.
  \item
    Discuss available options, including continuation or termination of
    pregnancy.
  \item
    Offer psychological support and refer to genetic counseling
  \end{itemize}
\item
  \textbf{Postnatal Counselling}:

  \begin{itemize}
  \tightlist
  \item
    Educate families on the medical challenges and potential
    interventions.
  \item
    Provide information on support systems and resources.
  \item
    Encourage the involvement of multidisciplinary teams, including
    neonatologists, geneticists, and social workers.
  \end{itemize}
\item
  \textbf{Genetic Counselling}:

  \begin{itemize}
  \tightlist
  \item
    For families with a history of Edward syndrome, offer genetic
    testing to identify potential translocations or chromosomal
    abnormalities in parents.
  \item
    Explain recurrence risks:

    \begin{itemize}
    \tightlist
    \item
      \textless1\% for full trisomy 18 due to de novo events.
    \item
      Higher if a parent is a carrier of a balanced translocation.
    \end{itemize}
  \end{itemize}
\item
  \textbf{Psychosocial Support}:

  \begin{itemize}
  \tightlist
  \item
    Assist families in coping with grief and decision-making.
  \item
    Connect them with support groups and advocacy organizations
  \end{itemize}
\end{enumerate}

\section{\texorpdfstring{\textbf{Conclusion}}{Conclusion}}\label{conclusion-35}

Edward syndrome is a severe chromosomal disorder characterized by
profound developmental and physical abnormalities, with a high risk of
perinatal mortality. Advances in prenatal diagnostic techniques allow
for early detection and informed decision-making. While there is no
definitive cure, multidisciplinary care focusing on symptom management,
supportive therapy, and counseling is critical in improving the quality
of life for affected individuals and their families. Further research
into the molecular mechanisms of trisomy 18 may offer insights into
potential therapeutic strategies in the future.

\chapter{Patau Syndrome}\label{patau-syndrome}

\section{Definition}\label{definition-41}

Patau syndrome is a rare genetic disorder with the patient having an
extra copy of chromosome 13 (Trisomy 13).

\section{Genetics}\label{genetics-2}

Patau Syndrome is due most commonly to non-disjunction in meiosis,
occurring more frequently in mothers of advanced age (greater than 35).
Also, an unbalanced Robertsian translocation which results in 2 normal
copies of chromosome 13 and an additional long arm of chromosome 13 can
cause Edward's syndrome. A less common cause is mosaicism which results
in 3 copies of chromosome 13 in some cells and two copies in the others.
Mosaics are the outcome of a mitotic non-disjunction error and are
unrelated to maternal age. The prognosis is better in patients with
mosaicism and patients with unbalanced translocations.

\section{Clinical features}\label{clinical-features-73}

This extra copy of chromosome 13 disrupts normal embryonic development
and leads to multiple defects. Some of these are:

\begin{itemize}
\tightlist
\item
  \emph{CNS}: Alobar holoprosencephaly, Anophthalmia, Microphthalmia,
  Coloboma, Intellectual disability, Hypotonia, Microcephaly, Seizures
\item
  \emph{Gastrointestinal}: Exomphalos, Hernia, Meckel diverticulum,
  Omphalocele
\item
  \emph{Genitourinary}: Polycystic kidneys, Cryptorchidism, Hypospadias,
  Labia Minora Hypoplasia
\item
  \emph{Cardiovascular}: \href{cvs-vsd.qmd}{Ventricular Septal Defect},
  \href{cvs-asd.qmd}{Atrial Septal Defect}, Atrioventricualr Canal
  Defect, \href{cvs-tof.qmd}{Tetralogy of Fallot}
\item
  \emph{Skeletal}: Polydactyly, Congenital talipes
\item
  \emph{Craniofacial}: Cleft lip /cleft palate, Micrognathia
\item
  \emph{Others}: Intrauterine Growth Restriction, Rocker bottom feet,
  Psychomotor disorders, Capillary hemangioma, Pre-auricular tags
\end{itemize}

\section{Investigation}\label{investigation-2}

\begin{itemize}
\tightlist
\item
  Prenatally with chorionic villi sampling, amniocentesis or fetal free
  DNA analysis.
\item
  Tissue microarray (especially in fetal death)
\item
  FISH(fluorescent in-situ hybridization)
\end{itemize}

\section{Treatment}\label{treatment-24}

There is no definitive treatment for Patau syndrome. A multidisciplinary
approach is required. Some of these include:

\begin{itemize}
\tightlist
\item
  Counselling
\item
  Rehabilitation team/ palliative team\\
\item
  Occupational therapy
\item
  Psychology
\item
  Cardiology
\end{itemize}

\section{Counselling}\label{counselling-2}

There should be genetic counselling offered to the parents. Genetic
testing for parents The recurrence risk for Trisomy is approximately
0.5\% above the mother's age-related risk for autosomal trisomy. Consult
a genetic counsellor or medical geneticist regarding recurrence risks
for structural rearrangements that involve chromosome 13. Prognosis is
generally poor (82\% die within 1 month, 95\% die within 6 months)

\chapter{Turner's Syndrome}\label{turners-syndrome}

\section{Definition}\label{definition-42}

It is defined as a combination of peculiar phenotypic features with
either a complete or partial absence of the second X chromosome. It is
one the most common chromosomal abnormalities, occurring in
approximately 1 in 1,500 to 2,500 live-birth females.

\section{Genetics}\label{genetics-3}

About half the patients with Turner syndrome have 45,X chromosomal
complement. The single X chromosome is of maternal origin in 80\%, thus
the missing X is usually of paternal origin. A similar clinical feature
is seen in 46,XXiq karyotype and mosaicism (45,X/ 46, XX). Deletion of
the SHOX gene (a gene thought to be important in controlling growth in
children) is a consistent finding in Turner syndrome. A similar skeletal
phenotype is seen in Leri-Weill dyschondrosteosis.

\section{Clinical Features}\label{clinical-features-74}

\begin{itemize}
\tightlist
\item
  \textbf{At birth}: Low birthweight Decreased birth length, Lymphedema
  of hands and feet, Loose skinfolds at the nape.
\item
  \textbf{During childhood into early adolescence}: Short stature,
  webbed neck, low posterior hairline, epicanthal folds, posteriorly
  rotated ears, high arched palate, micrognathia, shield chest, cubitus
  valgus, short 4th metacarpal bone, and hyperconvex fingernails, normal
  mental intelligence (except in mathematics).
\item
  \textbf{Late adolescence into adulthood}: Delayed puberty. Adrenarche
  occurs at a normal age and delayed breast development but will
  completely be absent if ovarian failure occurs before puberty. Primary
  or secondary amenorrhoea occurs with ovarian failure.
\end{itemize}

\section{Associations}\label{associations}

\begin{itemize}
\tightlist
\item
  \textbf{Ovarian dysgenesis}: The normal fetal ovary contains about
  Seven million oocytes. However, these begin to disappear rapidly after
  the 5th month of gestation. At birth, there are only 2 million and by
  age 2 almost all oocytes are depleted.
\item
  \textbf{Cardiac defects}: Bicuspid aortic valve (33-50\%),
  \href{cvs-coa.qmd}{Coarctation of the Aorta} (20\%), Aortic stenosis,
  Mitral valve prolapse, Partial anomalous pulmonary venous drainage.
\item
  \textbf{Renal malformation}: Seen in 25-50\%. They include horseshoe
  kidneys, double collecting systems Pelvic kidneys, and the Complete
  absence of one kidney.
\item
  \textbf{Others}: Autoimmune thyroid disease (10-30\%). Inflammatory
  Bowel Disease and Celiac disease (screening is recommended because the
  risk of celiac disease is increased in Turner syndrome (4-6\%).
  Recurrent otitis media (75\%), resulting in sensorineural hearing
  deficits. Increased risk of Gonadoblastoma, therefore prophylactic
  gonadectomy is recommended even in the presence of MRI or CT scan
  evidence.
\end{itemize}

\section{Investigation}\label{investigation-3}

\begin{itemize}
\tightlist
\item
  Genetic testing in all females with short stature and other related
  clinical features.
\item
  Echocardiogram to screen for cardiac defects.
\item
  Abdominopelvic ultrasound scan to assess the kidneys and the ovaries.
  -
\item
  Plasma levels of follicle-stimulating hormone (FSH) are markedly
  raised while the are low levels of estradiol.
\item
  Transglutaminase immunoglobulin A antibodies to rule in or out Celiac
  disease.
\item
  Thyroid peroxidase and thyroglobulin antibodies should be checked
  periodically to detect autoimmune.
\end{itemize}

\section{Treatment}\label{treatment-25}

\begin{itemize}
\tightlist
\item
  Recombinant human growth hormone helps with height velocity.
\item
  Estrogen replacement therapy for the development of sexual
  characteristics.
\item
  Psychological support is very crucial for the management of patients;
  to help them accept diagnosis and treatment modalities.
\end{itemize}

\section{Complications}\label{complications-37}

\begin{itemize}
\tightlist
\item
  \href{nep-hypertension.qmd}{Hypertension}
\item
  Abnormal glucose tolerance and insulin resistance
\item
  Infertility
\item
  Sensorineural hearing loss from recurrent bilateral otitis media
\end{itemize}

\section{Counselling}\label{counselling-3}

\begin{itemize}
\tightlist
\item
  There should be genetic screening for couples or individuals with
  Turner syndrome.
\item
  There is currently no cure and treatment options are focused on
  symptomatic management, in a multidisciplinary team approach.
\end{itemize}

\section{Review Schedule}\label{review-schedule-1}

\textbf{Birth -- 1 Month}

\begin{itemize}
\tightlist
\item
  Confirm diagnosis with karyotype
\item
  Full physical examination (check for webbed neck, lymphedema)
\item
  Echocardiogram to assess for:

  \begin{itemize}
  \tightlist
  \item
    Coarctation of the aorta
  \item
    Bicuspid aortic valve
  \end{itemize}
\item
  Renal ultrasound to detect congenital anomalies
\item
  Assess feeding, weight gain, and growth
\item
  Begin Turner-specific growth charting
\item
  Referrals:

  \begin{itemize}
  \tightlist
  \item
    Pediatric cardiology
  \item
    Pediatric nephrology
  \item
    Genetic counseling
  \end{itemize}
\end{itemize}

\textbf{1 -- 6 Months}

\begin{itemize}
\tightlist
\item
  Monitor growth and development
\item
  Repeat cardiac evaluation if abnormalities found
\item
  Thyroid function test (TSH and T4)
\item
  Hearing screen
\item
  First ophthalmologic (eye) exam
\item
  Monitor for:

  \begin{itemize}
  \tightlist
  \item
    Feeding difficulties
  \item
    Constipation
  \end{itemize}
\item
  Referrals:

  \begin{itemize}
  \tightlist
  \item
    Endocrinologist
  \item
    Audiologist
  \item
    Ophthalmologist
  \end{itemize}
\end{itemize}

\textbf{6 -- 12 Months}

\begin{itemize}
\tightlist
\item
  Track developmental milestones
\item
  Monitor growth using TS-specific charts
\item
  Repeat hearing test if needed
\item
  Screen for recurrent otitis media
\item
  Begin early intervention if developmental delays are noted
\item
  Referral to ENT if frequent ear infections
\end{itemize}

\textbf{1 -- 2 Years}

\begin{itemize}
\tightlist
\item
  Annual thyroid function testing
\item
  Monitor growth velocity
\item
  Repeat cardiac evaluation if previously abnormal
\item
  Begin dental evaluation
\item
  Continue hearing and speech assessment
\item
  Consider initiating Growth Hormone Therapy (GH) around 2--5 years
\item
  Referral to:

  \begin{itemize}
  \tightlist
  \item
    Speech therapist
  \item
    Endocrinologist
  \end{itemize}
\end{itemize}

\textbf{2 -- 5 Years}

\begin{itemize}
\tightlist
\item
  Measure height and weight every 3--6 months
\item
  Annual:

  \begin{itemize}
  \tightlist
  \item
    Hearing test
  \item
    Vision exam
  \item
    Thyroid function
  \end{itemize}
\item
  ENT review for chronic ear infections
\item
  Cardiac re-evaluation every 3--5 years
\item
  Continue/Start Growth Hormaone therapy.
\item
  Regular dental check-ups every 6 months
\item
  Begin developmental and learning support if needed
\end{itemize}

\textbf{5 -- 10 Years}

\begin{itemize}
\tightlist
\item
  Annual:

  \begin{itemize}
  \tightlist
  \item
    Thyroid, vision, hearing, dental exams
  \end{itemize}
\item
  Bone age x-ray every 1--2 years
\item
  Monitor for signs of puberty
\item
  Screen for learning and behavioral issues
\item
  Check blood pressure annually
\item
  Continued GH therapy
\item
  Referrals:

  \begin{itemize}
  \tightlist
  \item
    Psychologist or special educator
  \item
    Neurodevelopmental team for learning difficulties
  \end{itemize}
\end{itemize}

\textbf{10 -- 13 Years (Early Adolescence)}

\begin{itemize}
\tightlist
\item
  Assess for pubertal development

  \begin{itemize}
  \tightlist
  \item
    If absent by 12--13 years, begin estrogen replacement
  \end{itemize}
\item
  Annual:

  \begin{itemize}
  \tightlist
  \item
    Thyroid
  \item
    Hearing
  \item
    Vision
  \item
    Liver function and lipid profile
  \end{itemize}
\item
  DEXA scan if available to assess bone health
\item
  Cardiac MRI or echocardiogram if CHD present
\item
  Continue GH if appropriate
\item
  Begin psychosocial counseling
\end{itemize}

\textbf{13 -- 16 Years (Mid-Adolescence)}

\begin{itemize}
\item
  Continue estrogen therapy

  \begin{itemize}
  \tightlist
  \item
    Introduce progesterone if uterus is present
  \end{itemize}
\item
  Annual:

  \begin{itemize}
  \tightlist
  \item
    Thyroid
  \item
    Liver function
  \item
    Lipid profile
  \end{itemize}
\item
  Discuss:

  \begin{itemize}
  \tightlist
  \item
    Fertility options
  \item
    Sexual health education
  \end{itemize}
\item
  Monitor for emotional well-being, social adjustment
\item
  Referrals:

  \begin{itemize}
  \tightlist
  \item
    Gynecologist or adolescent endocrinologist
  \item
    Mental health support
  \item
    Career and vocational guidance
  \end{itemize}
\end{itemize}

\chapter{Noonan Syndrome}\label{noonan-syndrome}

\section{Definition}\label{definition-43}

Noonan syndrome (NS) is a genetic disorder in which multiple organ
systems are affected and is typically characterised by distinct facial
features, short stature, congenital heart disease and neurodevelopment
abnormalities.

\section{Genetics}\label{genetics-4}

NS occurs sporadically or in an autosomal dominant manner, the latter
being the most common mode of transmission. In either case, males and
females are affected equally. In familial cases, the mother is usually
the transmitting parent because a considerable proportion of affected
males have problems with fertility. Several mutations in the genes
involved in the renin-angiotensin system (RAS)--mitogen-activated
protein kinase (MAPK) pathway have been implicated. Such mutations are
currently detected in approximately 70\% of cases. Mutations in the
PTPN11 gene on chromosome 12q24.1 are by far the commonest. There is a
correlation between specific gene mutations and the clinical
manifestations of the syndrome. Autosomal recessive forms of the
disorder have also been identified.

\section{Clinical features}\label{clinical-features-75}

The clinical manifestations depend on the affected gene and the stage of
life at which the patient is presenting. Suggestive findings on prenatal
ultrasound scans include lymphatic system anomalies such as increased
nuchal translucency, cystic hygroma and fetal ascites. Other supportive
findings include polyhydramnios, short femur length, renal and cardiac
defects (particularly pulmonary valve dysplasia) in the fetus. At birth,
most affected neonates appear unremarkable. Birth weight and head
circumference are usually within the normal range. A cardiac murmur may
be detected on routine newborn examination. Male neonates may have
cryptorchidism. The affected infant may present with feeding
difficulties requiring ongoing nasogastric feeding or gastrostomy, which
can impair growth. Talipes equinovarus may be present in 5\% of
patients. The characteristic facial features of NS become more apparent
in early childhood and include a triangular-shaped face, hypertelorism,
blue-green irises, down-slanting palpebral fissures, epicanthal folds,
ptosis, low-set ears with thickened posteriorly rotated auricles, high
nasal bridge and upturned nose, short webbed neck and micrognathia.
Short stature is present in 85\% of cases. Cutaneous features in
childhood include lymphedema affecting the lower limbs and genitals,
follicular keratosis of the face and extensor surfaces, multiple
lentigines and café au lait spots. Musculoskeletal features include a
shield chest, cubitus valgus, pectus carinatum or excavatum, scoliosis,
and joint laxity. Affected children are usually hypotonic with delayed
motor and speech development. Expressive language skills are more
affected. A careful family history is important with emphasis on the
presence of congenital heart disease, short stature or unusual facies
among first-degree relatives.

\section{Associations}\label{associations-1}

Congenital heart disease is present in almost 80\% of patients with NS.
The main cardiac lesion is a dysplastic or stenotic pulmonic valve.
Other associated cardiac defects include hypertrophic cardiomyopathy
(30\%), \href{cvs-asd.qmd}{atrial septal defect},
\href{cvs-vsd.qmd}{ventricular septal defects},
\href{cvs-tof.qmd}{tetralogy of Fallot}, and
\href{cvs-coa.qmd}{coarctation of the aorta}. Most patients have more
than one cardiac defect. Renal anomalies such as ectopic kidneys are
present in 10\% of cases. Fifty per cent of affected males have
cryptorchidism with hypoplastic testes. There is an increased risk of
the development of certain cancers because of gene mutations affecting
the RAS-MAPK pathway, an important cell signalling pathway. These
include neuroblastoma, acute lymphoblastic leukaemia, low-grade glioma,
rhabdomyosarcoma, juvenile myelomonocytic leukaemia and
myeloproliferative disorders. There are associated autoimmune diseases
such as systemic lupus erythematosus and autoimmune thyroiditis. Seizure
disorders, unexplained peripheral neuropathy and intellectual
disabilities have been documented in association with NS.

\section{Investigations}\label{investigations-53}

Amniocentesis for the fetal karyotype can be done if prenatal USG
findings are suggestive. A normal karyotype rules out
\href{synd-turners.qmd}{Turner syndrome} which is a close differential.
Genetic studies to identify the specific gene mutation are important to
help confirm the diagnosis and determine prognosis. Other investigations
are conducted based on the affected organ systems. These include
echocardiography and electrocardiography for cardiac defects,
coagulation screen with full hematologic workup for associated bleeding
and neoplastic disorders, renal ultrasonographic examination for
genitourinary problems and brain and spine MRI if neurologic symptoms
are present.

\section{Treatment}\label{treatment-26}

Treatment is focused on the presenting symptoms and the affected organ
systems. A multidisciplinary team made up of paediatricians,
geneticists, cardiologists, haematologists, ophthalmologists,
neurologists, urologists, endocrinologists, audiologists, physical
therapists, speech therapists, occupational therapists, clinical
psychologists, and social workers is needed to manage the various
complications associated with NS. Growth hormone therapy can be given to
affected children with growth impairment to accelerate growth to a
near-normal adult height.

\section{Complications}\label{complications-38}

These include delayed or absent puberty, neurodevelopment and
behavioural problems such as autism spectrum disorders, attention
deficit hyperactivity disorder and seizure disorders. Other
complications are bleeding diathesis most commonly from factors XI and
XII deficiency and platelet disorders. Ophthalmologic complications
include strabismus, amblyopia, refractive errors, posterior embryotoxon,
optic nerve hypoplasia and nystagmus. Progressive hearing loss
(sensorineural, conductive, or mixed loss) and recurrent otitis media
are known to occur. Orthopaedic complications include joint
contractures, fusion of the cervical spine and scoliosis.

\section{Notes on counselling}\label{notes-on-counselling}

Parents need to be counselled regarding recurrence risk with each
pregnancy in familial cases and prenatal diagnosis offered. Families of
affected children need to be informed that although there is no
definitive cure for NS, the various associations and complications can
be managed very well by a multidisciplinary team to ensure the child has
an excellent quality of life. Affected children with mild or no learning
difficulties do not require special education but can integrate well
into mainstream schools. Families of affected children should be linked
to support groups. Patients with bleeding disorders must be advised
against the use of aspirin and aspirin-containing products or other
medications that may interfere with coagulation or platelet function.

\section{Review schedule}\label{review-schedule-2}

\textbf{Birth -- 1 Month}

\begin{itemize}
\tightlist
\item
  Confirm diagnosis (clinical or genetic testing)
\item
  Full physical examination (facial features, webbed neck, pectus
  deformity)
\item
  Echocardiogram (commonly pulmonary valve stenosis or hypertrophic
  cardiomyopathy)
\item
  Renal ultrasound (identify structural anomalies)
\item
  Feeding and growth assessment
\item
  Hematologic evaluation (coagulation profile due to bleeding diathesis)
\item
  Referrals:

  \begin{itemize}
  \tightlist
  \item
    Cardiology
  \item
    Nephrology
  \item
    Genetic counseling
  \item
    Nutritionist (if feeding difficulties)
  \end{itemize}
\end{itemize}

\textbf{1 -- 6 Months}

\begin{itemize}
\tightlist
\item
  Monitor growth, weight, and development
\item
  Repeat echocardiogram if initial findings abnormal
\item
  Assess for feeding problems, reflux
\item
  Ophthalmologic evaluation (strabismus, refractive errors)
\item
  Hearing screen (sensorineural or conductive hearing loss risk)
\item
  Coagulation profile if surgical procedure is planned
\item
  Referrals:

  \begin{itemize}
  \tightlist
  \item
    ENT for hearing concerns
  \item
    Ophthalmology
  \item
    Developmental services if delays are present
  \end{itemize}
\end{itemize}

\textbf{6 -- 12 Months}

\begin{itemize}
\tightlist
\item
  Track developmental milestones
\item
  Monitor growth using NS-specific growth charts if available
\item
  Repeat vision and hearing assessment
\item
  Screen for neuromotor or cognitive delay
\item
  Evaluate for bleeding/bruising
\item
  Begin early intervention therapies if delays
\item
  Continue feeding and nutrition support
\end{itemize}

\textbf{1 -- 2 Years}

\begin{itemize}
\tightlist
\item
  Annual:

  \begin{itemize}
  \tightlist
  \item
    Hearing
  \item
    Vision
  \item
    Coagulation profile if needed
  \item
    Cardiac follow-up if heart defect presen
  \end{itemize}
\item
  Monitor:

  \begin{itemize}
  \tightlist
  \item
    Growth velocity
  \item
    Milestone achievement
  \item
    Behavioral or speech delays
  \end{itemize}
\item
  Consider speech therapy if language is delayed
\item
  Dental evaluation (high arched palate, crowded teeth)
\end{itemize}

\textbf{2 -- 5 Years}

\begin{itemize}
\tightlist
\item
  Annual:

  \begin{itemize}
  \tightlist
  \item
    Vision, hearing, and cardiac exams
  \item
    Thyroid function (especially if sluggish growth)
  \end{itemize}
\item
  Track growth and nutritional status
\item
  Evaluate for:

  \begin{itemize}
  \tightlist
  \item
    Speech delays
  \item
    Behavioral concerns (ADHD, anxiety)
  \end{itemize}
\item
  Monitor for bleeding/bruising during procedures
\item
  Begin preschool/school readiness assessment
\item
  Referrals:

  \begin{itemize}
  \tightlist
  \item
    Speech/language therapy
  \item
    Child psychologist if behavioral issues arise
  \end{itemize}
\end{itemize}

\textbf{5 -- 10 Years}

\begin{itemize}
\tightlist
\item
  Annual:

  \begin{itemize}
  \tightlist
  \item
    Echocardiogram (especially for hypertrophic cardiomyopathy)
  \item
    Vision, hearing, and dental
  \item
    Developmental and behavioral assessment
  \end{itemize}
\item
  Monitor:

  \begin{itemize}
  \tightlist
  \item
    Learning difficulties and school performance
  \item
    Emotional and social development
  \end{itemize}
\item
  Assess for short stature and consider growth hormone therapy if growth
  failure persists
\item
  Continue monitoring for bleeding diathesis
\item
  Referrals:

  \begin{itemize}
  \tightlist
  \item
    Special educator
  \item
    Developmental pediatrician
  \item
    Endocrinologist for GH evaluation
  \end{itemize}
\end{itemize}

\textbf{10 -- 13 Years (Early Adolescence)}

\begin{itemize}
\tightlist
\item
  Annual:

  \begin{itemize}
  \tightlist
  \item
    Thyroid function
  \item
    Vision and hearing
  \item
    Echocardiogram or cardiac MRI
  \item
    Dental and orthodontic evaluation
  \end{itemize}
\item
  Monitor:

  \begin{itemize}
  \tightlist
  \item
    Pubertal development (may be delayed)
  \item
    Psychosocial and emotional development
  \item
    School performance and peer interaction
  \end{itemize}
\item
  Evaluate for:

  \begin{itemize}
  \tightlist
  \item
    Menstrual issues (girls)
  \item
    Fertility discussions (if appropriate)
  \end{itemize}
\item
  Continue growth hormone if previously initiated
\item
  Provide psychological counseling if needed
\end{itemize}

\textbf{13 -- 16 Years (Mid-Adolescence)}

\begin{itemize}
\tightlist
\item
  Continue annually:

  \begin{itemize}
  \tightlist
  \item
    Cardiac imaging
  \item
    Thyroid, hearing, vision
  \item
    Behavioural and emotional health assessments
  \end{itemize}
\item
  Monitor:

  \begin{itemize}
  \tightlist
  \item
    Final height
  \item
    Completion of puberty
  \item
    Menstrual regularity and fertility counseling
  \end{itemize}
\item
  Discuss:

  \begin{itemize}
  \tightlist
  \item
    Transition to adult care
  \item
    Career and vocational planning
  \end{itemize}
\item
  Referrals:

  \begin{itemize}
  \tightlist
  \item
    Adolescent medicine or gynecology
  \item
    Endocrinology
  \item
    Mental health support
  \item
    Social services
  \end{itemize}
\end{itemize}

\chapter{DiGeorge Syndrome}\label{digeorge-syndrome}

\section{Introduction}\label{introduction-69}

DiGeorge Syndrome (DGS), or 22q11.2 deletion syndrome, is a rare genetic
disorder with significant clinical variability. It affects multiple body
systems, including the heart, immune system, endocrine organs, and
facial structures. Due to its diverse presentations, this condition may
go undiagnosed or misdiagnosed without appropriate genetic screening.
Advances in medical research have improved the understanding and
management of DiGeorge Syndrome, benefiting individuals and families
impacted by this complex condition.

\section{Definition}\label{definition-44}

DiGeorge Syndrome is a chromosomal disorder caused by the deletion of a
small segment of chromosome 22 at the q11.2 locus. It is characterized
by congenital abnormalities affecting the heart, thymus, and parathyroid
glands, leading to immune deficiency, hypocalcemia, and developmental
delays. Also referred to as velocardiofacial syndrome (VCFS) or
conotruncal anomaly face syndrome, DiGeorge Syndrome belongs to a
spectrum of conditions grouped under 22q11.2 deletion syndromes.

\section{Genetics}\label{genetics-5}

DiGeorge Syndrome occurs due to a microdeletion on the long arm (q) of
chromosome 22 at position 11.2. This deletion results in the loss of
approximately 30 to 40 genes. One of the most crucial genes affected is
TBX1, which plays a vital role in embryonic development. The condition
typically arises sporadically, although it can be inherited in an
autosomal dominant manner. Inherited cases account for about 10\% of all
cases, meaning an affected individual has a 50\% chance of passing the
deletion to offspring. Most cases, however, result from de novo
deletions during embryogenesis.

\section{Clinical Features}\label{clinical-features-76}

The manifestations of DiGeorge Syndrome are highly variable, even among
affected members of the same family. Clinical features can range from
mild to severe and may affect multiple organ systems.

\begin{enumerate}
\def\labelenumi{\arabic{enumi}.}
\tightlist
\item
  \textbf{Cardiac Abnormalities}: Congenital heart defects are present
  in up to 75\% of individuals. Common defects include:

  \begin{itemize}
  \tightlist
  \item
    \href{cvs-tof.qmd}{Tetralogy of Fallot}
  \item
    Interrupted aortic arch
  \item
    \href{cvs-vsd.qmd}{Ventricular septal defects}
  \item
    Truncus arteriosus
  \end{itemize}
\item
  \textbf{Immune Dysfunction}: Hypoplasia or aplasia of the thymus
  results in impaired T-cell production, leading to varying degrees of
  immune deficiency. Severe cases may resemble complete DiGeorge
  Syndrome, which mimics severe combined immunodeficiency (SCID).
\item
  \textbf{Endocrine Abnormalities}: These include

  \begin{itemize}
  \tightlist
  \item
    Hypoparathyroidism causing hypocalcemia
  \item
    Thyroid dysfunction in some cases
  \end{itemize}
\item
  \textbf{Craniofacial Features}:

  \begin{itemize}
  \tightlist
  \item
    Cleft palate or submucosal cleft palate
  \item
    Small or malformed ears
  \item
    Long face with a prominent nasal bridge
  \end{itemize}
\item
  \textbf{Neurodevelopmental and Psychiatric Disorders}:

  \begin{itemize}
  \tightlist
  \item
    Developmental delays, learning disabilities
  \item
    Speech and language difficulties
  \item
    Increased risk of psychiatric conditions such as schizophrenia and
    anxiety disorders
  \end{itemize}
\item
  \textbf{Gastrointestinal and Renal Anomalies}:

  \begin{itemize}
  \tightlist
  \item
    Feeding difficulties in infancy
  \item
    Structural abnormalities of the kidneys
  \end{itemize}
\end{enumerate}

\section{Investigations}\label{investigations-54}

A thorough clinical assessment is essential for diagnosing DiGeorge
Syndrome. The following investigations aid in confirming the diagnosis
and assessing disease severity:

\begin{enumerate}
\def\labelenumi{\arabic{enumi}.}
\item
  \textbf{Genetic Testing}:

  \begin{itemize}
  \tightlist
  \item
    Fluorescence in situ hybridization (FISH) to detect the 22q11.2
    deletion
  \item
    Microarray analysis for more comprehensive chromosomal evaluation
  \end{itemize}
\item
  \textbf{Immunological Studies}:

  \begin{itemize}
  \tightlist
  \item
    Lymphocyte subset analysis to evaluate T-cell function
  \item
    Immunoglobulin level measurement
  \end{itemize}
\item
  \textbf{Cardiac Imaging}:

  \begin{itemize}
  \tightlist
  \item
    Echocardiography to assess for congenital heart defects
  \end{itemize}
\item
  \textbf{Calcium Levels}:

  \begin{itemize}
  \tightlist
  \item
    Serum calcium and parathyroid hormone (PTH) measurements
  \end{itemize}
\item
  \textbf{Endocrine Evaluation}:

  \begin{itemize}
  \tightlist
  \item
    Thyroid function tests
  \end{itemize}
\item
  \textbf{Radiological Imaging}:

  \begin{itemize}
  \tightlist
  \item
    Renal ultrasound for structural anomalies
  \end{itemize}
\end{enumerate}

\section{Treatment}\label{treatment-27}

Management of DiGeorge Syndrome requires a multidisciplinary approach
tailored to each individual's needs. Early diagnosis and intervention
significantly improve outcomes.

\begin{enumerate}
\def\labelenumi{\arabic{enumi}.}
\tightlist
\item
  \textbf{Cardiac Care}:

  \begin{itemize}
  \tightlist
  \item
    Surgical correction of congenital heart defects when indicated
  \end{itemize}
\item
  \textbf{Immune Management}:

  \begin{itemize}
  \tightlist
  \item
    Prophylactic antibiotics and immunoglobulin replacement in cases of
    immune deficiency
  \item
    Thymus transplantation in severe cases
  \end{itemize}
\item
  \textbf{Calcium and Endocrine Management}:

  \begin{itemize}
  \tightlist
  \item
    Calcium and vitamin D supplementation for hypocalcemia
  \item
    Hormonal therapy for endocrine dysfunctions
  \end{itemize}
\item
  \textbf{Speech and Developmental Support:}

  \begin{itemize}
  \tightlist
  \item
    Speech therapy for communication difficulties
  \item
    Educational interventions for cognitive delays
  \end{itemize}
\item
  \textbf{Psychiatric Care:}

  \begin{itemize}
  \tightlist
  \item
    Psychological counseling and psychiatric treatment for mental health
    conditions
  \end{itemize}
\end{enumerate}

\section{Counseling}\label{counseling}

Genetic counseling plays a pivotal role in helping affected individuals
and families understand the nature of the condition, its inheritance
patterns, and reproductive risks. Key counseling points include:

\begin{enumerate}
\def\labelenumi{\arabic{enumi}.}
\tightlist
\item
  \textbf{Risk of Recurrence}: Parents of an affected child should be
  informed of the potential 50\% inheritance risk if a parent carries
  the deletion.
\item
  \textbf{Prenatal Diagnosis}: Options such as chorionic villus sampling
  (CVS) and amniocentesis can be offered to at-risk couples.
\item
  \textbf{Support Services}: Families benefit from connections to
  support groups and advocacy organizations to navigate the challenges
  associated with the condition.
\end{enumerate}

\section{Conclusion}\label{conclusion-36}

DiGeorge Syndrome is a complex genetic disorder with wide-ranging
clinical manifestations. Advances in genetic testing have greatly
improved diagnostic accuracy, enabling earlier interventions and better
outcomes. Multidisciplinary care, including cardiac, immunological,
developmental, and psychological support, is essential for managing the
condition effectively. Through continued research and awareness,
individuals living with DiGeorge Syndrome can achieved.

\chapter{Williams Syndrome}\label{williams-syndrome}

\section{Definition}\label{definition-45}

Williams Syndrome also known as Williams-Beuren Syndrome is a rare
genetic disorder that is characterized by pre and postnatal growth
delays, developmental delays, cardiovascular diseases, endocrine
abnormalities, connective tissue abnormalities, intellectual
disabilities, feeding difficulties and distinctive facies.

\section{Genetics}\label{genetics-6}

It has an autosomal dominant pattern of inheritance. This disorder
results from the deletion of the genetic material 7q11.23 and has been
designated ``Williams-Beuren Syndrome Chromosome Region 1'' (WBSCR1). It
codes for numerous genes which include ELN (elastin) gene, the LIMK1 (or
LIM Kinase -1) gene and the RFC2 (replication factor, C subunit 2) gene.
Penetrance is 100\%.

\section{Clinical features}\label{clinical-features-77}

Williams Syndrome is a multisystem disorder characterized by the
following features:

\subsection{Neurodevelopmental \&
Neurobehavioural}\label{neurodevelopmental-neurobehavioural}

Autism, ADHD, Developmental Delay, Sleep disorders, Intellectual
Disability, Overfriendliness

\subsection{Cardiovascular}\label{cardiovascular}

Chest pain, fatigue, dizziness, murmurs and
\href{nep-hypertension.qmd}{hypertension}

\subsection{Gastrointestinal}\label{gastrointestinal}

Constipation, Vomiting, Irritability, Muscle cramps and pain, Loss of
appetite, Abdominal pain and Confusion

\subsection{Endocrine}\label{endocrine}

Short stature, Polyuria, Daytime wetting, hoarse voice, short stature

\subsection{Craniofacial \&
Musculoskeletal}\label{craniofacial-musculoskeletal}

Median flare of the eyebrows, Perioral and periorbital fullness,
Depressed nasal bridge, wide mouth, long philtrum, small jaw, ataxia,
dysmetria, and tremor, and hypotonia

\subsection{Urinary}\label{urinary}

Increased Urinary frequency, and Enuresis

\subsection{Ocular, Auditory \& Dental}\label{ocular-auditory-dental}

Hyperacusis, star-like (stellate) pattern in the iris of the eye. Blue
iris. Strabismus, Hearing loss

\section{Investigation}\label{investigation-4}

that may be used in the diagnosis of William Syndrome include Body Mass
Index (BMI), Complete Blood count (CBC), Complete Metabolic Panel (CMP),
Serum calcium for hypercalcemia, Thyroid stimulating hormone including
Free T3 and Free T4, Vision and Hearing Assessment, Echocardiogram for
associated structural anomalies, Electrocardiogram, Fluorescent in Situ
Hybridization (FISH)

\section{Treatment}\label{treatment-28}

Effective treatment and management of children with Williams Syndrome
require a multidisciplinary approach including:

\begin{itemize}
\tightlist
\item
  Genetic counselling after the diagnosis of Williams Syndrome is made.
\item
  Cardiology \& Cardiothoracic Surgery - The use of antihypertensives to
  control hypertension. Open heart surgeries for surgical correction of
  anomalies such as supra-valvular aortic stenosis
\item
  Endocrinology -- Dietary modifications, exercise, the use of growth
  hormones, thyroid hormone replacement therapy, the use of oral
  corticosteroids or IV pamidronate
\item
  Nephrology -- Lithotripsy for the management of renal calculi
\item
  Gastroenterology -- The placement of a permanent feeding tube for
  patients with feeding difficulties
\item
  Ophthalmology, ENT -- Corrective lenses for hyperopia, Ear protection
  for hyperacusis
\item
  Psychiatry -- Medical or psychotherapy for management of conditions
  like Attention-Deficit/Hyperactivity Disorder, Obsessive Compulsory
  Disorder.
\item
  Ancillary Services -- Occupational therapy, speech therapy, physical
  therapy
\end{itemize}

\section{Complications}\label{complications-39}

Some compliations associated with Williams Syndrome are malnutrition,
Left Ventricular Hypertrophy, \href{cvs-heart-failure.qmd}{Heart
Failure}, Myocardial Infarction, Dysrhythmias, Acute Pancreatitis,
Cholelithiasis, Peptic Ulcer Disease, Renal Artery Stenosis, Cataract,
Chronic Otitis Media, Sudden death.

\section{Notes on counselling}\label{notes-on-counselling-1}

The individual and the families/caretakers will be counselled on the
fact that the disorder is a genetic disorder with an autosomal dominant
inheritance thus there is at least a 50\% chance of the offspring of the
affected individual having the disorder. There is however no available
cure for the underlying disorder, some associated disorders can be
treated or managed. Affected individuals will also be advised to have
preconception counselling with the Obstetrician and Gynecologist if
he/she wishes to have children.

\chapter{Fetal Alcohol Syndrome}\label{fetal-alcohol-syndrome}

\section{Definition}\label{definition-46}

Fetal Alcohol Spectrum Disorder (FASD) is a continuum of adverse
outcomes described as preventable central nervous system dysfunction and
birth defects that result from prenatal alcohol exposure. The term has
been used to describe 4 main entities namely:

\begin{itemize}
\tightlist
\item
  Fetal alcohol syndrome (FAS)
\item
  Partial fetal alcohol syndrome (PFAS)
\item
  Alcohol-related neurodevelopmental disorder (ARND) and
\item
  Alcohol-related birth defects (ARBDs)
\end{itemize}

FAS is considered the most severe and distinct subset of FASD. The
diagnosis is based on findings in three fundamental areas:
characteristic facial dysmorphology, growth retardation and CNS
involvement. In addition to these three primary features, prenatal
alcohol exposure impairs cardiogenesis and subsequently results in
congenital heart disease occurring in about 40 to 54\% of children
diagnosed with FAS. The estimated global prevalence of FASD among the
general population is 7.7 cases per 1,000 individuals. In Africa,
studies have been done in South Africa, where prevalence is among the
highest reported in the world.

\section{Pathogenesis}\label{pathogenesis-8}

Alcohol is considered a teratogen that causes irreversible damage when
taken during pregnancy. Although its effect can affect every stage of
pregnancy, the risk of facial anomalies and major structural anomalies
occurs with exposure during the first trimester, with irreversible
damage. The possible teratogenic effects depend on the timing,
frequency, duration, amount of alcohol exposure as well as genetic
susceptibility. Ethanol and its metabolite acetaldehyde can alter fetal
development by disrupting cellular differentiation and growth,
disrupting DNA and protein synthesis and inhibiting cell migration
resulting in the various anomalies that can occur. The pathogenesis of
congenital heart disease in FAS is not well known but it is hypothesised
that fetal alcohol exposure has deleterious effects on the composition
of the cardiac extracellular matrix or cardiac fibroblasts based on
studies performed in animals (mice).

Additionally, prenatal alcohol exposure has been shown to impair cardiac
development through numerous cellular mechanisms, including reduction of
retinoic acid levels, apoptosis of neural crest cells, suppression of
histone acetylation, and altered gene expression. Animal studies have
shown that prenatal alcohol exposure can induce a range of abnormalities
in heart structure and function, including thinning of the ventricular
walls, decreased ejection fraction (EF), reduced heart weight, and left
ventricular hypertrophy. Overt congenital heart defect phenotypes such
as \href{cvs-vsd.qmd}{ventricular septal defect}, atrioventricular canal
malformation, cardiac chamber malformations, great vessel defects, and
double outlet right ventricle have also been identified in animals.

\section{Clinical Features}\label{clinical-features-78}

The clinical features of FAS require a history of prenatal alcohol
exposure and 1. At least 2 characteristic facial features
(characteristical short palpebral fissures, thin vermillion border, and
smooth philtrum). Other facial featues are microcephaly, small palpebral
fissures, low nasal bridge, flat midface, epicanthal folds and railroad
track ears, 2. Prenatal and postnatal growth retardation (height and//or
weight \textless10th centile); 3. Deficient brain growth, abnormal
morphogenesis, or abnormal neurophysiology, including at least one of
the following: - Head circumference ≤10th percentile - Structural brain
anomalies; - Recurrent non-febrile seizures (other causes of seizures
having been ruled out) and - Neurobehavioral impairment. 4. Associated
CHDs may include \href{cvs-asd.qmd}{atrial septal defect},
\href{cvs-vsd.qmd}{ventricular septal defect}, d-transposition of great
arteries, \href{cvs-pda.qmd}{patent ductus arteriosus}, endocardial
cushion defects and conotruncal heart defects (eg, aberrant great
vessels, \href{cvs-tof.qmd}{Tetralogy of Fallot})

\section{Associations}\label{associations-2}

The associations of FAS include other disorders of the FASD that require
a longitudinal study that assesses growth, neurodevelopment features and
facial dysmorphology to identify these children up to the ages of 9-18
months. Other associations of prenatal alcohol exposure include
miscarriage, still birth,
\href{neo-preterm-low-birth-weight.qmd}{preterm birth and low birth
weight}, and developmental delays.

\section{Investigation}\label{investigation-5}

\subsection{Prenatal}\label{prenatal-1}

\begin{enumerate}
\def\labelenumi{\arabic{enumi}.}
\tightlist
\item
  Optical coherence tomography (OCT) -- Can be used to identify
  embryonic structure and cardiac anomalies.
\end{enumerate}

\subsection{Post-natal.}\label{post-natal.}

\begin{itemize}
\tightlist
\item
  MRI and CT scan for structural brain abnormalities.
\item
  Echocardiogram
\item
  Neurodevelopmental evaluation
\end{itemize}

\section{Treatment}\label{treatment-29}

FAS has a varied presentation of congenital anomalies with attendant CNS
effects; therefore, management must be individualised based on the
strengths and needs of the patient and their families. Typically, these
children require a multidisciplinary approach to their management.
Management is typically lifelong long and interventions may change over
time based on the needs identified. Treatment modalities may include
referral to infant developmental services, vision and hearing screening,
preschool speech and language therapy, school-based support for learning
disorders, occupational and physical therapy, behavioural and
psychological interventions, pharmacotherapy, vocational support, and
support for independent living in adolescence and adulthood. Specialized
medical or surgical interventions may be required for congenital
anomalies and accompanying comorbidities.

\section{Complications}\label{complications-40}

Children with FASD, with neurological impairment, may lead to lifelong
``secondary disabilities'' which include academic failure, inappropriate
sexual behaviour, disrupted school experience, trouble with the law and
incarceration, homelessness, unemployment, substance use, chronic mental
health problems, premature death (most likely from impulsivity and poor
judgment, such as car accidents or human immunodeficiency virus (HIV)
infection, or from comorbidities such as substance use, unhealthy
lifestyle, suicide, or homicide)

\section{Counselling}\label{counselling-4}

In counselling the relatives of a child with FAS, it is important to
eliminate the stigmatization of the mother. Counselling should emphasise
the importance of an individualised lifelong treatment plan for the
child. For subsequent pregnancies, mothers must be counselled on the
abstinence from drinking alcohol before conception and during pregnancy
as there are no safe levels of alcohol in pregnancy.

\chapter{Fragile X Syndrome}\label{fragile-x-syndrome}

\section{Definition}\label{definition-47}

Fragile X syndrome, previously known as Martin-Bell syndrome or marker X
syndrome is the commonest cause of X-linked intellectual disability and
constitutes about 30\% of all X-linked learning disabilities as well as
emotional problems ranging from mood instability to autism. Prevalence
varies according to the method of diagnosis but has been estimated at
1;4000 in males and 1;6000 in females but may rise to 1:2000 in the
general population when mildly affected individuals are included.

\section{Genetics}\label{genetics-7}

It is an X-linked dominant disorder with variable expressivity and
reduced penetrance. Fragile X syndrome (FXS) was the first of a group of
genetic conditions called trinucleotide repeat expansion disorders. The
FMR1 gene responsible for FXS encodes a protein called FMRP that is
important for brain development. When a female carries this premutation,
it usually expands to a full mutation of more than 200 CGG repeats in
her offspring. The variable clinical phenotypes arise from a deficiency
or complete absence of FRMP and correlate with the size of the
expansion. Note that point mutations, deletions and missense mutations
can also result in FXS.

\section{Clinical features}\label{clinical-features-79}

Presentation depends on the sex, age, mutation state (full mutation vs
premutation), degree of methylation, magnitude of FMRP deficit and
mosaicism. Females tend to have milder expression. Features may be
classified into physical and cognitive aspects.

\subsection{Full mutation}\label{full-mutation}

\begin{enumerate}
\def\labelenumi{\arabic{enumi}.}
\tightlist
\item
  Physical

  \begin{enumerate}
  \def\labelenumii{\alph{enumii}.}
  \tightlist
  \item
    Facial features: Elongated face, high arched palate, large, cupped
    ears
  \item
    Connective tissue anomalies: Mitral valve prolapse, aortic root
    dilatation, scoliosis, flat feet, and hyperflexible joints.
  \item
    Other features: Hypotonia (infancy), macroorchidism (commonly
    post-pubertal), seizures, recurrent otitis media, strabismus, and
    refraction errors.
  \end{enumerate}
\item
  Cognitive features

  \begin{enumerate}
  \def\labelenumii{\alph{enumii}.}
  \tightlist
  \item
    Intellectual disability and developmental delay
  \item
    Autism 20-0\%
  \item
    ADHD 80\%
  \item
    Anxiety (70-100\%)
  \end{enumerate}
\end{enumerate}

\subsection{Permutation:}\label{permutation}

Affected females have been noted to develop premature ovarian failure,
Fragile X-associated tremor-ataxia syndrome (FXTAS) occurs in later life
along with other neurocognitive defects.

\section{Associations}\label{associations-3}

Fragile X syndrome is associated with autism spectrum disorders and
ADHD. Obsessive-compulsive behaviour has also been noted. Others include
Obstructive sleep apnoea, and urinary tract anomalies eg. Vesicoureteral
reflux.

\section{Investigations}\label{investigations-55}

Early diagnosis is important for timely genetic counselling. Genetic
testing for FMR1 DNA analysis is the recommended investigation. A
combination of PCR (measures CGG repeats) and Southern blot of genomic
DNA (determines methylation status) yields a sensitivity of 99\%. Other
investigations may be required for certain features of the disease
e.g.~an Echocardiogram for mitral valve prolapse.

\section{Treatment}\label{treatment-30}

There is currently no cure for FXS. A variety of interventions are
employed which work synergistically in ameliorating the symptoms. These
include.

\begin{itemize}
\tightlist
\item
  Counselling (including genetic counselling) and testing of family
  members.
\item
  Psychopharmacology

  \begin{itemize}
  \tightlist
  \item
    ADHD -- stimulants eg. Methylphenidate
  \item
    Anxiety and compulsive behaviours -- antidepressants eg. SSRIs
  \item
    Mood instability, aggression -- antipsychotics eg. Risperidone
  \item
    Seizure control eg. Carbamazepine, valproate. Phenytoin and
    phenobarbitone are not recommended.
  \end{itemize}
\item
  Speech and language therapy.
\item
  Motor therapy
\item
  Individualized educational plans
\item
  Health surveillance
\item
  Behaviour interventions tailored to maximise functioning. As such
  treatment requires a multidisciplinary approach.
\end{itemize}

\section{Complications}\label{complications-41}

Strongly associated with the symptoms of the condition eg. Chronic
otitis media and Epilepsy.

\section{Notes on Counselling}\label{notes-on-counselling-2}

It is important to note that life expectancy is the same as in the
general population though quality of life varies depending on the
associated features and degree of intellectual disability. Reproductive
considerations must also be discussed: Males with full mutations are
infertile, males with premutation are fertile carriers, Females with
full mutation are fertile, and females with premutation have impaired
fertility consequent from premature ovarian failure.

\chapter{Prada-Willi Syndrome}\label{prada-willi-syndrome}

\section{Introduction}\label{introduction-70}

Prader-Willi Syndrome (PWS) is a complex and rare genetic disorder that
affects multiple systems in the body, leading to developmental,
cognitive, and behavioral challenges. It is characterized by hypotonia
(low muscle tone), feeding difficulties in infancy, hyperphagia
(excessive hunger) in childhood and adulthood, and obesity-related
complications. With proper management, individuals with PWS can lead
healthier and more fulfilling lives.

\section{Definition}\label{definition-48}

Prader-Willi Syndrome is a genetic disorder caused by the loss of
function of specific genes on the paternal copy of chromosome 15
(15q11-q13). It is considered the most common genetic cause of
life-threatening obesity. Individuals with PWS often exhibit a
characteristic phenotype, including short stature, intellectual
disability, behavioral issues, and endocrine abnormalities.

\section{Genetics}\label{genetics-8}

Prader-Willi Syndrome is most commonly caused by the absence of
expression of paternally inherited genes on chromosome 15. Three main
genetic mechanisms are responsible for PWS:

\begin{enumerate}
\def\labelenumi{\arabic{enumi}.}
\tightlist
\item
  Paternal Deletion (about 70\% of cases): A segment of the paternal
  chromosome 15 is deleted, preventing normal gene expression.
\item
  Maternal Uniparental Disomy (UPD) (about 25\% of cases): The
  individual inherits two copies of chromosome 15 from the mother and
  none from the father.
\item
  Imprinting Defects (1-3\% of cases): Errors in genomic imprinting
  prevent proper gene activation. The region affected includes genes
  responsible for hypothalamic regulation, which plays a significant
  role in developing key features of the syndrome.
\end{enumerate}

\section{Clinical Features}\label{clinical-features-80}

Prader-Willi Syndrome presents a wide spectrum of clinical features that
evolve throughout an individual's life.

\begin{enumerate}
\def\labelenumi{\arabic{enumi}.}
\tightlist
\item
  \textbf{Neonatal and Infancy}:

  \begin{itemize}
  \tightlist
  \item
    Severe hypotonia (low muscle tone)
  \item
    Poor feeding and weak suck, often requiring feeding support
  \item
    Failure to thrive
  \item
    Delayed developmental milestones
  \end{itemize}
\item
  \textbf{Childhood and Adulthood}:

  \begin{itemize}
  \tightlist
  \item
    Hyperphagia (uncontrollable hunger)
  \item
    Rapid weight gain, leading to obesity if not managed
  \item
    Short stature
  \item
    Hypogonadism (underdeveloped sexual characteristics)
  \item
    Intellectual disabilities and learning difficulties
  \item
    Behavioral issues, such as temper tantrums and obsessive-compulsive
    tendencies
  \end{itemize}
\item
  \textbf{Physical Features}:

  \begin{itemize}
  \tightlist
  \item
    Almond-shaped eyes
  \item
    Narrow forehead
  \item
    Small hands and feet
  \item
    Soft, fair skin
  \end{itemize}
\item
  \textbf{Endocrine Abnormalities}:

  \begin{itemize}
  \tightlist
  \item
    Growth hormone deficiency
  \item
    Hypothyroidism
  \item
    Insulin resistance
  \end{itemize}
\item
  \textbf{Cognitive and Behavioral Features}:

  \begin{itemize}
  \tightlist
  \item
    Mild to moderate intellectual disability
  \item
    Obsessive-compulsive behaviors
  \item
    Anxiety and mood instability
  \item
    Sleep disturbances
  \end{itemize}
\end{enumerate}

\section{Investigations}\label{investigations-56}

Clinical evaluation and genetic testing are essential for diagnosing
Prader-Willi Syndrome.

\begin{enumerate}
\def\labelenumi{\arabic{enumi}.}
\item
  \textbf{Genetic Testing}:

  \begin{itemize}
  \tightlist
  \item
    Methylation Analysis: Detects all known causes of PWS by identifying
    the absence of paternal gene expression.
  \item
    Fluorescence In Situ Hybridization (FISH): Detects deletions on
    chromosome 15.
  \item
    DNA Microarray Analysis: Helps identify chromosomal abnormalities.
  \item
    Uniparental Disomy Testing: Confirms cases due to maternal UPD.
  \end{itemize}
\item
  \textbf{Hormonal Assessment:}

  \begin{itemize}
  \tightlist
  \item
    Evaluation of growth hormone and thyroid function
  \item
    Assessment of sex hormone levels
  \end{itemize}
\item
  \textbf{Developmental and Cognitive Testing}:

  \begin{itemize}
  \tightlist
  \item
    Psychological assessments for intellectual and behavioral
    evaluations
  \end{itemize}
\item
  \textbf{Imaging Studies:}

  \begin{itemize}
  \tightlist
  \item
    Brain MRI in cases with unusual neurological presentations
  \end{itemize}
\end{enumerate}

\section{Treatment}\label{treatment-31}

Management of Prader-Willi Syndrome requires a comprehensive,
multidisciplinary approach.

\begin{enumerate}
\def\labelenumi{\arabic{enumi}.}
\item
  Nutritional and Weight Management:

  \begin{itemize}
  \tightlist
  \item
    Controlled dietary intake to prevent obesity
  \item
    Structured meal plans and avoidance of access to food
  \item
    Close monitoring of weight and metabolic parameters
  \end{itemize}
\item
  Growth Hormone Therapy (GHT):

  \begin{itemize}
  \tightlist
  \item
    Improves growth, body composition, muscle tone, and possibly
    cognitive function
  \end{itemize}
\item
  Hormonal Replacement:

  \begin{itemize}
  \tightlist
  \item
    Testosterone for males and estrogen/progesterone for females to
    induce secondary sexual characteristics
  \item
    Thyroid hormone replacement if needed
  \end{itemize}
\item
  Behavioral and Psychological Interventions:

  \begin{itemize}
  \tightlist
  \item
    Behavioral therapy for tantrums and obsessive-compulsive behaviors
  \item
    Psychological counseling for anxiety and emotional instability
  \end{itemize}
\item
  Educational Support:

  \begin{itemize}
  \tightlist
  \item
    Special education services tailored to individual needs
  \item
    Speech and occupational therapy for communication and motor skill
    development
  \end{itemize}
\item
  Physical Activity:

  \begin{itemize}
  \tightlist
  \item
    Encouragement of regular physical exercise to maintain muscle
    strength and control weight
  \end{itemize}
\item
  Sleep Management:

  \begin{itemize}
  \tightlist
  \item
    Treatment for sleep apnea, often present in individuals with obesity
  \end{itemize}
\item
  Pharmacological Treatments:

  \begin{itemize}
  \tightlist
  \item
    Medications for mood stabilization, anxiety, or behavioral
    management when necessary
  \end{itemize}
\end{enumerate}

\section{Counseling}\label{counseling-1}

Genetic counseling plays a crucial role in supporting affected families
and individuals.

\begin{enumerate}
\def\labelenumi{\arabic{enumi}.}
\item
  \textbf{Diagnosis Counseling}: Parents are guided through the genetic
  basis of the condition and its clinical implications.
\item
  \textbf{Reproductive Counseling}:

  \begin{itemize}
  \tightlist
  \item
    Discussion of recurrence risks (less than 1\% in most cases but
    higher if imprinting errors are present)
  \item
    Prenatal testing options, such as chorionic villus sampling (CVS)
    and amniocentesis
  \end{itemize}
\item
  \textbf{Psychosocial Support:}

  \begin{itemize}
  \tightlist
  \item
    Addressing emotional and mental health challenges faced by families
  \item
    Providing connections to support groups and advocacy organizations
  \end{itemize}
\end{enumerate}

\section{Conclusion}\label{conclusion-37}

Prader-Willi Syndrome is a complex genetic disorder that poses numerous
challenges throughout an individual's life. However, advances in genetic
testing, growth hormone therapy, and behavioral interventions have
greatly improved outcomes for affected individuals. A structured,
multidisciplinary approach to care ensures that individuals with PWS can
achieve their developmental potential and maintain better health.
Ongoing research and awareness efforts remain essential to further
understanding and improving the management of this rare condition.

\part{{Parasitic Infections}}

\chapter{Overview of Parasitic
Infections}\label{overview-of-parasitic-infections}

\section{Introduction}\label{introduction-71}

Parasitic infections remain a major public health concern in the
tropics, particularly in \textbf{Ghana} and the wider
\textbf{sub-Saharan African region}, where they significantly contribute
to morbidity and mortality among children. These infections are closely
linked to environmental, socioeconomic, and behavioural factors such as
poverty, inadequate sanitation, unsafe water supply, and limited access
to healthcare.

In paediatric practice, parasitic diseases commonly present as
\textbf{anaemia}, \textbf{malnutrition}, \textbf{growth failure},
\textbf{diarrhoea}, and \textbf{neurological complications}, often
exacerbating other ongoing health conditions. Understanding the basic
biology, classification, and epidemiology of parasites is crucial for
diagnosis, treatment, and preventive interventions in endemic areas.

This chapter provides an overview of parasitic infections, defines key
concepts, and discusses the classification, modes of transmission, and
epidemiological patterns with specific reference to Ghana and
sub-Saharan Africa.

\section{Basic Definitions}\label{basic-definitions}

A \textbf{parasite} is an organism that lives on or within another
living organism, known as the \textbf{host}, deriving benefit at the
host's expense. The relationship is typically one-sided, with the
parasite gaining nourishment, shelter, or other support while causing
harm or disease to the host.

\subsection{Key Terminologies}\label{key-terminologies}

\begin{itemize}
\item
  \textbf{Host:} The organism that harbours the parasite.

  \begin{itemize}
  \tightlist
  \item
    \emph{Definitive host} -- harbours the adult or sexually mature
    stage.\\
  \item
    \emph{Intermediate host} -- harbours the larval or asexual stage.\\
  \item
    \emph{Paratenic (transport) host} -- carries the parasite without
    further development.
  \end{itemize}
\item
  \textbf{Vector:} An organism, often an arthropod (e.g., mosquito,
  tsetse fly), that transmits the parasite from one host to another.
\item
  \textbf{Reservoir host:} Animals that harbour parasites infectious to
  humans, maintaining the life cycle in nature.
\item
  \textbf{Infection vs.~Infestation:}

  \begin{itemize}
  \tightlist
  \item
    \emph{Infection} refers to internal invasion by protozoa or
    helminths.\\
  \item
    \emph{Infestation} refers to external colonization by ectoparasites
    (e.g., lice, ticks, mites).
  \end{itemize}
\item
  \textbf{Endoparasites} live inside the host's body (e.g.,
  \emph{Plasmodium}, \emph{Ascaris lumbricoides}), while
  \textbf{ectoparasites} live on the host's body surface (e.g.,
  \emph{Pediculus humanus capitis}).
\end{itemize}

Parasites have evolved to coexist with their hosts, but in
humans---especially in young or malnourished children---these
relationships often result in overt disease.

\section{Classification of Parasites}\label{classification-of-parasites}

Parasites of medical importance are broadly classified into three
groups: \textbf{protozoa}, \textbf{helminths}, and \textbf{arthropods}.

\subsection{Protozoa}\label{protozoa}

Protozoa are \textbf{unicellular microscopic organisms} that reproduce
by binary fission or multiple fission. Many are motile and can survive
in host tissues or extracellular fluids. They are responsible for some
of the most severe parasitic diseases in children.

\subsubsection{Major Groups of Protozoa}\label{major-groups-of-protozoa}

\begin{longtable}[]{@{}
  >{\raggedright\arraybackslash}p{(\linewidth - 6\tabcolsep) * \real{0.2500}}
  >{\raggedright\arraybackslash}p{(\linewidth - 6\tabcolsep) * \real{0.2500}}
  >{\raggedright\arraybackslash}p{(\linewidth - 6\tabcolsep) * \real{0.2500}}
  >{\raggedright\arraybackslash}p{(\linewidth - 6\tabcolsep) * \real{0.2500}}@{}}
\toprule\noalign{}
\begin{minipage}[b]{\linewidth}\raggedright
Group
\end{minipage} & \begin{minipage}[b]{\linewidth}\raggedright
Locomotory Organ
\end{minipage} & \begin{minipage}[b]{\linewidth}\raggedright
Example
\end{minipage} & \begin{minipage}[b]{\linewidth}\raggedright
Disease
\end{minipage} \\
\midrule\noalign{}
\endhead
\bottomrule\noalign{}
\endlastfoot
Amoebae & Pseudopodia & \emph{Entamoeba histolytica} & Amoebic
dysentery, liver abscess \\
Flagellates & Flagella & \emph{Giardia lamblia}, \emph{Trypanosoma
brucei}, \emph{Leishmania donovani} & Giardiasis, African sleeping
sickness, leishmaniasis \\
Ciliates & Cilia & \emph{Balantidium coli} & Balantidiasis \\
Sporozoa & None (complex life cycle) & \emph{Plasmodium spp.},
\emph{Toxoplasma gondii}, \emph{Cryptosporidium spp.} & Malaria,
toxoplasmosis, cryptosporidiosis \\
\end{longtable}

\subsubsection{Clinical Relevance in
Ghana}\label{clinical-relevance-in-ghana}

In Ghana, \textbf{malaria}, caused by \emph{Plasmodium falciparum}, is
by far the most important protozoal infection. It accounts for a
significant proportion of outpatient visits and hospital admissions in
children, with high mortality if untreated. Other protozoal infections
such as \textbf{giardiasis} and \textbf{cryptosporidiosis} are
increasingly recognized as causes of \textbf{persistent diarrhoea} in
both immunocompetent and immunocompromised children, including those
with HIV infection.

\subsection{Helminths}\label{helminths}

Helminths are \textbf{multicellular worms} that infect various tissues
and organs. They are typically large enough to be seen with the naked
eye and are classified into three main groups: nematodes, cestodes, and
trematodes.

\subsubsection{Nematodes -- Roundworms}\label{nematodes-roundworms}

These are cylindrical worms with separate sexes.

\textbf{Examples:} - \emph{Ascaris lumbricoides} (roundworm) -
\emph{Ancylostoma duodenale} and \emph{Necator americanus} (hookworms) -
\emph{Trichuris trichiura} (whipworm) - \emph{Strongyloides stercoralis}
- \emph{Enterobius vermicularis} (pinworm)

\textbf{Clinical relevance:} In Ghana, soil-transmitted helminth
infections remain endemic in many rural areas, contributing to chronic
anaemia, malnutrition, and impaired cognitive development among
school-aged children.

\subsubsection{Cestodes -- Tapeworms}\label{cestodes-tapeworms}

These are flat, ribbon-like worms with segmented bodies called
\textbf{proglottids}.

\textbf{Examples:} - \emph{Taenia saginata} (beef tapeworm) -
\emph{Taenia solium} (pork tapeworm) - \emph{Diphyllobothrium latum}
(fish tapeworm) - \emph{Hymenolepis nana} (dwarf tapeworm)

\textbf{Clinical relevance:} Tapeworm infections are relatively uncommon
in Ghana compared to nematodes, but \emph{Taenia solium} poses a risk of
\textbf{neurocysticercosis}, an emerging cause of seizures in children
and adults.

\subsubsection{Trematodes -- Flukes}\label{trematodes-flukes}

These are leaf-shaped flatworms that require at least one intermediate
host (usually a snail).

\textbf{Examples:} - \emph{Schistosoma haematobium} -- urinary
schistosomiasis - \emph{Schistosoma mansoni} -- intestinal
schistosomiasis

\textbf{Clinical relevance:} Schistosomiasis remains endemic in the
\textbf{Volta Basin}, \textbf{northern Ghana}, and other water-rich
communities. In children, it leads to \textbf{haematuria},
\textbf{hepatosplenomegaly}, \textbf{anaemia}, and \textbf{growth
retardation}.

\subsection{Arthropods}\label{arthropods}

Arthropods are \textbf{invertebrates with jointed appendages} and
chitinous exoskeletons. They are important as \textbf{vectors},
\textbf{intermediate hosts}, or \textbf{direct causes} of disease
through bites, stings, or infestation.

\subsubsection{Common Arthropods in Medical
Practice}\label{common-arthropods-in-medical-practice}

\begin{longtable}[]{@{}
  >{\raggedright\arraybackslash}p{(\linewidth - 4\tabcolsep) * \real{0.3333}}
  >{\raggedright\arraybackslash}p{(\linewidth - 4\tabcolsep) * \real{0.3333}}
  >{\raggedright\arraybackslash}p{(\linewidth - 4\tabcolsep) * \real{0.3333}}@{}}
\toprule\noalign{}
\begin{minipage}[b]{\linewidth}\raggedright
Group
\end{minipage} & \begin{minipage}[b]{\linewidth}\raggedright
Example
\end{minipage} & \begin{minipage}[b]{\linewidth}\raggedright
Role in Disease
\end{minipage} \\
\midrule\noalign{}
\endhead
\bottomrule\noalign{}
\endlastfoot
Mosquitoes & \emph{Anopheles}, \emph{Aedes}, \emph{Culex} & Vectors of
malaria, filariasis, dengue \\
Tsetse flies & \emph{Glossina spp.} & Vector of African
trypanosomiasis \\
Sandflies & \emph{Phlebotomus spp.} & Vector of leishmaniasis \\
Blackflies & \emph{Simulium spp.} & Vector of onchocerciasis \\
Fleas, lice, ticks, mites & Various species & Ectoparasitic
infestations; vectors of typhus, rickettsial infections \\
\end{longtable}

In Ghana, \textbf{malaria-transmitting Anopheles mosquitoes} are the
most significant arthropod vector, while \textbf{blackflies} are
responsible for \textbf{onchocerciasis} (river blindness) in some
riverine communities.

\section{Modes of Transmission of Parasitic
Infections}\label{modes-of-transmission-of-parasitic-infections}

Parasites employ diverse routes to infect humans. Understanding these is
crucial for preventive strategies.

\begin{enumerate}
\def\labelenumi{\arabic{enumi}.}
\tightlist
\item
  \textbf{Faeco--oral transmission} -- ingestion of cysts, oocysts, or
  eggs from contaminated water or food (\emph{E. histolytica},
  \emph{Giardia lamblia}, \emph{Ascaris lumbricoides}).
\item
  \textbf{Vector-borne transmission} -- via arthropods such as
  mosquitoes, tsetse flies, and sandflies (\emph{Plasmodium},
  \emph{Trypanosoma}, \emph{Leishmania}).
\item
  \textbf{Skin penetration} -- infective larvae penetrate intact skin
  (\emph{Schistosoma}, hookworms, \emph{Strongyloides}).
\item
  \textbf{Transplacental or perinatal transmission} -- \emph{Toxoplasma
  gondii}, \emph{Trypanosoma cruzi}.
\item
  \textbf{Ingestion of undercooked meat or fish} -- \emph{Taenia},
  \emph{Trichinella}, \emph{Diphyllobothrium}.
\item
  \textbf{Person-to-person contact or fomites} -- \emph{Enterobius
  vermicularis}, lice, scabies.
\end{enumerate}

In Ghana, the \textbf{faeco-oral route} and \textbf{vector-borne
transmission} dominate, reflecting poor sanitation and high vector
density due to favourable climatic conditions.

\section{Epidemiology of Parasitic Infections in Ghana and Sub-Saharan
Africa}\label{epidemiology-of-parasitic-infections-in-ghana-and-sub-saharan-africa}

\subsection{Burden and Impact}\label{burden-and-impact}

Parasitic infections constitute one of the \textbf{most widespread
causes of childhood morbidity} in sub-Saharan Africa. They coexist with
malnutrition, anaemia, and chronic infections like tuberculosis and HIV,
creating a vicious cycle of ill health and poor development.

In Ghana, the \textbf{Ghana Health Service (GHS)} and \textbf{Neglected
Tropical Disease Control Programme} have prioritized the control of
malaria, schistosomiasis, lymphatic filariasis, and soil-transmitted
helminths through \textbf{mass drug administration},
\textbf{insecticide-treated nets}, and \textbf{health education}.

\subsection{Key Epidemiological
Highlights}\label{key-epidemiological-highlights}

\begin{longtable}[]{@{}
  >{\raggedright\arraybackslash}p{(\linewidth - 6\tabcolsep) * \real{0.2500}}
  >{\raggedright\arraybackslash}p{(\linewidth - 6\tabcolsep) * \real{0.2500}}
  >{\raggedright\arraybackslash}p{(\linewidth - 6\tabcolsep) * \real{0.2500}}
  >{\raggedright\arraybackslash}p{(\linewidth - 6\tabcolsep) * \real{0.2500}}@{}}
\toprule\noalign{}
\begin{minipage}[b]{\linewidth}\raggedright
Disease
\end{minipage} & \begin{minipage}[b]{\linewidth}\raggedright
Agent
\end{minipage} & \begin{minipage}[b]{\linewidth}\raggedright
Mode of Transmission
\end{minipage} & \begin{minipage}[b]{\linewidth}\raggedright
Epidemiology in Ghana
\end{minipage} \\
\midrule\noalign{}
\endhead
\bottomrule\noalign{}
\endlastfoot
Malaria & \emph{Plasmodium falciparum} & Bite of female \emph{Anopheles}
mosquito & Endemic nationwide, seasonal peaks during rains \\
Schistosomiasis & \emph{S. haematobium}, \emph{S. mansoni} & Skin
penetration in infested water & Endemic in Volta, Northern, and Ashanti
regions \\
Soil-transmitted helminths & \emph{Ascaris}, \emph{Trichuris},
\emph{Hookworm} & Faeco-oral or skin penetration & Common in school-aged
children in rural areas \\
Onchocerciasis & \emph{Onchocerca volvulus} & Blackfly bite & Focal
along river basins \\
Filariasis & \emph{Wuchereria bancrofti} & Mosquito bite & Endemic in
parts of Northern Ghana \\
\end{longtable}

\section{Control and Prevention
Strategies}\label{control-and-prevention-strategies}

Effective control requires \textbf{multisectoral interventions}
integrating health, water, sanitation, and education systems.

\begin{enumerate}
\def\labelenumi{\arabic{enumi}.}
\tightlist
\item
  \textbf{Environmental sanitation:} Proper disposal of faeces, improved
  water supply, and waste management.
\item
  \textbf{Vector control:} Use of insecticide-treated bed nets, indoor
  residual spraying, and elimination of vector breeding sites.
\item
  \textbf{Mass drug administration (MDA):} Periodic deworming with
  albendazole, praziquantel, and ivermectin under the Ghana NTD
  programme.
\item
  \textbf{Health education:} Promoting hand hygiene, safe food
  practices, and community awareness.
\item
  \textbf{Nutrition support:} Addressing micronutrient deficiencies to
  reduce susceptibility to infection.
\item
  \textbf{Surveillance and research:} Strengthening laboratory capacity
  and epidemiological monitoring, especially in children.
\end{enumerate}

\section{Summary}\label{summary-6}

Parasitic infections remain a significant public health challenge in
Ghana and across sub-Saharan Africa, with children being the most
vulnerable group. Understanding their biology, modes of transmission,
and epidemiological context is essential for effective prevention,
diagnosis, and treatment.\\
Sustained control will require continued investment in sanitation,
health education, and access to quality paediatric care.

\section{Further Reading}\label{further-reading-10}

\begin{enumerate}
\def\labelenumi{\arabic{enumi}.}
\tightlist
\item
  \textbf{Kliegman RM et al.} Nelson Textbook of Pediatrics, 22nd
  Edition. Elsevier; 2023.\\
\item
  \textbf{Ghana Health Service.} Neglected Tropical Diseases Control
  Programme Annual Report, 2022.\\
\item
  \textbf{WHO.} World Malaria Report 2023. Geneva: World Health
  Organization.\\
\item
  \textbf{Hotez PJ, Kamath A.} Neglected Tropical Diseases in
  Sub-Saharan Africa: Review of Their Burden and Distribution.
  \emph{PLoS Neglected Tropical Diseases}. 2009;3(8):e412.\\
\item
  \textbf{Addo KK et al.} Intestinal Parasites among School Children in
  Accra. \emph{Ghana Medical Journal}. 2018;52(2):64--70.\\
\item
  \textbf{Brooker SJ et al.} Soil-Transmitted Helminth Infections in
  Africa: Distribution, Burden and Control. \emph{PLoS Negl Trop Dis}.
  2015;9(3):e0003565.
\end{enumerate}

\chapter{Malaria}\label{malaria-1}

\section{Introduction}\label{introduction-72}

Malaria is an infectious disease caused by plasmodium species. Five
types of species are known to affect man. \emph{Plasmodium falciparum},
\emph{Plasmodium vivax}, \emph{Plasmodium malaria}, \emph{Plasmodium
ovale}, and \emph{Plasmodium knowlesi}.(Poespoprodjo et al. 2023) In
sub-Saharan Africa, \emph{Plasmodium falciparum} infection is known to
be the highest cause of malaria and the most complicated form of malaria
in populations at risk of disease. The P. falciparum causes most
mortality from the disease. The geographic distribution of the species
on the African continent is well described.(Poespoprodjo et al. 2023)
The World Health Organization recognises malaria as an important disease
affecting children and pregnant women in sub-Saharan Africa and among
the diseases contributing to the high under-five mortality.(World Health
Organization 2023)

\section{Epidemiology}\label{epidemiology-18}

\textbf{Global}: The global estimate of malaria based on 2022 data
estimated over 249 million cases with about 608,000 deaths. There was
approximately a 28\% reduction in cases and a 50\% reduction in
mortality between 2000-2022. Ninety per cent of the cases and deaths
emanate from sub-Saharan Africa.(World Health Organization 2023)

\textbf{Regional}: In Africa, there were 232 million cases representing
about 94\% of global cases and about 580,000 deaths. Nigeria, Chad,
Niger, Sudan and DRC are among the top 5 countries with the highest
disease burdens. This is attributable to poverty, weak health systems
and environmental conditions.(World Health Organization 2023)

\textbf{Country}: In Ghana, malaria is the third leading cause of
under-5 morbidities and mortality. Pneumonia and diarrhoea account for
the first and second causes of under-five morbidities. The disease is
endemic in rural, overpopulated communities and slums. The northern
region of Ghana experiences peak incidence during the rainy season.

\section{Presentation}\label{presentation-2}

\textbf{Uncomplicated malaria}: Uncomplicated malaria is associated with
fever, chills, rigours, lethargy, nausea, vomiting, poor feeding, and
diarrhoea in younger children. Older children present with headaches,
myalgia, and abdominal pain. It is generally characterised by
non-specific clinical features that are well documented in the 2013
\href{https://factsforlife.org/pdf/management-of-severe-malaria-eng.pdf}{WHO
Handbook on Management of Severe Malaria}.

\textbf{Complicated malaria}: The transition between uncomplicated to
complicated malaria is not predictable. Therefore, there is a need for a
high index of suspicion of complicated malaria in managing children with
malaria.(K. Marsh et al. 1996; Kevin Marsh et al. 1995)

\begin{tcolorbox}[enhanced jigsaw, toprule=.15mm, left=2mm, leftrule=.75mm, opacitybacktitle=0.6, opacityback=0, arc=.35mm, toptitle=1mm, colbacktitle=quarto-callout-note-color!10!white, title=\textcolor{quarto-callout-note-color}{\faInfo}\hspace{0.5em}{Note}, titlerule=0mm, breakable, colframe=quarto-callout-note-color-frame, bottomtitle=1mm, colback=white, rightrule=.15mm, bottomrule=.15mm, coltitle=black]

There is no intermediary stage called \textbf{moderate malaria}.

\end{tcolorbox}

Clinical parameters that signal the presence of complicated (severe)
malaria include clinical signs and symptoms with laboratory indicators.
Table~\ref{tbl-gen-sev-malarai-clinical} and
Table~\ref{tbl-gen-sev-malarai-lab} outline the features suggestive of
complicated malaria.

\begin{longtable}[]{@{}
  >{\raggedright\arraybackslash}p{(\linewidth - 4\tabcolsep) * \real{0.0500}}
  >{\raggedright\arraybackslash}p{(\linewidth - 4\tabcolsep) * \real{0.4700}}
  >{\raggedright\arraybackslash}p{(\linewidth - 4\tabcolsep) * \real{0.4700}}@{}}
\caption{Signs and symptoms suggestive of complicated
malaria}\label{tbl-gen-sev-malarai-clinical}\tabularnewline
\toprule\noalign{}
\begin{minipage}[b]{\linewidth}\raggedright
SN
\end{minipage} & \begin{minipage}[b]{\linewidth}\raggedright
Clinical Signs and Symptoms
\end{minipage} & \begin{minipage}[b]{\linewidth}\raggedright
Comments
\end{minipage} \\
\midrule\noalign{}
\endfirsthead
\toprule\noalign{}
\begin{minipage}[b]{\linewidth}\raggedright
SN
\end{minipage} & \begin{minipage}[b]{\linewidth}\raggedright
Clinical Signs and Symptoms
\end{minipage} & \begin{minipage}[b]{\linewidth}\raggedright
Comments
\end{minipage} \\
\midrule\noalign{}
\endhead
\bottomrule\noalign{}
\endlastfoot
1 & Prostration & Weakness associated with the inability to sit, stand,
or walk. In younger children, the inability to suckle \\
2 & \begin{minipage}[t]{\linewidth}\raggedright
\textbf{Central Nervous System}

\begin{enumerate}
\def\labelenumi{\arabic{enumi}.}
\tightlist
\item
  Convulsions (Multiple \textgreater{} 2 in 24 hours)
\item
  Loss of consciousness (Blantyre Coma Score \textless5
\item
  Abnormal motor function
\item
  Abnormal Cranial nerve function. Specifically Cranial 3, 4,5, 6, 9 and

  \begin{enumerate}
  \def\labelenumii{\arabic{enumii}.}
  \setcounter{enumii}{9}
  \tightlist
  \item
  \end{enumerate}
\item
  Cotton wool and haemorrhage on the retina
\end{enumerate}
\end{minipage} & Blantyre Coma Score of \textless3 In the presence of
malaria parasite is classified as cerebral malaria. \\
3 & \begin{minipage}[t]{\linewidth}\raggedright
\textbf{Haematological}

\begin{enumerate}
\def\labelenumi{\arabic{enumi}.}
\tightlist
\item
  Severe pallor (anaemia)
\item
  Haemoglobinuria
\item
  Bleeding
\item
  Jaundice
\end{enumerate}
\end{minipage} & Obtained through history, physical examination and
complete blood count. \\
4 & \begin{minipage}[t]{\linewidth}\raggedright
\textbf{Renal}

\begin{enumerate}
\def\labelenumi{\arabic{enumi}.}
\tightlist
\item
  Oliguria
\item
  Anuria
\end{enumerate}
\end{minipage} & Measurement of the urine and fluid intake is critical
to estimate oliguria and anuria. \\
5 & \begin{minipage}[t]{\linewidth}\raggedright
\textbf{Respiratory}

\begin{enumerate}
\def\labelenumi{\arabic{enumi}.}
\tightlist
\item
  Respiratory Rate ~\textgreater{} age-specific limit
\item
  Oxygen saturation \textless{} 92\%
\item
  Pulmonary oedema
\item
  Chest wall indrawing
\item
  Crepitation on auscultation
\end{enumerate}
\end{minipage} & These could signal severe pneumonia. It should be
radiologically confirmed. \\
6 & \begin{minipage}[t]{\linewidth}\raggedright
\textbf{Cardiovascular}

\begin{enumerate}
\def\labelenumi{\arabic{enumi}.}
\tightlist
\item
  Reduce Capillary Refill Time.
\item
  Hypotension
\item
  Weak Pulse
\item
  Cold and clammy extremities
\end{enumerate}
\end{minipage} & Assessment based on clinical examination \\
\end{longtable}

\begin{longtable}[]{@{}
  >{\raggedright\arraybackslash}p{(\linewidth - 4\tabcolsep) * \real{0.0500}}
  >{\raggedright\arraybackslash}p{(\linewidth - 4\tabcolsep) * \real{0.4700}}
  >{\raggedright\arraybackslash}p{(\linewidth - 4\tabcolsep) * \real{0.4700}}@{}}
\caption{Laboratory features suggestive of complicated
malaria}\label{tbl-gen-sev-malarai-lab}\tabularnewline
\toprule\noalign{}
\begin{minipage}[b]{\linewidth}\raggedright
SN
\end{minipage} & \begin{minipage}[b]{\linewidth}\raggedright
Laboratory parameter
\end{minipage} & \begin{minipage}[b]{\linewidth}\raggedright
Comments
\end{minipage} \\
\midrule\noalign{}
\endfirsthead
\toprule\noalign{}
\begin{minipage}[b]{\linewidth}\raggedright
SN
\end{minipage} & \begin{minipage}[b]{\linewidth}\raggedright
Laboratory parameter
\end{minipage} & \begin{minipage}[b]{\linewidth}\raggedright
Comments
\end{minipage} \\
\midrule\noalign{}
\endhead
\bottomrule\noalign{}
\endlastfoot
1 & \textbf{Central Nervous System}

Coma Score of \textless3 In the presence of malaria parasite is
classified as cerebral malaria. & Rule out other causes of
encephalopathy, i.e.~Meningitis, encephalitis, hypoglycaemia, hepatic
and failure. \\
2 & \begin{minipage}[t]{\linewidth}\raggedright
\textbf{Haematological}

\begin{enumerate}
\def\labelenumi{\arabic{enumi}.}
\tightlist
\item
  Severe Anaemia Hb \textless5g/dL or PCV \textless15\%
\item
  Plasma Bilirubin \textgreater50 umol/L (3mg/dL)
\item
  Bleeding ~(thrombocytopenia)
\end{enumerate}
\end{minipage} & Bedside point-of-care devices and a complete blood
count are required. \\
3 & \begin{minipage}[t]{\linewidth}\raggedright
\textbf{Renal}

\begin{enumerate}
\def\labelenumi{\arabic{enumi}.}
\tightlist
\item
  Plasma creatinine~ \textgreater265umol/L (3mg/dL)
\item
  Blood urea \textgreater{} 20 mmol/L
\end{enumerate}
\end{minipage} & Blood chemistries are required to establish renal
complications. \\
4 & \begin{minipage}[t]{\linewidth}\raggedright
\textbf{Respiratory}

\begin{enumerate}
\def\labelenumi{\arabic{enumi}.}
\tightlist
\item
  Respiratory Rate
\item
  Oxygen saturation \textless{} 92\%
\end{enumerate}
\end{minipage} & Could signal severe pneumonia \\
5 & \begin{minipage}[t]{\linewidth}\raggedright
\textbf{Cardiovascular}

\begin{enumerate}
\def\labelenumi{\arabic{enumi}.}
\tightlist
\item
  Reduce Capillary Refill Time.
\item
  Reduce blood pressure
\item
  Weak Pulse
\item
  Cold and clammy extremities
\end{enumerate}
\end{minipage} & Basic bedside clinical examinations \\
6 & \begin{minipage}[t]{\linewidth}\raggedright
\textbf{Metabolic}

\begin{enumerate}
\def\labelenumi{\arabic{enumi}.}
\tightlist
\item
  Base deficit of \textgreater{} 8mEq/L
\item
  Plasma Bicarbonate~ of \textless15mmol/L
\item
  Lactate of \textless{} 5mmol/L
\item
  Hypoglycaemia~ \textless2.2 mmol/L
\end{enumerate}
\end{minipage} & Clinically, it will manifest as Kussmaul breathing
(deep, rapid breathing) \\
7 & \begin{minipage}[t]{\linewidth}\raggedright
Hyperparasitaemia:

\begin{quote}
\textgreater2\%/100 000/µL in low-intensity transmission areas or
\textgreater5\% or 250 000/µL in areas of high stable malaria
transmission intensity
\end{quote}
\end{minipage} & Microscopy in resource endowed institution \\
\end{longtable}

\section{Risk factors}\label{risk-factors-3}

There are genetic factors that protect against the development of
malaria. The genetic factor and the mechanism proposed to lead to the
protection against malaria are shown in Table 1 of the publication
titled
\href{https://malariajournal.biomedcentral.com/articles/10.1186/1475-2875-10-271\#citeas}{Genetic
polymorphisms linked to susceptibility to malaria} by Driss et al.
(2011). In sub-Saharan Africa, sickle cell anaemia and
glucose-6-phosphate dehydrogenase deficiency are the most common genetic
disorders.(Carter et al. 2011; Moeti et al. 2023)

\section{\texorpdfstring{\textbf{Management}}{Management}}\label{management-76}

\textbf{Uncomplicated malaria:} The WHO recommends Artemisinin-based
Combination Therapy (ACT) to manage uncomplicated malaria. Mono-therapy
is NOT recommended in patients living in endemic and high-transmission
settings. Treatment should be swift; therefore, home management of
malaria is highly recommended.

\textbf{Complicated malaria:} The aim of managing severe malaria is to
prevent death and complications. Severe malaria can result in death
within hours of progression from uncomplicated malaria to complicated.
Prompt and swift response to clinical changes is paramount.

The WHO recommends parenteral artesunate and appropriate support care
based on the clinical progression. Complicated malaria is a multi-system
disease; therefore, clinicians should prioritise their care, emphasising
supportive care. It is a medical emergency. Healthcare professionals
should identify life-threatening situations like hypoglycaemia and
hypoxaemia secondary to heart failure from complications such as severe
anaemia and pulmonary oedema, convulsion and other central nervous
manifestations from brain oedema and potentially raised intracranial
pressure. Children can develop compensated and decompensated shock. When
present, it should be corrected and monitored until the patient is
stable.

Full doses of parenteral artemisinin (artesunate or artemether) should
be provided, followed by complete doses of ACT. Rectal Artesunate is
recommended in situations where parenteral artemisinin is not available.

\textbf{Parenteral artesunate:} IV artesunate should be given
immediately after malaria is confirmed at~ 2·4mg/kg per dose at 0, 12,
and 24 hours, then every 24 hours till the child is stable and able to
tolerate oral ACT.

\begin{tcolorbox}[enhanced jigsaw, toprule=.15mm, left=2mm, leftrule=.75mm, opacitybacktitle=0.6, opacityback=0, arc=.35mm, toptitle=1mm, colbacktitle=quarto-callout-note-color!10!white, title=\textcolor{quarto-callout-note-color}{\faInfo}\hspace{0.5em}{Note}, titlerule=0mm, breakable, colframe=quarto-callout-note-color-frame, bottomtitle=1mm, colback=white, rightrule=.15mm, bottomrule=.15mm, coltitle=black]

Children with body weight \textless20 kg should receive 3·0 mg/kg per
dose to ensure equivalent exposure to the drug.

\end{tcolorbox}

Other alternate medications are:

\begin{itemize}
\item
  Intramuscular artemether: initial dose 3·2 mg/kg, then 1·6 mg/kg every
  24 h
\item
  Intravenous quinine diluted in 5\% dextrose: loading dose of 20 mg/kg
  infused over 4 hours, then 10 mg/kg every 8 hours infused no faster
  than 5 mg/kg per hour.

  \begin{tcolorbox}[enhanced jigsaw, toprule=.15mm, left=2mm, leftrule=.75mm, opacitybacktitle=0.6, opacityback=0, arc=.35mm, toptitle=1mm, colbacktitle=quarto-callout-note-color!10!white, title=\textcolor{quarto-callout-note-color}{\faInfo}\hspace{0.5em}{Note}, titlerule=0mm, breakable, colframe=quarto-callout-note-color-frame, bottomtitle=1mm, colback=white, rightrule=.15mm, bottomrule=.15mm, coltitle=black]

  Quinine is currently not recommended, and patients on Quinine are at
  risk of hypoglycaemia.

  \end{tcolorbox}
\item
  Pre-referral rectal artesunate: Recommended in primary health-care
  settings in which parenteral drug administration is not possible104
\item
  Children who are stable and can take oral medication should be given a
  full course of oral antimalarial treatment as per guidelines for
  uncomplicated malaria.
\end{itemize}

\begin{tcolorbox}[enhanced jigsaw, toprule=.15mm, left=2mm, leftrule=.75mm, opacitybacktitle=0.6, opacityback=0, arc=.35mm, toptitle=1mm, colbacktitle=quarto-callout-caution-color!10!white, title=\textcolor{quarto-callout-caution-color}{\faFire}\hspace{0.5em}{Caution!}, titlerule=0mm, breakable, colframe=quarto-callout-caution-color-frame, bottomtitle=1mm, colback=white, rightrule=.15mm, bottomrule=.15mm, coltitle=black]

Due to the risk of post-complicated malaria neurological syndrome.
Mefloquine-containing artemisinin-based combination therapies (ACT)
should be avoided.

\end{tcolorbox}

\begin{longtable}[]{@{}
  >{\raggedright\arraybackslash}p{(\linewidth - 6\tabcolsep) * \real{0.2500}}
  >{\raggedright\arraybackslash}p{(\linewidth - 6\tabcolsep) * \real{0.2500}}
  >{\raggedright\arraybackslash}p{(\linewidth - 6\tabcolsep) * \real{0.2500}}
  >{\raggedright\arraybackslash}p{(\linewidth - 6\tabcolsep) * \real{0.2500}}@{}}
\caption{Drugs recommended for the management of Uncomplicated
malaria}\label{tbl-gen-antimalarials}\tabularnewline
\toprule\noalign{}
\begin{minipage}[b]{\linewidth}\raggedright
SN
\end{minipage} & \begin{minipage}[b]{\linewidth}\raggedright
ACT
\end{minipage} & \begin{minipage}[b]{\linewidth}\raggedright
Dosage
\end{minipage} & \begin{minipage}[b]{\linewidth}\raggedright
Comments
\end{minipage} \\
\midrule\noalign{}
\endfirsthead
\toprule\noalign{}
\begin{minipage}[b]{\linewidth}\raggedright
SN
\end{minipage} & \begin{minipage}[b]{\linewidth}\raggedright
ACT
\end{minipage} & \begin{minipage}[b]{\linewidth}\raggedright
Dosage
\end{minipage} & \begin{minipage}[b]{\linewidth}\raggedright
Comments
\end{minipage} \\
\midrule\noalign{}
\endhead
\bottomrule\noalign{}
\endlastfoot
1 & Artemether-- lumefantrine & 0·83--4∙00 mg/kg artemether and
4·83--24∙00 mg/kg of lumefantrine & Take twice a day for 3 days with
fatty food; the first two doses should be given 8 hours apart. \\
2 & Artesunate--mefloquine & 4 mg/kg per day artesunate (range 2--10
mg/kg) and 8·3 mg/kg per day mefloquine (7--11 mg/kg) & Taken once a day
for 3 days \\
3 & Dihydroartemisinin --piperaquine & 4mg/kg per day dihydroartemisinin
(range 2--10 mg/kg) and 18 mg/kg per day piperaquine (16--27 mg/kg) &
Taken once a day for 3 days \\
4 & Artesunate --amodiaquine & 4 mg/kg per day of artesunate (range
2--10 mg/kg) and 10 mg/kg per day amodiaquine (7·5--15 mg/kg) once a day
for 3 days & \\
5 & Artesunate--sulf adoxine-pyrimethamine & 4 mg/kg per day of
artesunate (range 2--10 mg/kg) given once a day for 3 days and a single
administration of at least 25 mg/kg sulfadoxine (25--70 mg/kg) and 1·25
mg/kg pyrimethamine (1·25--3·5 mg/kg) given as a single dose on day 1
& \\
\end{longtable}

\section{\texorpdfstring{\textbf{Monitoring of
progress}}{Monitoring of progress}}\label{monitoring-of-progress}

The following clinical indices fluctuate, and their serial monitoring is
crucial to avoid complications:

\begin{enumerate}
\def\labelenumi{\arabic{enumi}.}
\tightlist
\item
  Hypoglycaemia
\item
  Shock
\item
  Raise Intracranial Pressure
\item
  Severe Anaemia
\item
  Haemoglobinuria~
\item
  Acute Kidney Injury
\item
  Pulmonary Edema
\end{enumerate}

\section{\texorpdfstring{\textbf{Prevention}}{Prevention}}\label{prevention-20}

\textbf{Behavioural change} to avoid the bite of infected female
\emph{anopheles} mosquitoes is important in preventing malaria.~Several
modes of behavioural techniques are known. Notable is the use of
insecticide-treated bed net.

\textbf{Indoor and Outdoor spraying residual spraying} at household
levels and in mass campaigns in wide geographic areas are known to
reduce malaria transmission.

\textbf{Seasonal Malaria Chemoprophylaxis} is the intermittent
administration of a curative dose of antimalaria medicine during the
malaria season, regardless of whether the child is infected. This is
recommended for children living within areas with high transmission.

\textbf{Perennial malaria~ chemoprophylaxis~ (PMC),} also known as
Intermittent Preventive treatment in infants (IPTi)

\textbf{Vaccines:} ~Two malaria vaccines are currently pre-qualified by
WHO. RTS, S/Vaccines and R21/Vaccine~ RTS,S/AS01 and R21/Matrix-M
vaccines are recommended by WHO to prevent malaria in children.~ Four
doses for children 5 months will benefit from the vaccines. Both RTS.S
and R21/Matrix malaria vaccines are safe and efficacious, and both have
been prequalified by WHO.(World Health Organization 2024)

Eight African countries are planning to roll out malaria vaccines as
part of routine childhood vaccinations, It is expected that more lives
will be saved with the introduction of the vaccine in sub-Saharan
Africa.(World Health Organization 2024)

\section{\texorpdfstring{\textbf{Differential
Diagnosis}}{Differential Diagnosis}}\label{differential-diagnosis-30}

The following differential diagnosis should be considered in children
presenting with clinical features suggestive of malaria but with
microscopy-negative slide and RDT-negative. History and physical
examination supported with appropriate laboratory investigation should
lead to other differential diagnoses shown below in
Table~\ref{tbl-malaria-differentials}.

\begin{longtable}[]{@{}
  >{\raggedright\arraybackslash}p{(\linewidth - 4\tabcolsep) * \real{0.3333}}
  >{\raggedright\arraybackslash}p{(\linewidth - 4\tabcolsep) * \real{0.3333}}
  >{\raggedright\arraybackslash}p{(\linewidth - 4\tabcolsep) * \real{0.3333}}@{}}
\caption{Possible differential diagnosis based on history and Anatomical
location}\label{tbl-malaria-differentials}\tabularnewline
\toprule\noalign{}
\begin{minipage}[b]{\linewidth}\raggedright
SN
\end{minipage} & \begin{minipage}[b]{\linewidth}\raggedright
Region
\end{minipage} & \begin{minipage}[b]{\linewidth}\raggedright
Differential
\end{minipage} \\
\midrule\noalign{}
\endfirsthead
\toprule\noalign{}
\begin{minipage}[b]{\linewidth}\raggedright
SN
\end{minipage} & \begin{minipage}[b]{\linewidth}\raggedright
Region
\end{minipage} & \begin{minipage}[b]{\linewidth}\raggedright
Differential
\end{minipage} \\
\midrule\noalign{}
\endhead
\bottomrule\noalign{}
\endlastfoot
1 & Head and Neck & \href{id-meningitis.qmd}{Bacterial M eningitis},
Viral Encephalitis, Otitis Media, Tonsilitis and Pharyngitis
Epiglottitis \\
2 & Thorax & Pneumonia, Bronchiolitis, Pericarditis, Carditis and
Endocarditis \\
3 & Gastrointestinal & \href{id-enteric-fever}{Typhoid fever},
Appendicitis, Hepatitis and Cholecystitis \\
4 & Genitourinary & Cystitis and Pyelonephritis \\
\end{longtable}

\section{\texorpdfstring{\textbf{Case
Discussion}}{Case Discussion}}\label{case-discussion-1}

These reviews present four hypothetical case scenarios of severe malaria
in children. It illustrates four key common pathways along which over
239,000 children die from malaria annually, either at rural health posts
or in higher tertiary institutions across sub-Saharan Africa.

\textbf{Case 1: Malaria with Haemoglobinuria}

A 4-year-old girl was seen at a Community Health Provider Services
(CHPS) compound with fever, vomiting and poor feeding. Clinical Care
received at the CHPS compound included oral anti-malaria, antipyretic
and haematinic drugs. The child's condition worsened in the subsequent
24 hours with the passage of coca-cola-like urine, which the mother had
noticed during the onset of the disease but not to the health worker at
the CHPS.~ The mother reported again to the CHPS compound the next day
with the complaint of prostration and persistent fever. The child was
subsequently referred to the primary health centre and, based on
Integrated Management of Childhood Illness (IMCI) criteria of
classification of severe disease, the child was finally referred to the
tertiary hospital as a case of severe malaria. The child convulsed on
the way to the hospital and, upon arrival, was still having focal
seizures. Blood glucose on admission was 1.2mmol/L, haemoglobin was
3.2g/dL, and lactate was 12.0mmol/L. Physical examination revealed a
child in respiratory distress with deep breathing. The respiratory rate
was 50 cycles per minute, and the heart rate was 146 beats per minute
with weak volume. Chest auscultation revealed adequate air entry and
clear lung fields. Extremities were cold with a capillary refill time of
more than 4 sec.~ Hypoglycaemia was immediately corrected with 10\%
dextrose, the shock was corrected with intravenous saline, and focal
seizures were controlled with diazepam per rectum.~ The Blantyre Coma
Score subsequently improved from 0/5 to 2/5. The child was subsequently
managed as cerebral malaria with severe anaemia secondary to
haemoglobinuria, lactic acidosis, and shock. The child was noted to have
scanty urine output (less than 0.3ml/kg/hour) on day 3 of admission.~
BUN and creatinine were raised with normal serum electrolytes. Acute
renal failure secondary to malaria with haemoglobinuria was diagnosed
and managed accordingly with IV fluids and frusemide. The child had two
episodes of blood transfusion during admission. She received quinine for
seven days, and by day 5, she had regained consciousness and was walking
by day 7. The child did not undergo dialysis but had complete recovery
of renal function by day 14 of admission and was discharged home on day
15. She has remained well in subsequent reviews.

\textbf{Case 2: Malaria with Lactic Acidosis and Pulmonary Oedema}

A 4-year-old boy was referred to the paediatric emergency unit with a
clinical diagnosis of severe malaria. He presented with prostration and
deep breathing. The Chest X-ray showed pulmonary oedema.

The patient had earlier been seen at a health centre with a 2-day
history of fever and vomiting, for which he received anti-malarial drugs
(type unknown). The child's condition did not improve as he continuously
vomited the medications and was feeding poorly the following day. The
mother noticed that the child's condition had worsened with increasing
breathlessness and inability to sit without support. The mother reported
to the health post, and the child was immediately referred to the
tertiary hospital.

The child arrived prostrated with acidotic deep breathing and had a
respiratory rate of 66 cycles per minute. Air entry was adequate
bilaterally, with crackles at the lung bases. Her heart rate was 148
beats/minute with normal heart sounds. There was no gallop rhythm. The
child had no hepatomegaly but a splenomegaly of 6cm below the
sub-coastal margin in the mid-clavicular line. The child was conscious
with a Blantyre coma score of 5/5 but prostrated (unable to stand or sit
without support). There were no other neurological deficits. Blood film
comments revealed malaria parasitaemia (P. falciparum), haemoglobin was
5.2g/dl, lactic acid was 13.3mmol/L, and normal blood glucose was
4.7mmol/L. The child was subsequently managed as a case of severe
malaria with lactic acidosis and pulmonary oedema (Acute Respiratory
Distress Syndrome, ARDS). The child was managed according to the WHO
guidelines for the management of complicated malaria with IM quinine.

\textbf{Case 3: Cerebral Malaria with Raised Intracranial Pressure}

A 3-year-old boy developed a febrile illness in a rural community.~
Mother gave acetaminophen at home with temporary relief of the fever.
After 12 hours, the fever recurred with vomiting. A single episode of
convulsion followed this. The child was sent to the health centre, where
he was found to have gained consciousness and had no neurological
deficit. The patient was managed as malaria with convulsion on oral
anti-malaria drugs and sent home. When the parents got home, the child
had another episode of convulsion, and the child was sent back to the
health centre. The child was subsequently referred to the children's
emergency unit. The child arrived in the hospital with ongoing
generalised tonic-clonic seizures. Random blood sugar was 1.2mmol/L, for
which 5ml/kg of 10\% dextrose was given immediately, and a maintenance
glucose infusion was set up. Rectal diazepam was given to abort the
seizures. Further examination revealed a deeply comatose child with a
Blantyre Coma score of 0/5.~ Oxygen saturation was 91\%. The child had
warm extremities, capillary refill time was less than 2 sec, and was
clinically pale, not cyanosed and well hydrated. The respiratory rate
was 62 cycles per minute, with occasional abnormal rhythm and effort.~
The chest was clinically clear. Heart rate was 140 beats per minute with
normal heart sounds. There was no hepato-splenomegaly. The child had
increased tone, decorticate posture, increased reflexes and extensor
plantar response. The pupil size was 1mm bilaterally and was
non-responsive to light. The child was diagnosed to have cerebral
malaria complicated by raised intracranial pressures. The patient was
managed according to the WHO guideline for cerebral malaria, and the
department treatment protocol-parenteral quinine was used.

\textbf{Case 4: Severe malaria with hyper-parasitemia and anaemia}

A 3-year-old child was referred to a tertiary hospital with a diagnosis
of severe malaria with moderate anaemia. Examination findings revealed a
child who was prostrated with normal blood glucose of 4.2 mmol/L,
Haemoglobin of 6.5g/dl, and malaria parasite density of 679,234/uL.~ The
child was managed as severe malaria with moderate anaemia based on WHO
guidelines and department protocol. A treatment course of parenteral
quinine was given. After 22 hours, the child's condition had worsened
with an increase in respiratory rate from 42 cycles per minute to 66
cycles per minute. The child looked more prostrated with a Blantyre coma
score of 2/5~ (initial score of 4/5). The chest was clinically clear;
the heart rate had increased from 126 to 160 beats per minute.
Hepatomegaly had increased from 3cm to 6cm below the sub-cost margin in
the mid-clavicular line. A repeat laboratory investigation revealed a
haemoglobin drop to 4.5g/dL. The child was given packed red blood cells
(20mL/kg BW) and further managed for malaria. The child responded to the
treatment.

\textbf{Discussions:}

\textbf{Introduction}

The over 480,000 deaths from malaria recorded annually
worldwide(Poespoprodjo et al. 2023) likely result from one of these case
scenarios. One critical question researchers and clinicians should ask
is: How can cases of malaria best be managed so that potentially
complicated cases are identified early and appropriate measures are
taken to avert death?

Too many complicated malaria cases are seen across sub-Saharan
Africa(World Health Organization 2023; K. Marsh et al. 1996).~ As we
advance in the knowledge of malaria pathogenesis and chemotherapy, we
have to translate the research knowledge into good clinical practice and
standard treatment. The case fatality rate from complicated malaria
varies across Africa.(World Health Organization 2023) Competencies of
health personnel in assessing children, among others, are important
factors.~ The bedside provides a good opportunity for health
practitioners to pick up clinical clues that signal impending
complications; if these clues are seen at the right time, appropriate
management can be instituted early to avoid death. How can these clues
be picked up to avoid these deaths in our health institutions across
Africa? The WHO Handbook on Hospital Care for Children provides
guidelines for assessing critically ill children.(Carter et al. 2011)
This document should serve as training material and medical companion
for lower-level health practitioners. Appropriate use of these materials
at the lower level with the correct application of the IMCI modules
should significantly impact the drive to reduce death from malaria.(de
Mendonça, Goncalves, and Barral-Netto 2012)

\textbf{Review of cases}

\textbf{Case 1:} In this case scenario, a history of the passage of
dark-coloured urine was enough to suspect haemolysis in this child. If
this essential symptom had been picked at the child's first appearance
at the CHPS, it should have led to an immediate referral to a higher
health centre.~

The passage of dark-coloured urine (haemoglobinuria) is a recognised
complication of severe malaria.(Crawley and Nahlen 2004; Crawley et al.
2010) Any child who presents with such symptoms is at risk of two
potentially fatal complications: severe anaemia and acute renal failure
from pigment nephropathy. Such cases should be referred to a centre
where facilities for safe blood transfusion exist. Acute renal failure
from haemoglobinuria usually responds to fluid therapy (approx 1.5 times
maintenance fluid, oral or intravenous). Haemoglobinuria is recognised
as an important cause of acute renal failure in children.(World Health
Organization 2024) The case under discussion presented haemoglobinuria
followed by the two complications of severe anaemia and acute renal
failure. The swift institution of haemotransfusion therapy averted
death. Her renal failure responded to conservative treatment with IV
fluid and frusemide with careful monitoring of the lung bases for any
evidence of pulmonary congestion.~ It is important to note that
primary-level health facilities across Africa will not be privileged to
have facilities to measure the haemoglobin levels of children reporting
severe malaria. However, a simple examination of the conjunctiva, palms
and soles by these first-level healthcare workers can detect significant
anaemia in children. Similarly, enquiry about urine colour or
macroscopic inspection of a urine sample could indicate haemolysis that
requires prompt review or referral. The scientific knowledge about the
pathogenesis of malaria-induced anaemia is widely known, and the bedside
clinical recognition of severe anaemia that is critical to warrant
transfusion is clear.(World Health Organization 2024) The current
challenge in most sub-Saharan African countries is recognising
potentially complicated malaria cases before the complications manifest
themselves. Healthcare professionals across sub-Saharan Africa should
continuously train and develop their skills in identifying the early
signs and symptoms of severe malaria.

\textbf{Case 2:} Prostration is a common presentation in children with
severe malaria and an important triage finding that health workers
should not miss in both the peripheral centres and the tertiary health
institutions. It is one clue that does not rely on any laboratory or
specialised care to detect. Recognising prostration does not require
extraordinary skills. Prostration is clearly defined as a criterion for
severe and complicated malaria; missing prostration could lead to an
underestimation of the disease condition and a serious delay in
treatment. In the case under discussion, the prostration was correctly
recognised by the healthcare worker at the periphery, which prompted the
referral.~ Clinicians in sub-Saharan Africa should develop the skill and
the consciousness to evaluate every child for prostration at the first
point of contact. Prostration in younger children who are yet to sit can
be difficult to assess. The inability to suckle or drink should raise
suspicion.

Early diagnosis of pulmonary complications is key to survival. A higher
respiratory rate, lower chest wall in-drawing and recessions may be key
findings in recognising pulmonary complications. Counting respiratory
rates accurately over one minute and relating them to other clinical
signs and symptoms is the simplest way of determining children who are
critically ill from malaria with possible pulmonary complications.

\textbf{Case 3:} This patient would have died from Hypoglycaemia,
Respiratory Arrest secondary to raised intracranial pressure or other
complications of cerebral malaria. In low-level health settings across
sub-Saharan Africa, glucose measurement is not routinely done. However,
a history of poor feeding and or vomiting in the presence of prostration
and unconsciousness should prompt health workers about the possibility
of hypoglycaemia. Hypoglycaemia secondary to malaria is known to be
associated with very high mortality.(World Health Organization 2023)
Management of the condition is simple and cheap.~ In advanced cases of
severe malaria where there has been the involvement of the brain, raised
intracranial pressures ensue. Failure to recognise this life-threatening
complication eventually leads to respiration arrest and eventual death.
Early clues known from scientific knowledge should be applied at the
bedside. Abnormal respiratory patterns, deep breathing without blood
gases and electrolytes, abnormal postures, and pupillary changes are the
bedside clinical signs of potential raised intracranial pressure.

\textbf{Case 4:}~ Early clinical signs in a child with severe malaria
and hyperparasitaemia can be misleading, especially if health staff do
not have much experience in monitoring the clinical signs of
deterioration.~ The progression of clinical disease and the care
available determine whether the child would survive severe malaria. In
the presence of heavy parasitaemia, health practitioners should be aware
that even moderate anaemia could rapidly lead to severe anaemia, which
could subsequently lead to heart failure. All parasitised red cells will
eventually haemolyse in the spleen and other reticuloendothelial
systems. In addition, because of the rosette formation, all the red
cells pulled into the spleen will be haemolysed. Children with acute
malaria, therefore, rapidly develop severe anaemia, and clinical teams
should have a high index of suspicion as well as prepare to administer
blood transfusion whenever the need arises.~ The majority of the over
one million malarial deaths that occur annually do occur because of the
complications associated with the disease. Prompt recognition of these
complications at all levels of health care and well-established referral
systems will go a long way in reducing these deaths. No death was
recorded among these four cases that presented multiple complications
associated with severe malaria. Sub-Saharan Africa's challenge is
translating the numerous research findings conducted in Africa into good
clinical practice.~

\textbf{Overall Comment:}

Basic clinical indices that are well known to signify impending
complication or life-threatening malaria should be applied in triaging
and reassessing all children who present with malaria. From the health
centre to the tertiary hospital, these clinical indices can guide health
officials in managing children according to standard protocols and
guidelines, thus avoiding death from complicated malaria.

\part{{General Pediatrics}}

\chapter{Therapeutics}\label{therapeutics}

\section{Introduction}\label{introduction-73}

Pediatric therapeutics involves the art and science of prescribing
medications and other treatments to children for disease management and
prevention. Children are not just small adults; they possess unique
physiological, metabolic, and developmental traits that affect how drugs
are absorbed, distributed, metabolized, and excreted. Consequently,
pediatric therapeutics demands a thorough understanding of child growth
stages, dosing principles, safety issues, and the specific disease
burden in the Ghanaian context.

\section{Principles of Pediatric
Therapeutics}\label{principles-of-pediatric-therapeutics}

\textbf{a. Age-related Pharmacokinetics and Pharmacodynamics}

\begin{itemize}
\tightlist
\item
  \textbf{Absorption}: Gastric pH is higher (less acidic) in neonates,
  which affects drug solubility and absorption. Gastric emptying is also
  slower.
\item
  \textbf{Distribution}: Neonates have higher body water content, which
  affects water-soluble drugs (e.g., gentamicin) and lower fat content,
  affecting lipophilic drugs.
\item
  \textbf{Metabolism}: Liver enzyme activity is immature at birth and
  develops over time, altering drug metabolism.
\item
  \textbf{Excretion}: Renal clearance is reduced in neonates due to
  immature kidneys, necessitating dose adjustments for renally-excreted
  drugs (e.g., aminoglycosides).
\end{itemize}

\textbf{b. Dosing Principles}

\begin{itemize}
\tightlist
\item
  Pediatric dosing is usually weight-based (mg/kg) or surface-area-based
  (mg/m²). For example:

  \begin{itemize}
  \item
    \textbf{Paracetamol}: 10--15 mg/kg per dose every 4--6 hours
  \item
    \textbf{Amoxicillin}: 20--40 mg/kg/day in divided doses
  \end{itemize}
\item
  Always confirm doses with a pediatric formulary or guidelines.
\end{itemize}

\section{Drug Formulations and
Administration}\label{drug-formulations-and-administration}

Children, especially infants and young kids, might not tolerate adult
formulas (e.g., tablets, capsules). Pediatric options include:

\begin{itemize}
\tightlist
\item
  \textbf{Syrups and suspensions}: Easier to swallow; dose flexibility
\item
  \textbf{Suppositories}: Useful for vomiting or unconscious children
\item
  \textbf{Inhalers/spacers}: For respiratory conditions like asthma
\item
  \textbf{Injectables}: Used in emergencies or when oral route is
  unsuitable
\end{itemize}

\textbf{Tip}: Always ensure accurate dosing with oral syringes or
calibrated spoons.

\section{Commonly Used Drugs in Ghanaian Pediatric
Practice}\label{commonly-used-drugs-in-ghanaian-pediatric-practice}

\textbf{a. Antipyretics and Analgesics}

\begin{itemize}
\tightlist
\item
  \textbf{Paracetamol}: First-line for fever and mild-to-moderate pain
\item
  \textbf{Ibuprofen}: Useful for inflammatory pain, but caution in
  dehydrated children due to renal risks
\end{itemize}

\textbf{b. Antibiotics}

\begin{itemize}
\tightlist
\item
  Used empirically or based on culture results. Indications include
  pneumonia, otitis media, and sepsis.
\item
  \textbf{First-line antibiotics}:

  \begin{itemize}
  \tightlist
  \item
    Amoxicillin
  \item
    Cloxacillin (for skin/soft tissue infections)
  \item
    Cefuroxime or Ceftriaxone for severe infections
  \end{itemize}
\item
  Rational use is essential to prevent resistance.
\end{itemize}

\textbf{c.~Antimalarials}

\begin{itemize}
\item
  Malaria is a leading cause of morbidity and mortality in children.
\item
  \textbf{First-line treatment (per Ghana NMCP)}:

  \begin{itemize}
  \tightlist
  \item
    Artesunate-amodiaquine (ASAQ)
  \item
    Artemether-lumefantrine (AL)
  \item
    Injectable artesunate for severe malaria
  \end{itemize}
\end{itemize}

\textbf{d.~Anthelmintics}

\begin{itemize}
\tightlist
\item
  Albendazole and mebendazole are used in deworming programs.
\item
  Dosing is age-based and used regularly in public health programs.
\end{itemize}

\textbf{e. Anticonvulsants}

\begin{itemize}
\tightlist
\item
  \textbf{Diazepam} (rectal or IV) for acute seizures
\item
  \textbf{Phenobarbital} and \textbf{Carbamazepine} for chronic seizure
  management
\end{itemize}

\textbf{f.~Asthma Medications}

\begin{itemize}
\tightlist
\item
  \textbf{Salbutamol}: Short-acting beta-agonist (inhaler or nebulizer)
\item
  \textbf{Beclomethasone}: Inhaled corticosteroid for maintenance
\item
  Use of a spacer device improves drug delivery in children.
\end{itemize}

\section{Rational Drug Use and
Safety}\label{rational-drug-use-and-safety}

\textbf{a. Preventing Medication Errors}

\begin{itemize}
\tightlist
\item
  Double-check drug names, doses, and units (mg vs mL)
\item
  Use weight-based dosing and avoid ``teaspoon'' measurements
\item
  Monitor for adverse drug reactions (e.g., rashes, GI upset)
\end{itemize}

\textbf{b. Drug Interactions}

\begin{itemize}
\tightlist
\item
  Be mindful of drugs that enhance or reduce each other's effects (e.g.,
  enzyme inducers/inhibitors)
\item
  Check compatibility, especially when using multiple IV medications
\end{itemize}

\textbf{c.~Avoid Contraindicated Medications}

\begin{itemize}
\tightlist
\item
  \textbf{Aspirin}: Avoid in febrile children due to risk of Reye's
  syndrome
\item
  \textbf{Tetracyclines}: Not recommended in children \textless8 years
  due to dental staining
\item
  \textbf{Chloramphenicol}: Risk of ``gray baby syndrome'' in neonates
\end{itemize}

\section{Therapeutics for Common Pediatric
Conditions}\label{therapeutics-for-common-pediatric-conditions}

\textbf{a. Pneumonia}

\begin{itemize}
\tightlist
\item
  First-line: Amoxicillin
\item
  Severe cases: Ceftriaxone or Benzylpenicillin + Gentamicin
\item
  Supportive care: Oxygen, fluids, antipyretics
\end{itemize}

\textbf{b. Acute Diarrhea}

\begin{itemize}
\tightlist
\item
  \textbf{Oral Rehydration Salts (ORS)}: Cornerstone of therapy
\item
  \textbf{Zinc supplementation}: 10--20 mg/day for 10--14 days
\item
  Avoid anti-diarrheal medications in children
\end{itemize}

\textbf{c.~Severe Acute Malnutrition (SAM)}

\begin{itemize}
\tightlist
\item
  \textbf{Therapeutic feeds}: F-75 and F-100
\item
  \textbf{Antibiotics}: Empirical (e.g., amoxicillin) due to immune
  suppression
\item
  \textbf{Micronutrients}: Vitamin A, iron (after stabilization), folate
\end{itemize}

\textbf{d.~HIV in Children}

\begin{itemize}
\tightlist
\item
  ART regimens are weight- and age-specific
\item
  Use pediatric fixed-dose combinations (FDCs)
\item
  Co-trimoxazole prophylaxis is essential
\end{itemize}

\section{Immunization and Preventive
Therapeutics}\label{immunization-and-preventive-therapeutics}

Vaccination is one of the most cost-effective therapeutic interventions
in pediatrics.

\textbf{Ghana EPI (Expanded Programme on Immunization) Schedule
includes:}

\begin{itemize}
\tightlist
\item
  BCG
\item
  Pentavalent vaccine (DPT-HepB-Hib)
\item
  OPV, IPV
\item
  PCV (Pneumococcal vaccine)
\item
  Rotavirus
\item
  Measles-Rubella
\item
  Yellow fever
\end{itemize}

\textbf{Vitamin A Supplementation}

\begin{itemize}
\tightlist
\item
  Prevents blindness and improves immunity
\item
  Given every 6 months starting at 6 months of age
\end{itemize}

\section{Monitoring and Follow-up}\label{monitoring-and-follow-up-1}

\textbf{a. Therapeutic Drug Monitoring (TDM)}

\begin{itemize}
\tightlist
\item
  Important for drugs with narrow therapeutic index (e.g.,
  aminoglycosides, phenobarbital)
\item
  Monitor renal and liver function where necessary
\end{itemize}

\textbf{b. Adherence and Education}

\begin{itemize}
\tightlist
\item
  Educate caregivers on correct drug use
\item
  Emphasize completing antibiotic courses
\item
  Address barriers to adherence (e.g., taste, cost)
\end{itemize}

\section{Ethical and Legal
Considerations}\label{ethical-and-legal-considerations}

\begin{itemize}
\tightlist
\item
  Obtain \textbf{informed consent} from parents/guardians before
  treatment
\item
  Respect \textbf{cultural beliefs} and involve families in care
\item
  Ensure \textbf{medication safety} and use only approved medications
\item
  Maintain \textbf{documentation} of all treatments given
\end{itemize}

\section{Conclusion}\label{conclusion-38}

Therapeutics in pediatrics involves more than simply prescribing
medications. It requires an understanding of developmental physiology,
rational drug use, dosing accuracy, and the sociocultural context of
care. For medical students in Ghana, mastery of local treatment
guidelines, National Standard Treatment Guidelines, and EPI protocols is
essential for safe and effective pediatric practice.

\section{Further Reading and
Resources}\label{further-reading-and-resources}

\begin{itemize}
\tightlist
\item
  Ghana Standard Treatment Guidelines (2021)
\item
  IMCI (Integrated Management of Childhood Illness) Guidelines
\item
  BNF for Children
\item
  WHO Pocket Book of Hospital Care for Children
\end{itemize}

\chapter{Congenital Malformations}\label{congenital-malformations}

\chapter{Social, Ethical and Legal
Issues}\label{social-ethical-and-legal-issues}

\section{Introduction}\label{introduction-74}

Social, ethical, and legal considerations are vital components of
holistic child healthcare. Understanding the child's family, social
background, and cultural context enables healthcare workers to diagnose
and manage effectively. These issues are particularly pertinent in
Ghanadue to diverse cultural beliefs, legal mandates, and socio-economic
disparities. This chapter explores these themes with real-life
illustrations to equip medical students with a broader understanding of
child health practice.

\section{Learning Objectives}\label{learning-objectives-1}

By the end of this chapter, students should be able to:

\begin{itemize}
\tightlist
\item
  State the importance of contextualizing patients in family and social
  history
\item
  Describe the major legal and ethical issues affecting children as
  patients
\item
  Define child abuse and state the key management issues.
\item
  Describe traditional practices in Ghana that affect children
\end{itemize}

\section{The Importance of Social Context in Clinical
Practice}\label{the-importance-of-social-context-in-clinical-practice}

Understanding a child's social context is essential for effective
diagnosis and treatment. For example, a child repeatedly brought to the
clinic by a grandparent for `fever' might be showing signs of more
profound social or emotional distress rather than a physical illness.
Medical students must go beyond clinical symptoms and explore the
child's home environment, caregivers, schooling, and socio-economic
conditions. Always ask, ``Why does THIS child fall sick in THIS way at
THIS time?''

\subsection{Taking a Social History}\label{taking-a-social-history}

A good social history is more than just a checklist. It should include
thoughtful questions about the child's living arrangements, primary
caregivers, and environmental exposures. Questions might include:

\begin{itemize}
\tightlist
\item
  Who are the child's primary caregivers?
\item
  Who feeds and bathes the child?
\item
  Where does the child live, and are there known environmental health
  risks?
\item
  Are there any concerns about school attendance or performance?
\end{itemize}

Medical students should also assess the reliability of the informant. If
the history is provided by someone other than the mother, consider
factors such as age, familiarity with the child, and the informant's
mental or physical capacity to provide accurate information.

\section{Legal and Ethical Considerations in Child
Health}\label{legal-and-ethical-considerations-in-child-health}

Child welfare is a cornerstone of legal and ethical practice in
paediatrics. According to Ghana's Children's Act (Act 560, 1998), the
child's best interest must always be the paramount consideration in any
matter affecting the child.

\subsection{Key Legal Provisions}\label{key-legal-provisions}

\begin{itemize}
\tightlist
\item
  No discrimination based on gender, disability, religion, ethnicity, or
  socio-economic status.
\item
  Every child has the right to medical care, regardless of parental or
  religious beliefs.
\item
  Children have a right to education, shelter, nutrition, and protection
  from abuse.
\end{itemize}

In practice, clinicians may encounter complex dilemmas, such as parents
refusing blood transfusions on religious grounds or the need for a court
order involving treatment. In such cases, the principle of the child's
best interest must guide decisions. However, in some cases, the child's
best interest may not be obvious, or there may be more than one child in
the situation with interests that do not coincide. An example of this
would be where the parents of a child, upon admission, request discharge
because another child at home has no one to care for them.

\section{Child Abuse}\label{child-abuse}

Child abuse is defined under the Children's Act as any contravention of
a child's rights resulting in physical or mental harm. It includes
physical, emotional, and sexual abuse, as well as neglect and
exploitation.

\subsection{Recognising Child Abuse}\label{recognising-child-abuse}

Health workers must be able to recognize signs of abuse, which may
include:

\begin{itemize}
\tightlist
\item
  Unexplained injuries
\item
  Multiple injuries at different stages of healing
\item
  Inconsistent explanations from caregivers
\item
  Signs of emotional withdrawal or fearfulness
\item
  Disturbed family dynamics or known parental mental health issues
\end{itemize}

Medical professionals are legally mandated to report suspected abuse to
the Department of Social Welfare. In many cases, health workers may be
the only advocates for the child.

\section{Fosterage and Adoption}\label{fosterage-and-adoption}

\subsection{Fosterage}\label{fosterage}

Foster parenting involves caring for a child without formal adoption.
Any person over 21 years of age with high moral character may be
eligible to foster a child. Foster parents are responsible for the
child's welfare, but do not have the same legal rights as adoptive
parents.

\subsection{Adoption}\label{adoption}

Key regulations around adoption in Ghana include:

\begin{itemize}
\tightlist
\item
  The applicant must be at least 25 years old and 21 years older than
  the child.
\item
  Single male applicants are generally discouraged except in exceptional
  circumstances.
\item
  The child must have lived with the applicant for at least three
  months.
\item
  Adoption is permanent and grants the child full rights, including
  inheritance.
\item
  Parental consent is required unless the child is abandoned or the
  parent is deemed unfit.
\end{itemize}

\section{Traditional Beliefs and
Practices}\label{traditional-beliefs-and-practices}

Cultural beliefs significantly influence health-seeking behaviour in
Ghana. Some traditional beliefs can interfere with timely medical
intervention.

\subsection{Common Traditional
Concepts}\label{common-traditional-concepts}

\begin{itemize}
\tightlist
\item
  `Asram': A broad term referring to various childhood illnesses
  believed to be caused spiritually or through a `bad eye'.
\item
  Practices such as squeezing the newborn's breast to treat engorgement
  or forcing cord separation can be harmful.
\item
  Other practices like head moulding and the use of herbal amulets may
  be harmless.
\item
  There are beneficial traditional practices, such as giving newly
  delivered mothers extra food or the wearing of beads, that help
  monitor the baby's growth.
\end{itemize}

Healthcare providers must acknowledge these beliefs while gently guiding
families toward evidence-based practices. Understanding and respecting
cultural perspectives can help bridge communication gaps and improve
health outcomes.

\section{Conclusion}\label{conclusion-39}

Social, ethical, and legal issues are central to paediatric practice.
The child must be understood within their social, cultural, and familial
context. Accurate social history-taking, sensitivity to cultural
practices, and awareness of legal mandates empower medical practitioners
to advocate effectively for their young patients. Collaborative efforts
with social workers, educators, and legal authorities may be necessary
to address complex challenges.

\chapter{Common Laboratory Investigation \&
Interpretation}\label{common-laboratory-investigation-interpretation}

\chapter{Common Bedside Tests}\label{common-bedside-tests}

\chapter{Pediatric Clinical Research}\label{pediatric-clinical-research}

\bookmarksetup{startatroot}

\chapter*{References}\label{references-3}
\addcontentsline{toc}{chapter}{References}

\markboth{References}{References}

\phantomsection\label{refs}
\begin{CSLReferences}{1}{0}
\bibitem[\citeproctext]{ref-adesegun2020}
Adesegun, OluwaseyitanA, OluwafunmilolaO Adeyemi, Osaze Ehioghae, DavidF
Rabor, TolulopeO Binuyo, BisolaA Alafin, OnyedikachiB Nnagha, AkoladeO
Idowu, and Ayokunle Osonuga. 2020. {``Current Trends in the Epidemiology
and Management of Enteric Fever in Africa: A Literature Review.''}
\emph{Asian Pacific Journal of Tropical Medicine} 13 (5): 204.
\url{https://doi.org/10.4103/1995-7645.283515}.

\bibitem[\citeproctext]{ref-alsayed2023}
Alsayed, Shahinda S. R., and Hendra Gunosewoyo. 2023. {``Tuberculosis:
Pathogenesis, Current Treatment Regimens and New Drug Targets.''}
\emph{International Journal of Molecular Sciences} 24 (6): 5202.
\url{https://doi.org/10.3390/ijms24065202}.

\bibitem[\citeproctext]{ref-andrews2023pathogenesis}
Andrews, Jason, and Richelle C Charles. 2023. {``Pathogenesis of Enteric
(Typhoid and Paratyphoid) Fever.''} \emph{UpToDate}.
\url{https://www.uptodate.com/contents/pathogenesis-of-enteric-typhoid-and-paratyphoid-fever}.

\bibitem[\citeproctext]{ref-appiah2018}
Appiah, J A, S Salie, A Argent, and B Morrow. 2018. {``Characteristics,
Course and Outcomes of Children Admitted to a Paediatric Intensive Care
Unit After Cardiac Arrest.''} \emph{Southern African Journal of Critical
Care} 34 (2): 58. \url{https://doi.org/10.7196/sajcc.2018.v34i2.355}.

\bibitem[\citeproctext]{ref-arias2024}
Arias, Anita V, Michael Lintner-Rivera, Nadeem I Shafi, Qalab Abbas,
Abdelhafeez H Abdelhafeez, Muhammad Ali, Halaashuor Ammar, et al. 2024.
{``A Research Definition and Framework for Acute Paediatric Critical
Illness Across Resource-Variable Settings: A Modified Delphi
Consensus.''} \emph{The Lancet Global Health} 12 (2): e331--40.
\url{https://doi.org/10.1016/s2214-109x(23)00537-5}.

\bibitem[\citeproctext]{ref-borges-lujan2022}
Borges-Lujan, Moreyba, Gema E. Gonzalez-Luis, Tom Roosen, Maurice J.
Huizing, and Eduardo Villamor. 2022. {``Sex Differences in Patent Ductus
Arteriosus Incidence and Response to Pharmacological Treatment in
Preterm Infants: A Systematic Review, Meta-Analysis and
Meta-Regression.''} \emph{Journal of Personalized Medicine} 12 (7):
1143. \url{https://doi.org/10.3390/jpm12071143}.

\bibitem[\citeproctext]{ref-caputo2005}
Caputo, Salvatore, Giovanbattista Capozzi, Maria Giovanna Russo, Teresa
Esposito, Lucia Martina, Dominga Cardaropoli, Concetta Ricci, Paola
Argiento, Giuseppe Pacileo, and Raffaele Calabrò. 2005. {``Familial
Recurrence of Congenital Heart Disease in Patients with Ostium Secundum
Atrial Septal Defect.''} \emph{European Heart Journal} 26 (20):
2179--84. \url{https://doi.org/10.1093/eurheartj/ehi378}.

\bibitem[\citeproctext]{ref-carter2011}
Carter, Nick, Allan Pamba, Stephan Duparc, and John N Waitumbi. 2011.
{``Frequency of Glucose-6-Phosphate Dehydrogenase Deficiency in Malaria
Patients from Six African Countries Enrolled in Two Randomized
Anti-Malarial Clinical Trials.''} \emph{Malaria Journal} 10 (1).
\url{https://doi.org/10.1186/1475-2875-10-241}.

\bibitem[\citeproctext]{ref-crawley2010}
Crawley, Jane, Cindy Chu, George Mtove, and François Nosten. 2010.
{``Malaria in Children.''} \emph{The Lancet} 375 (9724): 1468--81.
\url{https://doi.org/10.1016/s0140-6736(10)60447-3}.

\bibitem[\citeproctext]{ref-crawley2004}
Crawley, Jane, and Bernard Nahlen. 2004. {``Prevention and Treatment of
Malaria in Young African Children.''} \emph{Seminars in Pediatric
Infectious Diseases} 15 (3): 169--80.
\url{https://doi.org/10.1053/j.spid.2004.05.009}.

\bibitem[\citeproctext]{ref-cunningham2000}
Cunningham, A L, S Li, J Juarez, G Lynch, M Alali, and H Naif. 2000.
{``The Level of HIV Infection of Macrophages Is Determined by
Interaction of Viral and Host Cell Genotypes.''} \emph{Journal of
Leukocyte Biology} 68 (3): 311--17.
\url{https://doi.org/10.1189/jlb.68.3.311}.

\bibitem[\citeproctext]{ref-demendonuxe7a2012}
de Mendonça, Vitor R. R., Marilda Souza Goncalves, and Manoel
Barral-Netto. 2012. {``The Host Genetic Diversity in Malaria
Infection.''} \emph{Journal of Tropical Medicine} 2012: 1--17.
\url{https://doi.org/10.1155/2012/940616}.

\bibitem[\citeproctext]{ref-driss2011a}
Driss, Adel, Jacqueline M Hibbert, Nana O Wilson, Shareen A Iqbal,
Thomas V Adamkiewicz, and Jonathan K Stiles. 2011. {``Genetic
Polymorphisms Linked to Susceptibility to Malaria.''} \emph{Malaria
Journal} 10 (1). \url{https://doi.org/10.1186/1475-2875-10-271}.

\bibitem[\citeproctext]{ref-medscapeTyphoidFever}
emedicine. 2024. {``{T}yphoid {F}ever {F}ollow-up: {F}urther
{O}utpatient {C}are, {F}urther {I}npatient {C}are,
{D}eterrence/{P}revention --- Emedicine.medscape.com.''}
\url{https://emedicine.medscape.com/article/231135-followup?form=fpf}.

\bibitem[\citeproctext]{ref-ghs2023jointreview}
Ghana Health Service. 2023. {``Ghana Health Service Holds Annual Joint
TB/HIV Performance Review Meeting.''} \emph{GHS News}.
\url{https://ghs.gov.gh/2023/01/22/ghana-health-service-holds-annual-joint-tb-hiv-performance-review-meeting/}.

\bibitem[\citeproctext]{ref-hoffman2009}
Hoffman, Olaf, and Joerg R. Weber. 2009. {``Review: Pathophysiology and
Treatment of Bacterial Meningitis.''} \emph{Therapeutic Advances in
Neurological Disorders} 2 (6): 401--12.
\url{https://doi.org/10.1177/1756285609337975}.

\bibitem[\citeproctext]{ref-jortveit2016}
Jortveit, Jarle, Elisabeth Leirgul, Leif Eskedal, Gottfried Greve,
Tatiana Fomina, Gaute Døhlen, Grethe S Tell, Sigurd Birkeland, Nina
Øyen, and Henrik Holmstrøm. 2016. {``Mortality and Complications in 3495
Children with Isolated Ventricular Septal Defects.''} \emph{Archives of
Disease in Childhood} 101 (9): 808--13.
\url{https://doi.org/10.1136/archdischild-2015-310154}.

\bibitem[\citeproctext]{ref-Kalisch-Smith2020}
Kalisch-Smith, Jacinta Isabelle, Nikita Ved, and Duncan Burnaby Sparrow.
2019. {``Environmental Risk Factors for Congenital Heart Disease.''}
\emph{Cold Spring Harbor Perspectives in Biology} 12 (3): a037234.
\url{https://doi.org/10.1101/cshperspect.a037234}.

\bibitem[\citeproctext]{ref-ko2015}
Ko, Jung Min. 2015. {``Genetic Syndromes Associated with Congenital
Heart Disease.''} \emph{Korean Circulation Journal} 45 (5): 357.
\url{https://doi.org/10.4070/kcj.2015.45.5.357}.

\bibitem[\citeproctext]{ref-lewis2018}
Lewis, Grant, and Paul McConnell. 2018. {``Ethical Issues in
Resuscitation and Intensive Care.''} \emph{Anaesthesia \& Intensive Care
Medicine} 19 (12): 644--47.
\url{https://doi.org/10.1016/j.mpaic.2018.09.004}.

\bibitem[\citeproctext]{ref-marsh1996}
Marsh, K., M. English, J. Crawley, and N. Peshu. 1996. {``The
Pathogenesis of Severe Malaria in African Children.''} \emph{Annals of
Tropical Medicine \& Parasitology} 90 (4): 395--402.
\url{https://doi.org/10.1080/00034983.1996.11813068}.

\bibitem[\citeproctext]{ref-marsh1995}
Marsh, Kevin, Dayo Forster, Catherine Waruiru, Isiah Mwangi, Maria
Winstanley, Victoria Marsh, Charles Newton, et al. 1995. {``Indicators
of Life-Threatening Malaria in African Children.''} \emph{New England
Journal of Medicine} 332 (21): 1399--1404.
\url{https://doi.org/10.1056/nejm199505253322102}.

\bibitem[\citeproctext]{ref-meghani2021}
Meghani, Shaista. 2021. {``Witnessed Resuscitation: A Concept
Analysis.''} \emph{Intensive and Critical Care Nursing} 64 (June):
103003. \url{https://doi.org/10.1016/j.iccn.2020.103003}.

\bibitem[\citeproctext]{ref-moeti2023}
Moeti, Matshidiso R, Prebo Brango, Juliet Nabyonga-Orem, and Benido
Impouma. 2023. {``Ending the Burden of Sickle Cell Disease in Africa.''}
\emph{The Lancet Haematology} 10 (8): e567--69.
\url{https://doi.org/10.1016/s2352-3026(23)00120-5}.

\bibitem[\citeproctext]{ref-moir2011}
Moir, Susan, Tae-Wook Chun, and Anthony S. Fauci. 2011. {``Pathogenic
Mechanisms of HIV Disease.''} \emph{Annual Review of Pathology:
Mechanisms of Disease} 6 (1): 223--48.
\url{https://doi.org/10.1146/annurev-pathol-011110-130254}.

\bibitem[\citeproctext]{ref-who_meningitis}
Organization, World Health. 2021. {``Defeating Meningitis by 2030: A
Global Road Map.''} \emph{World Health Organization}.
\url{https://apps.who.int/iris/handle/10665/342010}.

\bibitem[\citeproctext]{ref-uxf8yen2009}
Øyen, Nina, Gry Poulsen, Heather A. Boyd, Jan Wohlfahrt, Peter K. A.
Jensen, and Mads Melbye. 2009. {``Recurrence of Congenital Heart Defects
in Families.''} \emph{Circulation} 120 (4): 295--301.
\url{https://doi.org/10.1161/circulationaha.109.857987}.

\bibitem[\citeproctext]{ref-path_meningitis}
PATH. 2021. {``Toward a World Without Meningitis.''} \emph{PATH Impact
Stories}.
\url{https://www.path.org/our-impact/articles/toward-world-without-meningitis/}.

\bibitem[\citeproctext]{ref-poespoprodjo2023}
Poespoprodjo, Jeanne Rini, Nicholas M Douglas, Daniel Ansong, Steven
Kho, and Nicholas M Anstey. 2023. {``Malaria.''} \emph{The Lancet} 402
(10419): 2328--45. \url{https://doi.org/10.1016/s0140-6736(23)01249-7}.

\bibitem[\citeproctext]{ref-sakaan2022}
Sakaan, Firas, Adrian Holloway, Qalab Abbas, John Appiah, Jonah
Attebery, Paula Caporal, Ericka Fink, et al. 2022. {``643: ESTIMATING
THE GLOBAL PREVALENCE OF PEDIATRIC ACUTE CRITICAL ILLNESS IN
RESOURCE-LIMITED SETTINGS.''} \emph{Critical Care Medicine} 51 (1):
311--11. \url{https://doi.org/10.1097/01.ccm.0000908304.12117.81}.

\bibitem[\citeproctext]{ref-skar2024meningitis}
Skar, Gwenn, Lillian Flannigan, Rebecca Latch, and Jessica Snowden.
2024. {``Meningitis in Children: Still a Can't-Miss Diagnosis.''}
\emph{Pediatrics In Review} 45 (6): 305--15.
\url{https://doi.org/10.1542/pir.2023-006013}.

\bibitem[\citeproctext]{ref-smith2019}
Smith, Clayton A., Courtney McCracken, Amanda S. Thomas, Logan G.
Spector, James D. St Louis, Matthew E. Oster, James H. Moller, and
Lazaros Kochilas. 2019. {``Long-Term Outcomes of Tetralogy of Fallot.''}
\emph{JAMA Cardiology} 4 (1): 34.
\url{https://doi.org/10.1001/jamacardio.2018.4255}.

\bibitem[\citeproctext]{ref-wang2023}
Wang, Huaming, Xi Lin, Guorong Lyu, Shaozheng He, Bingtian Dong, and
Yiru Yang. 2023. {``Chromosomal Abnormalities in Fetuses with Congenital
Heart Disease: A Meta-Analysis.''} \emph{Archives of Gynecology and
Obstetrics} 308 (3): 797--811.
\url{https://doi.org/10.1007/s00404-023-06910-3}.

\bibitem[\citeproctext]{ref-WHO_typhoid}
WHO. 2023. {``Typhoid.''} 2023.
\url{/url\%7Bhttps://www.who.int/news-room/fact-sheets/detail/typhoid\%7D}.

\bibitem[\citeproctext]{ref-wilen2012}
Wilen, C. B., J. C. Tilton, and R. W. Doms. 2012. {``HIV: Cell Binding
and Entry.''} \emph{Cold Spring Harbor Perspectives in Medicine} 2 (8):
a006866--66. \url{https://doi.org/10.1101/cshperspect.a006866}.

\bibitem[\citeproctext]{ref-who2023worldmalaria}
World Health Organization. 2023. {``World Malaria Report 2023.''}
\url{https://www.who.int/teams/global-malaria-programme/reports/world-malaria-report-2023}.

\bibitem[\citeproctext]{ref-WHO_MalariaVaccines}
---------. 2024. {``Malaria Vaccines (RTS,s and R21).''} 2024.
\url{https://www.who.int/news-room/questions-and-answers/item/q-a-on-rts-s-malaria-vaccine}.

\bibitem[\citeproctext]{ref-who2023hiv}
World Health Organization (WHO). 2023. {``HIV and AIDS.''} \emph{Fact
Sheets}.
\url{https://www.who.int/news-room/fact-sheets/detail/hiv-aids}.

\bibitem[\citeproctext]{ref-yuan2021}
Yuan, Zhen, Long-Zhen Zhang, Bin Li, Hung-Tao Chung, Jin-Xin Jiang, John
Y. Chiang, Hsin-Ju Chiang, Hon-Kan Yip, and Pei-Hsun Sung. 2021.
{``Investigation of Echocardiographic Characteristics and Predictors for
Persistent Defects of Patent Foramen Ovale or Patent Ductus Arteriosus
in Chinese Newborns.''} \emph{Biomedical Journal} 44 (2): 209--16.
\url{https://doi.org/10.1016/j.bj.2019.12.007}.

\bibitem[\citeproctext]{ref-zhang2022}
Zhang, Haobo, Mengda Liu, Weixing Fan, Shufang Sun, and Xiaoxu Fan.
2022. {``The Impact of Mycobacterium Tuberculosis Complex in the
Environment on One Health Approach.''} \emph{Frontiers in Public Health}
10 (September). \url{https://doi.org/10.3389/fpubh.2022.994745}.

\end{CSLReferences}




\end{document}
